\section{Results}
\label{ref_results}

The results from the the two literature reviews are presented here. Additionally, the results of the experiments with the STUMPY python library implementation of the Iterative Matrix Profile detector are presented.

\subsection{Data Set Survey}
In this section, the researchers surveyed 8 works focusing on streaming data outlier detection in machine learning. The datasets used in each paper were examined to determine which datasets were commonly used in the literature as shown in Table \ref{tab:datasets_outlier_detection}. 

\bigskip
\begin{longtable}{llp{7cm}}
\caption{Datasets for Streaming Outlier Detection} \\
\toprule
\textbf{Dataset} & \textbf{Citation Count} & \textbf{Description} \\
\midrule
\endfirsthead
\multicolumn{3}{c}%
{\tablename\ \thetable\ -- \textit{Continued from previous page}} \\
\hline
\textbf{Dataset} & \textbf{Citation Count} & \textbf{Description} \\
\hline
\endhead
\hline \multicolumn{3}{r}{\textit{Continued on next page}} \\
\endfoot
\hline
\endlastfoot
    KDD-CUP99 \footcite{kdd1999} & 6 \footcite{anomalies-detection-isolation}, \footcite{dilof-data-streams}, \footcite{fast-memory-efficent-lof-milof}, \footcite{fast-anomaly-detection-streaming}, \footcite{designing-streaming-alg-for-outlier-detection},\footcite{anomaly-pattern-detection}  & Network intrusion detection dataset. Includes a wide variety of malicious and normal connections simulated in a military network environment.   \\
    \midrule
    Covertype-Forest \footcite{covertype-dataset} & 4 \footcite{anomalies-detection-isolation}, \footcite{fast-memory-efficent-lof-milof}, \footcite{fast-anomaly-detection-streaming},
    \footcite{designing-streaming-alg-for-outlier-detection} & Predicting forest cover type from cartographic variables only.   \\
    \midrule
    Shuttle \footcite{shuttle-dataset} & 3\footcite{anomalies-detection-isolation},  \footcite{fast-anomaly-detection-streaming},
    \footcite{anomaly-pattern-detection} 
     & A multi-class classification dataset with dimensionality 9.   \\
    \midrule
    UCI-Vowel \footcite{uci-vowel-dataset} & 2
    \footcite{dilof-data-streams},
    \footcite{fast-memory-efficent-lof-milof}
     & Nine participants spoke two Japaneese vowels sucesevily. 640 time series were created using linear predictions.\\
    \midrule
    UCI-Pendigit \footcite{uci-pendigit-dataset} & 2
    \footcite{dilof-data-streams},
    \footcite{fast-memory-efficent-lof-milof}
     & Database of 250 hand-written digits from 44 participants.   \\
\label{tab:datasets_outlier_detection}
\end{longtable}



Figure \ref{fig_dataset_lit} shows the most commonly occurring data sets in the literature from table \ref{tab:datasets_outlier_detection} are KDD\_CUP99 \parencite{kdd1999} followed by Covertype-Forest \parencite{covertype-dataset}. A data set is included in table \ref{tab:datasets_outlier_detection} and figure \ref{fig_dataset_lit} if there are 2 or more occurrences of it in the literature surveyed. The most commonly used datasets found in the literature were originally created in the late 1990's. Most of the research surveyed has been conducted in the last 10 years but the standard benchmark datasets are from the late 90's. This suggests a lack of standardization in data sets to compare algorithm performance. Most papers surveyed included these seemingly standard datasets while also utilizing a wide variety of custom, private, or less well known datasets. Further research should be conducted to determine if the existing standard datasets are sufficient to benchmark performance for modern streaming outlier detection algorithms. If they are not, a standard test suite or data set should be proposed to uniformly evaluate performance across different works.

\begin{figure}[h]
    %%\centering
    \begin{tikzpicture}
        \pie[sum=auto]{6/KDD\_CUP99, 4/Covertype-Forest, 3/Shuttle, 2/UCI\_Pendigit, 2/UCI\_Vowel}
    \end{tikzpicture}
    \caption{Data Set Occurrence in Literature }
    \label{fig_dataset_lit}
\end{figure}

\subsubsection{Discussion with Experts}

The author began a discussion on the Github Discussion Board \parencite{RiverGithub2022} for the popular River \parencite{2020river} Machine Learning library. The post discusses plans to include a streaming Local Outlier Factor (LOF) methodology to the library and test its effectiveness. Through this discussion the author realized that this approach is not the best solution for this given task. Through this discussion, the author was introduced to the BETH dataset \parencite{beth-dataset}, which will be used as one of the experimental test datasets for the developed technique. Additionally, this discussion prompted the author of the BETH dataset, Kate Highnam, to add the dataset to a public repository at Imperial College London which provides easy access for researchers wishing to work on the dataset.

The author also connected with private cyber security researcher Matti Bispham \parencite{Bispham-email:private} at the National Composites Centre National Composites Centre (NCC) Group. Through this correspondence, Matti provided some practical issues with the most popular cited dataset, KDD\_CUP99, mostly corresponding to the age of the dataset including:

\begin{itemize}[leftmargin=2cm]
    \item Cyber-attacks have changed dramatically since 1999 (when KDD\_CUP99 was published). They have a different ontology and language (ex. Crypto mining, Ransomware, etc.)
    \item New network protocols have been developed and eclipsed or significantly modified older standards (ex. HTTP)
    \item System data logging has changed significantly
\end{itemize}

Matti also identified some potential shortcomings of the BETH dataset principally related to the data available including:

\begin{itemize}[leftmargin=2cm]
    \item The data was collected across many days, but only for a short period of consecutive time, around 5 hours.
    \item The exploited vulnerability was an SSH one where the attacker was granted access as long as they entered any password. This immediately classifies the types of attackers who will try to access the system as unsophisticated actors.
    \item Only DNS network logs (not SSH logs) are available.
\end{itemize}

The author additionally connected with Kate Highnam \parencite{Highnam-email:private}, the creator of the BETH dataset to understand additional aspects of the datasets discussed previously. Her original paper states ``The KDD1999, NSLKDD, and ISCX datasets contain network traffic, while the DARPA1998 dataset also includes limited process calls. However, these datasets are at best almost a decade old, and are collected on in-premise servers \parencite{beth-dataset}.'' In the BETH dataset, the honypots are deployed in a cloud environment because many major companies utilise compute through cloud providers. Therefore it is important to understand the attacks that are specifically scanning the cloud provider space compared to the prior datasets which are collected exclusively through on-prem servers. Additionally she expands that ``Typically one chooses a dataset for either of those reasons: to demonstrate the impact of a model in a specific area or the capabilities of the model (and thus extendable to other areas)  or the capabilities of the model (and thus extendable to other areas).'' The BETH dataset is designed in a way that machine learning researchers looking for applications can find it useful. In this work it is used in this way, by approaching the anomaly classification problem from a different angle than traditional machine learning techniques.

\subsection{Anomaly Detection Results}

In this section, the datasets from section \ref{ref_datasets} are tested using the developed matrix profile anomaly detection technique. The parameters for each trail are listed and the results are explain. Further analysis is performed in section \ref{sec:discussion}.

\subsubsection{Hydraulic Simulation Data Set}
\label{ref_results_hydraulic_sim}
The hydraulic simulation data set presented in section \ref{ref_hydraulic_dataset} is used to perform this simulation. Table \ref{tab:hydraulic_sim_params} specifies the parameters used with the anomaly detector to obtain the results in this section. This is the smallest dataset tested in this study and it outlines the performance and characteristics of the anomaly detector at small window sizes. The standard deviation multiplier is the default for the detector. If the data follows a standard Gaussian distribution, the detected points would fall outside 99.7\% of the data present in the window, which would represent a significant outlier comparative to the rest of the data. The same principal applies to the rolling range multiplier. The recent range detection debounce multiplier is also standard given the small window size and ensures that there are not over-noisy or duplicate detections.

\begin{table}[H]
%%\centering
\begin{tabular}{|l|c|l|}
    \hline
	\textbf{Parameter} & \textbf{Value} & \textbf{Description} \\ \hline
	m & 15 & Window Size \\ \hline
	ts$\_$size & 30 & Time Series Size \\ \hline
	std$\_$dev & 3 & Standard Deviation Multiplier \\ \hline
	range & 3 & Rolling Range Multiplier\\ \hline
	recent & 2 & Recent Detection Debounce\\ \hline
\end{tabular}
\caption{Hydraulic Simulation Detector Parameters}
\label{tab:hydraulic_sim_params}
\end{table}

Figures \ref{fig:hydraulic_sim_standard_force} and \ref{fig:hydraulic_sim_signal_force} show the original control signal with the anomaly detections from the experiment highlighted with red lines. In this case, the detector was able to locate all of the inflection points where the signal changed for both control signals without false positives. This is a 100\% accuracy rate for the two different control signals presented with the same detector parameters.

% \begin{figure}[H]
%     \begin{minipage}[t]{0.5\linewidth}
%         %%\centering
%         \resizebox{\linewidth}{!}{%% Creator: Matplotlib, PGF backend
%%
%% To include the figure in your LaTeX document, write
%%   \input{<filename>.pgf}
%%
%% Make sure the required packages are loaded in your preamble
%%   \usepackage{pgf}
%%
%% Also ensure that all the required font packages are loaded; for instance,
%% the lmodern package is sometimes necessary when using math font.
%%   \usepackage{lmodern}
%%
%% Figures using additional raster images can only be included by \input if
%% they are in the same directory as the main LaTeX file. For loading figures
%% from other directories you can use the `import` package
%%   \usepackage{import}
%%
%% and then include the figures with
%%   \import{<path to file>}{<filename>.pgf}
%%
%% Matplotlib used the following preamble
%%
\begingroup%
\makeatletter%
\begin{pgfpicture}%
\pgfpathrectangle{\pgfpointorigin}{\pgfqpoint{4.000000in}{3.000000in}}%
\pgfusepath{use as bounding box, clip}%
\begin{pgfscope}%
\pgfsetbuttcap%
\pgfsetmiterjoin%
\pgfsetlinewidth{0.000000pt}%
\definecolor{currentstroke}{rgb}{1.000000,1.000000,1.000000}%
\pgfsetstrokecolor{currentstroke}%
\pgfsetstrokeopacity{0.000000}%
\pgfsetdash{}{0pt}%
\pgfpathmoveto{\pgfqpoint{0.000000in}{0.000000in}}%
\pgfpathlineto{\pgfqpoint{4.000000in}{0.000000in}}%
\pgfpathlineto{\pgfqpoint{4.000000in}{3.000000in}}%
\pgfpathlineto{\pgfqpoint{0.000000in}{3.000000in}}%
\pgfpathlineto{\pgfqpoint{0.000000in}{0.000000in}}%
\pgfpathclose%
\pgfusepath{}%
\end{pgfscope}%
\begin{pgfscope}%
\pgfsetbuttcap%
\pgfsetmiterjoin%
\definecolor{currentfill}{rgb}{1.000000,1.000000,1.000000}%
\pgfsetfillcolor{currentfill}%
\pgfsetlinewidth{0.000000pt}%
\definecolor{currentstroke}{rgb}{0.000000,0.000000,0.000000}%
\pgfsetstrokecolor{currentstroke}%
\pgfsetstrokeopacity{0.000000}%
\pgfsetdash{}{0pt}%
\pgfpathmoveto{\pgfqpoint{0.500000in}{0.375000in}}%
\pgfpathlineto{\pgfqpoint{3.600000in}{0.375000in}}%
\pgfpathlineto{\pgfqpoint{3.600000in}{2.640000in}}%
\pgfpathlineto{\pgfqpoint{0.500000in}{2.640000in}}%
\pgfpathlineto{\pgfqpoint{0.500000in}{0.375000in}}%
\pgfpathclose%
\pgfusepath{fill}%
\end{pgfscope}%
\begin{pgfscope}%
\pgfsetbuttcap%
\pgfsetroundjoin%
\definecolor{currentfill}{rgb}{0.000000,0.000000,0.000000}%
\pgfsetfillcolor{currentfill}%
\pgfsetlinewidth{0.803000pt}%
\definecolor{currentstroke}{rgb}{0.000000,0.000000,0.000000}%
\pgfsetstrokecolor{currentstroke}%
\pgfsetdash{}{0pt}%
\pgfsys@defobject{currentmarker}{\pgfqpoint{0.000000in}{-0.048611in}}{\pgfqpoint{0.000000in}{0.000000in}}{%
\pgfpathmoveto{\pgfqpoint{0.000000in}{0.000000in}}%
\pgfpathlineto{\pgfqpoint{0.000000in}{-0.048611in}}%
\pgfusepath{stroke,fill}%
}%
\begin{pgfscope}%
\pgfsys@transformshift{0.640909in}{0.375000in}%
\pgfsys@useobject{currentmarker}{}%
\end{pgfscope}%
\end{pgfscope}%
\begin{pgfscope}%
\definecolor{textcolor}{rgb}{0.000000,0.000000,0.000000}%
\pgfsetstrokecolor{textcolor}%
\pgfsetfillcolor{textcolor}%
\pgftext[x=0.640909in,y=0.277778in,,top]{\color{textcolor}\rmfamily\fontsize{10.000000}{12.000000}\selectfont \(\displaystyle {0}\)}%
\end{pgfscope}%
\begin{pgfscope}%
\pgfsetbuttcap%
\pgfsetroundjoin%
\definecolor{currentfill}{rgb}{0.000000,0.000000,0.000000}%
\pgfsetfillcolor{currentfill}%
\pgfsetlinewidth{0.803000pt}%
\definecolor{currentstroke}{rgb}{0.000000,0.000000,0.000000}%
\pgfsetstrokecolor{currentstroke}%
\pgfsetdash{}{0pt}%
\pgfsys@defobject{currentmarker}{\pgfqpoint{0.000000in}{-0.048611in}}{\pgfqpoint{0.000000in}{0.000000in}}{%
\pgfpathmoveto{\pgfqpoint{0.000000in}{0.000000in}}%
\pgfpathlineto{\pgfqpoint{0.000000in}{-0.048611in}}%
\pgfusepath{stroke,fill}%
}%
\begin{pgfscope}%
\pgfsys@transformshift{1.139702in}{0.375000in}%
\pgfsys@useobject{currentmarker}{}%
\end{pgfscope}%
\end{pgfscope}%
\begin{pgfscope}%
\definecolor{textcolor}{rgb}{0.000000,0.000000,0.000000}%
\pgfsetstrokecolor{textcolor}%
\pgfsetfillcolor{textcolor}%
\pgftext[x=1.139702in,y=0.277778in,,top]{\color{textcolor}\rmfamily\fontsize{10.000000}{12.000000}\selectfont \(\displaystyle {100}\)}%
\end{pgfscope}%
\begin{pgfscope}%
\pgfsetbuttcap%
\pgfsetroundjoin%
\definecolor{currentfill}{rgb}{0.000000,0.000000,0.000000}%
\pgfsetfillcolor{currentfill}%
\pgfsetlinewidth{0.803000pt}%
\definecolor{currentstroke}{rgb}{0.000000,0.000000,0.000000}%
\pgfsetstrokecolor{currentstroke}%
\pgfsetdash{}{0pt}%
\pgfsys@defobject{currentmarker}{\pgfqpoint{0.000000in}{-0.048611in}}{\pgfqpoint{0.000000in}{0.000000in}}{%
\pgfpathmoveto{\pgfqpoint{0.000000in}{0.000000in}}%
\pgfpathlineto{\pgfqpoint{0.000000in}{-0.048611in}}%
\pgfusepath{stroke,fill}%
}%
\begin{pgfscope}%
\pgfsys@transformshift{1.638496in}{0.375000in}%
\pgfsys@useobject{currentmarker}{}%
\end{pgfscope}%
\end{pgfscope}%
\begin{pgfscope}%
\definecolor{textcolor}{rgb}{0.000000,0.000000,0.000000}%
\pgfsetstrokecolor{textcolor}%
\pgfsetfillcolor{textcolor}%
\pgftext[x=1.638496in,y=0.277778in,,top]{\color{textcolor}\rmfamily\fontsize{10.000000}{12.000000}\selectfont \(\displaystyle {200}\)}%
\end{pgfscope}%
\begin{pgfscope}%
\pgfsetbuttcap%
\pgfsetroundjoin%
\definecolor{currentfill}{rgb}{0.000000,0.000000,0.000000}%
\pgfsetfillcolor{currentfill}%
\pgfsetlinewidth{0.803000pt}%
\definecolor{currentstroke}{rgb}{0.000000,0.000000,0.000000}%
\pgfsetstrokecolor{currentstroke}%
\pgfsetdash{}{0pt}%
\pgfsys@defobject{currentmarker}{\pgfqpoint{0.000000in}{-0.048611in}}{\pgfqpoint{0.000000in}{0.000000in}}{%
\pgfpathmoveto{\pgfqpoint{0.000000in}{0.000000in}}%
\pgfpathlineto{\pgfqpoint{0.000000in}{-0.048611in}}%
\pgfusepath{stroke,fill}%
}%
\begin{pgfscope}%
\pgfsys@transformshift{2.137289in}{0.375000in}%
\pgfsys@useobject{currentmarker}{}%
\end{pgfscope}%
\end{pgfscope}%
\begin{pgfscope}%
\definecolor{textcolor}{rgb}{0.000000,0.000000,0.000000}%
\pgfsetstrokecolor{textcolor}%
\pgfsetfillcolor{textcolor}%
\pgftext[x=2.137289in,y=0.277778in,,top]{\color{textcolor}\rmfamily\fontsize{10.000000}{12.000000}\selectfont \(\displaystyle {300}\)}%
\end{pgfscope}%
\begin{pgfscope}%
\pgfsetbuttcap%
\pgfsetroundjoin%
\definecolor{currentfill}{rgb}{0.000000,0.000000,0.000000}%
\pgfsetfillcolor{currentfill}%
\pgfsetlinewidth{0.803000pt}%
\definecolor{currentstroke}{rgb}{0.000000,0.000000,0.000000}%
\pgfsetstrokecolor{currentstroke}%
\pgfsetdash{}{0pt}%
\pgfsys@defobject{currentmarker}{\pgfqpoint{0.000000in}{-0.048611in}}{\pgfqpoint{0.000000in}{0.000000in}}{%
\pgfpathmoveto{\pgfqpoint{0.000000in}{0.000000in}}%
\pgfpathlineto{\pgfqpoint{0.000000in}{-0.048611in}}%
\pgfusepath{stroke,fill}%
}%
\begin{pgfscope}%
\pgfsys@transformshift{2.636082in}{0.375000in}%
\pgfsys@useobject{currentmarker}{}%
\end{pgfscope}%
\end{pgfscope}%
\begin{pgfscope}%
\definecolor{textcolor}{rgb}{0.000000,0.000000,0.000000}%
\pgfsetstrokecolor{textcolor}%
\pgfsetfillcolor{textcolor}%
\pgftext[x=2.636082in,y=0.277778in,,top]{\color{textcolor}\rmfamily\fontsize{10.000000}{12.000000}\selectfont \(\displaystyle {400}\)}%
\end{pgfscope}%
\begin{pgfscope}%
\pgfsetbuttcap%
\pgfsetroundjoin%
\definecolor{currentfill}{rgb}{0.000000,0.000000,0.000000}%
\pgfsetfillcolor{currentfill}%
\pgfsetlinewidth{0.803000pt}%
\definecolor{currentstroke}{rgb}{0.000000,0.000000,0.000000}%
\pgfsetstrokecolor{currentstroke}%
\pgfsetdash{}{0pt}%
\pgfsys@defobject{currentmarker}{\pgfqpoint{0.000000in}{-0.048611in}}{\pgfqpoint{0.000000in}{0.000000in}}{%
\pgfpathmoveto{\pgfqpoint{0.000000in}{0.000000in}}%
\pgfpathlineto{\pgfqpoint{0.000000in}{-0.048611in}}%
\pgfusepath{stroke,fill}%
}%
\begin{pgfscope}%
\pgfsys@transformshift{3.134875in}{0.375000in}%
\pgfsys@useobject{currentmarker}{}%
\end{pgfscope}%
\end{pgfscope}%
\begin{pgfscope}%
\definecolor{textcolor}{rgb}{0.000000,0.000000,0.000000}%
\pgfsetstrokecolor{textcolor}%
\pgfsetfillcolor{textcolor}%
\pgftext[x=3.134875in,y=0.277778in,,top]{\color{textcolor}\rmfamily\fontsize{10.000000}{12.000000}\selectfont \(\displaystyle {500}\)}%
\end{pgfscope}%
\begin{pgfscope}%
\definecolor{textcolor}{rgb}{0.000000,0.000000,0.000000}%
\pgfsetstrokecolor{textcolor}%
\pgfsetfillcolor{textcolor}%
\pgftext[x=2.050000in,y=0.098766in,,top]{\color{textcolor}\rmfamily\fontsize{10.000000}{12.000000}\selectfont time}%
\end{pgfscope}%
\begin{pgfscope}%
\pgfsetbuttcap%
\pgfsetroundjoin%
\definecolor{currentfill}{rgb}{0.000000,0.000000,0.000000}%
\pgfsetfillcolor{currentfill}%
\pgfsetlinewidth{0.803000pt}%
\definecolor{currentstroke}{rgb}{0.000000,0.000000,0.000000}%
\pgfsetstrokecolor{currentstroke}%
\pgfsetdash{}{0pt}%
\pgfsys@defobject{currentmarker}{\pgfqpoint{-0.048611in}{0.000000in}}{\pgfqpoint{-0.000000in}{0.000000in}}{%
\pgfpathmoveto{\pgfqpoint{-0.000000in}{0.000000in}}%
\pgfpathlineto{\pgfqpoint{-0.048611in}{0.000000in}}%
\pgfusepath{stroke,fill}%
}%
\begin{pgfscope}%
\pgfsys@transformshift{0.500000in}{0.683744in}%
\pgfsys@useobject{currentmarker}{}%
\end{pgfscope}%
\end{pgfscope}%
\begin{pgfscope}%
\definecolor{textcolor}{rgb}{0.000000,0.000000,0.000000}%
\pgfsetstrokecolor{textcolor}%
\pgfsetfillcolor{textcolor}%
\pgftext[x=0.124999in, y=0.635519in, left, base]{\color{textcolor}\rmfamily\fontsize{10.000000}{12.000000}\selectfont \(\displaystyle {5000}\)}%
\end{pgfscope}%
\begin{pgfscope}%
\pgfsetbuttcap%
\pgfsetroundjoin%
\definecolor{currentfill}{rgb}{0.000000,0.000000,0.000000}%
\pgfsetfillcolor{currentfill}%
\pgfsetlinewidth{0.803000pt}%
\definecolor{currentstroke}{rgb}{0.000000,0.000000,0.000000}%
\pgfsetstrokecolor{currentstroke}%
\pgfsetdash{}{0pt}%
\pgfsys@defobject{currentmarker}{\pgfqpoint{-0.048611in}{0.000000in}}{\pgfqpoint{-0.000000in}{0.000000in}}{%
\pgfpathmoveto{\pgfqpoint{-0.000000in}{0.000000in}}%
\pgfpathlineto{\pgfqpoint{-0.048611in}{0.000000in}}%
\pgfusepath{stroke,fill}%
}%
\begin{pgfscope}%
\pgfsys@transformshift{0.500000in}{1.021765in}%
\pgfsys@useobject{currentmarker}{}%
\end{pgfscope}%
\end{pgfscope}%
\begin{pgfscope}%
\definecolor{textcolor}{rgb}{0.000000,0.000000,0.000000}%
\pgfsetstrokecolor{textcolor}%
\pgfsetfillcolor{textcolor}%
\pgftext[x=0.124999in, y=0.973540in, left, base]{\color{textcolor}\rmfamily\fontsize{10.000000}{12.000000}\selectfont \(\displaystyle {7500}\)}%
\end{pgfscope}%
\begin{pgfscope}%
\pgfsetbuttcap%
\pgfsetroundjoin%
\definecolor{currentfill}{rgb}{0.000000,0.000000,0.000000}%
\pgfsetfillcolor{currentfill}%
\pgfsetlinewidth{0.803000pt}%
\definecolor{currentstroke}{rgb}{0.000000,0.000000,0.000000}%
\pgfsetstrokecolor{currentstroke}%
\pgfsetdash{}{0pt}%
\pgfsys@defobject{currentmarker}{\pgfqpoint{-0.048611in}{0.000000in}}{\pgfqpoint{-0.000000in}{0.000000in}}{%
\pgfpathmoveto{\pgfqpoint{-0.000000in}{0.000000in}}%
\pgfpathlineto{\pgfqpoint{-0.048611in}{0.000000in}}%
\pgfusepath{stroke,fill}%
}%
\begin{pgfscope}%
\pgfsys@transformshift{0.500000in}{1.359787in}%
\pgfsys@useobject{currentmarker}{}%
\end{pgfscope}%
\end{pgfscope}%
\begin{pgfscope}%
\definecolor{textcolor}{rgb}{0.000000,0.000000,0.000000}%
\pgfsetstrokecolor{textcolor}%
\pgfsetfillcolor{textcolor}%
\pgftext[x=0.055554in, y=1.311561in, left, base]{\color{textcolor}\rmfamily\fontsize{10.000000}{12.000000}\selectfont \(\displaystyle {10000}\)}%
\end{pgfscope}%
\begin{pgfscope}%
\pgfsetbuttcap%
\pgfsetroundjoin%
\definecolor{currentfill}{rgb}{0.000000,0.000000,0.000000}%
\pgfsetfillcolor{currentfill}%
\pgfsetlinewidth{0.803000pt}%
\definecolor{currentstroke}{rgb}{0.000000,0.000000,0.000000}%
\pgfsetstrokecolor{currentstroke}%
\pgfsetdash{}{0pt}%
\pgfsys@defobject{currentmarker}{\pgfqpoint{-0.048611in}{0.000000in}}{\pgfqpoint{-0.000000in}{0.000000in}}{%
\pgfpathmoveto{\pgfqpoint{-0.000000in}{0.000000in}}%
\pgfpathlineto{\pgfqpoint{-0.048611in}{0.000000in}}%
\pgfusepath{stroke,fill}%
}%
\begin{pgfscope}%
\pgfsys@transformshift{0.500000in}{1.697808in}%
\pgfsys@useobject{currentmarker}{}%
\end{pgfscope}%
\end{pgfscope}%
\begin{pgfscope}%
\definecolor{textcolor}{rgb}{0.000000,0.000000,0.000000}%
\pgfsetstrokecolor{textcolor}%
\pgfsetfillcolor{textcolor}%
\pgftext[x=0.055554in, y=1.649583in, left, base]{\color{textcolor}\rmfamily\fontsize{10.000000}{12.000000}\selectfont \(\displaystyle {12500}\)}%
\end{pgfscope}%
\begin{pgfscope}%
\pgfsetbuttcap%
\pgfsetroundjoin%
\definecolor{currentfill}{rgb}{0.000000,0.000000,0.000000}%
\pgfsetfillcolor{currentfill}%
\pgfsetlinewidth{0.803000pt}%
\definecolor{currentstroke}{rgb}{0.000000,0.000000,0.000000}%
\pgfsetstrokecolor{currentstroke}%
\pgfsetdash{}{0pt}%
\pgfsys@defobject{currentmarker}{\pgfqpoint{-0.048611in}{0.000000in}}{\pgfqpoint{-0.000000in}{0.000000in}}{%
\pgfpathmoveto{\pgfqpoint{-0.000000in}{0.000000in}}%
\pgfpathlineto{\pgfqpoint{-0.048611in}{0.000000in}}%
\pgfusepath{stroke,fill}%
}%
\begin{pgfscope}%
\pgfsys@transformshift{0.500000in}{2.035829in}%
\pgfsys@useobject{currentmarker}{}%
\end{pgfscope}%
\end{pgfscope}%
\begin{pgfscope}%
\definecolor{textcolor}{rgb}{0.000000,0.000000,0.000000}%
\pgfsetstrokecolor{textcolor}%
\pgfsetfillcolor{textcolor}%
\pgftext[x=0.055554in, y=1.987604in, left, base]{\color{textcolor}\rmfamily\fontsize{10.000000}{12.000000}\selectfont \(\displaystyle {15000}\)}%
\end{pgfscope}%
\begin{pgfscope}%
\pgfsetbuttcap%
\pgfsetroundjoin%
\definecolor{currentfill}{rgb}{0.000000,0.000000,0.000000}%
\pgfsetfillcolor{currentfill}%
\pgfsetlinewidth{0.803000pt}%
\definecolor{currentstroke}{rgb}{0.000000,0.000000,0.000000}%
\pgfsetstrokecolor{currentstroke}%
\pgfsetdash{}{0pt}%
\pgfsys@defobject{currentmarker}{\pgfqpoint{-0.048611in}{0.000000in}}{\pgfqpoint{-0.000000in}{0.000000in}}{%
\pgfpathmoveto{\pgfqpoint{-0.000000in}{0.000000in}}%
\pgfpathlineto{\pgfqpoint{-0.048611in}{0.000000in}}%
\pgfusepath{stroke,fill}%
}%
\begin{pgfscope}%
\pgfsys@transformshift{0.500000in}{2.373850in}%
\pgfsys@useobject{currentmarker}{}%
\end{pgfscope}%
\end{pgfscope}%
\begin{pgfscope}%
\definecolor{textcolor}{rgb}{0.000000,0.000000,0.000000}%
\pgfsetstrokecolor{textcolor}%
\pgfsetfillcolor{textcolor}%
\pgftext[x=0.055554in, y=2.325625in, left, base]{\color{textcolor}\rmfamily\fontsize{10.000000}{12.000000}\selectfont \(\displaystyle {17500}\)}%
\end{pgfscope}%
\begin{pgfscope}%
\pgfpathrectangle{\pgfqpoint{0.500000in}{0.375000in}}{\pgfqpoint{3.100000in}{2.265000in}}%
\pgfusepath{clip}%
\pgfsetrectcap%
\pgfsetroundjoin%
\pgfsetlinewidth{1.505625pt}%
\definecolor{currentstroke}{rgb}{0.121569,0.466667,0.705882}%
\pgfsetstrokecolor{currentstroke}%
\pgfsetdash{}{0pt}%
\pgfpathmoveto{\pgfqpoint{0.640909in}{1.553591in}}%
\pgfpathlineto{\pgfqpoint{0.845414in}{1.554667in}}%
\pgfpathlineto{\pgfqpoint{0.855390in}{1.558366in}}%
\pgfpathlineto{\pgfqpoint{0.865366in}{1.565862in}}%
\pgfpathlineto{\pgfqpoint{0.875342in}{1.576770in}}%
\pgfpathlineto{\pgfqpoint{0.890306in}{1.599336in}}%
\pgfpathlineto{\pgfqpoint{0.905270in}{1.628995in}}%
\pgfpathlineto{\pgfqpoint{0.915245in}{1.658081in}}%
\pgfpathlineto{\pgfqpoint{0.925221in}{1.698967in}}%
\pgfpathlineto{\pgfqpoint{0.935197in}{1.756032in}}%
\pgfpathlineto{\pgfqpoint{0.945173in}{1.836049in}}%
\pgfpathlineto{\pgfqpoint{0.955149in}{1.949514in}}%
\pgfpathlineto{\pgfqpoint{0.975101in}{2.270268in}}%
\pgfpathlineto{\pgfqpoint{0.985076in}{2.394006in}}%
\pgfpathlineto{\pgfqpoint{0.995052in}{2.478397in}}%
\pgfpathlineto{\pgfqpoint{1.000040in}{2.497636in}}%
\pgfpathlineto{\pgfqpoint{1.010016in}{2.498465in}}%
\pgfpathlineto{\pgfqpoint{1.019992in}{2.498524in}}%
\pgfpathlineto{\pgfqpoint{1.024980in}{2.495775in}}%
\pgfpathlineto{\pgfqpoint{1.029968in}{2.487304in}}%
\pgfpathlineto{\pgfqpoint{1.034956in}{2.473225in}}%
\pgfpathlineto{\pgfqpoint{1.039944in}{2.443436in}}%
\pgfpathlineto{\pgfqpoint{1.044932in}{2.393278in}}%
\pgfpathlineto{\pgfqpoint{1.054907in}{2.192517in}}%
\pgfpathlineto{\pgfqpoint{1.069871in}{1.638608in}}%
\pgfpathlineto{\pgfqpoint{1.084835in}{1.099380in}}%
\pgfpathlineto{\pgfqpoint{1.094811in}{0.871234in}}%
\pgfpathlineto{\pgfqpoint{1.099799in}{0.814645in}}%
\pgfpathlineto{\pgfqpoint{1.104787in}{0.799602in}}%
\pgfpathlineto{\pgfqpoint{1.109775in}{0.825686in}}%
\pgfpathlineto{\pgfqpoint{1.114763in}{0.890258in}}%
\pgfpathlineto{\pgfqpoint{1.124739in}{1.114475in}}%
\pgfpathlineto{\pgfqpoint{1.154666in}{1.980555in}}%
\pgfpathlineto{\pgfqpoint{1.159654in}{2.068816in}}%
\pgfpathlineto{\pgfqpoint{1.164642in}{2.089380in}}%
\pgfpathlineto{\pgfqpoint{1.169630in}{2.089380in}}%
\pgfpathlineto{\pgfqpoint{1.174618in}{2.104671in}}%
\pgfpathlineto{\pgfqpoint{1.179606in}{2.130020in}}%
\pgfpathlineto{\pgfqpoint{1.184594in}{2.144837in}}%
\pgfpathlineto{\pgfqpoint{1.189582in}{2.147162in}}%
\pgfpathlineto{\pgfqpoint{1.194570in}{2.117713in}}%
\pgfpathlineto{\pgfqpoint{1.199558in}{2.055644in}}%
\pgfpathlineto{\pgfqpoint{1.209533in}{1.854608in}}%
\pgfpathlineto{\pgfqpoint{1.234473in}{1.242823in}}%
\pgfpathlineto{\pgfqpoint{1.244449in}{1.094440in}}%
\pgfpathlineto{\pgfqpoint{1.249437in}{1.071220in}}%
\pgfpathlineto{\pgfqpoint{1.254425in}{1.090740in}}%
\pgfpathlineto{\pgfqpoint{1.259413in}{1.149608in}}%
\pgfpathlineto{\pgfqpoint{1.269389in}{1.355824in}}%
\pgfpathlineto{\pgfqpoint{1.284352in}{1.726596in}}%
\pgfpathlineto{\pgfqpoint{1.294328in}{1.890062in}}%
\pgfpathlineto{\pgfqpoint{1.299316in}{1.924714in}}%
\pgfpathlineto{\pgfqpoint{1.304304in}{1.924802in}}%
\pgfpathlineto{\pgfqpoint{1.309292in}{1.891833in}}%
\pgfpathlineto{\pgfqpoint{1.314280in}{1.830133in}}%
\pgfpathlineto{\pgfqpoint{1.324256in}{1.648659in}}%
\pgfpathlineto{\pgfqpoint{1.339220in}{1.362651in}}%
\pgfpathlineto{\pgfqpoint{1.344208in}{1.296812in}}%
\pgfpathlineto{\pgfqpoint{1.349195in}{1.255520in}}%
\pgfpathlineto{\pgfqpoint{1.354183in}{1.241388in}}%
\pgfpathlineto{\pgfqpoint{1.359171in}{1.254559in}}%
\pgfpathlineto{\pgfqpoint{1.364159in}{1.292770in}}%
\pgfpathlineto{\pgfqpoint{1.374135in}{1.425145in}}%
\pgfpathlineto{\pgfqpoint{1.384111in}{1.587286in}}%
\pgfpathlineto{\pgfqpoint{1.389099in}{1.605841in}}%
\pgfpathlineto{\pgfqpoint{1.399075in}{1.605841in}}%
\pgfpathlineto{\pgfqpoint{1.404063in}{1.629112in}}%
\pgfpathlineto{\pgfqpoint{1.419027in}{1.758842in}}%
\pgfpathlineto{\pgfqpoint{1.424014in}{1.783765in}}%
\pgfpathlineto{\pgfqpoint{1.429002in}{1.784741in}}%
\pgfpathlineto{\pgfqpoint{1.433990in}{1.762746in}}%
\pgfpathlineto{\pgfqpoint{1.438978in}{1.720902in}}%
\pgfpathlineto{\pgfqpoint{1.443966in}{1.664072in}}%
\pgfpathlineto{\pgfqpoint{1.448954in}{1.584205in}}%
\pgfpathlineto{\pgfqpoint{1.453942in}{1.539548in}}%
\pgfpathlineto{\pgfqpoint{1.463918in}{1.539548in}}%
\pgfpathlineto{\pgfqpoint{1.468906in}{1.535065in}}%
\pgfpathlineto{\pgfqpoint{1.483870in}{1.504807in}}%
\pgfpathlineto{\pgfqpoint{1.488858in}{1.489044in}}%
\pgfpathlineto{\pgfqpoint{1.493846in}{1.466598in}}%
\pgfpathlineto{\pgfqpoint{1.498833in}{1.430523in}}%
\pgfpathlineto{\pgfqpoint{1.503821in}{1.365978in}}%
\pgfpathlineto{\pgfqpoint{1.518785in}{0.981175in}}%
\pgfpathlineto{\pgfqpoint{1.533749in}{0.624780in}}%
\pgfpathlineto{\pgfqpoint{1.543725in}{0.489069in}}%
\pgfpathlineto{\pgfqpoint{1.548713in}{0.477955in}}%
\pgfpathlineto{\pgfqpoint{1.578640in}{0.480204in}}%
\pgfpathlineto{\pgfqpoint{1.588616in}{0.482879in}}%
\pgfpathlineto{\pgfqpoint{1.593604in}{0.482879in}}%
\pgfpathlineto{\pgfqpoint{1.598592in}{0.486571in}}%
\pgfpathlineto{\pgfqpoint{1.608568in}{0.487481in}}%
\pgfpathlineto{\pgfqpoint{1.613556in}{0.493581in}}%
\pgfpathlineto{\pgfqpoint{1.623532in}{0.494757in}}%
\pgfpathlineto{\pgfqpoint{1.628520in}{0.504079in}}%
\pgfpathlineto{\pgfqpoint{1.638496in}{0.505560in}}%
\pgfpathlineto{\pgfqpoint{1.643484in}{0.519014in}}%
\pgfpathlineto{\pgfqpoint{1.648471in}{0.520835in}}%
\pgfpathlineto{\pgfqpoint{1.653459in}{0.520835in}}%
\pgfpathlineto{\pgfqpoint{1.658447in}{0.539408in}}%
\pgfpathlineto{\pgfqpoint{1.663435in}{0.541601in}}%
\pgfpathlineto{\pgfqpoint{1.668423in}{0.541601in}}%
\pgfpathlineto{\pgfqpoint{1.673411in}{0.564077in}}%
\pgfpathlineto{\pgfqpoint{1.678399in}{0.566653in}}%
\pgfpathlineto{\pgfqpoint{1.683387in}{0.566653in}}%
\pgfpathlineto{\pgfqpoint{1.688375in}{0.592867in}}%
\pgfpathlineto{\pgfqpoint{1.693363in}{0.595820in}}%
\pgfpathlineto{\pgfqpoint{1.698351in}{0.595820in}}%
\pgfpathlineto{\pgfqpoint{1.703339in}{0.625889in}}%
\pgfpathlineto{\pgfqpoint{1.708327in}{0.629211in}}%
\pgfpathlineto{\pgfqpoint{1.713315in}{0.629211in}}%
\pgfpathlineto{\pgfqpoint{1.718302in}{0.663290in}}%
\pgfpathlineto{\pgfqpoint{1.723290in}{0.666971in}}%
\pgfpathlineto{\pgfqpoint{1.728278in}{0.666971in}}%
\pgfpathlineto{\pgfqpoint{1.733266in}{0.705257in}}%
\pgfpathlineto{\pgfqpoint{1.738254in}{0.709286in}}%
\pgfpathlineto{\pgfqpoint{1.743242in}{0.709286in}}%
\pgfpathlineto{\pgfqpoint{1.748230in}{0.752035in}}%
\pgfpathlineto{\pgfqpoint{1.753218in}{0.756399in}}%
\pgfpathlineto{\pgfqpoint{1.758206in}{0.756399in}}%
\pgfpathlineto{\pgfqpoint{1.763194in}{0.803098in}}%
\pgfpathlineto{\pgfqpoint{1.768182in}{0.807782in}}%
\pgfpathlineto{\pgfqpoint{1.773170in}{0.807782in}}%
\pgfpathlineto{\pgfqpoint{1.778158in}{0.855847in}}%
\pgfpathlineto{\pgfqpoint{1.783146in}{0.860832in}}%
\pgfpathlineto{\pgfqpoint{1.788134in}{0.860832in}}%
\pgfpathlineto{\pgfqpoint{1.793121in}{0.910158in}}%
\pgfpathlineto{\pgfqpoint{1.798109in}{0.915415in}}%
\pgfpathlineto{\pgfqpoint{1.803097in}{0.915415in}}%
\pgfpathlineto{\pgfqpoint{1.808085in}{0.965837in}}%
\pgfpathlineto{\pgfqpoint{1.813073in}{0.971338in}}%
\pgfpathlineto{\pgfqpoint{1.818061in}{0.971338in}}%
\pgfpathlineto{\pgfqpoint{1.823049in}{1.022712in}}%
\pgfpathlineto{\pgfqpoint{1.828037in}{1.128510in}}%
\pgfpathlineto{\pgfqpoint{1.833025in}{1.140067in}}%
\pgfpathlineto{\pgfqpoint{1.838013in}{1.140067in}}%
\pgfpathlineto{\pgfqpoint{1.843001in}{1.188598in}}%
\pgfpathlineto{\pgfqpoint{1.852977in}{1.388239in}}%
\pgfpathlineto{\pgfqpoint{1.857965in}{1.544707in}}%
\pgfpathlineto{\pgfqpoint{1.862953in}{1.559231in}}%
\pgfpathlineto{\pgfqpoint{1.867940in}{1.559231in}}%
\pgfpathlineto{\pgfqpoint{1.872928in}{1.608560in}}%
\pgfpathlineto{\pgfqpoint{1.877916in}{1.615190in}}%
\pgfpathlineto{\pgfqpoint{1.882904in}{1.615190in}}%
\pgfpathlineto{\pgfqpoint{1.887892in}{1.669940in}}%
\pgfpathlineto{\pgfqpoint{1.892880in}{1.676471in}}%
\pgfpathlineto{\pgfqpoint{1.897868in}{1.676471in}}%
\pgfpathlineto{\pgfqpoint{1.902856in}{1.730937in}}%
\pgfpathlineto{\pgfqpoint{1.907844in}{1.737393in}}%
\pgfpathlineto{\pgfqpoint{1.912832in}{1.737393in}}%
\pgfpathlineto{\pgfqpoint{1.917820in}{1.791448in}}%
\pgfpathlineto{\pgfqpoint{1.922808in}{1.797805in}}%
\pgfpathlineto{\pgfqpoint{1.927796in}{1.797805in}}%
\pgfpathlineto{\pgfqpoint{1.932784in}{1.851342in}}%
\pgfpathlineto{\pgfqpoint{1.937772in}{1.857575in}}%
\pgfpathlineto{\pgfqpoint{1.942759in}{1.857575in}}%
\pgfpathlineto{\pgfqpoint{1.947747in}{1.910482in}}%
\pgfpathlineto{\pgfqpoint{1.952735in}{1.916566in}}%
\pgfpathlineto{\pgfqpoint{1.957723in}{1.916566in}}%
\pgfpathlineto{\pgfqpoint{1.962711in}{1.968725in}}%
\pgfpathlineto{\pgfqpoint{1.967699in}{1.974635in}}%
\pgfpathlineto{\pgfqpoint{1.972687in}{1.974635in}}%
\pgfpathlineto{\pgfqpoint{1.977675in}{2.025921in}}%
\pgfpathlineto{\pgfqpoint{1.982663in}{2.031631in}}%
\pgfpathlineto{\pgfqpoint{1.987651in}{2.031631in}}%
\pgfpathlineto{\pgfqpoint{1.992639in}{2.081908in}}%
\pgfpathlineto{\pgfqpoint{1.997627in}{2.087391in}}%
\pgfpathlineto{\pgfqpoint{2.002615in}{2.087391in}}%
\pgfpathlineto{\pgfqpoint{2.007603in}{2.136509in}}%
\pgfpathlineto{\pgfqpoint{2.012591in}{2.141738in}}%
\pgfpathlineto{\pgfqpoint{2.017578in}{2.141738in}}%
\pgfpathlineto{\pgfqpoint{2.022566in}{2.189531in}}%
\pgfpathlineto{\pgfqpoint{2.027554in}{2.194476in}}%
\pgfpathlineto{\pgfqpoint{2.032542in}{2.194476in}}%
\pgfpathlineto{\pgfqpoint{2.037530in}{2.240755in}}%
\pgfpathlineto{\pgfqpoint{2.042518in}{2.245387in}}%
\pgfpathlineto{\pgfqpoint{2.057482in}{2.245387in}}%
\pgfpathlineto{\pgfqpoint{2.062470in}{2.289485in}}%
\pgfpathlineto{\pgfqpoint{2.067458in}{2.296278in}}%
\pgfpathlineto{\pgfqpoint{2.072446in}{2.296278in}}%
\pgfpathlineto{\pgfqpoint{2.077434in}{2.335490in}}%
\pgfpathlineto{\pgfqpoint{2.082422in}{2.339429in}}%
\pgfpathlineto{\pgfqpoint{2.087409in}{2.339429in}}%
\pgfpathlineto{\pgfqpoint{2.092397in}{2.374258in}}%
\pgfpathlineto{\pgfqpoint{2.097385in}{2.377849in}}%
\pgfpathlineto{\pgfqpoint{2.102373in}{2.377849in}}%
\pgfpathlineto{\pgfqpoint{2.107361in}{2.408576in}}%
\pgfpathlineto{\pgfqpoint{2.112349in}{2.411810in}}%
\pgfpathlineto{\pgfqpoint{2.117337in}{2.411810in}}%
\pgfpathlineto{\pgfqpoint{2.122325in}{2.438607in}}%
\pgfpathlineto{\pgfqpoint{2.127313in}{2.441477in}}%
\pgfpathlineto{\pgfqpoint{2.132301in}{2.441477in}}%
\pgfpathlineto{\pgfqpoint{2.137289in}{2.464482in}}%
\pgfpathlineto{\pgfqpoint{2.142277in}{2.466980in}}%
\pgfpathlineto{\pgfqpoint{2.147265in}{2.466980in}}%
\pgfpathlineto{\pgfqpoint{2.152253in}{2.486304in}}%
\pgfpathlineto{\pgfqpoint{2.157241in}{2.488426in}}%
\pgfpathlineto{\pgfqpoint{2.162228in}{2.488426in}}%
\pgfpathlineto{\pgfqpoint{2.167216in}{2.504182in}}%
\pgfpathlineto{\pgfqpoint{2.172204in}{2.505921in}}%
\pgfpathlineto{\pgfqpoint{2.177192in}{2.505921in}}%
\pgfpathlineto{\pgfqpoint{2.182180in}{2.518157in}}%
\pgfpathlineto{\pgfqpoint{2.192156in}{2.519510in}}%
\pgfpathlineto{\pgfqpoint{2.197144in}{2.528250in}}%
\pgfpathlineto{\pgfqpoint{2.207120in}{2.529214in}}%
\pgfpathlineto{\pgfqpoint{2.212108in}{2.534480in}}%
\pgfpathlineto{\pgfqpoint{2.237047in}{2.537045in}}%
\pgfpathlineto{\pgfqpoint{2.247023in}{2.535206in}}%
\pgfpathlineto{\pgfqpoint{2.252011in}{2.535206in}}%
\pgfpathlineto{\pgfqpoint{2.256999in}{2.530132in}}%
\pgfpathlineto{\pgfqpoint{2.266975in}{2.529535in}}%
\pgfpathlineto{\pgfqpoint{2.271963in}{2.521004in}}%
\pgfpathlineto{\pgfqpoint{2.281939in}{2.520021in}}%
\pgfpathlineto{\pgfqpoint{2.286927in}{2.508005in}}%
\pgfpathlineto{\pgfqpoint{2.296903in}{2.506639in}}%
\pgfpathlineto{\pgfqpoint{2.301891in}{2.491097in}}%
\pgfpathlineto{\pgfqpoint{2.306879in}{2.489353in}}%
\pgfpathlineto{\pgfqpoint{2.311866in}{2.489353in}}%
\pgfpathlineto{\pgfqpoint{2.316854in}{2.470226in}}%
\pgfpathlineto{\pgfqpoint{2.321842in}{2.468110in}}%
\pgfpathlineto{\pgfqpoint{2.326830in}{2.468110in}}%
\pgfpathlineto{\pgfqpoint{2.331818in}{2.445318in}}%
\pgfpathlineto{\pgfqpoint{2.336806in}{2.442836in}}%
\pgfpathlineto{\pgfqpoint{2.341794in}{2.442836in}}%
\pgfpathlineto{\pgfqpoint{2.346782in}{2.416276in}}%
\pgfpathlineto{\pgfqpoint{2.356758in}{2.286860in}}%
\pgfpathlineto{\pgfqpoint{2.361746in}{2.160168in}}%
\pgfpathlineto{\pgfqpoint{2.376710in}{1.538831in}}%
\pgfpathlineto{\pgfqpoint{2.391673in}{0.921117in}}%
\pgfpathlineto{\pgfqpoint{2.401649in}{0.653146in}}%
\pgfpathlineto{\pgfqpoint{2.406637in}{0.614780in}}%
\pgfpathlineto{\pgfqpoint{2.416613in}{0.614780in}}%
\pgfpathlineto{\pgfqpoint{2.421601in}{0.601602in}}%
\pgfpathlineto{\pgfqpoint{2.426589in}{0.580598in}}%
\pgfpathlineto{\pgfqpoint{2.431577in}{0.570398in}}%
\pgfpathlineto{\pgfqpoint{2.436565in}{0.575772in}}%
\pgfpathlineto{\pgfqpoint{2.441553in}{0.620562in}}%
\pgfpathlineto{\pgfqpoint{2.446541in}{0.706293in}}%
\pgfpathlineto{\pgfqpoint{2.456516in}{0.983807in}}%
\pgfpathlineto{\pgfqpoint{2.471480in}{1.563920in}}%
\pgfpathlineto{\pgfqpoint{2.486444in}{2.134811in}}%
\pgfpathlineto{\pgfqpoint{2.496420in}{2.397691in}}%
\pgfpathlineto{\pgfqpoint{2.501408in}{2.401192in}}%
\pgfpathlineto{\pgfqpoint{2.506396in}{2.401192in}}%
\pgfpathlineto{\pgfqpoint{2.511384in}{2.423049in}}%
\pgfpathlineto{\pgfqpoint{2.521360in}{2.489124in}}%
\pgfpathlineto{\pgfqpoint{2.526348in}{2.509790in}}%
\pgfpathlineto{\pgfqpoint{2.531335in}{2.498339in}}%
\pgfpathlineto{\pgfqpoint{2.536323in}{2.443774in}}%
\pgfpathlineto{\pgfqpoint{2.541311in}{2.348667in}}%
\pgfpathlineto{\pgfqpoint{2.551287in}{2.055932in}}%
\pgfpathlineto{\pgfqpoint{2.586203in}{0.772215in}}%
\pgfpathlineto{\pgfqpoint{2.591191in}{0.668109in}}%
\pgfpathlineto{\pgfqpoint{2.596179in}{0.603539in}}%
\pgfpathlineto{\pgfqpoint{2.601167in}{0.581461in}}%
\pgfpathlineto{\pgfqpoint{2.606154in}{0.602919in}}%
\pgfpathlineto{\pgfqpoint{2.611142in}{0.666989in}}%
\pgfpathlineto{\pgfqpoint{2.616130in}{0.770804in}}%
\pgfpathlineto{\pgfqpoint{2.626106in}{1.077304in}}%
\pgfpathlineto{\pgfqpoint{2.661022in}{2.343648in}}%
\pgfpathlineto{\pgfqpoint{2.666010in}{2.436689in}}%
\pgfpathlineto{\pgfqpoint{2.670998in}{2.489141in}}%
\pgfpathlineto{\pgfqpoint{2.675986in}{2.498732in}}%
\pgfpathlineto{\pgfqpoint{2.680973in}{2.465138in}}%
\pgfpathlineto{\pgfqpoint{2.685961in}{2.389986in}}%
\pgfpathlineto{\pgfqpoint{2.695937in}{2.130645in}}%
\pgfpathlineto{\pgfqpoint{2.710901in}{1.566640in}}%
\pgfpathlineto{\pgfqpoint{2.725865in}{0.994858in}}%
\pgfpathlineto{\pgfqpoint{2.735841in}{0.722734in}}%
\pgfpathlineto{\pgfqpoint{2.740829in}{0.639541in}}%
\pgfpathlineto{\pgfqpoint{2.745817in}{0.597192in}}%
\pgfpathlineto{\pgfqpoint{2.750805in}{0.597647in}}%
\pgfpathlineto{\pgfqpoint{2.755792in}{0.640929in}}%
\pgfpathlineto{\pgfqpoint{2.760780in}{0.725120in}}%
\pgfpathlineto{\pgfqpoint{2.770756in}{0.999370in}}%
\pgfpathlineto{\pgfqpoint{2.785720in}{1.573078in}}%
\pgfpathlineto{\pgfqpoint{2.790708in}{1.773486in}}%
\pgfpathlineto{\pgfqpoint{2.795696in}{1.882623in}}%
\pgfpathlineto{\pgfqpoint{2.805672in}{1.882623in}}%
\pgfpathlineto{\pgfqpoint{2.810660in}{1.926535in}}%
\pgfpathlineto{\pgfqpoint{2.820636in}{2.092897in}}%
\pgfpathlineto{\pgfqpoint{2.835599in}{2.441423in}}%
\pgfpathlineto{\pgfqpoint{2.840587in}{2.485101in}}%
\pgfpathlineto{\pgfqpoint{2.845575in}{2.479320in}}%
\pgfpathlineto{\pgfqpoint{2.850563in}{2.424519in}}%
\pgfpathlineto{\pgfqpoint{2.855551in}{2.323693in}}%
\pgfpathlineto{\pgfqpoint{2.865527in}{2.007575in}}%
\pgfpathlineto{\pgfqpoint{2.895455in}{0.830488in}}%
\pgfpathlineto{\pgfqpoint{2.900442in}{0.709913in}}%
\pgfpathlineto{\pgfqpoint{2.905430in}{0.632999in}}%
\pgfpathlineto{\pgfqpoint{2.910418in}{0.603818in}}%
\pgfpathlineto{\pgfqpoint{2.915406in}{0.623953in}}%
\pgfpathlineto{\pgfqpoint{2.920394in}{0.692401in}}%
\pgfpathlineto{\pgfqpoint{2.925382in}{0.805608in}}%
\pgfpathlineto{\pgfqpoint{2.935358in}{1.140491in}}%
\pgfpathlineto{\pgfqpoint{2.960298in}{2.151622in}}%
\pgfpathlineto{\pgfqpoint{2.970274in}{2.403751in}}%
\pgfpathlineto{\pgfqpoint{2.975261in}{2.464527in}}%
\pgfpathlineto{\pgfqpoint{2.980249in}{2.476877in}}%
\pgfpathlineto{\pgfqpoint{2.985237in}{2.440284in}}%
\pgfpathlineto{\pgfqpoint{2.990225in}{2.356795in}}%
\pgfpathlineto{\pgfqpoint{3.000201in}{2.069241in}}%
\pgfpathlineto{\pgfqpoint{3.015165in}{1.461149in}}%
\pgfpathlineto{\pgfqpoint{3.030129in}{0.891780in}}%
\pgfpathlineto{\pgfqpoint{3.040105in}{0.665599in}}%
\pgfpathlineto{\pgfqpoint{3.045093in}{0.619082in}}%
\pgfpathlineto{\pgfqpoint{3.050080in}{0.621060in}}%
\pgfpathlineto{\pgfqpoint{3.055068in}{0.671484in}}%
\pgfpathlineto{\pgfqpoint{3.060056in}{0.767751in}}%
\pgfpathlineto{\pgfqpoint{3.070032in}{1.075506in}}%
\pgfpathlineto{\pgfqpoint{3.099960in}{2.240508in}}%
\pgfpathlineto{\pgfqpoint{3.104948in}{2.360715in}}%
\pgfpathlineto{\pgfqpoint{3.109936in}{2.437796in}}%
\pgfpathlineto{\pgfqpoint{3.114924in}{2.467809in}}%
\pgfpathlineto{\pgfqpoint{3.119912in}{2.449307in}}%
\pgfpathlineto{\pgfqpoint{3.124899in}{2.383390in}}%
\pgfpathlineto{\pgfqpoint{3.129887in}{2.273625in}}%
\pgfpathlineto{\pgfqpoint{3.139863in}{1.947843in}}%
\pgfpathlineto{\pgfqpoint{3.169791in}{0.810891in}}%
\pgfpathlineto{\pgfqpoint{3.174779in}{0.703757in}}%
\pgfpathlineto{\pgfqpoint{3.179767in}{0.640620in}}%
\pgfpathlineto{\pgfqpoint{3.184755in}{0.624830in}}%
\pgfpathlineto{\pgfqpoint{3.189743in}{0.657269in}}%
\pgfpathlineto{\pgfqpoint{3.194730in}{0.736281in}}%
\pgfpathlineto{\pgfqpoint{3.204706in}{1.015294in}}%
\pgfpathlineto{\pgfqpoint{3.219670in}{1.614428in}}%
\pgfpathlineto{\pgfqpoint{3.234634in}{2.179612in}}%
\pgfpathlineto{\pgfqpoint{3.244610in}{2.405227in}}%
\pgfpathlineto{\pgfqpoint{3.249598in}{2.452321in}}%
\pgfpathlineto{\pgfqpoint{3.254586in}{2.451666in}}%
\pgfpathlineto{\pgfqpoint{3.259574in}{2.403424in}}%
\pgfpathlineto{\pgfqpoint{3.264562in}{2.310243in}}%
\pgfpathlineto{\pgfqpoint{3.274537in}{2.011030in}}%
\pgfpathlineto{\pgfqpoint{3.304465in}{0.867734in}}%
\pgfpathlineto{\pgfqpoint{3.309453in}{0.747077in}}%
\pgfpathlineto{\pgfqpoint{3.314441in}{0.668138in}}%
\pgfpathlineto{\pgfqpoint{3.319429in}{0.635088in}}%
\pgfpathlineto{\pgfqpoint{3.324417in}{0.649709in}}%
\pgfpathlineto{\pgfqpoint{3.329405in}{0.711282in}}%
\pgfpathlineto{\pgfqpoint{3.334393in}{0.816613in}}%
\pgfpathlineto{\pgfqpoint{3.344368in}{1.134432in}}%
\pgfpathlineto{\pgfqpoint{3.374296in}{2.260262in}}%
\pgfpathlineto{\pgfqpoint{3.379284in}{2.367176in}}%
\pgfpathlineto{\pgfqpoint{3.384272in}{2.430657in}}%
\pgfpathlineto{\pgfqpoint{3.389260in}{2.447481in}}%
\pgfpathlineto{\pgfqpoint{3.394248in}{2.416891in}}%
\pgfpathlineto{\pgfqpoint{3.399236in}{2.340612in}}%
\pgfpathlineto{\pgfqpoint{3.409212in}{2.069539in}}%
\pgfpathlineto{\pgfqpoint{3.424175in}{1.484878in}}%
\pgfpathlineto{\pgfqpoint{3.439139in}{0.928161in}}%
\pgfpathlineto{\pgfqpoint{3.449115in}{0.701309in}}%
\pgfpathlineto{\pgfqpoint{3.454103in}{0.651617in}}%
\pgfpathlineto{\pgfqpoint{3.459091in}{0.650607in}}%
\pgfpathlineto{\pgfqpoint{3.459091in}{0.650607in}}%
\pgfusepath{stroke}%
\end{pgfscope}%
\begin{pgfscope}%
\pgfpathrectangle{\pgfqpoint{0.500000in}{0.375000in}}{\pgfqpoint{3.100000in}{2.265000in}}%
\pgfusepath{clip}%
\pgfsetrectcap%
\pgfsetroundjoin%
\pgfsetlinewidth{1.505625pt}%
\definecolor{currentstroke}{rgb}{1.000000,0.000000,0.000000}%
\pgfsetstrokecolor{currentstroke}%
\pgfsetdash{}{0pt}%
\pgfpathmoveto{\pgfqpoint{0.865366in}{0.375000in}}%
\pgfpathlineto{\pgfqpoint{0.865366in}{2.640000in}}%
\pgfusepath{stroke}%
\end{pgfscope}%
\begin{pgfscope}%
\pgfpathrectangle{\pgfqpoint{0.500000in}{0.375000in}}{\pgfqpoint{3.100000in}{2.265000in}}%
\pgfusepath{clip}%
\pgfsetrectcap%
\pgfsetroundjoin%
\pgfsetlinewidth{1.505625pt}%
\definecolor{currentstroke}{rgb}{1.000000,0.000000,0.000000}%
\pgfsetstrokecolor{currentstroke}%
\pgfsetdash{}{0pt}%
\pgfpathmoveto{\pgfqpoint{1.518785in}{0.375000in}}%
\pgfpathlineto{\pgfqpoint{1.518785in}{2.640000in}}%
\pgfusepath{stroke}%
\end{pgfscope}%
\begin{pgfscope}%
\pgfpathrectangle{\pgfqpoint{0.500000in}{0.375000in}}{\pgfqpoint{3.100000in}{2.265000in}}%
\pgfusepath{clip}%
\pgfsetrectcap%
\pgfsetroundjoin%
\pgfsetlinewidth{1.505625pt}%
\definecolor{currentstroke}{rgb}{1.000000,0.000000,0.000000}%
\pgfsetstrokecolor{currentstroke}%
\pgfsetdash{}{0pt}%
\pgfpathmoveto{\pgfqpoint{1.987651in}{0.375000in}}%
\pgfpathlineto{\pgfqpoint{1.987651in}{2.640000in}}%
\pgfusepath{stroke}%
\end{pgfscope}%
\begin{pgfscope}%
\pgfpathrectangle{\pgfqpoint{0.500000in}{0.375000in}}{\pgfqpoint{3.100000in}{2.265000in}}%
\pgfusepath{clip}%
\pgfsetrectcap%
\pgfsetroundjoin%
\pgfsetlinewidth{1.505625pt}%
\definecolor{currentstroke}{rgb}{1.000000,0.000000,0.000000}%
\pgfsetstrokecolor{currentstroke}%
\pgfsetdash{}{0pt}%
\pgfpathmoveto{\pgfqpoint{2.366734in}{0.375000in}}%
\pgfpathlineto{\pgfqpoint{2.366734in}{2.640000in}}%
\pgfusepath{stroke}%
\end{pgfscope}%
\begin{pgfscope}%
\pgfsetrectcap%
\pgfsetmiterjoin%
\pgfsetlinewidth{0.803000pt}%
\definecolor{currentstroke}{rgb}{0.000000,0.000000,0.000000}%
\pgfsetstrokecolor{currentstroke}%
\pgfsetdash{}{0pt}%
\pgfpathmoveto{\pgfqpoint{0.500000in}{0.375000in}}%
\pgfpathlineto{\pgfqpoint{0.500000in}{2.640000in}}%
\pgfusepath{stroke}%
\end{pgfscope}%
\begin{pgfscope}%
\pgfsetrectcap%
\pgfsetmiterjoin%
\pgfsetlinewidth{0.803000pt}%
\definecolor{currentstroke}{rgb}{0.000000,0.000000,0.000000}%
\pgfsetstrokecolor{currentstroke}%
\pgfsetdash{}{0pt}%
\pgfpathmoveto{\pgfqpoint{3.600000in}{0.375000in}}%
\pgfpathlineto{\pgfqpoint{3.600000in}{2.640000in}}%
\pgfusepath{stroke}%
\end{pgfscope}%
\begin{pgfscope}%
\pgfsetrectcap%
\pgfsetmiterjoin%
\pgfsetlinewidth{0.803000pt}%
\definecolor{currentstroke}{rgb}{0.000000,0.000000,0.000000}%
\pgfsetstrokecolor{currentstroke}%
\pgfsetdash{}{0pt}%
\pgfpathmoveto{\pgfqpoint{0.500000in}{0.375000in}}%
\pgfpathlineto{\pgfqpoint{3.600000in}{0.375000in}}%
\pgfusepath{stroke}%
\end{pgfscope}%
\begin{pgfscope}%
\pgfsetrectcap%
\pgfsetmiterjoin%
\pgfsetlinewidth{0.803000pt}%
\definecolor{currentstroke}{rgb}{0.000000,0.000000,0.000000}%
\pgfsetstrokecolor{currentstroke}%
\pgfsetdash{}{0pt}%
\pgfpathmoveto{\pgfqpoint{0.500000in}{2.640000in}}%
\pgfpathlineto{\pgfqpoint{3.600000in}{2.640000in}}%
\pgfusepath{stroke}%
\end{pgfscope}%
\begin{pgfscope}%
\pgfsetbuttcap%
\pgfsetmiterjoin%
\definecolor{currentfill}{rgb}{1.000000,1.000000,1.000000}%
\pgfsetfillcolor{currentfill}%
\pgfsetfillopacity{0.800000}%
\pgfsetlinewidth{1.003750pt}%
\definecolor{currentstroke}{rgb}{0.800000,0.800000,0.800000}%
\pgfsetstrokecolor{currentstroke}%
\pgfsetstrokeopacity{0.800000}%
\pgfsetdash{}{0pt}%
\pgfpathmoveto{\pgfqpoint{0.597222in}{0.444444in}}%
\pgfpathlineto{\pgfqpoint{1.439430in}{0.444444in}}%
\pgfpathquadraticcurveto{\pgfqpoint{1.467208in}{0.444444in}}{\pgfqpoint{1.467208in}{0.472222in}}%
\pgfpathlineto{\pgfqpoint{1.467208in}{0.652006in}}%
\pgfpathquadraticcurveto{\pgfqpoint{1.467208in}{0.679784in}}{\pgfqpoint{1.439430in}{0.679784in}}%
\pgfpathlineto{\pgfqpoint{0.597222in}{0.679784in}}%
\pgfpathquadraticcurveto{\pgfqpoint{0.569444in}{0.679784in}}{\pgfqpoint{0.569444in}{0.652006in}}%
\pgfpathlineto{\pgfqpoint{0.569444in}{0.472222in}}%
\pgfpathquadraticcurveto{\pgfqpoint{0.569444in}{0.444444in}}{\pgfqpoint{0.597222in}{0.444444in}}%
\pgfpathlineto{\pgfqpoint{0.597222in}{0.444444in}}%
\pgfpathclose%
\pgfusepath{stroke,fill}%
\end{pgfscope}%
\begin{pgfscope}%
\pgfsetrectcap%
\pgfsetroundjoin%
\pgfsetlinewidth{1.505625pt}%
\definecolor{currentstroke}{rgb}{0.121569,0.466667,0.705882}%
\pgfsetstrokecolor{currentstroke}%
\pgfsetdash{}{0pt}%
\pgfpathmoveto{\pgfqpoint{0.625000in}{0.575617in}}%
\pgfpathlineto{\pgfqpoint{0.763889in}{0.575617in}}%
\pgfpathlineto{\pgfqpoint{0.902778in}{0.575617in}}%
\pgfusepath{stroke}%
\end{pgfscope}%
\begin{pgfscope}%
\definecolor{textcolor}{rgb}{0.000000,0.000000,0.000000}%
\pgfsetstrokecolor{textcolor}%
\pgfsetfillcolor{textcolor}%
\pgftext[x=1.013889in,y=0.527006in,left,base]{\color{textcolor}\rmfamily\fontsize{10.000000}{12.000000}\selectfont force:1}%
\end{pgfscope}%
\end{pgfpicture}%
\makeatother%
\endgroup%
}
%         \caption{Force Anomalies}
%         \label{fig:hydraulic_sim_standard_force}
%     \end{minipage}
%     \begin{minipage}[t]{0.5\linewidth}
%         %%\centering
%         \resizebox{\linewidth}{!}{%% Creator: Matplotlib, PGF backend
%%
%% To include the figure in your LaTeX document, write
%%   \input{<filename>.pgf}
%%
%% Make sure the required packages are loaded in your preamble
%%   \usepackage{pgf}
%%
%% Also ensure that all the required font packages are loaded; for instance,
%% the lmodern package is sometimes necessary when using math font.
%%   \usepackage{lmodern}
%%
%% Figures using additional raster images can only be included by \input if
%% they are in the same directory as the main LaTeX file. For loading figures
%% from other directories you can use the `import` package
%%   \usepackage{import}
%%
%% and then include the figures with
%%   \import{<path to file>}{<filename>.pgf}
%%
%% Matplotlib used the following preamble
%%
\begingroup%
\makeatletter%
\begin{pgfpicture}%
\pgfpathrectangle{\pgfpointorigin}{\pgfqpoint{4.000000in}{3.000000in}}%
\pgfusepath{use as bounding box, clip}%
\begin{pgfscope}%
\pgfsetbuttcap%
\pgfsetmiterjoin%
\pgfsetlinewidth{0.000000pt}%
\definecolor{currentstroke}{rgb}{1.000000,1.000000,1.000000}%
\pgfsetstrokecolor{currentstroke}%
\pgfsetstrokeopacity{0.000000}%
\pgfsetdash{}{0pt}%
\pgfpathmoveto{\pgfqpoint{0.000000in}{0.000000in}}%
\pgfpathlineto{\pgfqpoint{4.000000in}{0.000000in}}%
\pgfpathlineto{\pgfqpoint{4.000000in}{3.000000in}}%
\pgfpathlineto{\pgfqpoint{0.000000in}{3.000000in}}%
\pgfpathlineto{\pgfqpoint{0.000000in}{0.000000in}}%
\pgfpathclose%
\pgfusepath{}%
\end{pgfscope}%
\begin{pgfscope}%
\pgfsetbuttcap%
\pgfsetmiterjoin%
\definecolor{currentfill}{rgb}{1.000000,1.000000,1.000000}%
\pgfsetfillcolor{currentfill}%
\pgfsetlinewidth{0.000000pt}%
\definecolor{currentstroke}{rgb}{0.000000,0.000000,0.000000}%
\pgfsetstrokecolor{currentstroke}%
\pgfsetstrokeopacity{0.000000}%
\pgfsetdash{}{0pt}%
\pgfpathmoveto{\pgfqpoint{0.500000in}{0.375000in}}%
\pgfpathlineto{\pgfqpoint{3.600000in}{0.375000in}}%
\pgfpathlineto{\pgfqpoint{3.600000in}{2.640000in}}%
\pgfpathlineto{\pgfqpoint{0.500000in}{2.640000in}}%
\pgfpathlineto{\pgfqpoint{0.500000in}{0.375000in}}%
\pgfpathclose%
\pgfusepath{fill}%
\end{pgfscope}%
\begin{pgfscope}%
\pgfsetbuttcap%
\pgfsetroundjoin%
\definecolor{currentfill}{rgb}{0.000000,0.000000,0.000000}%
\pgfsetfillcolor{currentfill}%
\pgfsetlinewidth{0.803000pt}%
\definecolor{currentstroke}{rgb}{0.000000,0.000000,0.000000}%
\pgfsetstrokecolor{currentstroke}%
\pgfsetdash{}{0pt}%
\pgfsys@defobject{currentmarker}{\pgfqpoint{0.000000in}{-0.048611in}}{\pgfqpoint{0.000000in}{0.000000in}}{%
\pgfpathmoveto{\pgfqpoint{0.000000in}{0.000000in}}%
\pgfpathlineto{\pgfqpoint{0.000000in}{-0.048611in}}%
\pgfusepath{stroke,fill}%
}%
\begin{pgfscope}%
\pgfsys@transformshift{0.640909in}{0.375000in}%
\pgfsys@useobject{currentmarker}{}%
\end{pgfscope}%
\end{pgfscope}%
\begin{pgfscope}%
\definecolor{textcolor}{rgb}{0.000000,0.000000,0.000000}%
\pgfsetstrokecolor{textcolor}%
\pgfsetfillcolor{textcolor}%
\pgftext[x=0.640909in,y=0.277778in,,top]{\color{textcolor}\rmfamily\fontsize{10.000000}{12.000000}\selectfont \(\displaystyle {0}\)}%
\end{pgfscope}%
\begin{pgfscope}%
\pgfsetbuttcap%
\pgfsetroundjoin%
\definecolor{currentfill}{rgb}{0.000000,0.000000,0.000000}%
\pgfsetfillcolor{currentfill}%
\pgfsetlinewidth{0.803000pt}%
\definecolor{currentstroke}{rgb}{0.000000,0.000000,0.000000}%
\pgfsetstrokecolor{currentstroke}%
\pgfsetdash{}{0pt}%
\pgfsys@defobject{currentmarker}{\pgfqpoint{0.000000in}{-0.048611in}}{\pgfqpoint{0.000000in}{0.000000in}}{%
\pgfpathmoveto{\pgfqpoint{0.000000in}{0.000000in}}%
\pgfpathlineto{\pgfqpoint{0.000000in}{-0.048611in}}%
\pgfusepath{stroke,fill}%
}%
\begin{pgfscope}%
\pgfsys@transformshift{1.361672in}{0.375000in}%
\pgfsys@useobject{currentmarker}{}%
\end{pgfscope}%
\end{pgfscope}%
\begin{pgfscope}%
\definecolor{textcolor}{rgb}{0.000000,0.000000,0.000000}%
\pgfsetstrokecolor{textcolor}%
\pgfsetfillcolor{textcolor}%
\pgftext[x=1.361672in,y=0.277778in,,top]{\color{textcolor}\rmfamily\fontsize{10.000000}{12.000000}\selectfont \(\displaystyle {100}\)}%
\end{pgfscope}%
\begin{pgfscope}%
\pgfsetbuttcap%
\pgfsetroundjoin%
\definecolor{currentfill}{rgb}{0.000000,0.000000,0.000000}%
\pgfsetfillcolor{currentfill}%
\pgfsetlinewidth{0.803000pt}%
\definecolor{currentstroke}{rgb}{0.000000,0.000000,0.000000}%
\pgfsetstrokecolor{currentstroke}%
\pgfsetdash{}{0pt}%
\pgfsys@defobject{currentmarker}{\pgfqpoint{0.000000in}{-0.048611in}}{\pgfqpoint{0.000000in}{0.000000in}}{%
\pgfpathmoveto{\pgfqpoint{0.000000in}{0.000000in}}%
\pgfpathlineto{\pgfqpoint{0.000000in}{-0.048611in}}%
\pgfusepath{stroke,fill}%
}%
\begin{pgfscope}%
\pgfsys@transformshift{2.082434in}{0.375000in}%
\pgfsys@useobject{currentmarker}{}%
\end{pgfscope}%
\end{pgfscope}%
\begin{pgfscope}%
\definecolor{textcolor}{rgb}{0.000000,0.000000,0.000000}%
\pgfsetstrokecolor{textcolor}%
\pgfsetfillcolor{textcolor}%
\pgftext[x=2.082434in,y=0.277778in,,top]{\color{textcolor}\rmfamily\fontsize{10.000000}{12.000000}\selectfont \(\displaystyle {200}\)}%
\end{pgfscope}%
\begin{pgfscope}%
\pgfsetbuttcap%
\pgfsetroundjoin%
\definecolor{currentfill}{rgb}{0.000000,0.000000,0.000000}%
\pgfsetfillcolor{currentfill}%
\pgfsetlinewidth{0.803000pt}%
\definecolor{currentstroke}{rgb}{0.000000,0.000000,0.000000}%
\pgfsetstrokecolor{currentstroke}%
\pgfsetdash{}{0pt}%
\pgfsys@defobject{currentmarker}{\pgfqpoint{0.000000in}{-0.048611in}}{\pgfqpoint{0.000000in}{0.000000in}}{%
\pgfpathmoveto{\pgfqpoint{0.000000in}{0.000000in}}%
\pgfpathlineto{\pgfqpoint{0.000000in}{-0.048611in}}%
\pgfusepath{stroke,fill}%
}%
\begin{pgfscope}%
\pgfsys@transformshift{2.803197in}{0.375000in}%
\pgfsys@useobject{currentmarker}{}%
\end{pgfscope}%
\end{pgfscope}%
\begin{pgfscope}%
\definecolor{textcolor}{rgb}{0.000000,0.000000,0.000000}%
\pgfsetstrokecolor{textcolor}%
\pgfsetfillcolor{textcolor}%
\pgftext[x=2.803197in,y=0.277778in,,top]{\color{textcolor}\rmfamily\fontsize{10.000000}{12.000000}\selectfont \(\displaystyle {300}\)}%
\end{pgfscope}%
\begin{pgfscope}%
\pgfsetbuttcap%
\pgfsetroundjoin%
\definecolor{currentfill}{rgb}{0.000000,0.000000,0.000000}%
\pgfsetfillcolor{currentfill}%
\pgfsetlinewidth{0.803000pt}%
\definecolor{currentstroke}{rgb}{0.000000,0.000000,0.000000}%
\pgfsetstrokecolor{currentstroke}%
\pgfsetdash{}{0pt}%
\pgfsys@defobject{currentmarker}{\pgfqpoint{0.000000in}{-0.048611in}}{\pgfqpoint{0.000000in}{0.000000in}}{%
\pgfpathmoveto{\pgfqpoint{0.000000in}{0.000000in}}%
\pgfpathlineto{\pgfqpoint{0.000000in}{-0.048611in}}%
\pgfusepath{stroke,fill}%
}%
\begin{pgfscope}%
\pgfsys@transformshift{3.523960in}{0.375000in}%
\pgfsys@useobject{currentmarker}{}%
\end{pgfscope}%
\end{pgfscope}%
\begin{pgfscope}%
\definecolor{textcolor}{rgb}{0.000000,0.000000,0.000000}%
\pgfsetstrokecolor{textcolor}%
\pgfsetfillcolor{textcolor}%
\pgftext[x=3.523960in,y=0.277778in,,top]{\color{textcolor}\rmfamily\fontsize{10.000000}{12.000000}\selectfont \(\displaystyle {400}\)}%
\end{pgfscope}%
\begin{pgfscope}%
\definecolor{textcolor}{rgb}{0.000000,0.000000,0.000000}%
\pgfsetstrokecolor{textcolor}%
\pgfsetfillcolor{textcolor}%
\pgftext[x=2.050000in,y=0.098766in,,top]{\color{textcolor}\rmfamily\fontsize{10.000000}{12.000000}\selectfont time}%
\end{pgfscope}%
\begin{pgfscope}%
\pgfsetbuttcap%
\pgfsetroundjoin%
\definecolor{currentfill}{rgb}{0.000000,0.000000,0.000000}%
\pgfsetfillcolor{currentfill}%
\pgfsetlinewidth{0.803000pt}%
\definecolor{currentstroke}{rgb}{0.000000,0.000000,0.000000}%
\pgfsetstrokecolor{currentstroke}%
\pgfsetdash{}{0pt}%
\pgfsys@defobject{currentmarker}{\pgfqpoint{-0.048611in}{0.000000in}}{\pgfqpoint{-0.000000in}{0.000000in}}{%
\pgfpathmoveto{\pgfqpoint{-0.000000in}{0.000000in}}%
\pgfpathlineto{\pgfqpoint{-0.048611in}{0.000000in}}%
\pgfusepath{stroke,fill}%
}%
\begin{pgfscope}%
\pgfsys@transformshift{0.500000in}{0.687667in}%
\pgfsys@useobject{currentmarker}{}%
\end{pgfscope}%
\end{pgfscope}%
\begin{pgfscope}%
\definecolor{textcolor}{rgb}{0.000000,0.000000,0.000000}%
\pgfsetstrokecolor{textcolor}%
\pgfsetfillcolor{textcolor}%
\pgftext[x=0.155863in, y=0.639442in, left, base]{\color{textcolor}\rmfamily\fontsize{10.000000}{12.000000}\selectfont \(\displaystyle {0.50}\)}%
\end{pgfscope}%
\begin{pgfscope}%
\pgfsetbuttcap%
\pgfsetroundjoin%
\definecolor{currentfill}{rgb}{0.000000,0.000000,0.000000}%
\pgfsetfillcolor{currentfill}%
\pgfsetlinewidth{0.803000pt}%
\definecolor{currentstroke}{rgb}{0.000000,0.000000,0.000000}%
\pgfsetstrokecolor{currentstroke}%
\pgfsetdash{}{0pt}%
\pgfsys@defobject{currentmarker}{\pgfqpoint{-0.048611in}{0.000000in}}{\pgfqpoint{-0.000000in}{0.000000in}}{%
\pgfpathmoveto{\pgfqpoint{-0.000000in}{0.000000in}}%
\pgfpathlineto{\pgfqpoint{-0.048611in}{0.000000in}}%
\pgfusepath{stroke,fill}%
}%
\begin{pgfscope}%
\pgfsys@transformshift{0.500000in}{1.032133in}%
\pgfsys@useobject{currentmarker}{}%
\end{pgfscope}%
\end{pgfscope}%
\begin{pgfscope}%
\definecolor{textcolor}{rgb}{0.000000,0.000000,0.000000}%
\pgfsetstrokecolor{textcolor}%
\pgfsetfillcolor{textcolor}%
\pgftext[x=0.155863in, y=0.983908in, left, base]{\color{textcolor}\rmfamily\fontsize{10.000000}{12.000000}\selectfont \(\displaystyle {0.75}\)}%
\end{pgfscope}%
\begin{pgfscope}%
\pgfsetbuttcap%
\pgfsetroundjoin%
\definecolor{currentfill}{rgb}{0.000000,0.000000,0.000000}%
\pgfsetfillcolor{currentfill}%
\pgfsetlinewidth{0.803000pt}%
\definecolor{currentstroke}{rgb}{0.000000,0.000000,0.000000}%
\pgfsetstrokecolor{currentstroke}%
\pgfsetdash{}{0pt}%
\pgfsys@defobject{currentmarker}{\pgfqpoint{-0.048611in}{0.000000in}}{\pgfqpoint{-0.000000in}{0.000000in}}{%
\pgfpathmoveto{\pgfqpoint{-0.000000in}{0.000000in}}%
\pgfpathlineto{\pgfqpoint{-0.048611in}{0.000000in}}%
\pgfusepath{stroke,fill}%
}%
\begin{pgfscope}%
\pgfsys@transformshift{0.500000in}{1.376598in}%
\pgfsys@useobject{currentmarker}{}%
\end{pgfscope}%
\end{pgfscope}%
\begin{pgfscope}%
\definecolor{textcolor}{rgb}{0.000000,0.000000,0.000000}%
\pgfsetstrokecolor{textcolor}%
\pgfsetfillcolor{textcolor}%
\pgftext[x=0.155863in, y=1.328373in, left, base]{\color{textcolor}\rmfamily\fontsize{10.000000}{12.000000}\selectfont \(\displaystyle {1.00}\)}%
\end{pgfscope}%
\begin{pgfscope}%
\pgfsetbuttcap%
\pgfsetroundjoin%
\definecolor{currentfill}{rgb}{0.000000,0.000000,0.000000}%
\pgfsetfillcolor{currentfill}%
\pgfsetlinewidth{0.803000pt}%
\definecolor{currentstroke}{rgb}{0.000000,0.000000,0.000000}%
\pgfsetstrokecolor{currentstroke}%
\pgfsetdash{}{0pt}%
\pgfsys@defobject{currentmarker}{\pgfqpoint{-0.048611in}{0.000000in}}{\pgfqpoint{-0.000000in}{0.000000in}}{%
\pgfpathmoveto{\pgfqpoint{-0.000000in}{0.000000in}}%
\pgfpathlineto{\pgfqpoint{-0.048611in}{0.000000in}}%
\pgfusepath{stroke,fill}%
}%
\begin{pgfscope}%
\pgfsys@transformshift{0.500000in}{1.721064in}%
\pgfsys@useobject{currentmarker}{}%
\end{pgfscope}%
\end{pgfscope}%
\begin{pgfscope}%
\definecolor{textcolor}{rgb}{0.000000,0.000000,0.000000}%
\pgfsetstrokecolor{textcolor}%
\pgfsetfillcolor{textcolor}%
\pgftext[x=0.155863in, y=1.672839in, left, base]{\color{textcolor}\rmfamily\fontsize{10.000000}{12.000000}\selectfont \(\displaystyle {1.25}\)}%
\end{pgfscope}%
\begin{pgfscope}%
\pgfsetbuttcap%
\pgfsetroundjoin%
\definecolor{currentfill}{rgb}{0.000000,0.000000,0.000000}%
\pgfsetfillcolor{currentfill}%
\pgfsetlinewidth{0.803000pt}%
\definecolor{currentstroke}{rgb}{0.000000,0.000000,0.000000}%
\pgfsetstrokecolor{currentstroke}%
\pgfsetdash{}{0pt}%
\pgfsys@defobject{currentmarker}{\pgfqpoint{-0.048611in}{0.000000in}}{\pgfqpoint{-0.000000in}{0.000000in}}{%
\pgfpathmoveto{\pgfqpoint{-0.000000in}{0.000000in}}%
\pgfpathlineto{\pgfqpoint{-0.048611in}{0.000000in}}%
\pgfusepath{stroke,fill}%
}%
\begin{pgfscope}%
\pgfsys@transformshift{0.500000in}{2.065530in}%
\pgfsys@useobject{currentmarker}{}%
\end{pgfscope}%
\end{pgfscope}%
\begin{pgfscope}%
\definecolor{textcolor}{rgb}{0.000000,0.000000,0.000000}%
\pgfsetstrokecolor{textcolor}%
\pgfsetfillcolor{textcolor}%
\pgftext[x=0.155863in, y=2.017304in, left, base]{\color{textcolor}\rmfamily\fontsize{10.000000}{12.000000}\selectfont \(\displaystyle {1.50}\)}%
\end{pgfscope}%
\begin{pgfscope}%
\pgfsetbuttcap%
\pgfsetroundjoin%
\definecolor{currentfill}{rgb}{0.000000,0.000000,0.000000}%
\pgfsetfillcolor{currentfill}%
\pgfsetlinewidth{0.803000pt}%
\definecolor{currentstroke}{rgb}{0.000000,0.000000,0.000000}%
\pgfsetstrokecolor{currentstroke}%
\pgfsetdash{}{0pt}%
\pgfsys@defobject{currentmarker}{\pgfqpoint{-0.048611in}{0.000000in}}{\pgfqpoint{-0.000000in}{0.000000in}}{%
\pgfpathmoveto{\pgfqpoint{-0.000000in}{0.000000in}}%
\pgfpathlineto{\pgfqpoint{-0.048611in}{0.000000in}}%
\pgfusepath{stroke,fill}%
}%
\begin{pgfscope}%
\pgfsys@transformshift{0.500000in}{2.409995in}%
\pgfsys@useobject{currentmarker}{}%
\end{pgfscope}%
\end{pgfscope}%
\begin{pgfscope}%
\definecolor{textcolor}{rgb}{0.000000,0.000000,0.000000}%
\pgfsetstrokecolor{textcolor}%
\pgfsetfillcolor{textcolor}%
\pgftext[x=0.155863in, y=2.361770in, left, base]{\color{textcolor}\rmfamily\fontsize{10.000000}{12.000000}\selectfont \(\displaystyle {1.75}\)}%
\end{pgfscope}%
\begin{pgfscope}%
\definecolor{textcolor}{rgb}{0.000000,0.000000,0.000000}%
\pgfsetstrokecolor{textcolor}%
\pgfsetfillcolor{textcolor}%
\pgftext[x=0.156558in,y=1.507500in,,bottom,rotate=90.000000]{\color{textcolor}\rmfamily\fontsize{10.000000}{12.000000}\selectfont Force (N)}%
\end{pgfscope}%
\begin{pgfscope}%
\definecolor{textcolor}{rgb}{0.000000,0.000000,0.000000}%
\pgfsetstrokecolor{textcolor}%
\pgfsetfillcolor{textcolor}%
\pgftext[x=0.500000in,y=2.681667in,left,base]{\color{textcolor}\rmfamily\fontsize{10.000000}{12.000000}\selectfont \(\displaystyle \times{10^{4}}{}\)}%
\end{pgfscope}%
\begin{pgfscope}%
\pgfpathrectangle{\pgfqpoint{0.500000in}{0.375000in}}{\pgfqpoint{3.100000in}{2.265000in}}%
\pgfusepath{clip}%
\pgfsetrectcap%
\pgfsetroundjoin%
\pgfsetlinewidth{1.505625pt}%
\definecolor{currentstroke}{rgb}{0.121569,0.466667,0.705882}%
\pgfsetstrokecolor{currentstroke}%
\pgfsetdash{}{0pt}%
\pgfpathmoveto{\pgfqpoint{0.640909in}{1.574098in}}%
\pgfpathlineto{\pgfqpoint{0.936422in}{1.575194in}}%
\pgfpathlineto{\pgfqpoint{0.950837in}{1.578964in}}%
\pgfpathlineto{\pgfqpoint{0.965252in}{1.586603in}}%
\pgfpathlineto{\pgfqpoint{0.979668in}{1.597718in}}%
\pgfpathlineto{\pgfqpoint{0.994083in}{1.612224in}}%
\pgfpathlineto{\pgfqpoint{1.008498in}{1.630010in}}%
\pgfpathlineto{\pgfqpoint{1.022913in}{1.650939in}}%
\pgfpathlineto{\pgfqpoint{1.037329in}{1.680580in}}%
\pgfpathlineto{\pgfqpoint{1.051744in}{1.722246in}}%
\pgfpathlineto{\pgfqpoint{1.066159in}{1.780398in}}%
\pgfpathlineto{\pgfqpoint{1.080574in}{1.861941in}}%
\pgfpathlineto{\pgfqpoint{1.094990in}{1.977569in}}%
\pgfpathlineto{\pgfqpoint{1.123820in}{2.304438in}}%
\pgfpathlineto{\pgfqpoint{1.138235in}{2.430535in}}%
\pgfpathlineto{\pgfqpoint{1.152651in}{2.516535in}}%
\pgfpathlineto{\pgfqpoint{1.159858in}{2.536140in}}%
\pgfpathlineto{\pgfqpoint{1.174273in}{2.536985in}}%
\pgfpathlineto{\pgfqpoint{1.188689in}{2.537045in}}%
\pgfpathlineto{\pgfqpoint{1.195896in}{2.534245in}}%
\pgfpathlineto{\pgfqpoint{1.203104in}{2.525612in}}%
\pgfpathlineto{\pgfqpoint{1.210312in}{2.511264in}}%
\pgfpathlineto{\pgfqpoint{1.217519in}{2.480907in}}%
\pgfpathlineto{\pgfqpoint{1.224727in}{2.429793in}}%
\pgfpathlineto{\pgfqpoint{1.231934in}{2.338739in}}%
\pgfpathlineto{\pgfqpoint{1.239142in}{2.225204in}}%
\pgfpathlineto{\pgfqpoint{1.253557in}{1.861476in}}%
\pgfpathlineto{\pgfqpoint{1.275180in}{1.275623in}}%
\pgfpathlineto{\pgfqpoint{1.289595in}{0.976843in}}%
\pgfpathlineto{\pgfqpoint{1.296803in}{0.878732in}}%
\pgfpathlineto{\pgfqpoint{1.304011in}{0.821064in}}%
\pgfpathlineto{\pgfqpoint{1.311218in}{0.805734in}}%
\pgfpathlineto{\pgfqpoint{1.318426in}{0.832315in}}%
\pgfpathlineto{\pgfqpoint{1.325634in}{0.898118in}}%
\pgfpathlineto{\pgfqpoint{1.332841in}{0.998389in}}%
\pgfpathlineto{\pgfqpoint{1.347256in}{1.274907in}}%
\pgfpathlineto{\pgfqpoint{1.376087in}{1.891389in}}%
\pgfpathlineto{\pgfqpoint{1.383295in}{2.009202in}}%
\pgfpathlineto{\pgfqpoint{1.390502in}{2.099145in}}%
\pgfpathlineto{\pgfqpoint{1.397710in}{2.120101in}}%
\pgfpathlineto{\pgfqpoint{1.404917in}{2.120101in}}%
\pgfpathlineto{\pgfqpoint{1.412125in}{2.135684in}}%
\pgfpathlineto{\pgfqpoint{1.419333in}{2.161516in}}%
\pgfpathlineto{\pgfqpoint{1.426540in}{2.176616in}}%
\pgfpathlineto{\pgfqpoint{1.433748in}{2.178985in}}%
\pgfpathlineto{\pgfqpoint{1.440956in}{2.148974in}}%
\pgfpathlineto{\pgfqpoint{1.448163in}{2.085722in}}%
\pgfpathlineto{\pgfqpoint{1.455371in}{1.994204in}}%
\pgfpathlineto{\pgfqpoint{1.469786in}{1.753134in}}%
\pgfpathlineto{\pgfqpoint{1.491409in}{1.363883in}}%
\pgfpathlineto{\pgfqpoint{1.505824in}{1.173036in}}%
\pgfpathlineto{\pgfqpoint{1.513032in}{1.106193in}}%
\pgfpathlineto{\pgfqpoint{1.520239in}{1.082530in}}%
\pgfpathlineto{\pgfqpoint{1.527447in}{1.102422in}}%
\pgfpathlineto{\pgfqpoint{1.534655in}{1.162413in}}%
\pgfpathlineto{\pgfqpoint{1.541862in}{1.255642in}}%
\pgfpathlineto{\pgfqpoint{1.556278in}{1.501881in}}%
\pgfpathlineto{\pgfqpoint{1.570693in}{1.750401in}}%
\pgfpathlineto{\pgfqpoint{1.577900in}{1.848080in}}%
\pgfpathlineto{\pgfqpoint{1.585108in}{1.916984in}}%
\pgfpathlineto{\pgfqpoint{1.592316in}{1.952296in}}%
\pgfpathlineto{\pgfqpoint{1.599523in}{1.952386in}}%
\pgfpathlineto{\pgfqpoint{1.606731in}{1.918789in}}%
\pgfpathlineto{\pgfqpoint{1.613939in}{1.855912in}}%
\pgfpathlineto{\pgfqpoint{1.628354in}{1.670979in}}%
\pgfpathlineto{\pgfqpoint{1.649977in}{1.379518in}}%
\pgfpathlineto{\pgfqpoint{1.657184in}{1.312424in}}%
\pgfpathlineto{\pgfqpoint{1.664392in}{1.270344in}}%
\pgfpathlineto{\pgfqpoint{1.671600in}{1.255943in}}%
\pgfpathlineto{\pgfqpoint{1.678807in}{1.269365in}}%
\pgfpathlineto{\pgfqpoint{1.686015in}{1.308305in}}%
\pgfpathlineto{\pgfqpoint{1.693223in}{1.368301in}}%
\pgfpathlineto{\pgfqpoint{1.714845in}{1.608436in}}%
\pgfpathlineto{\pgfqpoint{1.722053in}{1.627344in}}%
\pgfpathlineto{\pgfqpoint{1.736468in}{1.627344in}}%
\pgfpathlineto{\pgfqpoint{1.743676in}{1.651059in}}%
\pgfpathlineto{\pgfqpoint{1.765299in}{1.783262in}}%
\pgfpathlineto{\pgfqpoint{1.772506in}{1.808660in}}%
\pgfpathlineto{\pgfqpoint{1.779714in}{1.809655in}}%
\pgfpathlineto{\pgfqpoint{1.786922in}{1.787241in}}%
\pgfpathlineto{\pgfqpoint{1.794129in}{1.744598in}}%
\pgfpathlineto{\pgfqpoint{1.801337in}{1.686685in}}%
\pgfpathlineto{\pgfqpoint{1.808545in}{1.605296in}}%
\pgfpathlineto{\pgfqpoint{1.815752in}{1.559787in}}%
\pgfpathlineto{\pgfqpoint{1.830167in}{1.559787in}}%
\pgfpathlineto{\pgfqpoint{1.837375in}{1.555219in}}%
\pgfpathlineto{\pgfqpoint{1.858998in}{1.524384in}}%
\pgfpathlineto{\pgfqpoint{1.866206in}{1.508320in}}%
\pgfpathlineto{\pgfqpoint{1.873413in}{1.485446in}}%
\pgfpathlineto{\pgfqpoint{1.880621in}{1.448684in}}%
\pgfpathlineto{\pgfqpoint{1.887828in}{1.382908in}}%
\pgfpathlineto{\pgfqpoint{1.902244in}{1.128955in}}%
\pgfpathlineto{\pgfqpoint{1.923867in}{0.732569in}}%
\pgfpathlineto{\pgfqpoint{1.931074in}{0.627579in}}%
\pgfpathlineto{\pgfqpoint{1.938282in}{0.547090in}}%
\pgfpathlineto{\pgfqpoint{1.945489in}{0.489281in}}%
\pgfpathlineto{\pgfqpoint{1.952697in}{0.477955in}}%
\pgfpathlineto{\pgfqpoint{2.010358in}{0.480866in}}%
\pgfpathlineto{\pgfqpoint{2.017566in}{0.485560in}}%
\pgfpathlineto{\pgfqpoint{2.024773in}{0.499539in}}%
\pgfpathlineto{\pgfqpoint{2.031981in}{0.520209in}}%
\pgfpathlineto{\pgfqpoint{2.039189in}{0.547354in}}%
\pgfpathlineto{\pgfqpoint{2.053604in}{0.639345in}}%
\pgfpathlineto{\pgfqpoint{2.060811in}{0.722464in}}%
\pgfpathlineto{\pgfqpoint{2.068019in}{0.846357in}}%
\pgfpathlineto{\pgfqpoint{2.104057in}{1.735872in}}%
\pgfpathlineto{\pgfqpoint{2.111265in}{1.857373in}}%
\pgfpathlineto{\pgfqpoint{2.118472in}{1.940420in}}%
\pgfpathlineto{\pgfqpoint{2.125680in}{1.981601in}}%
\pgfpathlineto{\pgfqpoint{2.132888in}{1.985047in}}%
\pgfpathlineto{\pgfqpoint{2.140095in}{1.985047in}}%
\pgfpathlineto{\pgfqpoint{2.147303in}{1.981030in}}%
\pgfpathlineto{\pgfqpoint{2.161718in}{1.959298in}}%
\pgfpathlineto{\pgfqpoint{2.168926in}{1.945658in}}%
\pgfpathlineto{\pgfqpoint{2.183341in}{1.881047in}}%
\pgfpathlineto{\pgfqpoint{2.190549in}{1.815600in}}%
\pgfpathlineto{\pgfqpoint{2.219379in}{1.410699in}}%
\pgfpathlineto{\pgfqpoint{2.226587in}{1.342761in}}%
\pgfpathlineto{\pgfqpoint{2.233794in}{1.300914in}}%
\pgfpathlineto{\pgfqpoint{2.241002in}{1.287767in}}%
\pgfpathlineto{\pgfqpoint{2.262625in}{1.288011in}}%
\pgfpathlineto{\pgfqpoint{2.269833in}{1.293191in}}%
\pgfpathlineto{\pgfqpoint{2.277040in}{1.307339in}}%
\pgfpathlineto{\pgfqpoint{2.284248in}{1.350372in}}%
\pgfpathlineto{\pgfqpoint{2.298663in}{1.482373in}}%
\pgfpathlineto{\pgfqpoint{2.313078in}{1.624041in}}%
\pgfpathlineto{\pgfqpoint{2.320286in}{1.680733in}}%
\pgfpathlineto{\pgfqpoint{2.327494in}{1.721147in}}%
\pgfpathlineto{\pgfqpoint{2.334701in}{1.736751in}}%
\pgfpathlineto{\pgfqpoint{2.349116in}{1.736751in}}%
\pgfpathlineto{\pgfqpoint{2.363532in}{1.744648in}}%
\pgfpathlineto{\pgfqpoint{2.370739in}{1.738237in}}%
\pgfpathlineto{\pgfqpoint{2.377947in}{1.708726in}}%
\pgfpathlineto{\pgfqpoint{2.392362in}{1.604915in}}%
\pgfpathlineto{\pgfqpoint{2.406777in}{1.499770in}}%
\pgfpathlineto{\pgfqpoint{2.413985in}{1.465613in}}%
\pgfpathlineto{\pgfqpoint{2.421193in}{1.449314in}}%
\pgfpathlineto{\pgfqpoint{2.428400in}{1.451429in}}%
\pgfpathlineto{\pgfqpoint{2.435608in}{1.469803in}}%
\pgfpathlineto{\pgfqpoint{2.442816in}{1.500179in}}%
\pgfpathlineto{\pgfqpoint{2.457231in}{1.588949in}}%
\pgfpathlineto{\pgfqpoint{2.464439in}{1.623271in}}%
\pgfpathlineto{\pgfqpoint{2.471646in}{1.642004in}}%
\pgfpathlineto{\pgfqpoint{2.478854in}{1.643099in}}%
\pgfpathlineto{\pgfqpoint{2.486061in}{1.628364in}}%
\pgfpathlineto{\pgfqpoint{2.500477in}{1.572635in}}%
\pgfpathlineto{\pgfqpoint{2.507684in}{1.544980in}}%
\pgfpathlineto{\pgfqpoint{2.514892in}{1.525177in}}%
\pgfpathlineto{\pgfqpoint{2.522100in}{1.516002in}}%
\pgfpathlineto{\pgfqpoint{2.529307in}{1.523869in}}%
\pgfpathlineto{\pgfqpoint{2.543722in}{1.568685in}}%
\pgfpathlineto{\pgfqpoint{2.550930in}{1.574492in}}%
\pgfpathlineto{\pgfqpoint{2.565345in}{1.574492in}}%
\pgfpathlineto{\pgfqpoint{2.572553in}{1.582891in}}%
\pgfpathlineto{\pgfqpoint{2.579761in}{1.596123in}}%
\pgfpathlineto{\pgfqpoint{2.586968in}{1.601593in}}%
\pgfpathlineto{\pgfqpoint{2.594176in}{1.595099in}}%
\pgfpathlineto{\pgfqpoint{2.601383in}{1.576764in}}%
\pgfpathlineto{\pgfqpoint{2.608591in}{1.572332in}}%
\pgfpathlineto{\pgfqpoint{2.651837in}{1.571799in}}%
\pgfpathlineto{\pgfqpoint{2.659044in}{1.574006in}}%
\pgfpathlineto{\pgfqpoint{2.666252in}{1.578603in}}%
\pgfpathlineto{\pgfqpoint{2.673460in}{1.588888in}}%
\pgfpathlineto{\pgfqpoint{2.680667in}{1.611222in}}%
\pgfpathlineto{\pgfqpoint{2.687875in}{1.659244in}}%
\pgfpathlineto{\pgfqpoint{2.702290in}{1.821191in}}%
\pgfpathlineto{\pgfqpoint{2.716705in}{1.993855in}}%
\pgfpathlineto{\pgfqpoint{2.723913in}{2.063215in}}%
\pgfpathlineto{\pgfqpoint{2.731121in}{2.105149in}}%
\pgfpathlineto{\pgfqpoint{2.745536in}{2.105149in}}%
\pgfpathlineto{\pgfqpoint{2.759951in}{2.107704in}}%
\pgfpathlineto{\pgfqpoint{2.767159in}{2.107704in}}%
\pgfpathlineto{\pgfqpoint{2.781574in}{2.112364in}}%
\pgfpathlineto{\pgfqpoint{2.788782in}{2.116966in}}%
\pgfpathlineto{\pgfqpoint{2.795989in}{2.117043in}}%
\pgfpathlineto{\pgfqpoint{2.803197in}{2.088815in}}%
\pgfpathlineto{\pgfqpoint{2.810405in}{2.026971in}}%
\pgfpathlineto{\pgfqpoint{2.817612in}{1.935577in}}%
\pgfpathlineto{\pgfqpoint{2.832027in}{1.689478in}}%
\pgfpathlineto{\pgfqpoint{2.853650in}{1.286036in}}%
\pgfpathlineto{\pgfqpoint{2.860858in}{1.177200in}}%
\pgfpathlineto{\pgfqpoint{2.868066in}{1.093807in}}%
\pgfpathlineto{\pgfqpoint{2.875273in}{1.041257in}}%
\pgfpathlineto{\pgfqpoint{2.882481in}{1.022962in}}%
\pgfpathlineto{\pgfqpoint{2.889688in}{1.040120in}}%
\pgfpathlineto{\pgfqpoint{2.896896in}{1.091628in}}%
\pgfpathlineto{\pgfqpoint{2.904104in}{1.174142in}}%
\pgfpathlineto{\pgfqpoint{2.918519in}{1.409039in}}%
\pgfpathlineto{\pgfqpoint{2.947349in}{1.930032in}}%
\pgfpathlineto{\pgfqpoint{2.954557in}{2.021109in}}%
\pgfpathlineto{\pgfqpoint{2.961765in}{2.082722in}}%
\pgfpathlineto{\pgfqpoint{2.968972in}{2.110929in}}%
\pgfpathlineto{\pgfqpoint{2.976180in}{2.103970in}}%
\pgfpathlineto{\pgfqpoint{2.983388in}{2.062369in}}%
\pgfpathlineto{\pgfqpoint{2.990595in}{1.988890in}}%
\pgfpathlineto{\pgfqpoint{3.005010in}{1.767277in}}%
\pgfpathlineto{\pgfqpoint{3.033841in}{1.243392in}}%
\pgfpathlineto{\pgfqpoint{3.041049in}{1.145099in}}%
\pgfpathlineto{\pgfqpoint{3.048256in}{1.074329in}}%
\pgfpathlineto{\pgfqpoint{3.055464in}{1.035666in}}%
\pgfpathlineto{\pgfqpoint{3.062671in}{1.031625in}}%
\pgfpathlineto{\pgfqpoint{3.069879in}{1.062478in}}%
\pgfpathlineto{\pgfqpoint{3.077087in}{1.126231in}}%
\pgfpathlineto{\pgfqpoint{3.084294in}{1.218740in}}%
\pgfpathlineto{\pgfqpoint{3.098710in}{1.464442in}}%
\pgfpathlineto{\pgfqpoint{3.120332in}{1.860458in}}%
\pgfpathlineto{\pgfqpoint{3.127540in}{1.965296in}}%
\pgfpathlineto{\pgfqpoint{3.134748in}{2.044270in}}%
\pgfpathlineto{\pgfqpoint{3.141955in}{2.092299in}}%
\pgfpathlineto{\pgfqpoint{3.149163in}{2.106328in}}%
\pgfpathlineto{\pgfqpoint{3.156371in}{2.085519in}}%
\pgfpathlineto{\pgfqpoint{3.163578in}{2.031292in}}%
\pgfpathlineto{\pgfqpoint{3.170786in}{1.947222in}}%
\pgfpathlineto{\pgfqpoint{3.185201in}{1.713063in}}%
\pgfpathlineto{\pgfqpoint{3.206824in}{1.315730in}}%
\pgfpathlineto{\pgfqpoint{3.221239in}{1.118212in}}%
\pgfpathlineto{\pgfqpoint{3.228447in}{1.060536in}}%
\pgfpathlineto{\pgfqpoint{3.235654in}{1.035864in}}%
\pgfpathlineto{\pgfqpoint{3.242862in}{1.045805in}}%
\pgfpathlineto{\pgfqpoint{3.250070in}{1.089724in}}%
\pgfpathlineto{\pgfqpoint{3.257277in}{1.164771in}}%
\pgfpathlineto{\pgfqpoint{3.271693in}{1.387007in}}%
\pgfpathlineto{\pgfqpoint{3.300523in}{1.901422in}}%
\pgfpathlineto{\pgfqpoint{3.307731in}{1.995543in}}%
\pgfpathlineto{\pgfqpoint{3.314938in}{2.061853in}}%
\pgfpathlineto{\pgfqpoint{3.322146in}{2.096100in}}%
\pgfpathlineto{\pgfqpoint{3.329354in}{2.096130in}}%
\pgfpathlineto{\pgfqpoint{3.336561in}{2.062011in}}%
\pgfpathlineto{\pgfqpoint{3.343769in}{1.996024in}}%
\pgfpathlineto{\pgfqpoint{3.350977in}{1.902496in}}%
\pgfpathlineto{\pgfqpoint{3.365392in}{1.658557in}}%
\pgfpathlineto{\pgfqpoint{3.387015in}{1.272361in}}%
\pgfpathlineto{\pgfqpoint{3.394222in}{1.171635in}}%
\pgfpathlineto{\pgfqpoint{3.401430in}{1.096715in}}%
\pgfpathlineto{\pgfqpoint{3.408638in}{1.052453in}}%
\pgfpathlineto{\pgfqpoint{3.415845in}{1.041723in}}%
\pgfpathlineto{\pgfqpoint{3.423053in}{1.065229in}}%
\pgfpathlineto{\pgfqpoint{3.430260in}{1.121453in}}%
\pgfpathlineto{\pgfqpoint{3.437468in}{1.206742in}}%
\pgfpathlineto{\pgfqpoint{3.451883in}{1.440766in}}%
\pgfpathlineto{\pgfqpoint{3.459091in}{1.480108in}}%
\pgfpathlineto{\pgfqpoint{3.459091in}{1.480108in}}%
\pgfusepath{stroke}%
\end{pgfscope}%
\begin{pgfscope}%
\pgfpathrectangle{\pgfqpoint{0.500000in}{0.375000in}}{\pgfqpoint{3.100000in}{2.265000in}}%
\pgfusepath{clip}%
\pgfsetrectcap%
\pgfsetroundjoin%
\pgfsetlinewidth{1.505625pt}%
\definecolor{currentstroke}{rgb}{1.000000,0.000000,0.000000}%
\pgfsetstrokecolor{currentstroke}%
\pgfsetdash{}{0pt}%
\pgfpathmoveto{\pgfqpoint{0.965252in}{0.375000in}}%
\pgfpathlineto{\pgfqpoint{0.965252in}{2.640000in}}%
\pgfusepath{stroke}%
\end{pgfscope}%
\begin{pgfscope}%
\pgfpathrectangle{\pgfqpoint{0.500000in}{0.375000in}}{\pgfqpoint{3.100000in}{2.265000in}}%
\pgfusepath{clip}%
\pgfsetrectcap%
\pgfsetroundjoin%
\pgfsetlinewidth{1.505625pt}%
\definecolor{currentstroke}{rgb}{1.000000,0.000000,0.000000}%
\pgfsetstrokecolor{currentstroke}%
\pgfsetdash{}{0pt}%
\pgfpathmoveto{\pgfqpoint{1.909451in}{0.375000in}}%
\pgfpathlineto{\pgfqpoint{1.909451in}{2.640000in}}%
\pgfusepath{stroke}%
\end{pgfscope}%
\begin{pgfscope}%
\pgfpathrectangle{\pgfqpoint{0.500000in}{0.375000in}}{\pgfqpoint{3.100000in}{2.265000in}}%
\pgfusepath{clip}%
\pgfsetrectcap%
\pgfsetroundjoin%
\pgfsetlinewidth{1.505625pt}%
\definecolor{currentstroke}{rgb}{1.000000,0.000000,0.000000}%
\pgfsetstrokecolor{currentstroke}%
\pgfsetdash{}{0pt}%
\pgfpathmoveto{\pgfqpoint{2.709498in}{0.375000in}}%
\pgfpathlineto{\pgfqpoint{2.709498in}{2.640000in}}%
\pgfusepath{stroke}%
\end{pgfscope}%
\begin{pgfscope}%
\pgfsetrectcap%
\pgfsetmiterjoin%
\pgfsetlinewidth{0.803000pt}%
\definecolor{currentstroke}{rgb}{0.000000,0.000000,0.000000}%
\pgfsetstrokecolor{currentstroke}%
\pgfsetdash{}{0pt}%
\pgfpathmoveto{\pgfqpoint{0.500000in}{0.375000in}}%
\pgfpathlineto{\pgfqpoint{0.500000in}{2.640000in}}%
\pgfusepath{stroke}%
\end{pgfscope}%
\begin{pgfscope}%
\pgfsetrectcap%
\pgfsetmiterjoin%
\pgfsetlinewidth{0.803000pt}%
\definecolor{currentstroke}{rgb}{0.000000,0.000000,0.000000}%
\pgfsetstrokecolor{currentstroke}%
\pgfsetdash{}{0pt}%
\pgfpathmoveto{\pgfqpoint{3.600000in}{0.375000in}}%
\pgfpathlineto{\pgfqpoint{3.600000in}{2.640000in}}%
\pgfusepath{stroke}%
\end{pgfscope}%
\begin{pgfscope}%
\pgfsetrectcap%
\pgfsetmiterjoin%
\pgfsetlinewidth{0.803000pt}%
\definecolor{currentstroke}{rgb}{0.000000,0.000000,0.000000}%
\pgfsetstrokecolor{currentstroke}%
\pgfsetdash{}{0pt}%
\pgfpathmoveto{\pgfqpoint{0.500000in}{0.375000in}}%
\pgfpathlineto{\pgfqpoint{3.600000in}{0.375000in}}%
\pgfusepath{stroke}%
\end{pgfscope}%
\begin{pgfscope}%
\pgfsetrectcap%
\pgfsetmiterjoin%
\pgfsetlinewidth{0.803000pt}%
\definecolor{currentstroke}{rgb}{0.000000,0.000000,0.000000}%
\pgfsetstrokecolor{currentstroke}%
\pgfsetdash{}{0pt}%
\pgfpathmoveto{\pgfqpoint{0.500000in}{2.640000in}}%
\pgfpathlineto{\pgfqpoint{3.600000in}{2.640000in}}%
\pgfusepath{stroke}%
\end{pgfscope}%
\end{pgfpicture}%
\makeatother%
\endgroup%
}
%         \caption{Force Anomalies New CS}
%         \label{fig:hydraulic_sim_signal_force}
%     \end{minipage}
% \end{figure}

Figures \ref{fig:mp_hist_standard_force} and \ref{fig:mp_hist_signal_force} shows the values calculated during the matrix profile for the force generated by the system with the original control signal (figure \ref{fig:hydraulic_sim_standard_force}) and modified control signal (figure \ref{fig:hydraulic_sim_signal_force}) respectively. The values remain relatively similar until the differences in the two signals occur near time step 150. This reveals how the detector is operating and provides an explanation for the system output. 

% \begin{figure}[H]
%     \begin{minipage}[t]{0.5\linewidth}
%         %%\centering
%         \resizebox{\linewidth}{!}{%% Creator: Matplotlib, PGF backend
%%
%% To include the figure in your LaTeX document, write
%%   \input{<filename>.pgf}
%%
%% Make sure the required packages are loaded in your preamble
%%   \usepackage{pgf}
%%
%% Also ensure that all the required font packages are loaded; for instance,
%% the lmodern package is sometimes necessary when using math font.
%%   \usepackage{lmodern}
%%
%% Figures using additional raster images can only be included by \input if
%% they are in the same directory as the main LaTeX file. For loading figures
%% from other directories you can use the `import` package
%%   \usepackage{import}
%%
%% and then include the figures with
%%   \import{<path to file>}{<filename>.pgf}
%%
%% Matplotlib used the following preamble
%%
\begingroup%
\makeatletter%
\begin{pgfpicture}%
\pgfpathrectangle{\pgfpointorigin}{\pgfqpoint{4.000000in}{3.000000in}}%
\pgfusepath{use as bounding box, clip}%
\begin{pgfscope}%
\pgfsetbuttcap%
\pgfsetmiterjoin%
\pgfsetlinewidth{0.000000pt}%
\definecolor{currentstroke}{rgb}{1.000000,1.000000,1.000000}%
\pgfsetstrokecolor{currentstroke}%
\pgfsetstrokeopacity{0.000000}%
\pgfsetdash{}{0pt}%
\pgfpathmoveto{\pgfqpoint{0.000000in}{0.000000in}}%
\pgfpathlineto{\pgfqpoint{4.000000in}{0.000000in}}%
\pgfpathlineto{\pgfqpoint{4.000000in}{3.000000in}}%
\pgfpathlineto{\pgfqpoint{0.000000in}{3.000000in}}%
\pgfpathlineto{\pgfqpoint{0.000000in}{0.000000in}}%
\pgfpathclose%
\pgfusepath{}%
\end{pgfscope}%
\begin{pgfscope}%
\pgfsetbuttcap%
\pgfsetmiterjoin%
\definecolor{currentfill}{rgb}{1.000000,1.000000,1.000000}%
\pgfsetfillcolor{currentfill}%
\pgfsetlinewidth{0.000000pt}%
\definecolor{currentstroke}{rgb}{0.000000,0.000000,0.000000}%
\pgfsetstrokecolor{currentstroke}%
\pgfsetstrokeopacity{0.000000}%
\pgfsetdash{}{0pt}%
\pgfpathmoveto{\pgfqpoint{0.500000in}{0.375000in}}%
\pgfpathlineto{\pgfqpoint{3.600000in}{0.375000in}}%
\pgfpathlineto{\pgfqpoint{3.600000in}{2.640000in}}%
\pgfpathlineto{\pgfqpoint{0.500000in}{2.640000in}}%
\pgfpathlineto{\pgfqpoint{0.500000in}{0.375000in}}%
\pgfpathclose%
\pgfusepath{fill}%
\end{pgfscope}%
\begin{pgfscope}%
\pgfsetbuttcap%
\pgfsetroundjoin%
\definecolor{currentfill}{rgb}{0.000000,0.000000,0.000000}%
\pgfsetfillcolor{currentfill}%
\pgfsetlinewidth{0.803000pt}%
\definecolor{currentstroke}{rgb}{0.000000,0.000000,0.000000}%
\pgfsetstrokecolor{currentstroke}%
\pgfsetdash{}{0pt}%
\pgfsys@defobject{currentmarker}{\pgfqpoint{0.000000in}{-0.048611in}}{\pgfqpoint{0.000000in}{0.000000in}}{%
\pgfpathmoveto{\pgfqpoint{0.000000in}{0.000000in}}%
\pgfpathlineto{\pgfqpoint{0.000000in}{-0.048611in}}%
\pgfusepath{stroke,fill}%
}%
\begin{pgfscope}%
\pgfsys@transformshift{0.640909in}{0.375000in}%
\pgfsys@useobject{currentmarker}{}%
\end{pgfscope}%
\end{pgfscope}%
\begin{pgfscope}%
\definecolor{textcolor}{rgb}{0.000000,0.000000,0.000000}%
\pgfsetstrokecolor{textcolor}%
\pgfsetfillcolor{textcolor}%
\pgftext[x=0.640909in,y=0.277778in,,top]{\color{textcolor}\rmfamily\fontsize{10.000000}{12.000000}\selectfont \(\displaystyle {0}\)}%
\end{pgfscope}%
\begin{pgfscope}%
\pgfsetbuttcap%
\pgfsetroundjoin%
\definecolor{currentfill}{rgb}{0.000000,0.000000,0.000000}%
\pgfsetfillcolor{currentfill}%
\pgfsetlinewidth{0.803000pt}%
\definecolor{currentstroke}{rgb}{0.000000,0.000000,0.000000}%
\pgfsetstrokecolor{currentstroke}%
\pgfsetdash{}{0pt}%
\pgfsys@defobject{currentmarker}{\pgfqpoint{0.000000in}{-0.048611in}}{\pgfqpoint{0.000000in}{0.000000in}}{%
\pgfpathmoveto{\pgfqpoint{0.000000in}{0.000000in}}%
\pgfpathlineto{\pgfqpoint{0.000000in}{-0.048611in}}%
\pgfusepath{stroke,fill}%
}%
\begin{pgfscope}%
\pgfsys@transformshift{1.167672in}{0.375000in}%
\pgfsys@useobject{currentmarker}{}%
\end{pgfscope}%
\end{pgfscope}%
\begin{pgfscope}%
\definecolor{textcolor}{rgb}{0.000000,0.000000,0.000000}%
\pgfsetstrokecolor{textcolor}%
\pgfsetfillcolor{textcolor}%
\pgftext[x=1.167672in,y=0.277778in,,top]{\color{textcolor}\rmfamily\fontsize{10.000000}{12.000000}\selectfont \(\displaystyle {100}\)}%
\end{pgfscope}%
\begin{pgfscope}%
\pgfsetbuttcap%
\pgfsetroundjoin%
\definecolor{currentfill}{rgb}{0.000000,0.000000,0.000000}%
\pgfsetfillcolor{currentfill}%
\pgfsetlinewidth{0.803000pt}%
\definecolor{currentstroke}{rgb}{0.000000,0.000000,0.000000}%
\pgfsetstrokecolor{currentstroke}%
\pgfsetdash{}{0pt}%
\pgfsys@defobject{currentmarker}{\pgfqpoint{0.000000in}{-0.048611in}}{\pgfqpoint{0.000000in}{0.000000in}}{%
\pgfpathmoveto{\pgfqpoint{0.000000in}{0.000000in}}%
\pgfpathlineto{\pgfqpoint{0.000000in}{-0.048611in}}%
\pgfusepath{stroke,fill}%
}%
\begin{pgfscope}%
\pgfsys@transformshift{1.694435in}{0.375000in}%
\pgfsys@useobject{currentmarker}{}%
\end{pgfscope}%
\end{pgfscope}%
\begin{pgfscope}%
\definecolor{textcolor}{rgb}{0.000000,0.000000,0.000000}%
\pgfsetstrokecolor{textcolor}%
\pgfsetfillcolor{textcolor}%
\pgftext[x=1.694435in,y=0.277778in,,top]{\color{textcolor}\rmfamily\fontsize{10.000000}{12.000000}\selectfont \(\displaystyle {200}\)}%
\end{pgfscope}%
\begin{pgfscope}%
\pgfsetbuttcap%
\pgfsetroundjoin%
\definecolor{currentfill}{rgb}{0.000000,0.000000,0.000000}%
\pgfsetfillcolor{currentfill}%
\pgfsetlinewidth{0.803000pt}%
\definecolor{currentstroke}{rgb}{0.000000,0.000000,0.000000}%
\pgfsetstrokecolor{currentstroke}%
\pgfsetdash{}{0pt}%
\pgfsys@defobject{currentmarker}{\pgfqpoint{0.000000in}{-0.048611in}}{\pgfqpoint{0.000000in}{0.000000in}}{%
\pgfpathmoveto{\pgfqpoint{0.000000in}{0.000000in}}%
\pgfpathlineto{\pgfqpoint{0.000000in}{-0.048611in}}%
\pgfusepath{stroke,fill}%
}%
\begin{pgfscope}%
\pgfsys@transformshift{2.221198in}{0.375000in}%
\pgfsys@useobject{currentmarker}{}%
\end{pgfscope}%
\end{pgfscope}%
\begin{pgfscope}%
\definecolor{textcolor}{rgb}{0.000000,0.000000,0.000000}%
\pgfsetstrokecolor{textcolor}%
\pgfsetfillcolor{textcolor}%
\pgftext[x=2.221198in,y=0.277778in,,top]{\color{textcolor}\rmfamily\fontsize{10.000000}{12.000000}\selectfont \(\displaystyle {300}\)}%
\end{pgfscope}%
\begin{pgfscope}%
\pgfsetbuttcap%
\pgfsetroundjoin%
\definecolor{currentfill}{rgb}{0.000000,0.000000,0.000000}%
\pgfsetfillcolor{currentfill}%
\pgfsetlinewidth{0.803000pt}%
\definecolor{currentstroke}{rgb}{0.000000,0.000000,0.000000}%
\pgfsetstrokecolor{currentstroke}%
\pgfsetdash{}{0pt}%
\pgfsys@defobject{currentmarker}{\pgfqpoint{0.000000in}{-0.048611in}}{\pgfqpoint{0.000000in}{0.000000in}}{%
\pgfpathmoveto{\pgfqpoint{0.000000in}{0.000000in}}%
\pgfpathlineto{\pgfqpoint{0.000000in}{-0.048611in}}%
\pgfusepath{stroke,fill}%
}%
\begin{pgfscope}%
\pgfsys@transformshift{2.747961in}{0.375000in}%
\pgfsys@useobject{currentmarker}{}%
\end{pgfscope}%
\end{pgfscope}%
\begin{pgfscope}%
\definecolor{textcolor}{rgb}{0.000000,0.000000,0.000000}%
\pgfsetstrokecolor{textcolor}%
\pgfsetfillcolor{textcolor}%
\pgftext[x=2.747961in,y=0.277778in,,top]{\color{textcolor}\rmfamily\fontsize{10.000000}{12.000000}\selectfont \(\displaystyle {400}\)}%
\end{pgfscope}%
\begin{pgfscope}%
\pgfsetbuttcap%
\pgfsetroundjoin%
\definecolor{currentfill}{rgb}{0.000000,0.000000,0.000000}%
\pgfsetfillcolor{currentfill}%
\pgfsetlinewidth{0.803000pt}%
\definecolor{currentstroke}{rgb}{0.000000,0.000000,0.000000}%
\pgfsetstrokecolor{currentstroke}%
\pgfsetdash{}{0pt}%
\pgfsys@defobject{currentmarker}{\pgfqpoint{0.000000in}{-0.048611in}}{\pgfqpoint{0.000000in}{0.000000in}}{%
\pgfpathmoveto{\pgfqpoint{0.000000in}{0.000000in}}%
\pgfpathlineto{\pgfqpoint{0.000000in}{-0.048611in}}%
\pgfusepath{stroke,fill}%
}%
\begin{pgfscope}%
\pgfsys@transformshift{3.274724in}{0.375000in}%
\pgfsys@useobject{currentmarker}{}%
\end{pgfscope}%
\end{pgfscope}%
\begin{pgfscope}%
\definecolor{textcolor}{rgb}{0.000000,0.000000,0.000000}%
\pgfsetstrokecolor{textcolor}%
\pgfsetfillcolor{textcolor}%
\pgftext[x=3.274724in,y=0.277778in,,top]{\color{textcolor}\rmfamily\fontsize{10.000000}{12.000000}\selectfont \(\displaystyle {500}\)}%
\end{pgfscope}%
\begin{pgfscope}%
\definecolor{textcolor}{rgb}{0.000000,0.000000,0.000000}%
\pgfsetstrokecolor{textcolor}%
\pgfsetfillcolor{textcolor}%
\pgftext[x=2.050000in,y=0.098766in,,top]{\color{textcolor}\rmfamily\fontsize{10.000000}{12.000000}\selectfont time}%
\end{pgfscope}%
\begin{pgfscope}%
\pgfsetbuttcap%
\pgfsetroundjoin%
\definecolor{currentfill}{rgb}{0.000000,0.000000,0.000000}%
\pgfsetfillcolor{currentfill}%
\pgfsetlinewidth{0.803000pt}%
\definecolor{currentstroke}{rgb}{0.000000,0.000000,0.000000}%
\pgfsetstrokecolor{currentstroke}%
\pgfsetdash{}{0pt}%
\pgfsys@defobject{currentmarker}{\pgfqpoint{-0.048611in}{0.000000in}}{\pgfqpoint{-0.000000in}{0.000000in}}{%
\pgfpathmoveto{\pgfqpoint{-0.000000in}{0.000000in}}%
\pgfpathlineto{\pgfqpoint{-0.048611in}{0.000000in}}%
\pgfusepath{stroke,fill}%
}%
\begin{pgfscope}%
\pgfsys@transformshift{0.500000in}{0.477952in}%
\pgfsys@useobject{currentmarker}{}%
\end{pgfscope}%
\end{pgfscope}%
\begin{pgfscope}%
\definecolor{textcolor}{rgb}{0.000000,0.000000,0.000000}%
\pgfsetstrokecolor{textcolor}%
\pgfsetfillcolor{textcolor}%
\pgftext[x=0.333333in, y=0.429727in, left, base]{\color{textcolor}\rmfamily\fontsize{10.000000}{12.000000}\selectfont \(\displaystyle {0}\)}%
\end{pgfscope}%
\begin{pgfscope}%
\pgfsetbuttcap%
\pgfsetroundjoin%
\definecolor{currentfill}{rgb}{0.000000,0.000000,0.000000}%
\pgfsetfillcolor{currentfill}%
\pgfsetlinewidth{0.803000pt}%
\definecolor{currentstroke}{rgb}{0.000000,0.000000,0.000000}%
\pgfsetstrokecolor{currentstroke}%
\pgfsetdash{}{0pt}%
\pgfsys@defobject{currentmarker}{\pgfqpoint{-0.048611in}{0.000000in}}{\pgfqpoint{-0.000000in}{0.000000in}}{%
\pgfpathmoveto{\pgfqpoint{-0.000000in}{0.000000in}}%
\pgfpathlineto{\pgfqpoint{-0.048611in}{0.000000in}}%
\pgfusepath{stroke,fill}%
}%
\begin{pgfscope}%
\pgfsys@transformshift{0.500000in}{0.955912in}%
\pgfsys@useobject{currentmarker}{}%
\end{pgfscope}%
\end{pgfscope}%
\begin{pgfscope}%
\definecolor{textcolor}{rgb}{0.000000,0.000000,0.000000}%
\pgfsetstrokecolor{textcolor}%
\pgfsetfillcolor{textcolor}%
\pgftext[x=0.124999in, y=0.907686in, left, base]{\color{textcolor}\rmfamily\fontsize{10.000000}{12.000000}\selectfont \(\displaystyle {5000}\)}%
\end{pgfscope}%
\begin{pgfscope}%
\pgfsetbuttcap%
\pgfsetroundjoin%
\definecolor{currentfill}{rgb}{0.000000,0.000000,0.000000}%
\pgfsetfillcolor{currentfill}%
\pgfsetlinewidth{0.803000pt}%
\definecolor{currentstroke}{rgb}{0.000000,0.000000,0.000000}%
\pgfsetstrokecolor{currentstroke}%
\pgfsetdash{}{0pt}%
\pgfsys@defobject{currentmarker}{\pgfqpoint{-0.048611in}{0.000000in}}{\pgfqpoint{-0.000000in}{0.000000in}}{%
\pgfpathmoveto{\pgfqpoint{-0.000000in}{0.000000in}}%
\pgfpathlineto{\pgfqpoint{-0.048611in}{0.000000in}}%
\pgfusepath{stroke,fill}%
}%
\begin{pgfscope}%
\pgfsys@transformshift{0.500000in}{1.433872in}%
\pgfsys@useobject{currentmarker}{}%
\end{pgfscope}%
\end{pgfscope}%
\begin{pgfscope}%
\definecolor{textcolor}{rgb}{0.000000,0.000000,0.000000}%
\pgfsetstrokecolor{textcolor}%
\pgfsetfillcolor{textcolor}%
\pgftext[x=0.055554in, y=1.385646in, left, base]{\color{textcolor}\rmfamily\fontsize{10.000000}{12.000000}\selectfont \(\displaystyle {10000}\)}%
\end{pgfscope}%
\begin{pgfscope}%
\pgfsetbuttcap%
\pgfsetroundjoin%
\definecolor{currentfill}{rgb}{0.000000,0.000000,0.000000}%
\pgfsetfillcolor{currentfill}%
\pgfsetlinewidth{0.803000pt}%
\definecolor{currentstroke}{rgb}{0.000000,0.000000,0.000000}%
\pgfsetstrokecolor{currentstroke}%
\pgfsetdash{}{0pt}%
\pgfsys@defobject{currentmarker}{\pgfqpoint{-0.048611in}{0.000000in}}{\pgfqpoint{-0.000000in}{0.000000in}}{%
\pgfpathmoveto{\pgfqpoint{-0.000000in}{0.000000in}}%
\pgfpathlineto{\pgfqpoint{-0.048611in}{0.000000in}}%
\pgfusepath{stroke,fill}%
}%
\begin{pgfscope}%
\pgfsys@transformshift{0.500000in}{1.911831in}%
\pgfsys@useobject{currentmarker}{}%
\end{pgfscope}%
\end{pgfscope}%
\begin{pgfscope}%
\definecolor{textcolor}{rgb}{0.000000,0.000000,0.000000}%
\pgfsetstrokecolor{textcolor}%
\pgfsetfillcolor{textcolor}%
\pgftext[x=0.055554in, y=1.863606in, left, base]{\color{textcolor}\rmfamily\fontsize{10.000000}{12.000000}\selectfont \(\displaystyle {15000}\)}%
\end{pgfscope}%
\begin{pgfscope}%
\pgfsetbuttcap%
\pgfsetroundjoin%
\definecolor{currentfill}{rgb}{0.000000,0.000000,0.000000}%
\pgfsetfillcolor{currentfill}%
\pgfsetlinewidth{0.803000pt}%
\definecolor{currentstroke}{rgb}{0.000000,0.000000,0.000000}%
\pgfsetstrokecolor{currentstroke}%
\pgfsetdash{}{0pt}%
\pgfsys@defobject{currentmarker}{\pgfqpoint{-0.048611in}{0.000000in}}{\pgfqpoint{-0.000000in}{0.000000in}}{%
\pgfpathmoveto{\pgfqpoint{-0.000000in}{0.000000in}}%
\pgfpathlineto{\pgfqpoint{-0.048611in}{0.000000in}}%
\pgfusepath{stroke,fill}%
}%
\begin{pgfscope}%
\pgfsys@transformshift{0.500000in}{2.389791in}%
\pgfsys@useobject{currentmarker}{}%
\end{pgfscope}%
\end{pgfscope}%
\begin{pgfscope}%
\definecolor{textcolor}{rgb}{0.000000,0.000000,0.000000}%
\pgfsetstrokecolor{textcolor}%
\pgfsetfillcolor{textcolor}%
\pgftext[x=0.055554in, y=2.341566in, left, base]{\color{textcolor}\rmfamily\fontsize{10.000000}{12.000000}\selectfont \(\displaystyle {20000}\)}%
\end{pgfscope}%
\begin{pgfscope}%
\pgfpathrectangle{\pgfqpoint{0.500000in}{0.375000in}}{\pgfqpoint{3.100000in}{2.265000in}}%
\pgfusepath{clip}%
\pgfsetrectcap%
\pgfsetroundjoin%
\pgfsetlinewidth{1.505625pt}%
\definecolor{currentstroke}{rgb}{0.000000,0.000000,1.000000}%
\pgfsetstrokecolor{currentstroke}%
\pgfsetdash{}{0pt}%
\pgfpathmoveto{\pgfqpoint{0.640909in}{0.477963in}}%
\pgfpathlineto{\pgfqpoint{0.698853in}{0.478783in}}%
\pgfpathlineto{\pgfqpoint{0.709388in}{0.481612in}}%
\pgfpathlineto{\pgfqpoint{0.719924in}{0.488512in}}%
\pgfpathlineto{\pgfqpoint{0.730459in}{0.499411in}}%
\pgfpathlineto{\pgfqpoint{0.746262in}{0.519970in}}%
\pgfpathlineto{\pgfqpoint{0.762065in}{0.544062in}}%
\pgfpathlineto{\pgfqpoint{0.772600in}{0.564249in}}%
\pgfpathlineto{\pgfqpoint{0.783135in}{0.592395in}}%
\pgfpathlineto{\pgfqpoint{0.793670in}{0.633526in}}%
\pgfpathlineto{\pgfqpoint{0.804206in}{0.692159in}}%
\pgfpathlineto{\pgfqpoint{0.814741in}{0.775896in}}%
\pgfpathlineto{\pgfqpoint{0.825276in}{0.899351in}}%
\pgfpathlineto{\pgfqpoint{0.841079in}{1.099499in}}%
\pgfpathlineto{\pgfqpoint{0.851614in}{1.185661in}}%
\pgfpathlineto{\pgfqpoint{0.856882in}{1.211682in}}%
\pgfpathlineto{\pgfqpoint{0.862150in}{1.226919in}}%
\pgfpathlineto{\pgfqpoint{0.867417in}{1.232422in}}%
\pgfpathlineto{\pgfqpoint{0.893755in}{1.231805in}}%
\pgfpathlineto{\pgfqpoint{0.899023in}{1.226718in}}%
\pgfpathlineto{\pgfqpoint{0.909558in}{1.211682in}}%
\pgfpathlineto{\pgfqpoint{0.930629in}{1.211682in}}%
\pgfpathlineto{\pgfqpoint{0.935896in}{1.306109in}}%
\pgfpathlineto{\pgfqpoint{0.951699in}{1.846401in}}%
\pgfpathlineto{\pgfqpoint{0.962234in}{2.072234in}}%
\pgfpathlineto{\pgfqpoint{0.967502in}{2.133075in}}%
\pgfpathlineto{\pgfqpoint{0.972770in}{2.164855in}}%
\pgfpathlineto{\pgfqpoint{0.978037in}{2.175823in}}%
\pgfpathlineto{\pgfqpoint{0.988573in}{2.176655in}}%
\pgfpathlineto{\pgfqpoint{0.993840in}{2.184448in}}%
\pgfpathlineto{\pgfqpoint{1.009643in}{2.238118in}}%
\pgfpathlineto{\pgfqpoint{1.030714in}{2.238118in}}%
\pgfpathlineto{\pgfqpoint{1.035981in}{2.234771in}}%
\pgfpathlineto{\pgfqpoint{1.046517in}{2.184448in}}%
\pgfpathlineto{\pgfqpoint{1.072855in}{2.184448in}}%
\pgfpathlineto{\pgfqpoint{1.078122in}{2.158260in}}%
\pgfpathlineto{\pgfqpoint{1.088658in}{2.015579in}}%
\pgfpathlineto{\pgfqpoint{1.099193in}{1.880244in}}%
\pgfpathlineto{\pgfqpoint{1.104460in}{1.845912in}}%
\pgfpathlineto{\pgfqpoint{1.109728in}{1.833224in}}%
\pgfpathlineto{\pgfqpoint{1.114996in}{1.832739in}}%
\pgfpathlineto{\pgfqpoint{1.120263in}{1.830342in}}%
\pgfpathlineto{\pgfqpoint{1.125531in}{1.814075in}}%
\pgfpathlineto{\pgfqpoint{1.130799in}{1.779009in}}%
\pgfpathlineto{\pgfqpoint{1.141334in}{1.666627in}}%
\pgfpathlineto{\pgfqpoint{1.146602in}{1.607200in}}%
\pgfpathlineto{\pgfqpoint{1.151869in}{1.642235in}}%
\pgfpathlineto{\pgfqpoint{1.162404in}{1.743336in}}%
\pgfpathlineto{\pgfqpoint{1.167672in}{1.775197in}}%
\pgfpathlineto{\pgfqpoint{1.172940in}{1.778086in}}%
\pgfpathlineto{\pgfqpoint{1.194010in}{1.778086in}}%
\pgfpathlineto{\pgfqpoint{1.199278in}{1.748900in}}%
\pgfpathlineto{\pgfqpoint{1.204545in}{1.691867in}}%
\pgfpathlineto{\pgfqpoint{1.230884in}{1.691867in}}%
\pgfpathlineto{\pgfqpoint{1.241419in}{1.618195in}}%
\pgfpathlineto{\pgfqpoint{1.246686in}{1.612417in}}%
\pgfpathlineto{\pgfqpoint{1.251954in}{1.609156in}}%
\pgfpathlineto{\pgfqpoint{1.257222in}{1.569398in}}%
\pgfpathlineto{\pgfqpoint{1.262489in}{1.544824in}}%
\pgfpathlineto{\pgfqpoint{1.273025in}{1.544824in}}%
\pgfpathlineto{\pgfqpoint{1.278292in}{1.499505in}}%
\pgfpathlineto{\pgfqpoint{1.283560in}{1.499505in}}%
\pgfpathlineto{\pgfqpoint{1.288828in}{1.468129in}}%
\pgfpathlineto{\pgfqpoint{1.294095in}{1.414476in}}%
\pgfpathlineto{\pgfqpoint{1.299363in}{1.413980in}}%
\pgfpathlineto{\pgfqpoint{1.304630in}{1.335809in}}%
\pgfpathlineto{\pgfqpoint{1.330969in}{1.335809in}}%
\pgfpathlineto{\pgfqpoint{1.336236in}{1.285822in}}%
\pgfpathlineto{\pgfqpoint{1.341504in}{1.276820in}}%
\pgfpathlineto{\pgfqpoint{1.346771in}{1.233075in}}%
\pgfpathlineto{\pgfqpoint{1.357307in}{1.092152in}}%
\pgfpathlineto{\pgfqpoint{1.362574in}{1.039921in}}%
\pgfpathlineto{\pgfqpoint{1.367842in}{1.008491in}}%
\pgfpathlineto{\pgfqpoint{1.373110in}{1.000332in}}%
\pgfpathlineto{\pgfqpoint{1.388912in}{1.000332in}}%
\pgfpathlineto{\pgfqpoint{1.394180in}{0.995527in}}%
\pgfpathlineto{\pgfqpoint{1.399448in}{0.965239in}}%
\pgfpathlineto{\pgfqpoint{1.404715in}{0.955824in}}%
\pgfpathlineto{\pgfqpoint{1.431054in}{1.467851in}}%
\pgfpathlineto{\pgfqpoint{1.436321in}{1.523240in}}%
\pgfpathlineto{\pgfqpoint{1.441589in}{1.554852in}}%
\pgfpathlineto{\pgfqpoint{1.446856in}{1.569069in}}%
\pgfpathlineto{\pgfqpoint{1.452124in}{1.573695in}}%
\pgfpathlineto{\pgfqpoint{1.473195in}{1.574151in}}%
\pgfpathlineto{\pgfqpoint{1.483730in}{1.573101in}}%
\pgfpathlineto{\pgfqpoint{1.488997in}{1.571032in}}%
\pgfpathlineto{\pgfqpoint{1.494265in}{1.565899in}}%
\pgfpathlineto{\pgfqpoint{1.525871in}{1.565899in}}%
\pgfpathlineto{\pgfqpoint{1.531138in}{1.550609in}}%
\pgfpathlineto{\pgfqpoint{1.536406in}{1.517823in}}%
\pgfpathlineto{\pgfqpoint{1.541674in}{1.459561in}}%
\pgfpathlineto{\pgfqpoint{1.546941in}{1.369305in}}%
\pgfpathlineto{\pgfqpoint{1.557477in}{1.108057in}}%
\pgfpathlineto{\pgfqpoint{1.573280in}{0.688652in}}%
\pgfpathlineto{\pgfqpoint{1.578547in}{0.593816in}}%
\pgfpathlineto{\pgfqpoint{1.583815in}{0.549649in}}%
\pgfpathlineto{\pgfqpoint{1.594350in}{0.566245in}}%
\pgfpathlineto{\pgfqpoint{1.599618in}{0.568869in}}%
\pgfpathlineto{\pgfqpoint{1.610153in}{0.586002in}}%
\pgfpathlineto{\pgfqpoint{1.615421in}{0.588692in}}%
\pgfpathlineto{\pgfqpoint{1.625956in}{0.606151in}}%
\pgfpathlineto{\pgfqpoint{1.631223in}{0.608865in}}%
\pgfpathlineto{\pgfqpoint{1.641759in}{0.626316in}}%
\pgfpathlineto{\pgfqpoint{1.647026in}{0.628996in}}%
\pgfpathlineto{\pgfqpoint{1.657562in}{0.646607in}}%
\pgfpathlineto{\pgfqpoint{1.662829in}{0.649361in}}%
\pgfpathlineto{\pgfqpoint{1.673364in}{0.667283in}}%
\pgfpathlineto{\pgfqpoint{1.678632in}{0.670035in}}%
\pgfpathlineto{\pgfqpoint{1.689167in}{0.686712in}}%
\pgfpathlineto{\pgfqpoint{1.694435in}{0.689108in}}%
\pgfpathlineto{\pgfqpoint{1.704970in}{0.703363in}}%
\pgfpathlineto{\pgfqpoint{1.710238in}{0.705417in}}%
\pgfpathlineto{\pgfqpoint{1.720773in}{0.717242in}}%
\pgfpathlineto{\pgfqpoint{1.726041in}{0.718936in}}%
\pgfpathlineto{\pgfqpoint{1.731308in}{0.723331in}}%
\pgfpathlineto{\pgfqpoint{1.736576in}{0.758864in}}%
\pgfpathlineto{\pgfqpoint{1.741844in}{0.782546in}}%
\pgfpathlineto{\pgfqpoint{1.747111in}{0.798278in}}%
\pgfpathlineto{\pgfqpoint{1.752379in}{0.826310in}}%
\pgfpathlineto{\pgfqpoint{1.762914in}{0.906358in}}%
\pgfpathlineto{\pgfqpoint{1.773449in}{1.066235in}}%
\pgfpathlineto{\pgfqpoint{1.778717in}{1.117875in}}%
\pgfpathlineto{\pgfqpoint{1.783985in}{1.152743in}}%
\pgfpathlineto{\pgfqpoint{1.789252in}{1.170277in}}%
\pgfpathlineto{\pgfqpoint{1.831393in}{1.169225in}}%
\pgfpathlineto{\pgfqpoint{1.836661in}{1.157553in}}%
\pgfpathlineto{\pgfqpoint{1.868267in}{1.157553in}}%
\pgfpathlineto{\pgfqpoint{1.873534in}{1.151858in}}%
\pgfpathlineto{\pgfqpoint{1.878802in}{1.142928in}}%
\pgfpathlineto{\pgfqpoint{1.884070in}{1.130031in}}%
\pgfpathlineto{\pgfqpoint{1.894605in}{1.083035in}}%
\pgfpathlineto{\pgfqpoint{1.905140in}{0.936397in}}%
\pgfpathlineto{\pgfqpoint{1.915675in}{0.791654in}}%
\pgfpathlineto{\pgfqpoint{1.920943in}{0.760287in}}%
\pgfpathlineto{\pgfqpoint{1.963084in}{0.754391in}}%
\pgfpathlineto{\pgfqpoint{1.984155in}{0.747983in}}%
\pgfpathlineto{\pgfqpoint{1.994690in}{0.745139in}}%
\pgfpathlineto{\pgfqpoint{2.015760in}{0.736544in}}%
\pgfpathlineto{\pgfqpoint{2.021028in}{0.735580in}}%
\pgfpathlineto{\pgfqpoint{2.031563in}{0.715371in}}%
\pgfpathlineto{\pgfqpoint{2.036831in}{0.713699in}}%
\pgfpathlineto{\pgfqpoint{2.047366in}{0.697562in}}%
\pgfpathlineto{\pgfqpoint{2.052634in}{0.697562in}}%
\pgfpathlineto{\pgfqpoint{2.057901in}{0.695095in}}%
\pgfpathlineto{\pgfqpoint{2.063169in}{0.686730in}}%
\pgfpathlineto{\pgfqpoint{2.068437in}{0.686730in}}%
\pgfpathlineto{\pgfqpoint{2.073704in}{0.682156in}}%
\pgfpathlineto{\pgfqpoint{2.078972in}{0.671794in}}%
\pgfpathlineto{\pgfqpoint{2.084240in}{0.671794in}}%
\pgfpathlineto{\pgfqpoint{2.089507in}{0.665816in}}%
\pgfpathlineto{\pgfqpoint{2.094775in}{0.655833in}}%
\pgfpathlineto{\pgfqpoint{2.105310in}{0.655833in}}%
\pgfpathlineto{\pgfqpoint{2.115845in}{0.654266in}}%
\pgfpathlineto{\pgfqpoint{2.126381in}{0.652370in}}%
\pgfpathlineto{\pgfqpoint{2.136916in}{0.633030in}}%
\pgfpathlineto{\pgfqpoint{2.142184in}{0.629902in}}%
\pgfpathlineto{\pgfqpoint{2.152719in}{0.612760in}}%
\pgfpathlineto{\pgfqpoint{2.157986in}{0.610121in}}%
\pgfpathlineto{\pgfqpoint{2.168522in}{0.593645in}}%
\pgfpathlineto{\pgfqpoint{2.173789in}{0.591078in}}%
\pgfpathlineto{\pgfqpoint{2.184325in}{0.575158in}}%
\pgfpathlineto{\pgfqpoint{2.189592in}{0.572657in}}%
\pgfpathlineto{\pgfqpoint{2.200127in}{0.557285in}}%
\pgfpathlineto{\pgfqpoint{2.205395in}{0.554858in}}%
\pgfpathlineto{\pgfqpoint{2.215930in}{0.540147in}}%
\pgfpathlineto{\pgfqpoint{2.221198in}{0.537826in}}%
\pgfpathlineto{\pgfqpoint{2.231733in}{0.524161in}}%
\pgfpathlineto{\pgfqpoint{2.242268in}{0.520689in}}%
\pgfpathlineto{\pgfqpoint{2.247536in}{0.522640in}}%
\pgfpathlineto{\pgfqpoint{2.258071in}{0.536276in}}%
\pgfpathlineto{\pgfqpoint{2.263339in}{0.538487in}}%
\pgfpathlineto{\pgfqpoint{2.273874in}{0.553207in}}%
\pgfpathlineto{\pgfqpoint{2.279142in}{0.555547in}}%
\pgfpathlineto{\pgfqpoint{2.284410in}{0.563176in}}%
\pgfpathlineto{\pgfqpoint{2.289677in}{0.593238in}}%
\pgfpathlineto{\pgfqpoint{2.294945in}{0.638545in}}%
\pgfpathlineto{\pgfqpoint{2.300212in}{0.733875in}}%
\pgfpathlineto{\pgfqpoint{2.310748in}{1.103272in}}%
\pgfpathlineto{\pgfqpoint{2.331818in}{1.994139in}}%
\pgfpathlineto{\pgfqpoint{2.342353in}{2.262218in}}%
\pgfpathlineto{\pgfqpoint{2.347621in}{2.329814in}}%
\pgfpathlineto{\pgfqpoint{2.352889in}{2.362054in}}%
\pgfpathlineto{\pgfqpoint{2.358156in}{2.374363in}}%
\pgfpathlineto{\pgfqpoint{2.363424in}{2.377605in}}%
\pgfpathlineto{\pgfqpoint{2.389762in}{2.376680in}}%
\pgfpathlineto{\pgfqpoint{2.395030in}{2.373632in}}%
\pgfpathlineto{\pgfqpoint{2.400297in}{2.363270in}}%
\pgfpathlineto{\pgfqpoint{2.410833in}{2.362054in}}%
\pgfpathlineto{\pgfqpoint{2.431903in}{2.362054in}}%
\pgfpathlineto{\pgfqpoint{2.437171in}{2.333586in}}%
\pgfpathlineto{\pgfqpoint{2.442438in}{2.276292in}}%
\pgfpathlineto{\pgfqpoint{2.447706in}{2.294380in}}%
\pgfpathlineto{\pgfqpoint{2.452974in}{2.320654in}}%
\pgfpathlineto{\pgfqpoint{2.458241in}{2.331683in}}%
\pgfpathlineto{\pgfqpoint{2.463509in}{2.335979in}}%
\pgfpathlineto{\pgfqpoint{2.489847in}{2.335629in}}%
\pgfpathlineto{\pgfqpoint{2.495115in}{2.328890in}}%
\pgfpathlineto{\pgfqpoint{2.500382in}{2.306958in}}%
\pgfpathlineto{\pgfqpoint{2.531988in}{2.306958in}}%
\pgfpathlineto{\pgfqpoint{2.537256in}{2.259231in}}%
\pgfpathlineto{\pgfqpoint{2.542523in}{2.254885in}}%
\pgfpathlineto{\pgfqpoint{2.547791in}{2.314629in}}%
\pgfpathlineto{\pgfqpoint{2.553059in}{2.346380in}}%
\pgfpathlineto{\pgfqpoint{2.558326in}{2.357789in}}%
\pgfpathlineto{\pgfqpoint{2.574129in}{2.361554in}}%
\pgfpathlineto{\pgfqpoint{2.600467in}{2.386518in}}%
\pgfpathlineto{\pgfqpoint{2.626805in}{2.386361in}}%
\pgfpathlineto{\pgfqpoint{2.632073in}{2.383270in}}%
\pgfpathlineto{\pgfqpoint{2.637341in}{2.377640in}}%
\pgfpathlineto{\pgfqpoint{2.700552in}{2.376806in}}%
\pgfpathlineto{\pgfqpoint{2.705820in}{2.373404in}}%
\pgfpathlineto{\pgfqpoint{2.711088in}{2.367548in}}%
\pgfpathlineto{\pgfqpoint{2.742693in}{2.367548in}}%
\pgfpathlineto{\pgfqpoint{2.747961in}{2.360918in}}%
\pgfpathlineto{\pgfqpoint{2.753229in}{2.364645in}}%
\pgfpathlineto{\pgfqpoint{2.774299in}{2.364645in}}%
\pgfpathlineto{\pgfqpoint{2.779567in}{2.346033in}}%
\pgfpathlineto{\pgfqpoint{2.784834in}{2.343440in}}%
\pgfpathlineto{\pgfqpoint{2.811172in}{2.343440in}}%
\pgfpathlineto{\pgfqpoint{2.816440in}{2.287561in}}%
\pgfpathlineto{\pgfqpoint{2.826975in}{2.116477in}}%
\pgfpathlineto{\pgfqpoint{2.837511in}{2.060951in}}%
\pgfpathlineto{\pgfqpoint{2.853314in}{2.060951in}}%
\pgfpathlineto{\pgfqpoint{2.858581in}{2.124927in}}%
\pgfpathlineto{\pgfqpoint{2.863849in}{2.258100in}}%
\pgfpathlineto{\pgfqpoint{2.869116in}{2.350401in}}%
\pgfpathlineto{\pgfqpoint{2.874384in}{2.395462in}}%
\pgfpathlineto{\pgfqpoint{2.879652in}{2.403470in}}%
\pgfpathlineto{\pgfqpoint{2.900722in}{2.403470in}}%
\pgfpathlineto{\pgfqpoint{2.905990in}{2.412588in}}%
\pgfpathlineto{\pgfqpoint{2.921793in}{2.518392in}}%
\pgfpathlineto{\pgfqpoint{2.927060in}{2.535996in}}%
\pgfpathlineto{\pgfqpoint{2.937596in}{2.537045in}}%
\pgfpathlineto{\pgfqpoint{2.953398in}{2.537045in}}%
\pgfpathlineto{\pgfqpoint{2.958666in}{2.521170in}}%
\pgfpathlineto{\pgfqpoint{2.963934in}{2.491168in}}%
\pgfpathlineto{\pgfqpoint{2.969201in}{2.487669in}}%
\pgfpathlineto{\pgfqpoint{2.985004in}{2.487669in}}%
\pgfpathlineto{\pgfqpoint{2.990272in}{2.491641in}}%
\pgfpathlineto{\pgfqpoint{2.995540in}{2.517058in}}%
\pgfpathlineto{\pgfqpoint{3.000807in}{2.527455in}}%
\pgfpathlineto{\pgfqpoint{3.021878in}{2.527455in}}%
\pgfpathlineto{\pgfqpoint{3.027145in}{2.520801in}}%
\pgfpathlineto{\pgfqpoint{3.032413in}{2.498232in}}%
\pgfpathlineto{\pgfqpoint{3.037681in}{2.463853in}}%
\pgfpathlineto{\pgfqpoint{3.058751in}{2.463853in}}%
\pgfpathlineto{\pgfqpoint{3.069286in}{2.508832in}}%
\pgfpathlineto{\pgfqpoint{3.074554in}{2.516083in}}%
\pgfpathlineto{\pgfqpoint{3.095624in}{2.516083in}}%
\pgfpathlineto{\pgfqpoint{3.100892in}{2.506362in}}%
\pgfpathlineto{\pgfqpoint{3.111427in}{2.452353in}}%
\pgfpathlineto{\pgfqpoint{3.127230in}{2.452353in}}%
\pgfpathlineto{\pgfqpoint{3.132498in}{2.457470in}}%
\pgfpathlineto{\pgfqpoint{3.137766in}{2.486971in}}%
\pgfpathlineto{\pgfqpoint{3.143033in}{2.503121in}}%
\pgfpathlineto{\pgfqpoint{3.169371in}{2.502783in}}%
\pgfpathlineto{\pgfqpoint{3.174639in}{2.485931in}}%
\pgfpathlineto{\pgfqpoint{3.179907in}{2.455591in}}%
\pgfpathlineto{\pgfqpoint{3.206245in}{2.455591in}}%
\pgfpathlineto{\pgfqpoint{3.211512in}{2.479931in}}%
\pgfpathlineto{\pgfqpoint{3.216780in}{2.493044in}}%
\pgfpathlineto{\pgfqpoint{3.237850in}{2.493044in}}%
\pgfpathlineto{\pgfqpoint{3.243118in}{2.489563in}}%
\pgfpathlineto{\pgfqpoint{3.248386in}{2.469960in}}%
\pgfpathlineto{\pgfqpoint{3.253653in}{2.437762in}}%
\pgfpathlineto{\pgfqpoint{3.274724in}{2.437762in}}%
\pgfpathlineto{\pgfqpoint{3.285259in}{2.476833in}}%
\pgfpathlineto{\pgfqpoint{3.290527in}{2.482671in}}%
\pgfpathlineto{\pgfqpoint{3.311597in}{2.482671in}}%
\pgfpathlineto{\pgfqpoint{3.316865in}{2.471823in}}%
\pgfpathlineto{\pgfqpoint{3.327400in}{2.422788in}}%
\pgfpathlineto{\pgfqpoint{3.348471in}{2.422788in}}%
\pgfpathlineto{\pgfqpoint{3.353738in}{2.449530in}}%
\pgfpathlineto{\pgfqpoint{3.359006in}{2.468029in}}%
\pgfpathlineto{\pgfqpoint{3.364274in}{2.470715in}}%
\pgfpathlineto{\pgfqpoint{3.385344in}{2.470715in}}%
\pgfpathlineto{\pgfqpoint{3.390612in}{2.456924in}}%
\pgfpathlineto{\pgfqpoint{3.395879in}{2.429064in}}%
\pgfpathlineto{\pgfqpoint{3.401147in}{2.418819in}}%
\pgfpathlineto{\pgfqpoint{3.416950in}{2.418819in}}%
\pgfpathlineto{\pgfqpoint{3.422218in}{2.422898in}}%
\pgfpathlineto{\pgfqpoint{3.427485in}{2.448769in}}%
\pgfpathlineto{\pgfqpoint{3.432753in}{2.460489in}}%
\pgfpathlineto{\pgfqpoint{3.453823in}{2.460489in}}%
\pgfpathlineto{\pgfqpoint{3.459091in}{2.455776in}}%
\pgfpathlineto{\pgfqpoint{3.459091in}{2.455776in}}%
\pgfusepath{stroke}%
\end{pgfscope}%
\begin{pgfscope}%
\pgfpathrectangle{\pgfqpoint{0.500000in}{0.375000in}}{\pgfqpoint{3.100000in}{2.265000in}}%
\pgfusepath{clip}%
\pgfsetrectcap%
\pgfsetroundjoin%
\pgfsetlinewidth{1.505625pt}%
\definecolor{currentstroke}{rgb}{1.000000,0.000000,0.000000}%
\pgfsetstrokecolor{currentstroke}%
\pgfsetdash{}{0pt}%
\pgfpathmoveto{\pgfqpoint{0.640909in}{0.477955in}}%
\pgfpathlineto{\pgfqpoint{0.719924in}{0.479496in}}%
\pgfpathlineto{\pgfqpoint{0.735726in}{0.483549in}}%
\pgfpathlineto{\pgfqpoint{0.751529in}{0.491444in}}%
\pgfpathlineto{\pgfqpoint{0.767332in}{0.503888in}}%
\pgfpathlineto{\pgfqpoint{0.777867in}{0.515438in}}%
\pgfpathlineto{\pgfqpoint{0.788403in}{0.530780in}}%
\pgfpathlineto{\pgfqpoint{0.798938in}{0.551302in}}%
\pgfpathlineto{\pgfqpoint{0.809473in}{0.578898in}}%
\pgfpathlineto{\pgfqpoint{0.820008in}{0.616681in}}%
\pgfpathlineto{\pgfqpoint{0.830544in}{0.668941in}}%
\pgfpathlineto{\pgfqpoint{0.846347in}{0.773801in}}%
\pgfpathlineto{\pgfqpoint{0.883220in}{1.045060in}}%
\pgfpathlineto{\pgfqpoint{0.893755in}{1.101674in}}%
\pgfpathlineto{\pgfqpoint{0.899023in}{1.121357in}}%
\pgfpathlineto{\pgfqpoint{0.904291in}{1.133037in}}%
\pgfpathlineto{\pgfqpoint{0.909558in}{1.135195in}}%
\pgfpathlineto{\pgfqpoint{0.914826in}{1.127979in}}%
\pgfpathlineto{\pgfqpoint{0.925361in}{1.101863in}}%
\pgfpathlineto{\pgfqpoint{0.930629in}{1.101439in}}%
\pgfpathlineto{\pgfqpoint{0.935896in}{1.108967in}}%
\pgfpathlineto{\pgfqpoint{0.941164in}{1.126539in}}%
\pgfpathlineto{\pgfqpoint{0.946432in}{1.156335in}}%
\pgfpathlineto{\pgfqpoint{0.956967in}{1.248439in}}%
\pgfpathlineto{\pgfqpoint{0.967502in}{1.373778in}}%
\pgfpathlineto{\pgfqpoint{1.020178in}{2.074443in}}%
\pgfpathlineto{\pgfqpoint{1.025446in}{2.113256in}}%
\pgfpathlineto{\pgfqpoint{1.030714in}{2.130753in}}%
\pgfpathlineto{\pgfqpoint{1.035981in}{2.127418in}}%
\pgfpathlineto{\pgfqpoint{1.041249in}{2.106579in}}%
\pgfpathlineto{\pgfqpoint{1.093925in}{1.771855in}}%
\pgfpathlineto{\pgfqpoint{1.109728in}{1.659929in}}%
\pgfpathlineto{\pgfqpoint{1.120263in}{1.611770in}}%
\pgfpathlineto{\pgfqpoint{1.130799in}{1.582054in}}%
\pgfpathlineto{\pgfqpoint{1.141334in}{1.561724in}}%
\pgfpathlineto{\pgfqpoint{1.146602in}{1.557925in}}%
\pgfpathlineto{\pgfqpoint{1.151869in}{1.560114in}}%
\pgfpathlineto{\pgfqpoint{1.157137in}{1.568570in}}%
\pgfpathlineto{\pgfqpoint{1.167672in}{1.598513in}}%
\pgfpathlineto{\pgfqpoint{1.178207in}{1.626161in}}%
\pgfpathlineto{\pgfqpoint{1.183475in}{1.634447in}}%
\pgfpathlineto{\pgfqpoint{1.188743in}{1.632398in}}%
\pgfpathlineto{\pgfqpoint{1.194010in}{1.619951in}}%
\pgfpathlineto{\pgfqpoint{1.209813in}{1.557940in}}%
\pgfpathlineto{\pgfqpoint{1.215081in}{1.549155in}}%
\pgfpathlineto{\pgfqpoint{1.220348in}{1.548687in}}%
\pgfpathlineto{\pgfqpoint{1.230884in}{1.550248in}}%
\pgfpathlineto{\pgfqpoint{1.236151in}{1.543550in}}%
\pgfpathlineto{\pgfqpoint{1.241419in}{1.531547in}}%
\pgfpathlineto{\pgfqpoint{1.246686in}{1.508310in}}%
\pgfpathlineto{\pgfqpoint{1.267757in}{1.390883in}}%
\pgfpathlineto{\pgfqpoint{1.278292in}{1.360569in}}%
\pgfpathlineto{\pgfqpoint{1.283560in}{1.349505in}}%
\pgfpathlineto{\pgfqpoint{1.294095in}{1.312376in}}%
\pgfpathlineto{\pgfqpoint{1.299363in}{1.287676in}}%
\pgfpathlineto{\pgfqpoint{1.325701in}{1.117403in}}%
\pgfpathlineto{\pgfqpoint{1.336236in}{1.066850in}}%
\pgfpathlineto{\pgfqpoint{1.346771in}{1.026597in}}%
\pgfpathlineto{\pgfqpoint{1.362574in}{0.961949in}}%
\pgfpathlineto{\pgfqpoint{1.373110in}{0.930078in}}%
\pgfpathlineto{\pgfqpoint{1.399448in}{0.874163in}}%
\pgfpathlineto{\pgfqpoint{1.404715in}{0.873575in}}%
\pgfpathlineto{\pgfqpoint{1.409983in}{0.881816in}}%
\pgfpathlineto{\pgfqpoint{1.415251in}{0.899018in}}%
\pgfpathlineto{\pgfqpoint{1.425786in}{0.956603in}}%
\pgfpathlineto{\pgfqpoint{1.441589in}{1.078508in}}%
\pgfpathlineto{\pgfqpoint{1.478462in}{1.390925in}}%
\pgfpathlineto{\pgfqpoint{1.483730in}{1.427464in}}%
\pgfpathlineto{\pgfqpoint{1.488997in}{1.451851in}}%
\pgfpathlineto{\pgfqpoint{1.494265in}{1.458421in}}%
\pgfpathlineto{\pgfqpoint{1.499533in}{1.442731in}}%
\pgfpathlineto{\pgfqpoint{1.504800in}{1.403512in}}%
\pgfpathlineto{\pgfqpoint{1.510068in}{1.341948in}}%
\pgfpathlineto{\pgfqpoint{1.520603in}{1.166061in}}%
\pgfpathlineto{\pgfqpoint{1.541674in}{0.788165in}}%
\pgfpathlineto{\pgfqpoint{1.552209in}{0.659783in}}%
\pgfpathlineto{\pgfqpoint{1.562744in}{0.577722in}}%
\pgfpathlineto{\pgfqpoint{1.568012in}{0.550814in}}%
\pgfpathlineto{\pgfqpoint{1.573280in}{0.533687in}}%
\pgfpathlineto{\pgfqpoint{1.578547in}{0.524841in}}%
\pgfpathlineto{\pgfqpoint{1.583815in}{0.522080in}}%
\pgfpathlineto{\pgfqpoint{1.589082in}{0.523732in}}%
\pgfpathlineto{\pgfqpoint{1.615421in}{0.544007in}}%
\pgfpathlineto{\pgfqpoint{1.647026in}{0.581094in}}%
\pgfpathlineto{\pgfqpoint{1.726041in}{0.679007in}}%
\pgfpathlineto{\pgfqpoint{1.736576in}{0.692480in}}%
\pgfpathlineto{\pgfqpoint{1.752379in}{0.720794in}}%
\pgfpathlineto{\pgfqpoint{1.762914in}{0.748318in}}%
\pgfpathlineto{\pgfqpoint{1.773449in}{0.791389in}}%
\pgfpathlineto{\pgfqpoint{1.794520in}{0.904832in}}%
\pgfpathlineto{\pgfqpoint{1.820858in}{1.039833in}}%
\pgfpathlineto{\pgfqpoint{1.831393in}{1.078913in}}%
\pgfpathlineto{\pgfqpoint{1.836661in}{1.092756in}}%
\pgfpathlineto{\pgfqpoint{1.841929in}{1.101329in}}%
\pgfpathlineto{\pgfqpoint{1.847196in}{1.099839in}}%
\pgfpathlineto{\pgfqpoint{1.852464in}{1.083720in}}%
\pgfpathlineto{\pgfqpoint{1.857732in}{1.053581in}}%
\pgfpathlineto{\pgfqpoint{1.884070in}{0.862142in}}%
\pgfpathlineto{\pgfqpoint{1.894605in}{0.814892in}}%
\pgfpathlineto{\pgfqpoint{1.905140in}{0.777717in}}%
\pgfpathlineto{\pgfqpoint{1.910408in}{0.766009in}}%
\pgfpathlineto{\pgfqpoint{1.915675in}{0.759062in}}%
\pgfpathlineto{\pgfqpoint{1.920943in}{0.756202in}}%
\pgfpathlineto{\pgfqpoint{1.947281in}{0.749724in}}%
\pgfpathlineto{\pgfqpoint{1.968352in}{0.742936in}}%
\pgfpathlineto{\pgfqpoint{1.978887in}{0.736337in}}%
\pgfpathlineto{\pgfqpoint{1.989422in}{0.727678in}}%
\pgfpathlineto{\pgfqpoint{1.999958in}{0.716800in}}%
\pgfpathlineto{\pgfqpoint{2.021028in}{0.698631in}}%
\pgfpathlineto{\pgfqpoint{2.031563in}{0.687939in}}%
\pgfpathlineto{\pgfqpoint{2.042099in}{0.677817in}}%
\pgfpathlineto{\pgfqpoint{2.052634in}{0.669723in}}%
\pgfpathlineto{\pgfqpoint{2.063169in}{0.662577in}}%
\pgfpathlineto{\pgfqpoint{2.073704in}{0.658541in}}%
\pgfpathlineto{\pgfqpoint{2.089507in}{0.647599in}}%
\pgfpathlineto{\pgfqpoint{2.115845in}{0.621748in}}%
\pgfpathlineto{\pgfqpoint{2.142184in}{0.590477in}}%
\pgfpathlineto{\pgfqpoint{2.163254in}{0.565703in}}%
\pgfpathlineto{\pgfqpoint{2.200127in}{0.527260in}}%
\pgfpathlineto{\pgfqpoint{2.215930in}{0.515035in}}%
\pgfpathlineto{\pgfqpoint{2.231733in}{0.509296in}}%
\pgfpathlineto{\pgfqpoint{2.247536in}{0.509002in}}%
\pgfpathlineto{\pgfqpoint{2.263339in}{0.513867in}}%
\pgfpathlineto{\pgfqpoint{2.279142in}{0.522705in}}%
\pgfpathlineto{\pgfqpoint{2.284410in}{0.526138in}}%
\pgfpathlineto{\pgfqpoint{2.289677in}{0.531551in}}%
\pgfpathlineto{\pgfqpoint{2.294945in}{0.539973in}}%
\pgfpathlineto{\pgfqpoint{2.300212in}{0.554366in}}%
\pgfpathlineto{\pgfqpoint{2.305480in}{0.578867in}}%
\pgfpathlineto{\pgfqpoint{2.310748in}{0.616056in}}%
\pgfpathlineto{\pgfqpoint{2.321283in}{0.733300in}}%
\pgfpathlineto{\pgfqpoint{2.331818in}{0.905231in}}%
\pgfpathlineto{\pgfqpoint{2.347621in}{1.226266in}}%
\pgfpathlineto{\pgfqpoint{2.384494in}{1.997327in}}%
\pgfpathlineto{\pgfqpoint{2.389762in}{2.077149in}}%
\pgfpathlineto{\pgfqpoint{2.395030in}{2.133366in}}%
\pgfpathlineto{\pgfqpoint{2.400297in}{2.161097in}}%
\pgfpathlineto{\pgfqpoint{2.405565in}{2.158864in}}%
\pgfpathlineto{\pgfqpoint{2.410833in}{2.131933in}}%
\pgfpathlineto{\pgfqpoint{2.426636in}{2.014488in}}%
\pgfpathlineto{\pgfqpoint{2.437171in}{1.981547in}}%
\pgfpathlineto{\pgfqpoint{2.442438in}{1.975633in}}%
\pgfpathlineto{\pgfqpoint{2.447706in}{1.976764in}}%
\pgfpathlineto{\pgfqpoint{2.452974in}{1.984948in}}%
\pgfpathlineto{\pgfqpoint{2.458241in}{2.001084in}}%
\pgfpathlineto{\pgfqpoint{2.468777in}{2.055519in}}%
\pgfpathlineto{\pgfqpoint{2.479312in}{2.117699in}}%
\pgfpathlineto{\pgfqpoint{2.484579in}{2.135279in}}%
\pgfpathlineto{\pgfqpoint{2.489847in}{2.140295in}}%
\pgfpathlineto{\pgfqpoint{2.495115in}{2.138959in}}%
\pgfpathlineto{\pgfqpoint{2.500382in}{2.127384in}}%
\pgfpathlineto{\pgfqpoint{2.505650in}{2.103962in}}%
\pgfpathlineto{\pgfqpoint{2.526720in}{1.976748in}}%
\pgfpathlineto{\pgfqpoint{2.531988in}{1.961841in}}%
\pgfpathlineto{\pgfqpoint{2.537256in}{1.953106in}}%
\pgfpathlineto{\pgfqpoint{2.542523in}{1.952835in}}%
\pgfpathlineto{\pgfqpoint{2.547791in}{1.961077in}}%
\pgfpathlineto{\pgfqpoint{2.553059in}{1.977549in}}%
\pgfpathlineto{\pgfqpoint{2.563594in}{2.031638in}}%
\pgfpathlineto{\pgfqpoint{2.605735in}{2.299258in}}%
\pgfpathlineto{\pgfqpoint{2.616270in}{2.347234in}}%
\pgfpathlineto{\pgfqpoint{2.621538in}{2.359088in}}%
\pgfpathlineto{\pgfqpoint{2.626805in}{2.364345in}}%
\pgfpathlineto{\pgfqpoint{2.632073in}{2.365059in}}%
\pgfpathlineto{\pgfqpoint{2.653144in}{2.359390in}}%
\pgfpathlineto{\pgfqpoint{2.700552in}{2.359564in}}%
\pgfpathlineto{\pgfqpoint{2.732158in}{2.346139in}}%
\pgfpathlineto{\pgfqpoint{2.758496in}{2.344598in}}%
\pgfpathlineto{\pgfqpoint{2.763764in}{2.337133in}}%
\pgfpathlineto{\pgfqpoint{2.769031in}{2.319138in}}%
\pgfpathlineto{\pgfqpoint{2.774299in}{2.290080in}}%
\pgfpathlineto{\pgfqpoint{2.795370in}{2.136033in}}%
\pgfpathlineto{\pgfqpoint{2.805905in}{2.099849in}}%
\pgfpathlineto{\pgfqpoint{2.816440in}{2.065234in}}%
\pgfpathlineto{\pgfqpoint{2.821708in}{2.040310in}}%
\pgfpathlineto{\pgfqpoint{2.826975in}{2.003616in}}%
\pgfpathlineto{\pgfqpoint{2.842778in}{1.862017in}}%
\pgfpathlineto{\pgfqpoint{2.848046in}{1.841977in}}%
\pgfpathlineto{\pgfqpoint{2.853314in}{1.838851in}}%
\pgfpathlineto{\pgfqpoint{2.858581in}{1.842850in}}%
\pgfpathlineto{\pgfqpoint{2.863849in}{1.856277in}}%
\pgfpathlineto{\pgfqpoint{2.874384in}{1.903748in}}%
\pgfpathlineto{\pgfqpoint{2.884919in}{1.966414in}}%
\pgfpathlineto{\pgfqpoint{2.916525in}{2.219232in}}%
\pgfpathlineto{\pgfqpoint{2.932328in}{2.371932in}}%
\pgfpathlineto{\pgfqpoint{2.937596in}{2.406472in}}%
\pgfpathlineto{\pgfqpoint{2.942863in}{2.429362in}}%
\pgfpathlineto{\pgfqpoint{2.948131in}{2.437418in}}%
\pgfpathlineto{\pgfqpoint{2.953398in}{2.433700in}}%
\pgfpathlineto{\pgfqpoint{2.969201in}{2.402256in}}%
\pgfpathlineto{\pgfqpoint{2.974469in}{2.398743in}}%
\pgfpathlineto{\pgfqpoint{2.985004in}{2.406264in}}%
\pgfpathlineto{\pgfqpoint{2.995540in}{2.415406in}}%
\pgfpathlineto{\pgfqpoint{3.000807in}{2.417893in}}%
\pgfpathlineto{\pgfqpoint{3.016610in}{2.430842in}}%
\pgfpathlineto{\pgfqpoint{3.021878in}{2.428750in}}%
\pgfpathlineto{\pgfqpoint{3.027145in}{2.421268in}}%
\pgfpathlineto{\pgfqpoint{3.042948in}{2.387884in}}%
\pgfpathlineto{\pgfqpoint{3.048216in}{2.386029in}}%
\pgfpathlineto{\pgfqpoint{3.058751in}{2.393211in}}%
\pgfpathlineto{\pgfqpoint{3.090357in}{2.417868in}}%
\pgfpathlineto{\pgfqpoint{3.095624in}{2.415017in}}%
\pgfpathlineto{\pgfqpoint{3.100892in}{2.406889in}}%
\pgfpathlineto{\pgfqpoint{3.111427in}{2.383615in}}%
\pgfpathlineto{\pgfqpoint{3.116695in}{2.377738in}}%
\pgfpathlineto{\pgfqpoint{3.121963in}{2.378541in}}%
\pgfpathlineto{\pgfqpoint{3.153568in}{2.403733in}}%
\pgfpathlineto{\pgfqpoint{3.158836in}{2.408902in}}%
\pgfpathlineto{\pgfqpoint{3.164104in}{2.408879in}}%
\pgfpathlineto{\pgfqpoint{3.169371in}{2.403361in}}%
\pgfpathlineto{\pgfqpoint{3.190442in}{2.365478in}}%
\pgfpathlineto{\pgfqpoint{3.200977in}{2.371614in}}%
\pgfpathlineto{\pgfqpoint{3.211512in}{2.379638in}}%
\pgfpathlineto{\pgfqpoint{3.216780in}{2.381978in}}%
\pgfpathlineto{\pgfqpoint{3.232583in}{2.396886in}}%
\pgfpathlineto{\pgfqpoint{3.237850in}{2.396138in}}%
\pgfpathlineto{\pgfqpoint{3.243118in}{2.389892in}}%
\pgfpathlineto{\pgfqpoint{3.258921in}{2.358458in}}%
\pgfpathlineto{\pgfqpoint{3.264189in}{2.356783in}}%
\pgfpathlineto{\pgfqpoint{3.274724in}{2.364836in}}%
\pgfpathlineto{\pgfqpoint{3.285259in}{2.373319in}}%
\pgfpathlineto{\pgfqpoint{3.295794in}{2.380669in}}%
\pgfpathlineto{\pgfqpoint{3.301062in}{2.385936in}}%
\pgfpathlineto{\pgfqpoint{3.306330in}{2.387897in}}%
\pgfpathlineto{\pgfqpoint{3.311597in}{2.384408in}}%
\pgfpathlineto{\pgfqpoint{3.316865in}{2.375856in}}%
\pgfpathlineto{\pgfqpoint{3.327400in}{2.351721in}}%
\pgfpathlineto{\pgfqpoint{3.332668in}{2.345154in}}%
\pgfpathlineto{\pgfqpoint{3.337935in}{2.345100in}}%
\pgfpathlineto{\pgfqpoint{3.364274in}{2.363266in}}%
\pgfpathlineto{\pgfqpoint{3.374809in}{2.374349in}}%
\pgfpathlineto{\pgfqpoint{3.380076in}{2.375650in}}%
\pgfpathlineto{\pgfqpoint{3.385344in}{2.371397in}}%
\pgfpathlineto{\pgfqpoint{3.395879in}{2.350097in}}%
\pgfpathlineto{\pgfqpoint{3.401147in}{2.340171in}}%
\pgfpathlineto{\pgfqpoint{3.406415in}{2.336099in}}%
\pgfpathlineto{\pgfqpoint{3.411682in}{2.338614in}}%
\pgfpathlineto{\pgfqpoint{3.448556in}{2.367293in}}%
\pgfpathlineto{\pgfqpoint{3.453823in}{2.365843in}}%
\pgfpathlineto{\pgfqpoint{3.459091in}{2.359087in}}%
\pgfpathlineto{\pgfqpoint{3.459091in}{2.359087in}}%
\pgfusepath{stroke}%
\end{pgfscope}%
\begin{pgfscope}%
\pgfpathrectangle{\pgfqpoint{0.500000in}{0.375000in}}{\pgfqpoint{3.100000in}{2.265000in}}%
\pgfusepath{clip}%
\pgfsetrectcap%
\pgfsetroundjoin%
\pgfsetlinewidth{1.505625pt}%
\definecolor{currentstroke}{rgb}{0.000000,0.500000,0.000000}%
\pgfsetstrokecolor{currentstroke}%
\pgfsetdash{}{0pt}%
\pgfpathmoveto{\pgfqpoint{0.640909in}{0.477955in}}%
\pgfpathlineto{\pgfqpoint{0.709388in}{0.478884in}}%
\pgfpathlineto{\pgfqpoint{0.725191in}{0.482382in}}%
\pgfpathlineto{\pgfqpoint{0.746262in}{0.491405in}}%
\pgfpathlineto{\pgfqpoint{0.777867in}{0.508911in}}%
\pgfpathlineto{\pgfqpoint{0.788403in}{0.516769in}}%
\pgfpathlineto{\pgfqpoint{0.798938in}{0.528154in}}%
\pgfpathlineto{\pgfqpoint{0.809473in}{0.545565in}}%
\pgfpathlineto{\pgfqpoint{0.820008in}{0.572308in}}%
\pgfpathlineto{\pgfqpoint{0.830544in}{0.610930in}}%
\pgfpathlineto{\pgfqpoint{0.846347in}{0.672364in}}%
\pgfpathlineto{\pgfqpoint{0.856882in}{0.697523in}}%
\pgfpathlineto{\pgfqpoint{0.862150in}{0.702640in}}%
\pgfpathlineto{\pgfqpoint{0.867417in}{0.702289in}}%
\pgfpathlineto{\pgfqpoint{0.872685in}{0.696674in}}%
\pgfpathlineto{\pgfqpoint{0.877952in}{0.685978in}}%
\pgfpathlineto{\pgfqpoint{0.883220in}{0.670691in}}%
\pgfpathlineto{\pgfqpoint{0.893755in}{0.624766in}}%
\pgfpathlineto{\pgfqpoint{0.909558in}{0.544696in}}%
\pgfpathlineto{\pgfqpoint{0.914826in}{0.539271in}}%
\pgfpathlineto{\pgfqpoint{0.920093in}{0.547567in}}%
\pgfpathlineto{\pgfqpoint{0.925361in}{0.551111in}}%
\pgfpathlineto{\pgfqpoint{0.930629in}{0.550856in}}%
\pgfpathlineto{\pgfqpoint{0.935896in}{0.564168in}}%
\pgfpathlineto{\pgfqpoint{0.941164in}{0.603153in}}%
\pgfpathlineto{\pgfqpoint{0.967502in}{0.890050in}}%
\pgfpathlineto{\pgfqpoint{0.972770in}{0.921252in}}%
\pgfpathlineto{\pgfqpoint{0.978037in}{0.938596in}}%
\pgfpathlineto{\pgfqpoint{0.983305in}{0.946449in}}%
\pgfpathlineto{\pgfqpoint{0.988573in}{0.950325in}}%
\pgfpathlineto{\pgfqpoint{0.993840in}{0.941485in}}%
\pgfpathlineto{\pgfqpoint{0.999108in}{0.917955in}}%
\pgfpathlineto{\pgfqpoint{1.004376in}{0.877134in}}%
\pgfpathlineto{\pgfqpoint{1.014911in}{0.756198in}}%
\pgfpathlineto{\pgfqpoint{1.030714in}{0.575100in}}%
\pgfpathlineto{\pgfqpoint{1.035981in}{0.550828in}}%
\pgfpathlineto{\pgfqpoint{1.041249in}{0.562633in}}%
\pgfpathlineto{\pgfqpoint{1.051784in}{0.605296in}}%
\pgfpathlineto{\pgfqpoint{1.057052in}{0.615018in}}%
\pgfpathlineto{\pgfqpoint{1.062319in}{0.616616in}}%
\pgfpathlineto{\pgfqpoint{1.067587in}{0.611937in}}%
\pgfpathlineto{\pgfqpoint{1.072855in}{0.602047in}}%
\pgfpathlineto{\pgfqpoint{1.083390in}{0.571472in}}%
\pgfpathlineto{\pgfqpoint{1.088658in}{0.565939in}}%
\pgfpathlineto{\pgfqpoint{1.093925in}{0.571558in}}%
\pgfpathlineto{\pgfqpoint{1.104460in}{0.593676in}}%
\pgfpathlineto{\pgfqpoint{1.109728in}{0.599562in}}%
\pgfpathlineto{\pgfqpoint{1.114996in}{0.597605in}}%
\pgfpathlineto{\pgfqpoint{1.120263in}{0.587770in}}%
\pgfpathlineto{\pgfqpoint{1.130799in}{0.553643in}}%
\pgfpathlineto{\pgfqpoint{1.141334in}{0.517773in}}%
\pgfpathlineto{\pgfqpoint{1.146602in}{0.509656in}}%
\pgfpathlineto{\pgfqpoint{1.151869in}{0.513908in}}%
\pgfpathlineto{\pgfqpoint{1.172940in}{0.562604in}}%
\pgfpathlineto{\pgfqpoint{1.178207in}{0.567115in}}%
\pgfpathlineto{\pgfqpoint{1.183475in}{0.566547in}}%
\pgfpathlineto{\pgfqpoint{1.194010in}{0.548993in}}%
\pgfpathlineto{\pgfqpoint{1.199278in}{0.546376in}}%
\pgfpathlineto{\pgfqpoint{1.204545in}{0.547095in}}%
\pgfpathlineto{\pgfqpoint{1.220348in}{0.543200in}}%
\pgfpathlineto{\pgfqpoint{1.230884in}{0.544453in}}%
\pgfpathlineto{\pgfqpoint{1.241419in}{0.529115in}}%
\pgfpathlineto{\pgfqpoint{1.246686in}{0.534157in}}%
\pgfpathlineto{\pgfqpoint{1.257222in}{0.565391in}}%
\pgfpathlineto{\pgfqpoint{1.262489in}{0.574214in}}%
\pgfpathlineto{\pgfqpoint{1.267757in}{0.575357in}}%
\pgfpathlineto{\pgfqpoint{1.273025in}{0.569074in}}%
\pgfpathlineto{\pgfqpoint{1.283560in}{0.548736in}}%
\pgfpathlineto{\pgfqpoint{1.288828in}{0.529878in}}%
\pgfpathlineto{\pgfqpoint{1.294095in}{0.525335in}}%
\pgfpathlineto{\pgfqpoint{1.315166in}{0.593088in}}%
\pgfpathlineto{\pgfqpoint{1.320433in}{0.603381in}}%
\pgfpathlineto{\pgfqpoint{1.325701in}{0.605747in}}%
\pgfpathlineto{\pgfqpoint{1.330969in}{0.598614in}}%
\pgfpathlineto{\pgfqpoint{1.357307in}{0.525312in}}%
\pgfpathlineto{\pgfqpoint{1.362574in}{0.525062in}}%
\pgfpathlineto{\pgfqpoint{1.378377in}{0.529012in}}%
\pgfpathlineto{\pgfqpoint{1.388912in}{0.527920in}}%
\pgfpathlineto{\pgfqpoint{1.394180in}{0.522674in}}%
\pgfpathlineto{\pgfqpoint{1.399448in}{0.511640in}}%
\pgfpathlineto{\pgfqpoint{1.404715in}{0.510091in}}%
\pgfpathlineto{\pgfqpoint{1.409983in}{0.532648in}}%
\pgfpathlineto{\pgfqpoint{1.425786in}{0.650092in}}%
\pgfpathlineto{\pgfqpoint{1.436321in}{0.720827in}}%
\pgfpathlineto{\pgfqpoint{1.441589in}{0.746216in}}%
\pgfpathlineto{\pgfqpoint{1.446856in}{0.763209in}}%
\pgfpathlineto{\pgfqpoint{1.452124in}{0.772577in}}%
\pgfpathlineto{\pgfqpoint{1.457392in}{0.777238in}}%
\pgfpathlineto{\pgfqpoint{1.462659in}{0.774922in}}%
\pgfpathlineto{\pgfqpoint{1.467927in}{0.764617in}}%
\pgfpathlineto{\pgfqpoint{1.473195in}{0.744365in}}%
\pgfpathlineto{\pgfqpoint{1.478462in}{0.712189in}}%
\pgfpathlineto{\pgfqpoint{1.494265in}{0.586030in}}%
\pgfpathlineto{\pgfqpoint{1.499533in}{0.584816in}}%
\pgfpathlineto{\pgfqpoint{1.504800in}{0.625446in}}%
\pgfpathlineto{\pgfqpoint{1.520603in}{0.785833in}}%
\pgfpathlineto{\pgfqpoint{1.525871in}{0.818092in}}%
\pgfpathlineto{\pgfqpoint{1.531138in}{0.832041in}}%
\pgfpathlineto{\pgfqpoint{1.536406in}{0.826667in}}%
\pgfpathlineto{\pgfqpoint{1.541674in}{0.801747in}}%
\pgfpathlineto{\pgfqpoint{1.552209in}{0.712154in}}%
\pgfpathlineto{\pgfqpoint{1.568012in}{0.559353in}}%
\pgfpathlineto{\pgfqpoint{1.573280in}{0.523059in}}%
\pgfpathlineto{\pgfqpoint{1.578547in}{0.499558in}}%
\pgfpathlineto{\pgfqpoint{1.583815in}{0.492105in}}%
\pgfpathlineto{\pgfqpoint{1.615421in}{0.505082in}}%
\pgfpathlineto{\pgfqpoint{1.647026in}{0.508259in}}%
\pgfpathlineto{\pgfqpoint{1.715506in}{0.507342in}}%
\pgfpathlineto{\pgfqpoint{1.731308in}{0.504798in}}%
\pgfpathlineto{\pgfqpoint{1.736576in}{0.507371in}}%
\pgfpathlineto{\pgfqpoint{1.752379in}{0.523431in}}%
\pgfpathlineto{\pgfqpoint{1.762914in}{0.545327in}}%
\pgfpathlineto{\pgfqpoint{1.773449in}{0.587680in}}%
\pgfpathlineto{\pgfqpoint{1.783985in}{0.628212in}}%
\pgfpathlineto{\pgfqpoint{1.789252in}{0.642221in}}%
\pgfpathlineto{\pgfqpoint{1.794520in}{0.650192in}}%
\pgfpathlineto{\pgfqpoint{1.799788in}{0.653557in}}%
\pgfpathlineto{\pgfqpoint{1.805055in}{0.653052in}}%
\pgfpathlineto{\pgfqpoint{1.810323in}{0.647866in}}%
\pgfpathlineto{\pgfqpoint{1.820858in}{0.624152in}}%
\pgfpathlineto{\pgfqpoint{1.831393in}{0.588827in}}%
\pgfpathlineto{\pgfqpoint{1.841929in}{0.543719in}}%
\pgfpathlineto{\pgfqpoint{1.847196in}{0.529650in}}%
\pgfpathlineto{\pgfqpoint{1.852464in}{0.547602in}}%
\pgfpathlineto{\pgfqpoint{1.862999in}{0.607284in}}%
\pgfpathlineto{\pgfqpoint{1.868267in}{0.623621in}}%
\pgfpathlineto{\pgfqpoint{1.873534in}{0.628374in}}%
\pgfpathlineto{\pgfqpoint{1.878802in}{0.625417in}}%
\pgfpathlineto{\pgfqpoint{1.884070in}{0.615903in}}%
\pgfpathlineto{\pgfqpoint{1.889337in}{0.600110in}}%
\pgfpathlineto{\pgfqpoint{1.899873in}{0.552495in}}%
\pgfpathlineto{\pgfqpoint{1.910408in}{0.502912in}}%
\pgfpathlineto{\pgfqpoint{1.915675in}{0.486714in}}%
\pgfpathlineto{\pgfqpoint{1.920943in}{0.480863in}}%
\pgfpathlineto{\pgfqpoint{1.973619in}{0.485188in}}%
\pgfpathlineto{\pgfqpoint{1.984155in}{0.489879in}}%
\pgfpathlineto{\pgfqpoint{1.999958in}{0.495454in}}%
\pgfpathlineto{\pgfqpoint{2.015760in}{0.495042in}}%
\pgfpathlineto{\pgfqpoint{2.031563in}{0.491904in}}%
\pgfpathlineto{\pgfqpoint{2.047366in}{0.492802in}}%
\pgfpathlineto{\pgfqpoint{2.057901in}{0.491755in}}%
\pgfpathlineto{\pgfqpoint{2.078972in}{0.485532in}}%
\pgfpathlineto{\pgfqpoint{2.089507in}{0.488447in}}%
\pgfpathlineto{\pgfqpoint{2.110578in}{0.496168in}}%
\pgfpathlineto{\pgfqpoint{2.121113in}{0.498342in}}%
\pgfpathlineto{\pgfqpoint{2.142184in}{0.497332in}}%
\pgfpathlineto{\pgfqpoint{2.184325in}{0.495110in}}%
\pgfpathlineto{\pgfqpoint{2.205395in}{0.492827in}}%
\pgfpathlineto{\pgfqpoint{2.226466in}{0.486185in}}%
\pgfpathlineto{\pgfqpoint{2.237001in}{0.483078in}}%
\pgfpathlineto{\pgfqpoint{2.247536in}{0.483596in}}%
\pgfpathlineto{\pgfqpoint{2.284410in}{0.497610in}}%
\pgfpathlineto{\pgfqpoint{2.289677in}{0.502747in}}%
\pgfpathlineto{\pgfqpoint{2.294945in}{0.512753in}}%
\pgfpathlineto{\pgfqpoint{2.300212in}{0.535145in}}%
\pgfpathlineto{\pgfqpoint{2.305480in}{0.577433in}}%
\pgfpathlineto{\pgfqpoint{2.316015in}{0.710073in}}%
\pgfpathlineto{\pgfqpoint{2.337086in}{1.036553in}}%
\pgfpathlineto{\pgfqpoint{2.347621in}{1.144456in}}%
\pgfpathlineto{\pgfqpoint{2.352889in}{1.172958in}}%
\pgfpathlineto{\pgfqpoint{2.358156in}{1.183860in}}%
\pgfpathlineto{\pgfqpoint{2.363424in}{1.176732in}}%
\pgfpathlineto{\pgfqpoint{2.368692in}{1.151252in}}%
\pgfpathlineto{\pgfqpoint{2.373959in}{1.108049in}}%
\pgfpathlineto{\pgfqpoint{2.379227in}{1.045952in}}%
\pgfpathlineto{\pgfqpoint{2.389762in}{0.873375in}}%
\pgfpathlineto{\pgfqpoint{2.400297in}{0.691488in}}%
\pgfpathlineto{\pgfqpoint{2.405565in}{0.641910in}}%
\pgfpathlineto{\pgfqpoint{2.410833in}{0.645350in}}%
\pgfpathlineto{\pgfqpoint{2.416100in}{0.670854in}}%
\pgfpathlineto{\pgfqpoint{2.421368in}{0.686493in}}%
\pgfpathlineto{\pgfqpoint{2.426636in}{0.682881in}}%
\pgfpathlineto{\pgfqpoint{2.442438in}{0.634540in}}%
\pgfpathlineto{\pgfqpoint{2.447706in}{0.636756in}}%
\pgfpathlineto{\pgfqpoint{2.452974in}{0.650314in}}%
\pgfpathlineto{\pgfqpoint{2.463509in}{0.684665in}}%
\pgfpathlineto{\pgfqpoint{2.468777in}{0.692301in}}%
\pgfpathlineto{\pgfqpoint{2.474044in}{0.691520in}}%
\pgfpathlineto{\pgfqpoint{2.484579in}{0.676584in}}%
\pgfpathlineto{\pgfqpoint{2.489847in}{0.665118in}}%
\pgfpathlineto{\pgfqpoint{2.500382in}{0.625811in}}%
\pgfpathlineto{\pgfqpoint{2.505650in}{0.623911in}}%
\pgfpathlineto{\pgfqpoint{2.516185in}{0.661193in}}%
\pgfpathlineto{\pgfqpoint{2.521453in}{0.668289in}}%
\pgfpathlineto{\pgfqpoint{2.526720in}{0.658488in}}%
\pgfpathlineto{\pgfqpoint{2.537256in}{0.624826in}}%
\pgfpathlineto{\pgfqpoint{2.542523in}{0.624262in}}%
\pgfpathlineto{\pgfqpoint{2.547791in}{0.639865in}}%
\pgfpathlineto{\pgfqpoint{2.558326in}{0.684731in}}%
\pgfpathlineto{\pgfqpoint{2.563594in}{0.699048in}}%
\pgfpathlineto{\pgfqpoint{2.568862in}{0.704763in}}%
\pgfpathlineto{\pgfqpoint{2.574129in}{0.704300in}}%
\pgfpathlineto{\pgfqpoint{2.579397in}{0.702088in}}%
\pgfpathlineto{\pgfqpoint{2.584664in}{0.702928in}}%
\pgfpathlineto{\pgfqpoint{2.589932in}{0.696034in}}%
\pgfpathlineto{\pgfqpoint{2.595200in}{0.676937in}}%
\pgfpathlineto{\pgfqpoint{2.605735in}{0.606958in}}%
\pgfpathlineto{\pgfqpoint{2.616270in}{0.534529in}}%
\pgfpathlineto{\pgfqpoint{2.621538in}{0.509853in}}%
\pgfpathlineto{\pgfqpoint{2.626805in}{0.494263in}}%
\pgfpathlineto{\pgfqpoint{2.632073in}{0.487390in}}%
\pgfpathlineto{\pgfqpoint{2.642608in}{0.486109in}}%
\pgfpathlineto{\pgfqpoint{2.700552in}{0.485802in}}%
\pgfpathlineto{\pgfqpoint{2.711088in}{0.486314in}}%
\pgfpathlineto{\pgfqpoint{2.732158in}{0.488153in}}%
\pgfpathlineto{\pgfqpoint{2.758496in}{0.486474in}}%
\pgfpathlineto{\pgfqpoint{2.763764in}{0.498499in}}%
\pgfpathlineto{\pgfqpoint{2.784834in}{0.609584in}}%
\pgfpathlineto{\pgfqpoint{2.790102in}{0.619852in}}%
\pgfpathlineto{\pgfqpoint{2.795370in}{0.619300in}}%
\pgfpathlineto{\pgfqpoint{2.800637in}{0.611021in}}%
\pgfpathlineto{\pgfqpoint{2.805905in}{0.598153in}}%
\pgfpathlineto{\pgfqpoint{2.811172in}{0.581087in}}%
\pgfpathlineto{\pgfqpoint{2.816440in}{0.556641in}}%
\pgfpathlineto{\pgfqpoint{2.821708in}{0.540413in}}%
\pgfpathlineto{\pgfqpoint{2.826975in}{0.554995in}}%
\pgfpathlineto{\pgfqpoint{2.837511in}{0.619618in}}%
\pgfpathlineto{\pgfqpoint{2.842778in}{0.627652in}}%
\pgfpathlineto{\pgfqpoint{2.853314in}{0.618792in}}%
\pgfpathlineto{\pgfqpoint{2.858581in}{0.625776in}}%
\pgfpathlineto{\pgfqpoint{2.863849in}{0.650979in}}%
\pgfpathlineto{\pgfqpoint{2.879652in}{0.748936in}}%
\pgfpathlineto{\pgfqpoint{2.884919in}{0.769344in}}%
\pgfpathlineto{\pgfqpoint{2.890187in}{0.777549in}}%
\pgfpathlineto{\pgfqpoint{2.895455in}{0.774439in}}%
\pgfpathlineto{\pgfqpoint{2.900722in}{0.768037in}}%
\pgfpathlineto{\pgfqpoint{2.911257in}{0.766736in}}%
\pgfpathlineto{\pgfqpoint{2.916525in}{0.750860in}}%
\pgfpathlineto{\pgfqpoint{2.921793in}{0.716247in}}%
\pgfpathlineto{\pgfqpoint{2.937596in}{0.582845in}}%
\pgfpathlineto{\pgfqpoint{2.942863in}{0.555235in}}%
\pgfpathlineto{\pgfqpoint{2.948131in}{0.538345in}}%
\pgfpathlineto{\pgfqpoint{2.953398in}{0.529184in}}%
\pgfpathlineto{\pgfqpoint{2.963934in}{0.519340in}}%
\pgfpathlineto{\pgfqpoint{2.974469in}{0.513716in}}%
\pgfpathlineto{\pgfqpoint{2.979737in}{0.513194in}}%
\pgfpathlineto{\pgfqpoint{2.985004in}{0.515194in}}%
\pgfpathlineto{\pgfqpoint{3.006075in}{0.536477in}}%
\pgfpathlineto{\pgfqpoint{3.011342in}{0.539223in}}%
\pgfpathlineto{\pgfqpoint{3.016610in}{0.537284in}}%
\pgfpathlineto{\pgfqpoint{3.032413in}{0.519520in}}%
\pgfpathlineto{\pgfqpoint{3.048216in}{0.513197in}}%
\pgfpathlineto{\pgfqpoint{3.053483in}{0.513423in}}%
\pgfpathlineto{\pgfqpoint{3.058751in}{0.516559in}}%
\pgfpathlineto{\pgfqpoint{3.079822in}{0.538753in}}%
\pgfpathlineto{\pgfqpoint{3.085089in}{0.540852in}}%
\pgfpathlineto{\pgfqpoint{3.090357in}{0.537608in}}%
\pgfpathlineto{\pgfqpoint{3.100892in}{0.523482in}}%
\pgfpathlineto{\pgfqpoint{3.111427in}{0.515911in}}%
\pgfpathlineto{\pgfqpoint{3.121963in}{0.511581in}}%
\pgfpathlineto{\pgfqpoint{3.127230in}{0.512491in}}%
\pgfpathlineto{\pgfqpoint{3.132498in}{0.516930in}}%
\pgfpathlineto{\pgfqpoint{3.148301in}{0.534163in}}%
\pgfpathlineto{\pgfqpoint{3.153568in}{0.538006in}}%
\pgfpathlineto{\pgfqpoint{3.158836in}{0.538521in}}%
\pgfpathlineto{\pgfqpoint{3.179907in}{0.518267in}}%
\pgfpathlineto{\pgfqpoint{3.190442in}{0.514396in}}%
\pgfpathlineto{\pgfqpoint{3.195709in}{0.514049in}}%
\pgfpathlineto{\pgfqpoint{3.200977in}{0.515773in}}%
\pgfpathlineto{\pgfqpoint{3.227315in}{0.539712in}}%
\pgfpathlineto{\pgfqpoint{3.232583in}{0.538759in}}%
\pgfpathlineto{\pgfqpoint{3.248386in}{0.520268in}}%
\pgfpathlineto{\pgfqpoint{3.264189in}{0.511864in}}%
\pgfpathlineto{\pgfqpoint{3.269456in}{0.511822in}}%
\pgfpathlineto{\pgfqpoint{3.274724in}{0.514858in}}%
\pgfpathlineto{\pgfqpoint{3.290527in}{0.531654in}}%
\pgfpathlineto{\pgfqpoint{3.295794in}{0.536284in}}%
\pgfpathlineto{\pgfqpoint{3.301062in}{0.538129in}}%
\pgfpathlineto{\pgfqpoint{3.306330in}{0.534725in}}%
\pgfpathlineto{\pgfqpoint{3.316865in}{0.521381in}}%
\pgfpathlineto{\pgfqpoint{3.327400in}{0.515554in}}%
\pgfpathlineto{\pgfqpoint{3.337935in}{0.512394in}}%
\pgfpathlineto{\pgfqpoint{3.343203in}{0.513198in}}%
\pgfpathlineto{\pgfqpoint{3.348471in}{0.517112in}}%
\pgfpathlineto{\pgfqpoint{3.364274in}{0.533871in}}%
\pgfpathlineto{\pgfqpoint{3.369541in}{0.538305in}}%
\pgfpathlineto{\pgfqpoint{3.374809in}{0.539400in}}%
\pgfpathlineto{\pgfqpoint{3.380076in}{0.534843in}}%
\pgfpathlineto{\pgfqpoint{3.390612in}{0.521468in}}%
\pgfpathlineto{\pgfqpoint{3.406415in}{0.512316in}}%
\pgfpathlineto{\pgfqpoint{3.411682in}{0.511633in}}%
\pgfpathlineto{\pgfqpoint{3.416950in}{0.513257in}}%
\pgfpathlineto{\pgfqpoint{3.443288in}{0.537127in}}%
\pgfpathlineto{\pgfqpoint{3.448556in}{0.535889in}}%
\pgfpathlineto{\pgfqpoint{3.459091in}{0.522930in}}%
\pgfpathlineto{\pgfqpoint{3.459091in}{0.522930in}}%
\pgfusepath{stroke}%
\end{pgfscope}%
\begin{pgfscope}%
\pgfsetrectcap%
\pgfsetmiterjoin%
\pgfsetlinewidth{0.803000pt}%
\definecolor{currentstroke}{rgb}{0.000000,0.000000,0.000000}%
\pgfsetstrokecolor{currentstroke}%
\pgfsetdash{}{0pt}%
\pgfpathmoveto{\pgfqpoint{0.500000in}{0.375000in}}%
\pgfpathlineto{\pgfqpoint{0.500000in}{2.640000in}}%
\pgfusepath{stroke}%
\end{pgfscope}%
\begin{pgfscope}%
\pgfsetrectcap%
\pgfsetmiterjoin%
\pgfsetlinewidth{0.803000pt}%
\definecolor{currentstroke}{rgb}{0.000000,0.000000,0.000000}%
\pgfsetstrokecolor{currentstroke}%
\pgfsetdash{}{0pt}%
\pgfpathmoveto{\pgfqpoint{3.600000in}{0.375000in}}%
\pgfpathlineto{\pgfqpoint{3.600000in}{2.640000in}}%
\pgfusepath{stroke}%
\end{pgfscope}%
\begin{pgfscope}%
\pgfsetrectcap%
\pgfsetmiterjoin%
\pgfsetlinewidth{0.803000pt}%
\definecolor{currentstroke}{rgb}{0.000000,0.000000,0.000000}%
\pgfsetstrokecolor{currentstroke}%
\pgfsetdash{}{0pt}%
\pgfpathmoveto{\pgfqpoint{0.500000in}{0.375000in}}%
\pgfpathlineto{\pgfqpoint{3.600000in}{0.375000in}}%
\pgfusepath{stroke}%
\end{pgfscope}%
\begin{pgfscope}%
\pgfsetrectcap%
\pgfsetmiterjoin%
\pgfsetlinewidth{0.803000pt}%
\definecolor{currentstroke}{rgb}{0.000000,0.000000,0.000000}%
\pgfsetstrokecolor{currentstroke}%
\pgfsetdash{}{0pt}%
\pgfpathmoveto{\pgfqpoint{0.500000in}{2.640000in}}%
\pgfpathlineto{\pgfqpoint{3.600000in}{2.640000in}}%
\pgfusepath{stroke}%
\end{pgfscope}%
\begin{pgfscope}%
\pgfsetbuttcap%
\pgfsetmiterjoin%
\definecolor{currentfill}{rgb}{1.000000,1.000000,1.000000}%
\pgfsetfillcolor{currentfill}%
\pgfsetfillopacity{0.800000}%
\pgfsetlinewidth{1.003750pt}%
\definecolor{currentstroke}{rgb}{0.800000,0.800000,0.800000}%
\pgfsetstrokecolor{currentstroke}%
\pgfsetstrokeopacity{0.800000}%
\pgfsetdash{}{0pt}%
\pgfpathmoveto{\pgfqpoint{2.799845in}{1.196158in}}%
\pgfpathlineto{\pgfqpoint{3.502778in}{1.196158in}}%
\pgfpathquadraticcurveto{\pgfqpoint{3.530556in}{1.196158in}}{\pgfqpoint{3.530556in}{1.223935in}}%
\pgfpathlineto{\pgfqpoint{3.530556in}{1.791065in}}%
\pgfpathquadraticcurveto{\pgfqpoint{3.530556in}{1.818842in}}{\pgfqpoint{3.502778in}{1.818842in}}%
\pgfpathlineto{\pgfqpoint{2.799845in}{1.818842in}}%
\pgfpathquadraticcurveto{\pgfqpoint{2.772067in}{1.818842in}}{\pgfqpoint{2.772067in}{1.791065in}}%
\pgfpathlineto{\pgfqpoint{2.772067in}{1.223935in}}%
\pgfpathquadraticcurveto{\pgfqpoint{2.772067in}{1.196158in}}{\pgfqpoint{2.799845in}{1.196158in}}%
\pgfpathlineto{\pgfqpoint{2.799845in}{1.196158in}}%
\pgfpathclose%
\pgfusepath{stroke,fill}%
\end{pgfscope}%
\begin{pgfscope}%
\pgfsetrectcap%
\pgfsetroundjoin%
\pgfsetlinewidth{1.505625pt}%
\definecolor{currentstroke}{rgb}{0.000000,0.000000,1.000000}%
\pgfsetstrokecolor{currentstroke}%
\pgfsetdash{}{0pt}%
\pgfpathmoveto{\pgfqpoint{2.827622in}{1.714676in}}%
\pgfpathlineto{\pgfqpoint{2.966511in}{1.714676in}}%
\pgfpathlineto{\pgfqpoint{3.105400in}{1.714676in}}%
\pgfusepath{stroke}%
\end{pgfscope}%
\begin{pgfscope}%
\definecolor{textcolor}{rgb}{0.000000,0.000000,0.000000}%
\pgfsetstrokecolor{textcolor}%
\pgfsetfillcolor{textcolor}%
\pgftext[x=3.216511in,y=1.666065in,left,base]{\color{textcolor}\rmfamily\fontsize{10.000000}{12.000000}\selectfont max}%
\end{pgfscope}%
\begin{pgfscope}%
\pgfsetrectcap%
\pgfsetroundjoin%
\pgfsetlinewidth{1.505625pt}%
\definecolor{currentstroke}{rgb}{1.000000,0.000000,0.000000}%
\pgfsetstrokecolor{currentstroke}%
\pgfsetdash{}{0pt}%
\pgfpathmoveto{\pgfqpoint{2.827622in}{1.521003in}}%
\pgfpathlineto{\pgfqpoint{2.966511in}{1.521003in}}%
\pgfpathlineto{\pgfqpoint{3.105400in}{1.521003in}}%
\pgfusepath{stroke}%
\end{pgfscope}%
\begin{pgfscope}%
\definecolor{textcolor}{rgb}{0.000000,0.000000,0.000000}%
\pgfsetstrokecolor{textcolor}%
\pgfsetfillcolor{textcolor}%
\pgftext[x=3.216511in,y=1.472392in,left,base]{\color{textcolor}\rmfamily\fontsize{10.000000}{12.000000}\selectfont \(\displaystyle \mu\)}%
\end{pgfscope}%
\begin{pgfscope}%
\pgfsetrectcap%
\pgfsetroundjoin%
\pgfsetlinewidth{1.505625pt}%
\definecolor{currentstroke}{rgb}{0.000000,0.500000,0.000000}%
\pgfsetstrokecolor{currentstroke}%
\pgfsetdash{}{0pt}%
\pgfpathmoveto{\pgfqpoint{2.827622in}{1.327330in}}%
\pgfpathlineto{\pgfqpoint{2.966511in}{1.327330in}}%
\pgfpathlineto{\pgfqpoint{3.105400in}{1.327330in}}%
\pgfusepath{stroke}%
\end{pgfscope}%
\begin{pgfscope}%
\definecolor{textcolor}{rgb}{0.000000,0.000000,0.000000}%
\pgfsetstrokecolor{textcolor}%
\pgfsetfillcolor{textcolor}%
\pgftext[x=3.216511in,y=1.278719in,left,base]{\color{textcolor}\rmfamily\fontsize{10.000000}{12.000000}\selectfont \(\displaystyle \sigma\)}%
\end{pgfscope}%
\end{pgfpicture}%
\makeatother%
\endgroup%
}
%         \caption{Force Matrix Profile}
%         \label{fig:mp_hist_standard_force}
%     \end{minipage}
%     \begin{minipage}[t]{0.5\linewidth}
%         %%\centering
%         \resizebox{\linewidth}{!}{%% Creator: Matplotlib, PGF backend
%%
%% To include the figure in your LaTeX document, write
%%   \input{<filename>.pgf}
%%
%% Make sure the required packages are loaded in your preamble
%%   \usepackage{pgf}
%%
%% Also ensure that all the required font packages are loaded; for instance,
%% the lmodern package is sometimes necessary when using math font.
%%   \usepackage{lmodern}
%%
%% Figures using additional raster images can only be included by \input if
%% they are in the same directory as the main LaTeX file. For loading figures
%% from other directories you can use the `import` package
%%   \usepackage{import}
%%
%% and then include the figures with
%%   \import{<path to file>}{<filename>.pgf}
%%
%% Matplotlib used the following preamble
%%
\begingroup%
\makeatletter%
\begin{pgfpicture}%
\pgfpathrectangle{\pgfpointorigin}{\pgfqpoint{4.000000in}{3.000000in}}%
\pgfusepath{use as bounding box, clip}%
\begin{pgfscope}%
\pgfsetbuttcap%
\pgfsetmiterjoin%
\pgfsetlinewidth{0.000000pt}%
\definecolor{currentstroke}{rgb}{1.000000,1.000000,1.000000}%
\pgfsetstrokecolor{currentstroke}%
\pgfsetstrokeopacity{0.000000}%
\pgfsetdash{}{0pt}%
\pgfpathmoveto{\pgfqpoint{0.000000in}{0.000000in}}%
\pgfpathlineto{\pgfqpoint{4.000000in}{0.000000in}}%
\pgfpathlineto{\pgfqpoint{4.000000in}{3.000000in}}%
\pgfpathlineto{\pgfqpoint{0.000000in}{3.000000in}}%
\pgfpathlineto{\pgfqpoint{0.000000in}{0.000000in}}%
\pgfpathclose%
\pgfusepath{}%
\end{pgfscope}%
\begin{pgfscope}%
\pgfsetbuttcap%
\pgfsetmiterjoin%
\definecolor{currentfill}{rgb}{1.000000,1.000000,1.000000}%
\pgfsetfillcolor{currentfill}%
\pgfsetlinewidth{0.000000pt}%
\definecolor{currentstroke}{rgb}{0.000000,0.000000,0.000000}%
\pgfsetstrokecolor{currentstroke}%
\pgfsetstrokeopacity{0.000000}%
\pgfsetdash{}{0pt}%
\pgfpathmoveto{\pgfqpoint{0.500000in}{0.375000in}}%
\pgfpathlineto{\pgfqpoint{3.600000in}{0.375000in}}%
\pgfpathlineto{\pgfqpoint{3.600000in}{2.640000in}}%
\pgfpathlineto{\pgfqpoint{0.500000in}{2.640000in}}%
\pgfpathlineto{\pgfqpoint{0.500000in}{0.375000in}}%
\pgfpathclose%
\pgfusepath{fill}%
\end{pgfscope}%
\begin{pgfscope}%
\pgfsetbuttcap%
\pgfsetroundjoin%
\definecolor{currentfill}{rgb}{0.000000,0.000000,0.000000}%
\pgfsetfillcolor{currentfill}%
\pgfsetlinewidth{0.803000pt}%
\definecolor{currentstroke}{rgb}{0.000000,0.000000,0.000000}%
\pgfsetstrokecolor{currentstroke}%
\pgfsetdash{}{0pt}%
\pgfsys@defobject{currentmarker}{\pgfqpoint{0.000000in}{-0.048611in}}{\pgfqpoint{0.000000in}{0.000000in}}{%
\pgfpathmoveto{\pgfqpoint{0.000000in}{0.000000in}}%
\pgfpathlineto{\pgfqpoint{0.000000in}{-0.048611in}}%
\pgfusepath{stroke,fill}%
}%
\begin{pgfscope}%
\pgfsys@transformshift{0.640909in}{0.375000in}%
\pgfsys@useobject{currentmarker}{}%
\end{pgfscope}%
\end{pgfscope}%
\begin{pgfscope}%
\definecolor{textcolor}{rgb}{0.000000,0.000000,0.000000}%
\pgfsetstrokecolor{textcolor}%
\pgfsetfillcolor{textcolor}%
\pgftext[x=0.640909in,y=0.277778in,,top]{\color{textcolor}\rmfamily\fontsize{10.000000}{12.000000}\selectfont \(\displaystyle {0}\)}%
\end{pgfscope}%
\begin{pgfscope}%
\pgfsetbuttcap%
\pgfsetroundjoin%
\definecolor{currentfill}{rgb}{0.000000,0.000000,0.000000}%
\pgfsetfillcolor{currentfill}%
\pgfsetlinewidth{0.803000pt}%
\definecolor{currentstroke}{rgb}{0.000000,0.000000,0.000000}%
\pgfsetstrokecolor{currentstroke}%
\pgfsetdash{}{0pt}%
\pgfsys@defobject{currentmarker}{\pgfqpoint{0.000000in}{-0.048611in}}{\pgfqpoint{0.000000in}{0.000000in}}{%
\pgfpathmoveto{\pgfqpoint{0.000000in}{0.000000in}}%
\pgfpathlineto{\pgfqpoint{0.000000in}{-0.048611in}}%
\pgfusepath{stroke,fill}%
}%
\begin{pgfscope}%
\pgfsys@transformshift{1.421569in}{0.375000in}%
\pgfsys@useobject{currentmarker}{}%
\end{pgfscope}%
\end{pgfscope}%
\begin{pgfscope}%
\definecolor{textcolor}{rgb}{0.000000,0.000000,0.000000}%
\pgfsetstrokecolor{textcolor}%
\pgfsetfillcolor{textcolor}%
\pgftext[x=1.421569in,y=0.277778in,,top]{\color{textcolor}\rmfamily\fontsize{10.000000}{12.000000}\selectfont \(\displaystyle {100}\)}%
\end{pgfscope}%
\begin{pgfscope}%
\pgfsetbuttcap%
\pgfsetroundjoin%
\definecolor{currentfill}{rgb}{0.000000,0.000000,0.000000}%
\pgfsetfillcolor{currentfill}%
\pgfsetlinewidth{0.803000pt}%
\definecolor{currentstroke}{rgb}{0.000000,0.000000,0.000000}%
\pgfsetstrokecolor{currentstroke}%
\pgfsetdash{}{0pt}%
\pgfsys@defobject{currentmarker}{\pgfqpoint{0.000000in}{-0.048611in}}{\pgfqpoint{0.000000in}{0.000000in}}{%
\pgfpathmoveto{\pgfqpoint{0.000000in}{0.000000in}}%
\pgfpathlineto{\pgfqpoint{0.000000in}{-0.048611in}}%
\pgfusepath{stroke,fill}%
}%
\begin{pgfscope}%
\pgfsys@transformshift{2.202229in}{0.375000in}%
\pgfsys@useobject{currentmarker}{}%
\end{pgfscope}%
\end{pgfscope}%
\begin{pgfscope}%
\definecolor{textcolor}{rgb}{0.000000,0.000000,0.000000}%
\pgfsetstrokecolor{textcolor}%
\pgfsetfillcolor{textcolor}%
\pgftext[x=2.202229in,y=0.277778in,,top]{\color{textcolor}\rmfamily\fontsize{10.000000}{12.000000}\selectfont \(\displaystyle {200}\)}%
\end{pgfscope}%
\begin{pgfscope}%
\pgfsetbuttcap%
\pgfsetroundjoin%
\definecolor{currentfill}{rgb}{0.000000,0.000000,0.000000}%
\pgfsetfillcolor{currentfill}%
\pgfsetlinewidth{0.803000pt}%
\definecolor{currentstroke}{rgb}{0.000000,0.000000,0.000000}%
\pgfsetstrokecolor{currentstroke}%
\pgfsetdash{}{0pt}%
\pgfsys@defobject{currentmarker}{\pgfqpoint{0.000000in}{-0.048611in}}{\pgfqpoint{0.000000in}{0.000000in}}{%
\pgfpathmoveto{\pgfqpoint{0.000000in}{0.000000in}}%
\pgfpathlineto{\pgfqpoint{0.000000in}{-0.048611in}}%
\pgfusepath{stroke,fill}%
}%
\begin{pgfscope}%
\pgfsys@transformshift{2.982888in}{0.375000in}%
\pgfsys@useobject{currentmarker}{}%
\end{pgfscope}%
\end{pgfscope}%
\begin{pgfscope}%
\definecolor{textcolor}{rgb}{0.000000,0.000000,0.000000}%
\pgfsetstrokecolor{textcolor}%
\pgfsetfillcolor{textcolor}%
\pgftext[x=2.982888in,y=0.277778in,,top]{\color{textcolor}\rmfamily\fontsize{10.000000}{12.000000}\selectfont \(\displaystyle {300}\)}%
\end{pgfscope}%
\begin{pgfscope}%
\definecolor{textcolor}{rgb}{0.000000,0.000000,0.000000}%
\pgfsetstrokecolor{textcolor}%
\pgfsetfillcolor{textcolor}%
\pgftext[x=2.050000in,y=0.098766in,,top]{\color{textcolor}\rmfamily\fontsize{10.000000}{12.000000}\selectfont time}%
\end{pgfscope}%
\begin{pgfscope}%
\pgfsetbuttcap%
\pgfsetroundjoin%
\definecolor{currentfill}{rgb}{0.000000,0.000000,0.000000}%
\pgfsetfillcolor{currentfill}%
\pgfsetlinewidth{0.803000pt}%
\definecolor{currentstroke}{rgb}{0.000000,0.000000,0.000000}%
\pgfsetstrokecolor{currentstroke}%
\pgfsetdash{}{0pt}%
\pgfsys@defobject{currentmarker}{\pgfqpoint{-0.048611in}{0.000000in}}{\pgfqpoint{-0.000000in}{0.000000in}}{%
\pgfpathmoveto{\pgfqpoint{-0.000000in}{0.000000in}}%
\pgfpathlineto{\pgfqpoint{-0.048611in}{0.000000in}}%
\pgfusepath{stroke,fill}%
}%
\begin{pgfscope}%
\pgfsys@transformshift{0.500000in}{0.477951in}%
\pgfsys@useobject{currentmarker}{}%
\end{pgfscope}%
\end{pgfscope}%
\begin{pgfscope}%
\definecolor{textcolor}{rgb}{0.000000,0.000000,0.000000}%
\pgfsetstrokecolor{textcolor}%
\pgfsetfillcolor{textcolor}%
\pgftext[x=0.333333in, y=0.429726in, left, base]{\color{textcolor}\rmfamily\fontsize{10.000000}{12.000000}\selectfont \(\displaystyle {0}\)}%
\end{pgfscope}%
\begin{pgfscope}%
\pgfsetbuttcap%
\pgfsetroundjoin%
\definecolor{currentfill}{rgb}{0.000000,0.000000,0.000000}%
\pgfsetfillcolor{currentfill}%
\pgfsetlinewidth{0.803000pt}%
\definecolor{currentstroke}{rgb}{0.000000,0.000000,0.000000}%
\pgfsetstrokecolor{currentstroke}%
\pgfsetdash{}{0pt}%
\pgfsys@defobject{currentmarker}{\pgfqpoint{-0.048611in}{0.000000in}}{\pgfqpoint{-0.000000in}{0.000000in}}{%
\pgfpathmoveto{\pgfqpoint{-0.000000in}{0.000000in}}%
\pgfpathlineto{\pgfqpoint{-0.048611in}{0.000000in}}%
\pgfusepath{stroke,fill}%
}%
\begin{pgfscope}%
\pgfsys@transformshift{0.500000in}{1.037083in}%
\pgfsys@useobject{currentmarker}{}%
\end{pgfscope}%
\end{pgfscope}%
\begin{pgfscope}%
\definecolor{textcolor}{rgb}{0.000000,0.000000,0.000000}%
\pgfsetstrokecolor{textcolor}%
\pgfsetfillcolor{textcolor}%
\pgftext[x=0.124999in, y=0.988858in, left, base]{\color{textcolor}\rmfamily\fontsize{10.000000}{12.000000}\selectfont \(\displaystyle {5000}\)}%
\end{pgfscope}%
\begin{pgfscope}%
\pgfsetbuttcap%
\pgfsetroundjoin%
\definecolor{currentfill}{rgb}{0.000000,0.000000,0.000000}%
\pgfsetfillcolor{currentfill}%
\pgfsetlinewidth{0.803000pt}%
\definecolor{currentstroke}{rgb}{0.000000,0.000000,0.000000}%
\pgfsetstrokecolor{currentstroke}%
\pgfsetdash{}{0pt}%
\pgfsys@defobject{currentmarker}{\pgfqpoint{-0.048611in}{0.000000in}}{\pgfqpoint{-0.000000in}{0.000000in}}{%
\pgfpathmoveto{\pgfqpoint{-0.000000in}{0.000000in}}%
\pgfpathlineto{\pgfqpoint{-0.048611in}{0.000000in}}%
\pgfusepath{stroke,fill}%
}%
\begin{pgfscope}%
\pgfsys@transformshift{0.500000in}{1.596215in}%
\pgfsys@useobject{currentmarker}{}%
\end{pgfscope}%
\end{pgfscope}%
\begin{pgfscope}%
\definecolor{textcolor}{rgb}{0.000000,0.000000,0.000000}%
\pgfsetstrokecolor{textcolor}%
\pgfsetfillcolor{textcolor}%
\pgftext[x=0.055554in, y=1.547989in, left, base]{\color{textcolor}\rmfamily\fontsize{10.000000}{12.000000}\selectfont \(\displaystyle {10000}\)}%
\end{pgfscope}%
\begin{pgfscope}%
\pgfsetbuttcap%
\pgfsetroundjoin%
\definecolor{currentfill}{rgb}{0.000000,0.000000,0.000000}%
\pgfsetfillcolor{currentfill}%
\pgfsetlinewidth{0.803000pt}%
\definecolor{currentstroke}{rgb}{0.000000,0.000000,0.000000}%
\pgfsetstrokecolor{currentstroke}%
\pgfsetdash{}{0pt}%
\pgfsys@defobject{currentmarker}{\pgfqpoint{-0.048611in}{0.000000in}}{\pgfqpoint{-0.000000in}{0.000000in}}{%
\pgfpathmoveto{\pgfqpoint{-0.000000in}{0.000000in}}%
\pgfpathlineto{\pgfqpoint{-0.048611in}{0.000000in}}%
\pgfusepath{stroke,fill}%
}%
\begin{pgfscope}%
\pgfsys@transformshift{0.500000in}{2.155346in}%
\pgfsys@useobject{currentmarker}{}%
\end{pgfscope}%
\end{pgfscope}%
\begin{pgfscope}%
\definecolor{textcolor}{rgb}{0.000000,0.000000,0.000000}%
\pgfsetstrokecolor{textcolor}%
\pgfsetfillcolor{textcolor}%
\pgftext[x=0.055554in, y=2.107121in, left, base]{\color{textcolor}\rmfamily\fontsize{10.000000}{12.000000}\selectfont \(\displaystyle {15000}\)}%
\end{pgfscope}%
\begin{pgfscope}%
\pgfpathrectangle{\pgfqpoint{0.500000in}{0.375000in}}{\pgfqpoint{3.100000in}{2.265000in}}%
\pgfusepath{clip}%
\pgfsetrectcap%
\pgfsetroundjoin%
\pgfsetlinewidth{1.505625pt}%
\definecolor{currentstroke}{rgb}{0.000000,0.000000,1.000000}%
\pgfsetstrokecolor{currentstroke}%
\pgfsetdash{}{0pt}%
\pgfpathmoveto{\pgfqpoint{0.640909in}{0.477965in}}%
\pgfpathlineto{\pgfqpoint{0.726782in}{0.478923in}}%
\pgfpathlineto{\pgfqpoint{0.742395in}{0.482233in}}%
\pgfpathlineto{\pgfqpoint{0.758008in}{0.490305in}}%
\pgfpathlineto{\pgfqpoint{0.773621in}{0.503055in}}%
\pgfpathlineto{\pgfqpoint{0.797041in}{0.527105in}}%
\pgfpathlineto{\pgfqpoint{0.820461in}{0.555289in}}%
\pgfpathlineto{\pgfqpoint{0.836074in}{0.578904in}}%
\pgfpathlineto{\pgfqpoint{0.851687in}{0.611831in}}%
\pgfpathlineto{\pgfqpoint{0.867300in}{0.659947in}}%
\pgfpathlineto{\pgfqpoint{0.882914in}{0.728537in}}%
\pgfpathlineto{\pgfqpoint{0.898527in}{0.826495in}}%
\pgfpathlineto{\pgfqpoint{0.914140in}{0.970917in}}%
\pgfpathlineto{\pgfqpoint{0.937560in}{1.205056in}}%
\pgfpathlineto{\pgfqpoint{0.945366in}{1.262822in}}%
\pgfpathlineto{\pgfqpoint{0.953173in}{1.305851in}}%
\pgfpathlineto{\pgfqpoint{0.960980in}{1.336291in}}%
\pgfpathlineto{\pgfqpoint{0.968786in}{1.354115in}}%
\pgfpathlineto{\pgfqpoint{0.976593in}{1.360553in}}%
\pgfpathlineto{\pgfqpoint{1.015626in}{1.359831in}}%
\pgfpathlineto{\pgfqpoint{1.023432in}{1.353880in}}%
\pgfpathlineto{\pgfqpoint{1.039046in}{1.336291in}}%
\pgfpathlineto{\pgfqpoint{1.070272in}{1.336291in}}%
\pgfpathlineto{\pgfqpoint{1.078079in}{1.446754in}}%
\pgfpathlineto{\pgfqpoint{1.101498in}{2.078804in}}%
\pgfpathlineto{\pgfqpoint{1.109305in}{2.231759in}}%
\pgfpathlineto{\pgfqpoint{1.117112in}{2.342989in}}%
\pgfpathlineto{\pgfqpoint{1.124918in}{2.414163in}}%
\pgfpathlineto{\pgfqpoint{1.132725in}{2.451341in}}%
\pgfpathlineto{\pgfqpoint{1.140531in}{2.464172in}}%
\pgfpathlineto{\pgfqpoint{1.156145in}{2.465144in}}%
\pgfpathlineto{\pgfqpoint{1.163951in}{2.474261in}}%
\pgfpathlineto{\pgfqpoint{1.187371in}{2.537045in}}%
\pgfpathlineto{\pgfqpoint{1.218597in}{2.537045in}}%
\pgfpathlineto{\pgfqpoint{1.226404in}{2.533130in}}%
\pgfpathlineto{\pgfqpoint{1.242017in}{2.474261in}}%
\pgfpathlineto{\pgfqpoint{1.281050in}{2.474261in}}%
\pgfpathlineto{\pgfqpoint{1.288857in}{2.443625in}}%
\pgfpathlineto{\pgfqpoint{1.304470in}{2.276714in}}%
\pgfpathlineto{\pgfqpoint{1.312277in}{2.187423in}}%
\pgfpathlineto{\pgfqpoint{1.320083in}{2.118394in}}%
\pgfpathlineto{\pgfqpoint{1.327890in}{2.078232in}}%
\pgfpathlineto{\pgfqpoint{1.335696in}{2.063389in}}%
\pgfpathlineto{\pgfqpoint{1.343503in}{2.062822in}}%
\pgfpathlineto{\pgfqpoint{1.351309in}{2.060017in}}%
\pgfpathlineto{\pgfqpoint{1.359116in}{2.040988in}}%
\pgfpathlineto{\pgfqpoint{1.366923in}{1.999966in}}%
\pgfpathlineto{\pgfqpoint{1.382536in}{1.868499in}}%
\pgfpathlineto{\pgfqpoint{1.390342in}{1.798980in}}%
\pgfpathlineto{\pgfqpoint{1.398149in}{1.839965in}}%
\pgfpathlineto{\pgfqpoint{1.413762in}{1.958236in}}%
\pgfpathlineto{\pgfqpoint{1.421569in}{1.995507in}}%
\pgfpathlineto{\pgfqpoint{1.429375in}{1.998887in}}%
\pgfpathlineto{\pgfqpoint{1.460602in}{1.998887in}}%
\pgfpathlineto{\pgfqpoint{1.468408in}{1.964744in}}%
\pgfpathlineto{\pgfqpoint{1.476215in}{1.898026in}}%
\pgfpathlineto{\pgfqpoint{1.515248in}{1.898026in}}%
\pgfpathlineto{\pgfqpoint{1.530861in}{1.811842in}}%
\pgfpathlineto{\pgfqpoint{1.538668in}{1.805082in}}%
\pgfpathlineto{\pgfqpoint{1.546474in}{1.801268in}}%
\pgfpathlineto{\pgfqpoint{1.554281in}{1.754757in}}%
\pgfpathlineto{\pgfqpoint{1.562088in}{1.726010in}}%
\pgfpathlineto{\pgfqpoint{1.577701in}{1.726010in}}%
\pgfpathlineto{\pgfqpoint{1.585507in}{1.672994in}}%
\pgfpathlineto{\pgfqpoint{1.593314in}{1.672994in}}%
\pgfpathlineto{\pgfqpoint{1.601121in}{1.636290in}}%
\pgfpathlineto{\pgfqpoint{1.608927in}{1.573525in}}%
\pgfpathlineto{\pgfqpoint{1.616734in}{1.572945in}}%
\pgfpathlineto{\pgfqpoint{1.624540in}{1.481498in}}%
\pgfpathlineto{\pgfqpoint{1.663573in}{1.481498in}}%
\pgfpathlineto{\pgfqpoint{1.671380in}{1.423022in}}%
\pgfpathlineto{\pgfqpoint{1.679187in}{1.412491in}}%
\pgfpathlineto{\pgfqpoint{1.686993in}{1.361317in}}%
\pgfpathlineto{\pgfqpoint{1.702606in}{1.196461in}}%
\pgfpathlineto{\pgfqpoint{1.710413in}{1.135360in}}%
\pgfpathlineto{\pgfqpoint{1.718220in}{1.098592in}}%
\pgfpathlineto{\pgfqpoint{1.726026in}{1.089047in}}%
\pgfpathlineto{\pgfqpoint{1.749446in}{1.089047in}}%
\pgfpathlineto{\pgfqpoint{1.757253in}{1.083427in}}%
\pgfpathlineto{\pgfqpoint{1.765059in}{1.047994in}}%
\pgfpathlineto{\pgfqpoint{1.772866in}{1.036981in}}%
\pgfpathlineto{\pgfqpoint{1.804092in}{1.545886in}}%
\pgfpathlineto{\pgfqpoint{1.811899in}{1.635965in}}%
\pgfpathlineto{\pgfqpoint{1.819705in}{1.700761in}}%
\pgfpathlineto{\pgfqpoint{1.827512in}{1.737741in}}%
\pgfpathlineto{\pgfqpoint{1.835319in}{1.754372in}}%
\pgfpathlineto{\pgfqpoint{1.843125in}{1.759784in}}%
\pgfpathlineto{\pgfqpoint{1.866545in}{1.760317in}}%
\pgfpathlineto{\pgfqpoint{1.889965in}{1.759095in}}%
\pgfpathlineto{\pgfqpoint{1.897771in}{1.756673in}}%
\pgfpathlineto{\pgfqpoint{1.905578in}{1.750667in}}%
\pgfpathlineto{\pgfqpoint{1.952418in}{1.750667in}}%
\pgfpathlineto{\pgfqpoint{1.960224in}{1.732777in}}%
\pgfpathlineto{\pgfqpoint{1.968031in}{1.694420in}}%
\pgfpathlineto{\pgfqpoint{1.975837in}{1.626273in}}%
\pgfpathlineto{\pgfqpoint{1.983644in}{1.666112in}}%
\pgfpathlineto{\pgfqpoint{1.991451in}{1.868239in}}%
\pgfpathlineto{\pgfqpoint{1.999257in}{2.023795in}}%
\pgfpathlineto{\pgfqpoint{2.007064in}{2.132082in}}%
\pgfpathlineto{\pgfqpoint{2.014870in}{2.196999in}}%
\pgfpathlineto{\pgfqpoint{2.022677in}{2.227739in}}%
\pgfpathlineto{\pgfqpoint{2.030484in}{2.239069in}}%
\pgfpathlineto{\pgfqpoint{2.038290in}{2.241071in}}%
\pgfpathlineto{\pgfqpoint{2.069516in}{2.241071in}}%
\pgfpathlineto{\pgfqpoint{2.077323in}{2.239052in}}%
\pgfpathlineto{\pgfqpoint{2.085130in}{2.234717in}}%
\pgfpathlineto{\pgfqpoint{2.092936in}{2.225668in}}%
\pgfpathlineto{\pgfqpoint{2.139776in}{2.225668in}}%
\pgfpathlineto{\pgfqpoint{2.147582in}{2.206402in}}%
\pgfpathlineto{\pgfqpoint{2.155389in}{2.164386in}}%
\pgfpathlineto{\pgfqpoint{2.163196in}{2.090913in}}%
\pgfpathlineto{\pgfqpoint{2.171002in}{1.979815in}}%
\pgfpathlineto{\pgfqpoint{2.194422in}{1.539874in}}%
\pgfpathlineto{\pgfqpoint{2.202229in}{1.441501in}}%
\pgfpathlineto{\pgfqpoint{2.210035in}{1.384091in}}%
\pgfpathlineto{\pgfqpoint{2.217842in}{1.364782in}}%
\pgfpathlineto{\pgfqpoint{2.241262in}{1.364782in}}%
\pgfpathlineto{\pgfqpoint{2.249068in}{1.358648in}}%
\pgfpathlineto{\pgfqpoint{2.256875in}{1.355512in}}%
\pgfpathlineto{\pgfqpoint{2.295908in}{1.354909in}}%
\pgfpathlineto{\pgfqpoint{2.303714in}{1.334300in}}%
\pgfpathlineto{\pgfqpoint{2.311521in}{1.284717in}}%
\pgfpathlineto{\pgfqpoint{2.327134in}{1.153321in}}%
\pgfpathlineto{\pgfqpoint{2.334941in}{1.111421in}}%
\pgfpathlineto{\pgfqpoint{2.350554in}{1.111421in}}%
\pgfpathlineto{\pgfqpoint{2.358361in}{1.106275in}}%
\pgfpathlineto{\pgfqpoint{2.366167in}{1.087313in}}%
\pgfpathlineto{\pgfqpoint{2.381780in}{1.087313in}}%
\pgfpathlineto{\pgfqpoint{2.389587in}{1.079392in}}%
\pgfpathlineto{\pgfqpoint{2.397394in}{1.078148in}}%
\pgfpathlineto{\pgfqpoint{2.405200in}{1.062185in}}%
\pgfpathlineto{\pgfqpoint{2.413007in}{0.999494in}}%
\pgfpathlineto{\pgfqpoint{2.420813in}{0.970689in}}%
\pgfpathlineto{\pgfqpoint{2.428620in}{0.968828in}}%
\pgfpathlineto{\pgfqpoint{2.436427in}{0.968828in}}%
\pgfpathlineto{\pgfqpoint{2.444233in}{0.965134in}}%
\pgfpathlineto{\pgfqpoint{2.452040in}{0.945952in}}%
\pgfpathlineto{\pgfqpoint{2.459846in}{0.932555in}}%
\pgfpathlineto{\pgfqpoint{2.467653in}{0.901418in}}%
\pgfpathlineto{\pgfqpoint{2.483266in}{0.828898in}}%
\pgfpathlineto{\pgfqpoint{2.491073in}{0.806866in}}%
\pgfpathlineto{\pgfqpoint{2.498879in}{0.803682in}}%
\pgfpathlineto{\pgfqpoint{2.506686in}{0.788154in}}%
\pgfpathlineto{\pgfqpoint{2.530106in}{0.690036in}}%
\pgfpathlineto{\pgfqpoint{2.545719in}{0.673620in}}%
\pgfpathlineto{\pgfqpoint{2.553526in}{0.669328in}}%
\pgfpathlineto{\pgfqpoint{2.561332in}{0.657032in}}%
\pgfpathlineto{\pgfqpoint{2.569139in}{0.638123in}}%
\pgfpathlineto{\pgfqpoint{2.608172in}{0.584091in}}%
\pgfpathlineto{\pgfqpoint{2.615978in}{0.579542in}}%
\pgfpathlineto{\pgfqpoint{2.623785in}{0.578456in}}%
\pgfpathlineto{\pgfqpoint{2.631592in}{0.623792in}}%
\pgfpathlineto{\pgfqpoint{2.639398in}{0.712478in}}%
\pgfpathlineto{\pgfqpoint{2.655011in}{0.955421in}}%
\pgfpathlineto{\pgfqpoint{2.662818in}{1.056827in}}%
\pgfpathlineto{\pgfqpoint{2.670625in}{1.129884in}}%
\pgfpathlineto{\pgfqpoint{2.678431in}{1.169159in}}%
\pgfpathlineto{\pgfqpoint{2.686238in}{1.186961in}}%
\pgfpathlineto{\pgfqpoint{2.694044in}{1.192870in}}%
\pgfpathlineto{\pgfqpoint{2.709658in}{1.193782in}}%
\pgfpathlineto{\pgfqpoint{2.787723in}{1.192870in}}%
\pgfpathlineto{\pgfqpoint{2.795530in}{1.300870in}}%
\pgfpathlineto{\pgfqpoint{2.811143in}{1.570836in}}%
\pgfpathlineto{\pgfqpoint{2.818950in}{1.672443in}}%
\pgfpathlineto{\pgfqpoint{2.826756in}{1.741980in}}%
\pgfpathlineto{\pgfqpoint{2.834563in}{1.781065in}}%
\pgfpathlineto{\pgfqpoint{2.842370in}{1.796248in}}%
\pgfpathlineto{\pgfqpoint{2.857983in}{1.799638in}}%
\pgfpathlineto{\pgfqpoint{2.865789in}{1.813949in}}%
\pgfpathlineto{\pgfqpoint{2.873596in}{1.848571in}}%
\pgfpathlineto{\pgfqpoint{2.897016in}{2.011177in}}%
\pgfpathlineto{\pgfqpoint{2.904822in}{2.038855in}}%
\pgfpathlineto{\pgfqpoint{2.943855in}{2.038436in}}%
\pgfpathlineto{\pgfqpoint{2.951662in}{2.009906in}}%
\pgfpathlineto{\pgfqpoint{2.959469in}{1.959155in}}%
\pgfpathlineto{\pgfqpoint{2.982888in}{1.959155in}}%
\pgfpathlineto{\pgfqpoint{2.998502in}{2.025235in}}%
\pgfpathlineto{\pgfqpoint{3.006308in}{2.032781in}}%
\pgfpathlineto{\pgfqpoint{3.037535in}{2.032781in}}%
\pgfpathlineto{\pgfqpoint{3.045341in}{2.011783in}}%
\pgfpathlineto{\pgfqpoint{3.053148in}{1.966478in}}%
\pgfpathlineto{\pgfqpoint{3.060954in}{1.936978in}}%
\pgfpathlineto{\pgfqpoint{3.076568in}{1.936978in}}%
\pgfpathlineto{\pgfqpoint{3.084374in}{1.966705in}}%
\pgfpathlineto{\pgfqpoint{3.092181in}{2.008265in}}%
\pgfpathlineto{\pgfqpoint{3.099987in}{2.024855in}}%
\pgfpathlineto{\pgfqpoint{3.131214in}{2.024855in}}%
\pgfpathlineto{\pgfqpoint{3.139020in}{2.012945in}}%
\pgfpathlineto{\pgfqpoint{3.146827in}{1.974885in}}%
\pgfpathlineto{\pgfqpoint{3.154634in}{1.918749in}}%
\pgfpathlineto{\pgfqpoint{3.170247in}{1.918749in}}%
\pgfpathlineto{\pgfqpoint{3.178053in}{1.943689in}}%
\pgfpathlineto{\pgfqpoint{3.185860in}{1.990455in}}%
\pgfpathlineto{\pgfqpoint{3.193667in}{2.014401in}}%
\pgfpathlineto{\pgfqpoint{3.224893in}{2.014401in}}%
\pgfpathlineto{\pgfqpoint{3.232700in}{2.010400in}}%
\pgfpathlineto{\pgfqpoint{3.240506in}{1.979162in}}%
\pgfpathlineto{\pgfqpoint{3.248313in}{1.927154in}}%
\pgfpathlineto{\pgfqpoint{3.271733in}{1.927154in}}%
\pgfpathlineto{\pgfqpoint{3.279539in}{1.969031in}}%
\pgfpathlineto{\pgfqpoint{3.287346in}{2.000994in}}%
\pgfpathlineto{\pgfqpoint{3.295152in}{2.006173in}}%
\pgfpathlineto{\pgfqpoint{3.326379in}{2.006173in}}%
\pgfpathlineto{\pgfqpoint{3.334185in}{1.983406in}}%
\pgfpathlineto{\pgfqpoint{3.341992in}{1.937330in}}%
\pgfpathlineto{\pgfqpoint{3.349799in}{1.917164in}}%
\pgfpathlineto{\pgfqpoint{3.365412in}{1.917164in}}%
\pgfpathlineto{\pgfqpoint{3.373218in}{1.947887in}}%
\pgfpathlineto{\pgfqpoint{3.381025in}{1.986038in}}%
\pgfpathlineto{\pgfqpoint{3.388832in}{1.998910in}}%
\pgfpathlineto{\pgfqpoint{3.420058in}{1.998910in}}%
\pgfpathlineto{\pgfqpoint{3.427865in}{1.983736in}}%
\pgfpathlineto{\pgfqpoint{3.435671in}{1.943529in}}%
\pgfpathlineto{\pgfqpoint{3.443478in}{1.892841in}}%
\pgfpathlineto{\pgfqpoint{3.459091in}{1.892841in}}%
\pgfpathlineto{\pgfqpoint{3.459091in}{1.892841in}}%
\pgfusepath{stroke}%
\end{pgfscope}%
\begin{pgfscope}%
\pgfpathrectangle{\pgfqpoint{0.500000in}{0.375000in}}{\pgfqpoint{3.100000in}{2.265000in}}%
\pgfusepath{clip}%
\pgfsetrectcap%
\pgfsetroundjoin%
\pgfsetlinewidth{1.505625pt}%
\definecolor{currentstroke}{rgb}{1.000000,0.000000,0.000000}%
\pgfsetstrokecolor{currentstroke}%
\pgfsetdash{}{0pt}%
\pgfpathmoveto{\pgfqpoint{0.640909in}{0.477955in}}%
\pgfpathlineto{\pgfqpoint{0.750201in}{0.478986in}}%
\pgfpathlineto{\pgfqpoint{0.773621in}{0.482465in}}%
\pgfpathlineto{\pgfqpoint{0.797041in}{0.490105in}}%
\pgfpathlineto{\pgfqpoint{0.812654in}{0.497950in}}%
\pgfpathlineto{\pgfqpoint{0.828267in}{0.508292in}}%
\pgfpathlineto{\pgfqpoint{0.843881in}{0.521804in}}%
\pgfpathlineto{\pgfqpoint{0.859494in}{0.539751in}}%
\pgfpathlineto{\pgfqpoint{0.875107in}{0.563759in}}%
\pgfpathlineto{\pgfqpoint{0.890720in}{0.596041in}}%
\pgfpathlineto{\pgfqpoint{0.906333in}{0.640240in}}%
\pgfpathlineto{\pgfqpoint{0.921947in}{0.701376in}}%
\pgfpathlineto{\pgfqpoint{0.937560in}{0.779823in}}%
\pgfpathlineto{\pgfqpoint{0.968786in}{0.965097in}}%
\pgfpathlineto{\pgfqpoint{0.992206in}{1.100613in}}%
\pgfpathlineto{\pgfqpoint{1.007819in}{1.177382in}}%
\pgfpathlineto{\pgfqpoint{1.015626in}{1.207600in}}%
\pgfpathlineto{\pgfqpoint{1.023432in}{1.230626in}}%
\pgfpathlineto{\pgfqpoint{1.031239in}{1.244289in}}%
\pgfpathlineto{\pgfqpoint{1.039046in}{1.246814in}}%
\pgfpathlineto{\pgfqpoint{1.046852in}{1.238372in}}%
\pgfpathlineto{\pgfqpoint{1.062465in}{1.207821in}}%
\pgfpathlineto{\pgfqpoint{1.070272in}{1.207325in}}%
\pgfpathlineto{\pgfqpoint{1.078079in}{1.216131in}}%
\pgfpathlineto{\pgfqpoint{1.085885in}{1.236688in}}%
\pgfpathlineto{\pgfqpoint{1.093692in}{1.271544in}}%
\pgfpathlineto{\pgfqpoint{1.101498in}{1.319640in}}%
\pgfpathlineto{\pgfqpoint{1.117112in}{1.448640in}}%
\pgfpathlineto{\pgfqpoint{1.140531in}{1.694831in}}%
\pgfpathlineto{\pgfqpoint{1.195178in}{2.279530in}}%
\pgfpathlineto{\pgfqpoint{1.202984in}{2.345574in}}%
\pgfpathlineto{\pgfqpoint{1.210791in}{2.390978in}}%
\pgfpathlineto{\pgfqpoint{1.218597in}{2.411447in}}%
\pgfpathlineto{\pgfqpoint{1.226404in}{2.407545in}}%
\pgfpathlineto{\pgfqpoint{1.234211in}{2.383168in}}%
\pgfpathlineto{\pgfqpoint{1.249824in}{2.301232in}}%
\pgfpathlineto{\pgfqpoint{1.265437in}{2.219526in}}%
\pgfpathlineto{\pgfqpoint{1.296663in}{2.081102in}}%
\pgfpathlineto{\pgfqpoint{1.343503in}{1.828203in}}%
\pgfpathlineto{\pgfqpoint{1.351309in}{1.804325in}}%
\pgfpathlineto{\pgfqpoint{1.366923in}{1.769562in}}%
\pgfpathlineto{\pgfqpoint{1.374729in}{1.755746in}}%
\pgfpathlineto{\pgfqpoint{1.382536in}{1.745781in}}%
\pgfpathlineto{\pgfqpoint{1.390342in}{1.741336in}}%
\pgfpathlineto{\pgfqpoint{1.398149in}{1.743897in}}%
\pgfpathlineto{\pgfqpoint{1.405956in}{1.753789in}}%
\pgfpathlineto{\pgfqpoint{1.421569in}{1.788818in}}%
\pgfpathlineto{\pgfqpoint{1.429375in}{1.807107in}}%
\pgfpathlineto{\pgfqpoint{1.437182in}{1.821161in}}%
\pgfpathlineto{\pgfqpoint{1.444989in}{1.830854in}}%
\pgfpathlineto{\pgfqpoint{1.452795in}{1.828456in}}%
\pgfpathlineto{\pgfqpoint{1.460602in}{1.813896in}}%
\pgfpathlineto{\pgfqpoint{1.484022in}{1.741354in}}%
\pgfpathlineto{\pgfqpoint{1.491828in}{1.731077in}}%
\pgfpathlineto{\pgfqpoint{1.499635in}{1.730529in}}%
\pgfpathlineto{\pgfqpoint{1.515248in}{1.732355in}}%
\pgfpathlineto{\pgfqpoint{1.523055in}{1.724520in}}%
\pgfpathlineto{\pgfqpoint{1.530861in}{1.710479in}}%
\pgfpathlineto{\pgfqpoint{1.538668in}{1.683294in}}%
\pgfpathlineto{\pgfqpoint{1.569894in}{1.545925in}}%
\pgfpathlineto{\pgfqpoint{1.577701in}{1.525114in}}%
\pgfpathlineto{\pgfqpoint{1.593314in}{1.497520in}}%
\pgfpathlineto{\pgfqpoint{1.608927in}{1.454086in}}%
\pgfpathlineto{\pgfqpoint{1.616734in}{1.425191in}}%
\pgfpathlineto{\pgfqpoint{1.647960in}{1.261040in}}%
\pgfpathlineto{\pgfqpoint{1.663573in}{1.194284in}}%
\pgfpathlineto{\pgfqpoint{1.679187in}{1.144689in}}%
\pgfpathlineto{\pgfqpoint{1.718220in}{1.023531in}}%
\pgfpathlineto{\pgfqpoint{1.733833in}{0.992952in}}%
\pgfpathlineto{\pgfqpoint{1.765059in}{0.941451in}}%
\pgfpathlineto{\pgfqpoint{1.772866in}{0.940763in}}%
\pgfpathlineto{\pgfqpoint{1.780672in}{0.950403in}}%
\pgfpathlineto{\pgfqpoint{1.788479in}{0.970527in}}%
\pgfpathlineto{\pgfqpoint{1.796286in}{1.000303in}}%
\pgfpathlineto{\pgfqpoint{1.811899in}{1.081393in}}%
\pgfpathlineto{\pgfqpoint{1.835319in}{1.233811in}}%
\pgfpathlineto{\pgfqpoint{1.882158in}{1.545976in}}%
\pgfpathlineto{\pgfqpoint{1.889965in}{1.588720in}}%
\pgfpathlineto{\pgfqpoint{1.897771in}{1.617249in}}%
\pgfpathlineto{\pgfqpoint{1.905578in}{1.624951in}}%
\pgfpathlineto{\pgfqpoint{1.913385in}{1.606690in}}%
\pgfpathlineto{\pgfqpoint{1.921191in}{1.561178in}}%
\pgfpathlineto{\pgfqpoint{1.928998in}{1.490469in}}%
\pgfpathlineto{\pgfqpoint{1.952418in}{1.214717in}}%
\pgfpathlineto{\pgfqpoint{1.960224in}{1.155509in}}%
\pgfpathlineto{\pgfqpoint{1.968031in}{1.122962in}}%
\pgfpathlineto{\pgfqpoint{1.975837in}{1.107041in}}%
\pgfpathlineto{\pgfqpoint{1.983644in}{1.109530in}}%
\pgfpathlineto{\pgfqpoint{1.991451in}{1.131245in}}%
\pgfpathlineto{\pgfqpoint{1.999257in}{1.171606in}}%
\pgfpathlineto{\pgfqpoint{2.007064in}{1.228736in}}%
\pgfpathlineto{\pgfqpoint{2.022677in}{1.382818in}}%
\pgfpathlineto{\pgfqpoint{2.046097in}{1.634049in}}%
\pgfpathlineto{\pgfqpoint{2.061710in}{1.790230in}}%
\pgfpathlineto{\pgfqpoint{2.077323in}{1.951018in}}%
\pgfpathlineto{\pgfqpoint{2.085130in}{2.009274in}}%
\pgfpathlineto{\pgfqpoint{2.092936in}{2.041330in}}%
\pgfpathlineto{\pgfqpoint{2.100743in}{2.041810in}}%
\pgfpathlineto{\pgfqpoint{2.108549in}{2.009453in}}%
\pgfpathlineto{\pgfqpoint{2.116356in}{1.949899in}}%
\pgfpathlineto{\pgfqpoint{2.155389in}{1.567588in}}%
\pgfpathlineto{\pgfqpoint{2.171002in}{1.463889in}}%
\pgfpathlineto{\pgfqpoint{2.178809in}{1.424443in}}%
\pgfpathlineto{\pgfqpoint{2.186615in}{1.394314in}}%
\pgfpathlineto{\pgfqpoint{2.194422in}{1.374263in}}%
\pgfpathlineto{\pgfqpoint{2.202229in}{1.363131in}}%
\pgfpathlineto{\pgfqpoint{2.210035in}{1.358336in}}%
\pgfpathlineto{\pgfqpoint{2.225648in}{1.353798in}}%
\pgfpathlineto{\pgfqpoint{2.233455in}{1.344926in}}%
\pgfpathlineto{\pgfqpoint{2.241262in}{1.327274in}}%
\pgfpathlineto{\pgfqpoint{2.256875in}{1.270463in}}%
\pgfpathlineto{\pgfqpoint{2.272488in}{1.208267in}}%
\pgfpathlineto{\pgfqpoint{2.288101in}{1.160218in}}%
\pgfpathlineto{\pgfqpoint{2.303714in}{1.124685in}}%
\pgfpathlineto{\pgfqpoint{2.311521in}{1.110130in}}%
\pgfpathlineto{\pgfqpoint{2.319328in}{1.099299in}}%
\pgfpathlineto{\pgfqpoint{2.342747in}{1.078344in}}%
\pgfpathlineto{\pgfqpoint{2.350554in}{1.062885in}}%
\pgfpathlineto{\pgfqpoint{2.381780in}{0.981922in}}%
\pgfpathlineto{\pgfqpoint{2.389587in}{0.972128in}}%
\pgfpathlineto{\pgfqpoint{2.405200in}{0.955952in}}%
\pgfpathlineto{\pgfqpoint{2.413007in}{0.943854in}}%
\pgfpathlineto{\pgfqpoint{2.436427in}{0.927642in}}%
\pgfpathlineto{\pgfqpoint{2.444233in}{0.917203in}}%
\pgfpathlineto{\pgfqpoint{2.452040in}{0.902675in}}%
\pgfpathlineto{\pgfqpoint{2.459846in}{0.871090in}}%
\pgfpathlineto{\pgfqpoint{2.475460in}{0.794890in}}%
\pgfpathlineto{\pgfqpoint{2.483266in}{0.766680in}}%
\pgfpathlineto{\pgfqpoint{2.522299in}{0.670028in}}%
\pgfpathlineto{\pgfqpoint{2.530106in}{0.657184in}}%
\pgfpathlineto{\pgfqpoint{2.537912in}{0.648042in}}%
\pgfpathlineto{\pgfqpoint{2.553526in}{0.635498in}}%
\pgfpathlineto{\pgfqpoint{2.561332in}{0.625927in}}%
\pgfpathlineto{\pgfqpoint{2.576945in}{0.602351in}}%
\pgfpathlineto{\pgfqpoint{2.608172in}{0.568194in}}%
\pgfpathlineto{\pgfqpoint{2.615978in}{0.561281in}}%
\pgfpathlineto{\pgfqpoint{2.623785in}{0.560467in}}%
\pgfpathlineto{\pgfqpoint{2.631592in}{0.563921in}}%
\pgfpathlineto{\pgfqpoint{2.639398in}{0.573197in}}%
\pgfpathlineto{\pgfqpoint{2.647205in}{0.589719in}}%
\pgfpathlineto{\pgfqpoint{2.655011in}{0.613279in}}%
\pgfpathlineto{\pgfqpoint{2.670625in}{0.678100in}}%
\pgfpathlineto{\pgfqpoint{2.694044in}{0.796369in}}%
\pgfpathlineto{\pgfqpoint{2.740884in}{1.036641in}}%
\pgfpathlineto{\pgfqpoint{2.756497in}{1.104077in}}%
\pgfpathlineto{\pgfqpoint{2.764304in}{1.124498in}}%
\pgfpathlineto{\pgfqpoint{2.772110in}{1.133395in}}%
\pgfpathlineto{\pgfqpoint{2.779917in}{1.136342in}}%
\pgfpathlineto{\pgfqpoint{2.787723in}{1.143238in}}%
\pgfpathlineto{\pgfqpoint{2.795530in}{1.153925in}}%
\pgfpathlineto{\pgfqpoint{2.803337in}{1.171002in}}%
\pgfpathlineto{\pgfqpoint{2.811143in}{1.194994in}}%
\pgfpathlineto{\pgfqpoint{2.826756in}{1.259500in}}%
\pgfpathlineto{\pgfqpoint{2.850176in}{1.380043in}}%
\pgfpathlineto{\pgfqpoint{2.889209in}{1.611424in}}%
\pgfpathlineto{\pgfqpoint{2.912629in}{1.782336in}}%
\pgfpathlineto{\pgfqpoint{2.920436in}{1.826651in}}%
\pgfpathlineto{\pgfqpoint{2.928242in}{1.858814in}}%
\pgfpathlineto{\pgfqpoint{2.936049in}{1.872125in}}%
\pgfpathlineto{\pgfqpoint{2.943855in}{1.870192in}}%
\pgfpathlineto{\pgfqpoint{2.959469in}{1.847049in}}%
\pgfpathlineto{\pgfqpoint{2.967275in}{1.840437in}}%
\pgfpathlineto{\pgfqpoint{2.975082in}{1.844011in}}%
\pgfpathlineto{\pgfqpoint{2.982888in}{1.852595in}}%
\pgfpathlineto{\pgfqpoint{2.998502in}{1.874690in}}%
\pgfpathlineto{\pgfqpoint{3.006308in}{1.882892in}}%
\pgfpathlineto{\pgfqpoint{3.021921in}{1.890492in}}%
\pgfpathlineto{\pgfqpoint{3.029728in}{1.889388in}}%
\pgfpathlineto{\pgfqpoint{3.037535in}{1.880938in}}%
\pgfpathlineto{\pgfqpoint{3.060954in}{1.835731in}}%
\pgfpathlineto{\pgfqpoint{3.068761in}{1.832791in}}%
\pgfpathlineto{\pgfqpoint{3.076568in}{1.839430in}}%
\pgfpathlineto{\pgfqpoint{3.099987in}{1.870755in}}%
\pgfpathlineto{\pgfqpoint{3.115601in}{1.879767in}}%
\pgfpathlineto{\pgfqpoint{3.123407in}{1.881281in}}%
\pgfpathlineto{\pgfqpoint{3.131214in}{1.875322in}}%
\pgfpathlineto{\pgfqpoint{3.139020in}{1.863000in}}%
\pgfpathlineto{\pgfqpoint{3.154634in}{1.831868in}}%
\pgfpathlineto{\pgfqpoint{3.162440in}{1.825056in}}%
\pgfpathlineto{\pgfqpoint{3.170247in}{1.830213in}}%
\pgfpathlineto{\pgfqpoint{3.201473in}{1.868258in}}%
\pgfpathlineto{\pgfqpoint{3.217086in}{1.875350in}}%
\pgfpathlineto{\pgfqpoint{3.224893in}{1.871253in}}%
\pgfpathlineto{\pgfqpoint{3.232700in}{1.860345in}}%
\pgfpathlineto{\pgfqpoint{3.248313in}{1.828918in}}%
\pgfpathlineto{\pgfqpoint{3.256119in}{1.818001in}}%
\pgfpathlineto{\pgfqpoint{3.263926in}{1.820600in}}%
\pgfpathlineto{\pgfqpoint{3.271733in}{1.829045in}}%
\pgfpathlineto{\pgfqpoint{3.287346in}{1.851038in}}%
\pgfpathlineto{\pgfqpoint{3.295152in}{1.858631in}}%
\pgfpathlineto{\pgfqpoint{3.310766in}{1.866604in}}%
\pgfpathlineto{\pgfqpoint{3.318572in}{1.865109in}}%
\pgfpathlineto{\pgfqpoint{3.326379in}{1.856450in}}%
\pgfpathlineto{\pgfqpoint{3.349799in}{1.812722in}}%
\pgfpathlineto{\pgfqpoint{3.357605in}{1.811417in}}%
\pgfpathlineto{\pgfqpoint{3.365412in}{1.818639in}}%
\pgfpathlineto{\pgfqpoint{3.388832in}{1.849596in}}%
\pgfpathlineto{\pgfqpoint{3.404445in}{1.857850in}}%
\pgfpathlineto{\pgfqpoint{3.412251in}{1.858221in}}%
\pgfpathlineto{\pgfqpoint{3.420058in}{1.851293in}}%
\pgfpathlineto{\pgfqpoint{3.427865in}{1.838364in}}%
\pgfpathlineto{\pgfqpoint{3.443478in}{1.807405in}}%
\pgfpathlineto{\pgfqpoint{3.451284in}{1.801846in}}%
\pgfpathlineto{\pgfqpoint{3.459091in}{1.805978in}}%
\pgfpathlineto{\pgfqpoint{3.459091in}{1.805978in}}%
\pgfusepath{stroke}%
\end{pgfscope}%
\begin{pgfscope}%
\pgfpathrectangle{\pgfqpoint{0.500000in}{0.375000in}}{\pgfqpoint{3.100000in}{2.265000in}}%
\pgfusepath{clip}%
\pgfsetrectcap%
\pgfsetroundjoin%
\pgfsetlinewidth{1.505625pt}%
\definecolor{currentstroke}{rgb}{0.000000,0.500000,0.000000}%
\pgfsetstrokecolor{currentstroke}%
\pgfsetdash{}{0pt}%
\pgfpathmoveto{\pgfqpoint{0.640909in}{0.477955in}}%
\pgfpathlineto{\pgfqpoint{0.742395in}{0.479042in}}%
\pgfpathlineto{\pgfqpoint{0.765815in}{0.483134in}}%
\pgfpathlineto{\pgfqpoint{0.789234in}{0.490673in}}%
\pgfpathlineto{\pgfqpoint{0.836074in}{0.510361in}}%
\pgfpathlineto{\pgfqpoint{0.859494in}{0.523361in}}%
\pgfpathlineto{\pgfqpoint{0.875107in}{0.536679in}}%
\pgfpathlineto{\pgfqpoint{0.882914in}{0.545679in}}%
\pgfpathlineto{\pgfqpoint{0.898527in}{0.570830in}}%
\pgfpathlineto{\pgfqpoint{0.914140in}{0.609456in}}%
\pgfpathlineto{\pgfqpoint{0.945366in}{0.705380in}}%
\pgfpathlineto{\pgfqpoint{0.953173in}{0.722797in}}%
\pgfpathlineto{\pgfqpoint{0.960980in}{0.734812in}}%
\pgfpathlineto{\pgfqpoint{0.968786in}{0.740798in}}%
\pgfpathlineto{\pgfqpoint{0.976593in}{0.740388in}}%
\pgfpathlineto{\pgfqpoint{0.984399in}{0.733820in}}%
\pgfpathlineto{\pgfqpoint{0.992206in}{0.721307in}}%
\pgfpathlineto{\pgfqpoint{1.000013in}{0.703423in}}%
\pgfpathlineto{\pgfqpoint{1.007819in}{0.679147in}}%
\pgfpathlineto{\pgfqpoint{1.023432in}{0.615972in}}%
\pgfpathlineto{\pgfqpoint{1.031239in}{0.582087in}}%
\pgfpathlineto{\pgfqpoint{1.039046in}{0.556030in}}%
\pgfpathlineto{\pgfqpoint{1.046852in}{0.549685in}}%
\pgfpathlineto{\pgfqpoint{1.054659in}{0.559389in}}%
\pgfpathlineto{\pgfqpoint{1.062465in}{0.563535in}}%
\pgfpathlineto{\pgfqpoint{1.070272in}{0.563237in}}%
\pgfpathlineto{\pgfqpoint{1.078079in}{0.578809in}}%
\pgfpathlineto{\pgfqpoint{1.085885in}{0.624415in}}%
\pgfpathlineto{\pgfqpoint{1.124918in}{0.960036in}}%
\pgfpathlineto{\pgfqpoint{1.132725in}{0.996537in}}%
\pgfpathlineto{\pgfqpoint{1.140531in}{1.016826in}}%
\pgfpathlineto{\pgfqpoint{1.148338in}{1.026014in}}%
\pgfpathlineto{\pgfqpoint{1.156145in}{1.030547in}}%
\pgfpathlineto{\pgfqpoint{1.163951in}{1.020206in}}%
\pgfpathlineto{\pgfqpoint{1.171758in}{0.992680in}}%
\pgfpathlineto{\pgfqpoint{1.179564in}{0.944926in}}%
\pgfpathlineto{\pgfqpoint{1.195178in}{0.803453in}}%
\pgfpathlineto{\pgfqpoint{1.210791in}{0.651678in}}%
\pgfpathlineto{\pgfqpoint{1.218597in}{0.591599in}}%
\pgfpathlineto{\pgfqpoint{1.226404in}{0.563204in}}%
\pgfpathlineto{\pgfqpoint{1.234211in}{0.577014in}}%
\pgfpathlineto{\pgfqpoint{1.242017in}{0.604914in}}%
\pgfpathlineto{\pgfqpoint{1.249824in}{0.626923in}}%
\pgfpathlineto{\pgfqpoint{1.257630in}{0.638296in}}%
\pgfpathlineto{\pgfqpoint{1.265437in}{0.640165in}}%
\pgfpathlineto{\pgfqpoint{1.273244in}{0.634691in}}%
\pgfpathlineto{\pgfqpoint{1.281050in}{0.623122in}}%
\pgfpathlineto{\pgfqpoint{1.296663in}{0.587353in}}%
\pgfpathlineto{\pgfqpoint{1.304470in}{0.580882in}}%
\pgfpathlineto{\pgfqpoint{1.312277in}{0.587455in}}%
\pgfpathlineto{\pgfqpoint{1.327890in}{0.613329in}}%
\pgfpathlineto{\pgfqpoint{1.335696in}{0.620214in}}%
\pgfpathlineto{\pgfqpoint{1.343503in}{0.617925in}}%
\pgfpathlineto{\pgfqpoint{1.351309in}{0.606419in}}%
\pgfpathlineto{\pgfqpoint{1.366923in}{0.566497in}}%
\pgfpathlineto{\pgfqpoint{1.382536in}{0.524536in}}%
\pgfpathlineto{\pgfqpoint{1.390342in}{0.515040in}}%
\pgfpathlineto{\pgfqpoint{1.398149in}{0.520014in}}%
\pgfpathlineto{\pgfqpoint{1.429375in}{0.576980in}}%
\pgfpathlineto{\pgfqpoint{1.437182in}{0.582257in}}%
\pgfpathlineto{\pgfqpoint{1.444989in}{0.581593in}}%
\pgfpathlineto{\pgfqpoint{1.460602in}{0.561057in}}%
\pgfpathlineto{\pgfqpoint{1.468408in}{0.557995in}}%
\pgfpathlineto{\pgfqpoint{1.476215in}{0.558837in}}%
\pgfpathlineto{\pgfqpoint{1.499635in}{0.554280in}}%
\pgfpathlineto{\pgfqpoint{1.515248in}{0.555747in}}%
\pgfpathlineto{\pgfqpoint{1.530861in}{0.537803in}}%
\pgfpathlineto{\pgfqpoint{1.538668in}{0.543702in}}%
\pgfpathlineto{\pgfqpoint{1.554281in}{0.580240in}}%
\pgfpathlineto{\pgfqpoint{1.562088in}{0.590562in}}%
\pgfpathlineto{\pgfqpoint{1.569894in}{0.591899in}}%
\pgfpathlineto{\pgfqpoint{1.577701in}{0.584549in}}%
\pgfpathlineto{\pgfqpoint{1.593314in}{0.560757in}}%
\pgfpathlineto{\pgfqpoint{1.601121in}{0.538696in}}%
\pgfpathlineto{\pgfqpoint{1.608927in}{0.533381in}}%
\pgfpathlineto{\pgfqpoint{1.640154in}{0.612641in}}%
\pgfpathlineto{\pgfqpoint{1.647960in}{0.624682in}}%
\pgfpathlineto{\pgfqpoint{1.655767in}{0.627450in}}%
\pgfpathlineto{\pgfqpoint{1.663573in}{0.619106in}}%
\pgfpathlineto{\pgfqpoint{1.686993in}{0.561567in}}%
\pgfpathlineto{\pgfqpoint{1.694800in}{0.542071in}}%
\pgfpathlineto{\pgfqpoint{1.702606in}{0.533355in}}%
\pgfpathlineto{\pgfqpoint{1.710413in}{0.533063in}}%
\pgfpathlineto{\pgfqpoint{1.733833in}{0.537683in}}%
\pgfpathlineto{\pgfqpoint{1.749446in}{0.536405in}}%
\pgfpathlineto{\pgfqpoint{1.757253in}{0.530268in}}%
\pgfpathlineto{\pgfqpoint{1.765059in}{0.517361in}}%
\pgfpathlineto{\pgfqpoint{1.772866in}{0.515549in}}%
\pgfpathlineto{\pgfqpoint{1.780672in}{0.541937in}}%
\pgfpathlineto{\pgfqpoint{1.796286in}{0.631430in}}%
\pgfpathlineto{\pgfqpoint{1.811899in}{0.723747in}}%
\pgfpathlineto{\pgfqpoint{1.819705in}{0.762075in}}%
\pgfpathlineto{\pgfqpoint{1.827512in}{0.791774in}}%
\pgfpathlineto{\pgfqpoint{1.835319in}{0.811653in}}%
\pgfpathlineto{\pgfqpoint{1.843125in}{0.822613in}}%
\pgfpathlineto{\pgfqpoint{1.850932in}{0.828065in}}%
\pgfpathlineto{\pgfqpoint{1.858738in}{0.825356in}}%
\pgfpathlineto{\pgfqpoint{1.866545in}{0.813302in}}%
\pgfpathlineto{\pgfqpoint{1.874352in}{0.789610in}}%
\pgfpathlineto{\pgfqpoint{1.882158in}{0.751969in}}%
\pgfpathlineto{\pgfqpoint{1.905578in}{0.604371in}}%
\pgfpathlineto{\pgfqpoint{1.913385in}{0.602781in}}%
\pgfpathlineto{\pgfqpoint{1.921191in}{0.649654in}}%
\pgfpathlineto{\pgfqpoint{1.936804in}{0.773910in}}%
\pgfpathlineto{\pgfqpoint{1.944611in}{0.813462in}}%
\pgfpathlineto{\pgfqpoint{1.952418in}{0.819701in}}%
\pgfpathlineto{\pgfqpoint{1.960224in}{0.795029in}}%
\pgfpathlineto{\pgfqpoint{1.968031in}{0.758756in}}%
\pgfpathlineto{\pgfqpoint{1.975837in}{0.731834in}}%
\pgfpathlineto{\pgfqpoint{1.983644in}{0.737056in}}%
\pgfpathlineto{\pgfqpoint{1.991451in}{0.781380in}}%
\pgfpathlineto{\pgfqpoint{2.014870in}{0.970173in}}%
\pgfpathlineto{\pgfqpoint{2.022677in}{1.006593in}}%
\pgfpathlineto{\pgfqpoint{2.030484in}{1.023918in}}%
\pgfpathlineto{\pgfqpoint{2.046097in}{1.033912in}}%
\pgfpathlineto{\pgfqpoint{2.053903in}{1.030408in}}%
\pgfpathlineto{\pgfqpoint{2.061710in}{1.002120in}}%
\pgfpathlineto{\pgfqpoint{2.069516in}{0.945459in}}%
\pgfpathlineto{\pgfqpoint{2.085130in}{0.777104in}}%
\pgfpathlineto{\pgfqpoint{2.092936in}{0.691660in}}%
\pgfpathlineto{\pgfqpoint{2.100743in}{0.639713in}}%
\pgfpathlineto{\pgfqpoint{2.108549in}{0.654597in}}%
\pgfpathlineto{\pgfqpoint{2.124163in}{0.755929in}}%
\pgfpathlineto{\pgfqpoint{2.131969in}{0.786456in}}%
\pgfpathlineto{\pgfqpoint{2.139776in}{0.795420in}}%
\pgfpathlineto{\pgfqpoint{2.147582in}{0.784314in}}%
\pgfpathlineto{\pgfqpoint{2.155389in}{0.757536in}}%
\pgfpathlineto{\pgfqpoint{2.163196in}{0.717550in}}%
\pgfpathlineto{\pgfqpoint{2.194422in}{0.525787in}}%
\pgfpathlineto{\pgfqpoint{2.202229in}{0.499396in}}%
\pgfpathlineto{\pgfqpoint{2.210035in}{0.485245in}}%
\pgfpathlineto{\pgfqpoint{2.217842in}{0.480987in}}%
\pgfpathlineto{\pgfqpoint{2.225648in}{0.485826in}}%
\pgfpathlineto{\pgfqpoint{2.233455in}{0.502020in}}%
\pgfpathlineto{\pgfqpoint{2.256875in}{0.570324in}}%
\pgfpathlineto{\pgfqpoint{2.264681in}{0.582789in}}%
\pgfpathlineto{\pgfqpoint{2.272488in}{0.588699in}}%
\pgfpathlineto{\pgfqpoint{2.280295in}{0.589182in}}%
\pgfpathlineto{\pgfqpoint{2.288101in}{0.584314in}}%
\pgfpathlineto{\pgfqpoint{2.295908in}{0.573854in}}%
\pgfpathlineto{\pgfqpoint{2.303714in}{0.556948in}}%
\pgfpathlineto{\pgfqpoint{2.319328in}{0.513925in}}%
\pgfpathlineto{\pgfqpoint{2.327134in}{0.497523in}}%
\pgfpathlineto{\pgfqpoint{2.334941in}{0.490130in}}%
\pgfpathlineto{\pgfqpoint{2.342747in}{0.499969in}}%
\pgfpathlineto{\pgfqpoint{2.358361in}{0.536299in}}%
\pgfpathlineto{\pgfqpoint{2.366167in}{0.547494in}}%
\pgfpathlineto{\pgfqpoint{2.373974in}{0.551536in}}%
\pgfpathlineto{\pgfqpoint{2.381780in}{0.550412in}}%
\pgfpathlineto{\pgfqpoint{2.397394in}{0.539659in}}%
\pgfpathlineto{\pgfqpoint{2.405200in}{0.527727in}}%
\pgfpathlineto{\pgfqpoint{2.413007in}{0.507184in}}%
\pgfpathlineto{\pgfqpoint{2.420813in}{0.500284in}}%
\pgfpathlineto{\pgfqpoint{2.428620in}{0.498028in}}%
\pgfpathlineto{\pgfqpoint{2.436427in}{0.498847in}}%
\pgfpathlineto{\pgfqpoint{2.444233in}{0.505050in}}%
\pgfpathlineto{\pgfqpoint{2.452040in}{0.513889in}}%
\pgfpathlineto{\pgfqpoint{2.459846in}{0.513559in}}%
\pgfpathlineto{\pgfqpoint{2.467653in}{0.504203in}}%
\pgfpathlineto{\pgfqpoint{2.475460in}{0.509237in}}%
\pgfpathlineto{\pgfqpoint{2.491073in}{0.531229in}}%
\pgfpathlineto{\pgfqpoint{2.498879in}{0.535413in}}%
\pgfpathlineto{\pgfqpoint{2.506686in}{0.530364in}}%
\pgfpathlineto{\pgfqpoint{2.530106in}{0.499862in}}%
\pgfpathlineto{\pgfqpoint{2.537912in}{0.494942in}}%
\pgfpathlineto{\pgfqpoint{2.545719in}{0.493291in}}%
\pgfpathlineto{\pgfqpoint{2.561332in}{0.494882in}}%
\pgfpathlineto{\pgfqpoint{2.569139in}{0.496288in}}%
\pgfpathlineto{\pgfqpoint{2.584752in}{0.493198in}}%
\pgfpathlineto{\pgfqpoint{2.592559in}{0.490880in}}%
\pgfpathlineto{\pgfqpoint{2.623785in}{0.491427in}}%
\pgfpathlineto{\pgfqpoint{2.631592in}{0.498353in}}%
\pgfpathlineto{\pgfqpoint{2.639398in}{0.519297in}}%
\pgfpathlineto{\pgfqpoint{2.655011in}{0.594298in}}%
\pgfpathlineto{\pgfqpoint{2.670625in}{0.673209in}}%
\pgfpathlineto{\pgfqpoint{2.678431in}{0.703724in}}%
\pgfpathlineto{\pgfqpoint{2.686238in}{0.726089in}}%
\pgfpathlineto{\pgfqpoint{2.694044in}{0.740563in}}%
\pgfpathlineto{\pgfqpoint{2.701851in}{0.748307in}}%
\pgfpathlineto{\pgfqpoint{2.709658in}{0.751933in}}%
\pgfpathlineto{\pgfqpoint{2.717464in}{0.749576in}}%
\pgfpathlineto{\pgfqpoint{2.725271in}{0.740087in}}%
\pgfpathlineto{\pgfqpoint{2.733077in}{0.722485in}}%
\pgfpathlineto{\pgfqpoint{2.740884in}{0.695746in}}%
\pgfpathlineto{\pgfqpoint{2.748691in}{0.660735in}}%
\pgfpathlineto{\pgfqpoint{2.772110in}{0.539869in}}%
\pgfpathlineto{\pgfqpoint{2.779917in}{0.520655in}}%
\pgfpathlineto{\pgfqpoint{2.787723in}{0.515858in}}%
\pgfpathlineto{\pgfqpoint{2.795530in}{0.531473in}}%
\pgfpathlineto{\pgfqpoint{2.803337in}{0.566038in}}%
\pgfpathlineto{\pgfqpoint{2.826756in}{0.692295in}}%
\pgfpathlineto{\pgfqpoint{2.834563in}{0.725142in}}%
\pgfpathlineto{\pgfqpoint{2.842370in}{0.749509in}}%
\pgfpathlineto{\pgfqpoint{2.850176in}{0.764544in}}%
\pgfpathlineto{\pgfqpoint{2.857983in}{0.770737in}}%
\pgfpathlineto{\pgfqpoint{2.865789in}{0.771940in}}%
\pgfpathlineto{\pgfqpoint{2.873596in}{0.774773in}}%
\pgfpathlineto{\pgfqpoint{2.881403in}{0.773490in}}%
\pgfpathlineto{\pgfqpoint{2.889209in}{0.765127in}}%
\pgfpathlineto{\pgfqpoint{2.897016in}{0.745945in}}%
\pgfpathlineto{\pgfqpoint{2.904822in}{0.715322in}}%
\pgfpathlineto{\pgfqpoint{2.928242in}{0.611010in}}%
\pgfpathlineto{\pgfqpoint{2.936049in}{0.583214in}}%
\pgfpathlineto{\pgfqpoint{2.943855in}{0.561465in}}%
\pgfpathlineto{\pgfqpoint{2.951662in}{0.545938in}}%
\pgfpathlineto{\pgfqpoint{2.959469in}{0.534077in}}%
\pgfpathlineto{\pgfqpoint{2.967275in}{0.525084in}}%
\pgfpathlineto{\pgfqpoint{2.975082in}{0.523275in}}%
\pgfpathlineto{\pgfqpoint{2.982888in}{0.526917in}}%
\pgfpathlineto{\pgfqpoint{2.990695in}{0.536055in}}%
\pgfpathlineto{\pgfqpoint{3.006308in}{0.557459in}}%
\pgfpathlineto{\pgfqpoint{3.014115in}{0.561531in}}%
\pgfpathlineto{\pgfqpoint{3.021921in}{0.563753in}}%
\pgfpathlineto{\pgfqpoint{3.029728in}{0.559230in}}%
\pgfpathlineto{\pgfqpoint{3.053148in}{0.532311in}}%
\pgfpathlineto{\pgfqpoint{3.060954in}{0.523991in}}%
\pgfpathlineto{\pgfqpoint{3.068761in}{0.520617in}}%
\pgfpathlineto{\pgfqpoint{3.076568in}{0.523638in}}%
\pgfpathlineto{\pgfqpoint{3.084374in}{0.531595in}}%
\pgfpathlineto{\pgfqpoint{3.099987in}{0.554372in}}%
\pgfpathlineto{\pgfqpoint{3.107794in}{0.560457in}}%
\pgfpathlineto{\pgfqpoint{3.115601in}{0.563658in}}%
\pgfpathlineto{\pgfqpoint{3.123407in}{0.561569in}}%
\pgfpathlineto{\pgfqpoint{3.162440in}{0.519679in}}%
\pgfpathlineto{\pgfqpoint{3.170247in}{0.521588in}}%
\pgfpathlineto{\pgfqpoint{3.178053in}{0.527886in}}%
\pgfpathlineto{\pgfqpoint{3.201473in}{0.558200in}}%
\pgfpathlineto{\pgfqpoint{3.209280in}{0.561827in}}%
\pgfpathlineto{\pgfqpoint{3.217086in}{0.562034in}}%
\pgfpathlineto{\pgfqpoint{3.224893in}{0.555055in}}%
\pgfpathlineto{\pgfqpoint{3.240506in}{0.536364in}}%
\pgfpathlineto{\pgfqpoint{3.256119in}{0.521751in}}%
\pgfpathlineto{\pgfqpoint{3.263926in}{0.522155in}}%
\pgfpathlineto{\pgfqpoint{3.271733in}{0.526660in}}%
\pgfpathlineto{\pgfqpoint{3.295152in}{0.556957in}}%
\pgfpathlineto{\pgfqpoint{3.302959in}{0.561073in}}%
\pgfpathlineto{\pgfqpoint{3.310766in}{0.563049in}}%
\pgfpathlineto{\pgfqpoint{3.318572in}{0.557992in}}%
\pgfpathlineto{\pgfqpoint{3.341992in}{0.531195in}}%
\pgfpathlineto{\pgfqpoint{3.349799in}{0.522784in}}%
\pgfpathlineto{\pgfqpoint{3.357605in}{0.520359in}}%
\pgfpathlineto{\pgfqpoint{3.365412in}{0.523644in}}%
\pgfpathlineto{\pgfqpoint{3.373218in}{0.531776in}}%
\pgfpathlineto{\pgfqpoint{3.388832in}{0.553664in}}%
\pgfpathlineto{\pgfqpoint{3.396638in}{0.558911in}}%
\pgfpathlineto{\pgfqpoint{3.404445in}{0.561644in}}%
\pgfpathlineto{\pgfqpoint{3.412251in}{0.558652in}}%
\pgfpathlineto{\pgfqpoint{3.443478in}{0.524005in}}%
\pgfpathlineto{\pgfqpoint{3.451284in}{0.519014in}}%
\pgfpathlineto{\pgfqpoint{3.459091in}{0.519786in}}%
\pgfpathlineto{\pgfqpoint{3.459091in}{0.519786in}}%
\pgfusepath{stroke}%
\end{pgfscope}%
\begin{pgfscope}%
\pgfsetrectcap%
\pgfsetmiterjoin%
\pgfsetlinewidth{0.803000pt}%
\definecolor{currentstroke}{rgb}{0.000000,0.000000,0.000000}%
\pgfsetstrokecolor{currentstroke}%
\pgfsetdash{}{0pt}%
\pgfpathmoveto{\pgfqpoint{0.500000in}{0.375000in}}%
\pgfpathlineto{\pgfqpoint{0.500000in}{2.640000in}}%
\pgfusepath{stroke}%
\end{pgfscope}%
\begin{pgfscope}%
\pgfsetrectcap%
\pgfsetmiterjoin%
\pgfsetlinewidth{0.803000pt}%
\definecolor{currentstroke}{rgb}{0.000000,0.000000,0.000000}%
\pgfsetstrokecolor{currentstroke}%
\pgfsetdash{}{0pt}%
\pgfpathmoveto{\pgfqpoint{3.600000in}{0.375000in}}%
\pgfpathlineto{\pgfqpoint{3.600000in}{2.640000in}}%
\pgfusepath{stroke}%
\end{pgfscope}%
\begin{pgfscope}%
\pgfsetrectcap%
\pgfsetmiterjoin%
\pgfsetlinewidth{0.803000pt}%
\definecolor{currentstroke}{rgb}{0.000000,0.000000,0.000000}%
\pgfsetstrokecolor{currentstroke}%
\pgfsetdash{}{0pt}%
\pgfpathmoveto{\pgfqpoint{0.500000in}{0.375000in}}%
\pgfpathlineto{\pgfqpoint{3.600000in}{0.375000in}}%
\pgfusepath{stroke}%
\end{pgfscope}%
\begin{pgfscope}%
\pgfsetrectcap%
\pgfsetmiterjoin%
\pgfsetlinewidth{0.803000pt}%
\definecolor{currentstroke}{rgb}{0.000000,0.000000,0.000000}%
\pgfsetstrokecolor{currentstroke}%
\pgfsetdash{}{0pt}%
\pgfpathmoveto{\pgfqpoint{0.500000in}{2.640000in}}%
\pgfpathlineto{\pgfqpoint{3.600000in}{2.640000in}}%
\pgfusepath{stroke}%
\end{pgfscope}%
\begin{pgfscope}%
\pgfsetbuttcap%
\pgfsetmiterjoin%
\definecolor{currentfill}{rgb}{1.000000,1.000000,1.000000}%
\pgfsetfillcolor{currentfill}%
\pgfsetfillopacity{0.800000}%
\pgfsetlinewidth{1.003750pt}%
\definecolor{currentstroke}{rgb}{0.800000,0.800000,0.800000}%
\pgfsetstrokecolor{currentstroke}%
\pgfsetstrokeopacity{0.800000}%
\pgfsetdash{}{0pt}%
\pgfpathmoveto{\pgfqpoint{1.698533in}{1.947871in}}%
\pgfpathlineto{\pgfqpoint{2.401467in}{1.947871in}}%
\pgfpathquadraticcurveto{\pgfqpoint{2.429244in}{1.947871in}}{\pgfqpoint{2.429244in}{1.975648in}}%
\pgfpathlineto{\pgfqpoint{2.429244in}{2.542778in}}%
\pgfpathquadraticcurveto{\pgfqpoint{2.429244in}{2.570556in}}{\pgfqpoint{2.401467in}{2.570556in}}%
\pgfpathlineto{\pgfqpoint{1.698533in}{2.570556in}}%
\pgfpathquadraticcurveto{\pgfqpoint{1.670756in}{2.570556in}}{\pgfqpoint{1.670756in}{2.542778in}}%
\pgfpathlineto{\pgfqpoint{1.670756in}{1.975648in}}%
\pgfpathquadraticcurveto{\pgfqpoint{1.670756in}{1.947871in}}{\pgfqpoint{1.698533in}{1.947871in}}%
\pgfpathlineto{\pgfqpoint{1.698533in}{1.947871in}}%
\pgfpathclose%
\pgfusepath{stroke,fill}%
\end{pgfscope}%
\begin{pgfscope}%
\pgfsetrectcap%
\pgfsetroundjoin%
\pgfsetlinewidth{1.505625pt}%
\definecolor{currentstroke}{rgb}{0.000000,0.000000,1.000000}%
\pgfsetstrokecolor{currentstroke}%
\pgfsetdash{}{0pt}%
\pgfpathmoveto{\pgfqpoint{1.726311in}{2.466389in}}%
\pgfpathlineto{\pgfqpoint{1.865200in}{2.466389in}}%
\pgfpathlineto{\pgfqpoint{2.004089in}{2.466389in}}%
\pgfusepath{stroke}%
\end{pgfscope}%
\begin{pgfscope}%
\definecolor{textcolor}{rgb}{0.000000,0.000000,0.000000}%
\pgfsetstrokecolor{textcolor}%
\pgfsetfillcolor{textcolor}%
\pgftext[x=2.115200in,y=2.417778in,left,base]{\color{textcolor}\rmfamily\fontsize{10.000000}{12.000000}\selectfont max}%
\end{pgfscope}%
\begin{pgfscope}%
\pgfsetrectcap%
\pgfsetroundjoin%
\pgfsetlinewidth{1.505625pt}%
\definecolor{currentstroke}{rgb}{1.000000,0.000000,0.000000}%
\pgfsetstrokecolor{currentstroke}%
\pgfsetdash{}{0pt}%
\pgfpathmoveto{\pgfqpoint{1.726311in}{2.272716in}}%
\pgfpathlineto{\pgfqpoint{1.865200in}{2.272716in}}%
\pgfpathlineto{\pgfqpoint{2.004089in}{2.272716in}}%
\pgfusepath{stroke}%
\end{pgfscope}%
\begin{pgfscope}%
\definecolor{textcolor}{rgb}{0.000000,0.000000,0.000000}%
\pgfsetstrokecolor{textcolor}%
\pgfsetfillcolor{textcolor}%
\pgftext[x=2.115200in,y=2.224105in,left,base]{\color{textcolor}\rmfamily\fontsize{10.000000}{12.000000}\selectfont \(\displaystyle \mu\)}%
\end{pgfscope}%
\begin{pgfscope}%
\pgfsetrectcap%
\pgfsetroundjoin%
\pgfsetlinewidth{1.505625pt}%
\definecolor{currentstroke}{rgb}{0.000000,0.500000,0.000000}%
\pgfsetstrokecolor{currentstroke}%
\pgfsetdash{}{0pt}%
\pgfpathmoveto{\pgfqpoint{1.726311in}{2.079043in}}%
\pgfpathlineto{\pgfqpoint{1.865200in}{2.079043in}}%
\pgfpathlineto{\pgfqpoint{2.004089in}{2.079043in}}%
\pgfusepath{stroke}%
\end{pgfscope}%
\begin{pgfscope}%
\definecolor{textcolor}{rgb}{0.000000,0.000000,0.000000}%
\pgfsetstrokecolor{textcolor}%
\pgfsetfillcolor{textcolor}%
\pgftext[x=2.115200in,y=2.030432in,left,base]{\color{textcolor}\rmfamily\fontsize{10.000000}{12.000000}\selectfont \(\displaystyle \sigma\)}%
\end{pgfscope}%
\end{pgfpicture}%
\makeatother%
\endgroup%
}
%         \caption{Force Matrix Profile New CS}
%         \label{fig:mp_hist_signal_force}
%     \end{minipage}
% \end{figure}

Figures \ref{fig:hydraulic_sim_standard_pressure} and \ref{fig:hydraulic_sim_signal_pressure} show the original control signal with the anomaly detections from the experiment highlighted with red lines. In this case, the detector was able to locate all of the inflection points in the same locations as previously detected. This provides a 100\% accuracy rate for the two different control signals as is above. The amplitude and waveform of the signals presented here are identical to the force signal presented above. The primary difference is the different amplitudes. The pressure signal is two orders of magnitude larger in amplitude. Which illustrates the detector is robust to scale variations. 

% \begin{figure}[H]
%     \begin{minipage}[t]{0.5\linewidth}
%         %%\centering
%         \resizebox{\linewidth}{!}{%% Creator: Matplotlib, PGF backend
%%
%% To include the figure in your LaTeX document, write
%%   \input{<filename>.pgf}
%%
%% Make sure the required packages are loaded in your preamble
%%   \usepackage{pgf}
%%
%% Also ensure that all the required font packages are loaded; for instance,
%% the lmodern package is sometimes necessary when using math font.
%%   \usepackage{lmodern}
%%
%% Figures using additional raster images can only be included by \input if
%% they are in the same directory as the main LaTeX file. For loading figures
%% from other directories you can use the `import` package
%%   \usepackage{import}
%%
%% and then include the figures with
%%   \import{<path to file>}{<filename>.pgf}
%%
%% Matplotlib used the following preamble
%%
\begingroup%
\makeatletter%
\begin{pgfpicture}%
\pgfpathrectangle{\pgfpointorigin}{\pgfqpoint{4.000000in}{3.000000in}}%
\pgfusepath{use as bounding box, clip}%
\begin{pgfscope}%
\pgfsetbuttcap%
\pgfsetmiterjoin%
\pgfsetlinewidth{0.000000pt}%
\definecolor{currentstroke}{rgb}{1.000000,1.000000,1.000000}%
\pgfsetstrokecolor{currentstroke}%
\pgfsetstrokeopacity{0.000000}%
\pgfsetdash{}{0pt}%
\pgfpathmoveto{\pgfqpoint{0.000000in}{0.000000in}}%
\pgfpathlineto{\pgfqpoint{4.000000in}{0.000000in}}%
\pgfpathlineto{\pgfqpoint{4.000000in}{3.000000in}}%
\pgfpathlineto{\pgfqpoint{0.000000in}{3.000000in}}%
\pgfpathlineto{\pgfqpoint{0.000000in}{0.000000in}}%
\pgfpathclose%
\pgfusepath{}%
\end{pgfscope}%
\begin{pgfscope}%
\pgfsetbuttcap%
\pgfsetmiterjoin%
\definecolor{currentfill}{rgb}{1.000000,1.000000,1.000000}%
\pgfsetfillcolor{currentfill}%
\pgfsetlinewidth{0.000000pt}%
\definecolor{currentstroke}{rgb}{0.000000,0.000000,0.000000}%
\pgfsetstrokecolor{currentstroke}%
\pgfsetstrokeopacity{0.000000}%
\pgfsetdash{}{0pt}%
\pgfpathmoveto{\pgfqpoint{0.500000in}{0.375000in}}%
\pgfpathlineto{\pgfqpoint{3.600000in}{0.375000in}}%
\pgfpathlineto{\pgfqpoint{3.600000in}{2.640000in}}%
\pgfpathlineto{\pgfqpoint{0.500000in}{2.640000in}}%
\pgfpathlineto{\pgfqpoint{0.500000in}{0.375000in}}%
\pgfpathclose%
\pgfusepath{fill}%
\end{pgfscope}%
\begin{pgfscope}%
\pgfsetbuttcap%
\pgfsetroundjoin%
\definecolor{currentfill}{rgb}{0.000000,0.000000,0.000000}%
\pgfsetfillcolor{currentfill}%
\pgfsetlinewidth{0.803000pt}%
\definecolor{currentstroke}{rgb}{0.000000,0.000000,0.000000}%
\pgfsetstrokecolor{currentstroke}%
\pgfsetdash{}{0pt}%
\pgfsys@defobject{currentmarker}{\pgfqpoint{0.000000in}{-0.048611in}}{\pgfqpoint{0.000000in}{0.000000in}}{%
\pgfpathmoveto{\pgfqpoint{0.000000in}{0.000000in}}%
\pgfpathlineto{\pgfqpoint{0.000000in}{-0.048611in}}%
\pgfusepath{stroke,fill}%
}%
\begin{pgfscope}%
\pgfsys@transformshift{0.640909in}{0.375000in}%
\pgfsys@useobject{currentmarker}{}%
\end{pgfscope}%
\end{pgfscope}%
\begin{pgfscope}%
\definecolor{textcolor}{rgb}{0.000000,0.000000,0.000000}%
\pgfsetstrokecolor{textcolor}%
\pgfsetfillcolor{textcolor}%
\pgftext[x=0.640909in,y=0.277778in,,top]{\color{textcolor}\rmfamily\fontsize{10.000000}{12.000000}\selectfont \(\displaystyle {0}\)}%
\end{pgfscope}%
\begin{pgfscope}%
\pgfsetbuttcap%
\pgfsetroundjoin%
\definecolor{currentfill}{rgb}{0.000000,0.000000,0.000000}%
\pgfsetfillcolor{currentfill}%
\pgfsetlinewidth{0.803000pt}%
\definecolor{currentstroke}{rgb}{0.000000,0.000000,0.000000}%
\pgfsetstrokecolor{currentstroke}%
\pgfsetdash{}{0pt}%
\pgfsys@defobject{currentmarker}{\pgfqpoint{0.000000in}{-0.048611in}}{\pgfqpoint{0.000000in}{0.000000in}}{%
\pgfpathmoveto{\pgfqpoint{0.000000in}{0.000000in}}%
\pgfpathlineto{\pgfqpoint{0.000000in}{-0.048611in}}%
\pgfusepath{stroke,fill}%
}%
\begin{pgfscope}%
\pgfsys@transformshift{1.139702in}{0.375000in}%
\pgfsys@useobject{currentmarker}{}%
\end{pgfscope}%
\end{pgfscope}%
\begin{pgfscope}%
\definecolor{textcolor}{rgb}{0.000000,0.000000,0.000000}%
\pgfsetstrokecolor{textcolor}%
\pgfsetfillcolor{textcolor}%
\pgftext[x=1.139702in,y=0.277778in,,top]{\color{textcolor}\rmfamily\fontsize{10.000000}{12.000000}\selectfont \(\displaystyle {100}\)}%
\end{pgfscope}%
\begin{pgfscope}%
\pgfsetbuttcap%
\pgfsetroundjoin%
\definecolor{currentfill}{rgb}{0.000000,0.000000,0.000000}%
\pgfsetfillcolor{currentfill}%
\pgfsetlinewidth{0.803000pt}%
\definecolor{currentstroke}{rgb}{0.000000,0.000000,0.000000}%
\pgfsetstrokecolor{currentstroke}%
\pgfsetdash{}{0pt}%
\pgfsys@defobject{currentmarker}{\pgfqpoint{0.000000in}{-0.048611in}}{\pgfqpoint{0.000000in}{0.000000in}}{%
\pgfpathmoveto{\pgfqpoint{0.000000in}{0.000000in}}%
\pgfpathlineto{\pgfqpoint{0.000000in}{-0.048611in}}%
\pgfusepath{stroke,fill}%
}%
\begin{pgfscope}%
\pgfsys@transformshift{1.638496in}{0.375000in}%
\pgfsys@useobject{currentmarker}{}%
\end{pgfscope}%
\end{pgfscope}%
\begin{pgfscope}%
\definecolor{textcolor}{rgb}{0.000000,0.000000,0.000000}%
\pgfsetstrokecolor{textcolor}%
\pgfsetfillcolor{textcolor}%
\pgftext[x=1.638496in,y=0.277778in,,top]{\color{textcolor}\rmfamily\fontsize{10.000000}{12.000000}\selectfont \(\displaystyle {200}\)}%
\end{pgfscope}%
\begin{pgfscope}%
\pgfsetbuttcap%
\pgfsetroundjoin%
\definecolor{currentfill}{rgb}{0.000000,0.000000,0.000000}%
\pgfsetfillcolor{currentfill}%
\pgfsetlinewidth{0.803000pt}%
\definecolor{currentstroke}{rgb}{0.000000,0.000000,0.000000}%
\pgfsetstrokecolor{currentstroke}%
\pgfsetdash{}{0pt}%
\pgfsys@defobject{currentmarker}{\pgfqpoint{0.000000in}{-0.048611in}}{\pgfqpoint{0.000000in}{0.000000in}}{%
\pgfpathmoveto{\pgfqpoint{0.000000in}{0.000000in}}%
\pgfpathlineto{\pgfqpoint{0.000000in}{-0.048611in}}%
\pgfusepath{stroke,fill}%
}%
\begin{pgfscope}%
\pgfsys@transformshift{2.137289in}{0.375000in}%
\pgfsys@useobject{currentmarker}{}%
\end{pgfscope}%
\end{pgfscope}%
\begin{pgfscope}%
\definecolor{textcolor}{rgb}{0.000000,0.000000,0.000000}%
\pgfsetstrokecolor{textcolor}%
\pgfsetfillcolor{textcolor}%
\pgftext[x=2.137289in,y=0.277778in,,top]{\color{textcolor}\rmfamily\fontsize{10.000000}{12.000000}\selectfont \(\displaystyle {300}\)}%
\end{pgfscope}%
\begin{pgfscope}%
\pgfsetbuttcap%
\pgfsetroundjoin%
\definecolor{currentfill}{rgb}{0.000000,0.000000,0.000000}%
\pgfsetfillcolor{currentfill}%
\pgfsetlinewidth{0.803000pt}%
\definecolor{currentstroke}{rgb}{0.000000,0.000000,0.000000}%
\pgfsetstrokecolor{currentstroke}%
\pgfsetdash{}{0pt}%
\pgfsys@defobject{currentmarker}{\pgfqpoint{0.000000in}{-0.048611in}}{\pgfqpoint{0.000000in}{0.000000in}}{%
\pgfpathmoveto{\pgfqpoint{0.000000in}{0.000000in}}%
\pgfpathlineto{\pgfqpoint{0.000000in}{-0.048611in}}%
\pgfusepath{stroke,fill}%
}%
\begin{pgfscope}%
\pgfsys@transformshift{2.636082in}{0.375000in}%
\pgfsys@useobject{currentmarker}{}%
\end{pgfscope}%
\end{pgfscope}%
\begin{pgfscope}%
\definecolor{textcolor}{rgb}{0.000000,0.000000,0.000000}%
\pgfsetstrokecolor{textcolor}%
\pgfsetfillcolor{textcolor}%
\pgftext[x=2.636082in,y=0.277778in,,top]{\color{textcolor}\rmfamily\fontsize{10.000000}{12.000000}\selectfont \(\displaystyle {400}\)}%
\end{pgfscope}%
\begin{pgfscope}%
\pgfsetbuttcap%
\pgfsetroundjoin%
\definecolor{currentfill}{rgb}{0.000000,0.000000,0.000000}%
\pgfsetfillcolor{currentfill}%
\pgfsetlinewidth{0.803000pt}%
\definecolor{currentstroke}{rgb}{0.000000,0.000000,0.000000}%
\pgfsetstrokecolor{currentstroke}%
\pgfsetdash{}{0pt}%
\pgfsys@defobject{currentmarker}{\pgfqpoint{0.000000in}{-0.048611in}}{\pgfqpoint{0.000000in}{0.000000in}}{%
\pgfpathmoveto{\pgfqpoint{0.000000in}{0.000000in}}%
\pgfpathlineto{\pgfqpoint{0.000000in}{-0.048611in}}%
\pgfusepath{stroke,fill}%
}%
\begin{pgfscope}%
\pgfsys@transformshift{3.134875in}{0.375000in}%
\pgfsys@useobject{currentmarker}{}%
\end{pgfscope}%
\end{pgfscope}%
\begin{pgfscope}%
\definecolor{textcolor}{rgb}{0.000000,0.000000,0.000000}%
\pgfsetstrokecolor{textcolor}%
\pgfsetfillcolor{textcolor}%
\pgftext[x=3.134875in,y=0.277778in,,top]{\color{textcolor}\rmfamily\fontsize{10.000000}{12.000000}\selectfont \(\displaystyle {500}\)}%
\end{pgfscope}%
\begin{pgfscope}%
\definecolor{textcolor}{rgb}{0.000000,0.000000,0.000000}%
\pgfsetstrokecolor{textcolor}%
\pgfsetfillcolor{textcolor}%
\pgftext[x=2.050000in,y=0.098766in,,top]{\color{textcolor}\rmfamily\fontsize{10.000000}{12.000000}\selectfont time}%
\end{pgfscope}%
\begin{pgfscope}%
\pgfsetbuttcap%
\pgfsetroundjoin%
\definecolor{currentfill}{rgb}{0.000000,0.000000,0.000000}%
\pgfsetfillcolor{currentfill}%
\pgfsetlinewidth{0.803000pt}%
\definecolor{currentstroke}{rgb}{0.000000,0.000000,0.000000}%
\pgfsetstrokecolor{currentstroke}%
\pgfsetdash{}{0pt}%
\pgfsys@defobject{currentmarker}{\pgfqpoint{-0.048611in}{0.000000in}}{\pgfqpoint{-0.000000in}{0.000000in}}{%
\pgfpathmoveto{\pgfqpoint{-0.000000in}{0.000000in}}%
\pgfpathlineto{\pgfqpoint{-0.048611in}{0.000000in}}%
\pgfusepath{stroke,fill}%
}%
\begin{pgfscope}%
\pgfsys@transformshift{0.500000in}{0.390021in}%
\pgfsys@useobject{currentmarker}{}%
\end{pgfscope}%
\end{pgfscope}%
\begin{pgfscope}%
\definecolor{textcolor}{rgb}{0.000000,0.000000,0.000000}%
\pgfsetstrokecolor{textcolor}%
\pgfsetfillcolor{textcolor}%
\pgftext[x=0.333333in, y=0.341795in, left, base]{\color{textcolor}\rmfamily\fontsize{10.000000}{12.000000}\selectfont \(\displaystyle {1}\)}%
\end{pgfscope}%
\begin{pgfscope}%
\pgfsetbuttcap%
\pgfsetroundjoin%
\definecolor{currentfill}{rgb}{0.000000,0.000000,0.000000}%
\pgfsetfillcolor{currentfill}%
\pgfsetlinewidth{0.803000pt}%
\definecolor{currentstroke}{rgb}{0.000000,0.000000,0.000000}%
\pgfsetstrokecolor{currentstroke}%
\pgfsetdash{}{0pt}%
\pgfsys@defobject{currentmarker}{\pgfqpoint{-0.048611in}{0.000000in}}{\pgfqpoint{-0.000000in}{0.000000in}}{%
\pgfpathmoveto{\pgfqpoint{-0.000000in}{0.000000in}}%
\pgfpathlineto{\pgfqpoint{-0.048611in}{0.000000in}}%
\pgfusepath{stroke,fill}%
}%
\begin{pgfscope}%
\pgfsys@transformshift{0.500000in}{0.772342in}%
\pgfsys@useobject{currentmarker}{}%
\end{pgfscope}%
\end{pgfscope}%
\begin{pgfscope}%
\definecolor{textcolor}{rgb}{0.000000,0.000000,0.000000}%
\pgfsetstrokecolor{textcolor}%
\pgfsetfillcolor{textcolor}%
\pgftext[x=0.333333in, y=0.724117in, left, base]{\color{textcolor}\rmfamily\fontsize{10.000000}{12.000000}\selectfont \(\displaystyle {2}\)}%
\end{pgfscope}%
\begin{pgfscope}%
\pgfsetbuttcap%
\pgfsetroundjoin%
\definecolor{currentfill}{rgb}{0.000000,0.000000,0.000000}%
\pgfsetfillcolor{currentfill}%
\pgfsetlinewidth{0.803000pt}%
\definecolor{currentstroke}{rgb}{0.000000,0.000000,0.000000}%
\pgfsetstrokecolor{currentstroke}%
\pgfsetdash{}{0pt}%
\pgfsys@defobject{currentmarker}{\pgfqpoint{-0.048611in}{0.000000in}}{\pgfqpoint{-0.000000in}{0.000000in}}{%
\pgfpathmoveto{\pgfqpoint{-0.000000in}{0.000000in}}%
\pgfpathlineto{\pgfqpoint{-0.048611in}{0.000000in}}%
\pgfusepath{stroke,fill}%
}%
\begin{pgfscope}%
\pgfsys@transformshift{0.500000in}{1.154664in}%
\pgfsys@useobject{currentmarker}{}%
\end{pgfscope}%
\end{pgfscope}%
\begin{pgfscope}%
\definecolor{textcolor}{rgb}{0.000000,0.000000,0.000000}%
\pgfsetstrokecolor{textcolor}%
\pgfsetfillcolor{textcolor}%
\pgftext[x=0.333333in, y=1.106438in, left, base]{\color{textcolor}\rmfamily\fontsize{10.000000}{12.000000}\selectfont \(\displaystyle {3}\)}%
\end{pgfscope}%
\begin{pgfscope}%
\pgfsetbuttcap%
\pgfsetroundjoin%
\definecolor{currentfill}{rgb}{0.000000,0.000000,0.000000}%
\pgfsetfillcolor{currentfill}%
\pgfsetlinewidth{0.803000pt}%
\definecolor{currentstroke}{rgb}{0.000000,0.000000,0.000000}%
\pgfsetstrokecolor{currentstroke}%
\pgfsetdash{}{0pt}%
\pgfsys@defobject{currentmarker}{\pgfqpoint{-0.048611in}{0.000000in}}{\pgfqpoint{-0.000000in}{0.000000in}}{%
\pgfpathmoveto{\pgfqpoint{-0.000000in}{0.000000in}}%
\pgfpathlineto{\pgfqpoint{-0.048611in}{0.000000in}}%
\pgfusepath{stroke,fill}%
}%
\begin{pgfscope}%
\pgfsys@transformshift{0.500000in}{1.536985in}%
\pgfsys@useobject{currentmarker}{}%
\end{pgfscope}%
\end{pgfscope}%
\begin{pgfscope}%
\definecolor{textcolor}{rgb}{0.000000,0.000000,0.000000}%
\pgfsetstrokecolor{textcolor}%
\pgfsetfillcolor{textcolor}%
\pgftext[x=0.333333in, y=1.488760in, left, base]{\color{textcolor}\rmfamily\fontsize{10.000000}{12.000000}\selectfont \(\displaystyle {4}\)}%
\end{pgfscope}%
\begin{pgfscope}%
\pgfsetbuttcap%
\pgfsetroundjoin%
\definecolor{currentfill}{rgb}{0.000000,0.000000,0.000000}%
\pgfsetfillcolor{currentfill}%
\pgfsetlinewidth{0.803000pt}%
\definecolor{currentstroke}{rgb}{0.000000,0.000000,0.000000}%
\pgfsetstrokecolor{currentstroke}%
\pgfsetdash{}{0pt}%
\pgfsys@defobject{currentmarker}{\pgfqpoint{-0.048611in}{0.000000in}}{\pgfqpoint{-0.000000in}{0.000000in}}{%
\pgfpathmoveto{\pgfqpoint{-0.000000in}{0.000000in}}%
\pgfpathlineto{\pgfqpoint{-0.048611in}{0.000000in}}%
\pgfusepath{stroke,fill}%
}%
\begin{pgfscope}%
\pgfsys@transformshift{0.500000in}{1.919307in}%
\pgfsys@useobject{currentmarker}{}%
\end{pgfscope}%
\end{pgfscope}%
\begin{pgfscope}%
\definecolor{textcolor}{rgb}{0.000000,0.000000,0.000000}%
\pgfsetstrokecolor{textcolor}%
\pgfsetfillcolor{textcolor}%
\pgftext[x=0.333333in, y=1.871081in, left, base]{\color{textcolor}\rmfamily\fontsize{10.000000}{12.000000}\selectfont \(\displaystyle {5}\)}%
\end{pgfscope}%
\begin{pgfscope}%
\pgfsetbuttcap%
\pgfsetroundjoin%
\definecolor{currentfill}{rgb}{0.000000,0.000000,0.000000}%
\pgfsetfillcolor{currentfill}%
\pgfsetlinewidth{0.803000pt}%
\definecolor{currentstroke}{rgb}{0.000000,0.000000,0.000000}%
\pgfsetstrokecolor{currentstroke}%
\pgfsetdash{}{0pt}%
\pgfsys@defobject{currentmarker}{\pgfqpoint{-0.048611in}{0.000000in}}{\pgfqpoint{-0.000000in}{0.000000in}}{%
\pgfpathmoveto{\pgfqpoint{-0.000000in}{0.000000in}}%
\pgfpathlineto{\pgfqpoint{-0.048611in}{0.000000in}}%
\pgfusepath{stroke,fill}%
}%
\begin{pgfscope}%
\pgfsys@transformshift{0.500000in}{2.301628in}%
\pgfsys@useobject{currentmarker}{}%
\end{pgfscope}%
\end{pgfscope}%
\begin{pgfscope}%
\definecolor{textcolor}{rgb}{0.000000,0.000000,0.000000}%
\pgfsetstrokecolor{textcolor}%
\pgfsetfillcolor{textcolor}%
\pgftext[x=0.333333in, y=2.253403in, left, base]{\color{textcolor}\rmfamily\fontsize{10.000000}{12.000000}\selectfont \(\displaystyle {6}\)}%
\end{pgfscope}%
\begin{pgfscope}%
\definecolor{textcolor}{rgb}{0.000000,0.000000,0.000000}%
\pgfsetstrokecolor{textcolor}%
\pgfsetfillcolor{textcolor}%
\pgftext[x=0.500000in,y=2.681667in,left,base]{\color{textcolor}\rmfamily\fontsize{10.000000}{12.000000}\selectfont \(\displaystyle \times{10^{6}}{}\)}%
\end{pgfscope}%
\begin{pgfscope}%
\pgfpathrectangle{\pgfqpoint{0.500000in}{0.375000in}}{\pgfqpoint{3.100000in}{2.265000in}}%
\pgfusepath{clip}%
\pgfsetrectcap%
\pgfsetroundjoin%
\pgfsetlinewidth{1.505625pt}%
\definecolor{currentstroke}{rgb}{0.121569,0.466667,0.705882}%
\pgfsetstrokecolor{currentstroke}%
\pgfsetdash{}{0pt}%
\pgfpathmoveto{\pgfqpoint{0.640909in}{1.553704in}}%
\pgfpathlineto{\pgfqpoint{0.845414in}{1.554780in}}%
\pgfpathlineto{\pgfqpoint{0.855390in}{1.558480in}}%
\pgfpathlineto{\pgfqpoint{0.865366in}{1.565979in}}%
\pgfpathlineto{\pgfqpoint{0.875342in}{1.576893in}}%
\pgfpathlineto{\pgfqpoint{0.890306in}{1.599476in}}%
\pgfpathlineto{\pgfqpoint{0.905270in}{1.629165in}}%
\pgfpathlineto{\pgfqpoint{0.915245in}{1.658288in}}%
\pgfpathlineto{\pgfqpoint{0.925221in}{1.699238in}}%
\pgfpathlineto{\pgfqpoint{0.935197in}{1.756415in}}%
\pgfpathlineto{\pgfqpoint{0.945173in}{1.836638in}}%
\pgfpathlineto{\pgfqpoint{0.955149in}{1.950501in}}%
\pgfpathlineto{\pgfqpoint{0.975101in}{2.273397in}}%
\pgfpathlineto{\pgfqpoint{0.985076in}{2.398763in}}%
\pgfpathlineto{\pgfqpoint{0.995052in}{2.485218in}}%
\pgfpathlineto{\pgfqpoint{1.000040in}{2.505614in}}%
\pgfpathlineto{\pgfqpoint{1.010016in}{2.506554in}}%
\pgfpathlineto{\pgfqpoint{1.019992in}{2.506828in}}%
\pgfpathlineto{\pgfqpoint{1.024980in}{2.504497in}}%
\pgfpathlineto{\pgfqpoint{1.029968in}{2.496419in}}%
\pgfpathlineto{\pgfqpoint{1.034956in}{2.482704in}}%
\pgfpathlineto{\pgfqpoint{1.039944in}{2.453378in}}%
\pgfpathlineto{\pgfqpoint{1.044932in}{2.403670in}}%
\pgfpathlineto{\pgfqpoint{1.054907in}{2.203420in}}%
\pgfpathlineto{\pgfqpoint{1.069871in}{1.647748in}}%
\pgfpathlineto{\pgfqpoint{1.084835in}{1.105149in}}%
\pgfpathlineto{\pgfqpoint{1.094811in}{0.875253in}}%
\pgfpathlineto{\pgfqpoint{1.099799in}{0.818087in}}%
\pgfpathlineto{\pgfqpoint{1.104787in}{0.802657in}}%
\pgfpathlineto{\pgfqpoint{1.109775in}{0.828511in}}%
\pgfpathlineto{\pgfqpoint{1.114763in}{0.892972in}}%
\pgfpathlineto{\pgfqpoint{1.124739in}{1.117207in}}%
\pgfpathlineto{\pgfqpoint{1.154666in}{1.985522in}}%
\pgfpathlineto{\pgfqpoint{1.159654in}{2.074539in}}%
\pgfpathlineto{\pgfqpoint{1.164642in}{2.095340in}}%
\pgfpathlineto{\pgfqpoint{1.169630in}{2.095340in}}%
\pgfpathlineto{\pgfqpoint{1.174618in}{2.110854in}}%
\pgfpathlineto{\pgfqpoint{1.179606in}{2.136646in}}%
\pgfpathlineto{\pgfqpoint{1.184594in}{2.151894in}}%
\pgfpathlineto{\pgfqpoint{1.189582in}{2.154824in}}%
\pgfpathlineto{\pgfqpoint{1.194570in}{2.125976in}}%
\pgfpathlineto{\pgfqpoint{1.199558in}{2.064297in}}%
\pgfpathlineto{\pgfqpoint{1.209533in}{1.863286in}}%
\pgfpathlineto{\pgfqpoint{1.234473in}{1.248587in}}%
\pgfpathlineto{\pgfqpoint{1.244449in}{1.098971in}}%
\pgfpathlineto{\pgfqpoint{1.249437in}{1.075307in}}%
\pgfpathlineto{\pgfqpoint{1.254425in}{1.094546in}}%
\pgfpathlineto{\pgfqpoint{1.259413in}{1.153279in}}%
\pgfpathlineto{\pgfqpoint{1.269389in}{1.359588in}}%
\pgfpathlineto{\pgfqpoint{1.284352in}{1.731384in}}%
\pgfpathlineto{\pgfqpoint{1.294328in}{1.896013in}}%
\pgfpathlineto{\pgfqpoint{1.299316in}{1.931259in}}%
\pgfpathlineto{\pgfqpoint{1.304304in}{1.931855in}}%
\pgfpathlineto{\pgfqpoint{1.309292in}{1.899244in}}%
\pgfpathlineto{\pgfqpoint{1.314280in}{1.837703in}}%
\pgfpathlineto{\pgfqpoint{1.324256in}{1.655928in}}%
\pgfpathlineto{\pgfqpoint{1.339220in}{1.368520in}}%
\pgfpathlineto{\pgfqpoint{1.344208in}{1.302199in}}%
\pgfpathlineto{\pgfqpoint{1.349195in}{1.260497in}}%
\pgfpathlineto{\pgfqpoint{1.354183in}{1.246050in}}%
\pgfpathlineto{\pgfqpoint{1.359171in}{1.259009in}}%
\pgfpathlineto{\pgfqpoint{1.364159in}{1.297112in}}%
\pgfpathlineto{\pgfqpoint{1.374135in}{1.429554in}}%
\pgfpathlineto{\pgfqpoint{1.384111in}{1.592113in}}%
\pgfpathlineto{\pgfqpoint{1.389099in}{1.610738in}}%
\pgfpathlineto{\pgfqpoint{1.399075in}{1.610738in}}%
\pgfpathlineto{\pgfqpoint{1.404063in}{1.634110in}}%
\pgfpathlineto{\pgfqpoint{1.419027in}{1.764681in}}%
\pgfpathlineto{\pgfqpoint{1.424014in}{1.789982in}}%
\pgfpathlineto{\pgfqpoint{1.429002in}{1.791275in}}%
\pgfpathlineto{\pgfqpoint{1.433990in}{1.769496in}}%
\pgfpathlineto{\pgfqpoint{1.438978in}{1.727739in}}%
\pgfpathlineto{\pgfqpoint{1.443966in}{1.670863in}}%
\pgfpathlineto{\pgfqpoint{1.448954in}{1.590781in}}%
\pgfpathlineto{\pgfqpoint{1.453942in}{1.545951in}}%
\pgfpathlineto{\pgfqpoint{1.463918in}{1.545951in}}%
\pgfpathlineto{\pgfqpoint{1.468906in}{1.541449in}}%
\pgfpathlineto{\pgfqpoint{1.483870in}{1.511055in}}%
\pgfpathlineto{\pgfqpoint{1.488858in}{1.495216in}}%
\pgfpathlineto{\pgfqpoint{1.493846in}{1.472661in}}%
\pgfpathlineto{\pgfqpoint{1.498833in}{1.436404in}}%
\pgfpathlineto{\pgfqpoint{1.503821in}{1.371518in}}%
\pgfpathlineto{\pgfqpoint{1.518785in}{0.984506in}}%
\pgfpathlineto{\pgfqpoint{1.533749in}{0.626058in}}%
\pgfpathlineto{\pgfqpoint{1.543725in}{0.489435in}}%
\pgfpathlineto{\pgfqpoint{1.548713in}{0.477955in}}%
\pgfpathlineto{\pgfqpoint{1.578640in}{0.480160in}}%
\pgfpathlineto{\pgfqpoint{1.588616in}{0.482802in}}%
\pgfpathlineto{\pgfqpoint{1.593604in}{0.482802in}}%
\pgfpathlineto{\pgfqpoint{1.598592in}{0.486460in}}%
\pgfpathlineto{\pgfqpoint{1.608568in}{0.487363in}}%
\pgfpathlineto{\pgfqpoint{1.613556in}{0.493420in}}%
\pgfpathlineto{\pgfqpoint{1.623532in}{0.494588in}}%
\pgfpathlineto{\pgfqpoint{1.628520in}{0.503857in}}%
\pgfpathlineto{\pgfqpoint{1.638496in}{0.505330in}}%
\pgfpathlineto{\pgfqpoint{1.643484in}{0.518720in}}%
\pgfpathlineto{\pgfqpoint{1.648471in}{0.520533in}}%
\pgfpathlineto{\pgfqpoint{1.653459in}{0.520533in}}%
\pgfpathlineto{\pgfqpoint{1.658447in}{0.539028in}}%
\pgfpathlineto{\pgfqpoint{1.663435in}{0.541212in}}%
\pgfpathlineto{\pgfqpoint{1.668423in}{0.541212in}}%
\pgfpathlineto{\pgfqpoint{1.673411in}{0.563602in}}%
\pgfpathlineto{\pgfqpoint{1.678399in}{0.566168in}}%
\pgfpathlineto{\pgfqpoint{1.683387in}{0.566168in}}%
\pgfpathlineto{\pgfqpoint{1.688375in}{0.592286in}}%
\pgfpathlineto{\pgfqpoint{1.693363in}{0.595228in}}%
\pgfpathlineto{\pgfqpoint{1.698351in}{0.595228in}}%
\pgfpathlineto{\pgfqpoint{1.703339in}{0.625192in}}%
\pgfpathlineto{\pgfqpoint{1.708327in}{0.628502in}}%
\pgfpathlineto{\pgfqpoint{1.713315in}{0.628502in}}%
\pgfpathlineto{\pgfqpoint{1.718302in}{0.662464in}}%
\pgfpathlineto{\pgfqpoint{1.723290in}{0.666132in}}%
\pgfpathlineto{\pgfqpoint{1.728278in}{0.666132in}}%
\pgfpathlineto{\pgfqpoint{1.733266in}{0.704288in}}%
\pgfpathlineto{\pgfqpoint{1.738254in}{0.708303in}}%
\pgfpathlineto{\pgfqpoint{1.743242in}{0.708303in}}%
\pgfpathlineto{\pgfqpoint{1.748230in}{0.750909in}}%
\pgfpathlineto{\pgfqpoint{1.753218in}{0.755258in}}%
\pgfpathlineto{\pgfqpoint{1.758206in}{0.755258in}}%
\pgfpathlineto{\pgfqpoint{1.763194in}{0.801801in}}%
\pgfpathlineto{\pgfqpoint{1.768182in}{0.806469in}}%
\pgfpathlineto{\pgfqpoint{1.773170in}{0.806469in}}%
\pgfpathlineto{\pgfqpoint{1.778158in}{0.854374in}}%
\pgfpathlineto{\pgfqpoint{1.783146in}{0.859342in}}%
\pgfpathlineto{\pgfqpoint{1.788134in}{0.859342in}}%
\pgfpathlineto{\pgfqpoint{1.793121in}{0.908503in}}%
\pgfpathlineto{\pgfqpoint{1.798109in}{0.913742in}}%
\pgfpathlineto{\pgfqpoint{1.803097in}{0.913742in}}%
\pgfpathlineto{\pgfqpoint{1.808085in}{0.963999in}}%
\pgfpathlineto{\pgfqpoint{1.813073in}{0.969480in}}%
\pgfpathlineto{\pgfqpoint{1.818061in}{0.969480in}}%
\pgfpathlineto{\pgfqpoint{1.823049in}{1.020687in}}%
\pgfpathlineto{\pgfqpoint{1.828037in}{1.126137in}}%
\pgfpathlineto{\pgfqpoint{1.833025in}{1.137656in}}%
\pgfpathlineto{\pgfqpoint{1.838013in}{1.137656in}}%
\pgfpathlineto{\pgfqpoint{1.843001in}{1.186036in}}%
\pgfpathlineto{\pgfqpoint{1.852977in}{1.385083in}}%
\pgfpathlineto{\pgfqpoint{1.857965in}{1.541154in}}%
\pgfpathlineto{\pgfqpoint{1.862953in}{1.555644in}}%
\pgfpathlineto{\pgfqpoint{1.867940in}{1.555644in}}%
\pgfpathlineto{\pgfqpoint{1.872928in}{1.604870in}}%
\pgfpathlineto{\pgfqpoint{1.877916in}{1.611487in}}%
\pgfpathlineto{\pgfqpoint{1.882904in}{1.611487in}}%
\pgfpathlineto{\pgfqpoint{1.887892in}{1.666136in}}%
\pgfpathlineto{\pgfqpoint{1.892880in}{1.672655in}}%
\pgfpathlineto{\pgfqpoint{1.897868in}{1.672655in}}%
\pgfpathlineto{\pgfqpoint{1.902856in}{1.727037in}}%
\pgfpathlineto{\pgfqpoint{1.907844in}{1.733483in}}%
\pgfpathlineto{\pgfqpoint{1.912832in}{1.733483in}}%
\pgfpathlineto{\pgfqpoint{1.917820in}{1.787471in}}%
\pgfpathlineto{\pgfqpoint{1.922808in}{1.793820in}}%
\pgfpathlineto{\pgfqpoint{1.927796in}{1.793820in}}%
\pgfpathlineto{\pgfqpoint{1.932784in}{1.847309in}}%
\pgfpathlineto{\pgfqpoint{1.937772in}{1.853537in}}%
\pgfpathlineto{\pgfqpoint{1.942759in}{1.853537in}}%
\pgfpathlineto{\pgfqpoint{1.947747in}{1.906416in}}%
\pgfpathlineto{\pgfqpoint{1.952735in}{1.912498in}}%
\pgfpathlineto{\pgfqpoint{1.957723in}{1.912498in}}%
\pgfpathlineto{\pgfqpoint{1.962711in}{1.964653in}}%
\pgfpathlineto{\pgfqpoint{1.967699in}{1.970563in}}%
\pgfpathlineto{\pgfqpoint{1.972687in}{1.970563in}}%
\pgfpathlineto{\pgfqpoint{1.977675in}{2.021869in}}%
\pgfpathlineto{\pgfqpoint{1.982663in}{2.027582in}}%
\pgfpathlineto{\pgfqpoint{1.987651in}{2.027582in}}%
\pgfpathlineto{\pgfqpoint{1.992639in}{2.077906in}}%
\pgfpathlineto{\pgfqpoint{1.997627in}{2.083395in}}%
\pgfpathlineto{\pgfqpoint{2.002615in}{2.083395in}}%
\pgfpathlineto{\pgfqpoint{2.007603in}{2.132590in}}%
\pgfpathlineto{\pgfqpoint{2.012591in}{2.137827in}}%
\pgfpathlineto{\pgfqpoint{2.017578in}{2.137827in}}%
\pgfpathlineto{\pgfqpoint{2.022566in}{2.185729in}}%
\pgfpathlineto{\pgfqpoint{2.027554in}{2.190687in}}%
\pgfpathlineto{\pgfqpoint{2.032542in}{2.190687in}}%
\pgfpathlineto{\pgfqpoint{2.037530in}{2.237109in}}%
\pgfpathlineto{\pgfqpoint{2.042518in}{2.241756in}}%
\pgfpathlineto{\pgfqpoint{2.057482in}{2.241756in}}%
\pgfpathlineto{\pgfqpoint{2.062470in}{2.286034in}}%
\pgfpathlineto{\pgfqpoint{2.067458in}{2.292856in}}%
\pgfpathlineto{\pgfqpoint{2.072446in}{2.292856in}}%
\pgfpathlineto{\pgfqpoint{2.077434in}{2.332273in}}%
\pgfpathlineto{\pgfqpoint{2.082422in}{2.336233in}}%
\pgfpathlineto{\pgfqpoint{2.087409in}{2.336233in}}%
\pgfpathlineto{\pgfqpoint{2.092397in}{2.371289in}}%
\pgfpathlineto{\pgfqpoint{2.097385in}{2.374905in}}%
\pgfpathlineto{\pgfqpoint{2.102373in}{2.374905in}}%
\pgfpathlineto{\pgfqpoint{2.107361in}{2.405877in}}%
\pgfpathlineto{\pgfqpoint{2.112349in}{2.409138in}}%
\pgfpathlineto{\pgfqpoint{2.117337in}{2.409138in}}%
\pgfpathlineto{\pgfqpoint{2.122325in}{2.436198in}}%
\pgfpathlineto{\pgfqpoint{2.127313in}{2.439096in}}%
\pgfpathlineto{\pgfqpoint{2.132301in}{2.439096in}}%
\pgfpathlineto{\pgfqpoint{2.137289in}{2.462379in}}%
\pgfpathlineto{\pgfqpoint{2.142277in}{2.464908in}}%
\pgfpathlineto{\pgfqpoint{2.147265in}{2.464908in}}%
\pgfpathlineto{\pgfqpoint{2.152253in}{2.484522in}}%
\pgfpathlineto{\pgfqpoint{2.157241in}{2.486676in}}%
\pgfpathlineto{\pgfqpoint{2.162228in}{2.486676in}}%
\pgfpathlineto{\pgfqpoint{2.167216in}{2.502734in}}%
\pgfpathlineto{\pgfqpoint{2.172204in}{2.504507in}}%
\pgfpathlineto{\pgfqpoint{2.177192in}{2.504507in}}%
\pgfpathlineto{\pgfqpoint{2.182180in}{2.517053in}}%
\pgfpathlineto{\pgfqpoint{2.192156in}{2.518440in}}%
\pgfpathlineto{\pgfqpoint{2.197144in}{2.527498in}}%
\pgfpathlineto{\pgfqpoint{2.207120in}{2.528497in}}%
\pgfpathlineto{\pgfqpoint{2.212108in}{2.534084in}}%
\pgfpathlineto{\pgfqpoint{2.237047in}{2.537045in}}%
\pgfpathlineto{\pgfqpoint{2.247023in}{2.535565in}}%
\pgfpathlineto{\pgfqpoint{2.252011in}{2.535565in}}%
\pgfpathlineto{\pgfqpoint{2.256999in}{2.530812in}}%
\pgfpathlineto{\pgfqpoint{2.266975in}{2.530251in}}%
\pgfpathlineto{\pgfqpoint{2.271963in}{2.522036in}}%
\pgfpathlineto{\pgfqpoint{2.281939in}{2.521088in}}%
\pgfpathlineto{\pgfqpoint{2.286927in}{2.509381in}}%
\pgfpathlineto{\pgfqpoint{2.296903in}{2.508050in}}%
\pgfpathlineto{\pgfqpoint{2.301891in}{2.492809in}}%
\pgfpathlineto{\pgfqpoint{2.306879in}{2.491098in}}%
\pgfpathlineto{\pgfqpoint{2.311866in}{2.491098in}}%
\pgfpathlineto{\pgfqpoint{2.316854in}{2.472261in}}%
\pgfpathlineto{\pgfqpoint{2.321842in}{2.470176in}}%
\pgfpathlineto{\pgfqpoint{2.326830in}{2.470176in}}%
\pgfpathlineto{\pgfqpoint{2.331818in}{2.447662in}}%
\pgfpathlineto{\pgfqpoint{2.336806in}{2.445209in}}%
\pgfpathlineto{\pgfqpoint{2.341794in}{2.445209in}}%
\pgfpathlineto{\pgfqpoint{2.346782in}{2.418912in}}%
\pgfpathlineto{\pgfqpoint{2.356758in}{2.290369in}}%
\pgfpathlineto{\pgfqpoint{2.361746in}{2.164095in}}%
\pgfpathlineto{\pgfqpoint{2.376710in}{1.542399in}}%
\pgfpathlineto{\pgfqpoint{2.391673in}{0.922770in}}%
\pgfpathlineto{\pgfqpoint{2.401649in}{0.653787in}}%
\pgfpathlineto{\pgfqpoint{2.406637in}{0.615242in}}%
\pgfpathlineto{\pgfqpoint{2.416613in}{0.615242in}}%
\pgfpathlineto{\pgfqpoint{2.421601in}{0.601985in}}%
\pgfpathlineto{\pgfqpoint{2.426589in}{0.580835in}}%
\pgfpathlineto{\pgfqpoint{2.431577in}{0.570497in}}%
\pgfpathlineto{\pgfqpoint{2.436565in}{0.575667in}}%
\pgfpathlineto{\pgfqpoint{2.441553in}{0.620168in}}%
\pgfpathlineto{\pgfqpoint{2.446541in}{0.705544in}}%
\pgfpathlineto{\pgfqpoint{2.456516in}{0.982091in}}%
\pgfpathlineto{\pgfqpoint{2.471480in}{1.560555in}}%
\pgfpathlineto{\pgfqpoint{2.486444in}{2.131191in}}%
\pgfpathlineto{\pgfqpoint{2.496420in}{2.395322in}}%
\pgfpathlineto{\pgfqpoint{2.501408in}{2.398856in}}%
\pgfpathlineto{\pgfqpoint{2.506396in}{2.398856in}}%
\pgfpathlineto{\pgfqpoint{2.511384in}{2.420956in}}%
\pgfpathlineto{\pgfqpoint{2.521360in}{2.488105in}}%
\pgfpathlineto{\pgfqpoint{2.526348in}{2.509659in}}%
\pgfpathlineto{\pgfqpoint{2.531335in}{2.499387in}}%
\pgfpathlineto{\pgfqpoint{2.536323in}{2.445915in}}%
\pgfpathlineto{\pgfqpoint{2.541311in}{2.351710in}}%
\pgfpathlineto{\pgfqpoint{2.551287in}{2.059944in}}%
\pgfpathlineto{\pgfqpoint{2.571239in}{1.269852in}}%
\pgfpathlineto{\pgfqpoint{2.581215in}{0.912715in}}%
\pgfpathlineto{\pgfqpoint{2.591191in}{0.668766in}}%
\pgfpathlineto{\pgfqpoint{2.596179in}{0.603864in}}%
\pgfpathlineto{\pgfqpoint{2.601167in}{0.581505in}}%
\pgfpathlineto{\pgfqpoint{2.606154in}{0.602686in}}%
\pgfpathlineto{\pgfqpoint{2.611142in}{0.666436in}}%
\pgfpathlineto{\pgfqpoint{2.616130in}{0.769854in}}%
\pgfpathlineto{\pgfqpoint{2.626106in}{1.075315in}}%
\pgfpathlineto{\pgfqpoint{2.661022in}{2.340972in}}%
\pgfpathlineto{\pgfqpoint{2.666010in}{2.434918in}}%
\pgfpathlineto{\pgfqpoint{2.670998in}{2.488455in}}%
\pgfpathlineto{\pgfqpoint{2.675986in}{2.499210in}}%
\pgfpathlineto{\pgfqpoint{2.680973in}{2.466747in}}%
\pgfpathlineto{\pgfqpoint{2.685961in}{2.392586in}}%
\pgfpathlineto{\pgfqpoint{2.695937in}{2.134483in}}%
\pgfpathlineto{\pgfqpoint{2.710901in}{1.570186in}}%
\pgfpathlineto{\pgfqpoint{2.725865in}{0.996704in}}%
\pgfpathlineto{\pgfqpoint{2.735841in}{0.723582in}}%
\pgfpathlineto{\pgfqpoint{2.740829in}{0.640020in}}%
\pgfpathlineto{\pgfqpoint{2.745817in}{0.597370in}}%
\pgfpathlineto{\pgfqpoint{2.750805in}{0.597549in}}%
\pgfpathlineto{\pgfqpoint{2.755792in}{0.640537in}}%
\pgfpathlineto{\pgfqpoint{2.760780in}{0.724372in}}%
\pgfpathlineto{\pgfqpoint{2.770756in}{0.997664in}}%
\pgfpathlineto{\pgfqpoint{2.785720in}{1.569775in}}%
\pgfpathlineto{\pgfqpoint{2.790708in}{1.769874in}}%
\pgfpathlineto{\pgfqpoint{2.795696in}{1.878935in}}%
\pgfpathlineto{\pgfqpoint{2.805672in}{1.878935in}}%
\pgfpathlineto{\pgfqpoint{2.810660in}{1.922845in}}%
\pgfpathlineto{\pgfqpoint{2.820636in}{2.089341in}}%
\pgfpathlineto{\pgfqpoint{2.835599in}{2.439917in}}%
\pgfpathlineto{\pgfqpoint{2.840587in}{2.484778in}}%
\pgfpathlineto{\pgfqpoint{2.845575in}{2.480231in}}%
\pgfpathlineto{\pgfqpoint{2.850563in}{2.426577in}}%
\pgfpathlineto{\pgfqpoint{2.855551in}{2.326690in}}%
\pgfpathlineto{\pgfqpoint{2.865527in}{2.011506in}}%
\pgfpathlineto{\pgfqpoint{2.895455in}{0.831713in}}%
\pgfpathlineto{\pgfqpoint{2.900442in}{0.710668in}}%
\pgfpathlineto{\pgfqpoint{2.905430in}{0.633381in}}%
\pgfpathlineto{\pgfqpoint{2.910418in}{0.603888in}}%
\pgfpathlineto{\pgfqpoint{2.915406in}{0.623722in}}%
\pgfpathlineto{\pgfqpoint{2.920394in}{0.691823in}}%
\pgfpathlineto{\pgfqpoint{2.925382in}{0.804599in}}%
\pgfpathlineto{\pgfqpoint{2.935358in}{1.138365in}}%
\pgfpathlineto{\pgfqpoint{2.960298in}{2.148256in}}%
\pgfpathlineto{\pgfqpoint{2.970274in}{2.401880in}}%
\pgfpathlineto{\pgfqpoint{2.975261in}{2.463770in}}%
\pgfpathlineto{\pgfqpoint{2.980249in}{2.477332in}}%
\pgfpathlineto{\pgfqpoint{2.985237in}{2.441916in}}%
\pgfpathlineto{\pgfqpoint{2.990225in}{2.359440in}}%
\pgfpathlineto{\pgfqpoint{3.000201in}{2.073052in}}%
\pgfpathlineto{\pgfqpoint{3.015165in}{1.464370in}}%
\pgfpathlineto{\pgfqpoint{3.030129in}{0.893199in}}%
\pgfpathlineto{\pgfqpoint{3.040105in}{0.666112in}}%
\pgfpathlineto{\pgfqpoint{3.045093in}{0.619265in}}%
\pgfpathlineto{\pgfqpoint{3.050080in}{0.620943in}}%
\pgfpathlineto{\pgfqpoint{3.055068in}{0.671041in}}%
\pgfpathlineto{\pgfqpoint{3.060056in}{0.766912in}}%
\pgfpathlineto{\pgfqpoint{3.070032in}{1.073615in}}%
\pgfpathlineto{\pgfqpoint{3.099960in}{2.237527in}}%
\pgfpathlineto{\pgfqpoint{3.104948in}{2.358519in}}%
\pgfpathlineto{\pgfqpoint{3.109936in}{2.436631in}}%
\pgfpathlineto{\pgfqpoint{3.114924in}{2.467818in}}%
\pgfpathlineto{\pgfqpoint{3.119912in}{2.450503in}}%
\pgfpathlineto{\pgfqpoint{3.124899in}{2.385656in}}%
\pgfpathlineto{\pgfqpoint{3.129887in}{2.276734in}}%
\pgfpathlineto{\pgfqpoint{3.139863in}{1.951693in}}%
\pgfpathlineto{\pgfqpoint{3.169791in}{0.811980in}}%
\pgfpathlineto{\pgfqpoint{3.174779in}{0.704410in}}%
\pgfpathlineto{\pgfqpoint{3.179767in}{0.640920in}}%
\pgfpathlineto{\pgfqpoint{3.184755in}{0.624824in}}%
\pgfpathlineto{\pgfqpoint{3.189743in}{0.656951in}}%
\pgfpathlineto{\pgfqpoint{3.194730in}{0.735597in}}%
\pgfpathlineto{\pgfqpoint{3.204706in}{1.013626in}}%
\pgfpathlineto{\pgfqpoint{3.219670in}{1.611165in}}%
\pgfpathlineto{\pgfqpoint{3.234634in}{2.176467in}}%
\pgfpathlineto{\pgfqpoint{3.244610in}{2.403686in}}%
\pgfpathlineto{\pgfqpoint{3.249598in}{2.451898in}}%
\pgfpathlineto{\pgfqpoint{3.254586in}{2.452423in}}%
\pgfpathlineto{\pgfqpoint{3.259574in}{2.405292in}}%
\pgfpathlineto{\pgfqpoint{3.264562in}{2.313035in}}%
\pgfpathlineto{\pgfqpoint{3.274537in}{2.014794in}}%
\pgfpathlineto{\pgfqpoint{3.304465in}{0.869006in}}%
\pgfpathlineto{\pgfqpoint{3.309453in}{0.747881in}}%
\pgfpathlineto{\pgfqpoint{3.314441in}{0.668562in}}%
\pgfpathlineto{\pgfqpoint{3.319429in}{0.635193in}}%
\pgfpathlineto{\pgfqpoint{3.324417in}{0.649509in}}%
\pgfpathlineto{\pgfqpoint{3.329405in}{0.710741in}}%
\pgfpathlineto{\pgfqpoint{3.334393in}{0.815656in}}%
\pgfpathlineto{\pgfqpoint{3.344368in}{1.132413in}}%
\pgfpathlineto{\pgfqpoint{3.374296in}{2.257545in}}%
\pgfpathlineto{\pgfqpoint{3.379284in}{2.365294in}}%
\pgfpathlineto{\pgfqpoint{3.384272in}{2.429824in}}%
\pgfpathlineto{\pgfqpoint{3.389260in}{2.447803in}}%
\pgfpathlineto{\pgfqpoint{3.394248in}{2.418346in}}%
\pgfpathlineto{\pgfqpoint{3.399236in}{2.343058in}}%
\pgfpathlineto{\pgfqpoint{3.409212in}{2.073174in}}%
\pgfpathlineto{\pgfqpoint{3.424175in}{1.488053in}}%
\pgfpathlineto{\pgfqpoint{3.439139in}{0.929624in}}%
\pgfpathlineto{\pgfqpoint{3.449115in}{0.701865in}}%
\pgfpathlineto{\pgfqpoint{3.454103in}{0.651836in}}%
\pgfpathlineto{\pgfqpoint{3.459091in}{0.650495in}}%
\pgfpathlineto{\pgfqpoint{3.459091in}{0.650495in}}%
\pgfusepath{stroke}%
\end{pgfscope}%
\begin{pgfscope}%
\pgfpathrectangle{\pgfqpoint{0.500000in}{0.375000in}}{\pgfqpoint{3.100000in}{2.265000in}}%
\pgfusepath{clip}%
\pgfsetrectcap%
\pgfsetroundjoin%
\pgfsetlinewidth{1.505625pt}%
\definecolor{currentstroke}{rgb}{1.000000,0.000000,0.000000}%
\pgfsetstrokecolor{currentstroke}%
\pgfsetdash{}{0pt}%
\pgfpathmoveto{\pgfqpoint{0.865366in}{0.375000in}}%
\pgfpathlineto{\pgfqpoint{0.865366in}{2.640000in}}%
\pgfusepath{stroke}%
\end{pgfscope}%
\begin{pgfscope}%
\pgfpathrectangle{\pgfqpoint{0.500000in}{0.375000in}}{\pgfqpoint{3.100000in}{2.265000in}}%
\pgfusepath{clip}%
\pgfsetrectcap%
\pgfsetroundjoin%
\pgfsetlinewidth{1.505625pt}%
\definecolor{currentstroke}{rgb}{1.000000,0.000000,0.000000}%
\pgfsetstrokecolor{currentstroke}%
\pgfsetdash{}{0pt}%
\pgfpathmoveto{\pgfqpoint{1.518785in}{0.375000in}}%
\pgfpathlineto{\pgfqpoint{1.518785in}{2.640000in}}%
\pgfusepath{stroke}%
\end{pgfscope}%
\begin{pgfscope}%
\pgfpathrectangle{\pgfqpoint{0.500000in}{0.375000in}}{\pgfqpoint{3.100000in}{2.265000in}}%
\pgfusepath{clip}%
\pgfsetrectcap%
\pgfsetroundjoin%
\pgfsetlinewidth{1.505625pt}%
\definecolor{currentstroke}{rgb}{1.000000,0.000000,0.000000}%
\pgfsetstrokecolor{currentstroke}%
\pgfsetdash{}{0pt}%
\pgfpathmoveto{\pgfqpoint{1.987651in}{0.375000in}}%
\pgfpathlineto{\pgfqpoint{1.987651in}{2.640000in}}%
\pgfusepath{stroke}%
\end{pgfscope}%
\begin{pgfscope}%
\pgfpathrectangle{\pgfqpoint{0.500000in}{0.375000in}}{\pgfqpoint{3.100000in}{2.265000in}}%
\pgfusepath{clip}%
\pgfsetrectcap%
\pgfsetroundjoin%
\pgfsetlinewidth{1.505625pt}%
\definecolor{currentstroke}{rgb}{1.000000,0.000000,0.000000}%
\pgfsetstrokecolor{currentstroke}%
\pgfsetdash{}{0pt}%
\pgfpathmoveto{\pgfqpoint{2.366734in}{0.375000in}}%
\pgfpathlineto{\pgfqpoint{2.366734in}{2.640000in}}%
\pgfusepath{stroke}%
\end{pgfscope}%
\begin{pgfscope}%
\pgfsetrectcap%
\pgfsetmiterjoin%
\pgfsetlinewidth{0.803000pt}%
\definecolor{currentstroke}{rgb}{0.000000,0.000000,0.000000}%
\pgfsetstrokecolor{currentstroke}%
\pgfsetdash{}{0pt}%
\pgfpathmoveto{\pgfqpoint{0.500000in}{0.375000in}}%
\pgfpathlineto{\pgfqpoint{0.500000in}{2.640000in}}%
\pgfusepath{stroke}%
\end{pgfscope}%
\begin{pgfscope}%
\pgfsetrectcap%
\pgfsetmiterjoin%
\pgfsetlinewidth{0.803000pt}%
\definecolor{currentstroke}{rgb}{0.000000,0.000000,0.000000}%
\pgfsetstrokecolor{currentstroke}%
\pgfsetdash{}{0pt}%
\pgfpathmoveto{\pgfqpoint{3.600000in}{0.375000in}}%
\pgfpathlineto{\pgfqpoint{3.600000in}{2.640000in}}%
\pgfusepath{stroke}%
\end{pgfscope}%
\begin{pgfscope}%
\pgfsetrectcap%
\pgfsetmiterjoin%
\pgfsetlinewidth{0.803000pt}%
\definecolor{currentstroke}{rgb}{0.000000,0.000000,0.000000}%
\pgfsetstrokecolor{currentstroke}%
\pgfsetdash{}{0pt}%
\pgfpathmoveto{\pgfqpoint{0.500000in}{0.375000in}}%
\pgfpathlineto{\pgfqpoint{3.600000in}{0.375000in}}%
\pgfusepath{stroke}%
\end{pgfscope}%
\begin{pgfscope}%
\pgfsetrectcap%
\pgfsetmiterjoin%
\pgfsetlinewidth{0.803000pt}%
\definecolor{currentstroke}{rgb}{0.000000,0.000000,0.000000}%
\pgfsetstrokecolor{currentstroke}%
\pgfsetdash{}{0pt}%
\pgfpathmoveto{\pgfqpoint{0.500000in}{2.640000in}}%
\pgfpathlineto{\pgfqpoint{3.600000in}{2.640000in}}%
\pgfusepath{stroke}%
\end{pgfscope}%
\begin{pgfscope}%
\pgfsetbuttcap%
\pgfsetmiterjoin%
\definecolor{currentfill}{rgb}{1.000000,1.000000,1.000000}%
\pgfsetfillcolor{currentfill}%
\pgfsetfillopacity{0.800000}%
\pgfsetlinewidth{1.003750pt}%
\definecolor{currentstroke}{rgb}{0.800000,0.800000,0.800000}%
\pgfsetstrokecolor{currentstroke}%
\pgfsetstrokeopacity{0.800000}%
\pgfsetdash{}{0pt}%
\pgfpathmoveto{\pgfqpoint{0.597222in}{0.444444in}}%
\pgfpathlineto{\pgfqpoint{1.331019in}{0.444444in}}%
\pgfpathquadraticcurveto{\pgfqpoint{1.358797in}{0.444444in}}{\pgfqpoint{1.358797in}{0.472222in}}%
\pgfpathlineto{\pgfqpoint{1.358797in}{0.652006in}}%
\pgfpathquadraticcurveto{\pgfqpoint{1.358797in}{0.679784in}}{\pgfqpoint{1.331019in}{0.679784in}}%
\pgfpathlineto{\pgfqpoint{0.597222in}{0.679784in}}%
\pgfpathquadraticcurveto{\pgfqpoint{0.569444in}{0.679784in}}{\pgfqpoint{0.569444in}{0.652006in}}%
\pgfpathlineto{\pgfqpoint{0.569444in}{0.472222in}}%
\pgfpathquadraticcurveto{\pgfqpoint{0.569444in}{0.444444in}}{\pgfqpoint{0.597222in}{0.444444in}}%
\pgfpathlineto{\pgfqpoint{0.597222in}{0.444444in}}%
\pgfpathclose%
\pgfusepath{stroke,fill}%
\end{pgfscope}%
\begin{pgfscope}%
\pgfsetrectcap%
\pgfsetroundjoin%
\pgfsetlinewidth{1.505625pt}%
\definecolor{currentstroke}{rgb}{0.121569,0.466667,0.705882}%
\pgfsetstrokecolor{currentstroke}%
\pgfsetdash{}{0pt}%
\pgfpathmoveto{\pgfqpoint{0.625000in}{0.575617in}}%
\pgfpathlineto{\pgfqpoint{0.763889in}{0.575617in}}%
\pgfpathlineto{\pgfqpoint{0.902778in}{0.575617in}}%
\pgfusepath{stroke}%
\end{pgfscope}%
\begin{pgfscope}%
\definecolor{textcolor}{rgb}{0.000000,0.000000,0.000000}%
\pgfsetstrokecolor{textcolor}%
\pgfsetfillcolor{textcolor}%
\pgftext[x=1.013889in,y=0.527006in,left,base]{\color{textcolor}\rmfamily\fontsize{10.000000}{12.000000}\selectfont pA:1}%
\end{pgfscope}%
\end{pgfpicture}%
\makeatother%
\endgroup%
}
%         \caption{Pressure Anomalies}
%         \label{fig:hydraulic_sim_standard_pressure}
%     \end{minipage}
%     \begin{minipage}[t]{0.5\linewidth}
%         %%\centering
%         \resizebox{\linewidth}{!}{%% Creator: Matplotlib, PGF backend
%%
%% To include the figure in your LaTeX document, write
%%   \input{<filename>.pgf}
%%
%% Make sure the required packages are loaded in your preamble
%%   \usepackage{pgf}
%%
%% Also ensure that all the required font packages are loaded; for instance,
%% the lmodern package is sometimes necessary when using math font.
%%   \usepackage{lmodern}
%%
%% Figures using additional raster images can only be included by \input if
%% they are in the same directory as the main LaTeX file. For loading figures
%% from other directories you can use the `import` package
%%   \usepackage{import}
%%
%% and then include the figures with
%%   \import{<path to file>}{<filename>.pgf}
%%
%% Matplotlib used the following preamble
%%
\begingroup%
\makeatletter%
\begin{pgfpicture}%
\pgfpathrectangle{\pgfpointorigin}{\pgfqpoint{4.000000in}{3.000000in}}%
\pgfusepath{use as bounding box, clip}%
\begin{pgfscope}%
\pgfsetbuttcap%
\pgfsetmiterjoin%
\pgfsetlinewidth{0.000000pt}%
\definecolor{currentstroke}{rgb}{1.000000,1.000000,1.000000}%
\pgfsetstrokecolor{currentstroke}%
\pgfsetstrokeopacity{0.000000}%
\pgfsetdash{}{0pt}%
\pgfpathmoveto{\pgfqpoint{0.000000in}{0.000000in}}%
\pgfpathlineto{\pgfqpoint{4.000000in}{0.000000in}}%
\pgfpathlineto{\pgfqpoint{4.000000in}{3.000000in}}%
\pgfpathlineto{\pgfqpoint{0.000000in}{3.000000in}}%
\pgfpathlineto{\pgfqpoint{0.000000in}{0.000000in}}%
\pgfpathclose%
\pgfusepath{}%
\end{pgfscope}%
\begin{pgfscope}%
\pgfsetbuttcap%
\pgfsetmiterjoin%
\definecolor{currentfill}{rgb}{1.000000,1.000000,1.000000}%
\pgfsetfillcolor{currentfill}%
\pgfsetlinewidth{0.000000pt}%
\definecolor{currentstroke}{rgb}{0.000000,0.000000,0.000000}%
\pgfsetstrokecolor{currentstroke}%
\pgfsetstrokeopacity{0.000000}%
\pgfsetdash{}{0pt}%
\pgfpathmoveto{\pgfqpoint{0.500000in}{0.375000in}}%
\pgfpathlineto{\pgfqpoint{3.600000in}{0.375000in}}%
\pgfpathlineto{\pgfqpoint{3.600000in}{2.640000in}}%
\pgfpathlineto{\pgfqpoint{0.500000in}{2.640000in}}%
\pgfpathlineto{\pgfqpoint{0.500000in}{0.375000in}}%
\pgfpathclose%
\pgfusepath{fill}%
\end{pgfscope}%
\begin{pgfscope}%
\pgfsetbuttcap%
\pgfsetroundjoin%
\definecolor{currentfill}{rgb}{0.000000,0.000000,0.000000}%
\pgfsetfillcolor{currentfill}%
\pgfsetlinewidth{0.803000pt}%
\definecolor{currentstroke}{rgb}{0.000000,0.000000,0.000000}%
\pgfsetstrokecolor{currentstroke}%
\pgfsetdash{}{0pt}%
\pgfsys@defobject{currentmarker}{\pgfqpoint{0.000000in}{-0.048611in}}{\pgfqpoint{0.000000in}{0.000000in}}{%
\pgfpathmoveto{\pgfqpoint{0.000000in}{0.000000in}}%
\pgfpathlineto{\pgfqpoint{0.000000in}{-0.048611in}}%
\pgfusepath{stroke,fill}%
}%
\begin{pgfscope}%
\pgfsys@transformshift{0.640909in}{0.375000in}%
\pgfsys@useobject{currentmarker}{}%
\end{pgfscope}%
\end{pgfscope}%
\begin{pgfscope}%
\definecolor{textcolor}{rgb}{0.000000,0.000000,0.000000}%
\pgfsetstrokecolor{textcolor}%
\pgfsetfillcolor{textcolor}%
\pgftext[x=0.640909in,y=0.277778in,,top]{\color{textcolor}\rmfamily\fontsize{10.000000}{12.000000}\selectfont \(\displaystyle {0}\)}%
\end{pgfscope}%
\begin{pgfscope}%
\pgfsetbuttcap%
\pgfsetroundjoin%
\definecolor{currentfill}{rgb}{0.000000,0.000000,0.000000}%
\pgfsetfillcolor{currentfill}%
\pgfsetlinewidth{0.803000pt}%
\definecolor{currentstroke}{rgb}{0.000000,0.000000,0.000000}%
\pgfsetstrokecolor{currentstroke}%
\pgfsetdash{}{0pt}%
\pgfsys@defobject{currentmarker}{\pgfqpoint{0.000000in}{-0.048611in}}{\pgfqpoint{0.000000in}{0.000000in}}{%
\pgfpathmoveto{\pgfqpoint{0.000000in}{0.000000in}}%
\pgfpathlineto{\pgfqpoint{0.000000in}{-0.048611in}}%
\pgfusepath{stroke,fill}%
}%
\begin{pgfscope}%
\pgfsys@transformshift{1.361672in}{0.375000in}%
\pgfsys@useobject{currentmarker}{}%
\end{pgfscope}%
\end{pgfscope}%
\begin{pgfscope}%
\definecolor{textcolor}{rgb}{0.000000,0.000000,0.000000}%
\pgfsetstrokecolor{textcolor}%
\pgfsetfillcolor{textcolor}%
\pgftext[x=1.361672in,y=0.277778in,,top]{\color{textcolor}\rmfamily\fontsize{10.000000}{12.000000}\selectfont \(\displaystyle {100}\)}%
\end{pgfscope}%
\begin{pgfscope}%
\pgfsetbuttcap%
\pgfsetroundjoin%
\definecolor{currentfill}{rgb}{0.000000,0.000000,0.000000}%
\pgfsetfillcolor{currentfill}%
\pgfsetlinewidth{0.803000pt}%
\definecolor{currentstroke}{rgb}{0.000000,0.000000,0.000000}%
\pgfsetstrokecolor{currentstroke}%
\pgfsetdash{}{0pt}%
\pgfsys@defobject{currentmarker}{\pgfqpoint{0.000000in}{-0.048611in}}{\pgfqpoint{0.000000in}{0.000000in}}{%
\pgfpathmoveto{\pgfqpoint{0.000000in}{0.000000in}}%
\pgfpathlineto{\pgfqpoint{0.000000in}{-0.048611in}}%
\pgfusepath{stroke,fill}%
}%
\begin{pgfscope}%
\pgfsys@transformshift{2.082434in}{0.375000in}%
\pgfsys@useobject{currentmarker}{}%
\end{pgfscope}%
\end{pgfscope}%
\begin{pgfscope}%
\definecolor{textcolor}{rgb}{0.000000,0.000000,0.000000}%
\pgfsetstrokecolor{textcolor}%
\pgfsetfillcolor{textcolor}%
\pgftext[x=2.082434in,y=0.277778in,,top]{\color{textcolor}\rmfamily\fontsize{10.000000}{12.000000}\selectfont \(\displaystyle {200}\)}%
\end{pgfscope}%
\begin{pgfscope}%
\pgfsetbuttcap%
\pgfsetroundjoin%
\definecolor{currentfill}{rgb}{0.000000,0.000000,0.000000}%
\pgfsetfillcolor{currentfill}%
\pgfsetlinewidth{0.803000pt}%
\definecolor{currentstroke}{rgb}{0.000000,0.000000,0.000000}%
\pgfsetstrokecolor{currentstroke}%
\pgfsetdash{}{0pt}%
\pgfsys@defobject{currentmarker}{\pgfqpoint{0.000000in}{-0.048611in}}{\pgfqpoint{0.000000in}{0.000000in}}{%
\pgfpathmoveto{\pgfqpoint{0.000000in}{0.000000in}}%
\pgfpathlineto{\pgfqpoint{0.000000in}{-0.048611in}}%
\pgfusepath{stroke,fill}%
}%
\begin{pgfscope}%
\pgfsys@transformshift{2.803197in}{0.375000in}%
\pgfsys@useobject{currentmarker}{}%
\end{pgfscope}%
\end{pgfscope}%
\begin{pgfscope}%
\definecolor{textcolor}{rgb}{0.000000,0.000000,0.000000}%
\pgfsetstrokecolor{textcolor}%
\pgfsetfillcolor{textcolor}%
\pgftext[x=2.803197in,y=0.277778in,,top]{\color{textcolor}\rmfamily\fontsize{10.000000}{12.000000}\selectfont \(\displaystyle {300}\)}%
\end{pgfscope}%
\begin{pgfscope}%
\pgfsetbuttcap%
\pgfsetroundjoin%
\definecolor{currentfill}{rgb}{0.000000,0.000000,0.000000}%
\pgfsetfillcolor{currentfill}%
\pgfsetlinewidth{0.803000pt}%
\definecolor{currentstroke}{rgb}{0.000000,0.000000,0.000000}%
\pgfsetstrokecolor{currentstroke}%
\pgfsetdash{}{0pt}%
\pgfsys@defobject{currentmarker}{\pgfqpoint{0.000000in}{-0.048611in}}{\pgfqpoint{0.000000in}{0.000000in}}{%
\pgfpathmoveto{\pgfqpoint{0.000000in}{0.000000in}}%
\pgfpathlineto{\pgfqpoint{0.000000in}{-0.048611in}}%
\pgfusepath{stroke,fill}%
}%
\begin{pgfscope}%
\pgfsys@transformshift{3.523960in}{0.375000in}%
\pgfsys@useobject{currentmarker}{}%
\end{pgfscope}%
\end{pgfscope}%
\begin{pgfscope}%
\definecolor{textcolor}{rgb}{0.000000,0.000000,0.000000}%
\pgfsetstrokecolor{textcolor}%
\pgfsetfillcolor{textcolor}%
\pgftext[x=3.523960in,y=0.277778in,,top]{\color{textcolor}\rmfamily\fontsize{10.000000}{12.000000}\selectfont \(\displaystyle {400}\)}%
\end{pgfscope}%
\begin{pgfscope}%
\definecolor{textcolor}{rgb}{0.000000,0.000000,0.000000}%
\pgfsetstrokecolor{textcolor}%
\pgfsetfillcolor{textcolor}%
\pgftext[x=2.050000in,y=0.098766in,,top]{\color{textcolor}\rmfamily\fontsize{10.000000}{12.000000}\selectfont time}%
\end{pgfscope}%
\begin{pgfscope}%
\pgfsetbuttcap%
\pgfsetroundjoin%
\definecolor{currentfill}{rgb}{0.000000,0.000000,0.000000}%
\pgfsetfillcolor{currentfill}%
\pgfsetlinewidth{0.803000pt}%
\definecolor{currentstroke}{rgb}{0.000000,0.000000,0.000000}%
\pgfsetstrokecolor{currentstroke}%
\pgfsetdash{}{0pt}%
\pgfsys@defobject{currentmarker}{\pgfqpoint{-0.048611in}{0.000000in}}{\pgfqpoint{-0.000000in}{0.000000in}}{%
\pgfpathmoveto{\pgfqpoint{-0.000000in}{0.000000in}}%
\pgfpathlineto{\pgfqpoint{-0.048611in}{0.000000in}}%
\pgfusepath{stroke,fill}%
}%
\begin{pgfscope}%
\pgfsys@transformshift{0.500000in}{0.388711in}%
\pgfsys@useobject{currentmarker}{}%
\end{pgfscope}%
\end{pgfscope}%
\begin{pgfscope}%
\definecolor{textcolor}{rgb}{0.000000,0.000000,0.000000}%
\pgfsetstrokecolor{textcolor}%
\pgfsetfillcolor{textcolor}%
\pgftext[x=0.333333in, y=0.340486in, left, base]{\color{textcolor}\rmfamily\fontsize{10.000000}{12.000000}\selectfont \(\displaystyle {1}\)}%
\end{pgfscope}%
\begin{pgfscope}%
\pgfsetbuttcap%
\pgfsetroundjoin%
\definecolor{currentfill}{rgb}{0.000000,0.000000,0.000000}%
\pgfsetfillcolor{currentfill}%
\pgfsetlinewidth{0.803000pt}%
\definecolor{currentstroke}{rgb}{0.000000,0.000000,0.000000}%
\pgfsetstrokecolor{currentstroke}%
\pgfsetdash{}{0pt}%
\pgfsys@defobject{currentmarker}{\pgfqpoint{-0.048611in}{0.000000in}}{\pgfqpoint{-0.000000in}{0.000000in}}{%
\pgfpathmoveto{\pgfqpoint{-0.000000in}{0.000000in}}%
\pgfpathlineto{\pgfqpoint{-0.048611in}{0.000000in}}%
\pgfusepath{stroke,fill}%
}%
\begin{pgfscope}%
\pgfsys@transformshift{0.500000in}{0.776727in}%
\pgfsys@useobject{currentmarker}{}%
\end{pgfscope}%
\end{pgfscope}%
\begin{pgfscope}%
\definecolor{textcolor}{rgb}{0.000000,0.000000,0.000000}%
\pgfsetstrokecolor{textcolor}%
\pgfsetfillcolor{textcolor}%
\pgftext[x=0.333333in, y=0.728501in, left, base]{\color{textcolor}\rmfamily\fontsize{10.000000}{12.000000}\selectfont \(\displaystyle {2}\)}%
\end{pgfscope}%
\begin{pgfscope}%
\pgfsetbuttcap%
\pgfsetroundjoin%
\definecolor{currentfill}{rgb}{0.000000,0.000000,0.000000}%
\pgfsetfillcolor{currentfill}%
\pgfsetlinewidth{0.803000pt}%
\definecolor{currentstroke}{rgb}{0.000000,0.000000,0.000000}%
\pgfsetstrokecolor{currentstroke}%
\pgfsetdash{}{0pt}%
\pgfsys@defobject{currentmarker}{\pgfqpoint{-0.048611in}{0.000000in}}{\pgfqpoint{-0.000000in}{0.000000in}}{%
\pgfpathmoveto{\pgfqpoint{-0.000000in}{0.000000in}}%
\pgfpathlineto{\pgfqpoint{-0.048611in}{0.000000in}}%
\pgfusepath{stroke,fill}%
}%
\begin{pgfscope}%
\pgfsys@transformshift{0.500000in}{1.164742in}%
\pgfsys@useobject{currentmarker}{}%
\end{pgfscope}%
\end{pgfscope}%
\begin{pgfscope}%
\definecolor{textcolor}{rgb}{0.000000,0.000000,0.000000}%
\pgfsetstrokecolor{textcolor}%
\pgfsetfillcolor{textcolor}%
\pgftext[x=0.333333in, y=1.116517in, left, base]{\color{textcolor}\rmfamily\fontsize{10.000000}{12.000000}\selectfont \(\displaystyle {3}\)}%
\end{pgfscope}%
\begin{pgfscope}%
\pgfsetbuttcap%
\pgfsetroundjoin%
\definecolor{currentfill}{rgb}{0.000000,0.000000,0.000000}%
\pgfsetfillcolor{currentfill}%
\pgfsetlinewidth{0.803000pt}%
\definecolor{currentstroke}{rgb}{0.000000,0.000000,0.000000}%
\pgfsetstrokecolor{currentstroke}%
\pgfsetdash{}{0pt}%
\pgfsys@defobject{currentmarker}{\pgfqpoint{-0.048611in}{0.000000in}}{\pgfqpoint{-0.000000in}{0.000000in}}{%
\pgfpathmoveto{\pgfqpoint{-0.000000in}{0.000000in}}%
\pgfpathlineto{\pgfqpoint{-0.048611in}{0.000000in}}%
\pgfusepath{stroke,fill}%
}%
\begin{pgfscope}%
\pgfsys@transformshift{0.500000in}{1.552758in}%
\pgfsys@useobject{currentmarker}{}%
\end{pgfscope}%
\end{pgfscope}%
\begin{pgfscope}%
\definecolor{textcolor}{rgb}{0.000000,0.000000,0.000000}%
\pgfsetstrokecolor{textcolor}%
\pgfsetfillcolor{textcolor}%
\pgftext[x=0.333333in, y=1.504533in, left, base]{\color{textcolor}\rmfamily\fontsize{10.000000}{12.000000}\selectfont \(\displaystyle {4}\)}%
\end{pgfscope}%
\begin{pgfscope}%
\pgfsetbuttcap%
\pgfsetroundjoin%
\definecolor{currentfill}{rgb}{0.000000,0.000000,0.000000}%
\pgfsetfillcolor{currentfill}%
\pgfsetlinewidth{0.803000pt}%
\definecolor{currentstroke}{rgb}{0.000000,0.000000,0.000000}%
\pgfsetstrokecolor{currentstroke}%
\pgfsetdash{}{0pt}%
\pgfsys@defobject{currentmarker}{\pgfqpoint{-0.048611in}{0.000000in}}{\pgfqpoint{-0.000000in}{0.000000in}}{%
\pgfpathmoveto{\pgfqpoint{-0.000000in}{0.000000in}}%
\pgfpathlineto{\pgfqpoint{-0.048611in}{0.000000in}}%
\pgfusepath{stroke,fill}%
}%
\begin{pgfscope}%
\pgfsys@transformshift{0.500000in}{1.940774in}%
\pgfsys@useobject{currentmarker}{}%
\end{pgfscope}%
\end{pgfscope}%
\begin{pgfscope}%
\definecolor{textcolor}{rgb}{0.000000,0.000000,0.000000}%
\pgfsetstrokecolor{textcolor}%
\pgfsetfillcolor{textcolor}%
\pgftext[x=0.333333in, y=1.892549in, left, base]{\color{textcolor}\rmfamily\fontsize{10.000000}{12.000000}\selectfont \(\displaystyle {5}\)}%
\end{pgfscope}%
\begin{pgfscope}%
\pgfsetbuttcap%
\pgfsetroundjoin%
\definecolor{currentfill}{rgb}{0.000000,0.000000,0.000000}%
\pgfsetfillcolor{currentfill}%
\pgfsetlinewidth{0.803000pt}%
\definecolor{currentstroke}{rgb}{0.000000,0.000000,0.000000}%
\pgfsetstrokecolor{currentstroke}%
\pgfsetdash{}{0pt}%
\pgfsys@defobject{currentmarker}{\pgfqpoint{-0.048611in}{0.000000in}}{\pgfqpoint{-0.000000in}{0.000000in}}{%
\pgfpathmoveto{\pgfqpoint{-0.000000in}{0.000000in}}%
\pgfpathlineto{\pgfqpoint{-0.048611in}{0.000000in}}%
\pgfusepath{stroke,fill}%
}%
\begin{pgfscope}%
\pgfsys@transformshift{0.500000in}{2.328790in}%
\pgfsys@useobject{currentmarker}{}%
\end{pgfscope}%
\end{pgfscope}%
\begin{pgfscope}%
\definecolor{textcolor}{rgb}{0.000000,0.000000,0.000000}%
\pgfsetstrokecolor{textcolor}%
\pgfsetfillcolor{textcolor}%
\pgftext[x=0.333333in, y=2.280565in, left, base]{\color{textcolor}\rmfamily\fontsize{10.000000}{12.000000}\selectfont \(\displaystyle {6}\)}%
\end{pgfscope}%
\begin{pgfscope}%
\definecolor{textcolor}{rgb}{0.000000,0.000000,0.000000}%
\pgfsetstrokecolor{textcolor}%
\pgfsetfillcolor{textcolor}%
\pgftext[x=0.500000in,y=2.681667in,left,base]{\color{textcolor}\rmfamily\fontsize{10.000000}{12.000000}\selectfont \(\displaystyle \times{10^{6}}{}\)}%
\end{pgfscope}%
\begin{pgfscope}%
\pgfpathrectangle{\pgfqpoint{0.500000in}{0.375000in}}{\pgfqpoint{3.100000in}{2.265000in}}%
\pgfusepath{clip}%
\pgfsetrectcap%
\pgfsetroundjoin%
\pgfsetlinewidth{1.505625pt}%
\definecolor{currentstroke}{rgb}{0.121569,0.466667,0.705882}%
\pgfsetstrokecolor{currentstroke}%
\pgfsetdash{}{0pt}%
\pgfpathmoveto{\pgfqpoint{0.640909in}{1.569726in}}%
\pgfpathlineto{\pgfqpoint{0.936422in}{1.570818in}}%
\pgfpathlineto{\pgfqpoint{0.950837in}{1.574573in}}%
\pgfpathlineto{\pgfqpoint{0.965252in}{1.582184in}}%
\pgfpathlineto{\pgfqpoint{0.979668in}{1.593260in}}%
\pgfpathlineto{\pgfqpoint{0.994083in}{1.607717in}}%
\pgfpathlineto{\pgfqpoint{1.008498in}{1.625445in}}%
\pgfpathlineto{\pgfqpoint{1.022913in}{1.646311in}}%
\pgfpathlineto{\pgfqpoint{1.037329in}{1.675867in}}%
\pgfpathlineto{\pgfqpoint{1.051744in}{1.717428in}}%
\pgfpathlineto{\pgfqpoint{1.066159in}{1.775457in}}%
\pgfpathlineto{\pgfqpoint{1.080574in}{1.856874in}}%
\pgfpathlineto{\pgfqpoint{1.094990in}{1.972433in}}%
\pgfpathlineto{\pgfqpoint{1.123820in}{2.300138in}}%
\pgfpathlineto{\pgfqpoint{1.138235in}{2.427371in}}%
\pgfpathlineto{\pgfqpoint{1.152651in}{2.515114in}}%
\pgfpathlineto{\pgfqpoint{1.159858in}{2.535814in}}%
\pgfpathlineto{\pgfqpoint{1.174273in}{2.536768in}}%
\pgfpathlineto{\pgfqpoint{1.188689in}{2.537045in}}%
\pgfpathlineto{\pgfqpoint{1.195896in}{2.534680in}}%
\pgfpathlineto{\pgfqpoint{1.203104in}{2.526482in}}%
\pgfpathlineto{\pgfqpoint{1.210312in}{2.512562in}}%
\pgfpathlineto{\pgfqpoint{1.217519in}{2.482800in}}%
\pgfpathlineto{\pgfqpoint{1.224727in}{2.432351in}}%
\pgfpathlineto{\pgfqpoint{1.231934in}{2.342063in}}%
\pgfpathlineto{\pgfqpoint{1.239142in}{2.229119in}}%
\pgfpathlineto{\pgfqpoint{1.253557in}{1.866021in}}%
\pgfpathlineto{\pgfqpoint{1.275180in}{1.279339in}}%
\pgfpathlineto{\pgfqpoint{1.289595in}{0.979672in}}%
\pgfpathlineto{\pgfqpoint{1.296803in}{0.881170in}}%
\pgfpathlineto{\pgfqpoint{1.304011in}{0.823153in}}%
\pgfpathlineto{\pgfqpoint{1.311218in}{0.807493in}}%
\pgfpathlineto{\pgfqpoint{1.318426in}{0.833732in}}%
\pgfpathlineto{\pgfqpoint{1.325634in}{0.899153in}}%
\pgfpathlineto{\pgfqpoint{1.332841in}{0.998988in}}%
\pgfpathlineto{\pgfqpoint{1.347256in}{1.274530in}}%
\pgfpathlineto{\pgfqpoint{1.376087in}{1.889966in}}%
\pgfpathlineto{\pgfqpoint{1.383295in}{2.007975in}}%
\pgfpathlineto{\pgfqpoint{1.390502in}{2.098319in}}%
\pgfpathlineto{\pgfqpoint{1.397710in}{2.119429in}}%
\pgfpathlineto{\pgfqpoint{1.404917in}{2.119429in}}%
\pgfpathlineto{\pgfqpoint{1.412125in}{2.135174in}}%
\pgfpathlineto{\pgfqpoint{1.419333in}{2.161350in}}%
\pgfpathlineto{\pgfqpoint{1.426540in}{2.176826in}}%
\pgfpathlineto{\pgfqpoint{1.433748in}{2.179799in}}%
\pgfpathlineto{\pgfqpoint{1.440956in}{2.150522in}}%
\pgfpathlineto{\pgfqpoint{1.448163in}{2.087924in}}%
\pgfpathlineto{\pgfqpoint{1.455371in}{1.996922in}}%
\pgfpathlineto{\pgfqpoint{1.469786in}{1.756369in}}%
\pgfpathlineto{\pgfqpoint{1.491409in}{1.366801in}}%
\pgfpathlineto{\pgfqpoint{1.505824in}{1.175415in}}%
\pgfpathlineto{\pgfqpoint{1.513032in}{1.108221in}}%
\pgfpathlineto{\pgfqpoint{1.520239in}{1.084204in}}%
\pgfpathlineto{\pgfqpoint{1.527447in}{1.103730in}}%
\pgfpathlineto{\pgfqpoint{1.534655in}{1.163337in}}%
\pgfpathlineto{\pgfqpoint{1.541862in}{1.256173in}}%
\pgfpathlineto{\pgfqpoint{1.556278in}{1.501735in}}%
\pgfpathlineto{\pgfqpoint{1.570693in}{1.750053in}}%
\pgfpathlineto{\pgfqpoint{1.577900in}{1.847896in}}%
\pgfpathlineto{\pgfqpoint{1.585108in}{1.917133in}}%
\pgfpathlineto{\pgfqpoint{1.592316in}{1.952904in}}%
\pgfpathlineto{\pgfqpoint{1.599523in}{1.953510in}}%
\pgfpathlineto{\pgfqpoint{1.606731in}{1.920412in}}%
\pgfpathlineto{\pgfqpoint{1.613939in}{1.857955in}}%
\pgfpathlineto{\pgfqpoint{1.628354in}{1.673473in}}%
\pgfpathlineto{\pgfqpoint{1.649977in}{1.381784in}}%
\pgfpathlineto{\pgfqpoint{1.657184in}{1.314475in}}%
\pgfpathlineto{\pgfqpoint{1.664392in}{1.272152in}}%
\pgfpathlineto{\pgfqpoint{1.671600in}{1.257490in}}%
\pgfpathlineto{\pgfqpoint{1.678807in}{1.270642in}}%
\pgfpathlineto{\pgfqpoint{1.686015in}{1.309312in}}%
\pgfpathlineto{\pgfqpoint{1.693223in}{1.369050in}}%
\pgfpathlineto{\pgfqpoint{1.714845in}{1.608708in}}%
\pgfpathlineto{\pgfqpoint{1.722053in}{1.627609in}}%
\pgfpathlineto{\pgfqpoint{1.736468in}{1.627609in}}%
\pgfpathlineto{\pgfqpoint{1.743676in}{1.651330in}}%
\pgfpathlineto{\pgfqpoint{1.765299in}{1.783846in}}%
\pgfpathlineto{\pgfqpoint{1.772506in}{1.809524in}}%
\pgfpathlineto{\pgfqpoint{1.779714in}{1.810836in}}%
\pgfpathlineto{\pgfqpoint{1.786922in}{1.788732in}}%
\pgfpathlineto{\pgfqpoint{1.794129in}{1.746353in}}%
\pgfpathlineto{\pgfqpoint{1.801337in}{1.688630in}}%
\pgfpathlineto{\pgfqpoint{1.808545in}{1.607355in}}%
\pgfpathlineto{\pgfqpoint{1.815752in}{1.561857in}}%
\pgfpathlineto{\pgfqpoint{1.830167in}{1.561857in}}%
\pgfpathlineto{\pgfqpoint{1.837375in}{1.557289in}}%
\pgfpathlineto{\pgfqpoint{1.858998in}{1.526441in}}%
\pgfpathlineto{\pgfqpoint{1.866206in}{1.510367in}}%
\pgfpathlineto{\pgfqpoint{1.873413in}{1.487476in}}%
\pgfpathlineto{\pgfqpoint{1.880621in}{1.450679in}}%
\pgfpathlineto{\pgfqpoint{1.887828in}{1.384826in}}%
\pgfpathlineto{\pgfqpoint{1.902244in}{1.130476in}}%
\pgfpathlineto{\pgfqpoint{1.923867in}{0.733420in}}%
\pgfpathlineto{\pgfqpoint{1.931074in}{0.628263in}}%
\pgfpathlineto{\pgfqpoint{1.938282in}{0.547625in}}%
\pgfpathlineto{\pgfqpoint{1.945489in}{0.489606in}}%
\pgfpathlineto{\pgfqpoint{1.952697in}{0.477955in}}%
\pgfpathlineto{\pgfqpoint{2.010358in}{0.480796in}}%
\pgfpathlineto{\pgfqpoint{2.017566in}{0.485416in}}%
\pgfpathlineto{\pgfqpoint{2.024773in}{0.499230in}}%
\pgfpathlineto{\pgfqpoint{2.031981in}{0.519700in}}%
\pgfpathlineto{\pgfqpoint{2.039189in}{0.546606in}}%
\pgfpathlineto{\pgfqpoint{2.053604in}{0.637847in}}%
\pgfpathlineto{\pgfqpoint{2.060811in}{0.720306in}}%
\pgfpathlineto{\pgfqpoint{2.068019in}{0.843216in}}%
\pgfpathlineto{\pgfqpoint{2.104057in}{1.726170in}}%
\pgfpathlineto{\pgfqpoint{2.111265in}{1.847087in}}%
\pgfpathlineto{\pgfqpoint{2.118472in}{1.929934in}}%
\pgfpathlineto{\pgfqpoint{2.125680in}{1.971272in}}%
\pgfpathlineto{\pgfqpoint{2.132888in}{1.975015in}}%
\pgfpathlineto{\pgfqpoint{2.140095in}{1.975015in}}%
\pgfpathlineto{\pgfqpoint{2.147303in}{1.971130in}}%
\pgfpathlineto{\pgfqpoint{2.161718in}{1.949887in}}%
\pgfpathlineto{\pgfqpoint{2.168926in}{1.936470in}}%
\pgfpathlineto{\pgfqpoint{2.176133in}{1.907740in}}%
\pgfpathlineto{\pgfqpoint{2.183341in}{1.872663in}}%
\pgfpathlineto{\pgfqpoint{2.190549in}{1.807840in}}%
\pgfpathlineto{\pgfqpoint{2.219379in}{1.405275in}}%
\pgfpathlineto{\pgfqpoint{2.226587in}{1.337515in}}%
\pgfpathlineto{\pgfqpoint{2.233794in}{1.295670in}}%
\pgfpathlineto{\pgfqpoint{2.241002in}{1.282415in}}%
\pgfpathlineto{\pgfqpoint{2.262625in}{1.282583in}}%
\pgfpathlineto{\pgfqpoint{2.269833in}{1.287584in}}%
\pgfpathlineto{\pgfqpoint{2.277040in}{1.301504in}}%
\pgfpathlineto{\pgfqpoint{2.284248in}{1.344046in}}%
\pgfpathlineto{\pgfqpoint{2.298663in}{1.474906in}}%
\pgfpathlineto{\pgfqpoint{2.313078in}{1.615649in}}%
\pgfpathlineto{\pgfqpoint{2.320286in}{1.672087in}}%
\pgfpathlineto{\pgfqpoint{2.327494in}{1.712419in}}%
\pgfpathlineto{\pgfqpoint{2.334701in}{1.728052in}}%
\pgfpathlineto{\pgfqpoint{2.349116in}{1.728052in}}%
\pgfpathlineto{\pgfqpoint{2.363532in}{1.736149in}}%
\pgfpathlineto{\pgfqpoint{2.370739in}{1.729939in}}%
\pgfpathlineto{\pgfqpoint{2.377947in}{1.700842in}}%
\pgfpathlineto{\pgfqpoint{2.392362in}{1.597895in}}%
\pgfpathlineto{\pgfqpoint{2.406777in}{1.493296in}}%
\pgfpathlineto{\pgfqpoint{2.413985in}{1.459221in}}%
\pgfpathlineto{\pgfqpoint{2.421193in}{1.442876in}}%
\pgfpathlineto{\pgfqpoint{2.428400in}{1.444829in}}%
\pgfpathlineto{\pgfqpoint{2.435608in}{1.462956in}}%
\pgfpathlineto{\pgfqpoint{2.442816in}{1.493042in}}%
\pgfpathlineto{\pgfqpoint{2.457231in}{1.581200in}}%
\pgfpathlineto{\pgfqpoint{2.464439in}{1.615382in}}%
\pgfpathlineto{\pgfqpoint{2.471646in}{1.634118in}}%
\pgfpathlineto{\pgfqpoint{2.478854in}{1.635339in}}%
\pgfpathlineto{\pgfqpoint{2.486061in}{1.620812in}}%
\pgfpathlineto{\pgfqpoint{2.500477in}{1.565525in}}%
\pgfpathlineto{\pgfqpoint{2.507684in}{1.538015in}}%
\pgfpathlineto{\pgfqpoint{2.514892in}{1.518273in}}%
\pgfpathlineto{\pgfqpoint{2.522100in}{1.509056in}}%
\pgfpathlineto{\pgfqpoint{2.529307in}{1.516779in}}%
\pgfpathlineto{\pgfqpoint{2.543722in}{1.561234in}}%
\pgfpathlineto{\pgfqpoint{2.550930in}{1.567007in}}%
\pgfpathlineto{\pgfqpoint{2.565345in}{1.567007in}}%
\pgfpathlineto{\pgfqpoint{2.572553in}{1.575368in}}%
\pgfpathlineto{\pgfqpoint{2.579761in}{1.588560in}}%
\pgfpathlineto{\pgfqpoint{2.586968in}{1.594052in}}%
\pgfpathlineto{\pgfqpoint{2.594176in}{1.587680in}}%
\pgfpathlineto{\pgfqpoint{2.601383in}{1.569500in}}%
\pgfpathlineto{\pgfqpoint{2.608591in}{1.565098in}}%
\pgfpathlineto{\pgfqpoint{2.651837in}{1.564570in}}%
\pgfpathlineto{\pgfqpoint{2.659044in}{1.566766in}}%
\pgfpathlineto{\pgfqpoint{2.666252in}{1.571338in}}%
\pgfpathlineto{\pgfqpoint{2.673460in}{1.581569in}}%
\pgfpathlineto{\pgfqpoint{2.680667in}{1.603795in}}%
\pgfpathlineto{\pgfqpoint{2.687875in}{1.651608in}}%
\pgfpathlineto{\pgfqpoint{2.702290in}{1.813097in}}%
\pgfpathlineto{\pgfqpoint{2.716705in}{1.985864in}}%
\pgfpathlineto{\pgfqpoint{2.723913in}{2.055617in}}%
\pgfpathlineto{\pgfqpoint{2.731121in}{2.098055in}}%
\pgfpathlineto{\pgfqpoint{2.745536in}{2.098055in}}%
\pgfpathlineto{\pgfqpoint{2.759951in}{2.100657in}}%
\pgfpathlineto{\pgfqpoint{2.767159in}{2.100657in}}%
\pgfpathlineto{\pgfqpoint{2.781574in}{2.105424in}}%
\pgfpathlineto{\pgfqpoint{2.788782in}{2.110197in}}%
\pgfpathlineto{\pgfqpoint{2.795989in}{2.110674in}}%
\pgfpathlineto{\pgfqpoint{2.803197in}{2.083202in}}%
\pgfpathlineto{\pgfqpoint{2.810405in}{2.022155in}}%
\pgfpathlineto{\pgfqpoint{2.817612in}{1.931528in}}%
\pgfpathlineto{\pgfqpoint{2.832027in}{1.686676in}}%
\pgfpathlineto{\pgfqpoint{2.853650in}{1.284180in}}%
\pgfpathlineto{\pgfqpoint{2.860858in}{1.175436in}}%
\pgfpathlineto{\pgfqpoint{2.868066in}{1.092032in}}%
\pgfpathlineto{\pgfqpoint{2.875273in}{1.039358in}}%
\pgfpathlineto{\pgfqpoint{2.882481in}{1.020814in}}%
\pgfpathlineto{\pgfqpoint{2.889688in}{1.037580in}}%
\pgfpathlineto{\pgfqpoint{2.896896in}{1.088547in}}%
\pgfpathlineto{\pgfqpoint{2.904104in}{1.170385in}}%
\pgfpathlineto{\pgfqpoint{2.918519in}{1.403682in}}%
\pgfpathlineto{\pgfqpoint{2.947349in}{1.922459in}}%
\pgfpathlineto{\pgfqpoint{2.954557in}{2.013561in}}%
\pgfpathlineto{\pgfqpoint{2.961765in}{2.075465in}}%
\pgfpathlineto{\pgfqpoint{2.968972in}{2.104190in}}%
\pgfpathlineto{\pgfqpoint{2.976180in}{2.097914in}}%
\pgfpathlineto{\pgfqpoint{2.983388in}{2.057084in}}%
\pgfpathlineto{\pgfqpoint{2.990595in}{1.984389in}}%
\pgfpathlineto{\pgfqpoint{3.005010in}{1.764137in}}%
\pgfpathlineto{\pgfqpoint{3.033841in}{1.241561in}}%
\pgfpathlineto{\pgfqpoint{3.041049in}{1.143320in}}%
\pgfpathlineto{\pgfqpoint{3.048256in}{1.072494in}}%
\pgfpathlineto{\pgfqpoint{3.055464in}{1.033659in}}%
\pgfpathlineto{\pgfqpoint{3.062671in}{1.029314in}}%
\pgfpathlineto{\pgfqpoint{3.069879in}{1.059720in}}%
\pgfpathlineto{\pgfqpoint{3.077087in}{1.122880in}}%
\pgfpathlineto{\pgfqpoint{3.084294in}{1.214675in}}%
\pgfpathlineto{\pgfqpoint{3.098710in}{1.458773in}}%
\pgfpathlineto{\pgfqpoint{3.120332in}{1.853043in}}%
\pgfpathlineto{\pgfqpoint{3.127540in}{1.957742in}}%
\pgfpathlineto{\pgfqpoint{3.134748in}{2.036848in}}%
\pgfpathlineto{\pgfqpoint{3.141955in}{2.085260in}}%
\pgfpathlineto{\pgfqpoint{3.149163in}{2.099873in}}%
\pgfpathlineto{\pgfqpoint{3.156371in}{2.079782in}}%
\pgfpathlineto{\pgfqpoint{3.163578in}{2.026329in}}%
\pgfpathlineto{\pgfqpoint{3.170786in}{1.943019in}}%
\pgfpathlineto{\pgfqpoint{3.185201in}{1.710127in}}%
\pgfpathlineto{\pgfqpoint{3.206824in}{1.313797in}}%
\pgfpathlineto{\pgfqpoint{3.221239in}{1.116400in}}%
\pgfpathlineto{\pgfqpoint{3.228447in}{1.058623in}}%
\pgfpathlineto{\pgfqpoint{3.235654in}{1.033729in}}%
\pgfpathlineto{\pgfqpoint{3.242862in}{1.043311in}}%
\pgfpathlineto{\pgfqpoint{3.250070in}{1.086727in}}%
\pgfpathlineto{\pgfqpoint{3.257277in}{1.161137in}}%
\pgfpathlineto{\pgfqpoint{3.271693in}{1.381829in}}%
\pgfpathlineto{\pgfqpoint{3.300523in}{1.893960in}}%
\pgfpathlineto{\pgfqpoint{3.307731in}{1.988051in}}%
\pgfpathlineto{\pgfqpoint{3.314938in}{2.054594in}}%
\pgfpathlineto{\pgfqpoint{3.322146in}{2.089306in}}%
\pgfpathlineto{\pgfqpoint{3.329354in}{2.089973in}}%
\pgfpathlineto{\pgfqpoint{3.336561in}{2.056594in}}%
\pgfpathlineto{\pgfqpoint{3.343769in}{1.991373in}}%
\pgfpathlineto{\pgfqpoint{3.350977in}{1.898574in}}%
\pgfpathlineto{\pgfqpoint{3.365392in}{1.655801in}}%
\pgfpathlineto{\pgfqpoint{3.387015in}{1.270462in}}%
\pgfpathlineto{\pgfqpoint{3.394222in}{1.169806in}}%
\pgfpathlineto{\pgfqpoint{3.401430in}{1.094852in}}%
\pgfpathlineto{\pgfqpoint{3.408638in}{1.050442in}}%
\pgfpathlineto{\pgfqpoint{3.415845in}{1.039438in}}%
\pgfpathlineto{\pgfqpoint{3.423053in}{1.062531in}}%
\pgfpathlineto{\pgfqpoint{3.430260in}{1.118201in}}%
\pgfpathlineto{\pgfqpoint{3.437468in}{1.202812in}}%
\pgfpathlineto{\pgfqpoint{3.451883in}{1.435281in}}%
\pgfpathlineto{\pgfqpoint{3.459091in}{1.474392in}}%
\pgfpathlineto{\pgfqpoint{3.459091in}{1.474392in}}%
\pgfusepath{stroke}%
\end{pgfscope}%
\begin{pgfscope}%
\pgfpathrectangle{\pgfqpoint{0.500000in}{0.375000in}}{\pgfqpoint{3.100000in}{2.265000in}}%
\pgfusepath{clip}%
\pgfsetrectcap%
\pgfsetroundjoin%
\pgfsetlinewidth{1.505625pt}%
\definecolor{currentstroke}{rgb}{1.000000,0.000000,0.000000}%
\pgfsetstrokecolor{currentstroke}%
\pgfsetdash{}{0pt}%
\pgfpathmoveto{\pgfqpoint{0.965252in}{0.375000in}}%
\pgfpathlineto{\pgfqpoint{0.965252in}{2.640000in}}%
\pgfusepath{stroke}%
\end{pgfscope}%
\begin{pgfscope}%
\pgfpathrectangle{\pgfqpoint{0.500000in}{0.375000in}}{\pgfqpoint{3.100000in}{2.265000in}}%
\pgfusepath{clip}%
\pgfsetrectcap%
\pgfsetroundjoin%
\pgfsetlinewidth{1.505625pt}%
\definecolor{currentstroke}{rgb}{1.000000,0.000000,0.000000}%
\pgfsetstrokecolor{currentstroke}%
\pgfsetdash{}{0pt}%
\pgfpathmoveto{\pgfqpoint{1.909451in}{0.375000in}}%
\pgfpathlineto{\pgfqpoint{1.909451in}{2.640000in}}%
\pgfusepath{stroke}%
\end{pgfscope}%
\begin{pgfscope}%
\pgfpathrectangle{\pgfqpoint{0.500000in}{0.375000in}}{\pgfqpoint{3.100000in}{2.265000in}}%
\pgfusepath{clip}%
\pgfsetrectcap%
\pgfsetroundjoin%
\pgfsetlinewidth{1.505625pt}%
\definecolor{currentstroke}{rgb}{1.000000,0.000000,0.000000}%
\pgfsetstrokecolor{currentstroke}%
\pgfsetdash{}{0pt}%
\pgfpathmoveto{\pgfqpoint{2.709498in}{0.375000in}}%
\pgfpathlineto{\pgfqpoint{2.709498in}{2.640000in}}%
\pgfusepath{stroke}%
\end{pgfscope}%
\begin{pgfscope}%
\pgfsetrectcap%
\pgfsetmiterjoin%
\pgfsetlinewidth{0.803000pt}%
\definecolor{currentstroke}{rgb}{0.000000,0.000000,0.000000}%
\pgfsetstrokecolor{currentstroke}%
\pgfsetdash{}{0pt}%
\pgfpathmoveto{\pgfqpoint{0.500000in}{0.375000in}}%
\pgfpathlineto{\pgfqpoint{0.500000in}{2.640000in}}%
\pgfusepath{stroke}%
\end{pgfscope}%
\begin{pgfscope}%
\pgfsetrectcap%
\pgfsetmiterjoin%
\pgfsetlinewidth{0.803000pt}%
\definecolor{currentstroke}{rgb}{0.000000,0.000000,0.000000}%
\pgfsetstrokecolor{currentstroke}%
\pgfsetdash{}{0pt}%
\pgfpathmoveto{\pgfqpoint{3.600000in}{0.375000in}}%
\pgfpathlineto{\pgfqpoint{3.600000in}{2.640000in}}%
\pgfusepath{stroke}%
\end{pgfscope}%
\begin{pgfscope}%
\pgfsetrectcap%
\pgfsetmiterjoin%
\pgfsetlinewidth{0.803000pt}%
\definecolor{currentstroke}{rgb}{0.000000,0.000000,0.000000}%
\pgfsetstrokecolor{currentstroke}%
\pgfsetdash{}{0pt}%
\pgfpathmoveto{\pgfqpoint{0.500000in}{0.375000in}}%
\pgfpathlineto{\pgfqpoint{3.600000in}{0.375000in}}%
\pgfusepath{stroke}%
\end{pgfscope}%
\begin{pgfscope}%
\pgfsetrectcap%
\pgfsetmiterjoin%
\pgfsetlinewidth{0.803000pt}%
\definecolor{currentstroke}{rgb}{0.000000,0.000000,0.000000}%
\pgfsetstrokecolor{currentstroke}%
\pgfsetdash{}{0pt}%
\pgfpathmoveto{\pgfqpoint{0.500000in}{2.640000in}}%
\pgfpathlineto{\pgfqpoint{3.600000in}{2.640000in}}%
\pgfusepath{stroke}%
\end{pgfscope}%
\begin{pgfscope}%
\pgfsetbuttcap%
\pgfsetmiterjoin%
\definecolor{currentfill}{rgb}{1.000000,1.000000,1.000000}%
\pgfsetfillcolor{currentfill}%
\pgfsetfillopacity{0.800000}%
\pgfsetlinewidth{1.003750pt}%
\definecolor{currentstroke}{rgb}{0.800000,0.800000,0.800000}%
\pgfsetstrokecolor{currentstroke}%
\pgfsetstrokeopacity{0.800000}%
\pgfsetdash{}{0pt}%
\pgfpathmoveto{\pgfqpoint{2.768981in}{2.335216in}}%
\pgfpathlineto{\pgfqpoint{3.502778in}{2.335216in}}%
\pgfpathquadraticcurveto{\pgfqpoint{3.530556in}{2.335216in}}{\pgfqpoint{3.530556in}{2.362994in}}%
\pgfpathlineto{\pgfqpoint{3.530556in}{2.542778in}}%
\pgfpathquadraticcurveto{\pgfqpoint{3.530556in}{2.570556in}}{\pgfqpoint{3.502778in}{2.570556in}}%
\pgfpathlineto{\pgfqpoint{2.768981in}{2.570556in}}%
\pgfpathquadraticcurveto{\pgfqpoint{2.741203in}{2.570556in}}{\pgfqpoint{2.741203in}{2.542778in}}%
\pgfpathlineto{\pgfqpoint{2.741203in}{2.362994in}}%
\pgfpathquadraticcurveto{\pgfqpoint{2.741203in}{2.335216in}}{\pgfqpoint{2.768981in}{2.335216in}}%
\pgfpathlineto{\pgfqpoint{2.768981in}{2.335216in}}%
\pgfpathclose%
\pgfusepath{stroke,fill}%
\end{pgfscope}%
\begin{pgfscope}%
\pgfsetrectcap%
\pgfsetroundjoin%
\pgfsetlinewidth{1.505625pt}%
\definecolor{currentstroke}{rgb}{0.121569,0.466667,0.705882}%
\pgfsetstrokecolor{currentstroke}%
\pgfsetdash{}{0pt}%
\pgfpathmoveto{\pgfqpoint{2.796758in}{2.466389in}}%
\pgfpathlineto{\pgfqpoint{2.935647in}{2.466389in}}%
\pgfpathlineto{\pgfqpoint{3.074536in}{2.466389in}}%
\pgfusepath{stroke}%
\end{pgfscope}%
\begin{pgfscope}%
\definecolor{textcolor}{rgb}{0.000000,0.000000,0.000000}%
\pgfsetstrokecolor{textcolor}%
\pgfsetfillcolor{textcolor}%
\pgftext[x=3.185647in,y=2.417778in,left,base]{\color{textcolor}\rmfamily\fontsize{10.000000}{12.000000}\selectfont pA:1}%
\end{pgfscope}%
\end{pgfpicture}%
\makeatother%
\endgroup%
}
%         \caption{Pressure Anomalies New CS}
%         \label{fig:hydraulic_sim_signal_pressure}
%     \end{minipage}
% \end{figure}

Figures \ref{fig:mp_hist_standard_pressure} and \ref{fig:mp_hist_signal_pressure} shows the values calculated during the matrix profile for the pressure generated by the system with the original control signal (figure \ref{fig:hydraulic_sim_standard_pressure}) and modified control signal (figure \ref{fig:hydraulic_sim_signal_pressure}) respectively. As in the previous signal, the values remain relatively similar until the differences in the two signals occur near time step 150. The waveform of the matrix profile values is identical to the previous force plots with the difference in the magnitude as is noted above. This provides insight into the robustness of the algorithm against different macro data scales of signals with the same shape.

% \begin{figure}[H]
%     \begin{minipage}[t]{0.5\linewidth}
%         %%\centering
%         \resizebox{\linewidth}{!}{%% Creator: Matplotlib, PGF backend
%%
%% To include the figure in your LaTeX document, write
%%   \input{<filename>.pgf}
%%
%% Make sure the required packages are loaded in your preamble
%%   \usepackage{pgf}
%%
%% Also ensure that all the required font packages are loaded; for instance,
%% the lmodern package is sometimes necessary when using math font.
%%   \usepackage{lmodern}
%%
%% Figures using additional raster images can only be included by \input if
%% they are in the same directory as the main LaTeX file. For loading figures
%% from other directories you can use the `import` package
%%   \usepackage{import}
%%
%% and then include the figures with
%%   \import{<path to file>}{<filename>.pgf}
%%
%% Matplotlib used the following preamble
%%
\begingroup%
\makeatletter%
\begin{pgfpicture}%
\pgfpathrectangle{\pgfpointorigin}{\pgfqpoint{4.000000in}{3.000000in}}%
\pgfusepath{use as bounding box, clip}%
\begin{pgfscope}%
\pgfsetbuttcap%
\pgfsetmiterjoin%
\pgfsetlinewidth{0.000000pt}%
\definecolor{currentstroke}{rgb}{1.000000,1.000000,1.000000}%
\pgfsetstrokecolor{currentstroke}%
\pgfsetstrokeopacity{0.000000}%
\pgfsetdash{}{0pt}%
\pgfpathmoveto{\pgfqpoint{0.000000in}{0.000000in}}%
\pgfpathlineto{\pgfqpoint{4.000000in}{0.000000in}}%
\pgfpathlineto{\pgfqpoint{4.000000in}{3.000000in}}%
\pgfpathlineto{\pgfqpoint{0.000000in}{3.000000in}}%
\pgfpathlineto{\pgfqpoint{0.000000in}{0.000000in}}%
\pgfpathclose%
\pgfusepath{}%
\end{pgfscope}%
\begin{pgfscope}%
\pgfsetbuttcap%
\pgfsetmiterjoin%
\definecolor{currentfill}{rgb}{1.000000,1.000000,1.000000}%
\pgfsetfillcolor{currentfill}%
\pgfsetlinewidth{0.000000pt}%
\definecolor{currentstroke}{rgb}{0.000000,0.000000,0.000000}%
\pgfsetstrokecolor{currentstroke}%
\pgfsetstrokeopacity{0.000000}%
\pgfsetdash{}{0pt}%
\pgfpathmoveto{\pgfqpoint{0.500000in}{0.375000in}}%
\pgfpathlineto{\pgfqpoint{3.600000in}{0.375000in}}%
\pgfpathlineto{\pgfqpoint{3.600000in}{2.640000in}}%
\pgfpathlineto{\pgfqpoint{0.500000in}{2.640000in}}%
\pgfpathlineto{\pgfqpoint{0.500000in}{0.375000in}}%
\pgfpathclose%
\pgfusepath{fill}%
\end{pgfscope}%
\begin{pgfscope}%
\pgfsetbuttcap%
\pgfsetroundjoin%
\definecolor{currentfill}{rgb}{0.000000,0.000000,0.000000}%
\pgfsetfillcolor{currentfill}%
\pgfsetlinewidth{0.803000pt}%
\definecolor{currentstroke}{rgb}{0.000000,0.000000,0.000000}%
\pgfsetstrokecolor{currentstroke}%
\pgfsetdash{}{0pt}%
\pgfsys@defobject{currentmarker}{\pgfqpoint{0.000000in}{-0.048611in}}{\pgfqpoint{0.000000in}{0.000000in}}{%
\pgfpathmoveto{\pgfqpoint{0.000000in}{0.000000in}}%
\pgfpathlineto{\pgfqpoint{0.000000in}{-0.048611in}}%
\pgfusepath{stroke,fill}%
}%
\begin{pgfscope}%
\pgfsys@transformshift{0.640909in}{0.375000in}%
\pgfsys@useobject{currentmarker}{}%
\end{pgfscope}%
\end{pgfscope}%
\begin{pgfscope}%
\definecolor{textcolor}{rgb}{0.000000,0.000000,0.000000}%
\pgfsetstrokecolor{textcolor}%
\pgfsetfillcolor{textcolor}%
\pgftext[x=0.640909in,y=0.277778in,,top]{\color{textcolor}\rmfamily\fontsize{10.000000}{12.000000}\selectfont \(\displaystyle {0}\)}%
\end{pgfscope}%
\begin{pgfscope}%
\pgfsetbuttcap%
\pgfsetroundjoin%
\definecolor{currentfill}{rgb}{0.000000,0.000000,0.000000}%
\pgfsetfillcolor{currentfill}%
\pgfsetlinewidth{0.803000pt}%
\definecolor{currentstroke}{rgb}{0.000000,0.000000,0.000000}%
\pgfsetstrokecolor{currentstroke}%
\pgfsetdash{}{0pt}%
\pgfsys@defobject{currentmarker}{\pgfqpoint{0.000000in}{-0.048611in}}{\pgfqpoint{0.000000in}{0.000000in}}{%
\pgfpathmoveto{\pgfqpoint{0.000000in}{0.000000in}}%
\pgfpathlineto{\pgfqpoint{0.000000in}{-0.048611in}}%
\pgfusepath{stroke,fill}%
}%
\begin{pgfscope}%
\pgfsys@transformshift{1.167672in}{0.375000in}%
\pgfsys@useobject{currentmarker}{}%
\end{pgfscope}%
\end{pgfscope}%
\begin{pgfscope}%
\definecolor{textcolor}{rgb}{0.000000,0.000000,0.000000}%
\pgfsetstrokecolor{textcolor}%
\pgfsetfillcolor{textcolor}%
\pgftext[x=1.167672in,y=0.277778in,,top]{\color{textcolor}\rmfamily\fontsize{10.000000}{12.000000}\selectfont \(\displaystyle {100}\)}%
\end{pgfscope}%
\begin{pgfscope}%
\pgfsetbuttcap%
\pgfsetroundjoin%
\definecolor{currentfill}{rgb}{0.000000,0.000000,0.000000}%
\pgfsetfillcolor{currentfill}%
\pgfsetlinewidth{0.803000pt}%
\definecolor{currentstroke}{rgb}{0.000000,0.000000,0.000000}%
\pgfsetstrokecolor{currentstroke}%
\pgfsetdash{}{0pt}%
\pgfsys@defobject{currentmarker}{\pgfqpoint{0.000000in}{-0.048611in}}{\pgfqpoint{0.000000in}{0.000000in}}{%
\pgfpathmoveto{\pgfqpoint{0.000000in}{0.000000in}}%
\pgfpathlineto{\pgfqpoint{0.000000in}{-0.048611in}}%
\pgfusepath{stroke,fill}%
}%
\begin{pgfscope}%
\pgfsys@transformshift{1.694435in}{0.375000in}%
\pgfsys@useobject{currentmarker}{}%
\end{pgfscope}%
\end{pgfscope}%
\begin{pgfscope}%
\definecolor{textcolor}{rgb}{0.000000,0.000000,0.000000}%
\pgfsetstrokecolor{textcolor}%
\pgfsetfillcolor{textcolor}%
\pgftext[x=1.694435in,y=0.277778in,,top]{\color{textcolor}\rmfamily\fontsize{10.000000}{12.000000}\selectfont \(\displaystyle {200}\)}%
\end{pgfscope}%
\begin{pgfscope}%
\pgfsetbuttcap%
\pgfsetroundjoin%
\definecolor{currentfill}{rgb}{0.000000,0.000000,0.000000}%
\pgfsetfillcolor{currentfill}%
\pgfsetlinewidth{0.803000pt}%
\definecolor{currentstroke}{rgb}{0.000000,0.000000,0.000000}%
\pgfsetstrokecolor{currentstroke}%
\pgfsetdash{}{0pt}%
\pgfsys@defobject{currentmarker}{\pgfqpoint{0.000000in}{-0.048611in}}{\pgfqpoint{0.000000in}{0.000000in}}{%
\pgfpathmoveto{\pgfqpoint{0.000000in}{0.000000in}}%
\pgfpathlineto{\pgfqpoint{0.000000in}{-0.048611in}}%
\pgfusepath{stroke,fill}%
}%
\begin{pgfscope}%
\pgfsys@transformshift{2.221198in}{0.375000in}%
\pgfsys@useobject{currentmarker}{}%
\end{pgfscope}%
\end{pgfscope}%
\begin{pgfscope}%
\definecolor{textcolor}{rgb}{0.000000,0.000000,0.000000}%
\pgfsetstrokecolor{textcolor}%
\pgfsetfillcolor{textcolor}%
\pgftext[x=2.221198in,y=0.277778in,,top]{\color{textcolor}\rmfamily\fontsize{10.000000}{12.000000}\selectfont \(\displaystyle {300}\)}%
\end{pgfscope}%
\begin{pgfscope}%
\pgfsetbuttcap%
\pgfsetroundjoin%
\definecolor{currentfill}{rgb}{0.000000,0.000000,0.000000}%
\pgfsetfillcolor{currentfill}%
\pgfsetlinewidth{0.803000pt}%
\definecolor{currentstroke}{rgb}{0.000000,0.000000,0.000000}%
\pgfsetstrokecolor{currentstroke}%
\pgfsetdash{}{0pt}%
\pgfsys@defobject{currentmarker}{\pgfqpoint{0.000000in}{-0.048611in}}{\pgfqpoint{0.000000in}{0.000000in}}{%
\pgfpathmoveto{\pgfqpoint{0.000000in}{0.000000in}}%
\pgfpathlineto{\pgfqpoint{0.000000in}{-0.048611in}}%
\pgfusepath{stroke,fill}%
}%
\begin{pgfscope}%
\pgfsys@transformshift{2.747961in}{0.375000in}%
\pgfsys@useobject{currentmarker}{}%
\end{pgfscope}%
\end{pgfscope}%
\begin{pgfscope}%
\definecolor{textcolor}{rgb}{0.000000,0.000000,0.000000}%
\pgfsetstrokecolor{textcolor}%
\pgfsetfillcolor{textcolor}%
\pgftext[x=2.747961in,y=0.277778in,,top]{\color{textcolor}\rmfamily\fontsize{10.000000}{12.000000}\selectfont \(\displaystyle {400}\)}%
\end{pgfscope}%
\begin{pgfscope}%
\pgfsetbuttcap%
\pgfsetroundjoin%
\definecolor{currentfill}{rgb}{0.000000,0.000000,0.000000}%
\pgfsetfillcolor{currentfill}%
\pgfsetlinewidth{0.803000pt}%
\definecolor{currentstroke}{rgb}{0.000000,0.000000,0.000000}%
\pgfsetstrokecolor{currentstroke}%
\pgfsetdash{}{0pt}%
\pgfsys@defobject{currentmarker}{\pgfqpoint{0.000000in}{-0.048611in}}{\pgfqpoint{0.000000in}{0.000000in}}{%
\pgfpathmoveto{\pgfqpoint{0.000000in}{0.000000in}}%
\pgfpathlineto{\pgfqpoint{0.000000in}{-0.048611in}}%
\pgfusepath{stroke,fill}%
}%
\begin{pgfscope}%
\pgfsys@transformshift{3.274724in}{0.375000in}%
\pgfsys@useobject{currentmarker}{}%
\end{pgfscope}%
\end{pgfscope}%
\begin{pgfscope}%
\definecolor{textcolor}{rgb}{0.000000,0.000000,0.000000}%
\pgfsetstrokecolor{textcolor}%
\pgfsetfillcolor{textcolor}%
\pgftext[x=3.274724in,y=0.277778in,,top]{\color{textcolor}\rmfamily\fontsize{10.000000}{12.000000}\selectfont \(\displaystyle {500}\)}%
\end{pgfscope}%
\begin{pgfscope}%
\definecolor{textcolor}{rgb}{0.000000,0.000000,0.000000}%
\pgfsetstrokecolor{textcolor}%
\pgfsetfillcolor{textcolor}%
\pgftext[x=2.050000in,y=0.098766in,,top]{\color{textcolor}\rmfamily\fontsize{10.000000}{12.000000}\selectfont time}%
\end{pgfscope}%
\begin{pgfscope}%
\pgfsetbuttcap%
\pgfsetroundjoin%
\definecolor{currentfill}{rgb}{0.000000,0.000000,0.000000}%
\pgfsetfillcolor{currentfill}%
\pgfsetlinewidth{0.803000pt}%
\definecolor{currentstroke}{rgb}{0.000000,0.000000,0.000000}%
\pgfsetstrokecolor{currentstroke}%
\pgfsetdash{}{0pt}%
\pgfsys@defobject{currentmarker}{\pgfqpoint{-0.048611in}{0.000000in}}{\pgfqpoint{-0.000000in}{0.000000in}}{%
\pgfpathmoveto{\pgfqpoint{-0.000000in}{0.000000in}}%
\pgfpathlineto{\pgfqpoint{-0.048611in}{0.000000in}}%
\pgfusepath{stroke,fill}%
}%
\begin{pgfscope}%
\pgfsys@transformshift{0.500000in}{0.477952in}%
\pgfsys@useobject{currentmarker}{}%
\end{pgfscope}%
\end{pgfscope}%
\begin{pgfscope}%
\definecolor{textcolor}{rgb}{0.000000,0.000000,0.000000}%
\pgfsetstrokecolor{textcolor}%
\pgfsetfillcolor{textcolor}%
\pgftext[x=0.333333in, y=0.429727in, left, base]{\color{textcolor}\rmfamily\fontsize{10.000000}{12.000000}\selectfont \(\displaystyle {0}\)}%
\end{pgfscope}%
\begin{pgfscope}%
\pgfsetbuttcap%
\pgfsetroundjoin%
\definecolor{currentfill}{rgb}{0.000000,0.000000,0.000000}%
\pgfsetfillcolor{currentfill}%
\pgfsetlinewidth{0.803000pt}%
\definecolor{currentstroke}{rgb}{0.000000,0.000000,0.000000}%
\pgfsetstrokecolor{currentstroke}%
\pgfsetdash{}{0pt}%
\pgfsys@defobject{currentmarker}{\pgfqpoint{-0.048611in}{0.000000in}}{\pgfqpoint{-0.000000in}{0.000000in}}{%
\pgfpathmoveto{\pgfqpoint{-0.000000in}{0.000000in}}%
\pgfpathlineto{\pgfqpoint{-0.048611in}{0.000000in}}%
\pgfusepath{stroke,fill}%
}%
\begin{pgfscope}%
\pgfsys@transformshift{0.500000in}{1.018569in}%
\pgfsys@useobject{currentmarker}{}%
\end{pgfscope}%
\end{pgfscope}%
\begin{pgfscope}%
\definecolor{textcolor}{rgb}{0.000000,0.000000,0.000000}%
\pgfsetstrokecolor{textcolor}%
\pgfsetfillcolor{textcolor}%
\pgftext[x=0.333333in, y=0.970344in, left, base]{\color{textcolor}\rmfamily\fontsize{10.000000}{12.000000}\selectfont \(\displaystyle {2}\)}%
\end{pgfscope}%
\begin{pgfscope}%
\pgfsetbuttcap%
\pgfsetroundjoin%
\definecolor{currentfill}{rgb}{0.000000,0.000000,0.000000}%
\pgfsetfillcolor{currentfill}%
\pgfsetlinewidth{0.803000pt}%
\definecolor{currentstroke}{rgb}{0.000000,0.000000,0.000000}%
\pgfsetstrokecolor{currentstroke}%
\pgfsetdash{}{0pt}%
\pgfsys@defobject{currentmarker}{\pgfqpoint{-0.048611in}{0.000000in}}{\pgfqpoint{-0.000000in}{0.000000in}}{%
\pgfpathmoveto{\pgfqpoint{-0.000000in}{0.000000in}}%
\pgfpathlineto{\pgfqpoint{-0.048611in}{0.000000in}}%
\pgfusepath{stroke,fill}%
}%
\begin{pgfscope}%
\pgfsys@transformshift{0.500000in}{1.559186in}%
\pgfsys@useobject{currentmarker}{}%
\end{pgfscope}%
\end{pgfscope}%
\begin{pgfscope}%
\definecolor{textcolor}{rgb}{0.000000,0.000000,0.000000}%
\pgfsetstrokecolor{textcolor}%
\pgfsetfillcolor{textcolor}%
\pgftext[x=0.333333in, y=1.510961in, left, base]{\color{textcolor}\rmfamily\fontsize{10.000000}{12.000000}\selectfont \(\displaystyle {4}\)}%
\end{pgfscope}%
\begin{pgfscope}%
\pgfsetbuttcap%
\pgfsetroundjoin%
\definecolor{currentfill}{rgb}{0.000000,0.000000,0.000000}%
\pgfsetfillcolor{currentfill}%
\pgfsetlinewidth{0.803000pt}%
\definecolor{currentstroke}{rgb}{0.000000,0.000000,0.000000}%
\pgfsetstrokecolor{currentstroke}%
\pgfsetdash{}{0pt}%
\pgfsys@defobject{currentmarker}{\pgfqpoint{-0.048611in}{0.000000in}}{\pgfqpoint{-0.000000in}{0.000000in}}{%
\pgfpathmoveto{\pgfqpoint{-0.000000in}{0.000000in}}%
\pgfpathlineto{\pgfqpoint{-0.048611in}{0.000000in}}%
\pgfusepath{stroke,fill}%
}%
\begin{pgfscope}%
\pgfsys@transformshift{0.500000in}{2.099804in}%
\pgfsys@useobject{currentmarker}{}%
\end{pgfscope}%
\end{pgfscope}%
\begin{pgfscope}%
\definecolor{textcolor}{rgb}{0.000000,0.000000,0.000000}%
\pgfsetstrokecolor{textcolor}%
\pgfsetfillcolor{textcolor}%
\pgftext[x=0.333333in, y=2.051578in, left, base]{\color{textcolor}\rmfamily\fontsize{10.000000}{12.000000}\selectfont \(\displaystyle {6}\)}%
\end{pgfscope}%
\begin{pgfscope}%
\definecolor{textcolor}{rgb}{0.000000,0.000000,0.000000}%
\pgfsetstrokecolor{textcolor}%
\pgfsetfillcolor{textcolor}%
\pgftext[x=0.500000in,y=2.681667in,left,base]{\color{textcolor}\rmfamily\fontsize{10.000000}{12.000000}\selectfont \(\displaystyle \times{10^{6}}{}\)}%
\end{pgfscope}%
\begin{pgfscope}%
\pgfpathrectangle{\pgfqpoint{0.500000in}{0.375000in}}{\pgfqpoint{3.100000in}{2.265000in}}%
\pgfusepath{clip}%
\pgfsetrectcap%
\pgfsetroundjoin%
\pgfsetlinewidth{1.505625pt}%
\definecolor{currentstroke}{rgb}{0.000000,0.000000,1.000000}%
\pgfsetstrokecolor{currentstroke}%
\pgfsetdash{}{0pt}%
\pgfpathmoveto{\pgfqpoint{0.640909in}{0.477963in}}%
\pgfpathlineto{\pgfqpoint{0.698853in}{0.478783in}}%
\pgfpathlineto{\pgfqpoint{0.709388in}{0.481613in}}%
\pgfpathlineto{\pgfqpoint{0.719924in}{0.488516in}}%
\pgfpathlineto{\pgfqpoint{0.730459in}{0.499421in}}%
\pgfpathlineto{\pgfqpoint{0.746262in}{0.519995in}}%
\pgfpathlineto{\pgfqpoint{0.762065in}{0.544113in}}%
\pgfpathlineto{\pgfqpoint{0.772600in}{0.564328in}}%
\pgfpathlineto{\pgfqpoint{0.783135in}{0.592523in}}%
\pgfpathlineto{\pgfqpoint{0.793670in}{0.633741in}}%
\pgfpathlineto{\pgfqpoint{0.804206in}{0.692535in}}%
\pgfpathlineto{\pgfqpoint{0.814741in}{0.776579in}}%
\pgfpathlineto{\pgfqpoint{0.825276in}{0.900692in}}%
\pgfpathlineto{\pgfqpoint{0.841079in}{1.102582in}}%
\pgfpathlineto{\pgfqpoint{0.851614in}{1.190105in}}%
\pgfpathlineto{\pgfqpoint{0.856882in}{1.216810in}}%
\pgfpathlineto{\pgfqpoint{0.862150in}{1.232676in}}%
\pgfpathlineto{\pgfqpoint{0.867417in}{1.238579in}}%
\pgfpathlineto{\pgfqpoint{0.893755in}{1.238184in}}%
\pgfpathlineto{\pgfqpoint{0.899023in}{1.233212in}}%
\pgfpathlineto{\pgfqpoint{0.909558in}{1.216810in}}%
\pgfpathlineto{\pgfqpoint{0.930629in}{1.216810in}}%
\pgfpathlineto{\pgfqpoint{0.935896in}{1.306970in}}%
\pgfpathlineto{\pgfqpoint{0.951699in}{1.850056in}}%
\pgfpathlineto{\pgfqpoint{0.962234in}{2.077995in}}%
\pgfpathlineto{\pgfqpoint{0.967502in}{2.139502in}}%
\pgfpathlineto{\pgfqpoint{0.972770in}{2.171689in}}%
\pgfpathlineto{\pgfqpoint{0.978037in}{2.182862in}}%
\pgfpathlineto{\pgfqpoint{0.988573in}{2.183785in}}%
\pgfpathlineto{\pgfqpoint{0.993840in}{2.191599in}}%
\pgfpathlineto{\pgfqpoint{1.009643in}{2.245503in}}%
\pgfpathlineto{\pgfqpoint{1.030714in}{2.245503in}}%
\pgfpathlineto{\pgfqpoint{1.035981in}{2.242124in}}%
\pgfpathlineto{\pgfqpoint{1.046517in}{2.191599in}}%
\pgfpathlineto{\pgfqpoint{1.072855in}{2.191599in}}%
\pgfpathlineto{\pgfqpoint{1.078122in}{2.164930in}}%
\pgfpathlineto{\pgfqpoint{1.088658in}{2.020910in}}%
\pgfpathlineto{\pgfqpoint{1.099193in}{1.884092in}}%
\pgfpathlineto{\pgfqpoint{1.104460in}{1.849301in}}%
\pgfpathlineto{\pgfqpoint{1.109728in}{1.836383in}}%
\pgfpathlineto{\pgfqpoint{1.114996in}{1.835876in}}%
\pgfpathlineto{\pgfqpoint{1.120263in}{1.833566in}}%
\pgfpathlineto{\pgfqpoint{1.125531in}{1.817372in}}%
\pgfpathlineto{\pgfqpoint{1.130799in}{1.782295in}}%
\pgfpathlineto{\pgfqpoint{1.141334in}{1.669672in}}%
\pgfpathlineto{\pgfqpoint{1.146602in}{1.610162in}}%
\pgfpathlineto{\pgfqpoint{1.151869in}{1.645902in}}%
\pgfpathlineto{\pgfqpoint{1.162404in}{1.747402in}}%
\pgfpathlineto{\pgfqpoint{1.167672in}{1.779649in}}%
\pgfpathlineto{\pgfqpoint{1.172940in}{1.782884in}}%
\pgfpathlineto{\pgfqpoint{1.194010in}{1.782884in}}%
\pgfpathlineto{\pgfqpoint{1.199278in}{1.753903in}}%
\pgfpathlineto{\pgfqpoint{1.204545in}{1.696858in}}%
\pgfpathlineto{\pgfqpoint{1.230884in}{1.696858in}}%
\pgfpathlineto{\pgfqpoint{1.241419in}{1.622913in}}%
\pgfpathlineto{\pgfqpoint{1.246686in}{1.615681in}}%
\pgfpathlineto{\pgfqpoint{1.251954in}{1.612846in}}%
\pgfpathlineto{\pgfqpoint{1.257222in}{1.573404in}}%
\pgfpathlineto{\pgfqpoint{1.262489in}{1.549021in}}%
\pgfpathlineto{\pgfqpoint{1.273025in}{1.549021in}}%
\pgfpathlineto{\pgfqpoint{1.278292in}{1.503648in}}%
\pgfpathlineto{\pgfqpoint{1.283560in}{1.503648in}}%
\pgfpathlineto{\pgfqpoint{1.288828in}{1.471235in}}%
\pgfpathlineto{\pgfqpoint{1.294095in}{1.418524in}}%
\pgfpathlineto{\pgfqpoint{1.299363in}{1.416429in}}%
\pgfpathlineto{\pgfqpoint{1.304630in}{1.339509in}}%
\pgfpathlineto{\pgfqpoint{1.330969in}{1.339509in}}%
\pgfpathlineto{\pgfqpoint{1.336236in}{1.288989in}}%
\pgfpathlineto{\pgfqpoint{1.341504in}{1.279471in}}%
\pgfpathlineto{\pgfqpoint{1.346771in}{1.235914in}}%
\pgfpathlineto{\pgfqpoint{1.357307in}{1.094369in}}%
\pgfpathlineto{\pgfqpoint{1.362574in}{1.041749in}}%
\pgfpathlineto{\pgfqpoint{1.367842in}{1.010016in}}%
\pgfpathlineto{\pgfqpoint{1.373110in}{1.001558in}}%
\pgfpathlineto{\pgfqpoint{1.388912in}{1.001558in}}%
\pgfpathlineto{\pgfqpoint{1.394180in}{0.996817in}}%
\pgfpathlineto{\pgfqpoint{1.399448in}{0.966558in}}%
\pgfpathlineto{\pgfqpoint{1.404715in}{0.957312in}}%
\pgfpathlineto{\pgfqpoint{1.431054in}{1.473498in}}%
\pgfpathlineto{\pgfqpoint{1.436321in}{1.529264in}}%
\pgfpathlineto{\pgfqpoint{1.441589in}{1.561135in}}%
\pgfpathlineto{\pgfqpoint{1.446856in}{1.575484in}}%
\pgfpathlineto{\pgfqpoint{1.452124in}{1.580167in}}%
\pgfpathlineto{\pgfqpoint{1.473195in}{1.580640in}}%
\pgfpathlineto{\pgfqpoint{1.483730in}{1.579589in}}%
\pgfpathlineto{\pgfqpoint{1.488997in}{1.577511in}}%
\pgfpathlineto{\pgfqpoint{1.494265in}{1.572356in}}%
\pgfpathlineto{\pgfqpoint{1.525871in}{1.572356in}}%
\pgfpathlineto{\pgfqpoint{1.531138in}{1.556994in}}%
\pgfpathlineto{\pgfqpoint{1.536406in}{1.524041in}}%
\pgfpathlineto{\pgfqpoint{1.541674in}{1.465468in}}%
\pgfpathlineto{\pgfqpoint{1.546941in}{1.374721in}}%
\pgfpathlineto{\pgfqpoint{1.557477in}{1.112057in}}%
\pgfpathlineto{\pgfqpoint{1.573280in}{0.690346in}}%
\pgfpathlineto{\pgfqpoint{1.578547in}{0.594862in}}%
\pgfpathlineto{\pgfqpoint{1.583815in}{0.549326in}}%
\pgfpathlineto{\pgfqpoint{1.594350in}{0.565878in}}%
\pgfpathlineto{\pgfqpoint{1.599618in}{0.568494in}}%
\pgfpathlineto{\pgfqpoint{1.610153in}{0.585583in}}%
\pgfpathlineto{\pgfqpoint{1.615421in}{0.588266in}}%
\pgfpathlineto{\pgfqpoint{1.625956in}{0.605679in}}%
\pgfpathlineto{\pgfqpoint{1.631223in}{0.608386in}}%
\pgfpathlineto{\pgfqpoint{1.641759in}{0.625791in}}%
\pgfpathlineto{\pgfqpoint{1.647026in}{0.628464in}}%
\pgfpathlineto{\pgfqpoint{1.657562in}{0.646026in}}%
\pgfpathlineto{\pgfqpoint{1.662829in}{0.648772in}}%
\pgfpathlineto{\pgfqpoint{1.673364in}{0.666642in}}%
\pgfpathlineto{\pgfqpoint{1.678632in}{0.669386in}}%
\pgfpathlineto{\pgfqpoint{1.689167in}{0.686012in}}%
\pgfpathlineto{\pgfqpoint{1.694435in}{0.688400in}}%
\pgfpathlineto{\pgfqpoint{1.704970in}{0.702612in}}%
\pgfpathlineto{\pgfqpoint{1.710238in}{0.704660in}}%
\pgfpathlineto{\pgfqpoint{1.720773in}{0.716449in}}%
\pgfpathlineto{\pgfqpoint{1.726041in}{0.718137in}}%
\pgfpathlineto{\pgfqpoint{1.731308in}{0.722520in}}%
\pgfpathlineto{\pgfqpoint{1.736576in}{0.757939in}}%
\pgfpathlineto{\pgfqpoint{1.741844in}{0.781546in}}%
\pgfpathlineto{\pgfqpoint{1.747111in}{0.797228in}}%
\pgfpathlineto{\pgfqpoint{1.752379in}{0.825173in}}%
\pgfpathlineto{\pgfqpoint{1.762914in}{0.904994in}}%
\pgfpathlineto{\pgfqpoint{1.773449in}{1.064479in}}%
\pgfpathlineto{\pgfqpoint{1.778717in}{1.115994in}}%
\pgfpathlineto{\pgfqpoint{1.783985in}{1.150794in}}%
\pgfpathlineto{\pgfqpoint{1.789252in}{1.168297in}}%
\pgfpathlineto{\pgfqpoint{1.831393in}{1.167284in}}%
\pgfpathlineto{\pgfqpoint{1.836661in}{1.155671in}}%
\pgfpathlineto{\pgfqpoint{1.868267in}{1.155671in}}%
\pgfpathlineto{\pgfqpoint{1.873534in}{1.150014in}}%
\pgfpathlineto{\pgfqpoint{1.878802in}{1.141124in}}%
\pgfpathlineto{\pgfqpoint{1.884070in}{1.128294in}}%
\pgfpathlineto{\pgfqpoint{1.894605in}{1.081489in}}%
\pgfpathlineto{\pgfqpoint{1.905140in}{0.935355in}}%
\pgfpathlineto{\pgfqpoint{1.915675in}{0.791127in}}%
\pgfpathlineto{\pgfqpoint{1.920943in}{0.759902in}}%
\pgfpathlineto{\pgfqpoint{1.957816in}{0.755759in}}%
\pgfpathlineto{\pgfqpoint{2.010493in}{0.739647in}}%
\pgfpathlineto{\pgfqpoint{2.015760in}{0.736869in}}%
\pgfpathlineto{\pgfqpoint{2.021028in}{0.735925in}}%
\pgfpathlineto{\pgfqpoint{2.031563in}{0.715770in}}%
\pgfpathlineto{\pgfqpoint{2.036831in}{0.714118in}}%
\pgfpathlineto{\pgfqpoint{2.042099in}{0.704491in}}%
\pgfpathlineto{\pgfqpoint{2.047366in}{0.698098in}}%
\pgfpathlineto{\pgfqpoint{2.052634in}{0.698098in}}%
\pgfpathlineto{\pgfqpoint{2.057901in}{0.695700in}}%
\pgfpathlineto{\pgfqpoint{2.063169in}{0.687425in}}%
\pgfpathlineto{\pgfqpoint{2.068437in}{0.687425in}}%
\pgfpathlineto{\pgfqpoint{2.073704in}{0.682913in}}%
\pgfpathlineto{\pgfqpoint{2.078972in}{0.672633in}}%
\pgfpathlineto{\pgfqpoint{2.084240in}{0.672633in}}%
\pgfpathlineto{\pgfqpoint{2.089507in}{0.666712in}}%
\pgfpathlineto{\pgfqpoint{2.094775in}{0.656871in}}%
\pgfpathlineto{\pgfqpoint{2.115845in}{0.655379in}}%
\pgfpathlineto{\pgfqpoint{2.126381in}{0.653556in}}%
\pgfpathlineto{\pgfqpoint{2.136916in}{0.634292in}}%
\pgfpathlineto{\pgfqpoint{2.142184in}{0.631176in}}%
\pgfpathlineto{\pgfqpoint{2.152719in}{0.614101in}}%
\pgfpathlineto{\pgfqpoint{2.157986in}{0.611472in}}%
\pgfpathlineto{\pgfqpoint{2.168522in}{0.595049in}}%
\pgfpathlineto{\pgfqpoint{2.173789in}{0.592490in}}%
\pgfpathlineto{\pgfqpoint{2.184325in}{0.576609in}}%
\pgfpathlineto{\pgfqpoint{2.189592in}{0.574113in}}%
\pgfpathlineto{\pgfqpoint{2.200127in}{0.558758in}}%
\pgfpathlineto{\pgfqpoint{2.205395in}{0.556332in}}%
\pgfpathlineto{\pgfqpoint{2.215930in}{0.541604in}}%
\pgfpathlineto{\pgfqpoint{2.221198in}{0.539278in}}%
\pgfpathlineto{\pgfqpoint{2.231733in}{0.525520in}}%
\pgfpathlineto{\pgfqpoint{2.237001in}{0.523382in}}%
\pgfpathlineto{\pgfqpoint{2.242268in}{0.519405in}}%
\pgfpathlineto{\pgfqpoint{2.247536in}{0.521324in}}%
\pgfpathlineto{\pgfqpoint{2.258071in}{0.534851in}}%
\pgfpathlineto{\pgfqpoint{2.263339in}{0.537051in}}%
\pgfpathlineto{\pgfqpoint{2.273874in}{0.551749in}}%
\pgfpathlineto{\pgfqpoint{2.279142in}{0.554088in}}%
\pgfpathlineto{\pgfqpoint{2.284410in}{0.561731in}}%
\pgfpathlineto{\pgfqpoint{2.289677in}{0.591714in}}%
\pgfpathlineto{\pgfqpoint{2.294945in}{0.636892in}}%
\pgfpathlineto{\pgfqpoint{2.300212in}{0.732013in}}%
\pgfpathlineto{\pgfqpoint{2.310748in}{1.101176in}}%
\pgfpathlineto{\pgfqpoint{2.331818in}{1.994810in}}%
\pgfpathlineto{\pgfqpoint{2.342353in}{2.264348in}}%
\pgfpathlineto{\pgfqpoint{2.347621in}{2.332353in}}%
\pgfpathlineto{\pgfqpoint{2.352889in}{2.364800in}}%
\pgfpathlineto{\pgfqpoint{2.358156in}{2.377190in}}%
\pgfpathlineto{\pgfqpoint{2.363424in}{2.380466in}}%
\pgfpathlineto{\pgfqpoint{2.389762in}{2.379581in}}%
\pgfpathlineto{\pgfqpoint{2.395030in}{2.376590in}}%
\pgfpathlineto{\pgfqpoint{2.400297in}{2.366366in}}%
\pgfpathlineto{\pgfqpoint{2.410833in}{2.364800in}}%
\pgfpathlineto{\pgfqpoint{2.431903in}{2.364800in}}%
\pgfpathlineto{\pgfqpoint{2.437171in}{2.336950in}}%
\pgfpathlineto{\pgfqpoint{2.442438in}{2.279972in}}%
\pgfpathlineto{\pgfqpoint{2.447706in}{2.291474in}}%
\pgfpathlineto{\pgfqpoint{2.452974in}{2.317963in}}%
\pgfpathlineto{\pgfqpoint{2.458241in}{2.329122in}}%
\pgfpathlineto{\pgfqpoint{2.463509in}{2.333501in}}%
\pgfpathlineto{\pgfqpoint{2.489847in}{2.333216in}}%
\pgfpathlineto{\pgfqpoint{2.495115in}{2.326628in}}%
\pgfpathlineto{\pgfqpoint{2.500382in}{2.304964in}}%
\pgfpathlineto{\pgfqpoint{2.531988in}{2.304964in}}%
\pgfpathlineto{\pgfqpoint{2.537256in}{2.257569in}}%
\pgfpathlineto{\pgfqpoint{2.542523in}{2.255891in}}%
\pgfpathlineto{\pgfqpoint{2.547791in}{2.316016in}}%
\pgfpathlineto{\pgfqpoint{2.553059in}{2.347963in}}%
\pgfpathlineto{\pgfqpoint{2.558326in}{2.359424in}}%
\pgfpathlineto{\pgfqpoint{2.574129in}{2.363491in}}%
\pgfpathlineto{\pgfqpoint{2.595200in}{2.385379in}}%
\pgfpathlineto{\pgfqpoint{2.600467in}{2.387205in}}%
\pgfpathlineto{\pgfqpoint{2.626805in}{2.386143in}}%
\pgfpathlineto{\pgfqpoint{2.632073in}{2.382243in}}%
\pgfpathlineto{\pgfqpoint{2.637341in}{2.376062in}}%
\pgfpathlineto{\pgfqpoint{2.700552in}{2.376748in}}%
\pgfpathlineto{\pgfqpoint{2.705820in}{2.374330in}}%
\pgfpathlineto{\pgfqpoint{2.711088in}{2.369300in}}%
\pgfpathlineto{\pgfqpoint{2.721623in}{2.367736in}}%
\pgfpathlineto{\pgfqpoint{2.742693in}{2.367736in}}%
\pgfpathlineto{\pgfqpoint{2.747961in}{2.362672in}}%
\pgfpathlineto{\pgfqpoint{2.753229in}{2.365722in}}%
\pgfpathlineto{\pgfqpoint{2.774299in}{2.365722in}}%
\pgfpathlineto{\pgfqpoint{2.779567in}{2.346294in}}%
\pgfpathlineto{\pgfqpoint{2.790102in}{2.345456in}}%
\pgfpathlineto{\pgfqpoint{2.811172in}{2.345456in}}%
\pgfpathlineto{\pgfqpoint{2.816440in}{2.286974in}}%
\pgfpathlineto{\pgfqpoint{2.826975in}{2.114213in}}%
\pgfpathlineto{\pgfqpoint{2.837511in}{2.057836in}}%
\pgfpathlineto{\pgfqpoint{2.853314in}{2.057836in}}%
\pgfpathlineto{\pgfqpoint{2.858581in}{2.125869in}}%
\pgfpathlineto{\pgfqpoint{2.863849in}{2.259705in}}%
\pgfpathlineto{\pgfqpoint{2.869116in}{2.352558in}}%
\pgfpathlineto{\pgfqpoint{2.874384in}{2.397948in}}%
\pgfpathlineto{\pgfqpoint{2.879652in}{2.406003in}}%
\pgfpathlineto{\pgfqpoint{2.900722in}{2.406003in}}%
\pgfpathlineto{\pgfqpoint{2.905990in}{2.414089in}}%
\pgfpathlineto{\pgfqpoint{2.921793in}{2.519506in}}%
\pgfpathlineto{\pgfqpoint{2.927060in}{2.536597in}}%
\pgfpathlineto{\pgfqpoint{2.948131in}{2.537045in}}%
\pgfpathlineto{\pgfqpoint{2.953398in}{2.537045in}}%
\pgfpathlineto{\pgfqpoint{2.958666in}{2.520624in}}%
\pgfpathlineto{\pgfqpoint{2.963934in}{2.490249in}}%
\pgfpathlineto{\pgfqpoint{2.974469in}{2.489100in}}%
\pgfpathlineto{\pgfqpoint{2.990272in}{2.489707in}}%
\pgfpathlineto{\pgfqpoint{2.995540in}{2.515896in}}%
\pgfpathlineto{\pgfqpoint{3.000807in}{2.527345in}}%
\pgfpathlineto{\pgfqpoint{3.021878in}{2.527345in}}%
\pgfpathlineto{\pgfqpoint{3.027145in}{2.521808in}}%
\pgfpathlineto{\pgfqpoint{3.032413in}{2.500184in}}%
\pgfpathlineto{\pgfqpoint{3.037681in}{2.466390in}}%
\pgfpathlineto{\pgfqpoint{3.058751in}{2.466390in}}%
\pgfpathlineto{\pgfqpoint{3.069286in}{2.509621in}}%
\pgfpathlineto{\pgfqpoint{3.074554in}{2.516303in}}%
\pgfpathlineto{\pgfqpoint{3.095624in}{2.516303in}}%
\pgfpathlineto{\pgfqpoint{3.100892in}{2.506015in}}%
\pgfpathlineto{\pgfqpoint{3.111427in}{2.453808in}}%
\pgfpathlineto{\pgfqpoint{3.127230in}{2.453808in}}%
\pgfpathlineto{\pgfqpoint{3.132498in}{2.455396in}}%
\pgfpathlineto{\pgfqpoint{3.137766in}{2.485513in}}%
\pgfpathlineto{\pgfqpoint{3.143033in}{2.502616in}}%
\pgfpathlineto{\pgfqpoint{3.158836in}{2.503378in}}%
\pgfpathlineto{\pgfqpoint{3.169371in}{2.503378in}}%
\pgfpathlineto{\pgfqpoint{3.174639in}{2.487540in}}%
\pgfpathlineto{\pgfqpoint{3.179907in}{2.457919in}}%
\pgfpathlineto{\pgfqpoint{3.185174in}{2.455396in}}%
\pgfpathlineto{\pgfqpoint{3.206245in}{2.455396in}}%
\pgfpathlineto{\pgfqpoint{3.211512in}{2.480879in}}%
\pgfpathlineto{\pgfqpoint{3.216780in}{2.493473in}}%
\pgfpathlineto{\pgfqpoint{3.237850in}{2.493473in}}%
\pgfpathlineto{\pgfqpoint{3.243118in}{2.489423in}}%
\pgfpathlineto{\pgfqpoint{3.248386in}{2.469337in}}%
\pgfpathlineto{\pgfqpoint{3.253653in}{2.436840in}}%
\pgfpathlineto{\pgfqpoint{3.274724in}{2.436840in}}%
\pgfpathlineto{\pgfqpoint{3.279992in}{2.453787in}}%
\pgfpathlineto{\pgfqpoint{3.285259in}{2.475968in}}%
\pgfpathlineto{\pgfqpoint{3.290527in}{2.482857in}}%
\pgfpathlineto{\pgfqpoint{3.311597in}{2.482857in}}%
\pgfpathlineto{\pgfqpoint{3.316865in}{2.473062in}}%
\pgfpathlineto{\pgfqpoint{3.327400in}{2.420636in}}%
\pgfpathlineto{\pgfqpoint{3.348471in}{2.420636in}}%
\pgfpathlineto{\pgfqpoint{3.353738in}{2.450607in}}%
\pgfpathlineto{\pgfqpoint{3.359006in}{2.468648in}}%
\pgfpathlineto{\pgfqpoint{3.364274in}{2.470782in}}%
\pgfpathlineto{\pgfqpoint{3.385344in}{2.470782in}}%
\pgfpathlineto{\pgfqpoint{3.390612in}{2.456477in}}%
\pgfpathlineto{\pgfqpoint{3.395879in}{2.428254in}}%
\pgfpathlineto{\pgfqpoint{3.401147in}{2.420167in}}%
\pgfpathlineto{\pgfqpoint{3.422218in}{2.421025in}}%
\pgfpathlineto{\pgfqpoint{3.427485in}{2.447588in}}%
\pgfpathlineto{\pgfqpoint{3.432753in}{2.460283in}}%
\pgfpathlineto{\pgfqpoint{3.453823in}{2.460283in}}%
\pgfpathlineto{\pgfqpoint{3.459091in}{2.456628in}}%
\pgfpathlineto{\pgfqpoint{3.459091in}{2.456628in}}%
\pgfusepath{stroke}%
\end{pgfscope}%
\begin{pgfscope}%
\pgfpathrectangle{\pgfqpoint{0.500000in}{0.375000in}}{\pgfqpoint{3.100000in}{2.265000in}}%
\pgfusepath{clip}%
\pgfsetrectcap%
\pgfsetroundjoin%
\pgfsetlinewidth{1.505625pt}%
\definecolor{currentstroke}{rgb}{1.000000,0.000000,0.000000}%
\pgfsetstrokecolor{currentstroke}%
\pgfsetdash{}{0pt}%
\pgfpathmoveto{\pgfqpoint{0.640909in}{0.477955in}}%
\pgfpathlineto{\pgfqpoint{0.719924in}{0.479497in}}%
\pgfpathlineto{\pgfqpoint{0.735726in}{0.483552in}}%
\pgfpathlineto{\pgfqpoint{0.751529in}{0.491451in}}%
\pgfpathlineto{\pgfqpoint{0.767332in}{0.503905in}}%
\pgfpathlineto{\pgfqpoint{0.777867in}{0.515466in}}%
\pgfpathlineto{\pgfqpoint{0.788403in}{0.530826in}}%
\pgfpathlineto{\pgfqpoint{0.798938in}{0.551380in}}%
\pgfpathlineto{\pgfqpoint{0.809473in}{0.579030in}}%
\pgfpathlineto{\pgfqpoint{0.820008in}{0.616914in}}%
\pgfpathlineto{\pgfqpoint{0.830544in}{0.669370in}}%
\pgfpathlineto{\pgfqpoint{0.841079in}{0.736768in}}%
\pgfpathlineto{\pgfqpoint{0.893755in}{1.106020in}}%
\pgfpathlineto{\pgfqpoint{0.899023in}{1.126096in}}%
\pgfpathlineto{\pgfqpoint{0.904291in}{1.138141in}}%
\pgfpathlineto{\pgfqpoint{0.909558in}{1.140609in}}%
\pgfpathlineto{\pgfqpoint{0.914826in}{1.133592in}}%
\pgfpathlineto{\pgfqpoint{0.925361in}{1.107207in}}%
\pgfpathlineto{\pgfqpoint{0.930629in}{1.106628in}}%
\pgfpathlineto{\pgfqpoint{0.935896in}{1.113932in}}%
\pgfpathlineto{\pgfqpoint{0.941164in}{1.131267in}}%
\pgfpathlineto{\pgfqpoint{0.946432in}{1.160800in}}%
\pgfpathlineto{\pgfqpoint{0.956967in}{1.252602in}}%
\pgfpathlineto{\pgfqpoint{0.967502in}{1.377886in}}%
\pgfpathlineto{\pgfqpoint{1.020178in}{2.080444in}}%
\pgfpathlineto{\pgfqpoint{1.025446in}{2.119558in}}%
\pgfpathlineto{\pgfqpoint{1.030714in}{2.137207in}}%
\pgfpathlineto{\pgfqpoint{1.035981in}{2.133849in}}%
\pgfpathlineto{\pgfqpoint{1.041249in}{2.112816in}}%
\pgfpathlineto{\pgfqpoint{1.057052in}{2.004670in}}%
\pgfpathlineto{\pgfqpoint{1.067587in}{1.942668in}}%
\pgfpathlineto{\pgfqpoint{1.078122in}{1.885828in}}%
\pgfpathlineto{\pgfqpoint{1.093925in}{1.775157in}}%
\pgfpathlineto{\pgfqpoint{1.109728in}{1.663014in}}%
\pgfpathlineto{\pgfqpoint{1.120263in}{1.614933in}}%
\pgfpathlineto{\pgfqpoint{1.130799in}{1.585297in}}%
\pgfpathlineto{\pgfqpoint{1.141334in}{1.565060in}}%
\pgfpathlineto{\pgfqpoint{1.146602in}{1.561304in}}%
\pgfpathlineto{\pgfqpoint{1.151869in}{1.563538in}}%
\pgfpathlineto{\pgfqpoint{1.157137in}{1.572044in}}%
\pgfpathlineto{\pgfqpoint{1.167672in}{1.602141in}}%
\pgfpathlineto{\pgfqpoint{1.178207in}{1.629978in}}%
\pgfpathlineto{\pgfqpoint{1.183475in}{1.638466in}}%
\pgfpathlineto{\pgfqpoint{1.188743in}{1.636564in}}%
\pgfpathlineto{\pgfqpoint{1.194010in}{1.624173in}}%
\pgfpathlineto{\pgfqpoint{1.209813in}{1.561683in}}%
\pgfpathlineto{\pgfqpoint{1.215081in}{1.552670in}}%
\pgfpathlineto{\pgfqpoint{1.220348in}{1.552127in}}%
\pgfpathlineto{\pgfqpoint{1.230884in}{1.553643in}}%
\pgfpathlineto{\pgfqpoint{1.236151in}{1.546877in}}%
\pgfpathlineto{\pgfqpoint{1.241419in}{1.535012in}}%
\pgfpathlineto{\pgfqpoint{1.246686in}{1.511809in}}%
\pgfpathlineto{\pgfqpoint{1.267757in}{1.394109in}}%
\pgfpathlineto{\pgfqpoint{1.278292in}{1.363582in}}%
\pgfpathlineto{\pgfqpoint{1.283560in}{1.352539in}}%
\pgfpathlineto{\pgfqpoint{1.294095in}{1.315309in}}%
\pgfpathlineto{\pgfqpoint{1.299363in}{1.290511in}}%
\pgfpathlineto{\pgfqpoint{1.325701in}{1.119507in}}%
\pgfpathlineto{\pgfqpoint{1.336236in}{1.068625in}}%
\pgfpathlineto{\pgfqpoint{1.346771in}{1.028156in}}%
\pgfpathlineto{\pgfqpoint{1.362574in}{0.963280in}}%
\pgfpathlineto{\pgfqpoint{1.373110in}{0.931337in}}%
\pgfpathlineto{\pgfqpoint{1.399448in}{0.875214in}}%
\pgfpathlineto{\pgfqpoint{1.404715in}{0.874636in}}%
\pgfpathlineto{\pgfqpoint{1.409983in}{0.882947in}}%
\pgfpathlineto{\pgfqpoint{1.415251in}{0.900285in}}%
\pgfpathlineto{\pgfqpoint{1.425786in}{0.958312in}}%
\pgfpathlineto{\pgfqpoint{1.441589in}{1.081129in}}%
\pgfpathlineto{\pgfqpoint{1.478462in}{1.395978in}}%
\pgfpathlineto{\pgfqpoint{1.483730in}{1.432836in}}%
\pgfpathlineto{\pgfqpoint{1.488997in}{1.457464in}}%
\pgfpathlineto{\pgfqpoint{1.494265in}{1.464153in}}%
\pgfpathlineto{\pgfqpoint{1.499533in}{1.448436in}}%
\pgfpathlineto{\pgfqpoint{1.504800in}{1.409043in}}%
\pgfpathlineto{\pgfqpoint{1.510068in}{1.347175in}}%
\pgfpathlineto{\pgfqpoint{1.520603in}{1.170373in}}%
\pgfpathlineto{\pgfqpoint{1.541674in}{0.790142in}}%
\pgfpathlineto{\pgfqpoint{1.552209in}{0.660826in}}%
\pgfpathlineto{\pgfqpoint{1.562744in}{0.578141in}}%
\pgfpathlineto{\pgfqpoint{1.568012in}{0.551014in}}%
\pgfpathlineto{\pgfqpoint{1.573280in}{0.533713in}}%
\pgfpathlineto{\pgfqpoint{1.578547in}{0.524742in}}%
\pgfpathlineto{\pgfqpoint{1.583815in}{0.521896in}}%
\pgfpathlineto{\pgfqpoint{1.589082in}{0.523496in}}%
\pgfpathlineto{\pgfqpoint{1.615421in}{0.543708in}}%
\pgfpathlineto{\pgfqpoint{1.647026in}{0.580689in}}%
\pgfpathlineto{\pgfqpoint{1.726041in}{0.678329in}}%
\pgfpathlineto{\pgfqpoint{1.736576in}{0.691762in}}%
\pgfpathlineto{\pgfqpoint{1.752379in}{0.719988in}}%
\pgfpathlineto{\pgfqpoint{1.762914in}{0.747428in}}%
\pgfpathlineto{\pgfqpoint{1.773449in}{0.790378in}}%
\pgfpathlineto{\pgfqpoint{1.794520in}{0.903518in}}%
\pgfpathlineto{\pgfqpoint{1.820858in}{1.038191in}}%
\pgfpathlineto{\pgfqpoint{1.831393in}{1.077201in}}%
\pgfpathlineto{\pgfqpoint{1.836661in}{1.091032in}}%
\pgfpathlineto{\pgfqpoint{1.841929in}{1.099609in}}%
\pgfpathlineto{\pgfqpoint{1.847196in}{1.098154in}}%
\pgfpathlineto{\pgfqpoint{1.852464in}{1.082114in}}%
\pgfpathlineto{\pgfqpoint{1.857732in}{1.052098in}}%
\pgfpathlineto{\pgfqpoint{1.884070in}{0.861399in}}%
\pgfpathlineto{\pgfqpoint{1.894605in}{0.814353in}}%
\pgfpathlineto{\pgfqpoint{1.905140in}{0.777355in}}%
\pgfpathlineto{\pgfqpoint{1.910408in}{0.765714in}}%
\pgfpathlineto{\pgfqpoint{1.915675in}{0.758824in}}%
\pgfpathlineto{\pgfqpoint{1.920943in}{0.756004in}}%
\pgfpathlineto{\pgfqpoint{1.947281in}{0.749719in}}%
\pgfpathlineto{\pgfqpoint{1.968352in}{0.743104in}}%
\pgfpathlineto{\pgfqpoint{1.978887in}{0.736586in}}%
\pgfpathlineto{\pgfqpoint{1.989422in}{0.727996in}}%
\pgfpathlineto{\pgfqpoint{1.999958in}{0.717189in}}%
\pgfpathlineto{\pgfqpoint{2.021028in}{0.699188in}}%
\pgfpathlineto{\pgfqpoint{2.031563in}{0.688581in}}%
\pgfpathlineto{\pgfqpoint{2.042099in}{0.678551in}}%
\pgfpathlineto{\pgfqpoint{2.052634in}{0.670555in}}%
\pgfpathlineto{\pgfqpoint{2.063169in}{0.663500in}}%
\pgfpathlineto{\pgfqpoint{2.073704in}{0.659542in}}%
\pgfpathlineto{\pgfqpoint{2.089507in}{0.648721in}}%
\pgfpathlineto{\pgfqpoint{2.115845in}{0.623030in}}%
\pgfpathlineto{\pgfqpoint{2.142184in}{0.591880in}}%
\pgfpathlineto{\pgfqpoint{2.163254in}{0.567151in}}%
\pgfpathlineto{\pgfqpoint{2.200127in}{0.528543in}}%
\pgfpathlineto{\pgfqpoint{2.215930in}{0.515906in}}%
\pgfpathlineto{\pgfqpoint{2.231733in}{0.509559in}}%
\pgfpathlineto{\pgfqpoint{2.247536in}{0.508789in}}%
\pgfpathlineto{\pgfqpoint{2.263339in}{0.513225in}}%
\pgfpathlineto{\pgfqpoint{2.279142in}{0.521803in}}%
\pgfpathlineto{\pgfqpoint{2.284410in}{0.525197in}}%
\pgfpathlineto{\pgfqpoint{2.289677in}{0.530475in}}%
\pgfpathlineto{\pgfqpoint{2.294945in}{0.538782in}}%
\pgfpathlineto{\pgfqpoint{2.300212in}{0.553057in}}%
\pgfpathlineto{\pgfqpoint{2.305480in}{0.577453in}}%
\pgfpathlineto{\pgfqpoint{2.310748in}{0.614562in}}%
\pgfpathlineto{\pgfqpoint{2.321283in}{0.731758in}}%
\pgfpathlineto{\pgfqpoint{2.331818in}{0.903879in}}%
\pgfpathlineto{\pgfqpoint{2.347621in}{1.225563in}}%
\pgfpathlineto{\pgfqpoint{2.384494in}{1.998632in}}%
\pgfpathlineto{\pgfqpoint{2.389762in}{2.078866in}}%
\pgfpathlineto{\pgfqpoint{2.395030in}{2.135515in}}%
\pgfpathlineto{\pgfqpoint{2.400297in}{2.163637in}}%
\pgfpathlineto{\pgfqpoint{2.405565in}{2.161658in}}%
\pgfpathlineto{\pgfqpoint{2.410833in}{2.134735in}}%
\pgfpathlineto{\pgfqpoint{2.426636in}{2.015617in}}%
\pgfpathlineto{\pgfqpoint{2.437171in}{1.981884in}}%
\pgfpathlineto{\pgfqpoint{2.442438in}{1.975557in}}%
\pgfpathlineto{\pgfqpoint{2.447706in}{1.976276in}}%
\pgfpathlineto{\pgfqpoint{2.452974in}{1.984050in}}%
\pgfpathlineto{\pgfqpoint{2.458241in}{1.999793in}}%
\pgfpathlineto{\pgfqpoint{2.468777in}{2.053630in}}%
\pgfpathlineto{\pgfqpoint{2.479312in}{2.115733in}}%
\pgfpathlineto{\pgfqpoint{2.484579in}{2.133470in}}%
\pgfpathlineto{\pgfqpoint{2.489847in}{2.138282in}}%
\pgfpathlineto{\pgfqpoint{2.495115in}{2.136852in}}%
\pgfpathlineto{\pgfqpoint{2.500382in}{2.125299in}}%
\pgfpathlineto{\pgfqpoint{2.505650in}{2.101994in}}%
\pgfpathlineto{\pgfqpoint{2.526720in}{1.975458in}}%
\pgfpathlineto{\pgfqpoint{2.531988in}{1.960726in}}%
\pgfpathlineto{\pgfqpoint{2.537256in}{1.952147in}}%
\pgfpathlineto{\pgfqpoint{2.542523in}{1.952042in}}%
\pgfpathlineto{\pgfqpoint{2.547791in}{1.960455in}}%
\pgfpathlineto{\pgfqpoint{2.553059in}{1.977098in}}%
\pgfpathlineto{\pgfqpoint{2.563594in}{2.031528in}}%
\pgfpathlineto{\pgfqpoint{2.605735in}{2.300445in}}%
\pgfpathlineto{\pgfqpoint{2.616270in}{2.348085in}}%
\pgfpathlineto{\pgfqpoint{2.621538in}{2.359590in}}%
\pgfpathlineto{\pgfqpoint{2.626805in}{2.364524in}}%
\pgfpathlineto{\pgfqpoint{2.632073in}{2.364973in}}%
\pgfpathlineto{\pgfqpoint{2.653144in}{2.358529in}}%
\pgfpathlineto{\pgfqpoint{2.679482in}{2.358706in}}%
\pgfpathlineto{\pgfqpoint{2.695285in}{2.360531in}}%
\pgfpathlineto{\pgfqpoint{2.705820in}{2.357468in}}%
\pgfpathlineto{\pgfqpoint{2.721623in}{2.349405in}}%
\pgfpathlineto{\pgfqpoint{2.732158in}{2.347496in}}%
\pgfpathlineto{\pgfqpoint{2.758496in}{2.346351in}}%
\pgfpathlineto{\pgfqpoint{2.763764in}{2.338535in}}%
\pgfpathlineto{\pgfqpoint{2.769031in}{2.320102in}}%
\pgfpathlineto{\pgfqpoint{2.774299in}{2.290543in}}%
\pgfpathlineto{\pgfqpoint{2.795370in}{2.134356in}}%
\pgfpathlineto{\pgfqpoint{2.805905in}{2.097558in}}%
\pgfpathlineto{\pgfqpoint{2.816440in}{2.062537in}}%
\pgfpathlineto{\pgfqpoint{2.821708in}{2.037812in}}%
\pgfpathlineto{\pgfqpoint{2.826975in}{2.001375in}}%
\pgfpathlineto{\pgfqpoint{2.842778in}{1.860537in}}%
\pgfpathlineto{\pgfqpoint{2.848046in}{1.840735in}}%
\pgfpathlineto{\pgfqpoint{2.853314in}{1.837852in}}%
\pgfpathlineto{\pgfqpoint{2.858581in}{1.842104in}}%
\pgfpathlineto{\pgfqpoint{2.863849in}{1.855818in}}%
\pgfpathlineto{\pgfqpoint{2.874384in}{1.904028in}}%
\pgfpathlineto{\pgfqpoint{2.884919in}{1.967043in}}%
\pgfpathlineto{\pgfqpoint{2.916525in}{2.220483in}}%
\pgfpathlineto{\pgfqpoint{2.932328in}{2.373343in}}%
\pgfpathlineto{\pgfqpoint{2.937596in}{2.407832in}}%
\pgfpathlineto{\pgfqpoint{2.942863in}{2.430605in}}%
\pgfpathlineto{\pgfqpoint{2.948131in}{2.438357in}}%
\pgfpathlineto{\pgfqpoint{2.953398in}{2.434330in}}%
\pgfpathlineto{\pgfqpoint{2.969201in}{2.402110in}}%
\pgfpathlineto{\pgfqpoint{2.974469in}{2.398303in}}%
\pgfpathlineto{\pgfqpoint{2.985004in}{2.405347in}}%
\pgfpathlineto{\pgfqpoint{2.995540in}{2.414105in}}%
\pgfpathlineto{\pgfqpoint{3.000807in}{2.416496in}}%
\pgfpathlineto{\pgfqpoint{3.016610in}{2.430271in}}%
\pgfpathlineto{\pgfqpoint{3.021878in}{2.428662in}}%
\pgfpathlineto{\pgfqpoint{3.027145in}{2.421590in}}%
\pgfpathlineto{\pgfqpoint{3.042948in}{2.388772in}}%
\pgfpathlineto{\pgfqpoint{3.048216in}{2.387005in}}%
\pgfpathlineto{\pgfqpoint{3.058751in}{2.394584in}}%
\pgfpathlineto{\pgfqpoint{3.069286in}{2.402537in}}%
\pgfpathlineto{\pgfqpoint{3.074554in}{2.405656in}}%
\pgfpathlineto{\pgfqpoint{3.090357in}{2.418779in}}%
\pgfpathlineto{\pgfqpoint{3.095624in}{2.415641in}}%
\pgfpathlineto{\pgfqpoint{3.100892in}{2.407277in}}%
\pgfpathlineto{\pgfqpoint{3.111427in}{2.383524in}}%
\pgfpathlineto{\pgfqpoint{3.116695in}{2.377365in}}%
\pgfpathlineto{\pgfqpoint{3.121963in}{2.377867in}}%
\pgfpathlineto{\pgfqpoint{3.148301in}{2.397404in}}%
\pgfpathlineto{\pgfqpoint{3.158836in}{2.408061in}}%
\pgfpathlineto{\pgfqpoint{3.164104in}{2.408528in}}%
\pgfpathlineto{\pgfqpoint{3.169371in}{2.403442in}}%
\pgfpathlineto{\pgfqpoint{3.179907in}{2.380752in}}%
\pgfpathlineto{\pgfqpoint{3.185174in}{2.370540in}}%
\pgfpathlineto{\pgfqpoint{3.190442in}{2.366343in}}%
\pgfpathlineto{\pgfqpoint{3.200977in}{2.372834in}}%
\pgfpathlineto{\pgfqpoint{3.211512in}{2.381272in}}%
\pgfpathlineto{\pgfqpoint{3.216780in}{2.383652in}}%
\pgfpathlineto{\pgfqpoint{3.232583in}{2.397856in}}%
\pgfpathlineto{\pgfqpoint{3.237850in}{2.396809in}}%
\pgfpathlineto{\pgfqpoint{3.243118in}{2.390315in}}%
\pgfpathlineto{\pgfqpoint{3.258921in}{2.358148in}}%
\pgfpathlineto{\pgfqpoint{3.264189in}{2.356184in}}%
\pgfpathlineto{\pgfqpoint{3.274724in}{2.363784in}}%
\pgfpathlineto{\pgfqpoint{3.285259in}{2.371926in}}%
\pgfpathlineto{\pgfqpoint{3.290527in}{2.374802in}}%
\pgfpathlineto{\pgfqpoint{3.306330in}{2.387393in}}%
\pgfpathlineto{\pgfqpoint{3.311597in}{2.384354in}}%
\pgfpathlineto{\pgfqpoint{3.316865in}{2.376173in}}%
\pgfpathlineto{\pgfqpoint{3.327400in}{2.352444in}}%
\pgfpathlineto{\pgfqpoint{3.332668in}{2.345958in}}%
\pgfpathlineto{\pgfqpoint{3.337935in}{2.345992in}}%
\pgfpathlineto{\pgfqpoint{3.364274in}{2.364772in}}%
\pgfpathlineto{\pgfqpoint{3.374809in}{2.375474in}}%
\pgfpathlineto{\pgfqpoint{3.380076in}{2.376467in}}%
\pgfpathlineto{\pgfqpoint{3.385344in}{2.371955in}}%
\pgfpathlineto{\pgfqpoint{3.395879in}{2.350245in}}%
\pgfpathlineto{\pgfqpoint{3.401147in}{2.340056in}}%
\pgfpathlineto{\pgfqpoint{3.406415in}{2.335708in}}%
\pgfpathlineto{\pgfqpoint{3.411682in}{2.337928in}}%
\pgfpathlineto{\pgfqpoint{3.448556in}{2.366682in}}%
\pgfpathlineto{\pgfqpoint{3.453823in}{2.365691in}}%
\pgfpathlineto{\pgfqpoint{3.459091in}{2.359331in}}%
\pgfpathlineto{\pgfqpoint{3.459091in}{2.359331in}}%
\pgfusepath{stroke}%
\end{pgfscope}%
\begin{pgfscope}%
\pgfpathrectangle{\pgfqpoint{0.500000in}{0.375000in}}{\pgfqpoint{3.100000in}{2.265000in}}%
\pgfusepath{clip}%
\pgfsetrectcap%
\pgfsetroundjoin%
\pgfsetlinewidth{1.505625pt}%
\definecolor{currentstroke}{rgb}{0.000000,0.500000,0.000000}%
\pgfsetstrokecolor{currentstroke}%
\pgfsetdash{}{0pt}%
\pgfpathmoveto{\pgfqpoint{0.640909in}{0.477955in}}%
\pgfpathlineto{\pgfqpoint{0.709388in}{0.478884in}}%
\pgfpathlineto{\pgfqpoint{0.725191in}{0.482384in}}%
\pgfpathlineto{\pgfqpoint{0.746262in}{0.491412in}}%
\pgfpathlineto{\pgfqpoint{0.777867in}{0.508940in}}%
\pgfpathlineto{\pgfqpoint{0.788403in}{0.516816in}}%
\pgfpathlineto{\pgfqpoint{0.798938in}{0.528232in}}%
\pgfpathlineto{\pgfqpoint{0.809473in}{0.545703in}}%
\pgfpathlineto{\pgfqpoint{0.820008in}{0.572566in}}%
\pgfpathlineto{\pgfqpoint{0.830544in}{0.611428in}}%
\pgfpathlineto{\pgfqpoint{0.846347in}{0.673464in}}%
\pgfpathlineto{\pgfqpoint{0.856882in}{0.699088in}}%
\pgfpathlineto{\pgfqpoint{0.862150in}{0.704423in}}%
\pgfpathlineto{\pgfqpoint{0.867417in}{0.704252in}}%
\pgfpathlineto{\pgfqpoint{0.872685in}{0.698766in}}%
\pgfpathlineto{\pgfqpoint{0.877952in}{0.688141in}}%
\pgfpathlineto{\pgfqpoint{0.883220in}{0.672869in}}%
\pgfpathlineto{\pgfqpoint{0.893755in}{0.626851in}}%
\pgfpathlineto{\pgfqpoint{0.909558in}{0.545734in}}%
\pgfpathlineto{\pgfqpoint{0.914826in}{0.539651in}}%
\pgfpathlineto{\pgfqpoint{0.920093in}{0.547824in}}%
\pgfpathlineto{\pgfqpoint{0.925361in}{0.551637in}}%
\pgfpathlineto{\pgfqpoint{0.930629in}{0.551314in}}%
\pgfpathlineto{\pgfqpoint{0.935896in}{0.563985in}}%
\pgfpathlineto{\pgfqpoint{0.941164in}{0.602355in}}%
\pgfpathlineto{\pgfqpoint{0.967502in}{0.890249in}}%
\pgfpathlineto{\pgfqpoint{0.972770in}{0.921812in}}%
\pgfpathlineto{\pgfqpoint{0.978037in}{0.939418in}}%
\pgfpathlineto{\pgfqpoint{0.983305in}{0.947355in}}%
\pgfpathlineto{\pgfqpoint{0.988573in}{0.951540in}}%
\pgfpathlineto{\pgfqpoint{0.993840in}{0.943036in}}%
\pgfpathlineto{\pgfqpoint{0.999108in}{0.919855in}}%
\pgfpathlineto{\pgfqpoint{1.004376in}{0.879336in}}%
\pgfpathlineto{\pgfqpoint{1.014911in}{0.758384in}}%
\pgfpathlineto{\pgfqpoint{1.030714in}{0.576073in}}%
\pgfpathlineto{\pgfqpoint{1.035981in}{0.551577in}}%
\pgfpathlineto{\pgfqpoint{1.041249in}{0.563505in}}%
\pgfpathlineto{\pgfqpoint{1.051784in}{0.606675in}}%
\pgfpathlineto{\pgfqpoint{1.057052in}{0.616533in}}%
\pgfpathlineto{\pgfqpoint{1.062319in}{0.618165in}}%
\pgfpathlineto{\pgfqpoint{1.067587in}{0.613425in}}%
\pgfpathlineto{\pgfqpoint{1.072855in}{0.603389in}}%
\pgfpathlineto{\pgfqpoint{1.083390in}{0.572266in}}%
\pgfpathlineto{\pgfqpoint{1.088658in}{0.566436in}}%
\pgfpathlineto{\pgfqpoint{1.093925in}{0.571842in}}%
\pgfpathlineto{\pgfqpoint{1.104460in}{0.593792in}}%
\pgfpathlineto{\pgfqpoint{1.109728in}{0.599651in}}%
\pgfpathlineto{\pgfqpoint{1.114996in}{0.597672in}}%
\pgfpathlineto{\pgfqpoint{1.120263in}{0.587811in}}%
\pgfpathlineto{\pgfqpoint{1.130799in}{0.553669in}}%
\pgfpathlineto{\pgfqpoint{1.141334in}{0.517889in}}%
\pgfpathlineto{\pgfqpoint{1.146602in}{0.509902in}}%
\pgfpathlineto{\pgfqpoint{1.151869in}{0.514200in}}%
\pgfpathlineto{\pgfqpoint{1.172940in}{0.563013in}}%
\pgfpathlineto{\pgfqpoint{1.178207in}{0.567605in}}%
\pgfpathlineto{\pgfqpoint{1.183475in}{0.567091in}}%
\pgfpathlineto{\pgfqpoint{1.194010in}{0.549408in}}%
\pgfpathlineto{\pgfqpoint{1.199278in}{0.546793in}}%
\pgfpathlineto{\pgfqpoint{1.204545in}{0.547618in}}%
\pgfpathlineto{\pgfqpoint{1.220348in}{0.543749in}}%
\pgfpathlineto{\pgfqpoint{1.230884in}{0.544961in}}%
\pgfpathlineto{\pgfqpoint{1.241419in}{0.529398in}}%
\pgfpathlineto{\pgfqpoint{1.246686in}{0.534193in}}%
\pgfpathlineto{\pgfqpoint{1.257222in}{0.565414in}}%
\pgfpathlineto{\pgfqpoint{1.262489in}{0.574341in}}%
\pgfpathlineto{\pgfqpoint{1.267757in}{0.575588in}}%
\pgfpathlineto{\pgfqpoint{1.273025in}{0.569422in}}%
\pgfpathlineto{\pgfqpoint{1.283560in}{0.549056in}}%
\pgfpathlineto{\pgfqpoint{1.288828in}{0.530102in}}%
\pgfpathlineto{\pgfqpoint{1.294095in}{0.525509in}}%
\pgfpathlineto{\pgfqpoint{1.315166in}{0.593555in}}%
\pgfpathlineto{\pgfqpoint{1.320433in}{0.603946in}}%
\pgfpathlineto{\pgfqpoint{1.325701in}{0.606406in}}%
\pgfpathlineto{\pgfqpoint{1.330969in}{0.599364in}}%
\pgfpathlineto{\pgfqpoint{1.357307in}{0.525489in}}%
\pgfpathlineto{\pgfqpoint{1.362574in}{0.525134in}}%
\pgfpathlineto{\pgfqpoint{1.378377in}{0.529036in}}%
\pgfpathlineto{\pgfqpoint{1.388912in}{0.528025in}}%
\pgfpathlineto{\pgfqpoint{1.394180in}{0.522843in}}%
\pgfpathlineto{\pgfqpoint{1.399448in}{0.511816in}}%
\pgfpathlineto{\pgfqpoint{1.404715in}{0.510297in}}%
\pgfpathlineto{\pgfqpoint{1.409983in}{0.533038in}}%
\pgfpathlineto{\pgfqpoint{1.425786in}{0.651442in}}%
\pgfpathlineto{\pgfqpoint{1.436321in}{0.722720in}}%
\pgfpathlineto{\pgfqpoint{1.441589in}{0.748306in}}%
\pgfpathlineto{\pgfqpoint{1.446856in}{0.765431in}}%
\pgfpathlineto{\pgfqpoint{1.452124in}{0.774864in}}%
\pgfpathlineto{\pgfqpoint{1.457392in}{0.779573in}}%
\pgfpathlineto{\pgfqpoint{1.462659in}{0.777256in}}%
\pgfpathlineto{\pgfqpoint{1.467927in}{0.766890in}}%
\pgfpathlineto{\pgfqpoint{1.473195in}{0.746497in}}%
\pgfpathlineto{\pgfqpoint{1.478462in}{0.714071in}}%
\pgfpathlineto{\pgfqpoint{1.494265in}{0.586815in}}%
\pgfpathlineto{\pgfqpoint{1.499533in}{0.585434in}}%
\pgfpathlineto{\pgfqpoint{1.504800in}{0.626217in}}%
\pgfpathlineto{\pgfqpoint{1.520603in}{0.787481in}}%
\pgfpathlineto{\pgfqpoint{1.525871in}{0.819958in}}%
\pgfpathlineto{\pgfqpoint{1.531138in}{0.834040in}}%
\pgfpathlineto{\pgfqpoint{1.536406in}{0.828731in}}%
\pgfpathlineto{\pgfqpoint{1.541674in}{0.803750in}}%
\pgfpathlineto{\pgfqpoint{1.552209in}{0.713759in}}%
\pgfpathlineto{\pgfqpoint{1.568012in}{0.560126in}}%
\pgfpathlineto{\pgfqpoint{1.573280in}{0.523563in}}%
\pgfpathlineto{\pgfqpoint{1.578547in}{0.499798in}}%
\pgfpathlineto{\pgfqpoint{1.583815in}{0.492081in}}%
\pgfpathlineto{\pgfqpoint{1.615421in}{0.504997in}}%
\pgfpathlineto{\pgfqpoint{1.647026in}{0.508178in}}%
\pgfpathlineto{\pgfqpoint{1.715506in}{0.507256in}}%
\pgfpathlineto{\pgfqpoint{1.731308in}{0.504718in}}%
\pgfpathlineto{\pgfqpoint{1.736576in}{0.507282in}}%
\pgfpathlineto{\pgfqpoint{1.752379in}{0.523289in}}%
\pgfpathlineto{\pgfqpoint{1.762914in}{0.545120in}}%
\pgfpathlineto{\pgfqpoint{1.773449in}{0.587368in}}%
\pgfpathlineto{\pgfqpoint{1.783985in}{0.627804in}}%
\pgfpathlineto{\pgfqpoint{1.789252in}{0.641783in}}%
\pgfpathlineto{\pgfqpoint{1.794520in}{0.649739in}}%
\pgfpathlineto{\pgfqpoint{1.799788in}{0.653102in}}%
\pgfpathlineto{\pgfqpoint{1.805055in}{0.652605in}}%
\pgfpathlineto{\pgfqpoint{1.810323in}{0.647440in}}%
\pgfpathlineto{\pgfqpoint{1.820858in}{0.623811in}}%
\pgfpathlineto{\pgfqpoint{1.831393in}{0.588593in}}%
\pgfpathlineto{\pgfqpoint{1.841929in}{0.543590in}}%
\pgfpathlineto{\pgfqpoint{1.847196in}{0.529498in}}%
\pgfpathlineto{\pgfqpoint{1.852464in}{0.547336in}}%
\pgfpathlineto{\pgfqpoint{1.862999in}{0.606810in}}%
\pgfpathlineto{\pgfqpoint{1.868267in}{0.623090in}}%
\pgfpathlineto{\pgfqpoint{1.873534in}{0.627826in}}%
\pgfpathlineto{\pgfqpoint{1.878802in}{0.624875in}}%
\pgfpathlineto{\pgfqpoint{1.884070in}{0.615390in}}%
\pgfpathlineto{\pgfqpoint{1.889337in}{0.599649in}}%
\pgfpathlineto{\pgfqpoint{1.899873in}{0.552185in}}%
\pgfpathlineto{\pgfqpoint{1.910408in}{0.502769in}}%
\pgfpathlineto{\pgfqpoint{1.915675in}{0.486617in}}%
\pgfpathlineto{\pgfqpoint{1.920943in}{0.480759in}}%
\pgfpathlineto{\pgfqpoint{1.973619in}{0.485060in}}%
\pgfpathlineto{\pgfqpoint{1.984155in}{0.489772in}}%
\pgfpathlineto{\pgfqpoint{1.999958in}{0.495360in}}%
\pgfpathlineto{\pgfqpoint{2.015760in}{0.494936in}}%
\pgfpathlineto{\pgfqpoint{2.031563in}{0.491769in}}%
\pgfpathlineto{\pgfqpoint{2.047366in}{0.492662in}}%
\pgfpathlineto{\pgfqpoint{2.057901in}{0.491621in}}%
\pgfpathlineto{\pgfqpoint{2.078972in}{0.485423in}}%
\pgfpathlineto{\pgfqpoint{2.089507in}{0.488341in}}%
\pgfpathlineto{\pgfqpoint{2.110578in}{0.496058in}}%
\pgfpathlineto{\pgfqpoint{2.121113in}{0.498235in}}%
\pgfpathlineto{\pgfqpoint{2.142184in}{0.497276in}}%
\pgfpathlineto{\pgfqpoint{2.189592in}{0.494942in}}%
\pgfpathlineto{\pgfqpoint{2.210663in}{0.491896in}}%
\pgfpathlineto{\pgfqpoint{2.247536in}{0.483192in}}%
\pgfpathlineto{\pgfqpoint{2.284410in}{0.497119in}}%
\pgfpathlineto{\pgfqpoint{2.289677in}{0.502365in}}%
\pgfpathlineto{\pgfqpoint{2.294945in}{0.512460in}}%
\pgfpathlineto{\pgfqpoint{2.300212in}{0.534903in}}%
\pgfpathlineto{\pgfqpoint{2.305480in}{0.577178in}}%
\pgfpathlineto{\pgfqpoint{2.316015in}{0.709872in}}%
\pgfpathlineto{\pgfqpoint{2.337086in}{1.037307in}}%
\pgfpathlineto{\pgfqpoint{2.347621in}{1.145760in}}%
\pgfpathlineto{\pgfqpoint{2.352889in}{1.174473in}}%
\pgfpathlineto{\pgfqpoint{2.358156in}{1.185538in}}%
\pgfpathlineto{\pgfqpoint{2.363424in}{1.178530in}}%
\pgfpathlineto{\pgfqpoint{2.368692in}{1.153139in}}%
\pgfpathlineto{\pgfqpoint{2.373959in}{1.109990in}}%
\pgfpathlineto{\pgfqpoint{2.379227in}{1.047911in}}%
\pgfpathlineto{\pgfqpoint{2.389762in}{0.875215in}}%
\pgfpathlineto{\pgfqpoint{2.400297in}{0.692658in}}%
\pgfpathlineto{\pgfqpoint{2.405565in}{0.642501in}}%
\pgfpathlineto{\pgfqpoint{2.410833in}{0.645753in}}%
\pgfpathlineto{\pgfqpoint{2.416100in}{0.671648in}}%
\pgfpathlineto{\pgfqpoint{2.421368in}{0.687819in}}%
\pgfpathlineto{\pgfqpoint{2.426636in}{0.684530in}}%
\pgfpathlineto{\pgfqpoint{2.442438in}{0.635075in}}%
\pgfpathlineto{\pgfqpoint{2.447706in}{0.636487in}}%
\pgfpathlineto{\pgfqpoint{2.452974in}{0.649470in}}%
\pgfpathlineto{\pgfqpoint{2.463509in}{0.683674in}}%
\pgfpathlineto{\pgfqpoint{2.468777in}{0.691476in}}%
\pgfpathlineto{\pgfqpoint{2.474044in}{0.690802in}}%
\pgfpathlineto{\pgfqpoint{2.484579in}{0.675816in}}%
\pgfpathlineto{\pgfqpoint{2.489847in}{0.664666in}}%
\pgfpathlineto{\pgfqpoint{2.500382in}{0.625806in}}%
\pgfpathlineto{\pgfqpoint{2.505650in}{0.623706in}}%
\pgfpathlineto{\pgfqpoint{2.516185in}{0.660662in}}%
\pgfpathlineto{\pgfqpoint{2.521453in}{0.667805in}}%
\pgfpathlineto{\pgfqpoint{2.526720in}{0.658117in}}%
\pgfpathlineto{\pgfqpoint{2.537256in}{0.624926in}}%
\pgfpathlineto{\pgfqpoint{2.542523in}{0.624708in}}%
\pgfpathlineto{\pgfqpoint{2.547791in}{0.640614in}}%
\pgfpathlineto{\pgfqpoint{2.558326in}{0.685799in}}%
\pgfpathlineto{\pgfqpoint{2.563594in}{0.700190in}}%
\pgfpathlineto{\pgfqpoint{2.568862in}{0.705954in}}%
\pgfpathlineto{\pgfqpoint{2.574129in}{0.705524in}}%
\pgfpathlineto{\pgfqpoint{2.579397in}{0.703333in}}%
\pgfpathlineto{\pgfqpoint{2.584664in}{0.704198in}}%
\pgfpathlineto{\pgfqpoint{2.589932in}{0.697260in}}%
\pgfpathlineto{\pgfqpoint{2.595200in}{0.678051in}}%
\pgfpathlineto{\pgfqpoint{2.605735in}{0.607606in}}%
\pgfpathlineto{\pgfqpoint{2.616270in}{0.534428in}}%
\pgfpathlineto{\pgfqpoint{2.621538in}{0.509401in}}%
\pgfpathlineto{\pgfqpoint{2.626805in}{0.493611in}}%
\pgfpathlineto{\pgfqpoint{2.632073in}{0.486945in}}%
\pgfpathlineto{\pgfqpoint{2.642608in}{0.486543in}}%
\pgfpathlineto{\pgfqpoint{2.668946in}{0.486470in}}%
\pgfpathlineto{\pgfqpoint{2.684749in}{0.488037in}}%
\pgfpathlineto{\pgfqpoint{2.695285in}{0.487433in}}%
\pgfpathlineto{\pgfqpoint{2.705820in}{0.485745in}}%
\pgfpathlineto{\pgfqpoint{2.753229in}{0.486649in}}%
\pgfpathlineto{\pgfqpoint{2.758496in}{0.486407in}}%
\pgfpathlineto{\pgfqpoint{2.763764in}{0.499151in}}%
\pgfpathlineto{\pgfqpoint{2.784834in}{0.611613in}}%
\pgfpathlineto{\pgfqpoint{2.790102in}{0.622006in}}%
\pgfpathlineto{\pgfqpoint{2.795370in}{0.621430in}}%
\pgfpathlineto{\pgfqpoint{2.800637in}{0.613003in}}%
\pgfpathlineto{\pgfqpoint{2.805905in}{0.599893in}}%
\pgfpathlineto{\pgfqpoint{2.811172in}{0.582485in}}%
\pgfpathlineto{\pgfqpoint{2.816440in}{0.557446in}}%
\pgfpathlineto{\pgfqpoint{2.821708in}{0.540511in}}%
\pgfpathlineto{\pgfqpoint{2.826975in}{0.554325in}}%
\pgfpathlineto{\pgfqpoint{2.837511in}{0.618495in}}%
\pgfpathlineto{\pgfqpoint{2.842778in}{0.626441in}}%
\pgfpathlineto{\pgfqpoint{2.848046in}{0.620951in}}%
\pgfpathlineto{\pgfqpoint{2.853314in}{0.617840in}}%
\pgfpathlineto{\pgfqpoint{2.858581in}{0.625298in}}%
\pgfpathlineto{\pgfqpoint{2.863849in}{0.651063in}}%
\pgfpathlineto{\pgfqpoint{2.879652in}{0.749968in}}%
\pgfpathlineto{\pgfqpoint{2.884919in}{0.770454in}}%
\pgfpathlineto{\pgfqpoint{2.890187in}{0.778682in}}%
\pgfpathlineto{\pgfqpoint{2.895455in}{0.775552in}}%
\pgfpathlineto{\pgfqpoint{2.900722in}{0.769036in}}%
\pgfpathlineto{\pgfqpoint{2.911257in}{0.767633in}}%
\pgfpathlineto{\pgfqpoint{2.916525in}{0.751675in}}%
\pgfpathlineto{\pgfqpoint{2.921793in}{0.716920in}}%
\pgfpathlineto{\pgfqpoint{2.937596in}{0.582536in}}%
\pgfpathlineto{\pgfqpoint{2.942863in}{0.554599in}}%
\pgfpathlineto{\pgfqpoint{2.948131in}{0.537538in}}%
\pgfpathlineto{\pgfqpoint{2.953398in}{0.528503in}}%
\pgfpathlineto{\pgfqpoint{2.958666in}{0.522722in}}%
\pgfpathlineto{\pgfqpoint{2.969201in}{0.516532in}}%
\pgfpathlineto{\pgfqpoint{2.974469in}{0.514320in}}%
\pgfpathlineto{\pgfqpoint{2.979737in}{0.513867in}}%
\pgfpathlineto{\pgfqpoint{2.985004in}{0.515776in}}%
\pgfpathlineto{\pgfqpoint{3.011342in}{0.539797in}}%
\pgfpathlineto{\pgfqpoint{3.016610in}{0.538190in}}%
\pgfpathlineto{\pgfqpoint{3.032413in}{0.519977in}}%
\pgfpathlineto{\pgfqpoint{3.048216in}{0.512816in}}%
\pgfpathlineto{\pgfqpoint{3.053483in}{0.512960in}}%
\pgfpathlineto{\pgfqpoint{3.058751in}{0.516116in}}%
\pgfpathlineto{\pgfqpoint{3.079822in}{0.538212in}}%
\pgfpathlineto{\pgfqpoint{3.085089in}{0.540212in}}%
\pgfpathlineto{\pgfqpoint{3.090357in}{0.536857in}}%
\pgfpathlineto{\pgfqpoint{3.100892in}{0.522927in}}%
\pgfpathlineto{\pgfqpoint{3.111427in}{0.515987in}}%
\pgfpathlineto{\pgfqpoint{3.121963in}{0.512097in}}%
\pgfpathlineto{\pgfqpoint{3.127230in}{0.512985in}}%
\pgfpathlineto{\pgfqpoint{3.132498in}{0.517264in}}%
\pgfpathlineto{\pgfqpoint{3.148301in}{0.534423in}}%
\pgfpathlineto{\pgfqpoint{3.153568in}{0.538504in}}%
\pgfpathlineto{\pgfqpoint{3.158836in}{0.539283in}}%
\pgfpathlineto{\pgfqpoint{3.164104in}{0.534272in}}%
\pgfpathlineto{\pgfqpoint{3.174639in}{0.521423in}}%
\pgfpathlineto{\pgfqpoint{3.190442in}{0.513944in}}%
\pgfpathlineto{\pgfqpoint{3.195709in}{0.513497in}}%
\pgfpathlineto{\pgfqpoint{3.200977in}{0.515200in}}%
\pgfpathlineto{\pgfqpoint{3.227315in}{0.539115in}}%
\pgfpathlineto{\pgfqpoint{3.232583in}{0.538013in}}%
\pgfpathlineto{\pgfqpoint{3.248386in}{0.519851in}}%
\pgfpathlineto{\pgfqpoint{3.264189in}{0.512215in}}%
\pgfpathlineto{\pgfqpoint{3.269456in}{0.512209in}}%
\pgfpathlineto{\pgfqpoint{3.274724in}{0.515143in}}%
\pgfpathlineto{\pgfqpoint{3.295794in}{0.536685in}}%
\pgfpathlineto{\pgfqpoint{3.301062in}{0.538729in}}%
\pgfpathlineto{\pgfqpoint{3.306330in}{0.535593in}}%
\pgfpathlineto{\pgfqpoint{3.316865in}{0.522008in}}%
\pgfpathlineto{\pgfqpoint{3.327400in}{0.515468in}}%
\pgfpathlineto{\pgfqpoint{3.337935in}{0.511809in}}%
\pgfpathlineto{\pgfqpoint{3.343203in}{0.512565in}}%
\pgfpathlineto{\pgfqpoint{3.348471in}{0.516558in}}%
\pgfpathlineto{\pgfqpoint{3.364274in}{0.533477in}}%
\pgfpathlineto{\pgfqpoint{3.369541in}{0.537757in}}%
\pgfpathlineto{\pgfqpoint{3.374809in}{0.538782in}}%
\pgfpathlineto{\pgfqpoint{3.380076in}{0.534155in}}%
\pgfpathlineto{\pgfqpoint{3.390612in}{0.521067in}}%
\pgfpathlineto{\pgfqpoint{3.406415in}{0.512832in}}%
\pgfpathlineto{\pgfqpoint{3.411682in}{0.512205in}}%
\pgfpathlineto{\pgfqpoint{3.416950in}{0.513768in}}%
\pgfpathlineto{\pgfqpoint{3.443288in}{0.537657in}}%
\pgfpathlineto{\pgfqpoint{3.448556in}{0.536755in}}%
\pgfpathlineto{\pgfqpoint{3.459091in}{0.523776in}}%
\pgfpathlineto{\pgfqpoint{3.459091in}{0.523776in}}%
\pgfusepath{stroke}%
\end{pgfscope}%
\begin{pgfscope}%
\pgfsetrectcap%
\pgfsetmiterjoin%
\pgfsetlinewidth{0.803000pt}%
\definecolor{currentstroke}{rgb}{0.000000,0.000000,0.000000}%
\pgfsetstrokecolor{currentstroke}%
\pgfsetdash{}{0pt}%
\pgfpathmoveto{\pgfqpoint{0.500000in}{0.375000in}}%
\pgfpathlineto{\pgfqpoint{0.500000in}{2.640000in}}%
\pgfusepath{stroke}%
\end{pgfscope}%
\begin{pgfscope}%
\pgfsetrectcap%
\pgfsetmiterjoin%
\pgfsetlinewidth{0.803000pt}%
\definecolor{currentstroke}{rgb}{0.000000,0.000000,0.000000}%
\pgfsetstrokecolor{currentstroke}%
\pgfsetdash{}{0pt}%
\pgfpathmoveto{\pgfqpoint{3.600000in}{0.375000in}}%
\pgfpathlineto{\pgfqpoint{3.600000in}{2.640000in}}%
\pgfusepath{stroke}%
\end{pgfscope}%
\begin{pgfscope}%
\pgfsetrectcap%
\pgfsetmiterjoin%
\pgfsetlinewidth{0.803000pt}%
\definecolor{currentstroke}{rgb}{0.000000,0.000000,0.000000}%
\pgfsetstrokecolor{currentstroke}%
\pgfsetdash{}{0pt}%
\pgfpathmoveto{\pgfqpoint{0.500000in}{0.375000in}}%
\pgfpathlineto{\pgfqpoint{3.600000in}{0.375000in}}%
\pgfusepath{stroke}%
\end{pgfscope}%
\begin{pgfscope}%
\pgfsetrectcap%
\pgfsetmiterjoin%
\pgfsetlinewidth{0.803000pt}%
\definecolor{currentstroke}{rgb}{0.000000,0.000000,0.000000}%
\pgfsetstrokecolor{currentstroke}%
\pgfsetdash{}{0pt}%
\pgfpathmoveto{\pgfqpoint{0.500000in}{2.640000in}}%
\pgfpathlineto{\pgfqpoint{3.600000in}{2.640000in}}%
\pgfusepath{stroke}%
\end{pgfscope}%
\begin{pgfscope}%
\pgfsetbuttcap%
\pgfsetmiterjoin%
\definecolor{currentfill}{rgb}{1.000000,1.000000,1.000000}%
\pgfsetfillcolor{currentfill}%
\pgfsetfillopacity{0.800000}%
\pgfsetlinewidth{1.003750pt}%
\definecolor{currentstroke}{rgb}{0.800000,0.800000,0.800000}%
\pgfsetstrokecolor{currentstroke}%
\pgfsetstrokeopacity{0.800000}%
\pgfsetdash{}{0pt}%
\pgfpathmoveto{\pgfqpoint{2.799845in}{1.196158in}}%
\pgfpathlineto{\pgfqpoint{3.502778in}{1.196158in}}%
\pgfpathquadraticcurveto{\pgfqpoint{3.530556in}{1.196158in}}{\pgfqpoint{3.530556in}{1.223935in}}%
\pgfpathlineto{\pgfqpoint{3.530556in}{1.791065in}}%
\pgfpathquadraticcurveto{\pgfqpoint{3.530556in}{1.818842in}}{\pgfqpoint{3.502778in}{1.818842in}}%
\pgfpathlineto{\pgfqpoint{2.799845in}{1.818842in}}%
\pgfpathquadraticcurveto{\pgfqpoint{2.772067in}{1.818842in}}{\pgfqpoint{2.772067in}{1.791065in}}%
\pgfpathlineto{\pgfqpoint{2.772067in}{1.223935in}}%
\pgfpathquadraticcurveto{\pgfqpoint{2.772067in}{1.196158in}}{\pgfqpoint{2.799845in}{1.196158in}}%
\pgfpathlineto{\pgfqpoint{2.799845in}{1.196158in}}%
\pgfpathclose%
\pgfusepath{stroke,fill}%
\end{pgfscope}%
\begin{pgfscope}%
\pgfsetrectcap%
\pgfsetroundjoin%
\pgfsetlinewidth{1.505625pt}%
\definecolor{currentstroke}{rgb}{0.000000,0.000000,1.000000}%
\pgfsetstrokecolor{currentstroke}%
\pgfsetdash{}{0pt}%
\pgfpathmoveto{\pgfqpoint{2.827622in}{1.714676in}}%
\pgfpathlineto{\pgfqpoint{2.966511in}{1.714676in}}%
\pgfpathlineto{\pgfqpoint{3.105400in}{1.714676in}}%
\pgfusepath{stroke}%
\end{pgfscope}%
\begin{pgfscope}%
\definecolor{textcolor}{rgb}{0.000000,0.000000,0.000000}%
\pgfsetstrokecolor{textcolor}%
\pgfsetfillcolor{textcolor}%
\pgftext[x=3.216511in,y=1.666065in,left,base]{\color{textcolor}\rmfamily\fontsize{10.000000}{12.000000}\selectfont max}%
\end{pgfscope}%
\begin{pgfscope}%
\pgfsetrectcap%
\pgfsetroundjoin%
\pgfsetlinewidth{1.505625pt}%
\definecolor{currentstroke}{rgb}{1.000000,0.000000,0.000000}%
\pgfsetstrokecolor{currentstroke}%
\pgfsetdash{}{0pt}%
\pgfpathmoveto{\pgfqpoint{2.827622in}{1.521003in}}%
\pgfpathlineto{\pgfqpoint{2.966511in}{1.521003in}}%
\pgfpathlineto{\pgfqpoint{3.105400in}{1.521003in}}%
\pgfusepath{stroke}%
\end{pgfscope}%
\begin{pgfscope}%
\definecolor{textcolor}{rgb}{0.000000,0.000000,0.000000}%
\pgfsetstrokecolor{textcolor}%
\pgfsetfillcolor{textcolor}%
\pgftext[x=3.216511in,y=1.472392in,left,base]{\color{textcolor}\rmfamily\fontsize{10.000000}{12.000000}\selectfont \(\displaystyle \mu\)}%
\end{pgfscope}%
\begin{pgfscope}%
\pgfsetrectcap%
\pgfsetroundjoin%
\pgfsetlinewidth{1.505625pt}%
\definecolor{currentstroke}{rgb}{0.000000,0.500000,0.000000}%
\pgfsetstrokecolor{currentstroke}%
\pgfsetdash{}{0pt}%
\pgfpathmoveto{\pgfqpoint{2.827622in}{1.327330in}}%
\pgfpathlineto{\pgfqpoint{2.966511in}{1.327330in}}%
\pgfpathlineto{\pgfqpoint{3.105400in}{1.327330in}}%
\pgfusepath{stroke}%
\end{pgfscope}%
\begin{pgfscope}%
\definecolor{textcolor}{rgb}{0.000000,0.000000,0.000000}%
\pgfsetstrokecolor{textcolor}%
\pgfsetfillcolor{textcolor}%
\pgftext[x=3.216511in,y=1.278719in,left,base]{\color{textcolor}\rmfamily\fontsize{10.000000}{12.000000}\selectfont \(\displaystyle \sigma\)}%
\end{pgfscope}%
\end{pgfpicture}%
\makeatother%
\endgroup%
}
%         \caption{Pressure Matrix Profile}
%         \label{fig:mp_hist_standard_pressure}
%     \end{minipage}
%     \begin{minipage}[t]{0.5\linewidth}
%         %%\centering
%         \resizebox{\linewidth}{!}{%% Creator: Matplotlib, PGF backend
%%
%% To include the figure in your LaTeX document, write
%%   \input{<filename>.pgf}
%%
%% Make sure the required packages are loaded in your preamble
%%   \usepackage{pgf}
%%
%% Also ensure that all the required font packages are loaded; for instance,
%% the lmodern package is sometimes necessary when using math font.
%%   \usepackage{lmodern}
%%
%% Figures using additional raster images can only be included by \input if
%% they are in the same directory as the main LaTeX file. For loading figures
%% from other directories you can use the `import` package
%%   \usepackage{import}
%%
%% and then include the figures with
%%   \import{<path to file>}{<filename>.pgf}
%%
%% Matplotlib used the following preamble
%%
\begingroup%
\makeatletter%
\begin{pgfpicture}%
\pgfpathrectangle{\pgfpointorigin}{\pgfqpoint{4.000000in}{3.000000in}}%
\pgfusepath{use as bounding box, clip}%
\begin{pgfscope}%
\pgfsetbuttcap%
\pgfsetmiterjoin%
\pgfsetlinewidth{0.000000pt}%
\definecolor{currentstroke}{rgb}{1.000000,1.000000,1.000000}%
\pgfsetstrokecolor{currentstroke}%
\pgfsetstrokeopacity{0.000000}%
\pgfsetdash{}{0pt}%
\pgfpathmoveto{\pgfqpoint{0.000000in}{0.000000in}}%
\pgfpathlineto{\pgfqpoint{4.000000in}{0.000000in}}%
\pgfpathlineto{\pgfqpoint{4.000000in}{3.000000in}}%
\pgfpathlineto{\pgfqpoint{0.000000in}{3.000000in}}%
\pgfpathlineto{\pgfqpoint{0.000000in}{0.000000in}}%
\pgfpathclose%
\pgfusepath{}%
\end{pgfscope}%
\begin{pgfscope}%
\pgfsetbuttcap%
\pgfsetmiterjoin%
\definecolor{currentfill}{rgb}{1.000000,1.000000,1.000000}%
\pgfsetfillcolor{currentfill}%
\pgfsetlinewidth{0.000000pt}%
\definecolor{currentstroke}{rgb}{0.000000,0.000000,0.000000}%
\pgfsetstrokecolor{currentstroke}%
\pgfsetstrokeopacity{0.000000}%
\pgfsetdash{}{0pt}%
\pgfpathmoveto{\pgfqpoint{0.500000in}{0.375000in}}%
\pgfpathlineto{\pgfqpoint{3.600000in}{0.375000in}}%
\pgfpathlineto{\pgfqpoint{3.600000in}{2.640000in}}%
\pgfpathlineto{\pgfqpoint{0.500000in}{2.640000in}}%
\pgfpathlineto{\pgfqpoint{0.500000in}{0.375000in}}%
\pgfpathclose%
\pgfusepath{fill}%
\end{pgfscope}%
\begin{pgfscope}%
\pgfsetbuttcap%
\pgfsetroundjoin%
\definecolor{currentfill}{rgb}{0.000000,0.000000,0.000000}%
\pgfsetfillcolor{currentfill}%
\pgfsetlinewidth{0.803000pt}%
\definecolor{currentstroke}{rgb}{0.000000,0.000000,0.000000}%
\pgfsetstrokecolor{currentstroke}%
\pgfsetdash{}{0pt}%
\pgfsys@defobject{currentmarker}{\pgfqpoint{0.000000in}{-0.048611in}}{\pgfqpoint{0.000000in}{0.000000in}}{%
\pgfpathmoveto{\pgfqpoint{0.000000in}{0.000000in}}%
\pgfpathlineto{\pgfqpoint{0.000000in}{-0.048611in}}%
\pgfusepath{stroke,fill}%
}%
\begin{pgfscope}%
\pgfsys@transformshift{0.640909in}{0.375000in}%
\pgfsys@useobject{currentmarker}{}%
\end{pgfscope}%
\end{pgfscope}%
\begin{pgfscope}%
\definecolor{textcolor}{rgb}{0.000000,0.000000,0.000000}%
\pgfsetstrokecolor{textcolor}%
\pgfsetfillcolor{textcolor}%
\pgftext[x=0.640909in,y=0.277778in,,top]{\color{textcolor}\rmfamily\fontsize{10.000000}{12.000000}\selectfont \(\displaystyle {0}\)}%
\end{pgfscope}%
\begin{pgfscope}%
\pgfsetbuttcap%
\pgfsetroundjoin%
\definecolor{currentfill}{rgb}{0.000000,0.000000,0.000000}%
\pgfsetfillcolor{currentfill}%
\pgfsetlinewidth{0.803000pt}%
\definecolor{currentstroke}{rgb}{0.000000,0.000000,0.000000}%
\pgfsetstrokecolor{currentstroke}%
\pgfsetdash{}{0pt}%
\pgfsys@defobject{currentmarker}{\pgfqpoint{0.000000in}{-0.048611in}}{\pgfqpoint{0.000000in}{0.000000in}}{%
\pgfpathmoveto{\pgfqpoint{0.000000in}{0.000000in}}%
\pgfpathlineto{\pgfqpoint{0.000000in}{-0.048611in}}%
\pgfusepath{stroke,fill}%
}%
\begin{pgfscope}%
\pgfsys@transformshift{1.421569in}{0.375000in}%
\pgfsys@useobject{currentmarker}{}%
\end{pgfscope}%
\end{pgfscope}%
\begin{pgfscope}%
\definecolor{textcolor}{rgb}{0.000000,0.000000,0.000000}%
\pgfsetstrokecolor{textcolor}%
\pgfsetfillcolor{textcolor}%
\pgftext[x=1.421569in,y=0.277778in,,top]{\color{textcolor}\rmfamily\fontsize{10.000000}{12.000000}\selectfont \(\displaystyle {100}\)}%
\end{pgfscope}%
\begin{pgfscope}%
\pgfsetbuttcap%
\pgfsetroundjoin%
\definecolor{currentfill}{rgb}{0.000000,0.000000,0.000000}%
\pgfsetfillcolor{currentfill}%
\pgfsetlinewidth{0.803000pt}%
\definecolor{currentstroke}{rgb}{0.000000,0.000000,0.000000}%
\pgfsetstrokecolor{currentstroke}%
\pgfsetdash{}{0pt}%
\pgfsys@defobject{currentmarker}{\pgfqpoint{0.000000in}{-0.048611in}}{\pgfqpoint{0.000000in}{0.000000in}}{%
\pgfpathmoveto{\pgfqpoint{0.000000in}{0.000000in}}%
\pgfpathlineto{\pgfqpoint{0.000000in}{-0.048611in}}%
\pgfusepath{stroke,fill}%
}%
\begin{pgfscope}%
\pgfsys@transformshift{2.202229in}{0.375000in}%
\pgfsys@useobject{currentmarker}{}%
\end{pgfscope}%
\end{pgfscope}%
\begin{pgfscope}%
\definecolor{textcolor}{rgb}{0.000000,0.000000,0.000000}%
\pgfsetstrokecolor{textcolor}%
\pgfsetfillcolor{textcolor}%
\pgftext[x=2.202229in,y=0.277778in,,top]{\color{textcolor}\rmfamily\fontsize{10.000000}{12.000000}\selectfont \(\displaystyle {200}\)}%
\end{pgfscope}%
\begin{pgfscope}%
\pgfsetbuttcap%
\pgfsetroundjoin%
\definecolor{currentfill}{rgb}{0.000000,0.000000,0.000000}%
\pgfsetfillcolor{currentfill}%
\pgfsetlinewidth{0.803000pt}%
\definecolor{currentstroke}{rgb}{0.000000,0.000000,0.000000}%
\pgfsetstrokecolor{currentstroke}%
\pgfsetdash{}{0pt}%
\pgfsys@defobject{currentmarker}{\pgfqpoint{0.000000in}{-0.048611in}}{\pgfqpoint{0.000000in}{0.000000in}}{%
\pgfpathmoveto{\pgfqpoint{0.000000in}{0.000000in}}%
\pgfpathlineto{\pgfqpoint{0.000000in}{-0.048611in}}%
\pgfusepath{stroke,fill}%
}%
\begin{pgfscope}%
\pgfsys@transformshift{2.982888in}{0.375000in}%
\pgfsys@useobject{currentmarker}{}%
\end{pgfscope}%
\end{pgfscope}%
\begin{pgfscope}%
\definecolor{textcolor}{rgb}{0.000000,0.000000,0.000000}%
\pgfsetstrokecolor{textcolor}%
\pgfsetfillcolor{textcolor}%
\pgftext[x=2.982888in,y=0.277778in,,top]{\color{textcolor}\rmfamily\fontsize{10.000000}{12.000000}\selectfont \(\displaystyle {300}\)}%
\end{pgfscope}%
\begin{pgfscope}%
\definecolor{textcolor}{rgb}{0.000000,0.000000,0.000000}%
\pgfsetstrokecolor{textcolor}%
\pgfsetfillcolor{textcolor}%
\pgftext[x=2.050000in,y=0.098766in,,top]{\color{textcolor}\rmfamily\fontsize{10.000000}{12.000000}\selectfont time}%
\end{pgfscope}%
\begin{pgfscope}%
\pgfsetbuttcap%
\pgfsetroundjoin%
\definecolor{currentfill}{rgb}{0.000000,0.000000,0.000000}%
\pgfsetfillcolor{currentfill}%
\pgfsetlinewidth{0.803000pt}%
\definecolor{currentstroke}{rgb}{0.000000,0.000000,0.000000}%
\pgfsetstrokecolor{currentstroke}%
\pgfsetdash{}{0pt}%
\pgfsys@defobject{currentmarker}{\pgfqpoint{-0.048611in}{0.000000in}}{\pgfqpoint{-0.000000in}{0.000000in}}{%
\pgfpathmoveto{\pgfqpoint{-0.000000in}{0.000000in}}%
\pgfpathlineto{\pgfqpoint{-0.048611in}{0.000000in}}%
\pgfusepath{stroke,fill}%
}%
\begin{pgfscope}%
\pgfsys@transformshift{0.500000in}{0.477951in}%
\pgfsys@useobject{currentmarker}{}%
\end{pgfscope}%
\end{pgfscope}%
\begin{pgfscope}%
\definecolor{textcolor}{rgb}{0.000000,0.000000,0.000000}%
\pgfsetstrokecolor{textcolor}%
\pgfsetfillcolor{textcolor}%
\pgftext[x=0.333333in, y=0.429726in, left, base]{\color{textcolor}\rmfamily\fontsize{10.000000}{12.000000}\selectfont \(\displaystyle {0}\)}%
\end{pgfscope}%
\begin{pgfscope}%
\pgfsetbuttcap%
\pgfsetroundjoin%
\definecolor{currentfill}{rgb}{0.000000,0.000000,0.000000}%
\pgfsetfillcolor{currentfill}%
\pgfsetlinewidth{0.803000pt}%
\definecolor{currentstroke}{rgb}{0.000000,0.000000,0.000000}%
\pgfsetstrokecolor{currentstroke}%
\pgfsetdash{}{0pt}%
\pgfsys@defobject{currentmarker}{\pgfqpoint{-0.048611in}{0.000000in}}{\pgfqpoint{-0.000000in}{0.000000in}}{%
\pgfpathmoveto{\pgfqpoint{-0.000000in}{0.000000in}}%
\pgfpathlineto{\pgfqpoint{-0.048611in}{0.000000in}}%
\pgfusepath{stroke,fill}%
}%
\begin{pgfscope}%
\pgfsys@transformshift{0.500000in}{0.792845in}%
\pgfsys@useobject{currentmarker}{}%
\end{pgfscope}%
\end{pgfscope}%
\begin{pgfscope}%
\definecolor{textcolor}{rgb}{0.000000,0.000000,0.000000}%
\pgfsetstrokecolor{textcolor}%
\pgfsetfillcolor{textcolor}%
\pgftext[x=0.333333in, y=0.744620in, left, base]{\color{textcolor}\rmfamily\fontsize{10.000000}{12.000000}\selectfont \(\displaystyle {1}\)}%
\end{pgfscope}%
\begin{pgfscope}%
\pgfsetbuttcap%
\pgfsetroundjoin%
\definecolor{currentfill}{rgb}{0.000000,0.000000,0.000000}%
\pgfsetfillcolor{currentfill}%
\pgfsetlinewidth{0.803000pt}%
\definecolor{currentstroke}{rgb}{0.000000,0.000000,0.000000}%
\pgfsetstrokecolor{currentstroke}%
\pgfsetdash{}{0pt}%
\pgfsys@defobject{currentmarker}{\pgfqpoint{-0.048611in}{0.000000in}}{\pgfqpoint{-0.000000in}{0.000000in}}{%
\pgfpathmoveto{\pgfqpoint{-0.000000in}{0.000000in}}%
\pgfpathlineto{\pgfqpoint{-0.048611in}{0.000000in}}%
\pgfusepath{stroke,fill}%
}%
\begin{pgfscope}%
\pgfsys@transformshift{0.500000in}{1.107739in}%
\pgfsys@useobject{currentmarker}{}%
\end{pgfscope}%
\end{pgfscope}%
\begin{pgfscope}%
\definecolor{textcolor}{rgb}{0.000000,0.000000,0.000000}%
\pgfsetstrokecolor{textcolor}%
\pgfsetfillcolor{textcolor}%
\pgftext[x=0.333333in, y=1.059514in, left, base]{\color{textcolor}\rmfamily\fontsize{10.000000}{12.000000}\selectfont \(\displaystyle {2}\)}%
\end{pgfscope}%
\begin{pgfscope}%
\pgfsetbuttcap%
\pgfsetroundjoin%
\definecolor{currentfill}{rgb}{0.000000,0.000000,0.000000}%
\pgfsetfillcolor{currentfill}%
\pgfsetlinewidth{0.803000pt}%
\definecolor{currentstroke}{rgb}{0.000000,0.000000,0.000000}%
\pgfsetstrokecolor{currentstroke}%
\pgfsetdash{}{0pt}%
\pgfsys@defobject{currentmarker}{\pgfqpoint{-0.048611in}{0.000000in}}{\pgfqpoint{-0.000000in}{0.000000in}}{%
\pgfpathmoveto{\pgfqpoint{-0.000000in}{0.000000in}}%
\pgfpathlineto{\pgfqpoint{-0.048611in}{0.000000in}}%
\pgfusepath{stroke,fill}%
}%
\begin{pgfscope}%
\pgfsys@transformshift{0.500000in}{1.422633in}%
\pgfsys@useobject{currentmarker}{}%
\end{pgfscope}%
\end{pgfscope}%
\begin{pgfscope}%
\definecolor{textcolor}{rgb}{0.000000,0.000000,0.000000}%
\pgfsetstrokecolor{textcolor}%
\pgfsetfillcolor{textcolor}%
\pgftext[x=0.333333in, y=1.374408in, left, base]{\color{textcolor}\rmfamily\fontsize{10.000000}{12.000000}\selectfont \(\displaystyle {3}\)}%
\end{pgfscope}%
\begin{pgfscope}%
\pgfsetbuttcap%
\pgfsetroundjoin%
\definecolor{currentfill}{rgb}{0.000000,0.000000,0.000000}%
\pgfsetfillcolor{currentfill}%
\pgfsetlinewidth{0.803000pt}%
\definecolor{currentstroke}{rgb}{0.000000,0.000000,0.000000}%
\pgfsetstrokecolor{currentstroke}%
\pgfsetdash{}{0pt}%
\pgfsys@defobject{currentmarker}{\pgfqpoint{-0.048611in}{0.000000in}}{\pgfqpoint{-0.000000in}{0.000000in}}{%
\pgfpathmoveto{\pgfqpoint{-0.000000in}{0.000000in}}%
\pgfpathlineto{\pgfqpoint{-0.048611in}{0.000000in}}%
\pgfusepath{stroke,fill}%
}%
\begin{pgfscope}%
\pgfsys@transformshift{0.500000in}{1.737527in}%
\pgfsys@useobject{currentmarker}{}%
\end{pgfscope}%
\end{pgfscope}%
\begin{pgfscope}%
\definecolor{textcolor}{rgb}{0.000000,0.000000,0.000000}%
\pgfsetstrokecolor{textcolor}%
\pgfsetfillcolor{textcolor}%
\pgftext[x=0.333333in, y=1.689302in, left, base]{\color{textcolor}\rmfamily\fontsize{10.000000}{12.000000}\selectfont \(\displaystyle {4}\)}%
\end{pgfscope}%
\begin{pgfscope}%
\pgfsetbuttcap%
\pgfsetroundjoin%
\definecolor{currentfill}{rgb}{0.000000,0.000000,0.000000}%
\pgfsetfillcolor{currentfill}%
\pgfsetlinewidth{0.803000pt}%
\definecolor{currentstroke}{rgb}{0.000000,0.000000,0.000000}%
\pgfsetstrokecolor{currentstroke}%
\pgfsetdash{}{0pt}%
\pgfsys@defobject{currentmarker}{\pgfqpoint{-0.048611in}{0.000000in}}{\pgfqpoint{-0.000000in}{0.000000in}}{%
\pgfpathmoveto{\pgfqpoint{-0.000000in}{0.000000in}}%
\pgfpathlineto{\pgfqpoint{-0.048611in}{0.000000in}}%
\pgfusepath{stroke,fill}%
}%
\begin{pgfscope}%
\pgfsys@transformshift{0.500000in}{2.052421in}%
\pgfsys@useobject{currentmarker}{}%
\end{pgfscope}%
\end{pgfscope}%
\begin{pgfscope}%
\definecolor{textcolor}{rgb}{0.000000,0.000000,0.000000}%
\pgfsetstrokecolor{textcolor}%
\pgfsetfillcolor{textcolor}%
\pgftext[x=0.333333in, y=2.004196in, left, base]{\color{textcolor}\rmfamily\fontsize{10.000000}{12.000000}\selectfont \(\displaystyle {5}\)}%
\end{pgfscope}%
\begin{pgfscope}%
\pgfsetbuttcap%
\pgfsetroundjoin%
\definecolor{currentfill}{rgb}{0.000000,0.000000,0.000000}%
\pgfsetfillcolor{currentfill}%
\pgfsetlinewidth{0.803000pt}%
\definecolor{currentstroke}{rgb}{0.000000,0.000000,0.000000}%
\pgfsetstrokecolor{currentstroke}%
\pgfsetdash{}{0pt}%
\pgfsys@defobject{currentmarker}{\pgfqpoint{-0.048611in}{0.000000in}}{\pgfqpoint{-0.000000in}{0.000000in}}{%
\pgfpathmoveto{\pgfqpoint{-0.000000in}{0.000000in}}%
\pgfpathlineto{\pgfqpoint{-0.048611in}{0.000000in}}%
\pgfusepath{stroke,fill}%
}%
\begin{pgfscope}%
\pgfsys@transformshift{0.500000in}{2.367315in}%
\pgfsys@useobject{currentmarker}{}%
\end{pgfscope}%
\end{pgfscope}%
\begin{pgfscope}%
\definecolor{textcolor}{rgb}{0.000000,0.000000,0.000000}%
\pgfsetstrokecolor{textcolor}%
\pgfsetfillcolor{textcolor}%
\pgftext[x=0.333333in, y=2.319090in, left, base]{\color{textcolor}\rmfamily\fontsize{10.000000}{12.000000}\selectfont \(\displaystyle {6}\)}%
\end{pgfscope}%
\begin{pgfscope}%
\definecolor{textcolor}{rgb}{0.000000,0.000000,0.000000}%
\pgfsetstrokecolor{textcolor}%
\pgfsetfillcolor{textcolor}%
\pgftext[x=0.500000in,y=2.681667in,left,base]{\color{textcolor}\rmfamily\fontsize{10.000000}{12.000000}\selectfont \(\displaystyle \times{10^{6}}{}\)}%
\end{pgfscope}%
\begin{pgfscope}%
\pgfpathrectangle{\pgfqpoint{0.500000in}{0.375000in}}{\pgfqpoint{3.100000in}{2.265000in}}%
\pgfusepath{clip}%
\pgfsetrectcap%
\pgfsetroundjoin%
\pgfsetlinewidth{1.505625pt}%
\definecolor{currentstroke}{rgb}{0.000000,0.000000,1.000000}%
\pgfsetstrokecolor{currentstroke}%
\pgfsetdash{}{0pt}%
\pgfpathmoveto{\pgfqpoint{0.640909in}{0.477965in}}%
\pgfpathlineto{\pgfqpoint{0.726782in}{0.478920in}}%
\pgfpathlineto{\pgfqpoint{0.742395in}{0.482216in}}%
\pgfpathlineto{\pgfqpoint{0.758008in}{0.490258in}}%
\pgfpathlineto{\pgfqpoint{0.773621in}{0.502962in}}%
\pgfpathlineto{\pgfqpoint{0.797041in}{0.526929in}}%
\pgfpathlineto{\pgfqpoint{0.820461in}{0.555025in}}%
\pgfpathlineto{\pgfqpoint{0.836074in}{0.578574in}}%
\pgfpathlineto{\pgfqpoint{0.851687in}{0.611420in}}%
\pgfpathlineto{\pgfqpoint{0.867300in}{0.659437in}}%
\pgfpathlineto{\pgfqpoint{0.882914in}{0.727928in}}%
\pgfpathlineto{\pgfqpoint{0.898527in}{0.825835in}}%
\pgfpathlineto{\pgfqpoint{0.914140in}{0.970420in}}%
\pgfpathlineto{\pgfqpoint{0.937560in}{1.205609in}}%
\pgfpathlineto{\pgfqpoint{0.945366in}{1.263931in}}%
\pgfpathlineto{\pgfqpoint{0.953173in}{1.307568in}}%
\pgfpathlineto{\pgfqpoint{0.960980in}{1.338678in}}%
\pgfpathlineto{\pgfqpoint{0.968786in}{1.357162in}}%
\pgfpathlineto{\pgfqpoint{0.976593in}{1.364037in}}%
\pgfpathlineto{\pgfqpoint{1.015626in}{1.363578in}}%
\pgfpathlineto{\pgfqpoint{1.023432in}{1.357785in}}%
\pgfpathlineto{\pgfqpoint{1.039046in}{1.338678in}}%
\pgfpathlineto{\pgfqpoint{1.070272in}{1.338678in}}%
\pgfpathlineto{\pgfqpoint{1.078079in}{1.443710in}}%
\pgfpathlineto{\pgfqpoint{1.101498in}{2.076373in}}%
\pgfpathlineto{\pgfqpoint{1.109305in}{2.230045in}}%
\pgfpathlineto{\pgfqpoint{1.117112in}{2.341908in}}%
\pgfpathlineto{\pgfqpoint{1.124918in}{2.413561in}}%
\pgfpathlineto{\pgfqpoint{1.132725in}{2.451057in}}%
\pgfpathlineto{\pgfqpoint{1.140531in}{2.464072in}}%
\pgfpathlineto{\pgfqpoint{1.156145in}{2.465148in}}%
\pgfpathlineto{\pgfqpoint{1.163951in}{2.474251in}}%
\pgfpathlineto{\pgfqpoint{1.187371in}{2.537045in}}%
\pgfpathlineto{\pgfqpoint{1.218597in}{2.537045in}}%
\pgfpathlineto{\pgfqpoint{1.226404in}{2.533109in}}%
\pgfpathlineto{\pgfqpoint{1.242017in}{2.474251in}}%
\pgfpathlineto{\pgfqpoint{1.281050in}{2.474251in}}%
\pgfpathlineto{\pgfqpoint{1.288857in}{2.443183in}}%
\pgfpathlineto{\pgfqpoint{1.304470in}{2.275408in}}%
\pgfpathlineto{\pgfqpoint{1.312277in}{2.185561in}}%
\pgfpathlineto{\pgfqpoint{1.320083in}{2.116024in}}%
\pgfpathlineto{\pgfqpoint{1.327890in}{2.075494in}}%
\pgfpathlineto{\pgfqpoint{1.335696in}{2.060445in}}%
\pgfpathlineto{\pgfqpoint{1.343503in}{2.059854in}}%
\pgfpathlineto{\pgfqpoint{1.351309in}{2.057163in}}%
\pgfpathlineto{\pgfqpoint{1.359116in}{2.038299in}}%
\pgfpathlineto{\pgfqpoint{1.366923in}{1.997435in}}%
\pgfpathlineto{\pgfqpoint{1.382536in}{1.866236in}}%
\pgfpathlineto{\pgfqpoint{1.390342in}{1.796911in}}%
\pgfpathlineto{\pgfqpoint{1.398149in}{1.838546in}}%
\pgfpathlineto{\pgfqpoint{1.413762in}{1.956788in}}%
\pgfpathlineto{\pgfqpoint{1.421569in}{1.994354in}}%
\pgfpathlineto{\pgfqpoint{1.429375in}{1.998122in}}%
\pgfpathlineto{\pgfqpoint{1.460602in}{1.998122in}}%
\pgfpathlineto{\pgfqpoint{1.468408in}{1.964361in}}%
\pgfpathlineto{\pgfqpoint{1.476215in}{1.897907in}}%
\pgfpathlineto{\pgfqpoint{1.515248in}{1.897907in}}%
\pgfpathlineto{\pgfqpoint{1.530861in}{1.811765in}}%
\pgfpathlineto{\pgfqpoint{1.538668in}{1.803340in}}%
\pgfpathlineto{\pgfqpoint{1.546474in}{1.800038in}}%
\pgfpathlineto{\pgfqpoint{1.554281in}{1.754089in}}%
\pgfpathlineto{\pgfqpoint{1.562088in}{1.725685in}}%
\pgfpathlineto{\pgfqpoint{1.577701in}{1.725685in}}%
\pgfpathlineto{\pgfqpoint{1.585507in}{1.672827in}}%
\pgfpathlineto{\pgfqpoint{1.593314in}{1.672827in}}%
\pgfpathlineto{\pgfqpoint{1.601121in}{1.635068in}}%
\pgfpathlineto{\pgfqpoint{1.608927in}{1.573663in}}%
\pgfpathlineto{\pgfqpoint{1.616734in}{1.571223in}}%
\pgfpathlineto{\pgfqpoint{1.624540in}{1.481616in}}%
\pgfpathlineto{\pgfqpoint{1.663573in}{1.481616in}}%
\pgfpathlineto{\pgfqpoint{1.671380in}{1.422762in}}%
\pgfpathlineto{\pgfqpoint{1.679187in}{1.411674in}}%
\pgfpathlineto{\pgfqpoint{1.686993in}{1.360933in}}%
\pgfpathlineto{\pgfqpoint{1.702606in}{1.196042in}}%
\pgfpathlineto{\pgfqpoint{1.710413in}{1.134743in}}%
\pgfpathlineto{\pgfqpoint{1.718220in}{1.097775in}}%
\pgfpathlineto{\pgfqpoint{1.726026in}{1.087922in}}%
\pgfpathlineto{\pgfqpoint{1.749446in}{1.087922in}}%
\pgfpathlineto{\pgfqpoint{1.757253in}{1.082399in}}%
\pgfpathlineto{\pgfqpoint{1.765059in}{1.047149in}}%
\pgfpathlineto{\pgfqpoint{1.772866in}{1.036378in}}%
\pgfpathlineto{\pgfqpoint{1.804092in}{1.547393in}}%
\pgfpathlineto{\pgfqpoint{1.811899in}{1.637705in}}%
\pgfpathlineto{\pgfqpoint{1.819705in}{1.702669in}}%
\pgfpathlineto{\pgfqpoint{1.827512in}{1.739797in}}%
\pgfpathlineto{\pgfqpoint{1.835319in}{1.756513in}}%
\pgfpathlineto{\pgfqpoint{1.843125in}{1.761968in}}%
\pgfpathlineto{\pgfqpoint{1.866545in}{1.762520in}}%
\pgfpathlineto{\pgfqpoint{1.889965in}{1.761300in}}%
\pgfpathlineto{\pgfqpoint{1.897771in}{1.758879in}}%
\pgfpathlineto{\pgfqpoint{1.905578in}{1.752873in}}%
\pgfpathlineto{\pgfqpoint{1.952418in}{1.752873in}}%
\pgfpathlineto{\pgfqpoint{1.960224in}{1.734973in}}%
\pgfpathlineto{\pgfqpoint{1.968031in}{1.696581in}}%
\pgfpathlineto{\pgfqpoint{1.975837in}{1.628357in}}%
\pgfpathlineto{\pgfqpoint{1.983644in}{1.656706in}}%
\pgfpathlineto{\pgfqpoint{1.991451in}{1.857415in}}%
\pgfpathlineto{\pgfqpoint{1.999257in}{2.012012in}}%
\pgfpathlineto{\pgfqpoint{2.007064in}{2.119781in}}%
\pgfpathlineto{\pgfqpoint{2.014870in}{2.184531in}}%
\pgfpathlineto{\pgfqpoint{2.022677in}{2.215295in}}%
\pgfpathlineto{\pgfqpoint{2.030484in}{2.226680in}}%
\pgfpathlineto{\pgfqpoint{2.038290in}{2.228733in}}%
\pgfpathlineto{\pgfqpoint{2.069516in}{2.228733in}}%
\pgfpathlineto{\pgfqpoint{2.077323in}{2.226751in}}%
\pgfpathlineto{\pgfqpoint{2.085130in}{2.222457in}}%
\pgfpathlineto{\pgfqpoint{2.092936in}{2.213488in}}%
\pgfpathlineto{\pgfqpoint{2.139776in}{2.213488in}}%
\pgfpathlineto{\pgfqpoint{2.147582in}{2.194383in}}%
\pgfpathlineto{\pgfqpoint{2.155389in}{2.152723in}}%
\pgfpathlineto{\pgfqpoint{2.163196in}{2.079878in}}%
\pgfpathlineto{\pgfqpoint{2.171002in}{1.969728in}}%
\pgfpathlineto{\pgfqpoint{2.194422in}{1.533504in}}%
\pgfpathlineto{\pgfqpoint{2.202229in}{1.435874in}}%
\pgfpathlineto{\pgfqpoint{2.210035in}{1.378784in}}%
\pgfpathlineto{\pgfqpoint{2.217842in}{1.359388in}}%
\pgfpathlineto{\pgfqpoint{2.241262in}{1.359388in}}%
\pgfpathlineto{\pgfqpoint{2.249068in}{1.353351in}}%
\pgfpathlineto{\pgfqpoint{2.256875in}{1.350137in}}%
\pgfpathlineto{\pgfqpoint{2.295908in}{1.349565in}}%
\pgfpathlineto{\pgfqpoint{2.303714in}{1.329230in}}%
\pgfpathlineto{\pgfqpoint{2.311521in}{1.280013in}}%
\pgfpathlineto{\pgfqpoint{2.327134in}{1.149139in}}%
\pgfpathlineto{\pgfqpoint{2.334941in}{1.106845in}}%
\pgfpathlineto{\pgfqpoint{2.350554in}{1.106845in}}%
\pgfpathlineto{\pgfqpoint{2.358361in}{1.101937in}}%
\pgfpathlineto{\pgfqpoint{2.366167in}{1.083067in}}%
\pgfpathlineto{\pgfqpoint{2.381780in}{1.083067in}}%
\pgfpathlineto{\pgfqpoint{2.389587in}{1.075107in}}%
\pgfpathlineto{\pgfqpoint{2.397394in}{1.073816in}}%
\pgfpathlineto{\pgfqpoint{2.405200in}{1.058664in}}%
\pgfpathlineto{\pgfqpoint{2.413007in}{0.995919in}}%
\pgfpathlineto{\pgfqpoint{2.420813in}{0.967689in}}%
\pgfpathlineto{\pgfqpoint{2.428620in}{0.965451in}}%
\pgfpathlineto{\pgfqpoint{2.436427in}{0.965451in}}%
\pgfpathlineto{\pgfqpoint{2.444233in}{0.961869in}}%
\pgfpathlineto{\pgfqpoint{2.452040in}{0.942610in}}%
\pgfpathlineto{\pgfqpoint{2.459846in}{0.929075in}}%
\pgfpathlineto{\pgfqpoint{2.467653in}{0.898884in}}%
\pgfpathlineto{\pgfqpoint{2.483266in}{0.826765in}}%
\pgfpathlineto{\pgfqpoint{2.491073in}{0.804732in}}%
\pgfpathlineto{\pgfqpoint{2.498879in}{0.801421in}}%
\pgfpathlineto{\pgfqpoint{2.506686in}{0.786261in}}%
\pgfpathlineto{\pgfqpoint{2.530106in}{0.688667in}}%
\pgfpathlineto{\pgfqpoint{2.545719in}{0.672233in}}%
\pgfpathlineto{\pgfqpoint{2.553526in}{0.668265in}}%
\pgfpathlineto{\pgfqpoint{2.561332in}{0.655905in}}%
\pgfpathlineto{\pgfqpoint{2.569139in}{0.637178in}}%
\pgfpathlineto{\pgfqpoint{2.608172in}{0.583348in}}%
\pgfpathlineto{\pgfqpoint{2.615978in}{0.579106in}}%
\pgfpathlineto{\pgfqpoint{2.623785in}{0.577643in}}%
\pgfpathlineto{\pgfqpoint{2.631592in}{0.623277in}}%
\pgfpathlineto{\pgfqpoint{2.639398in}{0.711783in}}%
\pgfpathlineto{\pgfqpoint{2.655011in}{0.954216in}}%
\pgfpathlineto{\pgfqpoint{2.662818in}{1.055849in}}%
\pgfpathlineto{\pgfqpoint{2.670625in}{1.129319in}}%
\pgfpathlineto{\pgfqpoint{2.678431in}{1.168898in}}%
\pgfpathlineto{\pgfqpoint{2.686238in}{1.186895in}}%
\pgfpathlineto{\pgfqpoint{2.694044in}{1.192901in}}%
\pgfpathlineto{\pgfqpoint{2.709658in}{1.193836in}}%
\pgfpathlineto{\pgfqpoint{2.787723in}{1.192901in}}%
\pgfpathlineto{\pgfqpoint{2.795530in}{1.295752in}}%
\pgfpathlineto{\pgfqpoint{2.811143in}{1.565289in}}%
\pgfpathlineto{\pgfqpoint{2.818950in}{1.667008in}}%
\pgfpathlineto{\pgfqpoint{2.826756in}{1.736713in}}%
\pgfpathlineto{\pgfqpoint{2.834563in}{1.775972in}}%
\pgfpathlineto{\pgfqpoint{2.842370in}{1.791296in}}%
\pgfpathlineto{\pgfqpoint{2.857983in}{1.794671in}}%
\pgfpathlineto{\pgfqpoint{2.865789in}{1.808734in}}%
\pgfpathlineto{\pgfqpoint{2.873596in}{1.842976in}}%
\pgfpathlineto{\pgfqpoint{2.897016in}{2.004742in}}%
\pgfpathlineto{\pgfqpoint{2.904822in}{2.032438in}}%
\pgfpathlineto{\pgfqpoint{2.943855in}{2.032167in}}%
\pgfpathlineto{\pgfqpoint{2.951662in}{2.003881in}}%
\pgfpathlineto{\pgfqpoint{2.959469in}{1.953405in}}%
\pgfpathlineto{\pgfqpoint{2.982888in}{1.953405in}}%
\pgfpathlineto{\pgfqpoint{2.998502in}{2.018402in}}%
\pgfpathlineto{\pgfqpoint{3.006308in}{2.026565in}}%
\pgfpathlineto{\pgfqpoint{3.037535in}{2.026565in}}%
\pgfpathlineto{\pgfqpoint{3.045341in}{2.006275in}}%
\pgfpathlineto{\pgfqpoint{3.053148in}{1.961629in}}%
\pgfpathlineto{\pgfqpoint{3.060954in}{1.929635in}}%
\pgfpathlineto{\pgfqpoint{3.076568in}{1.929635in}}%
\pgfpathlineto{\pgfqpoint{3.084374in}{1.960395in}}%
\pgfpathlineto{\pgfqpoint{3.092181in}{2.001890in}}%
\pgfpathlineto{\pgfqpoint{3.099987in}{2.018555in}}%
\pgfpathlineto{\pgfqpoint{3.131214in}{2.018555in}}%
\pgfpathlineto{\pgfqpoint{3.139020in}{2.006838in}}%
\pgfpathlineto{\pgfqpoint{3.146827in}{1.969043in}}%
\pgfpathlineto{\pgfqpoint{3.154634in}{1.913169in}}%
\pgfpathlineto{\pgfqpoint{3.170247in}{1.913169in}}%
\pgfpathlineto{\pgfqpoint{3.178053in}{1.936440in}}%
\pgfpathlineto{\pgfqpoint{3.185860in}{1.983433in}}%
\pgfpathlineto{\pgfqpoint{3.193667in}{2.007871in}}%
\pgfpathlineto{\pgfqpoint{3.224893in}{2.007871in}}%
\pgfpathlineto{\pgfqpoint{3.232700in}{2.004530in}}%
\pgfpathlineto{\pgfqpoint{3.240506in}{1.973988in}}%
\pgfpathlineto{\pgfqpoint{3.248313in}{1.922570in}}%
\pgfpathlineto{\pgfqpoint{3.271733in}{1.922570in}}%
\pgfpathlineto{\pgfqpoint{3.279539in}{1.962745in}}%
\pgfpathlineto{\pgfqpoint{3.287346in}{1.994704in}}%
\pgfpathlineto{\pgfqpoint{3.295152in}{2.000011in}}%
\pgfpathlineto{\pgfqpoint{3.326379in}{2.000011in}}%
\pgfpathlineto{\pgfqpoint{3.334185in}{1.977472in}}%
\pgfpathlineto{\pgfqpoint{3.341992in}{1.931671in}}%
\pgfpathlineto{\pgfqpoint{3.349799in}{1.911030in}}%
\pgfpathlineto{\pgfqpoint{3.365412in}{1.911030in}}%
\pgfpathlineto{\pgfqpoint{3.373218in}{1.940781in}}%
\pgfpathlineto{\pgfqpoint{3.381025in}{1.979268in}}%
\pgfpathlineto{\pgfqpoint{3.388832in}{1.992705in}}%
\pgfpathlineto{\pgfqpoint{3.420058in}{1.992705in}}%
\pgfpathlineto{\pgfqpoint{3.427865in}{1.978210in}}%
\pgfpathlineto{\pgfqpoint{3.435671in}{1.938664in}}%
\pgfpathlineto{\pgfqpoint{3.443478in}{1.885693in}}%
\pgfpathlineto{\pgfqpoint{3.459091in}{1.885693in}}%
\pgfpathlineto{\pgfqpoint{3.459091in}{1.885693in}}%
\pgfusepath{stroke}%
\end{pgfscope}%
\begin{pgfscope}%
\pgfpathrectangle{\pgfqpoint{0.500000in}{0.375000in}}{\pgfqpoint{3.100000in}{2.265000in}}%
\pgfusepath{clip}%
\pgfsetrectcap%
\pgfsetroundjoin%
\pgfsetlinewidth{1.505625pt}%
\definecolor{currentstroke}{rgb}{1.000000,0.000000,0.000000}%
\pgfsetstrokecolor{currentstroke}%
\pgfsetdash{}{0pt}%
\pgfpathmoveto{\pgfqpoint{0.640909in}{0.477955in}}%
\pgfpathlineto{\pgfqpoint{0.750201in}{0.478982in}}%
\pgfpathlineto{\pgfqpoint{0.773621in}{0.482448in}}%
\pgfpathlineto{\pgfqpoint{0.797041in}{0.490060in}}%
\pgfpathlineto{\pgfqpoint{0.812654in}{0.497878in}}%
\pgfpathlineto{\pgfqpoint{0.828267in}{0.508185in}}%
\pgfpathlineto{\pgfqpoint{0.843881in}{0.521653in}}%
\pgfpathlineto{\pgfqpoint{0.859494in}{0.539547in}}%
\pgfpathlineto{\pgfqpoint{0.875107in}{0.563491in}}%
\pgfpathlineto{\pgfqpoint{0.890720in}{0.595701in}}%
\pgfpathlineto{\pgfqpoint{0.906333in}{0.639834in}}%
\pgfpathlineto{\pgfqpoint{0.921947in}{0.700942in}}%
\pgfpathlineto{\pgfqpoint{0.937560in}{0.779457in}}%
\pgfpathlineto{\pgfqpoint{0.968786in}{0.965325in}}%
\pgfpathlineto{\pgfqpoint{0.992206in}{1.101623in}}%
\pgfpathlineto{\pgfqpoint{1.007819in}{1.179041in}}%
\pgfpathlineto{\pgfqpoint{1.015626in}{1.209615in}}%
\pgfpathlineto{\pgfqpoint{1.023432in}{1.233002in}}%
\pgfpathlineto{\pgfqpoint{1.031239in}{1.247033in}}%
\pgfpathlineto{\pgfqpoint{1.039046in}{1.249908in}}%
\pgfpathlineto{\pgfqpoint{1.046852in}{1.241734in}}%
\pgfpathlineto{\pgfqpoint{1.062465in}{1.210997in}}%
\pgfpathlineto{\pgfqpoint{1.070272in}{1.210322in}}%
\pgfpathlineto{\pgfqpoint{1.078079in}{1.218831in}}%
\pgfpathlineto{\pgfqpoint{1.085885in}{1.239026in}}%
\pgfpathlineto{\pgfqpoint{1.093692in}{1.273430in}}%
\pgfpathlineto{\pgfqpoint{1.101498in}{1.321109in}}%
\pgfpathlineto{\pgfqpoint{1.117112in}{1.449370in}}%
\pgfpathlineto{\pgfqpoint{1.140531in}{1.694789in}}%
\pgfpathlineto{\pgfqpoint{1.195178in}{2.278539in}}%
\pgfpathlineto{\pgfqpoint{1.202984in}{2.344762in}}%
\pgfpathlineto{\pgfqpoint{1.210791in}{2.390327in}}%
\pgfpathlineto{\pgfqpoint{1.218597in}{2.410888in}}%
\pgfpathlineto{\pgfqpoint{1.226404in}{2.406975in}}%
\pgfpathlineto{\pgfqpoint{1.234211in}{2.382473in}}%
\pgfpathlineto{\pgfqpoint{1.249824in}{2.300042in}}%
\pgfpathlineto{\pgfqpoint{1.265437in}{2.217754in}}%
\pgfpathlineto{\pgfqpoint{1.296663in}{2.078670in}}%
\pgfpathlineto{\pgfqpoint{1.343503in}{1.826175in}}%
\pgfpathlineto{\pgfqpoint{1.351309in}{1.802469in}}%
\pgfpathlineto{\pgfqpoint{1.366923in}{1.767944in}}%
\pgfpathlineto{\pgfqpoint{1.374729in}{1.754239in}}%
\pgfpathlineto{\pgfqpoint{1.382536in}{1.744369in}}%
\pgfpathlineto{\pgfqpoint{1.390342in}{1.739994in}}%
\pgfpathlineto{\pgfqpoint{1.398149in}{1.742596in}}%
\pgfpathlineto{\pgfqpoint{1.405956in}{1.752506in}}%
\pgfpathlineto{\pgfqpoint{1.421569in}{1.787566in}}%
\pgfpathlineto{\pgfqpoint{1.429375in}{1.805893in}}%
\pgfpathlineto{\pgfqpoint{1.437182in}{1.819995in}}%
\pgfpathlineto{\pgfqpoint{1.444989in}{1.829883in}}%
\pgfpathlineto{\pgfqpoint{1.452795in}{1.827668in}}%
\pgfpathlineto{\pgfqpoint{1.460602in}{1.813233in}}%
\pgfpathlineto{\pgfqpoint{1.484022in}{1.740435in}}%
\pgfpathlineto{\pgfqpoint{1.491828in}{1.729935in}}%
\pgfpathlineto{\pgfqpoint{1.499635in}{1.729304in}}%
\pgfpathlineto{\pgfqpoint{1.515248in}{1.731070in}}%
\pgfpathlineto{\pgfqpoint{1.523055in}{1.723187in}}%
\pgfpathlineto{\pgfqpoint{1.530861in}{1.709366in}}%
\pgfpathlineto{\pgfqpoint{1.538668in}{1.682335in}}%
\pgfpathlineto{\pgfqpoint{1.569894in}{1.545222in}}%
\pgfpathlineto{\pgfqpoint{1.577701in}{1.524367in}}%
\pgfpathlineto{\pgfqpoint{1.593314in}{1.496795in}}%
\pgfpathlineto{\pgfqpoint{1.608927in}{1.453424in}}%
\pgfpathlineto{\pgfqpoint{1.616734in}{1.424536in}}%
\pgfpathlineto{\pgfqpoint{1.647960in}{1.260399in}}%
\pgfpathlineto{\pgfqpoint{1.663573in}{1.193584in}}%
\pgfpathlineto{\pgfqpoint{1.679187in}{1.143818in}}%
\pgfpathlineto{\pgfqpoint{1.718220in}{1.022752in}}%
\pgfpathlineto{\pgfqpoint{1.733833in}{0.992241in}}%
\pgfpathlineto{\pgfqpoint{1.765059in}{0.940738in}}%
\pgfpathlineto{\pgfqpoint{1.772866in}{0.940065in}}%
\pgfpathlineto{\pgfqpoint{1.780672in}{0.949747in}}%
\pgfpathlineto{\pgfqpoint{1.788479in}{0.969945in}}%
\pgfpathlineto{\pgfqpoint{1.796286in}{0.999827in}}%
\pgfpathlineto{\pgfqpoint{1.811899in}{1.081187in}}%
\pgfpathlineto{\pgfqpoint{1.835319in}{1.234115in}}%
\pgfpathlineto{\pgfqpoint{1.882158in}{1.547400in}}%
\pgfpathlineto{\pgfqpoint{1.889965in}{1.590337in}}%
\pgfpathlineto{\pgfqpoint{1.897771in}{1.619027in}}%
\pgfpathlineto{\pgfqpoint{1.905578in}{1.626836in}}%
\pgfpathlineto{\pgfqpoint{1.913385in}{1.608618in}}%
\pgfpathlineto{\pgfqpoint{1.921191in}{1.563086in}}%
\pgfpathlineto{\pgfqpoint{1.928998in}{1.492295in}}%
\pgfpathlineto{\pgfqpoint{1.952418in}{1.215462in}}%
\pgfpathlineto{\pgfqpoint{1.960224in}{1.155486in}}%
\pgfpathlineto{\pgfqpoint{1.968031in}{1.122430in}}%
\pgfpathlineto{\pgfqpoint{1.975837in}{1.105891in}}%
\pgfpathlineto{\pgfqpoint{1.983644in}{1.107663in}}%
\pgfpathlineto{\pgfqpoint{1.991451in}{1.128579in}}%
\pgfpathlineto{\pgfqpoint{1.999257in}{1.168095in}}%
\pgfpathlineto{\pgfqpoint{2.007064in}{1.224361in}}%
\pgfpathlineto{\pgfqpoint{2.022677in}{1.376778in}}%
\pgfpathlineto{\pgfqpoint{2.046097in}{1.626178in}}%
\pgfpathlineto{\pgfqpoint{2.061710in}{1.780741in}}%
\pgfpathlineto{\pgfqpoint{2.077323in}{1.940329in}}%
\pgfpathlineto{\pgfqpoint{2.085130in}{1.998243in}}%
\pgfpathlineto{\pgfqpoint{2.092936in}{2.030175in}}%
\pgfpathlineto{\pgfqpoint{2.100743in}{2.030791in}}%
\pgfpathlineto{\pgfqpoint{2.108549in}{1.998835in}}%
\pgfpathlineto{\pgfqpoint{2.116356in}{1.939889in}}%
\pgfpathlineto{\pgfqpoint{2.155389in}{1.560754in}}%
\pgfpathlineto{\pgfqpoint{2.171002in}{1.457816in}}%
\pgfpathlineto{\pgfqpoint{2.178809in}{1.418662in}}%
\pgfpathlineto{\pgfqpoint{2.186615in}{1.388754in}}%
\pgfpathlineto{\pgfqpoint{2.194422in}{1.368838in}}%
\pgfpathlineto{\pgfqpoint{2.202229in}{1.357766in}}%
\pgfpathlineto{\pgfqpoint{2.210035in}{1.352986in}}%
\pgfpathlineto{\pgfqpoint{2.225648in}{1.348478in}}%
\pgfpathlineto{\pgfqpoint{2.233455in}{1.339686in}}%
\pgfpathlineto{\pgfqpoint{2.241262in}{1.322163in}}%
\pgfpathlineto{\pgfqpoint{2.256875in}{1.265643in}}%
\pgfpathlineto{\pgfqpoint{2.272488in}{1.203659in}}%
\pgfpathlineto{\pgfqpoint{2.288101in}{1.155734in}}%
\pgfpathlineto{\pgfqpoint{2.303714in}{1.120300in}}%
\pgfpathlineto{\pgfqpoint{2.311521in}{1.105773in}}%
\pgfpathlineto{\pgfqpoint{2.319328in}{1.094950in}}%
\pgfpathlineto{\pgfqpoint{2.342747in}{1.074167in}}%
\pgfpathlineto{\pgfqpoint{2.350554in}{1.058886in}}%
\pgfpathlineto{\pgfqpoint{2.381780in}{0.978504in}}%
\pgfpathlineto{\pgfqpoint{2.389587in}{0.968737in}}%
\pgfpathlineto{\pgfqpoint{2.405200in}{0.952741in}}%
\pgfpathlineto{\pgfqpoint{2.413007in}{0.940667in}}%
\pgfpathlineto{\pgfqpoint{2.428620in}{0.930362in}}%
\pgfpathlineto{\pgfqpoint{2.436427in}{0.924578in}}%
\pgfpathlineto{\pgfqpoint{2.444233in}{0.914243in}}%
\pgfpathlineto{\pgfqpoint{2.452040in}{0.899877in}}%
\pgfpathlineto{\pgfqpoint{2.459846in}{0.868653in}}%
\pgfpathlineto{\pgfqpoint{2.475460in}{0.792935in}}%
\pgfpathlineto{\pgfqpoint{2.483266in}{0.764870in}}%
\pgfpathlineto{\pgfqpoint{2.522299in}{0.668841in}}%
\pgfpathlineto{\pgfqpoint{2.530106in}{0.656070in}}%
\pgfpathlineto{\pgfqpoint{2.537912in}{0.646972in}}%
\pgfpathlineto{\pgfqpoint{2.553526in}{0.634549in}}%
\pgfpathlineto{\pgfqpoint{2.561332in}{0.625044in}}%
\pgfpathlineto{\pgfqpoint{2.576945in}{0.601569in}}%
\pgfpathlineto{\pgfqpoint{2.608172in}{0.567557in}}%
\pgfpathlineto{\pgfqpoint{2.615978in}{0.560711in}}%
\pgfpathlineto{\pgfqpoint{2.623785in}{0.559911in}}%
\pgfpathlineto{\pgfqpoint{2.631592in}{0.563356in}}%
\pgfpathlineto{\pgfqpoint{2.639398in}{0.572616in}}%
\pgfpathlineto{\pgfqpoint{2.647205in}{0.589113in}}%
\pgfpathlineto{\pgfqpoint{2.655011in}{0.612649in}}%
\pgfpathlineto{\pgfqpoint{2.670625in}{0.677478in}}%
\pgfpathlineto{\pgfqpoint{2.694044in}{0.795828in}}%
\pgfpathlineto{\pgfqpoint{2.740884in}{1.036309in}}%
\pgfpathlineto{\pgfqpoint{2.756497in}{1.103852in}}%
\pgfpathlineto{\pgfqpoint{2.764304in}{1.124311in}}%
\pgfpathlineto{\pgfqpoint{2.772110in}{1.133158in}}%
\pgfpathlineto{\pgfqpoint{2.779917in}{1.135862in}}%
\pgfpathlineto{\pgfqpoint{2.787723in}{1.142287in}}%
\pgfpathlineto{\pgfqpoint{2.795530in}{1.152689in}}%
\pgfpathlineto{\pgfqpoint{2.803337in}{1.169439in}}%
\pgfpathlineto{\pgfqpoint{2.811143in}{1.193088in}}%
\pgfpathlineto{\pgfqpoint{2.826756in}{1.256917in}}%
\pgfpathlineto{\pgfqpoint{2.850176in}{1.376502in}}%
\pgfpathlineto{\pgfqpoint{2.889209in}{1.606585in}}%
\pgfpathlineto{\pgfqpoint{2.912629in}{1.776774in}}%
\pgfpathlineto{\pgfqpoint{2.920436in}{1.821032in}}%
\pgfpathlineto{\pgfqpoint{2.928242in}{1.853213in}}%
\pgfpathlineto{\pgfqpoint{2.936049in}{1.866581in}}%
\pgfpathlineto{\pgfqpoint{2.943855in}{1.864701in}}%
\pgfpathlineto{\pgfqpoint{2.959469in}{1.841489in}}%
\pgfpathlineto{\pgfqpoint{2.967275in}{1.834709in}}%
\pgfpathlineto{\pgfqpoint{2.975082in}{1.838042in}}%
\pgfpathlineto{\pgfqpoint{2.982888in}{1.846477in}}%
\pgfpathlineto{\pgfqpoint{2.998502in}{1.868368in}}%
\pgfpathlineto{\pgfqpoint{3.006308in}{1.876556in}}%
\pgfpathlineto{\pgfqpoint{3.021921in}{1.884215in}}%
\pgfpathlineto{\pgfqpoint{3.029728in}{1.883360in}}%
\pgfpathlineto{\pgfqpoint{3.037535in}{1.875169in}}%
\pgfpathlineto{\pgfqpoint{3.060954in}{1.830459in}}%
\pgfpathlineto{\pgfqpoint{3.068761in}{1.827421in}}%
\pgfpathlineto{\pgfqpoint{3.076568in}{1.834036in}}%
\pgfpathlineto{\pgfqpoint{3.099987in}{1.865425in}}%
\pgfpathlineto{\pgfqpoint{3.115601in}{1.874423in}}%
\pgfpathlineto{\pgfqpoint{3.123407in}{1.875896in}}%
\pgfpathlineto{\pgfqpoint{3.131214in}{1.869928in}}%
\pgfpathlineto{\pgfqpoint{3.139020in}{1.857609in}}%
\pgfpathlineto{\pgfqpoint{3.154634in}{1.826398in}}%
\pgfpathlineto{\pgfqpoint{3.162440in}{1.819346in}}%
\pgfpathlineto{\pgfqpoint{3.170247in}{1.824325in}}%
\pgfpathlineto{\pgfqpoint{3.201473in}{1.862066in}}%
\pgfpathlineto{\pgfqpoint{3.217086in}{1.869403in}}%
\pgfpathlineto{\pgfqpoint{3.224893in}{1.865560in}}%
\pgfpathlineto{\pgfqpoint{3.232700in}{1.854901in}}%
\pgfpathlineto{\pgfqpoint{3.248313in}{1.823820in}}%
\pgfpathlineto{\pgfqpoint{3.256119in}{1.812853in}}%
\pgfpathlineto{\pgfqpoint{3.263926in}{1.815331in}}%
\pgfpathlineto{\pgfqpoint{3.271733in}{1.823755in}}%
\pgfpathlineto{\pgfqpoint{3.287346in}{1.845781in}}%
\pgfpathlineto{\pgfqpoint{3.295152in}{1.853437in}}%
\pgfpathlineto{\pgfqpoint{3.310766in}{1.861236in}}%
\pgfpathlineto{\pgfqpoint{3.318572in}{1.859716in}}%
\pgfpathlineto{\pgfqpoint{3.326379in}{1.851058in}}%
\pgfpathlineto{\pgfqpoint{3.349799in}{1.807214in}}%
\pgfpathlineto{\pgfqpoint{3.357605in}{1.805648in}}%
\pgfpathlineto{\pgfqpoint{3.365412in}{1.812700in}}%
\pgfpathlineto{\pgfqpoint{3.388832in}{1.843408in}}%
\pgfpathlineto{\pgfqpoint{3.404445in}{1.851735in}}%
\pgfpathlineto{\pgfqpoint{3.412251in}{1.852340in}}%
\pgfpathlineto{\pgfqpoint{3.420058in}{1.845665in}}%
\pgfpathlineto{\pgfqpoint{3.427865in}{1.832967in}}%
\pgfpathlineto{\pgfqpoint{3.443478in}{1.802293in}}%
\pgfpathlineto{\pgfqpoint{3.451284in}{1.796650in}}%
\pgfpathlineto{\pgfqpoint{3.459091in}{1.800768in}}%
\pgfpathlineto{\pgfqpoint{3.459091in}{1.800768in}}%
\pgfusepath{stroke}%
\end{pgfscope}%
\begin{pgfscope}%
\pgfpathrectangle{\pgfqpoint{0.500000in}{0.375000in}}{\pgfqpoint{3.100000in}{2.265000in}}%
\pgfusepath{clip}%
\pgfsetrectcap%
\pgfsetroundjoin%
\pgfsetlinewidth{1.505625pt}%
\definecolor{currentstroke}{rgb}{0.000000,0.500000,0.000000}%
\pgfsetstrokecolor{currentstroke}%
\pgfsetdash{}{0pt}%
\pgfpathmoveto{\pgfqpoint{0.640909in}{0.477955in}}%
\pgfpathlineto{\pgfqpoint{0.742395in}{0.479038in}}%
\pgfpathlineto{\pgfqpoint{0.765815in}{0.483114in}}%
\pgfpathlineto{\pgfqpoint{0.789234in}{0.490627in}}%
\pgfpathlineto{\pgfqpoint{0.836074in}{0.510254in}}%
\pgfpathlineto{\pgfqpoint{0.859494in}{0.523226in}}%
\pgfpathlineto{\pgfqpoint{0.875107in}{0.536525in}}%
\pgfpathlineto{\pgfqpoint{0.882914in}{0.545516in}}%
\pgfpathlineto{\pgfqpoint{0.898527in}{0.570659in}}%
\pgfpathlineto{\pgfqpoint{0.914140in}{0.609327in}}%
\pgfpathlineto{\pgfqpoint{0.945366in}{0.705712in}}%
\pgfpathlineto{\pgfqpoint{0.953173in}{0.723327in}}%
\pgfpathlineto{\pgfqpoint{0.960980in}{0.735562in}}%
\pgfpathlineto{\pgfqpoint{0.968786in}{0.741777in}}%
\pgfpathlineto{\pgfqpoint{0.976593in}{0.741578in}}%
\pgfpathlineto{\pgfqpoint{0.984399in}{0.735187in}}%
\pgfpathlineto{\pgfqpoint{0.992206in}{0.722810in}}%
\pgfpathlineto{\pgfqpoint{1.000013in}{0.705018in}}%
\pgfpathlineto{\pgfqpoint{1.007819in}{0.680841in}}%
\pgfpathlineto{\pgfqpoint{1.023432in}{0.617614in}}%
\pgfpathlineto{\pgfqpoint{1.031239in}{0.583506in}}%
\pgfpathlineto{\pgfqpoint{1.039046in}{0.556913in}}%
\pgfpathlineto{\pgfqpoint{1.046852in}{0.549828in}}%
\pgfpathlineto{\pgfqpoint{1.054659in}{0.559348in}}%
\pgfpathlineto{\pgfqpoint{1.062465in}{0.563791in}}%
\pgfpathlineto{\pgfqpoint{1.070272in}{0.563414in}}%
\pgfpathlineto{\pgfqpoint{1.078079in}{0.578175in}}%
\pgfpathlineto{\pgfqpoint{1.085885in}{0.622874in}}%
\pgfpathlineto{\pgfqpoint{1.101498in}{0.766364in}}%
\pgfpathlineto{\pgfqpoint{1.117112in}{0.905695in}}%
\pgfpathlineto{\pgfqpoint{1.124918in}{0.958253in}}%
\pgfpathlineto{\pgfqpoint{1.132725in}{0.995023in}}%
\pgfpathlineto{\pgfqpoint{1.140531in}{1.015532in}}%
\pgfpathlineto{\pgfqpoint{1.148338in}{1.024779in}}%
\pgfpathlineto{\pgfqpoint{1.156145in}{1.029654in}}%
\pgfpathlineto{\pgfqpoint{1.163951in}{1.019747in}}%
\pgfpathlineto{\pgfqpoint{1.171758in}{0.992743in}}%
\pgfpathlineto{\pgfqpoint{1.179564in}{0.945541in}}%
\pgfpathlineto{\pgfqpoint{1.195178in}{0.804638in}}%
\pgfpathlineto{\pgfqpoint{1.210791in}{0.652587in}}%
\pgfpathlineto{\pgfqpoint{1.218597in}{0.592257in}}%
\pgfpathlineto{\pgfqpoint{1.226404in}{0.563720in}}%
\pgfpathlineto{\pgfqpoint{1.234211in}{0.577616in}}%
\pgfpathlineto{\pgfqpoint{1.242017in}{0.605717in}}%
\pgfpathlineto{\pgfqpoint{1.249824in}{0.627906in}}%
\pgfpathlineto{\pgfqpoint{1.257630in}{0.639391in}}%
\pgfpathlineto{\pgfqpoint{1.265437in}{0.641291in}}%
\pgfpathlineto{\pgfqpoint{1.273244in}{0.635770in}}%
\pgfpathlineto{\pgfqpoint{1.281050in}{0.624079in}}%
\pgfpathlineto{\pgfqpoint{1.296663in}{0.587822in}}%
\pgfpathlineto{\pgfqpoint{1.304470in}{0.581030in}}%
\pgfpathlineto{\pgfqpoint{1.312277in}{0.587328in}}%
\pgfpathlineto{\pgfqpoint{1.327890in}{0.612899in}}%
\pgfpathlineto{\pgfqpoint{1.335696in}{0.619724in}}%
\pgfpathlineto{\pgfqpoint{1.343503in}{0.617419in}}%
\pgfpathlineto{\pgfqpoint{1.351309in}{0.605930in}}%
\pgfpathlineto{\pgfqpoint{1.366923in}{0.566157in}}%
\pgfpathlineto{\pgfqpoint{1.382536in}{0.524476in}}%
\pgfpathlineto{\pgfqpoint{1.390342in}{0.515171in}}%
\pgfpathlineto{\pgfqpoint{1.398149in}{0.520178in}}%
\pgfpathlineto{\pgfqpoint{1.413762in}{0.551381in}}%
\pgfpathlineto{\pgfqpoint{1.421569in}{0.566333in}}%
\pgfpathlineto{\pgfqpoint{1.429375in}{0.577043in}}%
\pgfpathlineto{\pgfqpoint{1.437182in}{0.582392in}}%
\pgfpathlineto{\pgfqpoint{1.444989in}{0.581793in}}%
\pgfpathlineto{\pgfqpoint{1.460602in}{0.561193in}}%
\pgfpathlineto{\pgfqpoint{1.468408in}{0.558148in}}%
\pgfpathlineto{\pgfqpoint{1.476215in}{0.559109in}}%
\pgfpathlineto{\pgfqpoint{1.499635in}{0.554601in}}%
\pgfpathlineto{\pgfqpoint{1.515248in}{0.556013in}}%
\pgfpathlineto{\pgfqpoint{1.523055in}{0.545644in}}%
\pgfpathlineto{\pgfqpoint{1.530861in}{0.537883in}}%
\pgfpathlineto{\pgfqpoint{1.538668in}{0.543469in}}%
\pgfpathlineto{\pgfqpoint{1.554281in}{0.579840in}}%
\pgfpathlineto{\pgfqpoint{1.562088in}{0.590239in}}%
\pgfpathlineto{\pgfqpoint{1.569894in}{0.591692in}}%
\pgfpathlineto{\pgfqpoint{1.577701in}{0.584509in}}%
\pgfpathlineto{\pgfqpoint{1.585507in}{0.571104in}}%
\pgfpathlineto{\pgfqpoint{1.593314in}{0.560783in}}%
\pgfpathlineto{\pgfqpoint{1.601121in}{0.538703in}}%
\pgfpathlineto{\pgfqpoint{1.608927in}{0.533352in}}%
\pgfpathlineto{\pgfqpoint{1.640154in}{0.612623in}}%
\pgfpathlineto{\pgfqpoint{1.647960in}{0.624728in}}%
\pgfpathlineto{\pgfqpoint{1.655767in}{0.627594in}}%
\pgfpathlineto{\pgfqpoint{1.663573in}{0.619390in}}%
\pgfpathlineto{\pgfqpoint{1.686993in}{0.561793in}}%
\pgfpathlineto{\pgfqpoint{1.694800in}{0.542207in}}%
\pgfpathlineto{\pgfqpoint{1.702606in}{0.533329in}}%
\pgfpathlineto{\pgfqpoint{1.710413in}{0.532916in}}%
\pgfpathlineto{\pgfqpoint{1.733833in}{0.537462in}}%
\pgfpathlineto{\pgfqpoint{1.749446in}{0.536284in}}%
\pgfpathlineto{\pgfqpoint{1.757253in}{0.530247in}}%
\pgfpathlineto{\pgfqpoint{1.765059in}{0.517401in}}%
\pgfpathlineto{\pgfqpoint{1.772866in}{0.515632in}}%
\pgfpathlineto{\pgfqpoint{1.780672in}{0.542124in}}%
\pgfpathlineto{\pgfqpoint{1.796286in}{0.631982in}}%
\pgfpathlineto{\pgfqpoint{1.811899in}{0.724634in}}%
\pgfpathlineto{\pgfqpoint{1.819705in}{0.763092in}}%
\pgfpathlineto{\pgfqpoint{1.827512in}{0.792898in}}%
\pgfpathlineto{\pgfqpoint{1.835319in}{0.812848in}}%
\pgfpathlineto{\pgfqpoint{1.843125in}{0.823837in}}%
\pgfpathlineto{\pgfqpoint{1.850932in}{0.829323in}}%
\pgfpathlineto{\pgfqpoint{1.858738in}{0.826624in}}%
\pgfpathlineto{\pgfqpoint{1.866545in}{0.814548in}}%
\pgfpathlineto{\pgfqpoint{1.874352in}{0.790792in}}%
\pgfpathlineto{\pgfqpoint{1.882158in}{0.753017in}}%
\pgfpathlineto{\pgfqpoint{1.905578in}{0.604758in}}%
\pgfpathlineto{\pgfqpoint{1.913385in}{0.602983in}}%
\pgfpathlineto{\pgfqpoint{1.921191in}{0.649851in}}%
\pgfpathlineto{\pgfqpoint{1.936804in}{0.774411in}}%
\pgfpathlineto{\pgfqpoint{1.944611in}{0.814317in}}%
\pgfpathlineto{\pgfqpoint{1.952418in}{0.820961in}}%
\pgfpathlineto{\pgfqpoint{1.960224in}{0.796496in}}%
\pgfpathlineto{\pgfqpoint{1.968031in}{0.759997in}}%
\pgfpathlineto{\pgfqpoint{1.975837in}{0.732238in}}%
\pgfpathlineto{\pgfqpoint{1.983644in}{0.735944in}}%
\pgfpathlineto{\pgfqpoint{1.991451in}{0.778761in}}%
\pgfpathlineto{\pgfqpoint{2.014870in}{0.965701in}}%
\pgfpathlineto{\pgfqpoint{2.022677in}{1.002034in}}%
\pgfpathlineto{\pgfqpoint{2.030484in}{1.019330in}}%
\pgfpathlineto{\pgfqpoint{2.046097in}{1.029182in}}%
\pgfpathlineto{\pgfqpoint{2.053903in}{1.026053in}}%
\pgfpathlineto{\pgfqpoint{2.061710in}{0.998342in}}%
\pgfpathlineto{\pgfqpoint{2.069516in}{0.942352in}}%
\pgfpathlineto{\pgfqpoint{2.085130in}{0.775364in}}%
\pgfpathlineto{\pgfqpoint{2.092936in}{0.690494in}}%
\pgfpathlineto{\pgfqpoint{2.100743in}{0.638683in}}%
\pgfpathlineto{\pgfqpoint{2.108549in}{0.653094in}}%
\pgfpathlineto{\pgfqpoint{2.124163in}{0.753497in}}%
\pgfpathlineto{\pgfqpoint{2.131969in}{0.783853in}}%
\pgfpathlineto{\pgfqpoint{2.139776in}{0.792831in}}%
\pgfpathlineto{\pgfqpoint{2.147582in}{0.781859in}}%
\pgfpathlineto{\pgfqpoint{2.155389in}{0.755340in}}%
\pgfpathlineto{\pgfqpoint{2.163196in}{0.715712in}}%
\pgfpathlineto{\pgfqpoint{2.194422in}{0.525542in}}%
\pgfpathlineto{\pgfqpoint{2.202229in}{0.499336in}}%
\pgfpathlineto{\pgfqpoint{2.210035in}{0.485253in}}%
\pgfpathlineto{\pgfqpoint{2.217842in}{0.480983in}}%
\pgfpathlineto{\pgfqpoint{2.225648in}{0.485734in}}%
\pgfpathlineto{\pgfqpoint{2.233455in}{0.501798in}}%
\pgfpathlineto{\pgfqpoint{2.256875in}{0.569850in}}%
\pgfpathlineto{\pgfqpoint{2.264681in}{0.582319in}}%
\pgfpathlineto{\pgfqpoint{2.272488in}{0.588246in}}%
\pgfpathlineto{\pgfqpoint{2.280295in}{0.588764in}}%
\pgfpathlineto{\pgfqpoint{2.288101in}{0.583954in}}%
\pgfpathlineto{\pgfqpoint{2.295908in}{0.573577in}}%
\pgfpathlineto{\pgfqpoint{2.303714in}{0.556780in}}%
\pgfpathlineto{\pgfqpoint{2.319328in}{0.513932in}}%
\pgfpathlineto{\pgfqpoint{2.327134in}{0.497536in}}%
\pgfpathlineto{\pgfqpoint{2.334941in}{0.490042in}}%
\pgfpathlineto{\pgfqpoint{2.342747in}{0.499657in}}%
\pgfpathlineto{\pgfqpoint{2.358361in}{0.535734in}}%
\pgfpathlineto{\pgfqpoint{2.366167in}{0.546907in}}%
\pgfpathlineto{\pgfqpoint{2.373974in}{0.550970in}}%
\pgfpathlineto{\pgfqpoint{2.381780in}{0.549893in}}%
\pgfpathlineto{\pgfqpoint{2.397394in}{0.539249in}}%
\pgfpathlineto{\pgfqpoint{2.405200in}{0.527544in}}%
\pgfpathlineto{\pgfqpoint{2.413007in}{0.507026in}}%
\pgfpathlineto{\pgfqpoint{2.420813in}{0.500153in}}%
\pgfpathlineto{\pgfqpoint{2.428620in}{0.497883in}}%
\pgfpathlineto{\pgfqpoint{2.436427in}{0.498639in}}%
\pgfpathlineto{\pgfqpoint{2.444233in}{0.504727in}}%
\pgfpathlineto{\pgfqpoint{2.452040in}{0.513513in}}%
\pgfpathlineto{\pgfqpoint{2.459846in}{0.513319in}}%
\pgfpathlineto{\pgfqpoint{2.467653in}{0.504110in}}%
\pgfpathlineto{\pgfqpoint{2.475460in}{0.508976in}}%
\pgfpathlineto{\pgfqpoint{2.491073in}{0.530773in}}%
\pgfpathlineto{\pgfqpoint{2.498879in}{0.534964in}}%
\pgfpathlineto{\pgfqpoint{2.506686in}{0.530041in}}%
\pgfpathlineto{\pgfqpoint{2.530106in}{0.499757in}}%
\pgfpathlineto{\pgfqpoint{2.537912in}{0.494826in}}%
\pgfpathlineto{\pgfqpoint{2.545719in}{0.493161in}}%
\pgfpathlineto{\pgfqpoint{2.561332in}{0.494765in}}%
\pgfpathlineto{\pgfqpoint{2.569139in}{0.496242in}}%
\pgfpathlineto{\pgfqpoint{2.584752in}{0.493232in}}%
\pgfpathlineto{\pgfqpoint{2.592559in}{0.490897in}}%
\pgfpathlineto{\pgfqpoint{2.623785in}{0.491314in}}%
\pgfpathlineto{\pgfqpoint{2.631592in}{0.498284in}}%
\pgfpathlineto{\pgfqpoint{2.639398in}{0.519238in}}%
\pgfpathlineto{\pgfqpoint{2.655011in}{0.594106in}}%
\pgfpathlineto{\pgfqpoint{2.670625in}{0.673048in}}%
\pgfpathlineto{\pgfqpoint{2.678431in}{0.703628in}}%
\pgfpathlineto{\pgfqpoint{2.686238in}{0.726064in}}%
\pgfpathlineto{\pgfqpoint{2.694044in}{0.740605in}}%
\pgfpathlineto{\pgfqpoint{2.701851in}{0.748408in}}%
\pgfpathlineto{\pgfqpoint{2.709658in}{0.752074in}}%
\pgfpathlineto{\pgfqpoint{2.717464in}{0.749736in}}%
\pgfpathlineto{\pgfqpoint{2.725271in}{0.740273in}}%
\pgfpathlineto{\pgfqpoint{2.733077in}{0.722706in}}%
\pgfpathlineto{\pgfqpoint{2.740884in}{0.696013in}}%
\pgfpathlineto{\pgfqpoint{2.748691in}{0.661045in}}%
\pgfpathlineto{\pgfqpoint{2.772110in}{0.540274in}}%
\pgfpathlineto{\pgfqpoint{2.779917in}{0.521230in}}%
\pgfpathlineto{\pgfqpoint{2.787723in}{0.516064in}}%
\pgfpathlineto{\pgfqpoint{2.795530in}{0.530922in}}%
\pgfpathlineto{\pgfqpoint{2.803337in}{0.564884in}}%
\pgfpathlineto{\pgfqpoint{2.826756in}{0.690568in}}%
\pgfpathlineto{\pgfqpoint{2.834563in}{0.723421in}}%
\pgfpathlineto{\pgfqpoint{2.842370in}{0.747859in}}%
\pgfpathlineto{\pgfqpoint{2.850176in}{0.763016in}}%
\pgfpathlineto{\pgfqpoint{2.857983in}{0.769320in}}%
\pgfpathlineto{\pgfqpoint{2.881403in}{0.772181in}}%
\pgfpathlineto{\pgfqpoint{2.889209in}{0.764113in}}%
\pgfpathlineto{\pgfqpoint{2.897016in}{0.745272in}}%
\pgfpathlineto{\pgfqpoint{2.904822in}{0.714909in}}%
\pgfpathlineto{\pgfqpoint{2.928242in}{0.610649in}}%
\pgfpathlineto{\pgfqpoint{2.936049in}{0.582839in}}%
\pgfpathlineto{\pgfqpoint{2.943855in}{0.561116in}}%
\pgfpathlineto{\pgfqpoint{2.951662in}{0.545685in}}%
\pgfpathlineto{\pgfqpoint{2.959469in}{0.533973in}}%
\pgfpathlineto{\pgfqpoint{2.967275in}{0.525073in}}%
\pgfpathlineto{\pgfqpoint{2.975082in}{0.523179in}}%
\pgfpathlineto{\pgfqpoint{2.982888in}{0.526748in}}%
\pgfpathlineto{\pgfqpoint{2.990695in}{0.535808in}}%
\pgfpathlineto{\pgfqpoint{3.006308in}{0.557230in}}%
\pgfpathlineto{\pgfqpoint{3.014115in}{0.561359in}}%
\pgfpathlineto{\pgfqpoint{3.021921in}{0.563663in}}%
\pgfpathlineto{\pgfqpoint{3.029728in}{0.559292in}}%
\pgfpathlineto{\pgfqpoint{3.053148in}{0.532107in}}%
\pgfpathlineto{\pgfqpoint{3.060954in}{0.523744in}}%
\pgfpathlineto{\pgfqpoint{3.068761in}{0.520233in}}%
\pgfpathlineto{\pgfqpoint{3.076568in}{0.523192in}}%
\pgfpathlineto{\pgfqpoint{3.084374in}{0.531043in}}%
\pgfpathlineto{\pgfqpoint{3.099987in}{0.553719in}}%
\pgfpathlineto{\pgfqpoint{3.107794in}{0.559864in}}%
\pgfpathlineto{\pgfqpoint{3.115601in}{0.563029in}}%
\pgfpathlineto{\pgfqpoint{3.123407in}{0.561002in}}%
\pgfpathlineto{\pgfqpoint{3.162440in}{0.519615in}}%
\pgfpathlineto{\pgfqpoint{3.170247in}{0.521467in}}%
\pgfpathlineto{\pgfqpoint{3.178053in}{0.527680in}}%
\pgfpathlineto{\pgfqpoint{3.201473in}{0.557990in}}%
\pgfpathlineto{\pgfqpoint{3.209280in}{0.561686in}}%
\pgfpathlineto{\pgfqpoint{3.217086in}{0.562081in}}%
\pgfpathlineto{\pgfqpoint{3.224893in}{0.555221in}}%
\pgfpathlineto{\pgfqpoint{3.240506in}{0.536430in}}%
\pgfpathlineto{\pgfqpoint{3.256119in}{0.521718in}}%
\pgfpathlineto{\pgfqpoint{3.263926in}{0.521984in}}%
\pgfpathlineto{\pgfqpoint{3.271733in}{0.526375in}}%
\pgfpathlineto{\pgfqpoint{3.295152in}{0.556424in}}%
\pgfpathlineto{\pgfqpoint{3.302959in}{0.560469in}}%
\pgfpathlineto{\pgfqpoint{3.310766in}{0.562439in}}%
\pgfpathlineto{\pgfqpoint{3.318572in}{0.557467in}}%
\pgfpathlineto{\pgfqpoint{3.341992in}{0.531024in}}%
\pgfpathlineto{\pgfqpoint{3.349799in}{0.522722in}}%
\pgfpathlineto{\pgfqpoint{3.357605in}{0.520193in}}%
\pgfpathlineto{\pgfqpoint{3.365412in}{0.523426in}}%
\pgfpathlineto{\pgfqpoint{3.373218in}{0.531497in}}%
\pgfpathlineto{\pgfqpoint{3.388832in}{0.553396in}}%
\pgfpathlineto{\pgfqpoint{3.396638in}{0.558724in}}%
\pgfpathlineto{\pgfqpoint{3.404445in}{0.561535in}}%
\pgfpathlineto{\pgfqpoint{3.412251in}{0.558710in}}%
\pgfpathlineto{\pgfqpoint{3.443478in}{0.523868in}}%
\pgfpathlineto{\pgfqpoint{3.451284in}{0.518760in}}%
\pgfpathlineto{\pgfqpoint{3.459091in}{0.519492in}}%
\pgfpathlineto{\pgfqpoint{3.459091in}{0.519492in}}%
\pgfusepath{stroke}%
\end{pgfscope}%
\begin{pgfscope}%
\pgfsetrectcap%
\pgfsetmiterjoin%
\pgfsetlinewidth{0.803000pt}%
\definecolor{currentstroke}{rgb}{0.000000,0.000000,0.000000}%
\pgfsetstrokecolor{currentstroke}%
\pgfsetdash{}{0pt}%
\pgfpathmoveto{\pgfqpoint{0.500000in}{0.375000in}}%
\pgfpathlineto{\pgfqpoint{0.500000in}{2.640000in}}%
\pgfusepath{stroke}%
\end{pgfscope}%
\begin{pgfscope}%
\pgfsetrectcap%
\pgfsetmiterjoin%
\pgfsetlinewidth{0.803000pt}%
\definecolor{currentstroke}{rgb}{0.000000,0.000000,0.000000}%
\pgfsetstrokecolor{currentstroke}%
\pgfsetdash{}{0pt}%
\pgfpathmoveto{\pgfqpoint{3.600000in}{0.375000in}}%
\pgfpathlineto{\pgfqpoint{3.600000in}{2.640000in}}%
\pgfusepath{stroke}%
\end{pgfscope}%
\begin{pgfscope}%
\pgfsetrectcap%
\pgfsetmiterjoin%
\pgfsetlinewidth{0.803000pt}%
\definecolor{currentstroke}{rgb}{0.000000,0.000000,0.000000}%
\pgfsetstrokecolor{currentstroke}%
\pgfsetdash{}{0pt}%
\pgfpathmoveto{\pgfqpoint{0.500000in}{0.375000in}}%
\pgfpathlineto{\pgfqpoint{3.600000in}{0.375000in}}%
\pgfusepath{stroke}%
\end{pgfscope}%
\begin{pgfscope}%
\pgfsetrectcap%
\pgfsetmiterjoin%
\pgfsetlinewidth{0.803000pt}%
\definecolor{currentstroke}{rgb}{0.000000,0.000000,0.000000}%
\pgfsetstrokecolor{currentstroke}%
\pgfsetdash{}{0pt}%
\pgfpathmoveto{\pgfqpoint{0.500000in}{2.640000in}}%
\pgfpathlineto{\pgfqpoint{3.600000in}{2.640000in}}%
\pgfusepath{stroke}%
\end{pgfscope}%
\begin{pgfscope}%
\pgfsetbuttcap%
\pgfsetmiterjoin%
\definecolor{currentfill}{rgb}{1.000000,1.000000,1.000000}%
\pgfsetfillcolor{currentfill}%
\pgfsetfillopacity{0.800000}%
\pgfsetlinewidth{1.003750pt}%
\definecolor{currentstroke}{rgb}{0.800000,0.800000,0.800000}%
\pgfsetstrokecolor{currentstroke}%
\pgfsetstrokeopacity{0.800000}%
\pgfsetdash{}{0pt}%
\pgfpathmoveto{\pgfqpoint{1.698533in}{1.947871in}}%
\pgfpathlineto{\pgfqpoint{2.401467in}{1.947871in}}%
\pgfpathquadraticcurveto{\pgfqpoint{2.429244in}{1.947871in}}{\pgfqpoint{2.429244in}{1.975648in}}%
\pgfpathlineto{\pgfqpoint{2.429244in}{2.542778in}}%
\pgfpathquadraticcurveto{\pgfqpoint{2.429244in}{2.570556in}}{\pgfqpoint{2.401467in}{2.570556in}}%
\pgfpathlineto{\pgfqpoint{1.698533in}{2.570556in}}%
\pgfpathquadraticcurveto{\pgfqpoint{1.670756in}{2.570556in}}{\pgfqpoint{1.670756in}{2.542778in}}%
\pgfpathlineto{\pgfqpoint{1.670756in}{1.975648in}}%
\pgfpathquadraticcurveto{\pgfqpoint{1.670756in}{1.947871in}}{\pgfqpoint{1.698533in}{1.947871in}}%
\pgfpathlineto{\pgfqpoint{1.698533in}{1.947871in}}%
\pgfpathclose%
\pgfusepath{stroke,fill}%
\end{pgfscope}%
\begin{pgfscope}%
\pgfsetrectcap%
\pgfsetroundjoin%
\pgfsetlinewidth{1.505625pt}%
\definecolor{currentstroke}{rgb}{0.000000,0.000000,1.000000}%
\pgfsetstrokecolor{currentstroke}%
\pgfsetdash{}{0pt}%
\pgfpathmoveto{\pgfqpoint{1.726311in}{2.466389in}}%
\pgfpathlineto{\pgfqpoint{1.865200in}{2.466389in}}%
\pgfpathlineto{\pgfqpoint{2.004089in}{2.466389in}}%
\pgfusepath{stroke}%
\end{pgfscope}%
\begin{pgfscope}%
\definecolor{textcolor}{rgb}{0.000000,0.000000,0.000000}%
\pgfsetstrokecolor{textcolor}%
\pgfsetfillcolor{textcolor}%
\pgftext[x=2.115200in,y=2.417778in,left,base]{\color{textcolor}\rmfamily\fontsize{10.000000}{12.000000}\selectfont max}%
\end{pgfscope}%
\begin{pgfscope}%
\pgfsetrectcap%
\pgfsetroundjoin%
\pgfsetlinewidth{1.505625pt}%
\definecolor{currentstroke}{rgb}{1.000000,0.000000,0.000000}%
\pgfsetstrokecolor{currentstroke}%
\pgfsetdash{}{0pt}%
\pgfpathmoveto{\pgfqpoint{1.726311in}{2.272716in}}%
\pgfpathlineto{\pgfqpoint{1.865200in}{2.272716in}}%
\pgfpathlineto{\pgfqpoint{2.004089in}{2.272716in}}%
\pgfusepath{stroke}%
\end{pgfscope}%
\begin{pgfscope}%
\definecolor{textcolor}{rgb}{0.000000,0.000000,0.000000}%
\pgfsetstrokecolor{textcolor}%
\pgfsetfillcolor{textcolor}%
\pgftext[x=2.115200in,y=2.224105in,left,base]{\color{textcolor}\rmfamily\fontsize{10.000000}{12.000000}\selectfont \(\displaystyle \mu\)}%
\end{pgfscope}%
\begin{pgfscope}%
\pgfsetrectcap%
\pgfsetroundjoin%
\pgfsetlinewidth{1.505625pt}%
\definecolor{currentstroke}{rgb}{0.000000,0.500000,0.000000}%
\pgfsetstrokecolor{currentstroke}%
\pgfsetdash{}{0pt}%
\pgfpathmoveto{\pgfqpoint{1.726311in}{2.079043in}}%
\pgfpathlineto{\pgfqpoint{1.865200in}{2.079043in}}%
\pgfpathlineto{\pgfqpoint{2.004089in}{2.079043in}}%
\pgfusepath{stroke}%
\end{pgfscope}%
\begin{pgfscope}%
\definecolor{textcolor}{rgb}{0.000000,0.000000,0.000000}%
\pgfsetstrokecolor{textcolor}%
\pgfsetfillcolor{textcolor}%
\pgftext[x=2.115200in,y=2.030432in,left,base]{\color{textcolor}\rmfamily\fontsize{10.000000}{12.000000}\selectfont \(\displaystyle \sigma\)}%
\end{pgfscope}%
\end{pgfpicture}%
\makeatother%
\endgroup%
}
%         \caption{Pressure Matrix Profile New CS}
%         \label{fig:mp_hist_signal_pressure}
%     \end{minipage}
% \end{figure}

\subsubsection{Power Electronic Converter Data Set}
\label{ref_results_pec_sim}
The Power Electronic Converter (PEC) data set presented in section \ref{ref_pec_dataset} is used to perform this simulation. Table \ref{tab:pec_sim_params} specifies the parameters used with the anomaly detector to obtain the results in this section. This dataset is significantly larger than the previous, containing approximately 300,000 time steps (~83 hour of signal data). This dataset outlines the scenarios at which this detector excels and its performance for medium window sizes. The standard deviation multiplier is significant. If the data follows a standard Gaussian distribution, the detected points would fall outside 99.999\% of the data present in the window, which would represent a significant outlier comparative to the rest of the data. The rolling range multiplier is disabled for this experiment since the outliers are not relational to each other. There are 4 different types of outliers which all have radically different behaviors and signal shapes, so it is not desired to compare or remember them in the context of the next outlier. The recent range detection debounce multiplier is only one due to the larger size of the time series window and overall window size.

\begin{table}[H]
%%\centering
\begin{tabular}{|l|c|l|}
    \hline
	\textbf{Parameter} & \textbf{Value} & \textbf{Description} \\ \hline
	m & 250 & Window Size \\ \hline
	ts$\_$size & 5000 & Time Series Size \\ \hline
	std$\_$dev & 4 & Standard Deviation Multiplier \\ \hline
	range & 0 & Rolling Range Multiplier\\ \hline
	recent & 1 & Recent Detection Debounce\\ \hline
\end{tabular}
\caption{PEC Data Set Detector Parameters}
\label{tab:pec_sim_params}
\end{table}

Figure \ref{fig:pec_mp_hist} shows the computed matrix profile values during the experiment. Because of the scale variance of the anomalies, only 2 of the 4 detected anomalies are visually represented in the figure. The 2 anomalies that are not depicted show similar patterns, but at much smaller scale. This shows the detector is able to perform on signals with widely differing anomaly characteristics and patterns.

\begin{figure}[H]
    %%\centering
    %% Creator: Matplotlib, PGF backend
%%
%% To include the figure in your LaTeX document, write
%%   \input{<filename>.pgf}
%%
%% Make sure the required packages are loaded in your preamble
%%   \usepackage{pgf}
%%
%% Also ensure that all the required font packages are loaded; for instance,
%% the lmodern package is sometimes necessary when using math font.
%%   \usepackage{lmodern}
%%
%% Figures using additional raster images can only be included by \input if
%% they are in the same directory as the main LaTeX file. For loading figures
%% from other directories you can use the `import` package
%%   \usepackage{import}
%%
%% and then include the figures with
%%   \import{<path to file>}{<filename>.pgf}
%%
%% Matplotlib used the following preamble
%%
\begingroup%
\makeatletter%
\begin{pgfpicture}%
\pgfpathrectangle{\pgfpointorigin}{\pgfqpoint{6.000000in}{4.000000in}}%
\pgfusepath{use as bounding box, clip}%
\begin{pgfscope}%
\pgfsetbuttcap%
\pgfsetmiterjoin%
\pgfsetlinewidth{0.000000pt}%
\definecolor{currentstroke}{rgb}{1.000000,1.000000,1.000000}%
\pgfsetstrokecolor{currentstroke}%
\pgfsetstrokeopacity{0.000000}%
\pgfsetdash{}{0pt}%
\pgfpathmoveto{\pgfqpoint{0.000000in}{0.000000in}}%
\pgfpathlineto{\pgfqpoint{6.000000in}{0.000000in}}%
\pgfpathlineto{\pgfqpoint{6.000000in}{4.000000in}}%
\pgfpathlineto{\pgfqpoint{0.000000in}{4.000000in}}%
\pgfpathlineto{\pgfqpoint{0.000000in}{0.000000in}}%
\pgfpathclose%
\pgfusepath{}%
\end{pgfscope}%
\begin{pgfscope}%
\pgfsetbuttcap%
\pgfsetmiterjoin%
\definecolor{currentfill}{rgb}{1.000000,1.000000,1.000000}%
\pgfsetfillcolor{currentfill}%
\pgfsetlinewidth{0.000000pt}%
\definecolor{currentstroke}{rgb}{0.000000,0.000000,0.000000}%
\pgfsetstrokecolor{currentstroke}%
\pgfsetstrokeopacity{0.000000}%
\pgfsetdash{}{0pt}%
\pgfpathmoveto{\pgfqpoint{0.750000in}{0.500000in}}%
\pgfpathlineto{\pgfqpoint{5.400000in}{0.500000in}}%
\pgfpathlineto{\pgfqpoint{5.400000in}{3.520000in}}%
\pgfpathlineto{\pgfqpoint{0.750000in}{3.520000in}}%
\pgfpathlineto{\pgfqpoint{0.750000in}{0.500000in}}%
\pgfpathclose%
\pgfusepath{fill}%
\end{pgfscope}%
\begin{pgfscope}%
\pgfsetbuttcap%
\pgfsetroundjoin%
\definecolor{currentfill}{rgb}{0.000000,0.000000,0.000000}%
\pgfsetfillcolor{currentfill}%
\pgfsetlinewidth{0.803000pt}%
\definecolor{currentstroke}{rgb}{0.000000,0.000000,0.000000}%
\pgfsetstrokecolor{currentstroke}%
\pgfsetdash{}{0pt}%
\pgfsys@defobject{currentmarker}{\pgfqpoint{0.000000in}{-0.048611in}}{\pgfqpoint{0.000000in}{0.000000in}}{%
\pgfpathmoveto{\pgfqpoint{0.000000in}{0.000000in}}%
\pgfpathlineto{\pgfqpoint{0.000000in}{-0.048611in}}%
\pgfusepath{stroke,fill}%
}%
\begin{pgfscope}%
\pgfsys@transformshift{0.961364in}{0.500000in}%
\pgfsys@useobject{currentmarker}{}%
\end{pgfscope}%
\end{pgfscope}%
\begin{pgfscope}%
\definecolor{textcolor}{rgb}{0.000000,0.000000,0.000000}%
\pgfsetstrokecolor{textcolor}%
\pgfsetfillcolor{textcolor}%
\pgftext[x=0.961364in,y=0.402778in,,top]{\color{textcolor}\rmfamily\fontsize{10.000000}{12.000000}\selectfont \(\displaystyle {0}\)}%
\end{pgfscope}%
\begin{pgfscope}%
\pgfsetbuttcap%
\pgfsetroundjoin%
\definecolor{currentfill}{rgb}{0.000000,0.000000,0.000000}%
\pgfsetfillcolor{currentfill}%
\pgfsetlinewidth{0.803000pt}%
\definecolor{currentstroke}{rgb}{0.000000,0.000000,0.000000}%
\pgfsetstrokecolor{currentstroke}%
\pgfsetdash{}{0pt}%
\pgfsys@defobject{currentmarker}{\pgfqpoint{0.000000in}{-0.048611in}}{\pgfqpoint{0.000000in}{0.000000in}}{%
\pgfpathmoveto{\pgfqpoint{0.000000in}{0.000000in}}%
\pgfpathlineto{\pgfqpoint{0.000000in}{-0.048611in}}%
\pgfusepath{stroke,fill}%
}%
\begin{pgfscope}%
\pgfsys@transformshift{1.632244in}{0.500000in}%
\pgfsys@useobject{currentmarker}{}%
\end{pgfscope}%
\end{pgfscope}%
\begin{pgfscope}%
\definecolor{textcolor}{rgb}{0.000000,0.000000,0.000000}%
\pgfsetstrokecolor{textcolor}%
\pgfsetfillcolor{textcolor}%
\pgftext[x=1.632244in,y=0.402778in,,top]{\color{textcolor}\rmfamily\fontsize{10.000000}{12.000000}\selectfont \(\displaystyle {50000}\)}%
\end{pgfscope}%
\begin{pgfscope}%
\pgfsetbuttcap%
\pgfsetroundjoin%
\definecolor{currentfill}{rgb}{0.000000,0.000000,0.000000}%
\pgfsetfillcolor{currentfill}%
\pgfsetlinewidth{0.803000pt}%
\definecolor{currentstroke}{rgb}{0.000000,0.000000,0.000000}%
\pgfsetstrokecolor{currentstroke}%
\pgfsetdash{}{0pt}%
\pgfsys@defobject{currentmarker}{\pgfqpoint{0.000000in}{-0.048611in}}{\pgfqpoint{0.000000in}{0.000000in}}{%
\pgfpathmoveto{\pgfqpoint{0.000000in}{0.000000in}}%
\pgfpathlineto{\pgfqpoint{0.000000in}{-0.048611in}}%
\pgfusepath{stroke,fill}%
}%
\begin{pgfscope}%
\pgfsys@transformshift{2.303125in}{0.500000in}%
\pgfsys@useobject{currentmarker}{}%
\end{pgfscope}%
\end{pgfscope}%
\begin{pgfscope}%
\definecolor{textcolor}{rgb}{0.000000,0.000000,0.000000}%
\pgfsetstrokecolor{textcolor}%
\pgfsetfillcolor{textcolor}%
\pgftext[x=2.303125in,y=0.402778in,,top]{\color{textcolor}\rmfamily\fontsize{10.000000}{12.000000}\selectfont \(\displaystyle {100000}\)}%
\end{pgfscope}%
\begin{pgfscope}%
\pgfsetbuttcap%
\pgfsetroundjoin%
\definecolor{currentfill}{rgb}{0.000000,0.000000,0.000000}%
\pgfsetfillcolor{currentfill}%
\pgfsetlinewidth{0.803000pt}%
\definecolor{currentstroke}{rgb}{0.000000,0.000000,0.000000}%
\pgfsetstrokecolor{currentstroke}%
\pgfsetdash{}{0pt}%
\pgfsys@defobject{currentmarker}{\pgfqpoint{0.000000in}{-0.048611in}}{\pgfqpoint{0.000000in}{0.000000in}}{%
\pgfpathmoveto{\pgfqpoint{0.000000in}{0.000000in}}%
\pgfpathlineto{\pgfqpoint{0.000000in}{-0.048611in}}%
\pgfusepath{stroke,fill}%
}%
\begin{pgfscope}%
\pgfsys@transformshift{2.974006in}{0.500000in}%
\pgfsys@useobject{currentmarker}{}%
\end{pgfscope}%
\end{pgfscope}%
\begin{pgfscope}%
\definecolor{textcolor}{rgb}{0.000000,0.000000,0.000000}%
\pgfsetstrokecolor{textcolor}%
\pgfsetfillcolor{textcolor}%
\pgftext[x=2.974006in,y=0.402778in,,top]{\color{textcolor}\rmfamily\fontsize{10.000000}{12.000000}\selectfont \(\displaystyle {150000}\)}%
\end{pgfscope}%
\begin{pgfscope}%
\pgfsetbuttcap%
\pgfsetroundjoin%
\definecolor{currentfill}{rgb}{0.000000,0.000000,0.000000}%
\pgfsetfillcolor{currentfill}%
\pgfsetlinewidth{0.803000pt}%
\definecolor{currentstroke}{rgb}{0.000000,0.000000,0.000000}%
\pgfsetstrokecolor{currentstroke}%
\pgfsetdash{}{0pt}%
\pgfsys@defobject{currentmarker}{\pgfqpoint{0.000000in}{-0.048611in}}{\pgfqpoint{0.000000in}{0.000000in}}{%
\pgfpathmoveto{\pgfqpoint{0.000000in}{0.000000in}}%
\pgfpathlineto{\pgfqpoint{0.000000in}{-0.048611in}}%
\pgfusepath{stroke,fill}%
}%
\begin{pgfscope}%
\pgfsys@transformshift{3.644886in}{0.500000in}%
\pgfsys@useobject{currentmarker}{}%
\end{pgfscope}%
\end{pgfscope}%
\begin{pgfscope}%
\definecolor{textcolor}{rgb}{0.000000,0.000000,0.000000}%
\pgfsetstrokecolor{textcolor}%
\pgfsetfillcolor{textcolor}%
\pgftext[x=3.644886in,y=0.402778in,,top]{\color{textcolor}\rmfamily\fontsize{10.000000}{12.000000}\selectfont \(\displaystyle {200000}\)}%
\end{pgfscope}%
\begin{pgfscope}%
\pgfsetbuttcap%
\pgfsetroundjoin%
\definecolor{currentfill}{rgb}{0.000000,0.000000,0.000000}%
\pgfsetfillcolor{currentfill}%
\pgfsetlinewidth{0.803000pt}%
\definecolor{currentstroke}{rgb}{0.000000,0.000000,0.000000}%
\pgfsetstrokecolor{currentstroke}%
\pgfsetdash{}{0pt}%
\pgfsys@defobject{currentmarker}{\pgfqpoint{0.000000in}{-0.048611in}}{\pgfqpoint{0.000000in}{0.000000in}}{%
\pgfpathmoveto{\pgfqpoint{0.000000in}{0.000000in}}%
\pgfpathlineto{\pgfqpoint{0.000000in}{-0.048611in}}%
\pgfusepath{stroke,fill}%
}%
\begin{pgfscope}%
\pgfsys@transformshift{4.315767in}{0.500000in}%
\pgfsys@useobject{currentmarker}{}%
\end{pgfscope}%
\end{pgfscope}%
\begin{pgfscope}%
\definecolor{textcolor}{rgb}{0.000000,0.000000,0.000000}%
\pgfsetstrokecolor{textcolor}%
\pgfsetfillcolor{textcolor}%
\pgftext[x=4.315767in,y=0.402778in,,top]{\color{textcolor}\rmfamily\fontsize{10.000000}{12.000000}\selectfont \(\displaystyle {250000}\)}%
\end{pgfscope}%
\begin{pgfscope}%
\pgfsetbuttcap%
\pgfsetroundjoin%
\definecolor{currentfill}{rgb}{0.000000,0.000000,0.000000}%
\pgfsetfillcolor{currentfill}%
\pgfsetlinewidth{0.803000pt}%
\definecolor{currentstroke}{rgb}{0.000000,0.000000,0.000000}%
\pgfsetstrokecolor{currentstroke}%
\pgfsetdash{}{0pt}%
\pgfsys@defobject{currentmarker}{\pgfqpoint{0.000000in}{-0.048611in}}{\pgfqpoint{0.000000in}{0.000000in}}{%
\pgfpathmoveto{\pgfqpoint{0.000000in}{0.000000in}}%
\pgfpathlineto{\pgfqpoint{0.000000in}{-0.048611in}}%
\pgfusepath{stroke,fill}%
}%
\begin{pgfscope}%
\pgfsys@transformshift{4.986648in}{0.500000in}%
\pgfsys@useobject{currentmarker}{}%
\end{pgfscope}%
\end{pgfscope}%
\begin{pgfscope}%
\definecolor{textcolor}{rgb}{0.000000,0.000000,0.000000}%
\pgfsetstrokecolor{textcolor}%
\pgfsetfillcolor{textcolor}%
\pgftext[x=4.986648in,y=0.402778in,,top]{\color{textcolor}\rmfamily\fontsize{10.000000}{12.000000}\selectfont \(\displaystyle {300000}\)}%
\end{pgfscope}%
\begin{pgfscope}%
\definecolor{textcolor}{rgb}{0.000000,0.000000,0.000000}%
\pgfsetstrokecolor{textcolor}%
\pgfsetfillcolor{textcolor}%
\pgftext[x=3.075000in,y=0.223766in,,top]{\color{textcolor}\rmfamily\fontsize{10.000000}{12.000000}\selectfont time}%
\end{pgfscope}%
\begin{pgfscope}%
\pgfsetbuttcap%
\pgfsetroundjoin%
\definecolor{currentfill}{rgb}{0.000000,0.000000,0.000000}%
\pgfsetfillcolor{currentfill}%
\pgfsetlinewidth{0.803000pt}%
\definecolor{currentstroke}{rgb}{0.000000,0.000000,0.000000}%
\pgfsetstrokecolor{currentstroke}%
\pgfsetdash{}{0pt}%
\pgfsys@defobject{currentmarker}{\pgfqpoint{-0.048611in}{0.000000in}}{\pgfqpoint{-0.000000in}{0.000000in}}{%
\pgfpathmoveto{\pgfqpoint{-0.000000in}{0.000000in}}%
\pgfpathlineto{\pgfqpoint{-0.048611in}{0.000000in}}%
\pgfusepath{stroke,fill}%
}%
\begin{pgfscope}%
\pgfsys@transformshift{0.750000in}{0.637273in}%
\pgfsys@useobject{currentmarker}{}%
\end{pgfscope}%
\end{pgfscope}%
\begin{pgfscope}%
\definecolor{textcolor}{rgb}{0.000000,0.000000,0.000000}%
\pgfsetstrokecolor{textcolor}%
\pgfsetfillcolor{textcolor}%
\pgftext[x=0.583333in, y=0.589047in, left, base]{\color{textcolor}\rmfamily\fontsize{10.000000}{12.000000}\selectfont \(\displaystyle {0}\)}%
\end{pgfscope}%
\begin{pgfscope}%
\pgfsetbuttcap%
\pgfsetroundjoin%
\definecolor{currentfill}{rgb}{0.000000,0.000000,0.000000}%
\pgfsetfillcolor{currentfill}%
\pgfsetlinewidth{0.803000pt}%
\definecolor{currentstroke}{rgb}{0.000000,0.000000,0.000000}%
\pgfsetstrokecolor{currentstroke}%
\pgfsetdash{}{0pt}%
\pgfsys@defobject{currentmarker}{\pgfqpoint{-0.048611in}{0.000000in}}{\pgfqpoint{-0.000000in}{0.000000in}}{%
\pgfpathmoveto{\pgfqpoint{-0.000000in}{0.000000in}}%
\pgfpathlineto{\pgfqpoint{-0.048611in}{0.000000in}}%
\pgfusepath{stroke,fill}%
}%
\begin{pgfscope}%
\pgfsys@transformshift{0.750000in}{1.112883in}%
\pgfsys@useobject{currentmarker}{}%
\end{pgfscope}%
\end{pgfscope}%
\begin{pgfscope}%
\definecolor{textcolor}{rgb}{0.000000,0.000000,0.000000}%
\pgfsetstrokecolor{textcolor}%
\pgfsetfillcolor{textcolor}%
\pgftext[x=0.305554in, y=1.064658in, left, base]{\color{textcolor}\rmfamily\fontsize{10.000000}{12.000000}\selectfont \(\displaystyle {10000}\)}%
\end{pgfscope}%
\begin{pgfscope}%
\pgfsetbuttcap%
\pgfsetroundjoin%
\definecolor{currentfill}{rgb}{0.000000,0.000000,0.000000}%
\pgfsetfillcolor{currentfill}%
\pgfsetlinewidth{0.803000pt}%
\definecolor{currentstroke}{rgb}{0.000000,0.000000,0.000000}%
\pgfsetstrokecolor{currentstroke}%
\pgfsetdash{}{0pt}%
\pgfsys@defobject{currentmarker}{\pgfqpoint{-0.048611in}{0.000000in}}{\pgfqpoint{-0.000000in}{0.000000in}}{%
\pgfpathmoveto{\pgfqpoint{-0.000000in}{0.000000in}}%
\pgfpathlineto{\pgfqpoint{-0.048611in}{0.000000in}}%
\pgfusepath{stroke,fill}%
}%
\begin{pgfscope}%
\pgfsys@transformshift{0.750000in}{1.588494in}%
\pgfsys@useobject{currentmarker}{}%
\end{pgfscope}%
\end{pgfscope}%
\begin{pgfscope}%
\definecolor{textcolor}{rgb}{0.000000,0.000000,0.000000}%
\pgfsetstrokecolor{textcolor}%
\pgfsetfillcolor{textcolor}%
\pgftext[x=0.305554in, y=1.540268in, left, base]{\color{textcolor}\rmfamily\fontsize{10.000000}{12.000000}\selectfont \(\displaystyle {20000}\)}%
\end{pgfscope}%
\begin{pgfscope}%
\pgfsetbuttcap%
\pgfsetroundjoin%
\definecolor{currentfill}{rgb}{0.000000,0.000000,0.000000}%
\pgfsetfillcolor{currentfill}%
\pgfsetlinewidth{0.803000pt}%
\definecolor{currentstroke}{rgb}{0.000000,0.000000,0.000000}%
\pgfsetstrokecolor{currentstroke}%
\pgfsetdash{}{0pt}%
\pgfsys@defobject{currentmarker}{\pgfqpoint{-0.048611in}{0.000000in}}{\pgfqpoint{-0.000000in}{0.000000in}}{%
\pgfpathmoveto{\pgfqpoint{-0.000000in}{0.000000in}}%
\pgfpathlineto{\pgfqpoint{-0.048611in}{0.000000in}}%
\pgfusepath{stroke,fill}%
}%
\begin{pgfscope}%
\pgfsys@transformshift{0.750000in}{2.064104in}%
\pgfsys@useobject{currentmarker}{}%
\end{pgfscope}%
\end{pgfscope}%
\begin{pgfscope}%
\definecolor{textcolor}{rgb}{0.000000,0.000000,0.000000}%
\pgfsetstrokecolor{textcolor}%
\pgfsetfillcolor{textcolor}%
\pgftext[x=0.305554in, y=2.015879in, left, base]{\color{textcolor}\rmfamily\fontsize{10.000000}{12.000000}\selectfont \(\displaystyle {30000}\)}%
\end{pgfscope}%
\begin{pgfscope}%
\pgfsetbuttcap%
\pgfsetroundjoin%
\definecolor{currentfill}{rgb}{0.000000,0.000000,0.000000}%
\pgfsetfillcolor{currentfill}%
\pgfsetlinewidth{0.803000pt}%
\definecolor{currentstroke}{rgb}{0.000000,0.000000,0.000000}%
\pgfsetstrokecolor{currentstroke}%
\pgfsetdash{}{0pt}%
\pgfsys@defobject{currentmarker}{\pgfqpoint{-0.048611in}{0.000000in}}{\pgfqpoint{-0.000000in}{0.000000in}}{%
\pgfpathmoveto{\pgfqpoint{-0.000000in}{0.000000in}}%
\pgfpathlineto{\pgfqpoint{-0.048611in}{0.000000in}}%
\pgfusepath{stroke,fill}%
}%
\begin{pgfscope}%
\pgfsys@transformshift{0.750000in}{2.539714in}%
\pgfsys@useobject{currentmarker}{}%
\end{pgfscope}%
\end{pgfscope}%
\begin{pgfscope}%
\definecolor{textcolor}{rgb}{0.000000,0.000000,0.000000}%
\pgfsetstrokecolor{textcolor}%
\pgfsetfillcolor{textcolor}%
\pgftext[x=0.305554in, y=2.491489in, left, base]{\color{textcolor}\rmfamily\fontsize{10.000000}{12.000000}\selectfont \(\displaystyle {40000}\)}%
\end{pgfscope}%
\begin{pgfscope}%
\pgfsetbuttcap%
\pgfsetroundjoin%
\definecolor{currentfill}{rgb}{0.000000,0.000000,0.000000}%
\pgfsetfillcolor{currentfill}%
\pgfsetlinewidth{0.803000pt}%
\definecolor{currentstroke}{rgb}{0.000000,0.000000,0.000000}%
\pgfsetstrokecolor{currentstroke}%
\pgfsetdash{}{0pt}%
\pgfsys@defobject{currentmarker}{\pgfqpoint{-0.048611in}{0.000000in}}{\pgfqpoint{-0.000000in}{0.000000in}}{%
\pgfpathmoveto{\pgfqpoint{-0.000000in}{0.000000in}}%
\pgfpathlineto{\pgfqpoint{-0.048611in}{0.000000in}}%
\pgfusepath{stroke,fill}%
}%
\begin{pgfscope}%
\pgfsys@transformshift{0.750000in}{3.015325in}%
\pgfsys@useobject{currentmarker}{}%
\end{pgfscope}%
\end{pgfscope}%
\begin{pgfscope}%
\definecolor{textcolor}{rgb}{0.000000,0.000000,0.000000}%
\pgfsetstrokecolor{textcolor}%
\pgfsetfillcolor{textcolor}%
\pgftext[x=0.305554in, y=2.967100in, left, base]{\color{textcolor}\rmfamily\fontsize{10.000000}{12.000000}\selectfont \(\displaystyle {50000}\)}%
\end{pgfscope}%
\begin{pgfscope}%
\pgfsetbuttcap%
\pgfsetroundjoin%
\definecolor{currentfill}{rgb}{0.000000,0.000000,0.000000}%
\pgfsetfillcolor{currentfill}%
\pgfsetlinewidth{0.803000pt}%
\definecolor{currentstroke}{rgb}{0.000000,0.000000,0.000000}%
\pgfsetstrokecolor{currentstroke}%
\pgfsetdash{}{0pt}%
\pgfsys@defobject{currentmarker}{\pgfqpoint{-0.048611in}{0.000000in}}{\pgfqpoint{-0.000000in}{0.000000in}}{%
\pgfpathmoveto{\pgfqpoint{-0.000000in}{0.000000in}}%
\pgfpathlineto{\pgfqpoint{-0.048611in}{0.000000in}}%
\pgfusepath{stroke,fill}%
}%
\begin{pgfscope}%
\pgfsys@transformshift{0.750000in}{3.490935in}%
\pgfsys@useobject{currentmarker}{}%
\end{pgfscope}%
\end{pgfscope}%
\begin{pgfscope}%
\definecolor{textcolor}{rgb}{0.000000,0.000000,0.000000}%
\pgfsetstrokecolor{textcolor}%
\pgfsetfillcolor{textcolor}%
\pgftext[x=0.305554in, y=3.442710in, left, base]{\color{textcolor}\rmfamily\fontsize{10.000000}{12.000000}\selectfont \(\displaystyle {60000}\)}%
\end{pgfscope}%
\begin{pgfscope}%
\pgfpathrectangle{\pgfqpoint{0.750000in}{0.500000in}}{\pgfqpoint{4.650000in}{3.020000in}}%
\pgfusepath{clip}%
\pgfsetrectcap%
\pgfsetroundjoin%
\pgfsetlinewidth{1.505625pt}%
\definecolor{currentstroke}{rgb}{0.000000,0.000000,1.000000}%
\pgfsetstrokecolor{currentstroke}%
\pgfsetdash{}{0pt}%
\pgfpathmoveto{\pgfqpoint{0.961364in}{0.637273in}}%
\pgfpathlineto{\pgfqpoint{1.435542in}{0.637273in}}%
\pgfpathlineto{\pgfqpoint{1.437125in}{0.911208in}}%
\pgfpathlineto{\pgfqpoint{1.439742in}{0.912562in}}%
\pgfpathlineto{\pgfqpoint{1.441298in}{0.908808in}}%
\pgfpathlineto{\pgfqpoint{1.441620in}{0.908683in}}%
\pgfpathlineto{\pgfqpoint{1.444988in}{2.989772in}}%
\pgfpathlineto{\pgfqpoint{1.446249in}{3.096074in}}%
\pgfpathlineto{\pgfqpoint{1.446424in}{2.964231in}}%
\pgfpathlineto{\pgfqpoint{1.448181in}{2.855428in}}%
\pgfpathlineto{\pgfqpoint{1.508561in}{2.855428in}}%
\pgfpathlineto{\pgfqpoint{1.510520in}{2.645898in}}%
\pgfpathlineto{\pgfqpoint{1.510775in}{2.645861in}}%
\pgfpathlineto{\pgfqpoint{1.514116in}{0.646721in}}%
\pgfpathlineto{\pgfqpoint{1.522850in}{0.645183in}}%
\pgfpathlineto{\pgfqpoint{1.527426in}{0.641315in}}%
\pgfpathlineto{\pgfqpoint{1.534671in}{0.639773in}}%
\pgfpathlineto{\pgfqpoint{1.546412in}{0.638308in}}%
\pgfpathlineto{\pgfqpoint{3.583112in}{0.639850in}}%
\pgfpathlineto{\pgfqpoint{3.589552in}{0.639139in}}%
\pgfpathlineto{\pgfqpoint{3.620413in}{0.637865in}}%
\pgfpathlineto{\pgfqpoint{4.656494in}{0.637273in}}%
\pgfpathlineto{\pgfqpoint{4.658144in}{3.382714in}}%
\pgfpathlineto{\pgfqpoint{4.722723in}{3.381537in}}%
\pgfpathlineto{\pgfqpoint{4.724280in}{3.379132in}}%
\pgfpathlineto{\pgfqpoint{4.775857in}{3.379125in}}%
\pgfpathlineto{\pgfqpoint{4.776031in}{3.106482in}}%
\pgfpathlineto{\pgfqpoint{4.778956in}{0.639340in}}%
\pgfpathlineto{\pgfqpoint{4.782203in}{0.638979in}}%
\pgfpathlineto{\pgfqpoint{4.815949in}{0.637436in}}%
\pgfpathlineto{\pgfqpoint{4.858362in}{0.637281in}}%
\pgfpathlineto{\pgfqpoint{5.188636in}{0.637273in}}%
\pgfpathlineto{\pgfqpoint{5.188636in}{0.637273in}}%
\pgfusepath{stroke}%
\end{pgfscope}%
\begin{pgfscope}%
\pgfpathrectangle{\pgfqpoint{0.750000in}{0.500000in}}{\pgfqpoint{4.650000in}{3.020000in}}%
\pgfusepath{clip}%
\pgfsetrectcap%
\pgfsetroundjoin%
\pgfsetlinewidth{1.505625pt}%
\definecolor{currentstroke}{rgb}{1.000000,0.000000,0.000000}%
\pgfsetstrokecolor{currentstroke}%
\pgfsetdash{}{0pt}%
\pgfpathmoveto{\pgfqpoint{0.961364in}{0.637273in}}%
\pgfpathlineto{\pgfqpoint{1.435904in}{0.638780in}}%
\pgfpathlineto{\pgfqpoint{1.439782in}{0.652044in}}%
\pgfpathlineto{\pgfqpoint{1.440480in}{0.651649in}}%
\pgfpathlineto{\pgfqpoint{1.441620in}{0.651577in}}%
\pgfpathlineto{\pgfqpoint{1.441808in}{0.652641in}}%
\pgfpathlineto{\pgfqpoint{1.442895in}{0.679423in}}%
\pgfpathlineto{\pgfqpoint{1.448785in}{0.827662in}}%
\pgfpathlineto{\pgfqpoint{1.449282in}{0.827620in}}%
\pgfpathlineto{\pgfqpoint{1.457641in}{0.828529in}}%
\pgfpathlineto{\pgfqpoint{1.481994in}{0.829362in}}%
\pgfpathlineto{\pgfqpoint{1.499665in}{0.828023in}}%
\pgfpathlineto{\pgfqpoint{1.504227in}{0.815387in}}%
\pgfpathlineto{\pgfqpoint{1.505582in}{0.813928in}}%
\pgfpathlineto{\pgfqpoint{1.506843in}{0.781270in}}%
\pgfpathlineto{\pgfqpoint{1.513029in}{0.639236in}}%
\pgfpathlineto{\pgfqpoint{1.554503in}{0.637388in}}%
\pgfpathlineto{\pgfqpoint{1.777316in}{0.637273in}}%
\pgfpathlineto{\pgfqpoint{4.656615in}{0.638667in}}%
\pgfpathlineto{\pgfqpoint{4.657930in}{0.687051in}}%
\pgfpathlineto{\pgfqpoint{4.660492in}{0.784806in}}%
\pgfpathlineto{\pgfqpoint{4.661230in}{0.763086in}}%
\pgfpathlineto{\pgfqpoint{4.661552in}{0.757392in}}%
\pgfpathlineto{\pgfqpoint{4.662975in}{0.757688in}}%
\pgfpathlineto{\pgfqpoint{4.678928in}{0.759260in}}%
\pgfpathlineto{\pgfqpoint{4.698867in}{0.759935in}}%
\pgfpathlineto{\pgfqpoint{4.709762in}{0.761407in}}%
\pgfpathlineto{\pgfqpoint{4.711077in}{0.809722in}}%
\pgfpathlineto{\pgfqpoint{4.714243in}{0.879746in}}%
\pgfpathlineto{\pgfqpoint{4.720335in}{0.879079in}}%
\pgfpathlineto{\pgfqpoint{4.720362in}{0.878658in}}%
\pgfpathlineto{\pgfqpoint{4.721677in}{0.830265in}}%
\pgfpathlineto{\pgfqpoint{4.724843in}{0.760013in}}%
\pgfpathlineto{\pgfqpoint{4.741682in}{0.758743in}}%
\pgfpathlineto{\pgfqpoint{4.773549in}{0.755339in}}%
\pgfpathlineto{\pgfqpoint{4.775119in}{0.695046in}}%
\pgfpathlineto{\pgfqpoint{4.777990in}{0.637778in}}%
\pgfpathlineto{\pgfqpoint{4.820631in}{0.637298in}}%
\pgfpathlineto{\pgfqpoint{5.188636in}{0.637273in}}%
\pgfpathlineto{\pgfqpoint{5.188636in}{0.637273in}}%
\pgfusepath{stroke}%
\end{pgfscope}%
\begin{pgfscope}%
\pgfpathrectangle{\pgfqpoint{0.750000in}{0.500000in}}{\pgfqpoint{4.650000in}{3.020000in}}%
\pgfusepath{clip}%
\pgfsetrectcap%
\pgfsetroundjoin%
\pgfsetlinewidth{1.505625pt}%
\definecolor{currentstroke}{rgb}{0.000000,0.500000,0.000000}%
\pgfsetstrokecolor{currentstroke}%
\pgfsetdash{}{0pt}%
\pgfpathmoveto{\pgfqpoint{0.961364in}{0.637273in}}%
\pgfpathlineto{\pgfqpoint{1.435556in}{0.638439in}}%
\pgfpathlineto{\pgfqpoint{1.437300in}{0.681870in}}%
\pgfpathlineto{\pgfqpoint{1.438923in}{0.698448in}}%
\pgfpathlineto{\pgfqpoint{1.440104in}{0.697684in}}%
\pgfpathlineto{\pgfqpoint{1.441754in}{0.698904in}}%
\pgfpathlineto{\pgfqpoint{1.442170in}{0.740756in}}%
\pgfpathlineto{\pgfqpoint{1.446303in}{1.133484in}}%
\pgfpathlineto{\pgfqpoint{1.449859in}{1.184509in}}%
\pgfpathlineto{\pgfqpoint{1.449886in}{1.184508in}}%
\pgfpathlineto{\pgfqpoint{1.499665in}{1.184020in}}%
\pgfpathlineto{\pgfqpoint{1.505555in}{1.185168in}}%
\pgfpathlineto{\pgfqpoint{1.505622in}{1.184899in}}%
\pgfpathlineto{\pgfqpoint{1.506172in}{1.172972in}}%
\pgfpathlineto{\pgfqpoint{1.509097in}{0.996424in}}%
\pgfpathlineto{\pgfqpoint{1.511432in}{0.680005in}}%
\pgfpathlineto{\pgfqpoint{1.513190in}{0.639392in}}%
\pgfpathlineto{\pgfqpoint{1.677220in}{0.637273in}}%
\pgfpathlineto{\pgfqpoint{4.656494in}{0.637273in}}%
\pgfpathlineto{\pgfqpoint{4.659486in}{1.175948in}}%
\pgfpathlineto{\pgfqpoint{4.660251in}{1.228988in}}%
\pgfpathlineto{\pgfqpoint{4.661069in}{1.180007in}}%
\pgfpathlineto{\pgfqpoint{4.662854in}{1.166981in}}%
\pgfpathlineto{\pgfqpoint{4.709453in}{1.166411in}}%
\pgfpathlineto{\pgfqpoint{4.709802in}{1.167734in}}%
\pgfpathlineto{\pgfqpoint{4.710513in}{1.208294in}}%
\pgfpathlineto{\pgfqpoint{4.712969in}{1.365780in}}%
\pgfpathlineto{\pgfqpoint{4.713760in}{1.365592in}}%
\pgfpathlineto{\pgfqpoint{4.720496in}{1.363548in}}%
\pgfpathlineto{\pgfqpoint{4.721301in}{1.322938in}}%
\pgfpathlineto{\pgfqpoint{4.723582in}{1.165572in}}%
\pgfpathlineto{\pgfqpoint{4.724991in}{1.165627in}}%
\pgfpathlineto{\pgfqpoint{4.773469in}{1.165884in}}%
\pgfpathlineto{\pgfqpoint{4.773952in}{1.147630in}}%
\pgfpathlineto{\pgfqpoint{4.775589in}{0.925642in}}%
\pgfpathlineto{\pgfqpoint{4.778567in}{0.637822in}}%
\pgfpathlineto{\pgfqpoint{4.819276in}{0.637303in}}%
\pgfpathlineto{\pgfqpoint{5.188636in}{0.637273in}}%
\pgfpathlineto{\pgfqpoint{5.188636in}{0.637273in}}%
\pgfusepath{stroke}%
\end{pgfscope}%
\begin{pgfscope}%
\pgfsetrectcap%
\pgfsetmiterjoin%
\pgfsetlinewidth{0.803000pt}%
\definecolor{currentstroke}{rgb}{0.000000,0.000000,0.000000}%
\pgfsetstrokecolor{currentstroke}%
\pgfsetdash{}{0pt}%
\pgfpathmoveto{\pgfqpoint{0.750000in}{0.500000in}}%
\pgfpathlineto{\pgfqpoint{0.750000in}{3.520000in}}%
\pgfusepath{stroke}%
\end{pgfscope}%
\begin{pgfscope}%
\pgfsetrectcap%
\pgfsetmiterjoin%
\pgfsetlinewidth{0.803000pt}%
\definecolor{currentstroke}{rgb}{0.000000,0.000000,0.000000}%
\pgfsetstrokecolor{currentstroke}%
\pgfsetdash{}{0pt}%
\pgfpathmoveto{\pgfqpoint{5.400000in}{0.500000in}}%
\pgfpathlineto{\pgfqpoint{5.400000in}{3.520000in}}%
\pgfusepath{stroke}%
\end{pgfscope}%
\begin{pgfscope}%
\pgfsetrectcap%
\pgfsetmiterjoin%
\pgfsetlinewidth{0.803000pt}%
\definecolor{currentstroke}{rgb}{0.000000,0.000000,0.000000}%
\pgfsetstrokecolor{currentstroke}%
\pgfsetdash{}{0pt}%
\pgfpathmoveto{\pgfqpoint{0.750000in}{0.500000in}}%
\pgfpathlineto{\pgfqpoint{5.400000in}{0.500000in}}%
\pgfusepath{stroke}%
\end{pgfscope}%
\begin{pgfscope}%
\pgfsetrectcap%
\pgfsetmiterjoin%
\pgfsetlinewidth{0.803000pt}%
\definecolor{currentstroke}{rgb}{0.000000,0.000000,0.000000}%
\pgfsetstrokecolor{currentstroke}%
\pgfsetdash{}{0pt}%
\pgfpathmoveto{\pgfqpoint{0.750000in}{3.520000in}}%
\pgfpathlineto{\pgfqpoint{5.400000in}{3.520000in}}%
\pgfusepath{stroke}%
\end{pgfscope}%
\begin{pgfscope}%
\pgfsetbuttcap%
\pgfsetmiterjoin%
\definecolor{currentfill}{rgb}{1.000000,1.000000,1.000000}%
\pgfsetfillcolor{currentfill}%
\pgfsetfillopacity{0.800000}%
\pgfsetlinewidth{1.003750pt}%
\definecolor{currentstroke}{rgb}{0.800000,0.800000,0.800000}%
\pgfsetstrokecolor{currentstroke}%
\pgfsetstrokeopacity{0.800000}%
\pgfsetdash{}{0pt}%
\pgfpathmoveto{\pgfqpoint{2.723533in}{2.827871in}}%
\pgfpathlineto{\pgfqpoint{3.426467in}{2.827871in}}%
\pgfpathquadraticcurveto{\pgfqpoint{3.454244in}{2.827871in}}{\pgfqpoint{3.454244in}{2.855648in}}%
\pgfpathlineto{\pgfqpoint{3.454244in}{3.422778in}}%
\pgfpathquadraticcurveto{\pgfqpoint{3.454244in}{3.450556in}}{\pgfqpoint{3.426467in}{3.450556in}}%
\pgfpathlineto{\pgfqpoint{2.723533in}{3.450556in}}%
\pgfpathquadraticcurveto{\pgfqpoint{2.695756in}{3.450556in}}{\pgfqpoint{2.695756in}{3.422778in}}%
\pgfpathlineto{\pgfqpoint{2.695756in}{2.855648in}}%
\pgfpathquadraticcurveto{\pgfqpoint{2.695756in}{2.827871in}}{\pgfqpoint{2.723533in}{2.827871in}}%
\pgfpathlineto{\pgfqpoint{2.723533in}{2.827871in}}%
\pgfpathclose%
\pgfusepath{stroke,fill}%
\end{pgfscope}%
\begin{pgfscope}%
\pgfsetrectcap%
\pgfsetroundjoin%
\pgfsetlinewidth{1.505625pt}%
\definecolor{currentstroke}{rgb}{0.000000,0.000000,1.000000}%
\pgfsetstrokecolor{currentstroke}%
\pgfsetdash{}{0pt}%
\pgfpathmoveto{\pgfqpoint{2.751311in}{3.346389in}}%
\pgfpathlineto{\pgfqpoint{2.890200in}{3.346389in}}%
\pgfpathlineto{\pgfqpoint{3.029089in}{3.346389in}}%
\pgfusepath{stroke}%
\end{pgfscope}%
\begin{pgfscope}%
\definecolor{textcolor}{rgb}{0.000000,0.000000,0.000000}%
\pgfsetstrokecolor{textcolor}%
\pgfsetfillcolor{textcolor}%
\pgftext[x=3.140200in,y=3.297778in,left,base]{\color{textcolor}\rmfamily\fontsize{10.000000}{12.000000}\selectfont max}%
\end{pgfscope}%
\begin{pgfscope}%
\pgfsetrectcap%
\pgfsetroundjoin%
\pgfsetlinewidth{1.505625pt}%
\definecolor{currentstroke}{rgb}{1.000000,0.000000,0.000000}%
\pgfsetstrokecolor{currentstroke}%
\pgfsetdash{}{0pt}%
\pgfpathmoveto{\pgfqpoint{2.751311in}{3.152716in}}%
\pgfpathlineto{\pgfqpoint{2.890200in}{3.152716in}}%
\pgfpathlineto{\pgfqpoint{3.029089in}{3.152716in}}%
\pgfusepath{stroke}%
\end{pgfscope}%
\begin{pgfscope}%
\definecolor{textcolor}{rgb}{0.000000,0.000000,0.000000}%
\pgfsetstrokecolor{textcolor}%
\pgfsetfillcolor{textcolor}%
\pgftext[x=3.140200in,y=3.104105in,left,base]{\color{textcolor}\rmfamily\fontsize{10.000000}{12.000000}\selectfont \(\displaystyle \mu\)}%
\end{pgfscope}%
\begin{pgfscope}%
\pgfsetrectcap%
\pgfsetroundjoin%
\pgfsetlinewidth{1.505625pt}%
\definecolor{currentstroke}{rgb}{0.000000,0.500000,0.000000}%
\pgfsetstrokecolor{currentstroke}%
\pgfsetdash{}{0pt}%
\pgfpathmoveto{\pgfqpoint{2.751311in}{2.959043in}}%
\pgfpathlineto{\pgfqpoint{2.890200in}{2.959043in}}%
\pgfpathlineto{\pgfqpoint{3.029089in}{2.959043in}}%
\pgfusepath{stroke}%
\end{pgfscope}%
\begin{pgfscope}%
\definecolor{textcolor}{rgb}{0.000000,0.000000,0.000000}%
\pgfsetstrokecolor{textcolor}%
\pgfsetfillcolor{textcolor}%
\pgftext[x=3.140200in,y=2.910432in,left,base]{\color{textcolor}\rmfamily\fontsize{10.000000}{12.000000}\selectfont \(\displaystyle \sigma\)}%
\end{pgfscope}%
\end{pgfpicture}%
\makeatother%
\endgroup%

    \caption{PEC Data Set Matrix Profile Values}
    \label{fig:pec_mp_hist}
\end{figure}

Figure \ref{fig:pec_outliers} shows the anomalies detected by the algorithm in red, over the manually labeled anomaly (true/false values) start and end periods. The detector accurately determined the start and end of each anomaly (100\% detection rate) with no false positives (0\% error rate). It detected the start of each anomaly almost immediately quickly detected the end of the corresponding anomaly once it was over, although this takes more time in certain cases. By nature, anomalies are rare so interpreting the detection and false positive rates in a traditional way is not possible. The interpretation of the results and significance will be provided in section \ref{sec:discussion}.
 
\begin{figure}[H]
    %%\centering
    %% Creator: Matplotlib, PGF backend
%%
%% To include the figure in your LaTeX document, write
%%   \input{<filename>.pgf}
%%
%% Make sure the required packages are loaded in your preamble
%%   \usepackage{pgf}
%%
%% Also ensure that all the required font packages are loaded; for instance,
%% the lmodern package is sometimes necessary when using math font.
%%   \usepackage{lmodern}
%%
%% Figures using additional raster images can only be included by \input if
%% they are in the same directory as the main LaTeX file. For loading figures
%% from other directories you can use the `import` package
%%   \usepackage{import}
%%
%% and then include the figures with
%%   \import{<path to file>}{<filename>.pgf}
%%
%% Matplotlib used the following preamble
%%
\begingroup%
\makeatletter%
\begin{pgfpicture}%
\pgfpathrectangle{\pgfpointorigin}{\pgfqpoint{6.000000in}{4.000000in}}%
\pgfusepath{use as bounding box, clip}%
\begin{pgfscope}%
\pgfsetbuttcap%
\pgfsetmiterjoin%
\pgfsetlinewidth{0.000000pt}%
\definecolor{currentstroke}{rgb}{1.000000,1.000000,1.000000}%
\pgfsetstrokecolor{currentstroke}%
\pgfsetstrokeopacity{0.000000}%
\pgfsetdash{}{0pt}%
\pgfpathmoveto{\pgfqpoint{0.000000in}{0.000000in}}%
\pgfpathlineto{\pgfqpoint{6.000000in}{0.000000in}}%
\pgfpathlineto{\pgfqpoint{6.000000in}{4.000000in}}%
\pgfpathlineto{\pgfqpoint{0.000000in}{4.000000in}}%
\pgfpathlineto{\pgfqpoint{0.000000in}{0.000000in}}%
\pgfpathclose%
\pgfusepath{}%
\end{pgfscope}%
\begin{pgfscope}%
\pgfsetbuttcap%
\pgfsetmiterjoin%
\definecolor{currentfill}{rgb}{1.000000,1.000000,1.000000}%
\pgfsetfillcolor{currentfill}%
\pgfsetlinewidth{0.000000pt}%
\definecolor{currentstroke}{rgb}{0.000000,0.000000,0.000000}%
\pgfsetstrokecolor{currentstroke}%
\pgfsetstrokeopacity{0.000000}%
\pgfsetdash{}{0pt}%
\pgfpathmoveto{\pgfqpoint{0.750000in}{0.500000in}}%
\pgfpathlineto{\pgfqpoint{5.400000in}{0.500000in}}%
\pgfpathlineto{\pgfqpoint{5.400000in}{3.520000in}}%
\pgfpathlineto{\pgfqpoint{0.750000in}{3.520000in}}%
\pgfpathlineto{\pgfqpoint{0.750000in}{0.500000in}}%
\pgfpathclose%
\pgfusepath{fill}%
\end{pgfscope}%
\begin{pgfscope}%
\pgfsetbuttcap%
\pgfsetroundjoin%
\definecolor{currentfill}{rgb}{0.000000,0.000000,0.000000}%
\pgfsetfillcolor{currentfill}%
\pgfsetlinewidth{0.803000pt}%
\definecolor{currentstroke}{rgb}{0.000000,0.000000,0.000000}%
\pgfsetstrokecolor{currentstroke}%
\pgfsetdash{}{0pt}%
\pgfsys@defobject{currentmarker}{\pgfqpoint{0.000000in}{-0.048611in}}{\pgfqpoint{0.000000in}{0.000000in}}{%
\pgfpathmoveto{\pgfqpoint{0.000000in}{0.000000in}}%
\pgfpathlineto{\pgfqpoint{0.000000in}{-0.048611in}}%
\pgfusepath{stroke,fill}%
}%
\begin{pgfscope}%
\pgfsys@transformshift{0.961364in}{0.500000in}%
\pgfsys@useobject{currentmarker}{}%
\end{pgfscope}%
\end{pgfscope}%
\begin{pgfscope}%
\definecolor{textcolor}{rgb}{0.000000,0.000000,0.000000}%
\pgfsetstrokecolor{textcolor}%
\pgfsetfillcolor{textcolor}%
\pgftext[x=0.961364in,y=0.402778in,,top]{\color{textcolor}\rmfamily\fontsize{10.000000}{12.000000}\selectfont \(\displaystyle {0}\)}%
\end{pgfscope}%
\begin{pgfscope}%
\pgfsetbuttcap%
\pgfsetroundjoin%
\definecolor{currentfill}{rgb}{0.000000,0.000000,0.000000}%
\pgfsetfillcolor{currentfill}%
\pgfsetlinewidth{0.803000pt}%
\definecolor{currentstroke}{rgb}{0.000000,0.000000,0.000000}%
\pgfsetstrokecolor{currentstroke}%
\pgfsetdash{}{0pt}%
\pgfsys@defobject{currentmarker}{\pgfqpoint{0.000000in}{-0.048611in}}{\pgfqpoint{0.000000in}{0.000000in}}{%
\pgfpathmoveto{\pgfqpoint{0.000000in}{0.000000in}}%
\pgfpathlineto{\pgfqpoint{0.000000in}{-0.048611in}}%
\pgfusepath{stroke,fill}%
}%
\begin{pgfscope}%
\pgfsys@transformshift{1.621764in}{0.500000in}%
\pgfsys@useobject{currentmarker}{}%
\end{pgfscope}%
\end{pgfscope}%
\begin{pgfscope}%
\definecolor{textcolor}{rgb}{0.000000,0.000000,0.000000}%
\pgfsetstrokecolor{textcolor}%
\pgfsetfillcolor{textcolor}%
\pgftext[x=1.621764in,y=0.402778in,,top]{\color{textcolor}\rmfamily\fontsize{10.000000}{12.000000}\selectfont \(\displaystyle {50000}\)}%
\end{pgfscope}%
\begin{pgfscope}%
\pgfsetbuttcap%
\pgfsetroundjoin%
\definecolor{currentfill}{rgb}{0.000000,0.000000,0.000000}%
\pgfsetfillcolor{currentfill}%
\pgfsetlinewidth{0.803000pt}%
\definecolor{currentstroke}{rgb}{0.000000,0.000000,0.000000}%
\pgfsetstrokecolor{currentstroke}%
\pgfsetdash{}{0pt}%
\pgfsys@defobject{currentmarker}{\pgfqpoint{0.000000in}{-0.048611in}}{\pgfqpoint{0.000000in}{0.000000in}}{%
\pgfpathmoveto{\pgfqpoint{0.000000in}{0.000000in}}%
\pgfpathlineto{\pgfqpoint{0.000000in}{-0.048611in}}%
\pgfusepath{stroke,fill}%
}%
\begin{pgfscope}%
\pgfsys@transformshift{2.282163in}{0.500000in}%
\pgfsys@useobject{currentmarker}{}%
\end{pgfscope}%
\end{pgfscope}%
\begin{pgfscope}%
\definecolor{textcolor}{rgb}{0.000000,0.000000,0.000000}%
\pgfsetstrokecolor{textcolor}%
\pgfsetfillcolor{textcolor}%
\pgftext[x=2.282163in,y=0.402778in,,top]{\color{textcolor}\rmfamily\fontsize{10.000000}{12.000000}\selectfont \(\displaystyle {100000}\)}%
\end{pgfscope}%
\begin{pgfscope}%
\pgfsetbuttcap%
\pgfsetroundjoin%
\definecolor{currentfill}{rgb}{0.000000,0.000000,0.000000}%
\pgfsetfillcolor{currentfill}%
\pgfsetlinewidth{0.803000pt}%
\definecolor{currentstroke}{rgb}{0.000000,0.000000,0.000000}%
\pgfsetstrokecolor{currentstroke}%
\pgfsetdash{}{0pt}%
\pgfsys@defobject{currentmarker}{\pgfqpoint{0.000000in}{-0.048611in}}{\pgfqpoint{0.000000in}{0.000000in}}{%
\pgfpathmoveto{\pgfqpoint{0.000000in}{0.000000in}}%
\pgfpathlineto{\pgfqpoint{0.000000in}{-0.048611in}}%
\pgfusepath{stroke,fill}%
}%
\begin{pgfscope}%
\pgfsys@transformshift{2.942563in}{0.500000in}%
\pgfsys@useobject{currentmarker}{}%
\end{pgfscope}%
\end{pgfscope}%
\begin{pgfscope}%
\definecolor{textcolor}{rgb}{0.000000,0.000000,0.000000}%
\pgfsetstrokecolor{textcolor}%
\pgfsetfillcolor{textcolor}%
\pgftext[x=2.942563in,y=0.402778in,,top]{\color{textcolor}\rmfamily\fontsize{10.000000}{12.000000}\selectfont \(\displaystyle {150000}\)}%
\end{pgfscope}%
\begin{pgfscope}%
\pgfsetbuttcap%
\pgfsetroundjoin%
\definecolor{currentfill}{rgb}{0.000000,0.000000,0.000000}%
\pgfsetfillcolor{currentfill}%
\pgfsetlinewidth{0.803000pt}%
\definecolor{currentstroke}{rgb}{0.000000,0.000000,0.000000}%
\pgfsetstrokecolor{currentstroke}%
\pgfsetdash{}{0pt}%
\pgfsys@defobject{currentmarker}{\pgfqpoint{0.000000in}{-0.048611in}}{\pgfqpoint{0.000000in}{0.000000in}}{%
\pgfpathmoveto{\pgfqpoint{0.000000in}{0.000000in}}%
\pgfpathlineto{\pgfqpoint{0.000000in}{-0.048611in}}%
\pgfusepath{stroke,fill}%
}%
\begin{pgfscope}%
\pgfsys@transformshift{3.602963in}{0.500000in}%
\pgfsys@useobject{currentmarker}{}%
\end{pgfscope}%
\end{pgfscope}%
\begin{pgfscope}%
\definecolor{textcolor}{rgb}{0.000000,0.000000,0.000000}%
\pgfsetstrokecolor{textcolor}%
\pgfsetfillcolor{textcolor}%
\pgftext[x=3.602963in,y=0.402778in,,top]{\color{textcolor}\rmfamily\fontsize{10.000000}{12.000000}\selectfont \(\displaystyle {200000}\)}%
\end{pgfscope}%
\begin{pgfscope}%
\pgfsetbuttcap%
\pgfsetroundjoin%
\definecolor{currentfill}{rgb}{0.000000,0.000000,0.000000}%
\pgfsetfillcolor{currentfill}%
\pgfsetlinewidth{0.803000pt}%
\definecolor{currentstroke}{rgb}{0.000000,0.000000,0.000000}%
\pgfsetstrokecolor{currentstroke}%
\pgfsetdash{}{0pt}%
\pgfsys@defobject{currentmarker}{\pgfqpoint{0.000000in}{-0.048611in}}{\pgfqpoint{0.000000in}{0.000000in}}{%
\pgfpathmoveto{\pgfqpoint{0.000000in}{0.000000in}}%
\pgfpathlineto{\pgfqpoint{0.000000in}{-0.048611in}}%
\pgfusepath{stroke,fill}%
}%
\begin{pgfscope}%
\pgfsys@transformshift{4.263363in}{0.500000in}%
\pgfsys@useobject{currentmarker}{}%
\end{pgfscope}%
\end{pgfscope}%
\begin{pgfscope}%
\definecolor{textcolor}{rgb}{0.000000,0.000000,0.000000}%
\pgfsetstrokecolor{textcolor}%
\pgfsetfillcolor{textcolor}%
\pgftext[x=4.263363in,y=0.402778in,,top]{\color{textcolor}\rmfamily\fontsize{10.000000}{12.000000}\selectfont \(\displaystyle {250000}\)}%
\end{pgfscope}%
\begin{pgfscope}%
\pgfsetbuttcap%
\pgfsetroundjoin%
\definecolor{currentfill}{rgb}{0.000000,0.000000,0.000000}%
\pgfsetfillcolor{currentfill}%
\pgfsetlinewidth{0.803000pt}%
\definecolor{currentstroke}{rgb}{0.000000,0.000000,0.000000}%
\pgfsetstrokecolor{currentstroke}%
\pgfsetdash{}{0pt}%
\pgfsys@defobject{currentmarker}{\pgfqpoint{0.000000in}{-0.048611in}}{\pgfqpoint{0.000000in}{0.000000in}}{%
\pgfpathmoveto{\pgfqpoint{0.000000in}{0.000000in}}%
\pgfpathlineto{\pgfqpoint{0.000000in}{-0.048611in}}%
\pgfusepath{stroke,fill}%
}%
\begin{pgfscope}%
\pgfsys@transformshift{4.923763in}{0.500000in}%
\pgfsys@useobject{currentmarker}{}%
\end{pgfscope}%
\end{pgfscope}%
\begin{pgfscope}%
\definecolor{textcolor}{rgb}{0.000000,0.000000,0.000000}%
\pgfsetstrokecolor{textcolor}%
\pgfsetfillcolor{textcolor}%
\pgftext[x=4.923763in,y=0.402778in,,top]{\color{textcolor}\rmfamily\fontsize{10.000000}{12.000000}\selectfont \(\displaystyle {300000}\)}%
\end{pgfscope}%
\begin{pgfscope}%
\definecolor{textcolor}{rgb}{0.000000,0.000000,0.000000}%
\pgfsetstrokecolor{textcolor}%
\pgfsetfillcolor{textcolor}%
\pgftext[x=3.075000in,y=0.223766in,,top]{\color{textcolor}\rmfamily\fontsize{10.000000}{12.000000}\selectfont time}%
\end{pgfscope}%
\begin{pgfscope}%
\pgfsetbuttcap%
\pgfsetroundjoin%
\definecolor{currentfill}{rgb}{0.000000,0.000000,0.000000}%
\pgfsetfillcolor{currentfill}%
\pgfsetlinewidth{0.803000pt}%
\definecolor{currentstroke}{rgb}{0.000000,0.000000,0.000000}%
\pgfsetstrokecolor{currentstroke}%
\pgfsetdash{}{0pt}%
\pgfsys@defobject{currentmarker}{\pgfqpoint{-0.048611in}{0.000000in}}{\pgfqpoint{-0.000000in}{0.000000in}}{%
\pgfpathmoveto{\pgfqpoint{-0.000000in}{0.000000in}}%
\pgfpathlineto{\pgfqpoint{-0.048611in}{0.000000in}}%
\pgfusepath{stroke,fill}%
}%
\begin{pgfscope}%
\pgfsys@transformshift{0.750000in}{0.637273in}%
\pgfsys@useobject{currentmarker}{}%
\end{pgfscope}%
\end{pgfscope}%
\begin{pgfscope}%
\definecolor{textcolor}{rgb}{0.000000,0.000000,0.000000}%
\pgfsetstrokecolor{textcolor}%
\pgfsetfillcolor{textcolor}%
\pgftext[x=0.475308in, y=0.589047in, left, base]{\color{textcolor}\rmfamily\fontsize{10.000000}{12.000000}\selectfont \(\displaystyle {0.0}\)}%
\end{pgfscope}%
\begin{pgfscope}%
\pgfsetbuttcap%
\pgfsetroundjoin%
\definecolor{currentfill}{rgb}{0.000000,0.000000,0.000000}%
\pgfsetfillcolor{currentfill}%
\pgfsetlinewidth{0.803000pt}%
\definecolor{currentstroke}{rgb}{0.000000,0.000000,0.000000}%
\pgfsetstrokecolor{currentstroke}%
\pgfsetdash{}{0pt}%
\pgfsys@defobject{currentmarker}{\pgfqpoint{-0.048611in}{0.000000in}}{\pgfqpoint{-0.000000in}{0.000000in}}{%
\pgfpathmoveto{\pgfqpoint{-0.000000in}{0.000000in}}%
\pgfpathlineto{\pgfqpoint{-0.048611in}{0.000000in}}%
\pgfusepath{stroke,fill}%
}%
\begin{pgfscope}%
\pgfsys@transformshift{0.750000in}{1.186364in}%
\pgfsys@useobject{currentmarker}{}%
\end{pgfscope}%
\end{pgfscope}%
\begin{pgfscope}%
\definecolor{textcolor}{rgb}{0.000000,0.000000,0.000000}%
\pgfsetstrokecolor{textcolor}%
\pgfsetfillcolor{textcolor}%
\pgftext[x=0.475308in, y=1.138138in, left, base]{\color{textcolor}\rmfamily\fontsize{10.000000}{12.000000}\selectfont \(\displaystyle {0.2}\)}%
\end{pgfscope}%
\begin{pgfscope}%
\pgfsetbuttcap%
\pgfsetroundjoin%
\definecolor{currentfill}{rgb}{0.000000,0.000000,0.000000}%
\pgfsetfillcolor{currentfill}%
\pgfsetlinewidth{0.803000pt}%
\definecolor{currentstroke}{rgb}{0.000000,0.000000,0.000000}%
\pgfsetstrokecolor{currentstroke}%
\pgfsetdash{}{0pt}%
\pgfsys@defobject{currentmarker}{\pgfqpoint{-0.048611in}{0.000000in}}{\pgfqpoint{-0.000000in}{0.000000in}}{%
\pgfpathmoveto{\pgfqpoint{-0.000000in}{0.000000in}}%
\pgfpathlineto{\pgfqpoint{-0.048611in}{0.000000in}}%
\pgfusepath{stroke,fill}%
}%
\begin{pgfscope}%
\pgfsys@transformshift{0.750000in}{1.735455in}%
\pgfsys@useobject{currentmarker}{}%
\end{pgfscope}%
\end{pgfscope}%
\begin{pgfscope}%
\definecolor{textcolor}{rgb}{0.000000,0.000000,0.000000}%
\pgfsetstrokecolor{textcolor}%
\pgfsetfillcolor{textcolor}%
\pgftext[x=0.475308in, y=1.687229in, left, base]{\color{textcolor}\rmfamily\fontsize{10.000000}{12.000000}\selectfont \(\displaystyle {0.4}\)}%
\end{pgfscope}%
\begin{pgfscope}%
\pgfsetbuttcap%
\pgfsetroundjoin%
\definecolor{currentfill}{rgb}{0.000000,0.000000,0.000000}%
\pgfsetfillcolor{currentfill}%
\pgfsetlinewidth{0.803000pt}%
\definecolor{currentstroke}{rgb}{0.000000,0.000000,0.000000}%
\pgfsetstrokecolor{currentstroke}%
\pgfsetdash{}{0pt}%
\pgfsys@defobject{currentmarker}{\pgfqpoint{-0.048611in}{0.000000in}}{\pgfqpoint{-0.000000in}{0.000000in}}{%
\pgfpathmoveto{\pgfqpoint{-0.000000in}{0.000000in}}%
\pgfpathlineto{\pgfqpoint{-0.048611in}{0.000000in}}%
\pgfusepath{stroke,fill}%
}%
\begin{pgfscope}%
\pgfsys@transformshift{0.750000in}{2.284545in}%
\pgfsys@useobject{currentmarker}{}%
\end{pgfscope}%
\end{pgfscope}%
\begin{pgfscope}%
\definecolor{textcolor}{rgb}{0.000000,0.000000,0.000000}%
\pgfsetstrokecolor{textcolor}%
\pgfsetfillcolor{textcolor}%
\pgftext[x=0.475308in, y=2.236320in, left, base]{\color{textcolor}\rmfamily\fontsize{10.000000}{12.000000}\selectfont \(\displaystyle {0.6}\)}%
\end{pgfscope}%
\begin{pgfscope}%
\pgfsetbuttcap%
\pgfsetroundjoin%
\definecolor{currentfill}{rgb}{0.000000,0.000000,0.000000}%
\pgfsetfillcolor{currentfill}%
\pgfsetlinewidth{0.803000pt}%
\definecolor{currentstroke}{rgb}{0.000000,0.000000,0.000000}%
\pgfsetstrokecolor{currentstroke}%
\pgfsetdash{}{0pt}%
\pgfsys@defobject{currentmarker}{\pgfqpoint{-0.048611in}{0.000000in}}{\pgfqpoint{-0.000000in}{0.000000in}}{%
\pgfpathmoveto{\pgfqpoint{-0.000000in}{0.000000in}}%
\pgfpathlineto{\pgfqpoint{-0.048611in}{0.000000in}}%
\pgfusepath{stroke,fill}%
}%
\begin{pgfscope}%
\pgfsys@transformshift{0.750000in}{2.833636in}%
\pgfsys@useobject{currentmarker}{}%
\end{pgfscope}%
\end{pgfscope}%
\begin{pgfscope}%
\definecolor{textcolor}{rgb}{0.000000,0.000000,0.000000}%
\pgfsetstrokecolor{textcolor}%
\pgfsetfillcolor{textcolor}%
\pgftext[x=0.475308in, y=2.785411in, left, base]{\color{textcolor}\rmfamily\fontsize{10.000000}{12.000000}\selectfont \(\displaystyle {0.8}\)}%
\end{pgfscope}%
\begin{pgfscope}%
\pgfsetbuttcap%
\pgfsetroundjoin%
\definecolor{currentfill}{rgb}{0.000000,0.000000,0.000000}%
\pgfsetfillcolor{currentfill}%
\pgfsetlinewidth{0.803000pt}%
\definecolor{currentstroke}{rgb}{0.000000,0.000000,0.000000}%
\pgfsetstrokecolor{currentstroke}%
\pgfsetdash{}{0pt}%
\pgfsys@defobject{currentmarker}{\pgfqpoint{-0.048611in}{0.000000in}}{\pgfqpoint{-0.000000in}{0.000000in}}{%
\pgfpathmoveto{\pgfqpoint{-0.000000in}{0.000000in}}%
\pgfpathlineto{\pgfqpoint{-0.048611in}{0.000000in}}%
\pgfusepath{stroke,fill}%
}%
\begin{pgfscope}%
\pgfsys@transformshift{0.750000in}{3.382727in}%
\pgfsys@useobject{currentmarker}{}%
\end{pgfscope}%
\end{pgfscope}%
\begin{pgfscope}%
\definecolor{textcolor}{rgb}{0.000000,0.000000,0.000000}%
\pgfsetstrokecolor{textcolor}%
\pgfsetfillcolor{textcolor}%
\pgftext[x=0.475308in, y=3.334502in, left, base]{\color{textcolor}\rmfamily\fontsize{10.000000}{12.000000}\selectfont \(\displaystyle {1.0}\)}%
\end{pgfscope}%
\begin{pgfscope}%
\pgfpathrectangle{\pgfqpoint{0.750000in}{0.500000in}}{\pgfqpoint{4.650000in}{3.020000in}}%
\pgfusepath{clip}%
\pgfsetrectcap%
\pgfsetroundjoin%
\pgfsetlinewidth{1.505625pt}%
\definecolor{currentstroke}{rgb}{0.121569,0.466667,0.705882}%
\pgfsetstrokecolor{currentstroke}%
\pgfsetdash{}{0pt}%
\pgfpathmoveto{\pgfqpoint{0.961364in}{0.637273in}}%
\pgfpathlineto{\pgfqpoint{1.494095in}{0.637273in}}%
\pgfpathlineto{\pgfqpoint{1.495640in}{3.382727in}}%
\pgfpathlineto{\pgfqpoint{1.535912in}{3.382727in}}%
\pgfpathlineto{\pgfqpoint{1.537457in}{0.637273in}}%
\pgfpathlineto{\pgfqpoint{2.546825in}{0.637273in}}%
\pgfpathlineto{\pgfqpoint{2.548371in}{3.382727in}}%
\pgfpathlineto{\pgfqpoint{3.075330in}{3.382727in}}%
\pgfpathlineto{\pgfqpoint{3.076876in}{0.637273in}}%
\pgfpathlineto{\pgfqpoint{3.603650in}{0.637273in}}%
\pgfpathlineto{\pgfqpoint{3.605195in}{3.382727in}}%
\pgfpathlineto{\pgfqpoint{4.131983in}{3.382727in}}%
\pgfpathlineto{\pgfqpoint{4.133529in}{0.637273in}}%
\pgfpathlineto{\pgfqpoint{4.664701in}{0.637273in}}%
\pgfpathlineto{\pgfqpoint{4.666247in}{3.382727in}}%
\pgfpathlineto{\pgfqpoint{4.682334in}{3.382727in}}%
\pgfpathlineto{\pgfqpoint{4.683879in}{0.637273in}}%
\pgfpathlineto{\pgfqpoint{5.188636in}{0.637273in}}%
\pgfpathlineto{\pgfqpoint{5.188636in}{0.637273in}}%
\pgfusepath{stroke}%
\end{pgfscope}%
\begin{pgfscope}%
\pgfpathrectangle{\pgfqpoint{0.750000in}{0.500000in}}{\pgfqpoint{4.650000in}{3.020000in}}%
\pgfusepath{clip}%
\pgfsetrectcap%
\pgfsetroundjoin%
\pgfsetlinewidth{1.505625pt}%
\definecolor{currentstroke}{rgb}{1.000000,0.000000,0.000000}%
\pgfsetstrokecolor{currentstroke}%
\pgfsetdash{}{0pt}%
\pgfpathmoveto{\pgfqpoint{1.494188in}{0.500000in}}%
\pgfpathlineto{\pgfqpoint{1.494188in}{3.520000in}}%
\pgfusepath{stroke}%
\end{pgfscope}%
\begin{pgfscope}%
\pgfpathrectangle{\pgfqpoint{0.750000in}{0.500000in}}{\pgfqpoint{4.650000in}{3.020000in}}%
\pgfusepath{clip}%
\pgfsetrectcap%
\pgfsetroundjoin%
\pgfsetlinewidth{1.505625pt}%
\definecolor{currentstroke}{rgb}{1.000000,0.000000,0.000000}%
\pgfsetstrokecolor{currentstroke}%
\pgfsetdash{}{0pt}%
\pgfpathmoveto{\pgfqpoint{1.566065in}{0.500000in}}%
\pgfpathlineto{\pgfqpoint{1.566065in}{3.520000in}}%
\pgfusepath{stroke}%
\end{pgfscope}%
\begin{pgfscope}%
\pgfpathrectangle{\pgfqpoint{0.750000in}{0.500000in}}{\pgfqpoint{4.650000in}{3.020000in}}%
\pgfusepath{clip}%
\pgfsetrectcap%
\pgfsetroundjoin%
\pgfsetlinewidth{1.505625pt}%
\definecolor{currentstroke}{rgb}{1.000000,0.000000,0.000000}%
\pgfsetstrokecolor{currentstroke}%
\pgfsetdash{}{0pt}%
\pgfpathmoveto{\pgfqpoint{2.546825in}{0.500000in}}%
\pgfpathlineto{\pgfqpoint{2.546825in}{3.520000in}}%
\pgfusepath{stroke}%
\end{pgfscope}%
\begin{pgfscope}%
\pgfpathrectangle{\pgfqpoint{0.750000in}{0.500000in}}{\pgfqpoint{4.650000in}{3.020000in}}%
\pgfusepath{clip}%
\pgfsetrectcap%
\pgfsetroundjoin%
\pgfsetlinewidth{1.505625pt}%
\definecolor{currentstroke}{rgb}{1.000000,0.000000,0.000000}%
\pgfsetstrokecolor{currentstroke}%
\pgfsetdash{}{0pt}%
\pgfpathmoveto{\pgfqpoint{3.075343in}{0.500000in}}%
\pgfpathlineto{\pgfqpoint{3.075343in}{3.520000in}}%
\pgfusepath{stroke}%
\end{pgfscope}%
\begin{pgfscope}%
\pgfpathrectangle{\pgfqpoint{0.750000in}{0.500000in}}{\pgfqpoint{4.650000in}{3.020000in}}%
\pgfusepath{clip}%
\pgfsetrectcap%
\pgfsetroundjoin%
\pgfsetlinewidth{1.505625pt}%
\definecolor{currentstroke}{rgb}{1.000000,0.000000,0.000000}%
\pgfsetstrokecolor{currentstroke}%
\pgfsetdash{}{0pt}%
\pgfpathmoveto{\pgfqpoint{3.604258in}{0.500000in}}%
\pgfpathlineto{\pgfqpoint{3.604258in}{3.520000in}}%
\pgfusepath{stroke}%
\end{pgfscope}%
\begin{pgfscope}%
\pgfpathrectangle{\pgfqpoint{0.750000in}{0.500000in}}{\pgfqpoint{4.650000in}{3.020000in}}%
\pgfusepath{clip}%
\pgfsetrectcap%
\pgfsetroundjoin%
\pgfsetlinewidth{1.505625pt}%
\definecolor{currentstroke}{rgb}{1.000000,0.000000,0.000000}%
\pgfsetstrokecolor{currentstroke}%
\pgfsetdash{}{0pt}%
\pgfpathmoveto{\pgfqpoint{4.131996in}{0.500000in}}%
\pgfpathlineto{\pgfqpoint{4.131996in}{3.520000in}}%
\pgfusepath{stroke}%
\end{pgfscope}%
\begin{pgfscope}%
\pgfpathrectangle{\pgfqpoint{0.750000in}{0.500000in}}{\pgfqpoint{4.650000in}{3.020000in}}%
\pgfusepath{clip}%
\pgfsetrectcap%
\pgfsetroundjoin%
\pgfsetlinewidth{1.505625pt}%
\definecolor{currentstroke}{rgb}{1.000000,0.000000,0.000000}%
\pgfsetstrokecolor{currentstroke}%
\pgfsetdash{}{0pt}%
\pgfpathmoveto{\pgfqpoint{4.664820in}{0.500000in}}%
\pgfpathlineto{\pgfqpoint{4.664820in}{3.520000in}}%
\pgfusepath{stroke}%
\end{pgfscope}%
\begin{pgfscope}%
\pgfpathrectangle{\pgfqpoint{0.750000in}{0.500000in}}{\pgfqpoint{4.650000in}{3.020000in}}%
\pgfusepath{clip}%
\pgfsetrectcap%
\pgfsetroundjoin%
\pgfsetlinewidth{1.505625pt}%
\definecolor{currentstroke}{rgb}{1.000000,0.000000,0.000000}%
\pgfsetstrokecolor{currentstroke}%
\pgfsetdash{}{0pt}%
\pgfpathmoveto{\pgfqpoint{4.730860in}{0.500000in}}%
\pgfpathlineto{\pgfqpoint{4.730860in}{3.520000in}}%
\pgfusepath{stroke}%
\end{pgfscope}%
\begin{pgfscope}%
\pgfsetrectcap%
\pgfsetmiterjoin%
\pgfsetlinewidth{0.803000pt}%
\definecolor{currentstroke}{rgb}{0.000000,0.000000,0.000000}%
\pgfsetstrokecolor{currentstroke}%
\pgfsetdash{}{0pt}%
\pgfpathmoveto{\pgfqpoint{0.750000in}{0.500000in}}%
\pgfpathlineto{\pgfqpoint{0.750000in}{3.520000in}}%
\pgfusepath{stroke}%
\end{pgfscope}%
\begin{pgfscope}%
\pgfsetrectcap%
\pgfsetmiterjoin%
\pgfsetlinewidth{0.803000pt}%
\definecolor{currentstroke}{rgb}{0.000000,0.000000,0.000000}%
\pgfsetstrokecolor{currentstroke}%
\pgfsetdash{}{0pt}%
\pgfpathmoveto{\pgfqpoint{5.400000in}{0.500000in}}%
\pgfpathlineto{\pgfqpoint{5.400000in}{3.520000in}}%
\pgfusepath{stroke}%
\end{pgfscope}%
\begin{pgfscope}%
\pgfsetrectcap%
\pgfsetmiterjoin%
\pgfsetlinewidth{0.803000pt}%
\definecolor{currentstroke}{rgb}{0.000000,0.000000,0.000000}%
\pgfsetstrokecolor{currentstroke}%
\pgfsetdash{}{0pt}%
\pgfpathmoveto{\pgfqpoint{0.750000in}{0.500000in}}%
\pgfpathlineto{\pgfqpoint{5.400000in}{0.500000in}}%
\pgfusepath{stroke}%
\end{pgfscope}%
\begin{pgfscope}%
\pgfsetrectcap%
\pgfsetmiterjoin%
\pgfsetlinewidth{0.803000pt}%
\definecolor{currentstroke}{rgb}{0.000000,0.000000,0.000000}%
\pgfsetstrokecolor{currentstroke}%
\pgfsetdash{}{0pt}%
\pgfpathmoveto{\pgfqpoint{0.750000in}{3.520000in}}%
\pgfpathlineto{\pgfqpoint{5.400000in}{3.520000in}}%
\pgfusepath{stroke}%
\end{pgfscope}%
\end{pgfpicture}%
\makeatother%
\endgroup%

    \caption{PEC Data Set Anomalies}
    \label{fig:pec_outliers}
\end{figure}

\subsubsection{Cyber Security BETH Data Set}
\label{ref_results_beth_sim}
The Cyber Security BETH data set presented in section \ref{ref_beth_dataset} is used to perform this simulation. Table \ref{tab:beth_sim_params} specifies the parameters used with the anomaly detector to obtain the results in this section. This dataset is the largest of all in this study, containing approximately 800,000 time steps (~9 days of signal data). This dataset outlines the most challenging of the test scenarios for the detector and its performance for large window sizes. The standard deviation multiplier is the default. If the data follows a standard Gaussian distribution, the detected points would fall outside 99.7\% of the data present in the window, which would represent a significant outlier comparative to the rest of the data. The rolling range multiplier is set to one below the standard deviation multiplier for this experiment which provides some recall of previous outliers, but provides more weight to the present outlier detection score. The attacks have been manually labelled and although there are different kinds of attacks, the baseline signal (userId) remains the same. The recent range detection debounce multiplier is two due to the nature and volatility of the data.

\begin{table}[H]
%%\centering
\begin{tabular}{|l|c|l|}
    \hline
	\textbf{Parameter} & \textbf{Value} & \textbf{Description} \\ \hline
	m & 1000 & Window Size \\ \hline
	ts$\_$size & 10,000 & Time Series Size \\ \hline
	std$\_$dev & 3 & Standard Deviation Multiplier \\ \hline
	range & 2 & Rolling Range Multiplier\\ \hline
	recent & 2 & Recent Detection Debounce\\ \hline
\end{tabular}
\caption{BETH Data Set Model Parameters}
\label{tab:beth_sim_params}
\end{table}

Figure \ref{fig:beth_mp_hist} shows the computed matrix profile values for this experiment. In this study, the outliers are only present when there is a significantly large range spike as shown near time step 200,000. The regular and periodic small peaks prior are not outliers and the rolling range technique ensures they do not trigger the detector. This provides an example of how the rolling range parameter can be used to tune the memory and ability of the detector. 

\begin{figure}[H]
    %%\centering
    %% Creator: Matplotlib, PGF backend
%%
%% To include the figure in your LaTeX document, write
%%   \input{<filename>.pgf}
%%
%% Make sure the required packages are loaded in your preamble
%%   \usepackage{pgf}
%%
%% Also ensure that all the required font packages are loaded; for instance,
%% the lmodern package is sometimes necessary when using math font.
%%   \usepackage{lmodern}
%%
%% Figures using additional raster images can only be included by \input if
%% they are in the same directory as the main LaTeX file. For loading figures
%% from other directories you can use the `import` package
%%   \usepackage{import}
%%
%% and then include the figures with
%%   \import{<path to file>}{<filename>.pgf}
%%
%% Matplotlib used the following preamble
%%
\begingroup%
\makeatletter%
\begin{pgfpicture}%
\pgfpathrectangle{\pgfpointorigin}{\pgfqpoint{6.000000in}{4.000000in}}%
\pgfusepath{use as bounding box, clip}%
\begin{pgfscope}%
\pgfsetbuttcap%
\pgfsetmiterjoin%
\pgfsetlinewidth{0.000000pt}%
\definecolor{currentstroke}{rgb}{1.000000,1.000000,1.000000}%
\pgfsetstrokecolor{currentstroke}%
\pgfsetstrokeopacity{0.000000}%
\pgfsetdash{}{0pt}%
\pgfpathmoveto{\pgfqpoint{0.000000in}{0.000000in}}%
\pgfpathlineto{\pgfqpoint{6.000000in}{0.000000in}}%
\pgfpathlineto{\pgfqpoint{6.000000in}{4.000000in}}%
\pgfpathlineto{\pgfqpoint{0.000000in}{4.000000in}}%
\pgfpathlineto{\pgfqpoint{0.000000in}{0.000000in}}%
\pgfpathclose%
\pgfusepath{}%
\end{pgfscope}%
\begin{pgfscope}%
\pgfsetbuttcap%
\pgfsetmiterjoin%
\definecolor{currentfill}{rgb}{1.000000,1.000000,1.000000}%
\pgfsetfillcolor{currentfill}%
\pgfsetlinewidth{0.000000pt}%
\definecolor{currentstroke}{rgb}{0.000000,0.000000,0.000000}%
\pgfsetstrokecolor{currentstroke}%
\pgfsetstrokeopacity{0.000000}%
\pgfsetdash{}{0pt}%
\pgfpathmoveto{\pgfqpoint{0.750000in}{0.500000in}}%
\pgfpathlineto{\pgfqpoint{5.400000in}{0.500000in}}%
\pgfpathlineto{\pgfqpoint{5.400000in}{3.520000in}}%
\pgfpathlineto{\pgfqpoint{0.750000in}{3.520000in}}%
\pgfpathlineto{\pgfqpoint{0.750000in}{0.500000in}}%
\pgfpathclose%
\pgfusepath{fill}%
\end{pgfscope}%
\begin{pgfscope}%
\pgfsetbuttcap%
\pgfsetroundjoin%
\definecolor{currentfill}{rgb}{0.000000,0.000000,0.000000}%
\pgfsetfillcolor{currentfill}%
\pgfsetlinewidth{0.803000pt}%
\definecolor{currentstroke}{rgb}{0.000000,0.000000,0.000000}%
\pgfsetstrokecolor{currentstroke}%
\pgfsetdash{}{0pt}%
\pgfsys@defobject{currentmarker}{\pgfqpoint{0.000000in}{-0.048611in}}{\pgfqpoint{0.000000in}{0.000000in}}{%
\pgfpathmoveto{\pgfqpoint{0.000000in}{0.000000in}}%
\pgfpathlineto{\pgfqpoint{0.000000in}{-0.048611in}}%
\pgfusepath{stroke,fill}%
}%
\begin{pgfscope}%
\pgfsys@transformshift{0.961364in}{0.500000in}%
\pgfsys@useobject{currentmarker}{}%
\end{pgfscope}%
\end{pgfscope}%
\begin{pgfscope}%
\definecolor{textcolor}{rgb}{0.000000,0.000000,0.000000}%
\pgfsetstrokecolor{textcolor}%
\pgfsetfillcolor{textcolor}%
\pgftext[x=0.961364in,y=0.402778in,,top]{\color{textcolor}\rmfamily\fontsize{10.000000}{12.000000}\selectfont \(\displaystyle {0}\)}%
\end{pgfscope}%
\begin{pgfscope}%
\pgfsetbuttcap%
\pgfsetroundjoin%
\definecolor{currentfill}{rgb}{0.000000,0.000000,0.000000}%
\pgfsetfillcolor{currentfill}%
\pgfsetlinewidth{0.803000pt}%
\definecolor{currentstroke}{rgb}{0.000000,0.000000,0.000000}%
\pgfsetstrokecolor{currentstroke}%
\pgfsetdash{}{0pt}%
\pgfsys@defobject{currentmarker}{\pgfqpoint{0.000000in}{-0.048611in}}{\pgfqpoint{0.000000in}{0.000000in}}{%
\pgfpathmoveto{\pgfqpoint{0.000000in}{0.000000in}}%
\pgfpathlineto{\pgfqpoint{0.000000in}{-0.048611in}}%
\pgfusepath{stroke,fill}%
}%
\begin{pgfscope}%
\pgfsys@transformshift{1.916495in}{0.500000in}%
\pgfsys@useobject{currentmarker}{}%
\end{pgfscope}%
\end{pgfscope}%
\begin{pgfscope}%
\definecolor{textcolor}{rgb}{0.000000,0.000000,0.000000}%
\pgfsetstrokecolor{textcolor}%
\pgfsetfillcolor{textcolor}%
\pgftext[x=1.916495in,y=0.402778in,,top]{\color{textcolor}\rmfamily\fontsize{10.000000}{12.000000}\selectfont \(\displaystyle {200000}\)}%
\end{pgfscope}%
\begin{pgfscope}%
\pgfsetbuttcap%
\pgfsetroundjoin%
\definecolor{currentfill}{rgb}{0.000000,0.000000,0.000000}%
\pgfsetfillcolor{currentfill}%
\pgfsetlinewidth{0.803000pt}%
\definecolor{currentstroke}{rgb}{0.000000,0.000000,0.000000}%
\pgfsetstrokecolor{currentstroke}%
\pgfsetdash{}{0pt}%
\pgfsys@defobject{currentmarker}{\pgfqpoint{0.000000in}{-0.048611in}}{\pgfqpoint{0.000000in}{0.000000in}}{%
\pgfpathmoveto{\pgfqpoint{0.000000in}{0.000000in}}%
\pgfpathlineto{\pgfqpoint{0.000000in}{-0.048611in}}%
\pgfusepath{stroke,fill}%
}%
\begin{pgfscope}%
\pgfsys@transformshift{2.871626in}{0.500000in}%
\pgfsys@useobject{currentmarker}{}%
\end{pgfscope}%
\end{pgfscope}%
\begin{pgfscope}%
\definecolor{textcolor}{rgb}{0.000000,0.000000,0.000000}%
\pgfsetstrokecolor{textcolor}%
\pgfsetfillcolor{textcolor}%
\pgftext[x=2.871626in,y=0.402778in,,top]{\color{textcolor}\rmfamily\fontsize{10.000000}{12.000000}\selectfont \(\displaystyle {400000}\)}%
\end{pgfscope}%
\begin{pgfscope}%
\pgfsetbuttcap%
\pgfsetroundjoin%
\definecolor{currentfill}{rgb}{0.000000,0.000000,0.000000}%
\pgfsetfillcolor{currentfill}%
\pgfsetlinewidth{0.803000pt}%
\definecolor{currentstroke}{rgb}{0.000000,0.000000,0.000000}%
\pgfsetstrokecolor{currentstroke}%
\pgfsetdash{}{0pt}%
\pgfsys@defobject{currentmarker}{\pgfqpoint{0.000000in}{-0.048611in}}{\pgfqpoint{0.000000in}{0.000000in}}{%
\pgfpathmoveto{\pgfqpoint{0.000000in}{0.000000in}}%
\pgfpathlineto{\pgfqpoint{0.000000in}{-0.048611in}}%
\pgfusepath{stroke,fill}%
}%
\begin{pgfscope}%
\pgfsys@transformshift{3.826758in}{0.500000in}%
\pgfsys@useobject{currentmarker}{}%
\end{pgfscope}%
\end{pgfscope}%
\begin{pgfscope}%
\definecolor{textcolor}{rgb}{0.000000,0.000000,0.000000}%
\pgfsetstrokecolor{textcolor}%
\pgfsetfillcolor{textcolor}%
\pgftext[x=3.826758in,y=0.402778in,,top]{\color{textcolor}\rmfamily\fontsize{10.000000}{12.000000}\selectfont \(\displaystyle {600000}\)}%
\end{pgfscope}%
\begin{pgfscope}%
\pgfsetbuttcap%
\pgfsetroundjoin%
\definecolor{currentfill}{rgb}{0.000000,0.000000,0.000000}%
\pgfsetfillcolor{currentfill}%
\pgfsetlinewidth{0.803000pt}%
\definecolor{currentstroke}{rgb}{0.000000,0.000000,0.000000}%
\pgfsetstrokecolor{currentstroke}%
\pgfsetdash{}{0pt}%
\pgfsys@defobject{currentmarker}{\pgfqpoint{0.000000in}{-0.048611in}}{\pgfqpoint{0.000000in}{0.000000in}}{%
\pgfpathmoveto{\pgfqpoint{0.000000in}{0.000000in}}%
\pgfpathlineto{\pgfqpoint{0.000000in}{-0.048611in}}%
\pgfusepath{stroke,fill}%
}%
\begin{pgfscope}%
\pgfsys@transformshift{4.781889in}{0.500000in}%
\pgfsys@useobject{currentmarker}{}%
\end{pgfscope}%
\end{pgfscope}%
\begin{pgfscope}%
\definecolor{textcolor}{rgb}{0.000000,0.000000,0.000000}%
\pgfsetstrokecolor{textcolor}%
\pgfsetfillcolor{textcolor}%
\pgftext[x=4.781889in,y=0.402778in,,top]{\color{textcolor}\rmfamily\fontsize{10.000000}{12.000000}\selectfont \(\displaystyle {800000}\)}%
\end{pgfscope}%
\begin{pgfscope}%
\definecolor{textcolor}{rgb}{0.000000,0.000000,0.000000}%
\pgfsetstrokecolor{textcolor}%
\pgfsetfillcolor{textcolor}%
\pgftext[x=3.075000in,y=0.223766in,,top]{\color{textcolor}\rmfamily\fontsize{10.000000}{12.000000}\selectfont time}%
\end{pgfscope}%
\begin{pgfscope}%
\pgfsetbuttcap%
\pgfsetroundjoin%
\definecolor{currentfill}{rgb}{0.000000,0.000000,0.000000}%
\pgfsetfillcolor{currentfill}%
\pgfsetlinewidth{0.803000pt}%
\definecolor{currentstroke}{rgb}{0.000000,0.000000,0.000000}%
\pgfsetstrokecolor{currentstroke}%
\pgfsetdash{}{0pt}%
\pgfsys@defobject{currentmarker}{\pgfqpoint{-0.048611in}{0.000000in}}{\pgfqpoint{-0.000000in}{0.000000in}}{%
\pgfpathmoveto{\pgfqpoint{-0.000000in}{0.000000in}}%
\pgfpathlineto{\pgfqpoint{-0.048611in}{0.000000in}}%
\pgfusepath{stroke,fill}%
}%
\begin{pgfscope}%
\pgfsys@transformshift{0.750000in}{0.637273in}%
\pgfsys@useobject{currentmarker}{}%
\end{pgfscope}%
\end{pgfscope}%
\begin{pgfscope}%
\definecolor{textcolor}{rgb}{0.000000,0.000000,0.000000}%
\pgfsetstrokecolor{textcolor}%
\pgfsetfillcolor{textcolor}%
\pgftext[x=0.583333in, y=0.589047in, left, base]{\color{textcolor}\rmfamily\fontsize{10.000000}{12.000000}\selectfont \(\displaystyle {0}\)}%
\end{pgfscope}%
\begin{pgfscope}%
\pgfsetbuttcap%
\pgfsetroundjoin%
\definecolor{currentfill}{rgb}{0.000000,0.000000,0.000000}%
\pgfsetfillcolor{currentfill}%
\pgfsetlinewidth{0.803000pt}%
\definecolor{currentstroke}{rgb}{0.000000,0.000000,0.000000}%
\pgfsetstrokecolor{currentstroke}%
\pgfsetdash{}{0pt}%
\pgfsys@defobject{currentmarker}{\pgfqpoint{-0.048611in}{0.000000in}}{\pgfqpoint{-0.000000in}{0.000000in}}{%
\pgfpathmoveto{\pgfqpoint{-0.000000in}{0.000000in}}%
\pgfpathlineto{\pgfqpoint{-0.048611in}{0.000000in}}%
\pgfusepath{stroke,fill}%
}%
\begin{pgfscope}%
\pgfsys@transformshift{0.750000in}{0.996115in}%
\pgfsys@useobject{currentmarker}{}%
\end{pgfscope}%
\end{pgfscope}%
\begin{pgfscope}%
\definecolor{textcolor}{rgb}{0.000000,0.000000,0.000000}%
\pgfsetstrokecolor{textcolor}%
\pgfsetfillcolor{textcolor}%
\pgftext[x=0.374999in, y=0.947890in, left, base]{\color{textcolor}\rmfamily\fontsize{10.000000}{12.000000}\selectfont \(\displaystyle {2500}\)}%
\end{pgfscope}%
\begin{pgfscope}%
\pgfsetbuttcap%
\pgfsetroundjoin%
\definecolor{currentfill}{rgb}{0.000000,0.000000,0.000000}%
\pgfsetfillcolor{currentfill}%
\pgfsetlinewidth{0.803000pt}%
\definecolor{currentstroke}{rgb}{0.000000,0.000000,0.000000}%
\pgfsetstrokecolor{currentstroke}%
\pgfsetdash{}{0pt}%
\pgfsys@defobject{currentmarker}{\pgfqpoint{-0.048611in}{0.000000in}}{\pgfqpoint{-0.000000in}{0.000000in}}{%
\pgfpathmoveto{\pgfqpoint{-0.000000in}{0.000000in}}%
\pgfpathlineto{\pgfqpoint{-0.048611in}{0.000000in}}%
\pgfusepath{stroke,fill}%
}%
\begin{pgfscope}%
\pgfsys@transformshift{0.750000in}{1.354958in}%
\pgfsys@useobject{currentmarker}{}%
\end{pgfscope}%
\end{pgfscope}%
\begin{pgfscope}%
\definecolor{textcolor}{rgb}{0.000000,0.000000,0.000000}%
\pgfsetstrokecolor{textcolor}%
\pgfsetfillcolor{textcolor}%
\pgftext[x=0.374999in, y=1.306733in, left, base]{\color{textcolor}\rmfamily\fontsize{10.000000}{12.000000}\selectfont \(\displaystyle {5000}\)}%
\end{pgfscope}%
\begin{pgfscope}%
\pgfsetbuttcap%
\pgfsetroundjoin%
\definecolor{currentfill}{rgb}{0.000000,0.000000,0.000000}%
\pgfsetfillcolor{currentfill}%
\pgfsetlinewidth{0.803000pt}%
\definecolor{currentstroke}{rgb}{0.000000,0.000000,0.000000}%
\pgfsetstrokecolor{currentstroke}%
\pgfsetdash{}{0pt}%
\pgfsys@defobject{currentmarker}{\pgfqpoint{-0.048611in}{0.000000in}}{\pgfqpoint{-0.000000in}{0.000000in}}{%
\pgfpathmoveto{\pgfqpoint{-0.000000in}{0.000000in}}%
\pgfpathlineto{\pgfqpoint{-0.048611in}{0.000000in}}%
\pgfusepath{stroke,fill}%
}%
\begin{pgfscope}%
\pgfsys@transformshift{0.750000in}{1.713801in}%
\pgfsys@useobject{currentmarker}{}%
\end{pgfscope}%
\end{pgfscope}%
\begin{pgfscope}%
\definecolor{textcolor}{rgb}{0.000000,0.000000,0.000000}%
\pgfsetstrokecolor{textcolor}%
\pgfsetfillcolor{textcolor}%
\pgftext[x=0.374999in, y=1.665576in, left, base]{\color{textcolor}\rmfamily\fontsize{10.000000}{12.000000}\selectfont \(\displaystyle {7500}\)}%
\end{pgfscope}%
\begin{pgfscope}%
\pgfsetbuttcap%
\pgfsetroundjoin%
\definecolor{currentfill}{rgb}{0.000000,0.000000,0.000000}%
\pgfsetfillcolor{currentfill}%
\pgfsetlinewidth{0.803000pt}%
\definecolor{currentstroke}{rgb}{0.000000,0.000000,0.000000}%
\pgfsetstrokecolor{currentstroke}%
\pgfsetdash{}{0pt}%
\pgfsys@defobject{currentmarker}{\pgfqpoint{-0.048611in}{0.000000in}}{\pgfqpoint{-0.000000in}{0.000000in}}{%
\pgfpathmoveto{\pgfqpoint{-0.000000in}{0.000000in}}%
\pgfpathlineto{\pgfqpoint{-0.048611in}{0.000000in}}%
\pgfusepath{stroke,fill}%
}%
\begin{pgfscope}%
\pgfsys@transformshift{0.750000in}{2.072644in}%
\pgfsys@useobject{currentmarker}{}%
\end{pgfscope}%
\end{pgfscope}%
\begin{pgfscope}%
\definecolor{textcolor}{rgb}{0.000000,0.000000,0.000000}%
\pgfsetstrokecolor{textcolor}%
\pgfsetfillcolor{textcolor}%
\pgftext[x=0.305554in, y=2.024418in, left, base]{\color{textcolor}\rmfamily\fontsize{10.000000}{12.000000}\selectfont \(\displaystyle {10000}\)}%
\end{pgfscope}%
\begin{pgfscope}%
\pgfsetbuttcap%
\pgfsetroundjoin%
\definecolor{currentfill}{rgb}{0.000000,0.000000,0.000000}%
\pgfsetfillcolor{currentfill}%
\pgfsetlinewidth{0.803000pt}%
\definecolor{currentstroke}{rgb}{0.000000,0.000000,0.000000}%
\pgfsetstrokecolor{currentstroke}%
\pgfsetdash{}{0pt}%
\pgfsys@defobject{currentmarker}{\pgfqpoint{-0.048611in}{0.000000in}}{\pgfqpoint{-0.000000in}{0.000000in}}{%
\pgfpathmoveto{\pgfqpoint{-0.000000in}{0.000000in}}%
\pgfpathlineto{\pgfqpoint{-0.048611in}{0.000000in}}%
\pgfusepath{stroke,fill}%
}%
\begin{pgfscope}%
\pgfsys@transformshift{0.750000in}{2.431486in}%
\pgfsys@useobject{currentmarker}{}%
\end{pgfscope}%
\end{pgfscope}%
\begin{pgfscope}%
\definecolor{textcolor}{rgb}{0.000000,0.000000,0.000000}%
\pgfsetstrokecolor{textcolor}%
\pgfsetfillcolor{textcolor}%
\pgftext[x=0.305554in, y=2.383261in, left, base]{\color{textcolor}\rmfamily\fontsize{10.000000}{12.000000}\selectfont \(\displaystyle {12500}\)}%
\end{pgfscope}%
\begin{pgfscope}%
\pgfsetbuttcap%
\pgfsetroundjoin%
\definecolor{currentfill}{rgb}{0.000000,0.000000,0.000000}%
\pgfsetfillcolor{currentfill}%
\pgfsetlinewidth{0.803000pt}%
\definecolor{currentstroke}{rgb}{0.000000,0.000000,0.000000}%
\pgfsetstrokecolor{currentstroke}%
\pgfsetdash{}{0pt}%
\pgfsys@defobject{currentmarker}{\pgfqpoint{-0.048611in}{0.000000in}}{\pgfqpoint{-0.000000in}{0.000000in}}{%
\pgfpathmoveto{\pgfqpoint{-0.000000in}{0.000000in}}%
\pgfpathlineto{\pgfqpoint{-0.048611in}{0.000000in}}%
\pgfusepath{stroke,fill}%
}%
\begin{pgfscope}%
\pgfsys@transformshift{0.750000in}{2.790329in}%
\pgfsys@useobject{currentmarker}{}%
\end{pgfscope}%
\end{pgfscope}%
\begin{pgfscope}%
\definecolor{textcolor}{rgb}{0.000000,0.000000,0.000000}%
\pgfsetstrokecolor{textcolor}%
\pgfsetfillcolor{textcolor}%
\pgftext[x=0.305554in, y=2.742104in, left, base]{\color{textcolor}\rmfamily\fontsize{10.000000}{12.000000}\selectfont \(\displaystyle {15000}\)}%
\end{pgfscope}%
\begin{pgfscope}%
\pgfsetbuttcap%
\pgfsetroundjoin%
\definecolor{currentfill}{rgb}{0.000000,0.000000,0.000000}%
\pgfsetfillcolor{currentfill}%
\pgfsetlinewidth{0.803000pt}%
\definecolor{currentstroke}{rgb}{0.000000,0.000000,0.000000}%
\pgfsetstrokecolor{currentstroke}%
\pgfsetdash{}{0pt}%
\pgfsys@defobject{currentmarker}{\pgfqpoint{-0.048611in}{0.000000in}}{\pgfqpoint{-0.000000in}{0.000000in}}{%
\pgfpathmoveto{\pgfqpoint{-0.000000in}{0.000000in}}%
\pgfpathlineto{\pgfqpoint{-0.048611in}{0.000000in}}%
\pgfusepath{stroke,fill}%
}%
\begin{pgfscope}%
\pgfsys@transformshift{0.750000in}{3.149172in}%
\pgfsys@useobject{currentmarker}{}%
\end{pgfscope}%
\end{pgfscope}%
\begin{pgfscope}%
\definecolor{textcolor}{rgb}{0.000000,0.000000,0.000000}%
\pgfsetstrokecolor{textcolor}%
\pgfsetfillcolor{textcolor}%
\pgftext[x=0.305554in, y=3.100946in, left, base]{\color{textcolor}\rmfamily\fontsize{10.000000}{12.000000}\selectfont \(\displaystyle {17500}\)}%
\end{pgfscope}%
\begin{pgfscope}%
\pgfsetbuttcap%
\pgfsetroundjoin%
\definecolor{currentfill}{rgb}{0.000000,0.000000,0.000000}%
\pgfsetfillcolor{currentfill}%
\pgfsetlinewidth{0.803000pt}%
\definecolor{currentstroke}{rgb}{0.000000,0.000000,0.000000}%
\pgfsetstrokecolor{currentstroke}%
\pgfsetdash{}{0pt}%
\pgfsys@defobject{currentmarker}{\pgfqpoint{-0.048611in}{0.000000in}}{\pgfqpoint{-0.000000in}{0.000000in}}{%
\pgfpathmoveto{\pgfqpoint{-0.000000in}{0.000000in}}%
\pgfpathlineto{\pgfqpoint{-0.048611in}{0.000000in}}%
\pgfusepath{stroke,fill}%
}%
\begin{pgfscope}%
\pgfsys@transformshift{0.750000in}{3.508014in}%
\pgfsys@useobject{currentmarker}{}%
\end{pgfscope}%
\end{pgfscope}%
\begin{pgfscope}%
\definecolor{textcolor}{rgb}{0.000000,0.000000,0.000000}%
\pgfsetstrokecolor{textcolor}%
\pgfsetfillcolor{textcolor}%
\pgftext[x=0.305554in, y=3.459789in, left, base]{\color{textcolor}\rmfamily\fontsize{10.000000}{12.000000}\selectfont \(\displaystyle {20000}\)}%
\end{pgfscope}%
\begin{pgfscope}%
\pgfpathrectangle{\pgfqpoint{0.750000in}{0.500000in}}{\pgfqpoint{4.650000in}{3.020000in}}%
\pgfusepath{clip}%
\pgfsetrectcap%
\pgfsetroundjoin%
\pgfsetlinewidth{1.505625pt}%
\definecolor{currentstroke}{rgb}{0.000000,0.000000,1.000000}%
\pgfsetstrokecolor{currentstroke}%
\pgfsetdash{}{0pt}%
\pgfpathmoveto{\pgfqpoint{0.961364in}{3.136814in}}%
\pgfpathlineto{\pgfqpoint{0.965643in}{3.136726in}}%
\pgfpathlineto{\pgfqpoint{0.968131in}{1.476377in}}%
\pgfpathlineto{\pgfqpoint{0.969969in}{0.703583in}}%
\pgfpathlineto{\pgfqpoint{0.992148in}{0.703552in}}%
\pgfpathlineto{\pgfqpoint{0.993814in}{0.662468in}}%
\pgfpathlineto{\pgfqpoint{1.004841in}{0.662468in}}%
\pgfpathlineto{\pgfqpoint{1.006389in}{0.657572in}}%
\pgfpathlineto{\pgfqpoint{1.016551in}{0.657572in}}%
\pgfpathlineto{\pgfqpoint{1.018137in}{0.700367in}}%
\pgfpathlineto{\pgfqpoint{1.064160in}{0.700367in}}%
\pgfpathlineto{\pgfqpoint{1.064704in}{0.657675in}}%
\pgfpathlineto{\pgfqpoint{1.065764in}{0.678249in}}%
\pgfpathlineto{\pgfqpoint{1.067340in}{0.678249in}}%
\pgfpathlineto{\pgfqpoint{1.068921in}{0.703651in}}%
\pgfpathlineto{\pgfqpoint{1.112074in}{0.703651in}}%
\pgfpathlineto{\pgfqpoint{1.113621in}{0.700400in}}%
\pgfpathlineto{\pgfqpoint{1.115407in}{0.701593in}}%
\pgfpathlineto{\pgfqpoint{1.117356in}{0.836741in}}%
\pgfpathlineto{\pgfqpoint{1.149066in}{0.835290in}}%
\pgfpathlineto{\pgfqpoint{1.150647in}{0.830837in}}%
\pgfpathlineto{\pgfqpoint{1.162958in}{0.829770in}}%
\pgfpathlineto{\pgfqpoint{1.165461in}{0.721757in}}%
\pgfpathlineto{\pgfqpoint{1.192028in}{0.720528in}}%
\pgfpathlineto{\pgfqpoint{1.193776in}{0.666053in}}%
\pgfpathlineto{\pgfqpoint{1.237621in}{0.666053in}}%
\pgfpathlineto{\pgfqpoint{1.239173in}{0.657775in}}%
\pgfpathlineto{\pgfqpoint{1.251523in}{0.657775in}}%
\pgfpathlineto{\pgfqpoint{1.253113in}{0.703583in}}%
\pgfpathlineto{\pgfqpoint{1.280053in}{0.704751in}}%
\pgfpathlineto{\pgfqpoint{1.282016in}{0.837771in}}%
\pgfpathlineto{\pgfqpoint{1.327556in}{0.836720in}}%
\pgfpathlineto{\pgfqpoint{1.330751in}{0.657775in}}%
\pgfpathlineto{\pgfqpoint{1.338507in}{0.657775in}}%
\pgfpathlineto{\pgfqpoint{1.340054in}{0.666053in}}%
\pgfpathlineto{\pgfqpoint{1.366459in}{0.666125in}}%
\pgfpathlineto{\pgfqpoint{1.368049in}{0.700400in}}%
\pgfpathlineto{\pgfqpoint{1.413752in}{0.700400in}}%
\pgfpathlineto{\pgfqpoint{1.415414in}{0.657775in}}%
\pgfpathlineto{\pgfqpoint{1.444779in}{0.657775in}}%
\pgfpathlineto{\pgfqpoint{1.446527in}{0.847513in}}%
\pgfpathlineto{\pgfqpoint{1.484632in}{0.846028in}}%
\pgfpathlineto{\pgfqpoint{1.486261in}{0.836751in}}%
\pgfpathlineto{\pgfqpoint{1.492460in}{0.835716in}}%
\pgfpathlineto{\pgfqpoint{1.495191in}{0.703583in}}%
\pgfpathlineto{\pgfqpoint{1.528478in}{0.703583in}}%
\pgfpathlineto{\pgfqpoint{1.530130in}{0.666053in}}%
\pgfpathlineto{\pgfqpoint{1.532035in}{0.666053in}}%
\pgfpathlineto{\pgfqpoint{1.533588in}{0.657573in}}%
\pgfpathlineto{\pgfqpoint{1.606555in}{0.657775in}}%
\pgfpathlineto{\pgfqpoint{1.608169in}{0.683412in}}%
\pgfpathlineto{\pgfqpoint{1.618045in}{0.683050in}}%
\pgfpathlineto{\pgfqpoint{1.619941in}{0.842994in}}%
\pgfpathlineto{\pgfqpoint{1.665730in}{0.841486in}}%
\pgfpathlineto{\pgfqpoint{1.668514in}{0.710184in}}%
\pgfpathlineto{\pgfqpoint{1.697994in}{0.710155in}}%
\pgfpathlineto{\pgfqpoint{1.699585in}{0.692992in}}%
\pgfpathlineto{\pgfqpoint{1.708635in}{0.692992in}}%
\pgfpathlineto{\pgfqpoint{1.710230in}{0.667300in}}%
\pgfpathlineto{\pgfqpoint{1.710416in}{0.667300in}}%
\pgfpathlineto{\pgfqpoint{1.711968in}{0.657775in}}%
\pgfpathlineto{\pgfqpoint{1.714891in}{0.657776in}}%
\pgfpathlineto{\pgfqpoint{1.716471in}{0.696978in}}%
\pgfpathlineto{\pgfqpoint{1.762628in}{0.696978in}}%
\pgfpathlineto{\pgfqpoint{1.764276in}{0.657775in}}%
\pgfpathlineto{\pgfqpoint{1.793235in}{0.657573in}}%
\pgfpathlineto{\pgfqpoint{1.794974in}{0.847513in}}%
\pgfpathlineto{\pgfqpoint{1.840935in}{0.846018in}}%
\pgfpathlineto{\pgfqpoint{1.844215in}{0.665980in}}%
\pgfpathlineto{\pgfqpoint{1.848919in}{0.665980in}}%
\pgfpathlineto{\pgfqpoint{1.850472in}{0.657775in}}%
\pgfpathlineto{\pgfqpoint{1.890143in}{0.657775in}}%
\pgfpathlineto{\pgfqpoint{1.891733in}{0.675203in}}%
\pgfpathlineto{\pgfqpoint{1.893710in}{0.675203in}}%
\pgfpathlineto{\pgfqpoint{1.895267in}{0.684209in}}%
\pgfpathlineto{\pgfqpoint{1.921404in}{0.684209in}}%
\pgfpathlineto{\pgfqpoint{1.923128in}{0.782640in}}%
\pgfpathlineto{\pgfqpoint{1.931214in}{0.781873in}}%
\pgfpathlineto{\pgfqpoint{1.933902in}{2.913650in}}%
\pgfpathlineto{\pgfqpoint{1.934017in}{2.913602in}}%
\pgfpathlineto{\pgfqpoint{1.952608in}{2.913602in}}%
\pgfpathlineto{\pgfqpoint{1.954672in}{3.382727in}}%
\pgfpathlineto{\pgfqpoint{1.955149in}{3.382727in}}%
\pgfpathlineto{\pgfqpoint{1.956945in}{3.220671in}}%
\pgfpathlineto{\pgfqpoint{1.995948in}{3.220671in}}%
\pgfpathlineto{\pgfqpoint{1.999014in}{1.637691in}}%
\pgfpathlineto{\pgfqpoint{2.001015in}{0.637273in}}%
\pgfpathlineto{\pgfqpoint{2.014620in}{0.637273in}}%
\pgfpathlineto{\pgfqpoint{2.016172in}{0.991692in}}%
\pgfpathlineto{\pgfqpoint{2.017161in}{0.993071in}}%
\pgfpathlineto{\pgfqpoint{2.019148in}{1.025090in}}%
\pgfpathlineto{\pgfqpoint{2.022997in}{1.025090in}}%
\pgfpathlineto{\pgfqpoint{2.024544in}{0.994885in}}%
\pgfpathlineto{\pgfqpoint{2.044139in}{0.994885in}}%
\pgfpathlineto{\pgfqpoint{2.045696in}{0.897707in}}%
\pgfpathlineto{\pgfqpoint{2.046340in}{0.897707in}}%
\pgfpathlineto{\pgfqpoint{2.048198in}{2.879490in}}%
\pgfpathlineto{\pgfqpoint{2.092979in}{2.879927in}}%
\pgfpathlineto{\pgfqpoint{2.096781in}{0.700400in}}%
\pgfpathlineto{\pgfqpoint{2.113257in}{0.701690in}}%
\pgfpathlineto{\pgfqpoint{2.115210in}{0.837802in}}%
\pgfpathlineto{\pgfqpoint{2.160808in}{0.836772in}}%
\pgfpathlineto{\pgfqpoint{2.163893in}{0.665980in}}%
\pgfpathlineto{\pgfqpoint{2.183999in}{0.665980in}}%
\pgfpathlineto{\pgfqpoint{2.185551in}{0.657572in}}%
\pgfpathlineto{\pgfqpoint{2.210499in}{0.657572in}}%
\pgfpathlineto{\pgfqpoint{2.212046in}{0.665980in}}%
\pgfpathlineto{\pgfqpoint{2.217251in}{0.665980in}}%
\pgfpathlineto{\pgfqpoint{2.218842in}{0.687432in}}%
\pgfpathlineto{\pgfqpoint{2.264922in}{0.687432in}}%
\pgfpathlineto{\pgfqpoint{2.266531in}{0.657573in}}%
\pgfpathlineto{\pgfqpoint{2.277979in}{0.657573in}}%
\pgfpathlineto{\pgfqpoint{2.279536in}{0.637273in}}%
\pgfpathlineto{\pgfqpoint{2.284120in}{0.637273in}}%
\pgfpathlineto{\pgfqpoint{2.285730in}{0.847513in}}%
\pgfpathlineto{\pgfqpoint{2.317259in}{0.846018in}}%
\pgfpathlineto{\pgfqpoint{2.318868in}{0.838837in}}%
\pgfpathlineto{\pgfqpoint{2.331920in}{0.837791in}}%
\pgfpathlineto{\pgfqpoint{2.334766in}{0.697047in}}%
\pgfpathlineto{\pgfqpoint{2.360908in}{0.697046in}}%
\pgfpathlineto{\pgfqpoint{2.362556in}{0.657775in}}%
\pgfpathlineto{\pgfqpoint{2.370923in}{0.657775in}}%
\pgfpathlineto{\pgfqpoint{2.372475in}{0.666124in}}%
\pgfpathlineto{\pgfqpoint{2.377761in}{0.666124in}}%
\pgfpathlineto{\pgfqpoint{2.379332in}{0.683001in}}%
\pgfpathlineto{\pgfqpoint{2.396802in}{0.684209in}}%
\pgfpathlineto{\pgfqpoint{2.398349in}{0.686748in}}%
\pgfpathlineto{\pgfqpoint{2.411267in}{0.686748in}}%
\pgfpathlineto{\pgfqpoint{2.412815in}{0.684209in}}%
\pgfpathlineto{\pgfqpoint{2.428441in}{0.684209in}}%
\pgfpathlineto{\pgfqpoint{2.429993in}{0.688492in}}%
\pgfpathlineto{\pgfqpoint{2.460901in}{0.688492in}}%
\pgfpathlineto{\pgfqpoint{2.462448in}{0.684209in}}%
\pgfpathlineto{\pgfqpoint{2.475242in}{0.683376in}}%
\pgfpathlineto{\pgfqpoint{2.476789in}{0.681525in}}%
\pgfpathlineto{\pgfqpoint{2.488270in}{0.681525in}}%
\pgfpathlineto{\pgfqpoint{2.489832in}{0.690828in}}%
\pgfpathlineto{\pgfqpoint{2.532884in}{0.690751in}}%
\pgfpathlineto{\pgfqpoint{2.534431in}{0.688411in}}%
\pgfpathlineto{\pgfqpoint{2.551681in}{0.688411in}}%
\pgfpathlineto{\pgfqpoint{2.553233in}{0.681525in}}%
\pgfpathlineto{\pgfqpoint{2.599037in}{0.681525in}}%
\pgfpathlineto{\pgfqpoint{2.600598in}{0.668564in}}%
\pgfpathlineto{\pgfqpoint{2.613072in}{0.668564in}}%
\pgfpathlineto{\pgfqpoint{2.614624in}{0.659399in}}%
\pgfpathlineto{\pgfqpoint{2.617432in}{0.659399in}}%
\pgfpathlineto{\pgfqpoint{2.618980in}{0.657572in}}%
\pgfpathlineto{\pgfqpoint{2.624839in}{0.657572in}}%
\pgfpathlineto{\pgfqpoint{2.626530in}{0.715548in}}%
\pgfpathlineto{\pgfqpoint{2.640222in}{0.715548in}}%
\pgfpathlineto{\pgfqpoint{2.641788in}{0.707599in}}%
\pgfpathlineto{\pgfqpoint{2.645322in}{0.707599in}}%
\pgfpathlineto{\pgfqpoint{2.646870in}{0.705525in}}%
\pgfpathlineto{\pgfqpoint{2.670991in}{0.705525in}}%
\pgfpathlineto{\pgfqpoint{2.672539in}{0.703707in}}%
\pgfpathlineto{\pgfqpoint{2.672596in}{0.702168in}}%
\pgfpathlineto{\pgfqpoint{2.674182in}{0.684209in}}%
\pgfpathlineto{\pgfqpoint{2.682090in}{0.684209in}}%
\pgfpathlineto{\pgfqpoint{2.683637in}{0.681525in}}%
\pgfpathlineto{\pgfqpoint{2.686398in}{0.681525in}}%
\pgfpathlineto{\pgfqpoint{2.687945in}{0.684209in}}%
\pgfpathlineto{\pgfqpoint{2.695113in}{0.685242in}}%
\pgfpathlineto{\pgfqpoint{2.696670in}{0.692368in}}%
\pgfpathlineto{\pgfqpoint{2.738366in}{0.692368in}}%
\pgfpathlineto{\pgfqpoint{2.739918in}{0.686748in}}%
\pgfpathlineto{\pgfqpoint{2.745167in}{0.686748in}}%
\pgfpathlineto{\pgfqpoint{2.746714in}{0.684209in}}%
\pgfpathlineto{\pgfqpoint{2.753386in}{0.684209in}}%
\pgfpathlineto{\pgfqpoint{2.754933in}{0.686748in}}%
\pgfpathlineto{\pgfqpoint{2.757703in}{0.687729in}}%
\pgfpathlineto{\pgfqpoint{2.759255in}{0.692368in}}%
\pgfpathlineto{\pgfqpoint{2.792489in}{0.693741in}}%
\pgfpathlineto{\pgfqpoint{2.794437in}{0.850079in}}%
\pgfpathlineto{\pgfqpoint{2.840145in}{0.848687in}}%
\pgfpathlineto{\pgfqpoint{2.840809in}{0.773401in}}%
\pgfpathlineto{\pgfqpoint{2.842901in}{0.684209in}}%
\pgfpathlineto{\pgfqpoint{2.854052in}{0.685242in}}%
\pgfpathlineto{\pgfqpoint{2.855604in}{0.690099in}}%
\pgfpathlineto{\pgfqpoint{2.897329in}{0.690099in}}%
\pgfpathlineto{\pgfqpoint{2.898881in}{0.684209in}}%
\pgfpathlineto{\pgfqpoint{2.966122in}{0.685242in}}%
\pgfpathlineto{\pgfqpoint{2.967674in}{0.690099in}}%
\pgfpathlineto{\pgfqpoint{2.979685in}{0.690099in}}%
\pgfpathlineto{\pgfqpoint{2.981237in}{0.685242in}}%
\pgfpathlineto{\pgfqpoint{3.022661in}{0.685242in}}%
\pgfpathlineto{\pgfqpoint{3.024232in}{0.668564in}}%
\pgfpathlineto{\pgfqpoint{3.040045in}{0.667300in}}%
\pgfpathlineto{\pgfqpoint{3.041592in}{0.665980in}}%
\pgfpathlineto{\pgfqpoint{3.085132in}{0.665980in}}%
\pgfpathlineto{\pgfqpoint{3.086708in}{0.637273in}}%
\pgfpathlineto{\pgfqpoint{3.087457in}{0.637273in}}%
\pgfpathlineto{\pgfqpoint{3.089009in}{0.666124in}}%
\pgfpathlineto{\pgfqpoint{3.120576in}{0.665980in}}%
\pgfpathlineto{\pgfqpoint{3.122372in}{0.846511in}}%
\pgfpathlineto{\pgfqpoint{3.168319in}{0.845019in}}%
\pgfpathlineto{\pgfqpoint{3.171022in}{0.721717in}}%
\pgfpathlineto{\pgfqpoint{3.174045in}{0.723027in}}%
\pgfpathlineto{\pgfqpoint{3.175912in}{0.810719in}}%
\pgfpathlineto{\pgfqpoint{3.175988in}{0.811999in}}%
\pgfpathlineto{\pgfqpoint{3.178009in}{0.875651in}}%
\pgfpathlineto{\pgfqpoint{3.178080in}{0.877050in}}%
\pgfpathlineto{\pgfqpoint{3.179771in}{0.891485in}}%
\pgfpathlineto{\pgfqpoint{3.182999in}{0.892796in}}%
\pgfpathlineto{\pgfqpoint{3.184589in}{0.897120in}}%
\pgfpathlineto{\pgfqpoint{3.186466in}{0.898404in}}%
\pgfpathlineto{\pgfqpoint{3.188042in}{0.901373in}}%
\pgfpathlineto{\pgfqpoint{3.191963in}{0.902636in}}%
\pgfpathlineto{\pgfqpoint{3.193940in}{0.941010in}}%
\pgfpathlineto{\pgfqpoint{3.194408in}{0.942473in}}%
\pgfpathlineto{\pgfqpoint{3.196008in}{0.946823in}}%
\pgfpathlineto{\pgfqpoint{3.196963in}{0.945380in}}%
\pgfpathlineto{\pgfqpoint{3.198601in}{0.938061in}}%
\pgfpathlineto{\pgfqpoint{3.200884in}{0.936575in}}%
\pgfpathlineto{\pgfqpoint{3.202450in}{0.934708in}}%
\pgfpathlineto{\pgfqpoint{3.209012in}{0.933206in}}%
\pgfpathlineto{\pgfqpoint{3.210583in}{0.930938in}}%
\pgfpathlineto{\pgfqpoint{3.210918in}{0.929417in}}%
\pgfpathlineto{\pgfqpoint{3.212522in}{0.924416in}}%
\pgfpathlineto{\pgfqpoint{3.231252in}{0.924416in}}%
\pgfpathlineto{\pgfqpoint{3.233177in}{2.752681in}}%
\pgfpathlineto{\pgfqpoint{3.233430in}{2.754207in}}%
\pgfpathlineto{\pgfqpoint{3.233549in}{2.772441in}}%
\pgfpathlineto{\pgfqpoint{3.235159in}{2.820186in}}%
\pgfpathlineto{\pgfqpoint{3.246362in}{2.819484in}}%
\pgfpathlineto{\pgfqpoint{3.248827in}{2.096380in}}%
\pgfpathlineto{\pgfqpoint{3.288188in}{2.094849in}}%
\pgfpathlineto{\pgfqpoint{3.289659in}{2.053371in}}%
\pgfpathlineto{\pgfqpoint{3.292085in}{0.702108in}}%
\pgfpathlineto{\pgfqpoint{3.303169in}{0.702108in}}%
\pgfpathlineto{\pgfqpoint{3.304716in}{0.700400in}}%
\pgfpathlineto{\pgfqpoint{3.308776in}{0.700400in}}%
\pgfpathlineto{\pgfqpoint{3.310170in}{0.662386in}}%
\pgfpathlineto{\pgfqpoint{3.310414in}{0.666124in}}%
\pgfpathlineto{\pgfqpoint{3.333576in}{0.666268in}}%
\pgfpathlineto{\pgfqpoint{3.335171in}{0.703583in}}%
\pgfpathlineto{\pgfqpoint{3.350066in}{0.703583in}}%
\pgfpathlineto{\pgfqpoint{3.351618in}{0.700400in}}%
\pgfpathlineto{\pgfqpoint{3.381194in}{0.700400in}}%
\pgfpathlineto{\pgfqpoint{3.382856in}{0.657775in}}%
\pgfpathlineto{\pgfqpoint{3.390267in}{0.657775in}}%
\pgfpathlineto{\pgfqpoint{3.391858in}{0.703614in}}%
\pgfpathlineto{\pgfqpoint{3.418716in}{0.704874in}}%
\pgfpathlineto{\pgfqpoint{3.420679in}{0.837791in}}%
\pgfpathlineto{\pgfqpoint{3.466277in}{0.836761in}}%
\pgfpathlineto{\pgfqpoint{3.469075in}{0.700400in}}%
\pgfpathlineto{\pgfqpoint{3.495408in}{0.700399in}}%
\pgfpathlineto{\pgfqpoint{3.497046in}{0.666053in}}%
\pgfpathlineto{\pgfqpoint{3.507276in}{0.666053in}}%
\pgfpathlineto{\pgfqpoint{3.508828in}{0.657775in}}%
\pgfpathlineto{\pgfqpoint{3.555796in}{0.657775in}}%
\pgfpathlineto{\pgfqpoint{3.557387in}{0.703614in}}%
\pgfpathlineto{\pgfqpoint{3.584250in}{0.704905in}}%
\pgfpathlineto{\pgfqpoint{3.586213in}{0.837791in}}%
\pgfpathlineto{\pgfqpoint{3.631811in}{0.836761in}}%
\pgfpathlineto{\pgfqpoint{3.635005in}{0.657572in}}%
\pgfpathlineto{\pgfqpoint{3.670737in}{0.657775in}}%
\pgfpathlineto{\pgfqpoint{3.672322in}{0.700400in}}%
\pgfpathlineto{\pgfqpoint{3.718326in}{0.700400in}}%
\pgfpathlineto{\pgfqpoint{3.719964in}{0.666124in}}%
\pgfpathlineto{\pgfqpoint{3.749034in}{0.666124in}}%
\pgfpathlineto{\pgfqpoint{3.750829in}{0.847513in}}%
\pgfpathlineto{\pgfqpoint{3.788853in}{0.846018in}}%
\pgfpathlineto{\pgfqpoint{3.790463in}{0.838837in}}%
\pgfpathlineto{\pgfqpoint{3.796752in}{0.837791in}}%
\pgfpathlineto{\pgfqpoint{3.799570in}{0.700400in}}%
\pgfpathlineto{\pgfqpoint{3.832746in}{0.700399in}}%
\pgfpathlineto{\pgfqpoint{3.834384in}{0.665980in}}%
\pgfpathlineto{\pgfqpoint{3.896009in}{0.666053in}}%
\pgfpathlineto{\pgfqpoint{3.897561in}{0.657775in}}%
\pgfpathlineto{\pgfqpoint{3.903990in}{0.657775in}}%
\pgfpathlineto{\pgfqpoint{3.905537in}{0.666124in}}%
\pgfpathlineto{\pgfqpoint{3.920327in}{0.666124in}}%
\pgfpathlineto{\pgfqpoint{3.922132in}{0.846511in}}%
\pgfpathlineto{\pgfqpoint{3.968064in}{0.845029in}}%
\pgfpathlineto{\pgfqpoint{3.971407in}{0.657775in}}%
\pgfpathlineto{\pgfqpoint{4.006762in}{0.657776in}}%
\pgfpathlineto{\pgfqpoint{4.008347in}{0.700367in}}%
\pgfpathlineto{\pgfqpoint{4.054475in}{0.700367in}}%
\pgfpathlineto{\pgfqpoint{4.056137in}{0.657775in}}%
\pgfpathlineto{\pgfqpoint{4.085144in}{0.657775in}}%
\pgfpathlineto{\pgfqpoint{4.086892in}{0.847513in}}%
\pgfpathlineto{\pgfqpoint{4.132892in}{0.846038in}}%
\pgfpathlineto{\pgfqpoint{4.136096in}{0.637273in}}%
\pgfpathlineto{\pgfqpoint{4.136254in}{0.657572in}}%
\pgfpathlineto{\pgfqpoint{4.226752in}{0.657776in}}%
\pgfpathlineto{\pgfqpoint{4.228333in}{0.697047in}}%
\pgfpathlineto{\pgfqpoint{4.255282in}{0.698374in}}%
\pgfpathlineto{\pgfqpoint{4.257231in}{0.839856in}}%
\pgfpathlineto{\pgfqpoint{4.302848in}{0.838325in}}%
\pgfpathlineto{\pgfqpoint{4.306071in}{0.657775in}}%
\pgfpathlineto{\pgfqpoint{4.353751in}{0.657775in}}%
\pgfpathlineto{\pgfqpoint{4.355375in}{0.687763in}}%
\pgfpathlineto{\pgfqpoint{4.401494in}{0.687763in}}%
\pgfpathlineto{\pgfqpoint{4.403093in}{0.657775in}}%
\pgfpathlineto{\pgfqpoint{4.420663in}{0.657775in}}%
\pgfpathlineto{\pgfqpoint{4.422411in}{0.846511in}}%
\pgfpathlineto{\pgfqpoint{4.453964in}{0.845029in}}%
\pgfpathlineto{\pgfqpoint{4.455573in}{0.837791in}}%
\pgfpathlineto{\pgfqpoint{4.468405in}{0.836761in}}%
\pgfpathlineto{\pgfqpoint{4.471204in}{0.700400in}}%
\pgfpathlineto{\pgfqpoint{4.497217in}{0.700400in}}%
\pgfpathlineto{\pgfqpoint{4.498879in}{0.657674in}}%
\pgfpathlineto{\pgfqpoint{4.516071in}{0.657674in}}%
\pgfpathlineto{\pgfqpoint{4.517680in}{0.683050in}}%
\pgfpathlineto{\pgfqpoint{4.561631in}{0.683050in}}%
\pgfpathlineto{\pgfqpoint{4.563178in}{0.678914in}}%
\pgfpathlineto{\pgfqpoint{4.567032in}{0.679835in}}%
\pgfpathlineto{\pgfqpoint{4.568584in}{0.686748in}}%
\pgfpathlineto{\pgfqpoint{4.577653in}{0.687729in}}%
\pgfpathlineto{\pgfqpoint{4.579205in}{0.692368in}}%
\pgfpathlineto{\pgfqpoint{4.579267in}{0.693684in}}%
\pgfpathlineto{\pgfqpoint{4.580824in}{0.701781in}}%
\pgfpathlineto{\pgfqpoint{4.598886in}{0.702536in}}%
\pgfpathlineto{\pgfqpoint{4.600433in}{0.703651in}}%
\pgfpathlineto{\pgfqpoint{4.634823in}{0.705021in}}%
\pgfpathlineto{\pgfqpoint{4.636819in}{0.851727in}}%
\pgfpathlineto{\pgfqpoint{4.674680in}{0.851336in}}%
\pgfpathlineto{\pgfqpoint{4.676452in}{2.913602in}}%
\pgfpathlineto{\pgfqpoint{4.696586in}{2.914518in}}%
\pgfpathlineto{\pgfqpoint{4.698239in}{3.242820in}}%
\pgfpathlineto{\pgfqpoint{4.698363in}{3.137786in}}%
\pgfpathlineto{\pgfqpoint{4.698845in}{3.137786in}}%
\pgfpathlineto{\pgfqpoint{4.700402in}{3.125371in}}%
\pgfpathlineto{\pgfqpoint{4.739920in}{3.125371in}}%
\pgfpathlineto{\pgfqpoint{4.745016in}{0.637273in}}%
\pgfpathlineto{\pgfqpoint{4.758503in}{0.637273in}}%
\pgfpathlineto{\pgfqpoint{4.760117in}{2.913602in}}%
\pgfpathlineto{\pgfqpoint{4.767295in}{2.913602in}}%
\pgfpathlineto{\pgfqpoint{4.769233in}{3.060643in}}%
\pgfpathlineto{\pgfqpoint{4.769673in}{3.060643in}}%
\pgfpathlineto{\pgfqpoint{4.771297in}{3.001850in}}%
\pgfpathlineto{\pgfqpoint{4.810829in}{3.001850in}}%
\pgfpathlineto{\pgfqpoint{4.814870in}{0.712957in}}%
\pgfpathlineto{\pgfqpoint{4.823719in}{0.712957in}}%
\pgfpathlineto{\pgfqpoint{4.825309in}{0.696645in}}%
\pgfpathlineto{\pgfqpoint{4.825457in}{0.696645in}}%
\pgfpathlineto{\pgfqpoint{4.827052in}{0.673632in}}%
\pgfpathlineto{\pgfqpoint{4.827119in}{0.673632in}}%
\pgfpathlineto{\pgfqpoint{4.828681in}{0.657573in}}%
\pgfpathlineto{\pgfqpoint{4.943932in}{0.657775in}}%
\pgfpathlineto{\pgfqpoint{4.945680in}{0.848491in}}%
\pgfpathlineto{\pgfqpoint{4.977672in}{0.847023in}}%
\pgfpathlineto{\pgfqpoint{4.979281in}{0.839856in}}%
\pgfpathlineto{\pgfqpoint{4.991684in}{0.838325in}}%
\pgfpathlineto{\pgfqpoint{4.994539in}{0.697047in}}%
\pgfpathlineto{\pgfqpoint{5.021574in}{0.696977in}}%
\pgfpathlineto{\pgfqpoint{5.023222in}{0.657674in}}%
\pgfpathlineto{\pgfqpoint{5.073491in}{0.657573in}}%
\pgfpathlineto{\pgfqpoint{5.075047in}{0.637273in}}%
\pgfpathlineto{\pgfqpoint{5.082999in}{0.637273in}}%
\pgfpathlineto{\pgfqpoint{5.084546in}{0.657573in}}%
\pgfpathlineto{\pgfqpoint{5.094413in}{0.657573in}}%
\pgfpathlineto{\pgfqpoint{5.095960in}{0.665981in}}%
\pgfpathlineto{\pgfqpoint{5.105970in}{0.666125in}}%
\pgfpathlineto{\pgfqpoint{5.107765in}{0.847513in}}%
\pgfpathlineto{\pgfqpoint{5.125158in}{0.846018in}}%
\pgfpathlineto{\pgfqpoint{5.126725in}{0.843507in}}%
\pgfpathlineto{\pgfqpoint{5.153612in}{0.841992in}}%
\pgfpathlineto{\pgfqpoint{5.156515in}{0.695199in}}%
\pgfpathlineto{\pgfqpoint{5.168464in}{0.695199in}}%
\pgfpathlineto{\pgfqpoint{5.170083in}{0.665981in}}%
\pgfpathlineto{\pgfqpoint{5.184568in}{0.667135in}}%
\pgfpathlineto{\pgfqpoint{5.186134in}{0.684479in}}%
\pgfpathlineto{\pgfqpoint{5.186325in}{0.684479in}}%
\pgfpathlineto{\pgfqpoint{5.187901in}{0.700340in}}%
\pgfpathlineto{\pgfqpoint{5.188216in}{0.700340in}}%
\pgfpathlineto{\pgfqpoint{5.188636in}{0.703785in}}%
\pgfpathlineto{\pgfqpoint{5.188636in}{0.703785in}}%
\pgfusepath{stroke}%
\end{pgfscope}%
\begin{pgfscope}%
\pgfpathrectangle{\pgfqpoint{0.750000in}{0.500000in}}{\pgfqpoint{4.650000in}{3.020000in}}%
\pgfusepath{clip}%
\pgfsetrectcap%
\pgfsetroundjoin%
\pgfsetlinewidth{1.505625pt}%
\definecolor{currentstroke}{rgb}{1.000000,0.000000,0.000000}%
\pgfsetstrokecolor{currentstroke}%
\pgfsetdash{}{0pt}%
\pgfpathmoveto{\pgfqpoint{0.961364in}{0.960794in}}%
\pgfpathlineto{\pgfqpoint{0.964821in}{0.793899in}}%
\pgfpathlineto{\pgfqpoint{0.968838in}{0.649979in}}%
\pgfpathlineto{\pgfqpoint{0.974220in}{0.651522in}}%
\pgfpathlineto{\pgfqpoint{0.982768in}{0.653853in}}%
\pgfpathlineto{\pgfqpoint{0.991909in}{0.647804in}}%
\pgfpathlineto{\pgfqpoint{0.994956in}{0.646804in}}%
\pgfpathlineto{\pgfqpoint{1.015725in}{0.645261in}}%
\pgfpathlineto{\pgfqpoint{1.017731in}{0.646016in}}%
\pgfpathlineto{\pgfqpoint{1.023103in}{0.649189in}}%
\pgfpathlineto{\pgfqpoint{1.027549in}{0.648797in}}%
\pgfpathlineto{\pgfqpoint{1.062125in}{0.647259in}}%
\pgfpathlineto{\pgfqpoint{1.064718in}{0.644003in}}%
\pgfpathlineto{\pgfqpoint{1.065631in}{0.644856in}}%
\pgfpathlineto{\pgfqpoint{1.080965in}{0.657335in}}%
\pgfpathlineto{\pgfqpoint{1.085827in}{0.655792in}}%
\pgfpathlineto{\pgfqpoint{1.087952in}{0.656488in}}%
\pgfpathlineto{\pgfqpoint{1.093664in}{0.661633in}}%
\pgfpathlineto{\pgfqpoint{1.109323in}{0.660093in}}%
\pgfpathlineto{\pgfqpoint{1.115469in}{0.654797in}}%
\pgfpathlineto{\pgfqpoint{1.115890in}{0.655946in}}%
\pgfpathlineto{\pgfqpoint{1.120871in}{0.673949in}}%
\pgfpathlineto{\pgfqpoint{1.122088in}{0.672471in}}%
\pgfpathlineto{\pgfqpoint{1.124586in}{0.672624in}}%
\pgfpathlineto{\pgfqpoint{1.130054in}{0.673532in}}%
\pgfpathlineto{\pgfqpoint{1.138550in}{0.664964in}}%
\pgfpathlineto{\pgfqpoint{1.146731in}{0.666503in}}%
\pgfpathlineto{\pgfqpoint{1.151568in}{0.672134in}}%
\pgfpathlineto{\pgfqpoint{1.156124in}{0.672579in}}%
\pgfpathlineto{\pgfqpoint{1.159467in}{0.669295in}}%
\pgfpathlineto{\pgfqpoint{1.165466in}{0.651512in}}%
\pgfpathlineto{\pgfqpoint{1.170943in}{0.653054in}}%
\pgfpathlineto{\pgfqpoint{1.176025in}{0.653745in}}%
\pgfpathlineto{\pgfqpoint{1.182200in}{0.655288in}}%
\pgfpathlineto{\pgfqpoint{1.187310in}{0.656008in}}%
\pgfpathlineto{\pgfqpoint{1.188460in}{0.654470in}}%
\pgfpathlineto{\pgfqpoint{1.193098in}{0.648300in}}%
\pgfpathlineto{\pgfqpoint{1.193274in}{0.648418in}}%
\pgfpathlineto{\pgfqpoint{1.196116in}{0.648468in}}%
\pgfpathlineto{\pgfqpoint{1.204301in}{0.646311in}}%
\pgfpathlineto{\pgfqpoint{1.214106in}{0.644801in}}%
\pgfpathlineto{\pgfqpoint{1.218552in}{0.642714in}}%
\pgfpathlineto{\pgfqpoint{1.226647in}{0.644255in}}%
\pgfpathlineto{\pgfqpoint{1.231718in}{0.644942in}}%
\pgfpathlineto{\pgfqpoint{1.237081in}{0.643400in}}%
\pgfpathlineto{\pgfqpoint{1.242015in}{0.644022in}}%
\pgfpathlineto{\pgfqpoint{1.247769in}{0.642479in}}%
\pgfpathlineto{\pgfqpoint{1.251776in}{0.642282in}}%
\pgfpathlineto{\pgfqpoint{1.258180in}{0.649326in}}%
\pgfpathlineto{\pgfqpoint{1.280912in}{0.650858in}}%
\pgfpathlineto{\pgfqpoint{1.287183in}{0.670369in}}%
\pgfpathlineto{\pgfqpoint{1.296896in}{0.668828in}}%
\pgfpathlineto{\pgfqpoint{1.301128in}{0.665146in}}%
\pgfpathlineto{\pgfqpoint{1.306424in}{0.663605in}}%
\pgfpathlineto{\pgfqpoint{1.310206in}{0.663523in}}%
\pgfpathlineto{\pgfqpoint{1.321759in}{0.667269in}}%
\pgfpathlineto{\pgfqpoint{1.323516in}{0.665731in}}%
\pgfpathlineto{\pgfqpoint{1.328631in}{0.645366in}}%
\pgfpathlineto{\pgfqpoint{1.330469in}{0.646094in}}%
\pgfpathlineto{\pgfqpoint{1.345427in}{0.647635in}}%
\pgfpathlineto{\pgfqpoint{1.352160in}{0.649125in}}%
\pgfpathlineto{\pgfqpoint{1.359720in}{0.643853in}}%
\pgfpathlineto{\pgfqpoint{1.363216in}{0.642651in}}%
\pgfpathlineto{\pgfqpoint{1.367509in}{0.644188in}}%
\pgfpathlineto{\pgfqpoint{1.371970in}{0.648003in}}%
\pgfpathlineto{\pgfqpoint{1.374773in}{0.648373in}}%
\pgfpathlineto{\pgfqpoint{1.383174in}{0.650632in}}%
\pgfpathlineto{\pgfqpoint{1.388723in}{0.648769in}}%
\pgfpathlineto{\pgfqpoint{1.392639in}{0.650311in}}%
\pgfpathlineto{\pgfqpoint{1.397706in}{0.650996in}}%
\pgfpathlineto{\pgfqpoint{1.403790in}{0.652538in}}%
\pgfpathlineto{\pgfqpoint{1.408867in}{0.653255in}}%
\pgfpathlineto{\pgfqpoint{1.410500in}{0.651713in}}%
\pgfpathlineto{\pgfqpoint{1.415013in}{0.647830in}}%
\pgfpathlineto{\pgfqpoint{1.417883in}{0.647494in}}%
\pgfpathlineto{\pgfqpoint{1.425715in}{0.645504in}}%
\pgfpathlineto{\pgfqpoint{1.445415in}{0.647092in}}%
\pgfpathlineto{\pgfqpoint{1.453361in}{0.668943in}}%
\pgfpathlineto{\pgfqpoint{1.457855in}{0.667401in}}%
\pgfpathlineto{\pgfqpoint{1.467602in}{0.664452in}}%
\pgfpathlineto{\pgfqpoint{1.480033in}{0.662910in}}%
\pgfpathlineto{\pgfqpoint{1.482240in}{0.663561in}}%
\pgfpathlineto{\pgfqpoint{1.487593in}{0.668157in}}%
\pgfpathlineto{\pgfqpoint{1.488429in}{0.666615in}}%
\pgfpathlineto{\pgfqpoint{1.493529in}{0.645139in}}%
\pgfpathlineto{\pgfqpoint{1.495487in}{0.646064in}}%
\pgfpathlineto{\pgfqpoint{1.501022in}{0.646985in}}%
\pgfpathlineto{\pgfqpoint{1.508353in}{0.648527in}}%
\pgfpathlineto{\pgfqpoint{1.513434in}{0.649219in}}%
\pgfpathlineto{\pgfqpoint{1.518616in}{0.650761in}}%
\pgfpathlineto{\pgfqpoint{1.523821in}{0.652395in}}%
\pgfpathlineto{\pgfqpoint{1.526176in}{0.648961in}}%
\pgfpathlineto{\pgfqpoint{1.530149in}{0.645668in}}%
\pgfpathlineto{\pgfqpoint{1.546095in}{0.648430in}}%
\pgfpathlineto{\pgfqpoint{1.550484in}{0.646887in}}%
\pgfpathlineto{\pgfqpoint{1.555775in}{0.644002in}}%
\pgfpathlineto{\pgfqpoint{1.575083in}{0.645544in}}%
\pgfpathlineto{\pgfqpoint{1.578173in}{0.645297in}}%
\pgfpathlineto{\pgfqpoint{1.585795in}{0.643407in}}%
\pgfpathlineto{\pgfqpoint{1.596144in}{0.644950in}}%
\pgfpathlineto{\pgfqpoint{1.605428in}{0.647677in}}%
\pgfpathlineto{\pgfqpoint{1.608226in}{0.648048in}}%
\pgfpathlineto{\pgfqpoint{1.619358in}{0.656464in}}%
\pgfpathlineto{\pgfqpoint{1.624798in}{0.675664in}}%
\pgfpathlineto{\pgfqpoint{1.625333in}{0.675418in}}%
\pgfpathlineto{\pgfqpoint{1.629507in}{0.671715in}}%
\pgfpathlineto{\pgfqpoint{1.632329in}{0.671151in}}%
\pgfpathlineto{\pgfqpoint{1.640381in}{0.669609in}}%
\pgfpathlineto{\pgfqpoint{1.652062in}{0.665736in}}%
\pgfpathlineto{\pgfqpoint{1.655056in}{0.667535in}}%
\pgfpathlineto{\pgfqpoint{1.659870in}{0.671615in}}%
\pgfpathlineto{\pgfqpoint{1.661547in}{0.670074in}}%
\pgfpathlineto{\pgfqpoint{1.668261in}{0.651928in}}%
\pgfpathlineto{\pgfqpoint{1.671895in}{0.653471in}}%
\pgfpathlineto{\pgfqpoint{1.676981in}{0.654193in}}%
\pgfpathlineto{\pgfqpoint{1.684618in}{0.655735in}}%
\pgfpathlineto{\pgfqpoint{1.689728in}{0.656455in}}%
\pgfpathlineto{\pgfqpoint{1.697789in}{0.654914in}}%
\pgfpathlineto{\pgfqpoint{1.701910in}{0.650958in}}%
\pgfpathlineto{\pgfqpoint{1.706987in}{0.649418in}}%
\pgfpathlineto{\pgfqpoint{1.711237in}{0.647278in}}%
\pgfpathlineto{\pgfqpoint{1.714489in}{0.647107in}}%
\pgfpathlineto{\pgfqpoint{1.716185in}{0.648004in}}%
\pgfpathlineto{\pgfqpoint{1.723721in}{0.652801in}}%
\pgfpathlineto{\pgfqpoint{1.755379in}{0.651259in}}%
\pgfpathlineto{\pgfqpoint{1.760278in}{0.647246in}}%
\pgfpathlineto{\pgfqpoint{1.764968in}{0.641775in}}%
\pgfpathlineto{\pgfqpoint{1.783947in}{0.640232in}}%
\pgfpathlineto{\pgfqpoint{1.789047in}{0.639504in}}%
\pgfpathlineto{\pgfqpoint{1.793837in}{0.641036in}}%
\pgfpathlineto{\pgfqpoint{1.801937in}{0.665392in}}%
\pgfpathlineto{\pgfqpoint{1.814477in}{0.669604in}}%
\pgfpathlineto{\pgfqpoint{1.819540in}{0.668919in}}%
\pgfpathlineto{\pgfqpoint{1.824841in}{0.669714in}}%
\pgfpathlineto{\pgfqpoint{1.830992in}{0.668172in}}%
\pgfpathlineto{\pgfqpoint{1.836073in}{0.667465in}}%
\pgfpathlineto{\pgfqpoint{1.836837in}{0.665922in}}%
\pgfpathlineto{\pgfqpoint{1.844168in}{0.644361in}}%
\pgfpathlineto{\pgfqpoint{1.848915in}{0.643827in}}%
\pgfpathlineto{\pgfqpoint{1.855882in}{0.644038in}}%
\pgfpathlineto{\pgfqpoint{1.864383in}{0.645579in}}%
\pgfpathlineto{\pgfqpoint{1.869331in}{0.646235in}}%
\pgfpathlineto{\pgfqpoint{1.875386in}{0.647778in}}%
\pgfpathlineto{\pgfqpoint{1.883428in}{0.649867in}}%
\pgfpathlineto{\pgfqpoint{1.891662in}{0.647687in}}%
\pgfpathlineto{\pgfqpoint{1.896991in}{0.648215in}}%
\pgfpathlineto{\pgfqpoint{1.899794in}{0.648270in}}%
\pgfpathlineto{\pgfqpoint{1.908529in}{0.649805in}}%
\pgfpathlineto{\pgfqpoint{1.910936in}{0.649796in}}%
\pgfpathlineto{\pgfqpoint{1.916290in}{0.651674in}}%
\pgfpathlineto{\pgfqpoint{1.921495in}{0.650921in}}%
\pgfpathlineto{\pgfqpoint{1.922422in}{0.653411in}}%
\pgfpathlineto{\pgfqpoint{1.928401in}{0.665985in}}%
\pgfpathlineto{\pgfqpoint{1.931366in}{0.669191in}}%
\pgfpathlineto{\pgfqpoint{1.932661in}{0.722594in}}%
\pgfpathlineto{\pgfqpoint{1.936877in}{0.924169in}}%
\pgfpathlineto{\pgfqpoint{1.937560in}{0.900420in}}%
\pgfpathlineto{\pgfqpoint{1.939666in}{0.875450in}}%
\pgfpathlineto{\pgfqpoint{1.945158in}{0.873188in}}%
\pgfpathlineto{\pgfqpoint{1.945330in}{0.875021in}}%
\pgfpathlineto{\pgfqpoint{1.951501in}{0.980128in}}%
\pgfpathlineto{\pgfqpoint{1.953358in}{1.071476in}}%
\pgfpathlineto{\pgfqpoint{1.956859in}{1.208554in}}%
\pgfpathlineto{\pgfqpoint{1.957088in}{1.208524in}}%
\pgfpathlineto{\pgfqpoint{1.964390in}{1.206253in}}%
\pgfpathlineto{\pgfqpoint{1.965827in}{1.202256in}}%
\pgfpathlineto{\pgfqpoint{1.971157in}{1.189739in}}%
\pgfpathlineto{\pgfqpoint{1.974323in}{1.186459in}}%
\pgfpathlineto{\pgfqpoint{1.975498in}{1.140387in}}%
\pgfpathlineto{\pgfqpoint{1.980088in}{0.978972in}}%
\pgfpathlineto{\pgfqpoint{1.988287in}{0.977441in}}%
\pgfpathlineto{\pgfqpoint{1.994520in}{0.867439in}}%
\pgfpathlineto{\pgfqpoint{1.996459in}{0.771149in}}%
\pgfpathlineto{\pgfqpoint{2.000265in}{0.637273in}}%
\pgfpathlineto{\pgfqpoint{2.014850in}{0.638813in}}%
\pgfpathlineto{\pgfqpoint{2.023011in}{0.688204in}}%
\pgfpathlineto{\pgfqpoint{2.025170in}{0.689562in}}%
\pgfpathlineto{\pgfqpoint{2.033957in}{0.689604in}}%
\pgfpathlineto{\pgfqpoint{2.040796in}{0.691143in}}%
\pgfpathlineto{\pgfqpoint{2.046488in}{0.706783in}}%
\pgfpathlineto{\pgfqpoint{2.048012in}{0.767430in}}%
\pgfpathlineto{\pgfqpoint{2.052697in}{0.977259in}}%
\pgfpathlineto{\pgfqpoint{2.053580in}{0.977055in}}%
\pgfpathlineto{\pgfqpoint{2.058399in}{0.975522in}}%
\pgfpathlineto{\pgfqpoint{2.067196in}{0.944025in}}%
\pgfpathlineto{\pgfqpoint{2.079593in}{0.942446in}}%
\pgfpathlineto{\pgfqpoint{2.082817in}{0.939237in}}%
\pgfpathlineto{\pgfqpoint{2.090076in}{0.892659in}}%
\pgfpathlineto{\pgfqpoint{2.098142in}{0.646302in}}%
\pgfpathlineto{\pgfqpoint{2.106046in}{0.644255in}}%
\pgfpathlineto{\pgfqpoint{2.110840in}{0.645797in}}%
\pgfpathlineto{\pgfqpoint{2.114026in}{0.649004in}}%
\pgfpathlineto{\pgfqpoint{2.120248in}{0.669735in}}%
\pgfpathlineto{\pgfqpoint{2.128596in}{0.671276in}}%
\pgfpathlineto{\pgfqpoint{2.133338in}{0.671823in}}%
\pgfpathlineto{\pgfqpoint{2.138830in}{0.666822in}}%
\pgfpathlineto{\pgfqpoint{2.143916in}{0.668360in}}%
\pgfpathlineto{\pgfqpoint{2.156768in}{0.666817in}}%
\pgfpathlineto{\pgfqpoint{2.161859in}{0.646445in}}%
\pgfpathlineto{\pgfqpoint{2.163549in}{0.647163in}}%
\pgfpathlineto{\pgfqpoint{2.181611in}{0.645621in}}%
\pgfpathlineto{\pgfqpoint{2.185154in}{0.645107in}}%
\pgfpathlineto{\pgfqpoint{2.190861in}{0.646097in}}%
\pgfpathlineto{\pgfqpoint{2.208436in}{0.639509in}}%
\pgfpathlineto{\pgfqpoint{2.212762in}{0.641052in}}%
\pgfpathlineto{\pgfqpoint{2.226106in}{0.648100in}}%
\pgfpathlineto{\pgfqpoint{2.231387in}{0.647313in}}%
\pgfpathlineto{\pgfqpoint{2.237109in}{0.648339in}}%
\pgfpathlineto{\pgfqpoint{2.243169in}{0.646796in}}%
\pgfpathlineto{\pgfqpoint{2.248279in}{0.646064in}}%
\pgfpathlineto{\pgfqpoint{2.260810in}{0.644523in}}%
\pgfpathlineto{\pgfqpoint{2.267377in}{0.639495in}}%
\pgfpathlineto{\pgfqpoint{2.276556in}{0.637954in}}%
\pgfpathlineto{\pgfqpoint{2.281613in}{0.637273in}}%
\pgfpathlineto{\pgfqpoint{2.284794in}{0.638809in}}%
\pgfpathlineto{\pgfqpoint{2.292922in}{0.663174in}}%
\pgfpathlineto{\pgfqpoint{2.299350in}{0.664536in}}%
\pgfpathlineto{\pgfqpoint{2.305176in}{0.662993in}}%
\pgfpathlineto{\pgfqpoint{2.310324in}{0.662255in}}%
\pgfpathlineto{\pgfqpoint{2.314708in}{0.663798in}}%
\pgfpathlineto{\pgfqpoint{2.320057in}{0.668082in}}%
\pgfpathlineto{\pgfqpoint{2.328133in}{0.666552in}}%
\pgfpathlineto{\pgfqpoint{2.334642in}{0.646243in}}%
\pgfpathlineto{\pgfqpoint{2.339246in}{0.644701in}}%
\pgfpathlineto{\pgfqpoint{2.344298in}{0.644010in}}%
\pgfpathlineto{\pgfqpoint{2.351944in}{0.645553in}}%
\pgfpathlineto{\pgfqpoint{2.356500in}{0.646011in}}%
\pgfpathlineto{\pgfqpoint{2.363344in}{0.641273in}}%
\pgfpathlineto{\pgfqpoint{2.368033in}{0.641792in}}%
\pgfpathlineto{\pgfqpoint{2.373029in}{0.643335in}}%
\pgfpathlineto{\pgfqpoint{2.389347in}{0.652461in}}%
\pgfpathlineto{\pgfqpoint{2.400030in}{0.654003in}}%
\pgfpathlineto{\pgfqpoint{2.406516in}{0.654987in}}%
\pgfpathlineto{\pgfqpoint{2.434506in}{0.652591in}}%
\pgfpathlineto{\pgfqpoint{2.436975in}{0.651858in}}%
\pgfpathlineto{\pgfqpoint{2.441870in}{0.649278in}}%
\pgfpathlineto{\pgfqpoint{2.448326in}{0.647735in}}%
\pgfpathlineto{\pgfqpoint{2.457491in}{0.646044in}}%
\pgfpathlineto{\pgfqpoint{2.467558in}{0.648685in}}%
\pgfpathlineto{\pgfqpoint{2.471297in}{0.648876in}}%
\pgfpathlineto{\pgfqpoint{2.479459in}{0.645770in}}%
\pgfpathlineto{\pgfqpoint{2.482659in}{0.645293in}}%
\pgfpathlineto{\pgfqpoint{2.485156in}{0.646025in}}%
\pgfpathlineto{\pgfqpoint{2.494999in}{0.649349in}}%
\pgfpathlineto{\pgfqpoint{2.500357in}{0.650892in}}%
\pgfpathlineto{\pgfqpoint{2.503891in}{0.649499in}}%
\pgfpathlineto{\pgfqpoint{2.506952in}{0.648770in}}%
\pgfpathlineto{\pgfqpoint{2.519976in}{0.652324in}}%
\pgfpathlineto{\pgfqpoint{2.522736in}{0.653446in}}%
\pgfpathlineto{\pgfqpoint{2.522793in}{0.653376in}}%
\pgfpathlineto{\pgfqpoint{2.525086in}{0.653095in}}%
\pgfpathlineto{\pgfqpoint{2.531046in}{0.651554in}}%
\pgfpathlineto{\pgfqpoint{2.536136in}{0.648978in}}%
\pgfpathlineto{\pgfqpoint{2.546777in}{0.650520in}}%
\pgfpathlineto{\pgfqpoint{2.553095in}{0.651176in}}%
\pgfpathlineto{\pgfqpoint{2.562412in}{0.650688in}}%
\pgfpathlineto{\pgfqpoint{2.569747in}{0.648115in}}%
\pgfpathlineto{\pgfqpoint{2.579122in}{0.649882in}}%
\pgfpathlineto{\pgfqpoint{2.588564in}{0.648342in}}%
\pgfpathlineto{\pgfqpoint{2.597083in}{0.644789in}}%
\pgfpathlineto{\pgfqpoint{2.604065in}{0.643161in}}%
\pgfpathlineto{\pgfqpoint{2.615498in}{0.641619in}}%
\pgfpathlineto{\pgfqpoint{2.621200in}{0.640423in}}%
\pgfpathlineto{\pgfqpoint{2.626797in}{0.641961in}}%
\pgfpathlineto{\pgfqpoint{2.632447in}{0.647600in}}%
\pgfpathlineto{\pgfqpoint{2.638632in}{0.649141in}}%
\pgfpathlineto{\pgfqpoint{2.643412in}{0.650977in}}%
\pgfpathlineto{\pgfqpoint{2.646755in}{0.649434in}}%
\pgfpathlineto{\pgfqpoint{2.651096in}{0.649613in}}%
\pgfpathlineto{\pgfqpoint{2.664907in}{0.655017in}}%
\pgfpathlineto{\pgfqpoint{2.668102in}{0.655109in}}%
\pgfpathlineto{\pgfqpoint{2.669038in}{0.653631in}}%
\pgfpathlineto{\pgfqpoint{2.674148in}{0.646473in}}%
\pgfpathlineto{\pgfqpoint{2.682954in}{0.644930in}}%
\pgfpathlineto{\pgfqpoint{2.686312in}{0.644032in}}%
\pgfpathlineto{\pgfqpoint{2.709732in}{0.650908in}}%
\pgfpathlineto{\pgfqpoint{2.716456in}{0.650104in}}%
\pgfpathlineto{\pgfqpoint{2.720935in}{0.651647in}}%
\pgfpathlineto{\pgfqpoint{2.729737in}{0.654439in}}%
\pgfpathlineto{\pgfqpoint{2.733538in}{0.654660in}}%
\pgfpathlineto{\pgfqpoint{2.739288in}{0.653120in}}%
\pgfpathlineto{\pgfqpoint{2.741509in}{0.653044in}}%
\pgfpathlineto{\pgfqpoint{2.744159in}{0.652366in}}%
\pgfpathlineto{\pgfqpoint{2.754312in}{0.648879in}}%
\pgfpathlineto{\pgfqpoint{2.763300in}{0.653508in}}%
\pgfpathlineto{\pgfqpoint{2.769876in}{0.654258in}}%
\pgfpathlineto{\pgfqpoint{2.775841in}{0.651826in}}%
\pgfpathlineto{\pgfqpoint{2.792852in}{0.653362in}}%
\pgfpathlineto{\pgfqpoint{2.796271in}{0.669068in}}%
\pgfpathlineto{\pgfqpoint{2.798239in}{0.675582in}}%
\pgfpathlineto{\pgfqpoint{2.799151in}{0.675028in}}%
\pgfpathlineto{\pgfqpoint{2.803931in}{0.673854in}}%
\pgfpathlineto{\pgfqpoint{2.806386in}{0.673388in}}%
\pgfpathlineto{\pgfqpoint{2.820789in}{0.669789in}}%
\pgfpathlineto{\pgfqpoint{2.832418in}{0.672377in}}%
\pgfpathlineto{\pgfqpoint{2.835422in}{0.670576in}}%
\pgfpathlineto{\pgfqpoint{2.836411in}{0.668096in}}%
\pgfpathlineto{\pgfqpoint{2.842996in}{0.646492in}}%
\pgfpathlineto{\pgfqpoint{2.850480in}{0.644379in}}%
\pgfpathlineto{\pgfqpoint{2.854128in}{0.645917in}}%
\pgfpathlineto{\pgfqpoint{2.857466in}{0.647578in}}%
\pgfpathlineto{\pgfqpoint{2.860327in}{0.648176in}}%
\pgfpathlineto{\pgfqpoint{2.863775in}{0.648219in}}%
\pgfpathlineto{\pgfqpoint{2.867696in}{0.646831in}}%
\pgfpathlineto{\pgfqpoint{2.871540in}{0.646587in}}%
\pgfpathlineto{\pgfqpoint{2.875366in}{0.646352in}}%
\pgfpathlineto{\pgfqpoint{2.895528in}{0.649020in}}%
\pgfpathlineto{\pgfqpoint{2.906780in}{0.642372in}}%
\pgfpathlineto{\pgfqpoint{2.911546in}{0.645746in}}%
\pgfpathlineto{\pgfqpoint{2.916011in}{0.647501in}}%
\pgfpathlineto{\pgfqpoint{2.931675in}{0.643445in}}%
\pgfpathlineto{\pgfqpoint{2.934679in}{0.644987in}}%
\pgfpathlineto{\pgfqpoint{2.948901in}{0.648518in}}%
\pgfpathlineto{\pgfqpoint{2.954899in}{0.644695in}}%
\pgfpathlineto{\pgfqpoint{2.960769in}{0.642813in}}%
\pgfpathlineto{\pgfqpoint{2.968582in}{0.647295in}}%
\pgfpathlineto{\pgfqpoint{2.973128in}{0.648827in}}%
\pgfpathlineto{\pgfqpoint{2.980974in}{0.650369in}}%
\pgfpathlineto{\pgfqpoint{2.989361in}{0.651780in}}%
\pgfpathlineto{\pgfqpoint{2.994872in}{0.649679in}}%
\pgfpathlineto{\pgfqpoint{2.998387in}{0.650030in}}%
\pgfpathlineto{\pgfqpoint{3.008100in}{0.654047in}}%
\pgfpathlineto{\pgfqpoint{3.017389in}{0.651626in}}%
\pgfpathlineto{\pgfqpoint{3.020627in}{0.651459in}}%
\pgfpathlineto{\pgfqpoint{3.024653in}{0.647656in}}%
\pgfpathlineto{\pgfqpoint{3.028153in}{0.646421in}}%
\pgfpathlineto{\pgfqpoint{3.039457in}{0.649576in}}%
\pgfpathlineto{\pgfqpoint{3.042662in}{0.649862in}}%
\pgfpathlineto{\pgfqpoint{3.051530in}{0.648278in}}%
\pgfpathlineto{\pgfqpoint{3.057012in}{0.649174in}}%
\pgfpathlineto{\pgfqpoint{3.060642in}{0.647631in}}%
\pgfpathlineto{\pgfqpoint{3.065766in}{0.644904in}}%
\pgfpathlineto{\pgfqpoint{3.071554in}{0.645948in}}%
\pgfpathlineto{\pgfqpoint{3.082729in}{0.638933in}}%
\pgfpathlineto{\pgfqpoint{3.087744in}{0.637441in}}%
\pgfpathlineto{\pgfqpoint{3.095027in}{0.640466in}}%
\pgfpathlineto{\pgfqpoint{3.101947in}{0.642008in}}%
\pgfpathlineto{\pgfqpoint{3.107004in}{0.642676in}}%
\pgfpathlineto{\pgfqpoint{3.112845in}{0.644217in}}%
\pgfpathlineto{\pgfqpoint{3.121493in}{0.650003in}}%
\pgfpathlineto{\pgfqpoint{3.127750in}{0.671895in}}%
\pgfpathlineto{\pgfqpoint{3.132788in}{0.670352in}}%
\pgfpathlineto{\pgfqpoint{3.135405in}{0.669910in}}%
\pgfpathlineto{\pgfqpoint{3.142989in}{0.675193in}}%
\pgfpathlineto{\pgfqpoint{3.163839in}{0.705298in}}%
\pgfpathlineto{\pgfqpoint{3.164966in}{0.702590in}}%
\pgfpathlineto{\pgfqpoint{3.169236in}{0.690508in}}%
\pgfpathlineto{\pgfqpoint{3.170119in}{0.691893in}}%
\pgfpathlineto{\pgfqpoint{3.176046in}{0.705019in}}%
\pgfpathlineto{\pgfqpoint{3.180836in}{0.728616in}}%
\pgfpathlineto{\pgfqpoint{3.188195in}{0.762171in}}%
\pgfpathlineto{\pgfqpoint{3.197465in}{0.804450in}}%
\pgfpathlineto{\pgfqpoint{3.203831in}{0.825726in}}%
\pgfpathlineto{\pgfqpoint{3.218559in}{0.855009in}}%
\pgfpathlineto{\pgfqpoint{3.220798in}{0.854397in}}%
\pgfpathlineto{\pgfqpoint{3.225507in}{0.850096in}}%
\pgfpathlineto{\pgfqpoint{3.230288in}{0.845053in}}%
\pgfpathlineto{\pgfqpoint{3.231248in}{0.843717in}}%
\pgfpathlineto{\pgfqpoint{3.231548in}{0.845881in}}%
\pgfpathlineto{\pgfqpoint{3.232828in}{0.890024in}}%
\pgfpathlineto{\pgfqpoint{3.238774in}{1.082387in}}%
\pgfpathlineto{\pgfqpoint{3.239524in}{1.082261in}}%
\pgfpathlineto{\pgfqpoint{3.238817in}{1.082146in}}%
\pgfpathlineto{\pgfqpoint{3.239925in}{1.081358in}}%
\pgfpathlineto{\pgfqpoint{3.240660in}{1.072661in}}%
\pgfpathlineto{\pgfqpoint{3.242876in}{1.051305in}}%
\pgfpathlineto{\pgfqpoint{3.243387in}{1.057458in}}%
\pgfpathlineto{\pgfqpoint{3.246797in}{1.099997in}}%
\pgfpathlineto{\pgfqpoint{3.247036in}{1.098873in}}%
\pgfpathlineto{\pgfqpoint{3.249061in}{1.090832in}}%
\pgfpathlineto{\pgfqpoint{3.269988in}{1.005808in}}%
\pgfpathlineto{\pgfqpoint{3.275332in}{0.968263in}}%
\pgfpathlineto{\pgfqpoint{3.278718in}{0.860943in}}%
\pgfpathlineto{\pgfqpoint{3.281612in}{0.819170in}}%
\pgfpathlineto{\pgfqpoint{3.283417in}{0.817720in}}%
\pgfpathlineto{\pgfqpoint{3.285017in}{0.816230in}}%
\pgfpathlineto{\pgfqpoint{3.286683in}{0.778109in}}%
\pgfpathlineto{\pgfqpoint{3.292261in}{0.651008in}}%
\pgfpathlineto{\pgfqpoint{3.305413in}{0.649545in}}%
\pgfpathlineto{\pgfqpoint{3.310853in}{0.644292in}}%
\pgfpathlineto{\pgfqpoint{3.317491in}{0.646904in}}%
\pgfpathlineto{\pgfqpoint{3.333829in}{0.643078in}}%
\pgfpathlineto{\pgfqpoint{3.340457in}{0.650047in}}%
\pgfpathlineto{\pgfqpoint{3.345357in}{0.648506in}}%
\pgfpathlineto{\pgfqpoint{3.348156in}{0.648876in}}%
\pgfpathlineto{\pgfqpoint{3.353595in}{0.649405in}}%
\pgfpathlineto{\pgfqpoint{3.358934in}{0.647669in}}%
\pgfpathlineto{\pgfqpoint{3.364928in}{0.648807in}}%
\pgfpathlineto{\pgfqpoint{3.371136in}{0.650350in}}%
\pgfpathlineto{\pgfqpoint{3.376174in}{0.651036in}}%
\pgfpathlineto{\pgfqpoint{3.377660in}{0.649494in}}%
\pgfpathlineto{\pgfqpoint{3.384785in}{0.641778in}}%
\pgfpathlineto{\pgfqpoint{3.392493in}{0.643317in}}%
\pgfpathlineto{\pgfqpoint{3.398023in}{0.646952in}}%
\pgfpathlineto{\pgfqpoint{3.405234in}{0.648493in}}%
\pgfpathlineto{\pgfqpoint{3.410354in}{0.649229in}}%
\pgfpathlineto{\pgfqpoint{3.419213in}{0.650756in}}%
\pgfpathlineto{\pgfqpoint{3.426787in}{0.672511in}}%
\pgfpathlineto{\pgfqpoint{3.435736in}{0.670973in}}%
\pgfpathlineto{\pgfqpoint{3.438979in}{0.668800in}}%
\pgfpathlineto{\pgfqpoint{3.443220in}{0.669095in}}%
\pgfpathlineto{\pgfqpoint{3.449586in}{0.667779in}}%
\pgfpathlineto{\pgfqpoint{3.456128in}{0.672100in}}%
\pgfpathlineto{\pgfqpoint{3.461592in}{0.671199in}}%
\pgfpathlineto{\pgfqpoint{3.462447in}{0.669290in}}%
\pgfpathlineto{\pgfqpoint{3.469333in}{0.649696in}}%
\pgfpathlineto{\pgfqpoint{3.473884in}{0.651238in}}%
\pgfpathlineto{\pgfqpoint{3.478937in}{0.651943in}}%
\pgfpathlineto{\pgfqpoint{3.493321in}{0.650402in}}%
\pgfpathlineto{\pgfqpoint{3.497323in}{0.648597in}}%
\pgfpathlineto{\pgfqpoint{3.504343in}{0.650244in}}%
\pgfpathlineto{\pgfqpoint{3.515294in}{0.649708in}}%
\pgfpathlineto{\pgfqpoint{3.532377in}{0.648165in}}%
\pgfpathlineto{\pgfqpoint{3.536689in}{0.647822in}}%
\pgfpathlineto{\pgfqpoint{3.539492in}{0.647453in}}%
\pgfpathlineto{\pgfqpoint{3.546861in}{0.645681in}}%
\pgfpathlineto{\pgfqpoint{3.557391in}{0.647220in}}%
\pgfpathlineto{\pgfqpoint{3.563466in}{0.651449in}}%
\pgfpathlineto{\pgfqpoint{3.567984in}{0.649907in}}%
\pgfpathlineto{\pgfqpoint{3.573323in}{0.646973in}}%
\pgfpathlineto{\pgfqpoint{3.584756in}{0.648512in}}%
\pgfpathlineto{\pgfqpoint{3.592273in}{0.670243in}}%
\pgfpathlineto{\pgfqpoint{3.601246in}{0.668706in}}%
\pgfpathlineto{\pgfqpoint{3.605334in}{0.665225in}}%
\pgfpathlineto{\pgfqpoint{3.610755in}{0.666767in}}%
\pgfpathlineto{\pgfqpoint{3.615779in}{0.667433in}}%
\pgfpathlineto{\pgfqpoint{3.621772in}{0.665894in}}%
\pgfpathlineto{\pgfqpoint{3.626032in}{0.663976in}}%
\pgfpathlineto{\pgfqpoint{3.627918in}{0.662430in}}%
\pgfpathlineto{\pgfqpoint{3.632775in}{0.643183in}}%
\pgfpathlineto{\pgfqpoint{3.634475in}{0.643917in}}%
\pgfpathlineto{\pgfqpoint{3.639356in}{0.643915in}}%
\pgfpathlineto{\pgfqpoint{3.642670in}{0.642373in}}%
\pgfpathlineto{\pgfqpoint{3.652351in}{0.639483in}}%
\pgfpathlineto{\pgfqpoint{3.670426in}{0.637941in}}%
\pgfpathlineto{\pgfqpoint{3.671902in}{0.638946in}}%
\pgfpathlineto{\pgfqpoint{3.677413in}{0.644304in}}%
\pgfpathlineto{\pgfqpoint{3.685752in}{0.645847in}}%
\pgfpathlineto{\pgfqpoint{3.690871in}{0.646585in}}%
\pgfpathlineto{\pgfqpoint{3.696788in}{0.648126in}}%
\pgfpathlineto{\pgfqpoint{3.701865in}{0.648842in}}%
\pgfpathlineto{\pgfqpoint{3.707973in}{0.650384in}}%
\pgfpathlineto{\pgfqpoint{3.713040in}{0.651096in}}%
\pgfpathlineto{\pgfqpoint{3.714797in}{0.649556in}}%
\pgfpathlineto{\pgfqpoint{3.718923in}{0.646227in}}%
\pgfpathlineto{\pgfqpoint{3.723231in}{0.647258in}}%
\pgfpathlineto{\pgfqpoint{3.749908in}{0.648799in}}%
\pgfpathlineto{\pgfqpoint{3.756804in}{0.669964in}}%
\pgfpathlineto{\pgfqpoint{3.760982in}{0.668423in}}%
\pgfpathlineto{\pgfqpoint{3.766197in}{0.666770in}}%
\pgfpathlineto{\pgfqpoint{3.777057in}{0.668312in}}%
\pgfpathlineto{\pgfqpoint{3.781518in}{0.668737in}}%
\pgfpathlineto{\pgfqpoint{3.786451in}{0.668089in}}%
\pgfpathlineto{\pgfqpoint{3.791747in}{0.672414in}}%
\pgfpathlineto{\pgfqpoint{3.792406in}{0.672116in}}%
\pgfpathlineto{\pgfqpoint{3.792741in}{0.671195in}}%
\pgfpathlineto{\pgfqpoint{3.800410in}{0.650920in}}%
\pgfpathlineto{\pgfqpoint{3.806179in}{0.652461in}}%
\pgfpathlineto{\pgfqpoint{3.812397in}{0.654772in}}%
\pgfpathlineto{\pgfqpoint{3.829981in}{0.653124in}}%
\pgfpathlineto{\pgfqpoint{3.837403in}{0.647914in}}%
\pgfpathlineto{\pgfqpoint{3.894706in}{0.646372in}}%
\pgfpathlineto{\pgfqpoint{3.900126in}{0.642509in}}%
\pgfpathlineto{\pgfqpoint{3.904486in}{0.642154in}}%
\pgfpathlineto{\pgfqpoint{3.911449in}{0.644964in}}%
\pgfpathlineto{\pgfqpoint{3.920934in}{0.646503in}}%
\pgfpathlineto{\pgfqpoint{3.926120in}{0.669190in}}%
\pgfpathlineto{\pgfqpoint{3.928369in}{0.668118in}}%
\pgfpathlineto{\pgfqpoint{3.932834in}{0.667692in}}%
\pgfpathlineto{\pgfqpoint{3.940132in}{0.670727in}}%
\pgfpathlineto{\pgfqpoint{3.958255in}{0.669187in}}%
\pgfpathlineto{\pgfqpoint{3.964803in}{0.660859in}}%
\pgfpathlineto{\pgfqpoint{3.969110in}{0.643048in}}%
\pgfpathlineto{\pgfqpoint{3.969951in}{0.643449in}}%
\pgfpathlineto{\pgfqpoint{3.974397in}{0.643864in}}%
\pgfpathlineto{\pgfqpoint{3.978867in}{0.642323in}}%
\pgfpathlineto{\pgfqpoint{3.983934in}{0.641636in}}%
\pgfpathlineto{\pgfqpoint{3.995763in}{0.640095in}}%
\pgfpathlineto{\pgfqpoint{4.000553in}{0.639539in}}%
\pgfpathlineto{\pgfqpoint{4.004135in}{0.641080in}}%
\pgfpathlineto{\pgfqpoint{4.009149in}{0.645151in}}%
\pgfpathlineto{\pgfqpoint{4.013061in}{0.646850in}}%
\pgfpathlineto{\pgfqpoint{4.017383in}{0.646494in}}%
\pgfpathlineto{\pgfqpoint{4.021791in}{0.648036in}}%
\pgfpathlineto{\pgfqpoint{4.026881in}{0.648759in}}%
\pgfpathlineto{\pgfqpoint{4.032775in}{0.650301in}}%
\pgfpathlineto{\pgfqpoint{4.042636in}{0.653306in}}%
\pgfpathlineto{\pgfqpoint{4.050736in}{0.651765in}}%
\pgfpathlineto{\pgfqpoint{4.058181in}{0.646551in}}%
\pgfpathlineto{\pgfqpoint{4.063797in}{0.647484in}}%
\pgfpathlineto{\pgfqpoint{4.069265in}{0.646577in}}%
\pgfpathlineto{\pgfqpoint{4.074772in}{0.647503in}}%
\pgfpathlineto{\pgfqpoint{4.082227in}{0.645670in}}%
\pgfpathlineto{\pgfqpoint{4.085584in}{0.647200in}}%
\pgfpathlineto{\pgfqpoint{4.088717in}{0.662028in}}%
\pgfpathlineto{\pgfqpoint{4.090937in}{0.670515in}}%
\pgfpathlineto{\pgfqpoint{4.091926in}{0.670045in}}%
\pgfpathlineto{\pgfqpoint{4.097580in}{0.669066in}}%
\pgfpathlineto{\pgfqpoint{4.103249in}{0.670054in}}%
\pgfpathlineto{\pgfqpoint{4.118641in}{0.664496in}}%
\pgfpathlineto{\pgfqpoint{4.125933in}{0.662954in}}%
\pgfpathlineto{\pgfqpoint{4.128971in}{0.659809in}}%
\pgfpathlineto{\pgfqpoint{4.135685in}{0.637273in}}%
\pgfpathlineto{\pgfqpoint{4.139367in}{0.638814in}}%
\pgfpathlineto{\pgfqpoint{4.149114in}{0.641708in}}%
\pgfpathlineto{\pgfqpoint{4.161393in}{0.643251in}}%
\pgfpathlineto{\pgfqpoint{4.171197in}{0.646171in}}%
\pgfpathlineto{\pgfqpoint{4.178241in}{0.647714in}}%
\pgfpathlineto{\pgfqpoint{4.181193in}{0.647400in}}%
\pgfpathlineto{\pgfqpoint{4.188523in}{0.645647in}}%
\pgfpathlineto{\pgfqpoint{4.199297in}{0.647190in}}%
\pgfpathlineto{\pgfqpoint{4.203266in}{0.647359in}}%
\pgfpathlineto{\pgfqpoint{4.211938in}{0.644973in}}%
\pgfpathlineto{\pgfqpoint{4.218018in}{0.646169in}}%
\pgfpathlineto{\pgfqpoint{4.227855in}{0.643229in}}%
\pgfpathlineto{\pgfqpoint{4.232316in}{0.646784in}}%
\pgfpathlineto{\pgfqpoint{4.235172in}{0.647133in}}%
\pgfpathlineto{\pgfqpoint{4.243114in}{0.649176in}}%
\pgfpathlineto{\pgfqpoint{4.255296in}{0.650791in}}%
\pgfpathlineto{\pgfqpoint{4.256576in}{0.655665in}}%
\pgfpathlineto{\pgfqpoint{4.262078in}{0.674177in}}%
\pgfpathlineto{\pgfqpoint{4.272517in}{0.672636in}}%
\pgfpathlineto{\pgfqpoint{4.276577in}{0.669629in}}%
\pgfpathlineto{\pgfqpoint{4.281352in}{0.669077in}}%
\pgfpathlineto{\pgfqpoint{4.284103in}{0.668682in}}%
\pgfpathlineto{\pgfqpoint{4.289767in}{0.667699in}}%
\pgfpathlineto{\pgfqpoint{4.295240in}{0.668607in}}%
\pgfpathlineto{\pgfqpoint{4.299022in}{0.665105in}}%
\pgfpathlineto{\pgfqpoint{4.303898in}{0.645564in}}%
\pgfpathlineto{\pgfqpoint{4.304911in}{0.646047in}}%
\pgfpathlineto{\pgfqpoint{4.311148in}{0.647302in}}%
\pgfpathlineto{\pgfqpoint{4.319944in}{0.648870in}}%
\pgfpathlineto{\pgfqpoint{4.324424in}{0.649289in}}%
\pgfpathlineto{\pgfqpoint{4.329668in}{0.648506in}}%
\pgfpathlineto{\pgfqpoint{4.344167in}{0.646966in}}%
\pgfpathlineto{\pgfqpoint{4.348660in}{0.644812in}}%
\pgfpathlineto{\pgfqpoint{4.353231in}{0.645275in}}%
\pgfpathlineto{\pgfqpoint{4.355299in}{0.645994in}}%
\pgfpathlineto{\pgfqpoint{4.360657in}{0.647851in}}%
\pgfpathlineto{\pgfqpoint{4.365036in}{0.647490in}}%
\pgfpathlineto{\pgfqpoint{4.368599in}{0.649033in}}%
\pgfpathlineto{\pgfqpoint{4.373475in}{0.649657in}}%
\pgfpathlineto{\pgfqpoint{4.378948in}{0.648749in}}%
\pgfpathlineto{\pgfqpoint{4.384645in}{0.649762in}}%
\pgfpathlineto{\pgfqpoint{4.396135in}{0.651304in}}%
\pgfpathlineto{\pgfqpoint{4.398084in}{0.650518in}}%
\pgfpathlineto{\pgfqpoint{4.402291in}{0.647964in}}%
\pgfpathlineto{\pgfqpoint{4.410395in}{0.650083in}}%
\pgfpathlineto{\pgfqpoint{4.421351in}{0.651511in}}%
\pgfpathlineto{\pgfqpoint{4.429565in}{0.674279in}}%
\pgfpathlineto{\pgfqpoint{4.432893in}{0.674146in}}%
\pgfpathlineto{\pgfqpoint{4.435329in}{0.674702in}}%
\pgfpathlineto{\pgfqpoint{4.438486in}{0.674957in}}%
\pgfpathlineto{\pgfqpoint{4.450625in}{0.670898in}}%
\pgfpathlineto{\pgfqpoint{4.456695in}{0.673747in}}%
\pgfpathlineto{\pgfqpoint{4.460989in}{0.673427in}}%
\pgfpathlineto{\pgfqpoint{4.464451in}{0.671897in}}%
\pgfpathlineto{\pgfqpoint{4.471643in}{0.652045in}}%
\pgfpathlineto{\pgfqpoint{4.486214in}{0.653586in}}%
\pgfpathlineto{\pgfqpoint{4.490378in}{0.653826in}}%
\pgfpathlineto{\pgfqpoint{4.497580in}{0.648698in}}%
\pgfpathlineto{\pgfqpoint{4.499328in}{0.648902in}}%
\pgfpathlineto{\pgfqpoint{4.506314in}{0.650514in}}%
\pgfpathlineto{\pgfqpoint{4.511907in}{0.649563in}}%
\pgfpathlineto{\pgfqpoint{4.522657in}{0.652962in}}%
\pgfpathlineto{\pgfqpoint{4.526597in}{0.649384in}}%
\pgfpathlineto{\pgfqpoint{4.530169in}{0.648127in}}%
\pgfpathlineto{\pgfqpoint{4.534882in}{0.647596in}}%
\pgfpathlineto{\pgfqpoint{4.538407in}{0.647573in}}%
\pgfpathlineto{\pgfqpoint{4.543784in}{0.646742in}}%
\pgfpathlineto{\pgfqpoint{4.555423in}{0.645199in}}%
\pgfpathlineto{\pgfqpoint{4.559071in}{0.643851in}}%
\pgfpathlineto{\pgfqpoint{4.562424in}{0.642510in}}%
\pgfpathlineto{\pgfqpoint{4.566044in}{0.642066in}}%
\pgfpathlineto{\pgfqpoint{4.569310in}{0.644928in}}%
\pgfpathlineto{\pgfqpoint{4.572997in}{0.646667in}}%
\pgfpathlineto{\pgfqpoint{4.575881in}{0.648209in}}%
\pgfpathlineto{\pgfqpoint{4.600104in}{0.666299in}}%
\pgfpathlineto{\pgfqpoint{4.602816in}{0.667653in}}%
\pgfpathlineto{\pgfqpoint{4.613237in}{0.667584in}}%
\pgfpathlineto{\pgfqpoint{4.617922in}{0.668460in}}%
\pgfpathlineto{\pgfqpoint{4.622005in}{0.667579in}}%
\pgfpathlineto{\pgfqpoint{4.633949in}{0.668025in}}%
\pgfpathlineto{\pgfqpoint{4.634808in}{0.667244in}}%
\pgfpathlineto{\pgfqpoint{4.635434in}{0.668558in}}%
\pgfpathlineto{\pgfqpoint{4.639794in}{0.687655in}}%
\pgfpathlineto{\pgfqpoint{4.640654in}{0.689185in}}%
\pgfpathlineto{\pgfqpoint{4.641647in}{0.688626in}}%
\pgfpathlineto{\pgfqpoint{4.647063in}{0.687818in}}%
\pgfpathlineto{\pgfqpoint{4.651709in}{0.690817in}}%
\pgfpathlineto{\pgfqpoint{4.659584in}{0.687561in}}%
\pgfpathlineto{\pgfqpoint{4.664155in}{0.688175in}}%
\pgfpathlineto{\pgfqpoint{4.665172in}{0.686693in}}%
\pgfpathlineto{\pgfqpoint{4.665960in}{0.688298in}}%
\pgfpathlineto{\pgfqpoint{4.673004in}{0.699271in}}%
\pgfpathlineto{\pgfqpoint{4.674924in}{0.701696in}}%
\pgfpathlineto{\pgfqpoint{4.676180in}{0.746713in}}%
\pgfpathlineto{\pgfqpoint{4.680435in}{0.940567in}}%
\pgfpathlineto{\pgfqpoint{4.681108in}{0.922216in}}%
\pgfpathlineto{\pgfqpoint{4.683439in}{0.882755in}}%
\pgfpathlineto{\pgfqpoint{4.684795in}{0.881871in}}%
\pgfpathlineto{\pgfqpoint{4.684924in}{0.883246in}}%
\pgfpathlineto{\pgfqpoint{4.688329in}{0.965753in}}%
\pgfpathlineto{\pgfqpoint{4.691686in}{1.010724in}}%
\pgfpathlineto{\pgfqpoint{4.693248in}{1.035005in}}%
\pgfpathlineto{\pgfqpoint{4.697102in}{1.149840in}}%
\pgfpathlineto{\pgfqpoint{4.700951in}{1.286982in}}%
\pgfpathlineto{\pgfqpoint{4.702957in}{1.285947in}}%
\pgfpathlineto{\pgfqpoint{4.706873in}{1.283052in}}%
\pgfpathlineto{\pgfqpoint{4.708855in}{1.280698in}}%
\pgfpathlineto{\pgfqpoint{4.715144in}{1.265153in}}%
\pgfpathlineto{\pgfqpoint{4.717890in}{1.260997in}}%
\pgfpathlineto{\pgfqpoint{4.719132in}{1.216301in}}%
\pgfpathlineto{\pgfqpoint{4.723736in}{1.058160in}}%
\pgfpathlineto{\pgfqpoint{4.727914in}{1.056643in}}%
\pgfpathlineto{\pgfqpoint{4.732322in}{0.962901in}}%
\pgfpathlineto{\pgfqpoint{4.737003in}{0.875899in}}%
\pgfpathlineto{\pgfqpoint{4.739467in}{0.810953in}}%
\pgfpathlineto{\pgfqpoint{4.744433in}{0.637273in}}%
\pgfpathlineto{\pgfqpoint{4.758631in}{0.638804in}}%
\pgfpathlineto{\pgfqpoint{4.759849in}{0.687554in}}%
\pgfpathlineto{\pgfqpoint{4.765480in}{0.904082in}}%
\pgfpathlineto{\pgfqpoint{4.767753in}{1.001283in}}%
\pgfpathlineto{\pgfqpoint{4.771702in}{1.136767in}}%
\pgfpathlineto{\pgfqpoint{4.774964in}{1.138053in}}%
\pgfpathlineto{\pgfqpoint{4.777763in}{1.139592in}}%
\pgfpathlineto{\pgfqpoint{4.786961in}{1.147344in}}%
\pgfpathlineto{\pgfqpoint{4.793102in}{1.148885in}}%
\pgfpathlineto{\pgfqpoint{4.798661in}{1.151452in}}%
\pgfpathlineto{\pgfqpoint{4.801503in}{1.152242in}}%
\pgfpathlineto{\pgfqpoint{4.801641in}{1.150492in}}%
\pgfpathlineto{\pgfqpoint{4.802907in}{1.098758in}}%
\pgfpathlineto{\pgfqpoint{4.808461in}{0.887378in}}%
\pgfpathlineto{\pgfqpoint{4.810767in}{0.787639in}}%
\pgfpathlineto{\pgfqpoint{4.814736in}{0.652845in}}%
\pgfpathlineto{\pgfqpoint{4.819292in}{0.649138in}}%
\pgfpathlineto{\pgfqpoint{4.830185in}{0.640771in}}%
\pgfpathlineto{\pgfqpoint{4.834894in}{0.641289in}}%
\pgfpathlineto{\pgfqpoint{4.842473in}{0.639397in}}%
\pgfpathlineto{\pgfqpoint{4.898682in}{0.640940in}}%
\pgfpathlineto{\pgfqpoint{4.904002in}{0.641745in}}%
\pgfpathlineto{\pgfqpoint{4.910082in}{0.643287in}}%
\pgfpathlineto{\pgfqpoint{4.915197in}{0.644023in}}%
\pgfpathlineto{\pgfqpoint{4.919194in}{0.642481in}}%
\pgfpathlineto{\pgfqpoint{4.924266in}{0.641794in}}%
\pgfpathlineto{\pgfqpoint{4.941888in}{0.640252in}}%
\pgfpathlineto{\pgfqpoint{4.944137in}{0.639845in}}%
\pgfpathlineto{\pgfqpoint{4.944452in}{0.640734in}}%
\pgfpathlineto{\pgfqpoint{4.948316in}{0.659330in}}%
\pgfpathlineto{\pgfqpoint{4.949830in}{0.664659in}}%
\pgfpathlineto{\pgfqpoint{4.950871in}{0.664165in}}%
\pgfpathlineto{\pgfqpoint{4.958182in}{0.662381in}}%
\pgfpathlineto{\pgfqpoint{4.975432in}{0.663921in}}%
\pgfpathlineto{\pgfqpoint{4.980499in}{0.667858in}}%
\pgfpathlineto{\pgfqpoint{4.987791in}{0.666319in}}%
\pgfpathlineto{\pgfqpoint{4.994640in}{0.646075in}}%
\pgfpathlineto{\pgfqpoint{5.000437in}{0.647618in}}%
\pgfpathlineto{\pgfqpoint{5.010223in}{0.650558in}}%
\pgfpathlineto{\pgfqpoint{5.017147in}{0.649023in}}%
\pgfpathlineto{\pgfqpoint{5.022787in}{0.641772in}}%
\pgfpathlineto{\pgfqpoint{5.043141in}{0.640230in}}%
\pgfpathlineto{\pgfqpoint{5.048519in}{0.639384in}}%
\pgfpathlineto{\pgfqpoint{5.072297in}{0.637842in}}%
\pgfpathlineto{\pgfqpoint{5.077115in}{0.637273in}}%
\pgfpathlineto{\pgfqpoint{5.086265in}{0.638815in}}%
\pgfpathlineto{\pgfqpoint{5.091113in}{0.639397in}}%
\pgfpathlineto{\pgfqpoint{5.096180in}{0.640939in}}%
\pgfpathlineto{\pgfqpoint{5.101676in}{0.642769in}}%
\pgfpathlineto{\pgfqpoint{5.106562in}{0.644306in}}%
\pgfpathlineto{\pgfqpoint{5.114509in}{0.667807in}}%
\pgfpathlineto{\pgfqpoint{5.122756in}{0.669348in}}%
\pgfpathlineto{\pgfqpoint{5.127298in}{0.672840in}}%
\pgfpathlineto{\pgfqpoint{5.131023in}{0.672785in}}%
\pgfpathlineto{\pgfqpoint{5.135904in}{0.673408in}}%
\pgfpathlineto{\pgfqpoint{5.139175in}{0.671867in}}%
\pgfpathlineto{\pgfqpoint{5.142556in}{0.671464in}}%
\pgfpathlineto{\pgfqpoint{5.149166in}{0.674834in}}%
\pgfpathlineto{\pgfqpoint{5.149447in}{0.674217in}}%
\pgfpathlineto{\pgfqpoint{5.152260in}{0.662021in}}%
\pgfpathlineto{\pgfqpoint{5.154724in}{0.653123in}}%
\pgfpathlineto{\pgfqpoint{5.155794in}{0.653614in}}%
\pgfpathlineto{\pgfqpoint{5.167413in}{0.652075in}}%
\pgfpathlineto{\pgfqpoint{5.170976in}{0.648980in}}%
\pgfpathlineto{\pgfqpoint{5.174558in}{0.647287in}}%
\pgfpathlineto{\pgfqpoint{5.184405in}{0.644291in}}%
\pgfpathlineto{\pgfqpoint{5.187638in}{0.645892in}}%
\pgfpathlineto{\pgfqpoint{5.188636in}{0.646834in}}%
\pgfpathlineto{\pgfqpoint{5.188636in}{0.646834in}}%
\pgfusepath{stroke}%
\end{pgfscope}%
\begin{pgfscope}%
\pgfpathrectangle{\pgfqpoint{0.750000in}{0.500000in}}{\pgfqpoint{4.650000in}{3.020000in}}%
\pgfusepath{clip}%
\pgfsetrectcap%
\pgfsetroundjoin%
\pgfsetlinewidth{1.505625pt}%
\definecolor{currentstroke}{rgb}{0.000000,0.500000,0.000000}%
\pgfsetstrokecolor{currentstroke}%
\pgfsetdash{}{0pt}%
\pgfpathmoveto{\pgfqpoint{0.961364in}{1.368563in}}%
\pgfpathlineto{\pgfqpoint{0.963188in}{1.304196in}}%
\pgfpathlineto{\pgfqpoint{0.965657in}{1.014953in}}%
\pgfpathlineto{\pgfqpoint{0.969286in}{0.658183in}}%
\pgfpathlineto{\pgfqpoint{0.988695in}{0.656642in}}%
\pgfpathlineto{\pgfqpoint{0.995935in}{0.647827in}}%
\pgfpathlineto{\pgfqpoint{1.017349in}{0.649369in}}%
\pgfpathlineto{\pgfqpoint{1.021819in}{0.657130in}}%
\pgfpathlineto{\pgfqpoint{1.050430in}{0.657219in}}%
\pgfpathlineto{\pgfqpoint{1.060497in}{0.655679in}}%
\pgfpathlineto{\pgfqpoint{1.064718in}{0.646828in}}%
\pgfpathlineto{\pgfqpoint{1.066557in}{0.648910in}}%
\pgfpathlineto{\pgfqpoint{1.069050in}{0.653521in}}%
\pgfpathlineto{\pgfqpoint{1.072460in}{0.656964in}}%
\pgfpathlineto{\pgfqpoint{1.090808in}{0.661672in}}%
\pgfpathlineto{\pgfqpoint{1.095158in}{0.661897in}}%
\pgfpathlineto{\pgfqpoint{1.112217in}{0.660354in}}%
\pgfpathlineto{\pgfqpoint{1.115407in}{0.658508in}}%
\pgfpathlineto{\pgfqpoint{1.115617in}{0.659137in}}%
\pgfpathlineto{\pgfqpoint{1.117069in}{0.674245in}}%
\pgfpathlineto{\pgfqpoint{1.120971in}{0.697073in}}%
\pgfpathlineto{\pgfqpoint{1.124873in}{0.696993in}}%
\pgfpathlineto{\pgfqpoint{1.132872in}{0.696956in}}%
\pgfpathlineto{\pgfqpoint{1.159210in}{0.695968in}}%
\pgfpathlineto{\pgfqpoint{1.162753in}{0.671063in}}%
\pgfpathlineto{\pgfqpoint{1.165098in}{0.662680in}}%
\pgfpathlineto{\pgfqpoint{1.188045in}{0.661139in}}%
\pgfpathlineto{\pgfqpoint{1.189664in}{0.657762in}}%
\pgfpathlineto{\pgfqpoint{1.194831in}{0.648504in}}%
\pgfpathlineto{\pgfqpoint{1.214249in}{0.648301in}}%
\pgfpathlineto{\pgfqpoint{1.223638in}{0.647657in}}%
\pgfpathlineto{\pgfqpoint{1.233538in}{0.648136in}}%
\pgfpathlineto{\pgfqpoint{1.240357in}{0.646847in}}%
\pgfpathlineto{\pgfqpoint{1.252172in}{0.648388in}}%
\pgfpathlineto{\pgfqpoint{1.256275in}{0.657960in}}%
\pgfpathlineto{\pgfqpoint{1.258352in}{0.658051in}}%
\pgfpathlineto{\pgfqpoint{1.280349in}{0.659584in}}%
\pgfpathlineto{\pgfqpoint{1.287317in}{0.698881in}}%
\pgfpathlineto{\pgfqpoint{1.318865in}{0.697338in}}%
\pgfpathlineto{\pgfqpoint{1.323869in}{0.695008in}}%
\pgfpathlineto{\pgfqpoint{1.327131in}{0.666809in}}%
\pgfpathlineto{\pgfqpoint{1.328626in}{0.647241in}}%
\pgfpathlineto{\pgfqpoint{1.330073in}{0.647366in}}%
\pgfpathlineto{\pgfqpoint{1.360298in}{0.647910in}}%
\pgfpathlineto{\pgfqpoint{1.366702in}{0.648318in}}%
\pgfpathlineto{\pgfqpoint{1.370976in}{0.657476in}}%
\pgfpathlineto{\pgfqpoint{1.373188in}{0.657845in}}%
\pgfpathlineto{\pgfqpoint{1.409707in}{0.656083in}}%
\pgfpathlineto{\pgfqpoint{1.413800in}{0.648515in}}%
\pgfpathlineto{\pgfqpoint{1.415930in}{0.647407in}}%
\pgfpathlineto{\pgfqpoint{1.445080in}{0.649124in}}%
\pgfpathlineto{\pgfqpoint{1.451155in}{0.700720in}}%
\pgfpathlineto{\pgfqpoint{1.478405in}{0.702262in}}%
\pgfpathlineto{\pgfqpoint{1.482827in}{0.701243in}}%
\pgfpathlineto{\pgfqpoint{1.488983in}{0.696107in}}%
\pgfpathlineto{\pgfqpoint{1.492374in}{0.667636in}}%
\pgfpathlineto{\pgfqpoint{1.494656in}{0.657656in}}%
\pgfpathlineto{\pgfqpoint{1.524304in}{0.657360in}}%
\pgfpathlineto{\pgfqpoint{1.527336in}{0.651882in}}%
\pgfpathlineto{\pgfqpoint{1.530732in}{0.647631in}}%
\pgfpathlineto{\pgfqpoint{1.538640in}{0.647384in}}%
\pgfpathlineto{\pgfqpoint{1.554319in}{0.646827in}}%
\pgfpathlineto{\pgfqpoint{1.607462in}{0.648369in}}%
\pgfpathlineto{\pgfqpoint{1.612290in}{0.651884in}}%
\pgfpathlineto{\pgfqpoint{1.618384in}{0.653383in}}%
\pgfpathlineto{\pgfqpoint{1.624573in}{0.697249in}}%
\pgfpathlineto{\pgfqpoint{1.629086in}{0.698790in}}%
\pgfpathlineto{\pgfqpoint{1.634402in}{0.699168in}}%
\pgfpathlineto{\pgfqpoint{1.647511in}{0.700710in}}%
\pgfpathlineto{\pgfqpoint{1.660658in}{0.700698in}}%
\pgfpathlineto{\pgfqpoint{1.661809in}{0.699168in}}%
\pgfpathlineto{\pgfqpoint{1.664746in}{0.678212in}}%
\pgfpathlineto{\pgfqpoint{1.668261in}{0.659717in}}%
\pgfpathlineto{\pgfqpoint{1.695143in}{0.658174in}}%
\pgfpathlineto{\pgfqpoint{1.697884in}{0.655282in}}%
\pgfpathlineto{\pgfqpoint{1.700540in}{0.653694in}}%
\pgfpathlineto{\pgfqpoint{1.706605in}{0.652011in}}%
\pgfpathlineto{\pgfqpoint{1.711309in}{0.647438in}}%
\pgfpathlineto{\pgfqpoint{1.715669in}{0.649006in}}%
\pgfpathlineto{\pgfqpoint{1.720607in}{0.655380in}}%
\pgfpathlineto{\pgfqpoint{1.759524in}{0.653841in}}%
\pgfpathlineto{\pgfqpoint{1.766535in}{0.645724in}}%
\pgfpathlineto{\pgfqpoint{1.784453in}{0.644183in}}%
\pgfpathlineto{\pgfqpoint{1.788889in}{0.643614in}}%
\pgfpathlineto{\pgfqpoint{1.793441in}{0.645141in}}%
\pgfpathlineto{\pgfqpoint{1.799343in}{0.702052in}}%
\pgfpathlineto{\pgfqpoint{1.809181in}{0.700509in}}%
\pgfpathlineto{\pgfqpoint{1.815690in}{0.700476in}}%
\pgfpathlineto{\pgfqpoint{1.823857in}{0.700238in}}%
\pgfpathlineto{\pgfqpoint{1.837186in}{0.698720in}}%
\pgfpathlineto{\pgfqpoint{1.840309in}{0.671736in}}%
\pgfpathlineto{\pgfqpoint{1.843657in}{0.648271in}}%
\pgfpathlineto{\pgfqpoint{1.864335in}{0.647277in}}%
\pgfpathlineto{\pgfqpoint{1.880009in}{0.647235in}}%
\pgfpathlineto{\pgfqpoint{1.890554in}{0.647737in}}%
\pgfpathlineto{\pgfqpoint{1.896069in}{0.649958in}}%
\pgfpathlineto{\pgfqpoint{1.902063in}{0.647736in}}%
\pgfpathlineto{\pgfqpoint{1.918591in}{0.648634in}}%
\pgfpathlineto{\pgfqpoint{1.921629in}{0.650954in}}%
\pgfpathlineto{\pgfqpoint{1.924957in}{0.674641in}}%
\pgfpathlineto{\pgfqpoint{1.927689in}{0.679179in}}%
\pgfpathlineto{\pgfqpoint{1.931218in}{0.677692in}}%
\pgfpathlineto{\pgfqpoint{1.931276in}{0.679706in}}%
\pgfpathlineto{\pgfqpoint{1.932126in}{0.838976in}}%
\pgfpathlineto{\pgfqpoint{1.935741in}{1.280681in}}%
\pgfpathlineto{\pgfqpoint{1.936510in}{1.315919in}}%
\pgfpathlineto{\pgfqpoint{1.937298in}{1.281746in}}%
\pgfpathlineto{\pgfqpoint{1.938329in}{1.249936in}}%
\pgfpathlineto{\pgfqpoint{1.939346in}{1.250112in}}%
\pgfpathlineto{\pgfqpoint{1.945932in}{1.252572in}}%
\pgfpathlineto{\pgfqpoint{1.948769in}{1.257368in}}%
\pgfpathlineto{\pgfqpoint{1.949003in}{1.255813in}}%
\pgfpathlineto{\pgfqpoint{1.949237in}{1.254420in}}%
\pgfpathlineto{\pgfqpoint{1.950135in}{1.258945in}}%
\pgfpathlineto{\pgfqpoint{1.951692in}{1.269093in}}%
\pgfpathlineto{\pgfqpoint{1.952852in}{1.323154in}}%
\pgfpathlineto{\pgfqpoint{1.955555in}{1.467954in}}%
\pgfpathlineto{\pgfqpoint{1.956338in}{1.463016in}}%
\pgfpathlineto{\pgfqpoint{1.958492in}{1.461017in}}%
\pgfpathlineto{\pgfqpoint{1.964696in}{1.462558in}}%
\pgfpathlineto{\pgfqpoint{1.966931in}{1.466904in}}%
\pgfpathlineto{\pgfqpoint{1.971339in}{1.472196in}}%
\pgfpathlineto{\pgfqpoint{1.974572in}{1.473983in}}%
\pgfpathlineto{\pgfqpoint{1.974844in}{1.471728in}}%
\pgfpathlineto{\pgfqpoint{1.975909in}{1.435431in}}%
\pgfpathlineto{\pgfqpoint{1.978865in}{1.315829in}}%
\pgfpathlineto{\pgfqpoint{1.980193in}{1.315972in}}%
\pgfpathlineto{\pgfqpoint{1.992065in}{1.314434in}}%
\pgfpathlineto{\pgfqpoint{1.992777in}{1.311292in}}%
\pgfpathlineto{\pgfqpoint{1.994954in}{1.285898in}}%
\pgfpathlineto{\pgfqpoint{1.996396in}{1.154499in}}%
\pgfpathlineto{\pgfqpoint{2.001922in}{0.637273in}}%
\pgfpathlineto{\pgfqpoint{2.014625in}{0.638787in}}%
\pgfpathlineto{\pgfqpoint{2.016459in}{0.707167in}}%
\pgfpathlineto{\pgfqpoint{2.019716in}{0.751076in}}%
\pgfpathlineto{\pgfqpoint{2.020246in}{0.750188in}}%
\pgfpathlineto{\pgfqpoint{2.028030in}{0.746308in}}%
\pgfpathlineto{\pgfqpoint{2.039669in}{0.744788in}}%
\pgfpathlineto{\pgfqpoint{2.043432in}{0.730866in}}%
\pgfpathlineto{\pgfqpoint{2.045294in}{0.729018in}}%
\pgfpathlineto{\pgfqpoint{2.045653in}{0.729749in}}%
\pgfpathlineto{\pgfqpoint{2.046460in}{0.734818in}}%
\pgfpathlineto{\pgfqpoint{2.047553in}{0.889573in}}%
\pgfpathlineto{\pgfqpoint{2.051336in}{1.283741in}}%
\pgfpathlineto{\pgfqpoint{2.052210in}{1.307293in}}%
\pgfpathlineto{\pgfqpoint{2.053303in}{1.305544in}}%
\pgfpathlineto{\pgfqpoint{2.059306in}{1.307083in}}%
\pgfpathlineto{\pgfqpoint{2.067415in}{1.316036in}}%
\pgfpathlineto{\pgfqpoint{2.082621in}{1.317624in}}%
\pgfpathlineto{\pgfqpoint{2.089398in}{1.325207in}}%
\pgfpathlineto{\pgfqpoint{2.089584in}{1.324400in}}%
\pgfpathlineto{\pgfqpoint{2.090324in}{1.305602in}}%
\pgfpathlineto{\pgfqpoint{2.092588in}{1.120962in}}%
\pgfpathlineto{\pgfqpoint{2.098467in}{0.657122in}}%
\pgfpathlineto{\pgfqpoint{2.113510in}{0.658166in}}%
\pgfpathlineto{\pgfqpoint{2.121060in}{0.698915in}}%
\pgfpathlineto{\pgfqpoint{2.140712in}{0.697372in}}%
\pgfpathlineto{\pgfqpoint{2.152431in}{0.696791in}}%
\pgfpathlineto{\pgfqpoint{2.156892in}{0.696032in}}%
\pgfpathlineto{\pgfqpoint{2.159285in}{0.679134in}}%
\pgfpathlineto{\pgfqpoint{2.164137in}{0.648605in}}%
\pgfpathlineto{\pgfqpoint{2.183827in}{0.647062in}}%
\pgfpathlineto{\pgfqpoint{2.187872in}{0.647307in}}%
\pgfpathlineto{\pgfqpoint{2.198049in}{0.645765in}}%
\pgfpathlineto{\pgfqpoint{2.210317in}{0.643617in}}%
\pgfpathlineto{\pgfqpoint{2.211898in}{0.645156in}}%
\pgfpathlineto{\pgfqpoint{2.216707in}{0.647573in}}%
\pgfpathlineto{\pgfqpoint{2.218173in}{0.649110in}}%
\pgfpathlineto{\pgfqpoint{2.222701in}{0.654214in}}%
\pgfpathlineto{\pgfqpoint{2.260887in}{0.652672in}}%
\pgfpathlineto{\pgfqpoint{2.264693in}{0.644904in}}%
\pgfpathlineto{\pgfqpoint{2.266680in}{0.643604in}}%
\pgfpathlineto{\pgfqpoint{2.275448in}{0.642064in}}%
\pgfpathlineto{\pgfqpoint{2.277964in}{0.637798in}}%
\pgfpathlineto{\pgfqpoint{2.279569in}{0.637273in}}%
\pgfpathlineto{\pgfqpoint{2.284249in}{0.638798in}}%
\pgfpathlineto{\pgfqpoint{2.288791in}{0.698643in}}%
\pgfpathlineto{\pgfqpoint{2.290935in}{0.702595in}}%
\pgfpathlineto{\pgfqpoint{2.315979in}{0.701053in}}%
\pgfpathlineto{\pgfqpoint{2.321791in}{0.699970in}}%
\pgfpathlineto{\pgfqpoint{2.327927in}{0.698437in}}%
\pgfpathlineto{\pgfqpoint{2.330038in}{0.684702in}}%
\pgfpathlineto{\pgfqpoint{2.335210in}{0.656116in}}%
\pgfpathlineto{\pgfqpoint{2.357345in}{0.654578in}}%
\pgfpathlineto{\pgfqpoint{2.364093in}{0.645638in}}%
\pgfpathlineto{\pgfqpoint{2.369256in}{0.645748in}}%
\pgfpathlineto{\pgfqpoint{2.373181in}{0.647290in}}%
\pgfpathlineto{\pgfqpoint{2.386076in}{0.652660in}}%
\pgfpathlineto{\pgfqpoint{2.427562in}{0.649350in}}%
\pgfpathlineto{\pgfqpoint{2.434892in}{0.649169in}}%
\pgfpathlineto{\pgfqpoint{2.446750in}{0.647974in}}%
\pgfpathlineto{\pgfqpoint{2.488752in}{0.649895in}}%
\pgfpathlineto{\pgfqpoint{2.500567in}{0.650907in}}%
\pgfpathlineto{\pgfqpoint{2.531580in}{0.650111in}}%
\pgfpathlineto{\pgfqpoint{2.536896in}{0.649281in}}%
\pgfpathlineto{\pgfqpoint{2.598712in}{0.647743in}}%
\pgfpathlineto{\pgfqpoint{2.601281in}{0.646668in}}%
\pgfpathlineto{\pgfqpoint{2.610460in}{0.646286in}}%
\pgfpathlineto{\pgfqpoint{2.616133in}{0.644745in}}%
\pgfpathlineto{\pgfqpoint{2.621391in}{0.643915in}}%
\pgfpathlineto{\pgfqpoint{2.625446in}{0.645445in}}%
\pgfpathlineto{\pgfqpoint{2.630169in}{0.661440in}}%
\pgfpathlineto{\pgfqpoint{2.634415in}{0.661510in}}%
\pgfpathlineto{\pgfqpoint{2.642060in}{0.659967in}}%
\pgfpathlineto{\pgfqpoint{2.649186in}{0.658117in}}%
\pgfpathlineto{\pgfqpoint{2.669043in}{0.656578in}}%
\pgfpathlineto{\pgfqpoint{2.672582in}{0.649344in}}%
\pgfpathlineto{\pgfqpoint{2.674406in}{0.648724in}}%
\pgfpathlineto{\pgfqpoint{2.695376in}{0.650260in}}%
\pgfpathlineto{\pgfqpoint{2.697449in}{0.650709in}}%
\pgfpathlineto{\pgfqpoint{2.749049in}{0.649478in}}%
\pgfpathlineto{\pgfqpoint{2.753386in}{0.649048in}}%
\pgfpathlineto{\pgfqpoint{2.756585in}{0.650805in}}%
\pgfpathlineto{\pgfqpoint{2.760330in}{0.651975in}}%
\pgfpathlineto{\pgfqpoint{2.792847in}{0.653491in}}%
\pgfpathlineto{\pgfqpoint{2.799189in}{0.699690in}}%
\pgfpathlineto{\pgfqpoint{2.828588in}{0.700704in}}%
\pgfpathlineto{\pgfqpoint{2.835341in}{0.700995in}}%
\pgfpathlineto{\pgfqpoint{2.836220in}{0.699484in}}%
\pgfpathlineto{\pgfqpoint{2.838631in}{0.681668in}}%
\pgfpathlineto{\pgfqpoint{2.843588in}{0.648991in}}%
\pgfpathlineto{\pgfqpoint{2.854095in}{0.649523in}}%
\pgfpathlineto{\pgfqpoint{2.856406in}{0.650498in}}%
\pgfpathlineto{\pgfqpoint{2.865418in}{0.650796in}}%
\pgfpathlineto{\pgfqpoint{2.867868in}{0.650480in}}%
\pgfpathlineto{\pgfqpoint{2.899015in}{0.647344in}}%
\pgfpathlineto{\pgfqpoint{2.905958in}{0.646018in}}%
\pgfpathlineto{\pgfqpoint{2.910094in}{0.648067in}}%
\pgfpathlineto{\pgfqpoint{2.913733in}{0.650249in}}%
\pgfpathlineto{\pgfqpoint{2.952860in}{0.650236in}}%
\pgfpathlineto{\pgfqpoint{2.962244in}{0.646496in}}%
\pgfpathlineto{\pgfqpoint{2.966901in}{0.648349in}}%
\pgfpathlineto{\pgfqpoint{2.973099in}{0.649883in}}%
\pgfpathlineto{\pgfqpoint{3.011042in}{0.648341in}}%
\pgfpathlineto{\pgfqpoint{3.016768in}{0.647527in}}%
\pgfpathlineto{\pgfqpoint{3.072562in}{0.648331in}}%
\pgfpathlineto{\pgfqpoint{3.080872in}{0.646061in}}%
\pgfpathlineto{\pgfqpoint{3.085136in}{0.638326in}}%
\pgfpathlineto{\pgfqpoint{3.086846in}{0.637273in}}%
\pgfpathlineto{\pgfqpoint{3.087634in}{0.638804in}}%
\pgfpathlineto{\pgfqpoint{3.090041in}{0.644058in}}%
\pgfpathlineto{\pgfqpoint{3.093771in}{0.646277in}}%
\pgfpathlineto{\pgfqpoint{3.110829in}{0.647820in}}%
\pgfpathlineto{\pgfqpoint{3.120978in}{0.651241in}}%
\pgfpathlineto{\pgfqpoint{3.126240in}{0.699449in}}%
\pgfpathlineto{\pgfqpoint{3.126417in}{0.699431in}}%
\pgfpathlineto{\pgfqpoint{3.153681in}{0.697888in}}%
\pgfpathlineto{\pgfqpoint{3.162688in}{0.695086in}}%
\pgfpathlineto{\pgfqpoint{3.164613in}{0.692605in}}%
\pgfpathlineto{\pgfqpoint{3.173137in}{0.664184in}}%
\pgfpathlineto{\pgfqpoint{3.175119in}{0.662428in}}%
\pgfpathlineto{\pgfqpoint{3.175630in}{0.663253in}}%
\pgfpathlineto{\pgfqpoint{3.177225in}{0.671076in}}%
\pgfpathlineto{\pgfqpoint{3.184819in}{0.703896in}}%
\pgfpathlineto{\pgfqpoint{3.196887in}{0.727077in}}%
\pgfpathlineto{\pgfqpoint{3.200325in}{0.727877in}}%
\pgfpathlineto{\pgfqpoint{3.200497in}{0.727562in}}%
\pgfpathlineto{\pgfqpoint{3.206037in}{0.713686in}}%
\pgfpathlineto{\pgfqpoint{3.213310in}{0.692233in}}%
\pgfpathlineto{\pgfqpoint{3.218945in}{0.675913in}}%
\pgfpathlineto{\pgfqpoint{3.231252in}{0.672194in}}%
\pgfpathlineto{\pgfqpoint{3.231372in}{0.673274in}}%
\pgfpathlineto{\pgfqpoint{3.231620in}{0.694660in}}%
\pgfpathlineto{\pgfqpoint{3.236988in}{1.243132in}}%
\pgfpathlineto{\pgfqpoint{3.237279in}{1.244558in}}%
\pgfpathlineto{\pgfqpoint{3.238573in}{1.243570in}}%
\pgfpathlineto{\pgfqpoint{3.240006in}{1.243031in}}%
\pgfpathlineto{\pgfqpoint{3.240235in}{1.244332in}}%
\pgfpathlineto{\pgfqpoint{3.241897in}{1.251970in}}%
\pgfpathlineto{\pgfqpoint{3.242088in}{1.246060in}}%
\pgfpathlineto{\pgfqpoint{3.244596in}{1.194529in}}%
\pgfpathlineto{\pgfqpoint{3.248292in}{1.103085in}}%
\pgfpathlineto{\pgfqpoint{3.249925in}{1.107621in}}%
\pgfpathlineto{\pgfqpoint{3.274787in}{1.168145in}}%
\pgfpathlineto{\pgfqpoint{3.275408in}{1.165260in}}%
\pgfpathlineto{\pgfqpoint{3.276382in}{1.146056in}}%
\pgfpathlineto{\pgfqpoint{3.281311in}{1.053865in}}%
\pgfpathlineto{\pgfqpoint{3.281812in}{1.054213in}}%
\pgfpathlineto{\pgfqpoint{3.285265in}{1.053692in}}%
\pgfpathlineto{\pgfqpoint{3.285298in}{1.053499in}}%
\pgfpathlineto{\pgfqpoint{3.286301in}{1.034477in}}%
\pgfpathlineto{\pgfqpoint{3.288828in}{0.884371in}}%
\pgfpathlineto{\pgfqpoint{3.292896in}{0.657584in}}%
\pgfpathlineto{\pgfqpoint{3.305361in}{0.656042in}}%
\pgfpathlineto{\pgfqpoint{3.312362in}{0.648281in}}%
\pgfpathlineto{\pgfqpoint{3.318532in}{0.649037in}}%
\pgfpathlineto{\pgfqpoint{3.333700in}{0.647994in}}%
\pgfpathlineto{\pgfqpoint{3.338227in}{0.658388in}}%
\pgfpathlineto{\pgfqpoint{3.340233in}{0.658555in}}%
\pgfpathlineto{\pgfqpoint{3.370382in}{0.657012in}}%
\pgfpathlineto{\pgfqpoint{3.377941in}{0.655015in}}%
\pgfpathlineto{\pgfqpoint{3.381327in}{0.646524in}}%
\pgfpathlineto{\pgfqpoint{3.383137in}{0.645918in}}%
\pgfpathlineto{\pgfqpoint{3.390325in}{0.645253in}}%
\pgfpathlineto{\pgfqpoint{3.390349in}{0.645348in}}%
\pgfpathlineto{\pgfqpoint{3.394460in}{0.657170in}}%
\pgfpathlineto{\pgfqpoint{3.396719in}{0.658146in}}%
\pgfpathlineto{\pgfqpoint{3.419045in}{0.659659in}}%
\pgfpathlineto{\pgfqpoint{3.425846in}{0.698130in}}%
\pgfpathlineto{\pgfqpoint{3.440584in}{0.696587in}}%
\pgfpathlineto{\pgfqpoint{3.445612in}{0.697262in}}%
\pgfpathlineto{\pgfqpoint{3.452795in}{0.697931in}}%
\pgfpathlineto{\pgfqpoint{3.462509in}{0.696410in}}%
\pgfpathlineto{\pgfqpoint{3.465856in}{0.670583in}}%
\pgfpathlineto{\pgfqpoint{3.468526in}{0.657531in}}%
\pgfpathlineto{\pgfqpoint{3.491626in}{0.655988in}}%
\pgfpathlineto{\pgfqpoint{3.498727in}{0.648517in}}%
\pgfpathlineto{\pgfqpoint{3.535223in}{0.647431in}}%
\pgfpathlineto{\pgfqpoint{3.556365in}{0.649035in}}%
\pgfpathlineto{\pgfqpoint{3.560763in}{0.657749in}}%
\pgfpathlineto{\pgfqpoint{3.584512in}{0.659282in}}%
\pgfpathlineto{\pgfqpoint{3.592115in}{0.699017in}}%
\pgfpathlineto{\pgfqpoint{3.611256in}{0.697474in}}%
\pgfpathlineto{\pgfqpoint{3.620607in}{0.697628in}}%
\pgfpathlineto{\pgfqpoint{3.627813in}{0.697153in}}%
\pgfpathlineto{\pgfqpoint{3.629886in}{0.682842in}}%
\pgfpathlineto{\pgfqpoint{3.635612in}{0.646782in}}%
\pgfpathlineto{\pgfqpoint{3.643998in}{0.645242in}}%
\pgfpathlineto{\pgfqpoint{3.648812in}{0.643588in}}%
\pgfpathlineto{\pgfqpoint{3.669309in}{0.642049in}}%
\pgfpathlineto{\pgfqpoint{3.670742in}{0.640510in}}%
\pgfpathlineto{\pgfqpoint{3.671014in}{0.642560in}}%
\pgfpathlineto{\pgfqpoint{3.674051in}{0.653973in}}%
\pgfpathlineto{\pgfqpoint{3.677017in}{0.657008in}}%
\pgfpathlineto{\pgfqpoint{3.714816in}{0.655469in}}%
\pgfpathlineto{\pgfqpoint{3.720996in}{0.648683in}}%
\pgfpathlineto{\pgfqpoint{3.749359in}{0.650193in}}%
\pgfpathlineto{\pgfqpoint{3.756240in}{0.700503in}}%
\pgfpathlineto{\pgfqpoint{3.788767in}{0.698960in}}%
\pgfpathlineto{\pgfqpoint{3.793347in}{0.694340in}}%
\pgfpathlineto{\pgfqpoint{3.796743in}{0.665219in}}%
\pgfpathlineto{\pgfqpoint{3.798901in}{0.656670in}}%
\pgfpathlineto{\pgfqpoint{3.829079in}{0.655131in}}%
\pgfpathlineto{\pgfqpoint{3.835693in}{0.648567in}}%
\pgfpathlineto{\pgfqpoint{3.864108in}{0.648475in}}%
\pgfpathlineto{\pgfqpoint{3.882614in}{0.648576in}}%
\pgfpathlineto{\pgfqpoint{3.896602in}{0.647033in}}%
\pgfpathlineto{\pgfqpoint{3.906201in}{0.647234in}}%
\pgfpathlineto{\pgfqpoint{3.911397in}{0.648400in}}%
\pgfpathlineto{\pgfqpoint{3.920633in}{0.649914in}}%
\pgfpathlineto{\pgfqpoint{3.927567in}{0.700666in}}%
\pgfpathlineto{\pgfqpoint{3.935748in}{0.700408in}}%
\pgfpathlineto{\pgfqpoint{3.947806in}{0.699660in}}%
\pgfpathlineto{\pgfqpoint{3.964230in}{0.699109in}}%
\pgfpathlineto{\pgfqpoint{3.967100in}{0.675293in}}%
\pgfpathlineto{\pgfqpoint{3.969106in}{0.646432in}}%
\pgfpathlineto{\pgfqpoint{3.971064in}{0.646781in}}%
\pgfpathlineto{\pgfqpoint{3.993609in}{0.645239in}}%
\pgfpathlineto{\pgfqpoint{4.001403in}{0.643978in}}%
\pgfpathlineto{\pgfqpoint{4.007793in}{0.649165in}}%
\pgfpathlineto{\pgfqpoint{4.011934in}{0.657151in}}%
\pgfpathlineto{\pgfqpoint{4.050588in}{0.655496in}}%
\pgfpathlineto{\pgfqpoint{4.057914in}{0.647417in}}%
\pgfpathlineto{\pgfqpoint{4.085555in}{0.650105in}}%
\pgfpathlineto{\pgfqpoint{4.091009in}{0.700163in}}%
\pgfpathlineto{\pgfqpoint{4.100274in}{0.700212in}}%
\pgfpathlineto{\pgfqpoint{4.106678in}{0.700680in}}%
\pgfpathlineto{\pgfqpoint{4.124138in}{0.702223in}}%
\pgfpathlineto{\pgfqpoint{4.128890in}{0.701242in}}%
\pgfpathlineto{\pgfqpoint{4.131048in}{0.685039in}}%
\pgfpathlineto{\pgfqpoint{4.133488in}{0.647126in}}%
\pgfpathlineto{\pgfqpoint{4.135470in}{0.637273in}}%
\pgfpathlineto{\pgfqpoint{4.136354in}{0.638804in}}%
\pgfpathlineto{\pgfqpoint{4.139176in}{0.642498in}}%
\pgfpathlineto{\pgfqpoint{4.145003in}{0.645123in}}%
\pgfpathlineto{\pgfqpoint{4.150614in}{0.645644in}}%
\pgfpathlineto{\pgfqpoint{4.166765in}{0.647187in}}%
\pgfpathlineto{\pgfqpoint{4.183647in}{0.647311in}}%
\pgfpathlineto{\pgfqpoint{4.212626in}{0.647202in}}%
\pgfpathlineto{\pgfqpoint{4.222201in}{0.646658in}}%
\pgfpathlineto{\pgfqpoint{4.226795in}{0.645734in}}%
\pgfpathlineto{\pgfqpoint{4.226967in}{0.646396in}}%
\pgfpathlineto{\pgfqpoint{4.230907in}{0.655389in}}%
\pgfpathlineto{\pgfqpoint{4.233328in}{0.656310in}}%
\pgfpathlineto{\pgfqpoint{4.255640in}{0.657779in}}%
\pgfpathlineto{\pgfqpoint{4.261911in}{0.697460in}}%
\pgfpathlineto{\pgfqpoint{4.299075in}{0.695933in}}%
\pgfpathlineto{\pgfqpoint{4.302136in}{0.670528in}}%
\pgfpathlineto{\pgfqpoint{4.305536in}{0.647410in}}%
\pgfpathlineto{\pgfqpoint{4.354988in}{0.648947in}}%
\pgfpathlineto{\pgfqpoint{4.359640in}{0.653531in}}%
\pgfpathlineto{\pgfqpoint{4.396871in}{0.652995in}}%
\pgfpathlineto{\pgfqpoint{4.399655in}{0.650211in}}%
\pgfpathlineto{\pgfqpoint{4.403805in}{0.647403in}}%
\pgfpathlineto{\pgfqpoint{4.421016in}{0.648903in}}%
\pgfpathlineto{\pgfqpoint{4.427187in}{0.698647in}}%
\pgfpathlineto{\pgfqpoint{4.464313in}{0.697107in}}%
\pgfpathlineto{\pgfqpoint{4.465965in}{0.687754in}}%
\pgfpathlineto{\pgfqpoint{4.472006in}{0.657172in}}%
\pgfpathlineto{\pgfqpoint{4.493487in}{0.655634in}}%
\pgfpathlineto{\pgfqpoint{4.501386in}{0.647121in}}%
\pgfpathlineto{\pgfqpoint{4.516128in}{0.646946in}}%
\pgfpathlineto{\pgfqpoint{4.526272in}{0.651916in}}%
\pgfpathlineto{\pgfqpoint{4.538569in}{0.652131in}}%
\pgfpathlineto{\pgfqpoint{4.560977in}{0.650659in}}%
\pgfpathlineto{\pgfqpoint{4.565494in}{0.648257in}}%
\pgfpathlineto{\pgfqpoint{4.568164in}{0.649800in}}%
\pgfpathlineto{\pgfqpoint{4.571679in}{0.652261in}}%
\pgfpathlineto{\pgfqpoint{4.579482in}{0.654298in}}%
\pgfpathlineto{\pgfqpoint{4.583628in}{0.655754in}}%
\pgfpathlineto{\pgfqpoint{4.590013in}{0.656223in}}%
\pgfpathlineto{\pgfqpoint{4.593188in}{0.656554in}}%
\pgfpathlineto{\pgfqpoint{4.601923in}{0.655441in}}%
\pgfpathlineto{\pgfqpoint{4.616164in}{0.655051in}}%
\pgfpathlineto{\pgfqpoint{4.617773in}{0.655948in}}%
\pgfpathlineto{\pgfqpoint{4.621193in}{0.656214in}}%
\pgfpathlineto{\pgfqpoint{4.635033in}{0.654718in}}%
\pgfpathlineto{\pgfqpoint{4.635047in}{0.654787in}}%
\pgfpathlineto{\pgfqpoint{4.636250in}{0.668660in}}%
\pgfpathlineto{\pgfqpoint{4.640066in}{0.694958in}}%
\pgfpathlineto{\pgfqpoint{4.642129in}{0.695512in}}%
\pgfpathlineto{\pgfqpoint{4.648863in}{0.694412in}}%
\pgfpathlineto{\pgfqpoint{4.652488in}{0.694151in}}%
\pgfpathlineto{\pgfqpoint{4.658496in}{0.695640in}}%
\pgfpathlineto{\pgfqpoint{4.666509in}{0.697540in}}%
\pgfpathlineto{\pgfqpoint{4.671094in}{0.701480in}}%
\pgfpathlineto{\pgfqpoint{4.674852in}{0.703110in}}%
\pgfpathlineto{\pgfqpoint{4.675535in}{0.792163in}}%
\pgfpathlineto{\pgfqpoint{4.679981in}{1.298202in}}%
\pgfpathlineto{\pgfqpoint{4.680120in}{1.299395in}}%
\pgfpathlineto{\pgfqpoint{4.680640in}{1.283084in}}%
\pgfpathlineto{\pgfqpoint{4.681915in}{1.235356in}}%
\pgfpathlineto{\pgfqpoint{4.682937in}{1.236772in}}%
\pgfpathlineto{\pgfqpoint{4.685382in}{1.241504in}}%
\pgfpathlineto{\pgfqpoint{4.688635in}{1.274946in}}%
\pgfpathlineto{\pgfqpoint{4.689485in}{1.273260in}}%
\pgfpathlineto{\pgfqpoint{4.691061in}{1.264894in}}%
\pgfpathlineto{\pgfqpoint{4.692603in}{1.257425in}}%
\pgfpathlineto{\pgfqpoint{4.693296in}{1.260616in}}%
\pgfpathlineto{\pgfqpoint{4.693907in}{1.262537in}}%
\pgfpathlineto{\pgfqpoint{4.695173in}{1.261613in}}%
\pgfpathlineto{\pgfqpoint{4.695865in}{1.265381in}}%
\pgfpathlineto{\pgfqpoint{4.696820in}{1.303950in}}%
\pgfpathlineto{\pgfqpoint{4.699623in}{1.409627in}}%
\pgfpathlineto{\pgfqpoint{4.699953in}{1.407899in}}%
\pgfpathlineto{\pgfqpoint{4.701811in}{1.403450in}}%
\pgfpathlineto{\pgfqpoint{4.701935in}{1.403490in}}%
\pgfpathlineto{\pgfqpoint{4.705445in}{1.406312in}}%
\pgfpathlineto{\pgfqpoint{4.709208in}{1.409582in}}%
\pgfpathlineto{\pgfqpoint{4.715178in}{1.420144in}}%
\pgfpathlineto{\pgfqpoint{4.718110in}{1.423785in}}%
\pgfpathlineto{\pgfqpoint{4.718239in}{1.423954in}}%
\pgfpathlineto{\pgfqpoint{4.718674in}{1.420814in}}%
\pgfpathlineto{\pgfqpoint{4.719863in}{1.380315in}}%
\pgfpathlineto{\pgfqpoint{4.722408in}{1.292827in}}%
\pgfpathlineto{\pgfqpoint{4.723430in}{1.294003in}}%
\pgfpathlineto{\pgfqpoint{4.728731in}{1.292474in}}%
\pgfpathlineto{\pgfqpoint{4.730971in}{1.290650in}}%
\pgfpathlineto{\pgfqpoint{4.731845in}{1.285331in}}%
\pgfpathlineto{\pgfqpoint{4.734529in}{1.276103in}}%
\pgfpathlineto{\pgfqpoint{4.735794in}{1.272504in}}%
\pgfpathlineto{\pgfqpoint{4.739319in}{1.224651in}}%
\pgfpathlineto{\pgfqpoint{4.741382in}{0.983389in}}%
\pgfpathlineto{\pgfqpoint{4.745097in}{0.637273in}}%
\pgfpathlineto{\pgfqpoint{4.758507in}{0.638787in}}%
\pgfpathlineto{\pgfqpoint{4.762767in}{1.269868in}}%
\pgfpathlineto{\pgfqpoint{4.763875in}{1.327333in}}%
\pgfpathlineto{\pgfqpoint{4.764663in}{1.297468in}}%
\pgfpathlineto{\pgfqpoint{4.765289in}{1.285687in}}%
\pgfpathlineto{\pgfqpoint{4.766163in}{1.298957in}}%
\pgfpathlineto{\pgfqpoint{4.767567in}{1.365778in}}%
\pgfpathlineto{\pgfqpoint{4.770107in}{1.473460in}}%
\pgfpathlineto{\pgfqpoint{4.770719in}{1.468588in}}%
\pgfpathlineto{\pgfqpoint{4.772724in}{1.464162in}}%
\pgfpathlineto{\pgfqpoint{4.775628in}{1.463670in}}%
\pgfpathlineto{\pgfqpoint{4.777806in}{1.462129in}}%
\pgfpathlineto{\pgfqpoint{4.787983in}{1.457500in}}%
\pgfpathlineto{\pgfqpoint{4.794339in}{1.455959in}}%
\pgfpathlineto{\pgfqpoint{4.800356in}{1.454697in}}%
\pgfpathlineto{\pgfqpoint{4.802085in}{1.452814in}}%
\pgfpathlineto{\pgfqpoint{4.803074in}{1.418424in}}%
\pgfpathlineto{\pgfqpoint{4.808064in}{1.257622in}}%
\pgfpathlineto{\pgfqpoint{4.809550in}{1.222676in}}%
\pgfpathlineto{\pgfqpoint{4.811340in}{1.049090in}}%
\pgfpathlineto{\pgfqpoint{4.816145in}{0.657732in}}%
\pgfpathlineto{\pgfqpoint{4.822496in}{0.655533in}}%
\pgfpathlineto{\pgfqpoint{4.827668in}{0.643716in}}%
\pgfpathlineto{\pgfqpoint{4.829149in}{0.644484in}}%
\pgfpathlineto{\pgfqpoint{4.834770in}{0.645400in}}%
\pgfpathlineto{\pgfqpoint{4.844789in}{0.643483in}}%
\pgfpathlineto{\pgfqpoint{4.872540in}{0.641942in}}%
\pgfpathlineto{\pgfqpoint{4.875095in}{0.641699in}}%
\pgfpathlineto{\pgfqpoint{4.879274in}{0.643618in}}%
\pgfpathlineto{\pgfqpoint{4.898888in}{0.645161in}}%
\pgfpathlineto{\pgfqpoint{4.904585in}{0.645684in}}%
\pgfpathlineto{\pgfqpoint{4.942580in}{0.644142in}}%
\pgfpathlineto{\pgfqpoint{4.943951in}{0.643742in}}%
\pgfpathlineto{\pgfqpoint{4.944142in}{0.645035in}}%
\pgfpathlineto{\pgfqpoint{4.950450in}{0.702396in}}%
\pgfpathlineto{\pgfqpoint{4.966974in}{0.702877in}}%
\pgfpathlineto{\pgfqpoint{4.976325in}{0.701336in}}%
\pgfpathlineto{\pgfqpoint{4.982371in}{0.700187in}}%
\pgfpathlineto{\pgfqpoint{4.987753in}{0.698662in}}%
\pgfpathlineto{\pgfqpoint{4.990170in}{0.681837in}}%
\pgfpathlineto{\pgfqpoint{4.994611in}{0.656111in}}%
\pgfpathlineto{\pgfqpoint{5.018040in}{0.654569in}}%
\pgfpathlineto{\pgfqpoint{5.021517in}{0.646138in}}%
\pgfpathlineto{\pgfqpoint{5.023212in}{0.645712in}}%
\pgfpathlineto{\pgfqpoint{5.043671in}{0.644170in}}%
\pgfpathlineto{\pgfqpoint{5.048394in}{0.643466in}}%
\pgfpathlineto{\pgfqpoint{5.071107in}{0.641924in}}%
\pgfpathlineto{\pgfqpoint{5.073481in}{0.637673in}}%
\pgfpathlineto{\pgfqpoint{5.075057in}{0.637273in}}%
\pgfpathlineto{\pgfqpoint{5.083247in}{0.638804in}}%
\pgfpathlineto{\pgfqpoint{5.086065in}{0.642495in}}%
\pgfpathlineto{\pgfqpoint{5.089809in}{0.643483in}}%
\pgfpathlineto{\pgfqpoint{5.095444in}{0.645024in}}%
\pgfpathlineto{\pgfqpoint{5.100201in}{0.647459in}}%
\pgfpathlineto{\pgfqpoint{5.106237in}{0.648953in}}%
\pgfpathlineto{\pgfqpoint{5.112770in}{0.701276in}}%
\pgfpathlineto{\pgfqpoint{5.124986in}{0.699735in}}%
\pgfpathlineto{\pgfqpoint{5.129179in}{0.699536in}}%
\pgfpathlineto{\pgfqpoint{5.149428in}{0.697997in}}%
\pgfpathlineto{\pgfqpoint{5.150536in}{0.692722in}}%
\pgfpathlineto{\pgfqpoint{5.153698in}{0.663834in}}%
\pgfpathlineto{\pgfqpoint{5.155837in}{0.654925in}}%
\pgfpathlineto{\pgfqpoint{5.165422in}{0.653386in}}%
\pgfpathlineto{\pgfqpoint{5.170646in}{0.648557in}}%
\pgfpathlineto{\pgfqpoint{5.186888in}{0.650780in}}%
\pgfpathlineto{\pgfqpoint{5.188636in}{0.654166in}}%
\pgfpathlineto{\pgfqpoint{5.188636in}{0.654166in}}%
\pgfusepath{stroke}%
\end{pgfscope}%
\begin{pgfscope}%
\pgfsetrectcap%
\pgfsetmiterjoin%
\pgfsetlinewidth{0.803000pt}%
\definecolor{currentstroke}{rgb}{0.000000,0.000000,0.000000}%
\pgfsetstrokecolor{currentstroke}%
\pgfsetdash{}{0pt}%
\pgfpathmoveto{\pgfqpoint{0.750000in}{0.500000in}}%
\pgfpathlineto{\pgfqpoint{0.750000in}{3.520000in}}%
\pgfusepath{stroke}%
\end{pgfscope}%
\begin{pgfscope}%
\pgfsetrectcap%
\pgfsetmiterjoin%
\pgfsetlinewidth{0.803000pt}%
\definecolor{currentstroke}{rgb}{0.000000,0.000000,0.000000}%
\pgfsetstrokecolor{currentstroke}%
\pgfsetdash{}{0pt}%
\pgfpathmoveto{\pgfqpoint{5.400000in}{0.500000in}}%
\pgfpathlineto{\pgfqpoint{5.400000in}{3.520000in}}%
\pgfusepath{stroke}%
\end{pgfscope}%
\begin{pgfscope}%
\pgfsetrectcap%
\pgfsetmiterjoin%
\pgfsetlinewidth{0.803000pt}%
\definecolor{currentstroke}{rgb}{0.000000,0.000000,0.000000}%
\pgfsetstrokecolor{currentstroke}%
\pgfsetdash{}{0pt}%
\pgfpathmoveto{\pgfqpoint{0.750000in}{0.500000in}}%
\pgfpathlineto{\pgfqpoint{5.400000in}{0.500000in}}%
\pgfusepath{stroke}%
\end{pgfscope}%
\begin{pgfscope}%
\pgfsetrectcap%
\pgfsetmiterjoin%
\pgfsetlinewidth{0.803000pt}%
\definecolor{currentstroke}{rgb}{0.000000,0.000000,0.000000}%
\pgfsetstrokecolor{currentstroke}%
\pgfsetdash{}{0pt}%
\pgfpathmoveto{\pgfqpoint{0.750000in}{3.520000in}}%
\pgfpathlineto{\pgfqpoint{5.400000in}{3.520000in}}%
\pgfusepath{stroke}%
\end{pgfscope}%
\begin{pgfscope}%
\pgfsetbuttcap%
\pgfsetmiterjoin%
\definecolor{currentfill}{rgb}{1.000000,1.000000,1.000000}%
\pgfsetfillcolor{currentfill}%
\pgfsetfillopacity{0.800000}%
\pgfsetlinewidth{1.003750pt}%
\definecolor{currentstroke}{rgb}{0.800000,0.800000,0.800000}%
\pgfsetstrokecolor{currentstroke}%
\pgfsetstrokeopacity{0.800000}%
\pgfsetdash{}{0pt}%
\pgfpathmoveto{\pgfqpoint{2.723533in}{2.827871in}}%
\pgfpathlineto{\pgfqpoint{3.426467in}{2.827871in}}%
\pgfpathquadraticcurveto{\pgfqpoint{3.454244in}{2.827871in}}{\pgfqpoint{3.454244in}{2.855648in}}%
\pgfpathlineto{\pgfqpoint{3.454244in}{3.422778in}}%
\pgfpathquadraticcurveto{\pgfqpoint{3.454244in}{3.450556in}}{\pgfqpoint{3.426467in}{3.450556in}}%
\pgfpathlineto{\pgfqpoint{2.723533in}{3.450556in}}%
\pgfpathquadraticcurveto{\pgfqpoint{2.695756in}{3.450556in}}{\pgfqpoint{2.695756in}{3.422778in}}%
\pgfpathlineto{\pgfqpoint{2.695756in}{2.855648in}}%
\pgfpathquadraticcurveto{\pgfqpoint{2.695756in}{2.827871in}}{\pgfqpoint{2.723533in}{2.827871in}}%
\pgfpathlineto{\pgfqpoint{2.723533in}{2.827871in}}%
\pgfpathclose%
\pgfusepath{stroke,fill}%
\end{pgfscope}%
\begin{pgfscope}%
\pgfsetrectcap%
\pgfsetroundjoin%
\pgfsetlinewidth{1.505625pt}%
\definecolor{currentstroke}{rgb}{0.000000,0.000000,1.000000}%
\pgfsetstrokecolor{currentstroke}%
\pgfsetdash{}{0pt}%
\pgfpathmoveto{\pgfqpoint{2.751311in}{3.346389in}}%
\pgfpathlineto{\pgfqpoint{2.890200in}{3.346389in}}%
\pgfpathlineto{\pgfqpoint{3.029089in}{3.346389in}}%
\pgfusepath{stroke}%
\end{pgfscope}%
\begin{pgfscope}%
\definecolor{textcolor}{rgb}{0.000000,0.000000,0.000000}%
\pgfsetstrokecolor{textcolor}%
\pgfsetfillcolor{textcolor}%
\pgftext[x=3.140200in,y=3.297778in,left,base]{\color{textcolor}\rmfamily\fontsize{10.000000}{12.000000}\selectfont max}%
\end{pgfscope}%
\begin{pgfscope}%
\pgfsetrectcap%
\pgfsetroundjoin%
\pgfsetlinewidth{1.505625pt}%
\definecolor{currentstroke}{rgb}{1.000000,0.000000,0.000000}%
\pgfsetstrokecolor{currentstroke}%
\pgfsetdash{}{0pt}%
\pgfpathmoveto{\pgfqpoint{2.751311in}{3.152716in}}%
\pgfpathlineto{\pgfqpoint{2.890200in}{3.152716in}}%
\pgfpathlineto{\pgfqpoint{3.029089in}{3.152716in}}%
\pgfusepath{stroke}%
\end{pgfscope}%
\begin{pgfscope}%
\definecolor{textcolor}{rgb}{0.000000,0.000000,0.000000}%
\pgfsetstrokecolor{textcolor}%
\pgfsetfillcolor{textcolor}%
\pgftext[x=3.140200in,y=3.104105in,left,base]{\color{textcolor}\rmfamily\fontsize{10.000000}{12.000000}\selectfont \(\displaystyle \mu\)}%
\end{pgfscope}%
\begin{pgfscope}%
\pgfsetrectcap%
\pgfsetroundjoin%
\pgfsetlinewidth{1.505625pt}%
\definecolor{currentstroke}{rgb}{0.000000,0.500000,0.000000}%
\pgfsetstrokecolor{currentstroke}%
\pgfsetdash{}{0pt}%
\pgfpathmoveto{\pgfqpoint{2.751311in}{2.959043in}}%
\pgfpathlineto{\pgfqpoint{2.890200in}{2.959043in}}%
\pgfpathlineto{\pgfqpoint{3.029089in}{2.959043in}}%
\pgfusepath{stroke}%
\end{pgfscope}%
\begin{pgfscope}%
\definecolor{textcolor}{rgb}{0.000000,0.000000,0.000000}%
\pgfsetstrokecolor{textcolor}%
\pgfsetfillcolor{textcolor}%
\pgftext[x=3.140200in,y=2.910432in,left,base]{\color{textcolor}\rmfamily\fontsize{10.000000}{12.000000}\selectfont \(\displaystyle \sigma\)}%
\end{pgfscope}%
\end{pgfpicture}%
\makeatother%
\endgroup%

    \caption{BETH Matrix Profile}
    \label{fig:beth_mp_hist}
\end{figure}

Figure \ref{fig:beth_sus_outliers} shows the algorithms detection results compared to the values considered suspicious by the creators of the BETH dataset. The BETH dataset researchers determined that many of the data points that are considered suspicious are not actually dangerous or evil as shown in figure \ref{fig:beth_evil_outliers}. In this case, the detector is able to discern fairly well where the evil outlier lie among the very noisy suspicious data points. This is the most challenging data set analyzed thus far and as such the results do not provide 100\% accuracy with no false positives.

 \begin{figure}[H]
    %%\centering
    %% Creator: Matplotlib, PGF backend
%%
%% To include the figure in your LaTeX document, write
%%   \input{<filename>.pgf}
%%
%% Make sure the required packages are loaded in your preamble
%%   \usepackage{pgf}
%%
%% Also ensure that all the required font packages are loaded; for instance,
%% the lmodern package is sometimes necessary when using math font.
%%   \usepackage{lmodern}
%%
%% Figures using additional raster images can only be included by \input if
%% they are in the same directory as the main LaTeX file. For loading figures
%% from other directories you can use the `import` package
%%   \usepackage{import}
%%
%% and then include the figures with
%%   \import{<path to file>}{<filename>.pgf}
%%
%% Matplotlib used the following preamble
%%
\begingroup%
\makeatletter%
\begin{pgfpicture}%
\pgfpathrectangle{\pgfpointorigin}{\pgfqpoint{6.000000in}{4.000000in}}%
\pgfusepath{use as bounding box, clip}%
\begin{pgfscope}%
\pgfsetbuttcap%
\pgfsetmiterjoin%
\pgfsetlinewidth{0.000000pt}%
\definecolor{currentstroke}{rgb}{1.000000,1.000000,1.000000}%
\pgfsetstrokecolor{currentstroke}%
\pgfsetstrokeopacity{0.000000}%
\pgfsetdash{}{0pt}%
\pgfpathmoveto{\pgfqpoint{0.000000in}{0.000000in}}%
\pgfpathlineto{\pgfqpoint{6.000000in}{0.000000in}}%
\pgfpathlineto{\pgfqpoint{6.000000in}{4.000000in}}%
\pgfpathlineto{\pgfqpoint{0.000000in}{4.000000in}}%
\pgfpathlineto{\pgfqpoint{0.000000in}{0.000000in}}%
\pgfpathclose%
\pgfusepath{}%
\end{pgfscope}%
\begin{pgfscope}%
\pgfsetbuttcap%
\pgfsetmiterjoin%
\definecolor{currentfill}{rgb}{1.000000,1.000000,1.000000}%
\pgfsetfillcolor{currentfill}%
\pgfsetlinewidth{0.000000pt}%
\definecolor{currentstroke}{rgb}{0.000000,0.000000,0.000000}%
\pgfsetstrokecolor{currentstroke}%
\pgfsetstrokeopacity{0.000000}%
\pgfsetdash{}{0pt}%
\pgfpathmoveto{\pgfqpoint{0.750000in}{0.500000in}}%
\pgfpathlineto{\pgfqpoint{5.400000in}{0.500000in}}%
\pgfpathlineto{\pgfqpoint{5.400000in}{3.520000in}}%
\pgfpathlineto{\pgfqpoint{0.750000in}{3.520000in}}%
\pgfpathlineto{\pgfqpoint{0.750000in}{0.500000in}}%
\pgfpathclose%
\pgfusepath{fill}%
\end{pgfscope}%
\begin{pgfscope}%
\pgfsetbuttcap%
\pgfsetroundjoin%
\definecolor{currentfill}{rgb}{0.000000,0.000000,0.000000}%
\pgfsetfillcolor{currentfill}%
\pgfsetlinewidth{0.803000pt}%
\definecolor{currentstroke}{rgb}{0.000000,0.000000,0.000000}%
\pgfsetstrokecolor{currentstroke}%
\pgfsetdash{}{0pt}%
\pgfsys@defobject{currentmarker}{\pgfqpoint{0.000000in}{-0.048611in}}{\pgfqpoint{0.000000in}{0.000000in}}{%
\pgfpathmoveto{\pgfqpoint{0.000000in}{0.000000in}}%
\pgfpathlineto{\pgfqpoint{0.000000in}{-0.048611in}}%
\pgfusepath{stroke,fill}%
}%
\begin{pgfscope}%
\pgfsys@transformshift{0.961364in}{0.500000in}%
\pgfsys@useobject{currentmarker}{}%
\end{pgfscope}%
\end{pgfscope}%
\begin{pgfscope}%
\definecolor{textcolor}{rgb}{0.000000,0.000000,0.000000}%
\pgfsetstrokecolor{textcolor}%
\pgfsetfillcolor{textcolor}%
\pgftext[x=0.961364in,y=0.402778in,,top]{\color{textcolor}\rmfamily\fontsize{10.000000}{12.000000}\selectfont \(\displaystyle {0}\)}%
\end{pgfscope}%
\begin{pgfscope}%
\pgfsetbuttcap%
\pgfsetroundjoin%
\definecolor{currentfill}{rgb}{0.000000,0.000000,0.000000}%
\pgfsetfillcolor{currentfill}%
\pgfsetlinewidth{0.803000pt}%
\definecolor{currentstroke}{rgb}{0.000000,0.000000,0.000000}%
\pgfsetstrokecolor{currentstroke}%
\pgfsetdash{}{0pt}%
\pgfsys@defobject{currentmarker}{\pgfqpoint{0.000000in}{-0.048611in}}{\pgfqpoint{0.000000in}{0.000000in}}{%
\pgfpathmoveto{\pgfqpoint{0.000000in}{0.000000in}}%
\pgfpathlineto{\pgfqpoint{0.000000in}{-0.048611in}}%
\pgfusepath{stroke,fill}%
}%
\begin{pgfscope}%
\pgfsys@transformshift{1.905825in}{0.500000in}%
\pgfsys@useobject{currentmarker}{}%
\end{pgfscope}%
\end{pgfscope}%
\begin{pgfscope}%
\definecolor{textcolor}{rgb}{0.000000,0.000000,0.000000}%
\pgfsetstrokecolor{textcolor}%
\pgfsetfillcolor{textcolor}%
\pgftext[x=1.905825in,y=0.402778in,,top]{\color{textcolor}\rmfamily\fontsize{10.000000}{12.000000}\selectfont \(\displaystyle {200000}\)}%
\end{pgfscope}%
\begin{pgfscope}%
\pgfsetbuttcap%
\pgfsetroundjoin%
\definecolor{currentfill}{rgb}{0.000000,0.000000,0.000000}%
\pgfsetfillcolor{currentfill}%
\pgfsetlinewidth{0.803000pt}%
\definecolor{currentstroke}{rgb}{0.000000,0.000000,0.000000}%
\pgfsetstrokecolor{currentstroke}%
\pgfsetdash{}{0pt}%
\pgfsys@defobject{currentmarker}{\pgfqpoint{0.000000in}{-0.048611in}}{\pgfqpoint{0.000000in}{0.000000in}}{%
\pgfpathmoveto{\pgfqpoint{0.000000in}{0.000000in}}%
\pgfpathlineto{\pgfqpoint{0.000000in}{-0.048611in}}%
\pgfusepath{stroke,fill}%
}%
\begin{pgfscope}%
\pgfsys@transformshift{2.850287in}{0.500000in}%
\pgfsys@useobject{currentmarker}{}%
\end{pgfscope}%
\end{pgfscope}%
\begin{pgfscope}%
\definecolor{textcolor}{rgb}{0.000000,0.000000,0.000000}%
\pgfsetstrokecolor{textcolor}%
\pgfsetfillcolor{textcolor}%
\pgftext[x=2.850287in,y=0.402778in,,top]{\color{textcolor}\rmfamily\fontsize{10.000000}{12.000000}\selectfont \(\displaystyle {400000}\)}%
\end{pgfscope}%
\begin{pgfscope}%
\pgfsetbuttcap%
\pgfsetroundjoin%
\definecolor{currentfill}{rgb}{0.000000,0.000000,0.000000}%
\pgfsetfillcolor{currentfill}%
\pgfsetlinewidth{0.803000pt}%
\definecolor{currentstroke}{rgb}{0.000000,0.000000,0.000000}%
\pgfsetstrokecolor{currentstroke}%
\pgfsetdash{}{0pt}%
\pgfsys@defobject{currentmarker}{\pgfqpoint{0.000000in}{-0.048611in}}{\pgfqpoint{0.000000in}{0.000000in}}{%
\pgfpathmoveto{\pgfqpoint{0.000000in}{0.000000in}}%
\pgfpathlineto{\pgfqpoint{0.000000in}{-0.048611in}}%
\pgfusepath{stroke,fill}%
}%
\begin{pgfscope}%
\pgfsys@transformshift{3.794748in}{0.500000in}%
\pgfsys@useobject{currentmarker}{}%
\end{pgfscope}%
\end{pgfscope}%
\begin{pgfscope}%
\definecolor{textcolor}{rgb}{0.000000,0.000000,0.000000}%
\pgfsetstrokecolor{textcolor}%
\pgfsetfillcolor{textcolor}%
\pgftext[x=3.794748in,y=0.402778in,,top]{\color{textcolor}\rmfamily\fontsize{10.000000}{12.000000}\selectfont \(\displaystyle {600000}\)}%
\end{pgfscope}%
\begin{pgfscope}%
\pgfsetbuttcap%
\pgfsetroundjoin%
\definecolor{currentfill}{rgb}{0.000000,0.000000,0.000000}%
\pgfsetfillcolor{currentfill}%
\pgfsetlinewidth{0.803000pt}%
\definecolor{currentstroke}{rgb}{0.000000,0.000000,0.000000}%
\pgfsetstrokecolor{currentstroke}%
\pgfsetdash{}{0pt}%
\pgfsys@defobject{currentmarker}{\pgfqpoint{0.000000in}{-0.048611in}}{\pgfqpoint{0.000000in}{0.000000in}}{%
\pgfpathmoveto{\pgfqpoint{0.000000in}{0.000000in}}%
\pgfpathlineto{\pgfqpoint{0.000000in}{-0.048611in}}%
\pgfusepath{stroke,fill}%
}%
\begin{pgfscope}%
\pgfsys@transformshift{4.739210in}{0.500000in}%
\pgfsys@useobject{currentmarker}{}%
\end{pgfscope}%
\end{pgfscope}%
\begin{pgfscope}%
\definecolor{textcolor}{rgb}{0.000000,0.000000,0.000000}%
\pgfsetstrokecolor{textcolor}%
\pgfsetfillcolor{textcolor}%
\pgftext[x=4.739210in,y=0.402778in,,top]{\color{textcolor}\rmfamily\fontsize{10.000000}{12.000000}\selectfont \(\displaystyle {800000}\)}%
\end{pgfscope}%
\begin{pgfscope}%
\definecolor{textcolor}{rgb}{0.000000,0.000000,0.000000}%
\pgfsetstrokecolor{textcolor}%
\pgfsetfillcolor{textcolor}%
\pgftext[x=3.075000in,y=0.223766in,,top]{\color{textcolor}\rmfamily\fontsize{10.000000}{12.000000}\selectfont time}%
\end{pgfscope}%
\begin{pgfscope}%
\pgfsetbuttcap%
\pgfsetroundjoin%
\definecolor{currentfill}{rgb}{0.000000,0.000000,0.000000}%
\pgfsetfillcolor{currentfill}%
\pgfsetlinewidth{0.803000pt}%
\definecolor{currentstroke}{rgb}{0.000000,0.000000,0.000000}%
\pgfsetstrokecolor{currentstroke}%
\pgfsetdash{}{0pt}%
\pgfsys@defobject{currentmarker}{\pgfqpoint{-0.048611in}{0.000000in}}{\pgfqpoint{-0.000000in}{0.000000in}}{%
\pgfpathmoveto{\pgfqpoint{-0.000000in}{0.000000in}}%
\pgfpathlineto{\pgfqpoint{-0.048611in}{0.000000in}}%
\pgfusepath{stroke,fill}%
}%
\begin{pgfscope}%
\pgfsys@transformshift{0.750000in}{0.637273in}%
\pgfsys@useobject{currentmarker}{}%
\end{pgfscope}%
\end{pgfscope}%
\begin{pgfscope}%
\definecolor{textcolor}{rgb}{0.000000,0.000000,0.000000}%
\pgfsetstrokecolor{textcolor}%
\pgfsetfillcolor{textcolor}%
\pgftext[x=0.475308in, y=0.589047in, left, base]{\color{textcolor}\rmfamily\fontsize{10.000000}{12.000000}\selectfont \(\displaystyle {0.0}\)}%
\end{pgfscope}%
\begin{pgfscope}%
\pgfsetbuttcap%
\pgfsetroundjoin%
\definecolor{currentfill}{rgb}{0.000000,0.000000,0.000000}%
\pgfsetfillcolor{currentfill}%
\pgfsetlinewidth{0.803000pt}%
\definecolor{currentstroke}{rgb}{0.000000,0.000000,0.000000}%
\pgfsetstrokecolor{currentstroke}%
\pgfsetdash{}{0pt}%
\pgfsys@defobject{currentmarker}{\pgfqpoint{-0.048611in}{0.000000in}}{\pgfqpoint{-0.000000in}{0.000000in}}{%
\pgfpathmoveto{\pgfqpoint{-0.000000in}{0.000000in}}%
\pgfpathlineto{\pgfqpoint{-0.048611in}{0.000000in}}%
\pgfusepath{stroke,fill}%
}%
\begin{pgfscope}%
\pgfsys@transformshift{0.750000in}{1.186364in}%
\pgfsys@useobject{currentmarker}{}%
\end{pgfscope}%
\end{pgfscope}%
\begin{pgfscope}%
\definecolor{textcolor}{rgb}{0.000000,0.000000,0.000000}%
\pgfsetstrokecolor{textcolor}%
\pgfsetfillcolor{textcolor}%
\pgftext[x=0.475308in, y=1.138138in, left, base]{\color{textcolor}\rmfamily\fontsize{10.000000}{12.000000}\selectfont \(\displaystyle {0.2}\)}%
\end{pgfscope}%
\begin{pgfscope}%
\pgfsetbuttcap%
\pgfsetroundjoin%
\definecolor{currentfill}{rgb}{0.000000,0.000000,0.000000}%
\pgfsetfillcolor{currentfill}%
\pgfsetlinewidth{0.803000pt}%
\definecolor{currentstroke}{rgb}{0.000000,0.000000,0.000000}%
\pgfsetstrokecolor{currentstroke}%
\pgfsetdash{}{0pt}%
\pgfsys@defobject{currentmarker}{\pgfqpoint{-0.048611in}{0.000000in}}{\pgfqpoint{-0.000000in}{0.000000in}}{%
\pgfpathmoveto{\pgfqpoint{-0.000000in}{0.000000in}}%
\pgfpathlineto{\pgfqpoint{-0.048611in}{0.000000in}}%
\pgfusepath{stroke,fill}%
}%
\begin{pgfscope}%
\pgfsys@transformshift{0.750000in}{1.735455in}%
\pgfsys@useobject{currentmarker}{}%
\end{pgfscope}%
\end{pgfscope}%
\begin{pgfscope}%
\definecolor{textcolor}{rgb}{0.000000,0.000000,0.000000}%
\pgfsetstrokecolor{textcolor}%
\pgfsetfillcolor{textcolor}%
\pgftext[x=0.475308in, y=1.687229in, left, base]{\color{textcolor}\rmfamily\fontsize{10.000000}{12.000000}\selectfont \(\displaystyle {0.4}\)}%
\end{pgfscope}%
\begin{pgfscope}%
\pgfsetbuttcap%
\pgfsetroundjoin%
\definecolor{currentfill}{rgb}{0.000000,0.000000,0.000000}%
\pgfsetfillcolor{currentfill}%
\pgfsetlinewidth{0.803000pt}%
\definecolor{currentstroke}{rgb}{0.000000,0.000000,0.000000}%
\pgfsetstrokecolor{currentstroke}%
\pgfsetdash{}{0pt}%
\pgfsys@defobject{currentmarker}{\pgfqpoint{-0.048611in}{0.000000in}}{\pgfqpoint{-0.000000in}{0.000000in}}{%
\pgfpathmoveto{\pgfqpoint{-0.000000in}{0.000000in}}%
\pgfpathlineto{\pgfqpoint{-0.048611in}{0.000000in}}%
\pgfusepath{stroke,fill}%
}%
\begin{pgfscope}%
\pgfsys@transformshift{0.750000in}{2.284545in}%
\pgfsys@useobject{currentmarker}{}%
\end{pgfscope}%
\end{pgfscope}%
\begin{pgfscope}%
\definecolor{textcolor}{rgb}{0.000000,0.000000,0.000000}%
\pgfsetstrokecolor{textcolor}%
\pgfsetfillcolor{textcolor}%
\pgftext[x=0.475308in, y=2.236320in, left, base]{\color{textcolor}\rmfamily\fontsize{10.000000}{12.000000}\selectfont \(\displaystyle {0.6}\)}%
\end{pgfscope}%
\begin{pgfscope}%
\pgfsetbuttcap%
\pgfsetroundjoin%
\definecolor{currentfill}{rgb}{0.000000,0.000000,0.000000}%
\pgfsetfillcolor{currentfill}%
\pgfsetlinewidth{0.803000pt}%
\definecolor{currentstroke}{rgb}{0.000000,0.000000,0.000000}%
\pgfsetstrokecolor{currentstroke}%
\pgfsetdash{}{0pt}%
\pgfsys@defobject{currentmarker}{\pgfqpoint{-0.048611in}{0.000000in}}{\pgfqpoint{-0.000000in}{0.000000in}}{%
\pgfpathmoveto{\pgfqpoint{-0.000000in}{0.000000in}}%
\pgfpathlineto{\pgfqpoint{-0.048611in}{0.000000in}}%
\pgfusepath{stroke,fill}%
}%
\begin{pgfscope}%
\pgfsys@transformshift{0.750000in}{2.833636in}%
\pgfsys@useobject{currentmarker}{}%
\end{pgfscope}%
\end{pgfscope}%
\begin{pgfscope}%
\definecolor{textcolor}{rgb}{0.000000,0.000000,0.000000}%
\pgfsetstrokecolor{textcolor}%
\pgfsetfillcolor{textcolor}%
\pgftext[x=0.475308in, y=2.785411in, left, base]{\color{textcolor}\rmfamily\fontsize{10.000000}{12.000000}\selectfont \(\displaystyle {0.8}\)}%
\end{pgfscope}%
\begin{pgfscope}%
\pgfsetbuttcap%
\pgfsetroundjoin%
\definecolor{currentfill}{rgb}{0.000000,0.000000,0.000000}%
\pgfsetfillcolor{currentfill}%
\pgfsetlinewidth{0.803000pt}%
\definecolor{currentstroke}{rgb}{0.000000,0.000000,0.000000}%
\pgfsetstrokecolor{currentstroke}%
\pgfsetdash{}{0pt}%
\pgfsys@defobject{currentmarker}{\pgfqpoint{-0.048611in}{0.000000in}}{\pgfqpoint{-0.000000in}{0.000000in}}{%
\pgfpathmoveto{\pgfqpoint{-0.000000in}{0.000000in}}%
\pgfpathlineto{\pgfqpoint{-0.048611in}{0.000000in}}%
\pgfusepath{stroke,fill}%
}%
\begin{pgfscope}%
\pgfsys@transformshift{0.750000in}{3.382727in}%
\pgfsys@useobject{currentmarker}{}%
\end{pgfscope}%
\end{pgfscope}%
\begin{pgfscope}%
\definecolor{textcolor}{rgb}{0.000000,0.000000,0.000000}%
\pgfsetstrokecolor{textcolor}%
\pgfsetfillcolor{textcolor}%
\pgftext[x=0.475308in, y=3.334502in, left, base]{\color{textcolor}\rmfamily\fontsize{10.000000}{12.000000}\selectfont \(\displaystyle {1.0}\)}%
\end{pgfscope}%
\begin{pgfscope}%
\definecolor{textcolor}{rgb}{0.000000,0.000000,0.000000}%
\pgfsetstrokecolor{textcolor}%
\pgfsetfillcolor{textcolor}%
\pgftext[x=0.419753in,y=2.010000in,,bottom,rotate=90.000000]{\color{textcolor}\rmfamily\fontsize{10.000000}{12.000000}\selectfont sus}%
\end{pgfscope}%
\begin{pgfscope}%
\pgfpathrectangle{\pgfqpoint{0.750000in}{0.500000in}}{\pgfqpoint{4.650000in}{3.020000in}}%
\pgfusepath{clip}%
\pgfsetrectcap%
\pgfsetroundjoin%
\pgfsetlinewidth{1.505625pt}%
\definecolor{currentstroke}{rgb}{0.121569,0.466667,0.705882}%
\pgfsetstrokecolor{currentstroke}%
\pgfsetdash{}{0pt}%
\pgfpathmoveto{\pgfqpoint{0.961364in}{0.637273in}}%
\pgfpathlineto{\pgfqpoint{0.961619in}{0.637273in}}%
\pgfpathlineto{\pgfqpoint{0.962941in}{3.382727in}}%
\pgfpathlineto{\pgfqpoint{0.963158in}{0.637273in}}%
\pgfpathlineto{\pgfqpoint{0.963328in}{0.637273in}}%
\pgfpathlineto{\pgfqpoint{0.964372in}{3.382727in}}%
\pgfpathlineto{\pgfqpoint{0.964868in}{0.637273in}}%
\pgfpathlineto{\pgfqpoint{0.964929in}{0.637273in}}%
\pgfpathlineto{\pgfqpoint{0.966473in}{3.382727in}}%
\pgfpathlineto{\pgfqpoint{0.966752in}{3.382727in}}%
\pgfpathlineto{\pgfqpoint{0.968296in}{0.637273in}}%
\pgfpathlineto{\pgfqpoint{0.968683in}{0.637273in}}%
\pgfpathlineto{\pgfqpoint{0.968693in}{3.382727in}}%
\pgfpathlineto{\pgfqpoint{0.970223in}{0.637273in}}%
\pgfpathlineto{\pgfqpoint{0.970327in}{0.637273in}}%
\pgfpathlineto{\pgfqpoint{0.970520in}{3.382727in}}%
\pgfpathlineto{\pgfqpoint{0.971866in}{0.637273in}}%
\pgfpathlineto{\pgfqpoint{0.971918in}{0.637273in}}%
\pgfpathlineto{\pgfqpoint{0.972716in}{3.382727in}}%
\pgfpathlineto{\pgfqpoint{0.973457in}{0.637273in}}%
\pgfpathlineto{\pgfqpoint{0.986151in}{0.637273in}}%
\pgfpathlineto{\pgfqpoint{0.986217in}{3.382727in}}%
\pgfpathlineto{\pgfqpoint{0.987691in}{0.637273in}}%
\pgfpathlineto{\pgfqpoint{0.998580in}{0.637273in}}%
\pgfpathlineto{\pgfqpoint{0.998627in}{3.382727in}}%
\pgfpathlineto{\pgfqpoint{1.000120in}{0.637273in}}%
\pgfpathlineto{\pgfqpoint{1.004308in}{0.637273in}}%
\pgfpathlineto{\pgfqpoint{1.004337in}{3.382727in}}%
\pgfpathlineto{\pgfqpoint{1.005848in}{0.637273in}}%
\pgfpathlineto{\pgfqpoint{1.015968in}{0.637273in}}%
\pgfpathlineto{\pgfqpoint{1.016015in}{3.382727in}}%
\pgfpathlineto{\pgfqpoint{1.017507in}{0.637273in}}%
\pgfpathlineto{\pgfqpoint{1.029034in}{0.637273in}}%
\pgfpathlineto{\pgfqpoint{1.029072in}{3.382727in}}%
\pgfpathlineto{\pgfqpoint{1.030574in}{0.637273in}}%
\pgfpathlineto{\pgfqpoint{1.039981in}{0.637273in}}%
\pgfpathlineto{\pgfqpoint{1.040047in}{3.382727in}}%
\pgfpathlineto{\pgfqpoint{1.041520in}{0.637273in}}%
\pgfpathlineto{\pgfqpoint{1.050804in}{0.637273in}}%
\pgfpathlineto{\pgfqpoint{1.050832in}{3.382727in}}%
\pgfpathlineto{\pgfqpoint{1.052344in}{0.637273in}}%
\pgfpathlineto{\pgfqpoint{1.074793in}{0.637273in}}%
\pgfpathlineto{\pgfqpoint{1.074841in}{3.382727in}}%
\pgfpathlineto{\pgfqpoint{1.076333in}{0.637273in}}%
\pgfpathlineto{\pgfqpoint{1.085825in}{0.637273in}}%
\pgfpathlineto{\pgfqpoint{1.085853in}{3.382727in}}%
\pgfpathlineto{\pgfqpoint{1.087364in}{0.637273in}}%
\pgfpathlineto{\pgfqpoint{1.096941in}{0.637273in}}%
\pgfpathlineto{\pgfqpoint{1.096951in}{3.382727in}}%
\pgfpathlineto{\pgfqpoint{1.098481in}{0.637273in}}%
\pgfpathlineto{\pgfqpoint{1.107812in}{0.637273in}}%
\pgfpathlineto{\pgfqpoint{1.109153in}{3.382727in}}%
\pgfpathlineto{\pgfqpoint{1.109351in}{0.637273in}}%
\pgfpathlineto{\pgfqpoint{1.109455in}{0.637273in}}%
\pgfpathlineto{\pgfqpoint{1.110999in}{3.382727in}}%
\pgfpathlineto{\pgfqpoint{1.111160in}{0.637273in}}%
\pgfpathlineto{\pgfqpoint{1.112539in}{3.382727in}}%
\pgfpathlineto{\pgfqpoint{1.113946in}{3.382727in}}%
\pgfpathlineto{\pgfqpoint{1.115490in}{0.637273in}}%
\pgfpathlineto{\pgfqpoint{1.115566in}{0.637273in}}%
\pgfpathlineto{\pgfqpoint{1.117105in}{3.382727in}}%
\pgfpathlineto{\pgfqpoint{1.118650in}{0.637273in}}%
\pgfpathlineto{\pgfqpoint{1.120194in}{3.382727in}}%
\pgfpathlineto{\pgfqpoint{1.121157in}{3.382727in}}%
\pgfpathlineto{\pgfqpoint{1.122701in}{0.637273in}}%
\pgfpathlineto{\pgfqpoint{1.126904in}{0.637273in}}%
\pgfpathlineto{\pgfqpoint{1.126951in}{3.382727in}}%
\pgfpathlineto{\pgfqpoint{1.128444in}{0.637273in}}%
\pgfpathlineto{\pgfqpoint{1.138946in}{0.637273in}}%
\pgfpathlineto{\pgfqpoint{1.138993in}{3.382727in}}%
\pgfpathlineto{\pgfqpoint{1.140485in}{0.637273in}}%
\pgfpathlineto{\pgfqpoint{1.149883in}{0.637273in}}%
\pgfpathlineto{\pgfqpoint{1.149930in}{3.382727in}}%
\pgfpathlineto{\pgfqpoint{1.151422in}{0.637273in}}%
\pgfpathlineto{\pgfqpoint{1.162029in}{0.637273in}}%
\pgfpathlineto{\pgfqpoint{1.162057in}{3.382727in}}%
\pgfpathlineto{\pgfqpoint{1.163568in}{0.637273in}}%
\pgfpathlineto{\pgfqpoint{1.172951in}{0.637273in}}%
\pgfpathlineto{\pgfqpoint{1.172980in}{3.382727in}}%
\pgfpathlineto{\pgfqpoint{1.174491in}{0.637273in}}%
\pgfpathlineto{\pgfqpoint{1.183784in}{0.637273in}}%
\pgfpathlineto{\pgfqpoint{1.183813in}{3.382727in}}%
\pgfpathlineto{\pgfqpoint{1.185324in}{0.637273in}}%
\pgfpathlineto{\pgfqpoint{1.196237in}{0.637273in}}%
\pgfpathlineto{\pgfqpoint{1.196265in}{3.382727in}}%
\pgfpathlineto{\pgfqpoint{1.197777in}{0.637273in}}%
\pgfpathlineto{\pgfqpoint{1.207160in}{0.637273in}}%
\pgfpathlineto{\pgfqpoint{1.207188in}{3.382727in}}%
\pgfpathlineto{\pgfqpoint{1.208699in}{0.637273in}}%
\pgfpathlineto{\pgfqpoint{1.218285in}{0.637273in}}%
\pgfpathlineto{\pgfqpoint{1.218314in}{3.382727in}}%
\pgfpathlineto{\pgfqpoint{1.219825in}{0.637273in}}%
\pgfpathlineto{\pgfqpoint{1.229203in}{0.637273in}}%
\pgfpathlineto{\pgfqpoint{1.229232in}{3.382727in}}%
\pgfpathlineto{\pgfqpoint{1.230743in}{0.637273in}}%
\pgfpathlineto{\pgfqpoint{1.240060in}{0.637273in}}%
\pgfpathlineto{\pgfqpoint{1.240088in}{3.382727in}}%
\pgfpathlineto{\pgfqpoint{1.241600in}{0.637273in}}%
\pgfpathlineto{\pgfqpoint{1.251427in}{0.637273in}}%
\pgfpathlineto{\pgfqpoint{1.251455in}{3.382727in}}%
\pgfpathlineto{\pgfqpoint{1.252966in}{0.637273in}}%
\pgfpathlineto{\pgfqpoint{1.262279in}{0.637273in}}%
\pgfpathlineto{\pgfqpoint{1.262302in}{3.382727in}}%
\pgfpathlineto{\pgfqpoint{1.263818in}{0.637273in}}%
\pgfpathlineto{\pgfqpoint{1.273329in}{0.637273in}}%
\pgfpathlineto{\pgfqpoint{1.273357in}{3.382727in}}%
\pgfpathlineto{\pgfqpoint{1.274868in}{0.637273in}}%
\pgfpathlineto{\pgfqpoint{1.284171in}{0.637273in}}%
\pgfpathlineto{\pgfqpoint{1.284199in}{3.382727in}}%
\pgfpathlineto{\pgfqpoint{1.285711in}{0.637273in}}%
\pgfpathlineto{\pgfqpoint{1.295080in}{0.637273in}}%
\pgfpathlineto{\pgfqpoint{1.295108in}{3.382727in}}%
\pgfpathlineto{\pgfqpoint{1.296619in}{0.637273in}}%
\pgfpathlineto{\pgfqpoint{1.307122in}{0.637273in}}%
\pgfpathlineto{\pgfqpoint{1.307150in}{3.382727in}}%
\pgfpathlineto{\pgfqpoint{1.308661in}{0.637273in}}%
\pgfpathlineto{\pgfqpoint{1.318044in}{0.637273in}}%
\pgfpathlineto{\pgfqpoint{1.318073in}{3.382727in}}%
\pgfpathlineto{\pgfqpoint{1.319584in}{0.637273in}}%
\pgfpathlineto{\pgfqpoint{1.330190in}{0.637273in}}%
\pgfpathlineto{\pgfqpoint{1.330218in}{3.382727in}}%
\pgfpathlineto{\pgfqpoint{1.331729in}{0.637273in}}%
\pgfpathlineto{\pgfqpoint{1.341037in}{0.637273in}}%
\pgfpathlineto{\pgfqpoint{1.341065in}{3.382727in}}%
\pgfpathlineto{\pgfqpoint{1.342577in}{0.637273in}}%
\pgfpathlineto{\pgfqpoint{1.351946in}{0.637273in}}%
\pgfpathlineto{\pgfqpoint{1.351974in}{3.382727in}}%
\pgfpathlineto{\pgfqpoint{1.353485in}{0.637273in}}%
\pgfpathlineto{\pgfqpoint{1.363076in}{0.637273in}}%
\pgfpathlineto{\pgfqpoint{1.363105in}{3.382727in}}%
\pgfpathlineto{\pgfqpoint{1.364616in}{0.637273in}}%
\pgfpathlineto{\pgfqpoint{1.373952in}{0.637273in}}%
\pgfpathlineto{\pgfqpoint{1.373961in}{3.382727in}}%
\pgfpathlineto{\pgfqpoint{1.375491in}{0.637273in}}%
\pgfpathlineto{\pgfqpoint{1.381545in}{0.637273in}}%
\pgfpathlineto{\pgfqpoint{1.381573in}{3.382727in}}%
\pgfpathlineto{\pgfqpoint{1.383085in}{0.637273in}}%
\pgfpathlineto{\pgfqpoint{1.392463in}{0.637273in}}%
\pgfpathlineto{\pgfqpoint{1.392491in}{3.382727in}}%
\pgfpathlineto{\pgfqpoint{1.394003in}{0.637273in}}%
\pgfpathlineto{\pgfqpoint{1.403334in}{0.637273in}}%
\pgfpathlineto{\pgfqpoint{1.403362in}{3.382727in}}%
\pgfpathlineto{\pgfqpoint{1.404873in}{0.637273in}}%
\pgfpathlineto{\pgfqpoint{1.415395in}{0.637273in}}%
\pgfpathlineto{\pgfqpoint{1.415423in}{3.382727in}}%
\pgfpathlineto{\pgfqpoint{1.416934in}{0.637273in}}%
\pgfpathlineto{\pgfqpoint{1.426341in}{0.637273in}}%
\pgfpathlineto{\pgfqpoint{1.426369in}{3.382727in}}%
\pgfpathlineto{\pgfqpoint{1.427880in}{0.637273in}}%
\pgfpathlineto{\pgfqpoint{1.437443in}{0.637273in}}%
\pgfpathlineto{\pgfqpoint{1.437471in}{3.382727in}}%
\pgfpathlineto{\pgfqpoint{1.438983in}{0.637273in}}%
\pgfpathlineto{\pgfqpoint{1.448333in}{0.637273in}}%
\pgfpathlineto{\pgfqpoint{1.448356in}{3.382727in}}%
\pgfpathlineto{\pgfqpoint{1.449872in}{0.637273in}}%
\pgfpathlineto{\pgfqpoint{1.459213in}{0.637273in}}%
\pgfpathlineto{\pgfqpoint{1.459241in}{3.382727in}}%
\pgfpathlineto{\pgfqpoint{1.460752in}{0.637273in}}%
\pgfpathlineto{\pgfqpoint{1.470329in}{0.637273in}}%
\pgfpathlineto{\pgfqpoint{1.470358in}{3.382727in}}%
\pgfpathlineto{\pgfqpoint{1.471869in}{0.637273in}}%
\pgfpathlineto{\pgfqpoint{1.481124in}{0.637273in}}%
\pgfpathlineto{\pgfqpoint{1.481153in}{3.382727in}}%
\pgfpathlineto{\pgfqpoint{1.482664in}{0.637273in}}%
\pgfpathlineto{\pgfqpoint{1.487618in}{0.637273in}}%
\pgfpathlineto{\pgfqpoint{1.489053in}{3.382727in}}%
\pgfpathlineto{\pgfqpoint{1.489157in}{0.637273in}}%
\pgfpathlineto{\pgfqpoint{1.494734in}{0.637273in}}%
\pgfpathlineto{\pgfqpoint{1.494762in}{3.382727in}}%
\pgfpathlineto{\pgfqpoint{1.496274in}{0.637273in}}%
\pgfpathlineto{\pgfqpoint{1.505529in}{0.637273in}}%
\pgfpathlineto{\pgfqpoint{1.505558in}{3.382727in}}%
\pgfpathlineto{\pgfqpoint{1.507069in}{0.637273in}}%
\pgfpathlineto{\pgfqpoint{1.516438in}{0.637273in}}%
\pgfpathlineto{\pgfqpoint{1.516466in}{3.382727in}}%
\pgfpathlineto{\pgfqpoint{1.517977in}{0.637273in}}%
\pgfpathlineto{\pgfqpoint{1.528513in}{0.637273in}}%
\pgfpathlineto{\pgfqpoint{1.528541in}{3.382727in}}%
\pgfpathlineto{\pgfqpoint{1.530052in}{0.637273in}}%
\pgfpathlineto{\pgfqpoint{1.539388in}{0.637273in}}%
\pgfpathlineto{\pgfqpoint{1.540829in}{3.382727in}}%
\pgfpathlineto{\pgfqpoint{1.540928in}{0.637273in}}%
\pgfpathlineto{\pgfqpoint{1.551879in}{0.637273in}}%
\pgfpathlineto{\pgfqpoint{1.551907in}{3.382727in}}%
\pgfpathlineto{\pgfqpoint{1.553418in}{0.637273in}}%
\pgfpathlineto{\pgfqpoint{1.562797in}{0.637273in}}%
\pgfpathlineto{\pgfqpoint{1.562825in}{3.382727in}}%
\pgfpathlineto{\pgfqpoint{1.564336in}{0.637273in}}%
\pgfpathlineto{\pgfqpoint{1.573677in}{0.637273in}}%
\pgfpathlineto{\pgfqpoint{1.573705in}{3.382727in}}%
\pgfpathlineto{\pgfqpoint{1.575216in}{0.637273in}}%
\pgfpathlineto{\pgfqpoint{1.584765in}{0.637273in}}%
\pgfpathlineto{\pgfqpoint{1.584793in}{3.382727in}}%
\pgfpathlineto{\pgfqpoint{1.586304in}{0.637273in}}%
\pgfpathlineto{\pgfqpoint{1.595636in}{0.637273in}}%
\pgfpathlineto{\pgfqpoint{1.595664in}{3.382727in}}%
\pgfpathlineto{\pgfqpoint{1.597175in}{0.637273in}}%
\pgfpathlineto{\pgfqpoint{1.606766in}{0.637273in}}%
\pgfpathlineto{\pgfqpoint{1.606794in}{3.382727in}}%
\pgfpathlineto{\pgfqpoint{1.608306in}{0.637273in}}%
\pgfpathlineto{\pgfqpoint{1.617717in}{0.637273in}}%
\pgfpathlineto{\pgfqpoint{1.617745in}{3.382727in}}%
\pgfpathlineto{\pgfqpoint{1.619257in}{0.637273in}}%
\pgfpathlineto{\pgfqpoint{1.628597in}{0.637273in}}%
\pgfpathlineto{\pgfqpoint{1.628607in}{3.382727in}}%
\pgfpathlineto{\pgfqpoint{1.630137in}{0.637273in}}%
\pgfpathlineto{\pgfqpoint{1.639728in}{0.637273in}}%
\pgfpathlineto{\pgfqpoint{1.639737in}{3.382727in}}%
\pgfpathlineto{\pgfqpoint{1.641267in}{0.637273in}}%
\pgfpathlineto{\pgfqpoint{1.645253in}{0.637273in}}%
\pgfpathlineto{\pgfqpoint{1.646674in}{3.382727in}}%
\pgfpathlineto{\pgfqpoint{1.646792in}{0.637273in}}%
\pgfpathlineto{\pgfqpoint{1.646939in}{0.637273in}}%
\pgfpathlineto{\pgfqpoint{1.647099in}{3.382727in}}%
\pgfpathlineto{\pgfqpoint{1.648478in}{0.637273in}}%
\pgfpathlineto{\pgfqpoint{1.659188in}{0.637273in}}%
\pgfpathlineto{\pgfqpoint{1.659217in}{3.382727in}}%
\pgfpathlineto{\pgfqpoint{1.660728in}{0.637273in}}%
\pgfpathlineto{\pgfqpoint{1.670078in}{0.637273in}}%
\pgfpathlineto{\pgfqpoint{1.670106in}{3.382727in}}%
\pgfpathlineto{\pgfqpoint{1.671618in}{0.637273in}}%
\pgfpathlineto{\pgfqpoint{1.681336in}{0.637273in}}%
\pgfpathlineto{\pgfqpoint{1.681364in}{3.382727in}}%
\pgfpathlineto{\pgfqpoint{1.682876in}{0.637273in}}%
\pgfpathlineto{\pgfqpoint{1.686653in}{0.637273in}}%
\pgfpathlineto{\pgfqpoint{1.688146in}{3.382727in}}%
\pgfpathlineto{\pgfqpoint{1.688193in}{0.637273in}}%
\pgfpathlineto{\pgfqpoint{1.688476in}{0.637273in}}%
\pgfpathlineto{\pgfqpoint{1.689855in}{3.382727in}}%
\pgfpathlineto{\pgfqpoint{1.690016in}{0.637273in}}%
\pgfpathlineto{\pgfqpoint{1.690020in}{0.637273in}}%
\pgfpathlineto{\pgfqpoint{1.691565in}{3.382727in}}%
\pgfpathlineto{\pgfqpoint{1.691569in}{3.382727in}}%
\pgfpathlineto{\pgfqpoint{1.691990in}{0.637273in}}%
\pgfpathlineto{\pgfqpoint{1.693109in}{3.382727in}}%
\pgfpathlineto{\pgfqpoint{1.693369in}{3.382727in}}%
\pgfpathlineto{\pgfqpoint{1.694913in}{0.637273in}}%
\pgfpathlineto{\pgfqpoint{1.698988in}{0.637273in}}%
\pgfpathlineto{\pgfqpoint{1.700367in}{3.382727in}}%
\pgfpathlineto{\pgfqpoint{1.700528in}{0.637273in}}%
\pgfpathlineto{\pgfqpoint{1.700594in}{0.637273in}}%
\pgfpathlineto{\pgfqpoint{1.702110in}{3.382727in}}%
\pgfpathlineto{\pgfqpoint{1.702133in}{0.637273in}}%
\pgfpathlineto{\pgfqpoint{1.702180in}{0.637273in}}%
\pgfpathlineto{\pgfqpoint{1.702520in}{3.382727in}}%
\pgfpathlineto{\pgfqpoint{1.703720in}{0.637273in}}%
\pgfpathlineto{\pgfqpoint{1.713377in}{0.637273in}}%
\pgfpathlineto{\pgfqpoint{1.713419in}{3.382727in}}%
\pgfpathlineto{\pgfqpoint{1.714916in}{0.637273in}}%
\pgfpathlineto{\pgfqpoint{1.726184in}{0.637273in}}%
\pgfpathlineto{\pgfqpoint{1.726231in}{3.382727in}}%
\pgfpathlineto{\pgfqpoint{1.727723in}{0.637273in}}%
\pgfpathlineto{\pgfqpoint{1.737007in}{0.637273in}}%
\pgfpathlineto{\pgfqpoint{1.737055in}{3.382727in}}%
\pgfpathlineto{\pgfqpoint{1.738547in}{0.637273in}}%
\pgfpathlineto{\pgfqpoint{1.747855in}{0.637273in}}%
\pgfpathlineto{\pgfqpoint{1.747902in}{3.382727in}}%
\pgfpathlineto{\pgfqpoint{1.749394in}{0.637273in}}%
\pgfpathlineto{\pgfqpoint{1.759934in}{0.637273in}}%
\pgfpathlineto{\pgfqpoint{1.759981in}{3.382727in}}%
\pgfpathlineto{\pgfqpoint{1.761474in}{0.637273in}}%
\pgfpathlineto{\pgfqpoint{1.770838in}{0.637273in}}%
\pgfpathlineto{\pgfqpoint{1.770885in}{3.382727in}}%
\pgfpathlineto{\pgfqpoint{1.772377in}{0.637273in}}%
\pgfpathlineto{\pgfqpoint{1.781888in}{0.637273in}}%
\pgfpathlineto{\pgfqpoint{1.781954in}{3.382727in}}%
\pgfpathlineto{\pgfqpoint{1.783428in}{0.637273in}}%
\pgfpathlineto{\pgfqpoint{1.792806in}{0.637273in}}%
\pgfpathlineto{\pgfqpoint{1.792853in}{3.382727in}}%
\pgfpathlineto{\pgfqpoint{1.794346in}{0.637273in}}%
\pgfpathlineto{\pgfqpoint{1.803686in}{0.637273in}}%
\pgfpathlineto{\pgfqpoint{1.803752in}{3.382727in}}%
\pgfpathlineto{\pgfqpoint{1.805226in}{0.637273in}}%
\pgfpathlineto{\pgfqpoint{1.814774in}{0.637273in}}%
\pgfpathlineto{\pgfqpoint{1.814840in}{3.382727in}}%
\pgfpathlineto{\pgfqpoint{1.816314in}{0.637273in}}%
\pgfpathlineto{\pgfqpoint{1.825673in}{0.637273in}}%
\pgfpathlineto{\pgfqpoint{1.825740in}{3.382727in}}%
\pgfpathlineto{\pgfqpoint{1.827213in}{0.637273in}}%
\pgfpathlineto{\pgfqpoint{1.837819in}{0.637273in}}%
\pgfpathlineto{\pgfqpoint{1.837866in}{3.382727in}}%
\pgfpathlineto{\pgfqpoint{1.839359in}{0.637273in}}%
\pgfpathlineto{\pgfqpoint{1.848737in}{0.637273in}}%
\pgfpathlineto{\pgfqpoint{1.848784in}{3.382727in}}%
\pgfpathlineto{\pgfqpoint{1.850277in}{0.637273in}}%
\pgfpathlineto{\pgfqpoint{1.859547in}{0.637273in}}%
\pgfpathlineto{\pgfqpoint{1.859594in}{3.382727in}}%
\pgfpathlineto{\pgfqpoint{1.861086in}{0.637273in}}%
\pgfpathlineto{\pgfqpoint{1.870705in}{0.637273in}}%
\pgfpathlineto{\pgfqpoint{1.870753in}{3.382727in}}%
\pgfpathlineto{\pgfqpoint{1.872245in}{0.637273in}}%
\pgfpathlineto{\pgfqpoint{1.881581in}{0.637273in}}%
\pgfpathlineto{\pgfqpoint{1.881628in}{3.382727in}}%
\pgfpathlineto{\pgfqpoint{1.883120in}{0.637273in}}%
\pgfpathlineto{\pgfqpoint{1.892730in}{0.637273in}}%
\pgfpathlineto{\pgfqpoint{1.892740in}{3.382727in}}%
\pgfpathlineto{\pgfqpoint{1.894270in}{0.637273in}}%
\pgfpathlineto{\pgfqpoint{1.903629in}{0.637273in}}%
\pgfpathlineto{\pgfqpoint{1.903639in}{3.382727in}}%
\pgfpathlineto{\pgfqpoint{1.905169in}{0.637273in}}%
\pgfpathlineto{\pgfqpoint{1.909178in}{0.637273in}}%
\pgfpathlineto{\pgfqpoint{1.909225in}{3.382727in}}%
\pgfpathlineto{\pgfqpoint{1.910717in}{0.637273in}}%
\pgfpathlineto{\pgfqpoint{1.920261in}{0.637273in}}%
\pgfpathlineto{\pgfqpoint{1.920308in}{3.382727in}}%
\pgfpathlineto{\pgfqpoint{1.921801in}{0.637273in}}%
\pgfpathlineto{\pgfqpoint{1.925673in}{0.637273in}}%
\pgfpathlineto{\pgfqpoint{1.927203in}{3.382727in}}%
\pgfpathlineto{\pgfqpoint{1.927212in}{0.637273in}}%
\pgfpathlineto{\pgfqpoint{1.927467in}{0.637273in}}%
\pgfpathlineto{\pgfqpoint{1.928946in}{3.382727in}}%
\pgfpathlineto{\pgfqpoint{1.929007in}{0.637273in}}%
\pgfpathlineto{\pgfqpoint{1.929064in}{0.637273in}}%
\pgfpathlineto{\pgfqpoint{1.930532in}{3.382727in}}%
\pgfpathlineto{\pgfqpoint{1.930603in}{0.637273in}}%
\pgfpathlineto{\pgfqpoint{1.930679in}{0.637273in}}%
\pgfpathlineto{\pgfqpoint{1.932223in}{3.382727in}}%
\pgfpathlineto{\pgfqpoint{1.932563in}{3.382727in}}%
\pgfpathlineto{\pgfqpoint{1.934102in}{0.637273in}}%
\pgfpathlineto{\pgfqpoint{1.935647in}{3.382727in}}%
\pgfpathlineto{\pgfqpoint{1.937129in}{0.637273in}}%
\pgfpathlineto{\pgfqpoint{1.937186in}{3.382727in}}%
\pgfpathlineto{\pgfqpoint{1.937191in}{3.382727in}}%
\pgfpathlineto{\pgfqpoint{1.938735in}{0.637273in}}%
\pgfpathlineto{\pgfqpoint{1.940274in}{3.382727in}}%
\pgfpathlineto{\pgfqpoint{1.941804in}{0.637273in}}%
\pgfpathlineto{\pgfqpoint{1.941814in}{3.382727in}}%
\pgfpathlineto{\pgfqpoint{1.941819in}{3.382727in}}%
\pgfpathlineto{\pgfqpoint{1.943297in}{0.637273in}}%
\pgfpathlineto{\pgfqpoint{1.943358in}{3.382727in}}%
\pgfpathlineto{\pgfqpoint{1.944638in}{3.382727in}}%
\pgfpathlineto{\pgfqpoint{1.946182in}{0.637273in}}%
\pgfpathlineto{\pgfqpoint{1.946602in}{3.382727in}}%
\pgfpathlineto{\pgfqpoint{1.947721in}{0.637273in}}%
\pgfpathlineto{\pgfqpoint{1.950607in}{0.637273in}}%
\pgfpathlineto{\pgfqpoint{1.951948in}{3.382727in}}%
\pgfpathlineto{\pgfqpoint{1.952146in}{0.637273in}}%
\pgfpathlineto{\pgfqpoint{1.952217in}{0.637273in}}%
\pgfpathlineto{\pgfqpoint{1.953761in}{3.382727in}}%
\pgfpathlineto{\pgfqpoint{1.954007in}{3.382727in}}%
\pgfpathlineto{\pgfqpoint{1.955348in}{0.637273in}}%
\pgfpathlineto{\pgfqpoint{1.955546in}{3.382727in}}%
\pgfpathlineto{\pgfqpoint{1.956689in}{3.382727in}}%
\pgfpathlineto{\pgfqpoint{1.958233in}{0.637273in}}%
\pgfpathlineto{\pgfqpoint{1.959763in}{3.382727in}}%
\pgfpathlineto{\pgfqpoint{1.959773in}{0.637273in}}%
\pgfpathlineto{\pgfqpoint{1.960713in}{0.637273in}}%
\pgfpathlineto{\pgfqpoint{1.962257in}{3.382727in}}%
\pgfpathlineto{\pgfqpoint{1.962290in}{3.382727in}}%
\pgfpathlineto{\pgfqpoint{1.963834in}{0.637273in}}%
\pgfpathlineto{\pgfqpoint{1.964859in}{0.637273in}}%
\pgfpathlineto{\pgfqpoint{1.966403in}{3.382727in}}%
\pgfpathlineto{\pgfqpoint{1.966955in}{3.382727in}}%
\pgfpathlineto{\pgfqpoint{1.967196in}{0.637273in}}%
\pgfpathlineto{\pgfqpoint{1.968495in}{3.382727in}}%
\pgfpathlineto{\pgfqpoint{1.981486in}{3.382727in}}%
\pgfpathlineto{\pgfqpoint{1.981580in}{0.637273in}}%
\pgfpathlineto{\pgfqpoint{1.983025in}{3.382727in}}%
\pgfpathlineto{\pgfqpoint{1.985410in}{3.382727in}}%
\pgfpathlineto{\pgfqpoint{1.986954in}{0.637273in}}%
\pgfpathlineto{\pgfqpoint{1.986969in}{0.637273in}}%
\pgfpathlineto{\pgfqpoint{1.988513in}{3.382727in}}%
\pgfpathlineto{\pgfqpoint{1.988527in}{3.382727in}}%
\pgfpathlineto{\pgfqpoint{1.989854in}{0.637273in}}%
\pgfpathlineto{\pgfqpoint{1.990066in}{3.382727in}}%
\pgfpathlineto{\pgfqpoint{2.049959in}{3.382727in}}%
\pgfpathlineto{\pgfqpoint{2.051267in}{0.637273in}}%
\pgfpathlineto{\pgfqpoint{2.051499in}{3.382727in}}%
\pgfpathlineto{\pgfqpoint{2.053227in}{3.382727in}}%
\pgfpathlineto{\pgfqpoint{2.054771in}{0.637273in}}%
\pgfpathlineto{\pgfqpoint{2.055315in}{0.637273in}}%
\pgfpathlineto{\pgfqpoint{2.056859in}{3.382727in}}%
\pgfpathlineto{\pgfqpoint{2.074435in}{3.382727in}}%
\pgfpathlineto{\pgfqpoint{2.075979in}{0.637273in}}%
\pgfpathlineto{\pgfqpoint{2.080338in}{0.637273in}}%
\pgfpathlineto{\pgfqpoint{2.081849in}{3.382727in}}%
\pgfpathlineto{\pgfqpoint{2.081877in}{0.637273in}}%
\pgfpathlineto{\pgfqpoint{2.081948in}{0.637273in}}%
\pgfpathlineto{\pgfqpoint{2.083190in}{3.382727in}}%
\pgfpathlineto{\pgfqpoint{2.083488in}{0.637273in}}%
\pgfpathlineto{\pgfqpoint{2.084092in}{0.637273in}}%
\pgfpathlineto{\pgfqpoint{2.085636in}{3.382727in}}%
\pgfpathlineto{\pgfqpoint{2.086288in}{3.382727in}}%
\pgfpathlineto{\pgfqpoint{2.087832in}{0.637273in}}%
\pgfpathlineto{\pgfqpoint{2.095478in}{0.637273in}}%
\pgfpathlineto{\pgfqpoint{2.095525in}{3.382727in}}%
\pgfpathlineto{\pgfqpoint{2.097017in}{0.637273in}}%
\pgfpathlineto{\pgfqpoint{2.106330in}{0.637273in}}%
\pgfpathlineto{\pgfqpoint{2.107869in}{3.382727in}}%
\pgfpathlineto{\pgfqpoint{2.109413in}{0.637273in}}%
\pgfpathlineto{\pgfqpoint{2.118938in}{0.637273in}}%
\pgfpathlineto{\pgfqpoint{2.119004in}{3.382727in}}%
\pgfpathlineto{\pgfqpoint{2.120478in}{0.637273in}}%
\pgfpathlineto{\pgfqpoint{2.131004in}{0.637273in}}%
\pgfpathlineto{\pgfqpoint{2.131070in}{3.382727in}}%
\pgfpathlineto{\pgfqpoint{2.132543in}{0.637273in}}%
\pgfpathlineto{\pgfqpoint{2.141922in}{0.637273in}}%
\pgfpathlineto{\pgfqpoint{2.141988in}{3.382727in}}%
\pgfpathlineto{\pgfqpoint{2.143461in}{0.637273in}}%
\pgfpathlineto{\pgfqpoint{2.154115in}{0.637273in}}%
\pgfpathlineto{\pgfqpoint{2.154162in}{3.382727in}}%
\pgfpathlineto{\pgfqpoint{2.155654in}{0.637273in}}%
\pgfpathlineto{\pgfqpoint{2.165009in}{0.637273in}}%
\pgfpathlineto{\pgfqpoint{2.165037in}{3.382727in}}%
\pgfpathlineto{\pgfqpoint{2.166548in}{0.637273in}}%
\pgfpathlineto{\pgfqpoint{2.175889in}{0.637273in}}%
\pgfpathlineto{\pgfqpoint{2.175918in}{3.382727in}}%
\pgfpathlineto{\pgfqpoint{2.177429in}{0.637273in}}%
\pgfpathlineto{\pgfqpoint{2.187020in}{0.637273in}}%
\pgfpathlineto{\pgfqpoint{2.187048in}{3.382727in}}%
\pgfpathlineto{\pgfqpoint{2.188559in}{0.637273in}}%
\pgfpathlineto{\pgfqpoint{2.197895in}{0.637273in}}%
\pgfpathlineto{\pgfqpoint{2.197923in}{3.382727in}}%
\pgfpathlineto{\pgfqpoint{2.199435in}{0.637273in}}%
\pgfpathlineto{\pgfqpoint{2.210768in}{0.637273in}}%
\pgfpathlineto{\pgfqpoint{2.210796in}{3.382727in}}%
\pgfpathlineto{\pgfqpoint{2.212308in}{0.637273in}}%
\pgfpathlineto{\pgfqpoint{2.221714in}{0.637273in}}%
\pgfpathlineto{\pgfqpoint{2.221743in}{3.382727in}}%
\pgfpathlineto{\pgfqpoint{2.223254in}{0.637273in}}%
\pgfpathlineto{\pgfqpoint{2.232566in}{0.637273in}}%
\pgfpathlineto{\pgfqpoint{2.232595in}{3.382727in}}%
\pgfpathlineto{\pgfqpoint{2.234106in}{0.637273in}}%
\pgfpathlineto{\pgfqpoint{2.243697in}{0.637273in}}%
\pgfpathlineto{\pgfqpoint{2.243725in}{3.382727in}}%
\pgfpathlineto{\pgfqpoint{2.245236in}{0.637273in}}%
\pgfpathlineto{\pgfqpoint{2.249118in}{0.637273in}}%
\pgfpathlineto{\pgfqpoint{2.250596in}{3.382727in}}%
\pgfpathlineto{\pgfqpoint{2.250657in}{0.637273in}}%
\pgfpathlineto{\pgfqpoint{2.250700in}{0.637273in}}%
\pgfpathlineto{\pgfqpoint{2.250851in}{3.382727in}}%
\pgfpathlineto{\pgfqpoint{2.252239in}{0.637273in}}%
\pgfpathlineto{\pgfqpoint{2.256357in}{0.637273in}}%
\pgfpathlineto{\pgfqpoint{2.256386in}{3.382727in}}%
\pgfpathlineto{\pgfqpoint{2.257897in}{0.637273in}}%
\pgfpathlineto{\pgfqpoint{2.267445in}{0.637273in}}%
\pgfpathlineto{\pgfqpoint{2.267474in}{3.382727in}}%
\pgfpathlineto{\pgfqpoint{2.268985in}{0.637273in}}%
\pgfpathlineto{\pgfqpoint{2.278316in}{0.637273in}}%
\pgfpathlineto{\pgfqpoint{2.278344in}{3.382727in}}%
\pgfpathlineto{\pgfqpoint{2.279856in}{0.637273in}}%
\pgfpathlineto{\pgfqpoint{2.289210in}{0.637273in}}%
\pgfpathlineto{\pgfqpoint{2.289239in}{3.382727in}}%
\pgfpathlineto{\pgfqpoint{2.290750in}{0.637273in}}%
\pgfpathlineto{\pgfqpoint{2.300341in}{0.637273in}}%
\pgfpathlineto{\pgfqpoint{2.300350in}{3.382727in}}%
\pgfpathlineto{\pgfqpoint{2.301880in}{0.637273in}}%
\pgfpathlineto{\pgfqpoint{2.311169in}{0.637273in}}%
\pgfpathlineto{\pgfqpoint{2.311216in}{3.382727in}}%
\pgfpathlineto{\pgfqpoint{2.312709in}{0.637273in}}%
\pgfpathlineto{\pgfqpoint{2.323249in}{0.637273in}}%
\pgfpathlineto{\pgfqpoint{2.323258in}{3.382727in}}%
\pgfpathlineto{\pgfqpoint{2.324788in}{0.637273in}}%
\pgfpathlineto{\pgfqpoint{2.328797in}{0.637273in}}%
\pgfpathlineto{\pgfqpoint{2.328826in}{3.382727in}}%
\pgfpathlineto{\pgfqpoint{2.330337in}{0.637273in}}%
\pgfpathlineto{\pgfqpoint{2.339588in}{0.637273in}}%
\pgfpathlineto{\pgfqpoint{2.339612in}{3.382727in}}%
\pgfpathlineto{\pgfqpoint{2.341127in}{0.637273in}}%
\pgfpathlineto{\pgfqpoint{2.351521in}{0.637273in}}%
\pgfpathlineto{\pgfqpoint{2.351550in}{3.382727in}}%
\pgfpathlineto{\pgfqpoint{2.353061in}{0.637273in}}%
\pgfpathlineto{\pgfqpoint{2.362392in}{0.637273in}}%
\pgfpathlineto{\pgfqpoint{2.362420in}{3.382727in}}%
\pgfpathlineto{\pgfqpoint{2.363931in}{0.637273in}}%
\pgfpathlineto{\pgfqpoint{2.373537in}{0.637273in}}%
\pgfpathlineto{\pgfqpoint{2.373565in}{3.382727in}}%
\pgfpathlineto{\pgfqpoint{2.375076in}{0.637273in}}%
\pgfpathlineto{\pgfqpoint{2.378958in}{0.637273in}}%
\pgfpathlineto{\pgfqpoint{2.380370in}{3.382727in}}%
\pgfpathlineto{\pgfqpoint{2.380497in}{0.637273in}}%
\pgfpathlineto{\pgfqpoint{2.385919in}{0.637273in}}%
\pgfpathlineto{\pgfqpoint{2.385947in}{3.382727in}}%
\pgfpathlineto{\pgfqpoint{2.387458in}{0.637273in}}%
\pgfpathlineto{\pgfqpoint{2.396704in}{0.637273in}}%
\pgfpathlineto{\pgfqpoint{2.396733in}{3.382727in}}%
\pgfpathlineto{\pgfqpoint{2.398244in}{0.637273in}}%
\pgfpathlineto{\pgfqpoint{2.407835in}{0.637273in}}%
\pgfpathlineto{\pgfqpoint{2.409209in}{3.382727in}}%
\pgfpathlineto{\pgfqpoint{2.409374in}{0.637273in}}%
\pgfpathlineto{\pgfqpoint{2.409473in}{0.637273in}}%
\pgfpathlineto{\pgfqpoint{2.409615in}{3.382727in}}%
\pgfpathlineto{\pgfqpoint{2.411013in}{0.637273in}}%
\pgfpathlineto{\pgfqpoint{2.420505in}{0.637273in}}%
\pgfpathlineto{\pgfqpoint{2.422049in}{3.382727in}}%
\pgfpathlineto{\pgfqpoint{2.422686in}{3.382727in}}%
\pgfpathlineto{\pgfqpoint{2.423749in}{0.637273in}}%
\pgfpathlineto{\pgfqpoint{2.424226in}{3.382727in}}%
\pgfpathlineto{\pgfqpoint{2.424372in}{3.382727in}}%
\pgfpathlineto{\pgfqpoint{2.425860in}{0.637273in}}%
\pgfpathlineto{\pgfqpoint{2.425912in}{3.382727in}}%
\pgfpathlineto{\pgfqpoint{2.425931in}{3.382727in}}%
\pgfpathlineto{\pgfqpoint{2.426337in}{0.637273in}}%
\pgfpathlineto{\pgfqpoint{2.427470in}{3.382727in}}%
\pgfpathlineto{\pgfqpoint{2.427782in}{3.382727in}}%
\pgfpathlineto{\pgfqpoint{2.429321in}{0.637273in}}%
\pgfpathlineto{\pgfqpoint{2.430865in}{3.382727in}}%
\pgfpathlineto{\pgfqpoint{2.432254in}{3.382727in}}%
\pgfpathlineto{\pgfqpoint{2.433798in}{0.637273in}}%
\pgfpathlineto{\pgfqpoint{2.435130in}{3.382727in}}%
\pgfpathlineto{\pgfqpoint{2.435337in}{0.637273in}}%
\pgfpathlineto{\pgfqpoint{2.440537in}{0.637273in}}%
\pgfpathlineto{\pgfqpoint{2.442062in}{3.382727in}}%
\pgfpathlineto{\pgfqpoint{2.442076in}{0.637273in}}%
\pgfpathlineto{\pgfqpoint{2.442298in}{0.637273in}}%
\pgfpathlineto{\pgfqpoint{2.443842in}{3.382727in}}%
\pgfpathlineto{\pgfqpoint{2.444862in}{0.637273in}}%
\pgfpathlineto{\pgfqpoint{2.445382in}{3.382727in}}%
\pgfpathlineto{\pgfqpoint{2.446303in}{3.382727in}}%
\pgfpathlineto{\pgfqpoint{2.447837in}{0.637273in}}%
\pgfpathlineto{\pgfqpoint{2.447842in}{3.382727in}}%
\pgfpathlineto{\pgfqpoint{2.447852in}{3.382727in}}%
\pgfpathlineto{\pgfqpoint{2.449056in}{0.637273in}}%
\pgfpathlineto{\pgfqpoint{2.449391in}{3.382727in}}%
\pgfpathlineto{\pgfqpoint{2.450496in}{3.382727in}}%
\pgfpathlineto{\pgfqpoint{2.452026in}{0.637273in}}%
\pgfpathlineto{\pgfqpoint{2.452036in}{3.382727in}}%
\pgfpathlineto{\pgfqpoint{2.452040in}{3.382727in}}%
\pgfpathlineto{\pgfqpoint{2.453414in}{0.637273in}}%
\pgfpathlineto{\pgfqpoint{2.453580in}{3.382727in}}%
\pgfpathlineto{\pgfqpoint{2.454855in}{3.382727in}}%
\pgfpathlineto{\pgfqpoint{2.456390in}{0.637273in}}%
\pgfpathlineto{\pgfqpoint{2.456394in}{3.382727in}}%
\pgfpathlineto{\pgfqpoint{2.456404in}{3.382727in}}%
\pgfpathlineto{\pgfqpoint{2.457948in}{0.637273in}}%
\pgfpathlineto{\pgfqpoint{2.459473in}{3.382727in}}%
\pgfpathlineto{\pgfqpoint{2.459487in}{0.637273in}}%
\pgfpathlineto{\pgfqpoint{2.459714in}{0.637273in}}%
\pgfpathlineto{\pgfqpoint{2.461258in}{3.382727in}}%
\pgfpathlineto{\pgfqpoint{2.461268in}{3.382727in}}%
\pgfpathlineto{\pgfqpoint{2.462307in}{0.637273in}}%
\pgfpathlineto{\pgfqpoint{2.462807in}{3.382727in}}%
\pgfpathlineto{\pgfqpoint{2.463747in}{3.382727in}}%
\pgfpathlineto{\pgfqpoint{2.465282in}{0.637273in}}%
\pgfpathlineto{\pgfqpoint{2.465286in}{3.382727in}}%
\pgfpathlineto{\pgfqpoint{2.465296in}{3.382727in}}%
\pgfpathlineto{\pgfqpoint{2.466840in}{0.637273in}}%
\pgfpathlineto{\pgfqpoint{2.471997in}{0.637273in}}%
\pgfpathlineto{\pgfqpoint{2.473541in}{3.382727in}}%
\pgfpathlineto{\pgfqpoint{2.473787in}{3.382727in}}%
\pgfpathlineto{\pgfqpoint{2.475317in}{0.637273in}}%
\pgfpathlineto{\pgfqpoint{2.475326in}{3.382727in}}%
\pgfpathlineto{\pgfqpoint{2.475331in}{3.382727in}}%
\pgfpathlineto{\pgfqpoint{2.476672in}{0.637273in}}%
\pgfpathlineto{\pgfqpoint{2.476870in}{3.382727in}}%
\pgfpathlineto{\pgfqpoint{2.478112in}{3.382727in}}%
\pgfpathlineto{\pgfqpoint{2.479647in}{0.637273in}}%
\pgfpathlineto{\pgfqpoint{2.479652in}{3.382727in}}%
\pgfpathlineto{\pgfqpoint{2.479661in}{3.382727in}}%
\pgfpathlineto{\pgfqpoint{2.480865in}{0.637273in}}%
\pgfpathlineto{\pgfqpoint{2.481201in}{3.382727in}}%
\pgfpathlineto{\pgfqpoint{2.482306in}{3.382727in}}%
\pgfpathlineto{\pgfqpoint{2.483840in}{0.637273in}}%
\pgfpathlineto{\pgfqpoint{2.483845in}{3.382727in}}%
\pgfpathlineto{\pgfqpoint{2.483854in}{3.382727in}}%
\pgfpathlineto{\pgfqpoint{2.485361in}{0.637273in}}%
\pgfpathlineto{\pgfqpoint{2.485394in}{3.382727in}}%
\pgfpathlineto{\pgfqpoint{2.486801in}{3.382727in}}%
\pgfpathlineto{\pgfqpoint{2.488317in}{0.637273in}}%
\pgfpathlineto{\pgfqpoint{2.488341in}{3.382727in}}%
\pgfpathlineto{\pgfqpoint{2.488388in}{3.382727in}}%
\pgfpathlineto{\pgfqpoint{2.489885in}{0.637273in}}%
\pgfpathlineto{\pgfqpoint{2.489927in}{3.382727in}}%
\pgfpathlineto{\pgfqpoint{2.491325in}{3.382727in}}%
\pgfpathlineto{\pgfqpoint{2.492860in}{0.637273in}}%
\pgfpathlineto{\pgfqpoint{2.492865in}{3.382727in}}%
\pgfpathlineto{\pgfqpoint{2.492874in}{3.382727in}}%
\pgfpathlineto{\pgfqpoint{2.494244in}{0.637273in}}%
\pgfpathlineto{\pgfqpoint{2.494414in}{3.382727in}}%
\pgfpathlineto{\pgfqpoint{2.495684in}{3.382727in}}%
\pgfpathlineto{\pgfqpoint{2.497219in}{0.637273in}}%
\pgfpathlineto{\pgfqpoint{2.497223in}{3.382727in}}%
\pgfpathlineto{\pgfqpoint{2.497233in}{3.382727in}}%
\pgfpathlineto{\pgfqpoint{2.498777in}{0.637273in}}%
\pgfpathlineto{\pgfqpoint{2.503546in}{0.637273in}}%
\pgfpathlineto{\pgfqpoint{2.505091in}{3.382727in}}%
\pgfpathlineto{\pgfqpoint{2.505473in}{3.382727in}}%
\pgfpathlineto{\pgfqpoint{2.507008in}{0.637273in}}%
\pgfpathlineto{\pgfqpoint{2.507013in}{3.382727in}}%
\pgfpathlineto{\pgfqpoint{2.507022in}{3.382727in}}%
\pgfpathlineto{\pgfqpoint{2.508363in}{0.637273in}}%
\pgfpathlineto{\pgfqpoint{2.508562in}{3.382727in}}%
\pgfpathlineto{\pgfqpoint{2.509804in}{3.382727in}}%
\pgfpathlineto{\pgfqpoint{2.511324in}{0.637273in}}%
\pgfpathlineto{\pgfqpoint{2.511343in}{3.382727in}}%
\pgfpathlineto{\pgfqpoint{2.511348in}{3.382727in}}%
\pgfpathlineto{\pgfqpoint{2.512694in}{0.637273in}}%
\pgfpathlineto{\pgfqpoint{2.512887in}{3.382727in}}%
\pgfpathlineto{\pgfqpoint{2.514134in}{3.382727in}}%
\pgfpathlineto{\pgfqpoint{2.515669in}{0.637273in}}%
\pgfpathlineto{\pgfqpoint{2.515673in}{3.382727in}}%
\pgfpathlineto{\pgfqpoint{2.515683in}{3.382727in}}%
\pgfpathlineto{\pgfqpoint{2.517118in}{0.637273in}}%
\pgfpathlineto{\pgfqpoint{2.517222in}{3.382727in}}%
\pgfpathlineto{\pgfqpoint{2.518559in}{3.382727in}}%
\pgfpathlineto{\pgfqpoint{2.520093in}{0.637273in}}%
\pgfpathlineto{\pgfqpoint{2.520098in}{3.382727in}}%
\pgfpathlineto{\pgfqpoint{2.520108in}{3.382727in}}%
\pgfpathlineto{\pgfqpoint{2.521312in}{0.637273in}}%
\pgfpathlineto{\pgfqpoint{2.521647in}{3.382727in}}%
\pgfpathlineto{\pgfqpoint{2.522752in}{3.382727in}}%
\pgfpathlineto{\pgfqpoint{2.524287in}{0.637273in}}%
\pgfpathlineto{\pgfqpoint{2.524292in}{3.382727in}}%
\pgfpathlineto{\pgfqpoint{2.524301in}{3.382727in}}%
\pgfpathlineto{\pgfqpoint{2.525534in}{0.637273in}}%
\pgfpathlineto{\pgfqpoint{2.525840in}{3.382727in}}%
\pgfpathlineto{\pgfqpoint{2.526974in}{3.382727in}}%
\pgfpathlineto{\pgfqpoint{2.528504in}{0.637273in}}%
\pgfpathlineto{\pgfqpoint{2.528513in}{3.382727in}}%
\pgfpathlineto{\pgfqpoint{2.528523in}{3.382727in}}%
\pgfpathlineto{\pgfqpoint{2.530067in}{0.637273in}}%
\pgfpathlineto{\pgfqpoint{2.534001in}{0.637273in}}%
\pgfpathlineto{\pgfqpoint{2.535545in}{3.382727in}}%
\pgfpathlineto{\pgfqpoint{2.536763in}{3.382727in}}%
\pgfpathlineto{\pgfqpoint{2.538293in}{0.637273in}}%
\pgfpathlineto{\pgfqpoint{2.538303in}{3.382727in}}%
\pgfpathlineto{\pgfqpoint{2.538307in}{3.382727in}}%
\pgfpathlineto{\pgfqpoint{2.539649in}{0.637273in}}%
\pgfpathlineto{\pgfqpoint{2.539847in}{3.382727in}}%
\pgfpathlineto{\pgfqpoint{2.541089in}{3.382727in}}%
\pgfpathlineto{\pgfqpoint{2.542624in}{0.637273in}}%
\pgfpathlineto{\pgfqpoint{2.542628in}{3.382727in}}%
\pgfpathlineto{\pgfqpoint{2.542638in}{3.382727in}}%
\pgfpathlineto{\pgfqpoint{2.543875in}{0.637273in}}%
\pgfpathlineto{\pgfqpoint{2.544177in}{3.382727in}}%
\pgfpathlineto{\pgfqpoint{2.545315in}{3.382727in}}%
\pgfpathlineto{\pgfqpoint{2.546850in}{0.637273in}}%
\pgfpathlineto{\pgfqpoint{2.546855in}{3.382727in}}%
\pgfpathlineto{\pgfqpoint{2.546864in}{3.382727in}}%
\pgfpathlineto{\pgfqpoint{2.548149in}{0.637273in}}%
\pgfpathlineto{\pgfqpoint{2.548404in}{3.382727in}}%
\pgfpathlineto{\pgfqpoint{2.549589in}{3.382727in}}%
\pgfpathlineto{\pgfqpoint{2.551124in}{0.637273in}}%
\pgfpathlineto{\pgfqpoint{2.551128in}{3.382727in}}%
\pgfpathlineto{\pgfqpoint{2.551138in}{3.382727in}}%
\pgfpathlineto{\pgfqpoint{2.552337in}{0.637273in}}%
\pgfpathlineto{\pgfqpoint{2.552677in}{3.382727in}}%
\pgfpathlineto{\pgfqpoint{2.553778in}{3.382727in}}%
\pgfpathlineto{\pgfqpoint{2.555312in}{0.637273in}}%
\pgfpathlineto{\pgfqpoint{2.555317in}{3.382727in}}%
\pgfpathlineto{\pgfqpoint{2.555327in}{3.382727in}}%
\pgfpathlineto{\pgfqpoint{2.556554in}{0.637273in}}%
\pgfpathlineto{\pgfqpoint{2.556866in}{3.382727in}}%
\pgfpathlineto{\pgfqpoint{2.557995in}{3.382727in}}%
\pgfpathlineto{\pgfqpoint{2.559529in}{0.637273in}}%
\pgfpathlineto{\pgfqpoint{2.559534in}{3.382727in}}%
\pgfpathlineto{\pgfqpoint{2.559544in}{3.382727in}}%
\pgfpathlineto{\pgfqpoint{2.561088in}{0.637273in}}%
\pgfpathlineto{\pgfqpoint{2.565187in}{0.637273in}}%
\pgfpathlineto{\pgfqpoint{2.566731in}{3.382727in}}%
\pgfpathlineto{\pgfqpoint{2.567921in}{3.382727in}}%
\pgfpathlineto{\pgfqpoint{2.569456in}{0.637273in}}%
\pgfpathlineto{\pgfqpoint{2.569460in}{3.382727in}}%
\pgfpathlineto{\pgfqpoint{2.569470in}{3.382727in}}%
\pgfpathlineto{\pgfqpoint{2.570811in}{0.637273in}}%
\pgfpathlineto{\pgfqpoint{2.571009in}{3.382727in}}%
\pgfpathlineto{\pgfqpoint{2.572251in}{3.382727in}}%
\pgfpathlineto{\pgfqpoint{2.573786in}{0.637273in}}%
\pgfpathlineto{\pgfqpoint{2.573791in}{3.382727in}}%
\pgfpathlineto{\pgfqpoint{2.573800in}{3.382727in}}%
\pgfpathlineto{\pgfqpoint{2.575156in}{0.637273in}}%
\pgfpathlineto{\pgfqpoint{2.575340in}{3.382727in}}%
\pgfpathlineto{\pgfqpoint{2.576596in}{3.382727in}}%
\pgfpathlineto{\pgfqpoint{2.578126in}{0.637273in}}%
\pgfpathlineto{\pgfqpoint{2.578135in}{3.382727in}}%
\pgfpathlineto{\pgfqpoint{2.578140in}{3.382727in}}%
\pgfpathlineto{\pgfqpoint{2.579439in}{0.637273in}}%
\pgfpathlineto{\pgfqpoint{2.579680in}{3.382727in}}%
\pgfpathlineto{\pgfqpoint{2.580879in}{3.382727in}}%
\pgfpathlineto{\pgfqpoint{2.582414in}{0.637273in}}%
\pgfpathlineto{\pgfqpoint{2.582418in}{3.382727in}}%
\pgfpathlineto{\pgfqpoint{2.582428in}{3.382727in}}%
\pgfpathlineto{\pgfqpoint{2.583797in}{0.637273in}}%
\pgfpathlineto{\pgfqpoint{2.583967in}{3.382727in}}%
\pgfpathlineto{\pgfqpoint{2.585238in}{3.382727in}}%
\pgfpathlineto{\pgfqpoint{2.586768in}{0.637273in}}%
\pgfpathlineto{\pgfqpoint{2.586777in}{3.382727in}}%
\pgfpathlineto{\pgfqpoint{2.586782in}{3.382727in}}%
\pgfpathlineto{\pgfqpoint{2.587986in}{0.637273in}}%
\pgfpathlineto{\pgfqpoint{2.588321in}{3.382727in}}%
\pgfpathlineto{\pgfqpoint{2.589426in}{3.382727in}}%
\pgfpathlineto{\pgfqpoint{2.590961in}{0.637273in}}%
\pgfpathlineto{\pgfqpoint{2.590966in}{3.382727in}}%
\pgfpathlineto{\pgfqpoint{2.590975in}{3.382727in}}%
\pgfpathlineto{\pgfqpoint{2.592519in}{0.637273in}}%
\pgfpathlineto{\pgfqpoint{2.596453in}{0.637273in}}%
\pgfpathlineto{\pgfqpoint{2.597997in}{3.382727in}}%
\pgfpathlineto{\pgfqpoint{2.599079in}{3.382727in}}%
\pgfpathlineto{\pgfqpoint{2.600609in}{0.637273in}}%
\pgfpathlineto{\pgfqpoint{2.600618in}{3.382727in}}%
\pgfpathlineto{\pgfqpoint{2.600623in}{3.382727in}}%
\pgfpathlineto{\pgfqpoint{2.601822in}{0.637273in}}%
\pgfpathlineto{\pgfqpoint{2.602162in}{3.382727in}}%
\pgfpathlineto{\pgfqpoint{2.603263in}{3.382727in}}%
\pgfpathlineto{\pgfqpoint{2.604797in}{0.637273in}}%
\pgfpathlineto{\pgfqpoint{2.604802in}{3.382727in}}%
\pgfpathlineto{\pgfqpoint{2.604812in}{3.382727in}}%
\pgfpathlineto{\pgfqpoint{2.606011in}{0.637273in}}%
\pgfpathlineto{\pgfqpoint{2.606351in}{3.382727in}}%
\pgfpathlineto{\pgfqpoint{2.607451in}{3.382727in}}%
\pgfpathlineto{\pgfqpoint{2.608986in}{0.637273in}}%
\pgfpathlineto{\pgfqpoint{2.608991in}{3.382727in}}%
\pgfpathlineto{\pgfqpoint{2.608996in}{3.382727in}}%
\pgfpathlineto{\pgfqpoint{2.610483in}{0.637273in}}%
\pgfpathlineto{\pgfqpoint{2.610535in}{3.382727in}}%
\pgfpathlineto{\pgfqpoint{2.610563in}{3.382727in}}%
\pgfpathlineto{\pgfqpoint{2.611763in}{0.637273in}}%
\pgfpathlineto{\pgfqpoint{2.612103in}{3.382727in}}%
\pgfpathlineto{\pgfqpoint{2.623163in}{3.382727in}}%
\pgfpathlineto{\pgfqpoint{2.623314in}{0.637273in}}%
\pgfpathlineto{\pgfqpoint{2.624702in}{3.382727in}}%
\pgfpathlineto{\pgfqpoint{2.627998in}{3.382727in}}%
\pgfpathlineto{\pgfqpoint{2.628083in}{0.637273in}}%
\pgfpathlineto{\pgfqpoint{2.629538in}{3.382727in}}%
\pgfpathlineto{\pgfqpoint{2.637268in}{3.382727in}}%
\pgfpathlineto{\pgfqpoint{2.637273in}{0.637273in}}%
\pgfpathlineto{\pgfqpoint{2.638808in}{3.382727in}}%
\pgfpathlineto{\pgfqpoint{2.653706in}{3.382727in}}%
\pgfpathlineto{\pgfqpoint{2.655251in}{0.637273in}}%
\pgfpathlineto{\pgfqpoint{2.655468in}{0.637273in}}%
\pgfpathlineto{\pgfqpoint{2.657012in}{3.382727in}}%
\pgfpathlineto{\pgfqpoint{2.657229in}{3.382727in}}%
\pgfpathlineto{\pgfqpoint{2.657385in}{0.637273in}}%
\pgfpathlineto{\pgfqpoint{2.658769in}{3.382727in}}%
\pgfpathlineto{\pgfqpoint{2.659600in}{3.382727in}}%
\pgfpathlineto{\pgfqpoint{2.660837in}{0.637273in}}%
\pgfpathlineto{\pgfqpoint{2.661139in}{3.382727in}}%
\pgfpathlineto{\pgfqpoint{2.661423in}{3.382727in}}%
\pgfpathlineto{\pgfqpoint{2.662910in}{0.637273in}}%
\pgfpathlineto{\pgfqpoint{2.662962in}{3.382727in}}%
\pgfpathlineto{\pgfqpoint{2.662981in}{3.382727in}}%
\pgfpathlineto{\pgfqpoint{2.663397in}{0.637273in}}%
\pgfpathlineto{\pgfqpoint{2.664521in}{3.382727in}}%
\pgfpathlineto{\pgfqpoint{2.664837in}{3.382727in}}%
\pgfpathlineto{\pgfqpoint{2.666376in}{0.637273in}}%
\pgfpathlineto{\pgfqpoint{2.667921in}{3.382727in}}%
\pgfpathlineto{\pgfqpoint{2.667925in}{3.382727in}}%
\pgfpathlineto{\pgfqpoint{2.669469in}{0.637273in}}%
\pgfpathlineto{\pgfqpoint{2.671014in}{3.382727in}}%
\pgfpathlineto{\pgfqpoint{2.671018in}{3.382727in}}%
\pgfpathlineto{\pgfqpoint{2.672563in}{0.637273in}}%
\pgfpathlineto{\pgfqpoint{2.676496in}{0.637273in}}%
\pgfpathlineto{\pgfqpoint{2.678040in}{3.382727in}}%
\pgfpathlineto{\pgfqpoint{2.679264in}{3.382727in}}%
\pgfpathlineto{\pgfqpoint{2.680794in}{0.637273in}}%
\pgfpathlineto{\pgfqpoint{2.680803in}{3.382727in}}%
\pgfpathlineto{\pgfqpoint{2.680808in}{3.382727in}}%
\pgfpathlineto{\pgfqpoint{2.682012in}{0.637273in}}%
\pgfpathlineto{\pgfqpoint{2.682347in}{3.382727in}}%
\pgfpathlineto{\pgfqpoint{2.683452in}{3.382727in}}%
\pgfpathlineto{\pgfqpoint{2.684987in}{0.637273in}}%
\pgfpathlineto{\pgfqpoint{2.684992in}{3.382727in}}%
\pgfpathlineto{\pgfqpoint{2.685001in}{3.382727in}}%
\pgfpathlineto{\pgfqpoint{2.686205in}{0.637273in}}%
\pgfpathlineto{\pgfqpoint{2.686541in}{3.382727in}}%
\pgfpathlineto{\pgfqpoint{2.687646in}{3.382727in}}%
\pgfpathlineto{\pgfqpoint{2.689176in}{0.637273in}}%
\pgfpathlineto{\pgfqpoint{2.689185in}{3.382727in}}%
\pgfpathlineto{\pgfqpoint{2.689190in}{3.382727in}}%
\pgfpathlineto{\pgfqpoint{2.690413in}{0.637273in}}%
\pgfpathlineto{\pgfqpoint{2.690729in}{3.382727in}}%
\pgfpathlineto{\pgfqpoint{2.691853in}{3.382727in}}%
\pgfpathlineto{\pgfqpoint{2.693341in}{0.637273in}}%
\pgfpathlineto{\pgfqpoint{2.693393in}{3.382727in}}%
\pgfpathlineto{\pgfqpoint{2.693412in}{3.382727in}}%
\pgfpathlineto{\pgfqpoint{2.694663in}{0.637273in}}%
\pgfpathlineto{\pgfqpoint{2.694951in}{3.382727in}}%
\pgfpathlineto{\pgfqpoint{2.696103in}{3.382727in}}%
\pgfpathlineto{\pgfqpoint{2.697633in}{0.637273in}}%
\pgfpathlineto{\pgfqpoint{2.697643in}{3.382727in}}%
\pgfpathlineto{\pgfqpoint{2.697662in}{3.382727in}}%
\pgfpathlineto{\pgfqpoint{2.698852in}{0.637273in}}%
\pgfpathlineto{\pgfqpoint{2.699201in}{3.382727in}}%
\pgfpathlineto{\pgfqpoint{2.700292in}{3.382727in}}%
\pgfpathlineto{\pgfqpoint{2.701836in}{0.637273in}}%
\pgfpathlineto{\pgfqpoint{2.702724in}{3.382727in}}%
\pgfpathlineto{\pgfqpoint{2.703376in}{0.637273in}}%
\pgfpathlineto{\pgfqpoint{2.708202in}{0.637273in}}%
\pgfpathlineto{\pgfqpoint{2.709746in}{3.382727in}}%
\pgfpathlineto{\pgfqpoint{2.710100in}{3.382727in}}%
\pgfpathlineto{\pgfqpoint{2.711630in}{0.637273in}}%
\pgfpathlineto{\pgfqpoint{2.711640in}{3.382727in}}%
\pgfpathlineto{\pgfqpoint{2.711644in}{3.382727in}}%
\pgfpathlineto{\pgfqpoint{2.713014in}{0.637273in}}%
\pgfpathlineto{\pgfqpoint{2.713184in}{3.382727in}}%
\pgfpathlineto{\pgfqpoint{2.714459in}{3.382727in}}%
\pgfpathlineto{\pgfqpoint{2.715989in}{0.637273in}}%
\pgfpathlineto{\pgfqpoint{2.715998in}{3.382727in}}%
\pgfpathlineto{\pgfqpoint{2.716003in}{3.382727in}}%
\pgfpathlineto{\pgfqpoint{2.717207in}{0.637273in}}%
\pgfpathlineto{\pgfqpoint{2.717543in}{3.382727in}}%
\pgfpathlineto{\pgfqpoint{2.718648in}{3.382727in}}%
\pgfpathlineto{\pgfqpoint{2.720178in}{0.637273in}}%
\pgfpathlineto{\pgfqpoint{2.720187in}{3.382727in}}%
\pgfpathlineto{\pgfqpoint{2.720192in}{3.382727in}}%
\pgfpathlineto{\pgfqpoint{2.721642in}{0.637273in}}%
\pgfpathlineto{\pgfqpoint{2.721731in}{3.382727in}}%
\pgfpathlineto{\pgfqpoint{2.723082in}{3.382727in}}%
\pgfpathlineto{\pgfqpoint{2.724565in}{0.637273in}}%
\pgfpathlineto{\pgfqpoint{2.724621in}{3.382727in}}%
\pgfpathlineto{\pgfqpoint{2.724635in}{3.382727in}}%
\pgfpathlineto{\pgfqpoint{2.726024in}{0.637273in}}%
\pgfpathlineto{\pgfqpoint{2.726175in}{3.382727in}}%
\pgfpathlineto{\pgfqpoint{2.727464in}{3.382727in}}%
\pgfpathlineto{\pgfqpoint{2.728999in}{0.637273in}}%
\pgfpathlineto{\pgfqpoint{2.729004in}{3.382727in}}%
\pgfpathlineto{\pgfqpoint{2.729013in}{3.382727in}}%
\pgfpathlineto{\pgfqpoint{2.730354in}{0.637273in}}%
\pgfpathlineto{\pgfqpoint{2.730553in}{3.382727in}}%
\pgfpathlineto{\pgfqpoint{2.731794in}{3.382727in}}%
\pgfpathlineto{\pgfqpoint{2.733325in}{0.637273in}}%
\pgfpathlineto{\pgfqpoint{2.733334in}{3.382727in}}%
\pgfpathlineto{\pgfqpoint{2.733339in}{3.382727in}}%
\pgfpathlineto{\pgfqpoint{2.734883in}{0.637273in}}%
\pgfpathlineto{\pgfqpoint{2.739704in}{0.637273in}}%
\pgfpathlineto{\pgfqpoint{2.741249in}{3.382727in}}%
\pgfpathlineto{\pgfqpoint{2.741461in}{3.382727in}}%
\pgfpathlineto{\pgfqpoint{2.742991in}{0.637273in}}%
\pgfpathlineto{\pgfqpoint{2.743001in}{3.382727in}}%
\pgfpathlineto{\pgfqpoint{2.743010in}{3.382727in}}%
\pgfpathlineto{\pgfqpoint{2.744242in}{0.637273in}}%
\pgfpathlineto{\pgfqpoint{2.744549in}{3.382727in}}%
\pgfpathlineto{\pgfqpoint{2.745678in}{3.382727in}}%
\pgfpathlineto{\pgfqpoint{2.747213in}{0.637273in}}%
\pgfpathlineto{\pgfqpoint{2.747218in}{3.382727in}}%
\pgfpathlineto{\pgfqpoint{2.747227in}{3.382727in}}%
\pgfpathlineto{\pgfqpoint{2.748469in}{0.637273in}}%
\pgfpathlineto{\pgfqpoint{2.748766in}{3.382727in}}%
\pgfpathlineto{\pgfqpoint{2.749909in}{3.382727in}}%
\pgfpathlineto{\pgfqpoint{2.751444in}{0.637273in}}%
\pgfpathlineto{\pgfqpoint{2.751449in}{3.382727in}}%
\pgfpathlineto{\pgfqpoint{2.751458in}{3.382727in}}%
\pgfpathlineto{\pgfqpoint{2.752681in}{0.637273in}}%
\pgfpathlineto{\pgfqpoint{2.752998in}{3.382727in}}%
\pgfpathlineto{\pgfqpoint{2.754122in}{3.382727in}}%
\pgfpathlineto{\pgfqpoint{2.755652in}{0.637273in}}%
\pgfpathlineto{\pgfqpoint{2.755661in}{3.382727in}}%
\pgfpathlineto{\pgfqpoint{2.755666in}{3.382727in}}%
\pgfpathlineto{\pgfqpoint{2.756927in}{0.637273in}}%
\pgfpathlineto{\pgfqpoint{2.757205in}{3.382727in}}%
\pgfpathlineto{\pgfqpoint{2.758367in}{3.382727in}}%
\pgfpathlineto{\pgfqpoint{2.759902in}{0.637273in}}%
\pgfpathlineto{\pgfqpoint{2.759906in}{3.382727in}}%
\pgfpathlineto{\pgfqpoint{2.759911in}{3.382727in}}%
\pgfpathlineto{\pgfqpoint{2.761120in}{0.637273in}}%
\pgfpathlineto{\pgfqpoint{2.761451in}{3.382727in}}%
\pgfpathlineto{\pgfqpoint{2.762560in}{3.382727in}}%
\pgfpathlineto{\pgfqpoint{2.764090in}{0.637273in}}%
\pgfpathlineto{\pgfqpoint{2.764100in}{3.382727in}}%
\pgfpathlineto{\pgfqpoint{2.764105in}{3.382727in}}%
\pgfpathlineto{\pgfqpoint{2.765649in}{0.637273in}}%
\pgfpathlineto{\pgfqpoint{2.770692in}{0.637273in}}%
\pgfpathlineto{\pgfqpoint{2.772217in}{3.382727in}}%
\pgfpathlineto{\pgfqpoint{2.772232in}{0.637273in}}%
\pgfpathlineto{\pgfqpoint{2.772458in}{0.637273in}}%
\pgfpathlineto{\pgfqpoint{2.774002in}{3.382727in}}%
\pgfpathlineto{\pgfqpoint{2.774914in}{0.637273in}}%
\pgfpathlineto{\pgfqpoint{2.775542in}{3.382727in}}%
\pgfpathlineto{\pgfqpoint{2.776354in}{3.382727in}}%
\pgfpathlineto{\pgfqpoint{2.777884in}{0.637273in}}%
\pgfpathlineto{\pgfqpoint{2.777894in}{3.382727in}}%
\pgfpathlineto{\pgfqpoint{2.777898in}{3.382727in}}%
\pgfpathlineto{\pgfqpoint{2.779254in}{0.637273in}}%
\pgfpathlineto{\pgfqpoint{2.779438in}{3.382727in}}%
\pgfpathlineto{\pgfqpoint{2.780694in}{3.382727in}}%
\pgfpathlineto{\pgfqpoint{2.782229in}{0.637273in}}%
\pgfpathlineto{\pgfqpoint{2.782233in}{3.382727in}}%
\pgfpathlineto{\pgfqpoint{2.782243in}{3.382727in}}%
\pgfpathlineto{\pgfqpoint{2.783551in}{0.637273in}}%
\pgfpathlineto{\pgfqpoint{2.783782in}{3.382727in}}%
\pgfpathlineto{\pgfqpoint{2.784991in}{3.382727in}}%
\pgfpathlineto{\pgfqpoint{2.786521in}{0.637273in}}%
\pgfpathlineto{\pgfqpoint{2.786531in}{3.382727in}}%
\pgfpathlineto{\pgfqpoint{2.786535in}{3.382727in}}%
\pgfpathlineto{\pgfqpoint{2.787740in}{0.637273in}}%
\pgfpathlineto{\pgfqpoint{2.788075in}{3.382727in}}%
\pgfpathlineto{\pgfqpoint{2.788226in}{3.382727in}}%
\pgfpathlineto{\pgfqpoint{2.789553in}{0.637273in}}%
\pgfpathlineto{\pgfqpoint{2.789766in}{3.382727in}}%
\pgfpathlineto{\pgfqpoint{2.790176in}{3.382727in}}%
\pgfpathlineto{\pgfqpoint{2.791640in}{0.637273in}}%
\pgfpathlineto{\pgfqpoint{2.791716in}{3.382727in}}%
\pgfpathlineto{\pgfqpoint{2.791725in}{3.382727in}}%
\pgfpathlineto{\pgfqpoint{2.792146in}{0.637273in}}%
\pgfpathlineto{\pgfqpoint{2.793265in}{3.382727in}}%
\pgfpathlineto{\pgfqpoint{2.793586in}{3.382727in}}%
\pgfpathlineto{\pgfqpoint{2.795116in}{0.637273in}}%
\pgfpathlineto{\pgfqpoint{2.795125in}{3.382727in}}%
\pgfpathlineto{\pgfqpoint{2.795130in}{3.382727in}}%
\pgfpathlineto{\pgfqpoint{2.796334in}{0.637273in}}%
\pgfpathlineto{\pgfqpoint{2.796670in}{3.382727in}}%
\pgfpathlineto{\pgfqpoint{2.797675in}{3.382727in}}%
\pgfpathlineto{\pgfqpoint{2.799220in}{0.637273in}}%
\pgfpathlineto{\pgfqpoint{2.803153in}{0.637273in}}%
\pgfpathlineto{\pgfqpoint{2.804697in}{3.382727in}}%
\pgfpathlineto{\pgfqpoint{2.804707in}{3.382727in}}%
\pgfpathlineto{\pgfqpoint{2.806143in}{0.637273in}}%
\pgfpathlineto{\pgfqpoint{2.806246in}{3.382727in}}%
\pgfpathlineto{\pgfqpoint{2.807583in}{3.382727in}}%
\pgfpathlineto{\pgfqpoint{2.809070in}{0.637273in}}%
\pgfpathlineto{\pgfqpoint{2.809122in}{3.382727in}}%
\pgfpathlineto{\pgfqpoint{2.809141in}{3.382727in}}%
\pgfpathlineto{\pgfqpoint{2.810501in}{0.637273in}}%
\pgfpathlineto{\pgfqpoint{2.810681in}{3.382727in}}%
\pgfpathlineto{\pgfqpoint{2.811941in}{3.382727in}}%
\pgfpathlineto{\pgfqpoint{2.813472in}{0.637273in}}%
\pgfpathlineto{\pgfqpoint{2.813481in}{3.382727in}}%
\pgfpathlineto{\pgfqpoint{2.813505in}{3.382727in}}%
\pgfpathlineto{\pgfqpoint{2.814827in}{0.637273in}}%
\pgfpathlineto{\pgfqpoint{2.815044in}{3.382727in}}%
\pgfpathlineto{\pgfqpoint{2.816267in}{3.382727in}}%
\pgfpathlineto{\pgfqpoint{2.817802in}{0.637273in}}%
\pgfpathlineto{\pgfqpoint{2.817807in}{3.382727in}}%
\pgfpathlineto{\pgfqpoint{2.817816in}{3.382727in}}%
\pgfpathlineto{\pgfqpoint{2.819360in}{0.637273in}}%
\pgfpathlineto{\pgfqpoint{2.820229in}{0.637273in}}%
\pgfpathlineto{\pgfqpoint{2.821764in}{3.382727in}}%
\pgfpathlineto{\pgfqpoint{2.821769in}{0.637273in}}%
\pgfpathlineto{\pgfqpoint{2.822000in}{0.637273in}}%
\pgfpathlineto{\pgfqpoint{2.823535in}{3.382727in}}%
\pgfpathlineto{\pgfqpoint{2.823539in}{0.637273in}}%
\pgfpathlineto{\pgfqpoint{2.823634in}{0.637273in}}%
\pgfpathlineto{\pgfqpoint{2.825178in}{3.382727in}}%
\pgfpathlineto{\pgfqpoint{2.826453in}{3.382727in}}%
\pgfpathlineto{\pgfqpoint{2.827988in}{0.637273in}}%
\pgfpathlineto{\pgfqpoint{2.827993in}{3.382727in}}%
\pgfpathlineto{\pgfqpoint{2.828002in}{3.382727in}}%
\pgfpathlineto{\pgfqpoint{2.829206in}{0.637273in}}%
\pgfpathlineto{\pgfqpoint{2.829542in}{3.382727in}}%
\pgfpathlineto{\pgfqpoint{2.830647in}{3.382727in}}%
\pgfpathlineto{\pgfqpoint{2.832181in}{0.637273in}}%
\pgfpathlineto{\pgfqpoint{2.832186in}{3.382727in}}%
\pgfpathlineto{\pgfqpoint{2.832195in}{3.382727in}}%
\pgfpathlineto{\pgfqpoint{2.833740in}{0.637273in}}%
\pgfpathlineto{\pgfqpoint{2.837626in}{0.637273in}}%
\pgfpathlineto{\pgfqpoint{2.839170in}{3.382727in}}%
\pgfpathlineto{\pgfqpoint{2.840223in}{3.382727in}}%
\pgfpathlineto{\pgfqpoint{2.841753in}{0.637273in}}%
\pgfpathlineto{\pgfqpoint{2.841763in}{3.382727in}}%
\pgfpathlineto{\pgfqpoint{2.841768in}{3.382727in}}%
\pgfpathlineto{\pgfqpoint{2.843000in}{0.637273in}}%
\pgfpathlineto{\pgfqpoint{2.843307in}{3.382727in}}%
\pgfpathlineto{\pgfqpoint{2.844440in}{3.382727in}}%
\pgfpathlineto{\pgfqpoint{2.845975in}{0.637273in}}%
\pgfpathlineto{\pgfqpoint{2.845980in}{3.382727in}}%
\pgfpathlineto{\pgfqpoint{2.845989in}{3.382727in}}%
\pgfpathlineto{\pgfqpoint{2.847378in}{0.637273in}}%
\pgfpathlineto{\pgfqpoint{2.847529in}{3.382727in}}%
\pgfpathlineto{\pgfqpoint{2.848818in}{3.382727in}}%
\pgfpathlineto{\pgfqpoint{2.850353in}{0.637273in}}%
\pgfpathlineto{\pgfqpoint{2.850357in}{3.382727in}}%
\pgfpathlineto{\pgfqpoint{2.850367in}{3.382727in}}%
\pgfpathlineto{\pgfqpoint{2.851599in}{0.637273in}}%
\pgfpathlineto{\pgfqpoint{2.851906in}{3.382727in}}%
\pgfpathlineto{\pgfqpoint{2.853040in}{3.382727in}}%
\pgfpathlineto{\pgfqpoint{2.854574in}{0.637273in}}%
\pgfpathlineto{\pgfqpoint{2.854579in}{3.382727in}}%
\pgfpathlineto{\pgfqpoint{2.854589in}{3.382727in}}%
\pgfpathlineto{\pgfqpoint{2.856095in}{0.637273in}}%
\pgfpathlineto{\pgfqpoint{2.856128in}{3.382727in}}%
\pgfpathlineto{\pgfqpoint{2.857535in}{3.382727in}}%
\pgfpathlineto{\pgfqpoint{2.859065in}{0.637273in}}%
\pgfpathlineto{\pgfqpoint{2.859075in}{3.382727in}}%
\pgfpathlineto{\pgfqpoint{2.859080in}{3.382727in}}%
\pgfpathlineto{\pgfqpoint{2.860421in}{0.637273in}}%
\pgfpathlineto{\pgfqpoint{2.860619in}{3.382727in}}%
\pgfpathlineto{\pgfqpoint{2.861861in}{3.382727in}}%
\pgfpathlineto{\pgfqpoint{2.863396in}{0.637273in}}%
\pgfpathlineto{\pgfqpoint{2.863400in}{3.382727in}}%
\pgfpathlineto{\pgfqpoint{2.863410in}{3.382727in}}%
\pgfpathlineto{\pgfqpoint{2.864954in}{0.637273in}}%
\pgfpathlineto{\pgfqpoint{2.868888in}{0.637273in}}%
\pgfpathlineto{\pgfqpoint{2.870432in}{3.382727in}}%
\pgfpathlineto{\pgfqpoint{2.871622in}{3.382727in}}%
\pgfpathlineto{\pgfqpoint{2.873152in}{0.637273in}}%
\pgfpathlineto{\pgfqpoint{2.873161in}{3.382727in}}%
\pgfpathlineto{\pgfqpoint{2.873166in}{3.382727in}}%
\pgfpathlineto{\pgfqpoint{2.874399in}{0.637273in}}%
\pgfpathlineto{\pgfqpoint{2.874706in}{3.382727in}}%
\pgfpathlineto{\pgfqpoint{2.875844in}{3.382727in}}%
\pgfpathlineto{\pgfqpoint{2.877374in}{0.637273in}}%
\pgfpathlineto{\pgfqpoint{2.877383in}{3.382727in}}%
\pgfpathlineto{\pgfqpoint{2.877388in}{3.382727in}}%
\pgfpathlineto{\pgfqpoint{2.878805in}{0.637273in}}%
\pgfpathlineto{\pgfqpoint{2.878927in}{3.382727in}}%
\pgfpathlineto{\pgfqpoint{2.880245in}{3.382727in}}%
\pgfpathlineto{\pgfqpoint{2.881780in}{0.637273in}}%
\pgfpathlineto{\pgfqpoint{2.881784in}{3.382727in}}%
\pgfpathlineto{\pgfqpoint{2.881794in}{3.382727in}}%
\pgfpathlineto{\pgfqpoint{2.883163in}{0.637273in}}%
\pgfpathlineto{\pgfqpoint{2.883333in}{3.382727in}}%
\pgfpathlineto{\pgfqpoint{2.884604in}{3.382727in}}%
\pgfpathlineto{\pgfqpoint{2.886134in}{0.637273in}}%
\pgfpathlineto{\pgfqpoint{2.886143in}{3.382727in}}%
\pgfpathlineto{\pgfqpoint{2.886148in}{3.382727in}}%
\pgfpathlineto{\pgfqpoint{2.887654in}{0.637273in}}%
\pgfpathlineto{\pgfqpoint{2.887687in}{3.382727in}}%
\pgfpathlineto{\pgfqpoint{2.889095in}{3.382727in}}%
\pgfpathlineto{\pgfqpoint{2.890629in}{0.637273in}}%
\pgfpathlineto{\pgfqpoint{2.890634in}{3.382727in}}%
\pgfpathlineto{\pgfqpoint{2.890639in}{3.382727in}}%
\pgfpathlineto{\pgfqpoint{2.892183in}{0.637273in}}%
\pgfpathlineto{\pgfqpoint{2.897458in}{0.637273in}}%
\pgfpathlineto{\pgfqpoint{2.898983in}{3.382727in}}%
\pgfpathlineto{\pgfqpoint{2.898997in}{0.637273in}}%
\pgfpathlineto{\pgfqpoint{2.899224in}{0.637273in}}%
\pgfpathlineto{\pgfqpoint{2.900697in}{3.382727in}}%
\pgfpathlineto{\pgfqpoint{2.900763in}{0.637273in}}%
\pgfpathlineto{\pgfqpoint{2.900811in}{0.637273in}}%
\pgfpathlineto{\pgfqpoint{2.902355in}{3.382727in}}%
\pgfpathlineto{\pgfqpoint{2.903228in}{3.382727in}}%
\pgfpathlineto{\pgfqpoint{2.904711in}{0.637273in}}%
\pgfpathlineto{\pgfqpoint{2.904768in}{3.382727in}}%
\pgfpathlineto{\pgfqpoint{2.904782in}{3.382727in}}%
\pgfpathlineto{\pgfqpoint{2.906322in}{0.637273in}}%
\pgfpathlineto{\pgfqpoint{2.907866in}{3.382727in}}%
\pgfpathlineto{\pgfqpoint{2.907894in}{3.382727in}}%
\pgfpathlineto{\pgfqpoint{2.909438in}{0.637273in}}%
\pgfpathlineto{\pgfqpoint{2.910982in}{3.382727in}}%
\pgfpathlineto{\pgfqpoint{2.912305in}{3.382727in}}%
\pgfpathlineto{\pgfqpoint{2.913839in}{0.637273in}}%
\pgfpathlineto{\pgfqpoint{2.913844in}{3.382727in}}%
\pgfpathlineto{\pgfqpoint{2.913854in}{3.382727in}}%
\pgfpathlineto{\pgfqpoint{2.915360in}{0.637273in}}%
\pgfpathlineto{\pgfqpoint{2.915393in}{3.382727in}}%
\pgfpathlineto{\pgfqpoint{2.916800in}{3.382727in}}%
\pgfpathlineto{\pgfqpoint{2.918335in}{0.637273in}}%
\pgfpathlineto{\pgfqpoint{2.918340in}{3.382727in}}%
\pgfpathlineto{\pgfqpoint{2.918349in}{3.382727in}}%
\pgfpathlineto{\pgfqpoint{2.919893in}{0.637273in}}%
\pgfpathlineto{\pgfqpoint{2.925164in}{0.637273in}}%
\pgfpathlineto{\pgfqpoint{2.926689in}{3.382727in}}%
\pgfpathlineto{\pgfqpoint{2.926703in}{0.637273in}}%
\pgfpathlineto{\pgfqpoint{2.926930in}{0.637273in}}%
\pgfpathlineto{\pgfqpoint{2.928403in}{3.382727in}}%
\pgfpathlineto{\pgfqpoint{2.928469in}{0.637273in}}%
\pgfpathlineto{\pgfqpoint{2.928483in}{0.637273in}}%
\pgfpathlineto{\pgfqpoint{2.930028in}{3.382727in}}%
\pgfpathlineto{\pgfqpoint{2.931085in}{3.382727in}}%
\pgfpathlineto{\pgfqpoint{2.932573in}{0.637273in}}%
\pgfpathlineto{\pgfqpoint{2.932625in}{3.382727in}}%
\pgfpathlineto{\pgfqpoint{2.932644in}{3.382727in}}%
\pgfpathlineto{\pgfqpoint{2.934188in}{0.637273in}}%
\pgfpathlineto{\pgfqpoint{2.934216in}{0.637273in}}%
\pgfpathlineto{\pgfqpoint{2.935742in}{3.382727in}}%
\pgfpathlineto{\pgfqpoint{2.935756in}{0.637273in}}%
\pgfpathlineto{\pgfqpoint{2.935982in}{0.637273in}}%
\pgfpathlineto{\pgfqpoint{2.937456in}{3.382727in}}%
\pgfpathlineto{\pgfqpoint{2.937522in}{0.637273in}}%
\pgfpathlineto{\pgfqpoint{2.937545in}{0.637273in}}%
\pgfpathlineto{\pgfqpoint{2.939090in}{3.382727in}}%
\pgfpathlineto{\pgfqpoint{2.940152in}{3.382727in}}%
\pgfpathlineto{\pgfqpoint{2.941687in}{0.637273in}}%
\pgfpathlineto{\pgfqpoint{2.941692in}{3.382727in}}%
\pgfpathlineto{\pgfqpoint{2.941706in}{3.382727in}}%
\pgfpathlineto{\pgfqpoint{2.943207in}{0.637273in}}%
\pgfpathlineto{\pgfqpoint{2.943245in}{3.382727in}}%
\pgfpathlineto{\pgfqpoint{2.944648in}{3.382727in}}%
\pgfpathlineto{\pgfqpoint{2.946183in}{0.637273in}}%
\pgfpathlineto{\pgfqpoint{2.946187in}{3.382727in}}%
\pgfpathlineto{\pgfqpoint{2.946197in}{3.382727in}}%
\pgfpathlineto{\pgfqpoint{2.947741in}{0.637273in}}%
\pgfpathlineto{\pgfqpoint{2.953011in}{0.637273in}}%
\pgfpathlineto{\pgfqpoint{2.954536in}{3.382727in}}%
\pgfpathlineto{\pgfqpoint{2.954550in}{0.637273in}}%
\pgfpathlineto{\pgfqpoint{2.954777in}{0.637273in}}%
\pgfpathlineto{\pgfqpoint{2.956307in}{3.382727in}}%
\pgfpathlineto{\pgfqpoint{2.956317in}{0.637273in}}%
\pgfpathlineto{\pgfqpoint{2.956321in}{0.637273in}}%
\pgfpathlineto{\pgfqpoint{2.957866in}{3.382727in}}%
\pgfpathlineto{\pgfqpoint{2.958782in}{3.382727in}}%
\pgfpathlineto{\pgfqpoint{2.960312in}{0.637273in}}%
\pgfpathlineto{\pgfqpoint{2.960321in}{3.382727in}}%
\pgfpathlineto{\pgfqpoint{2.960326in}{3.382727in}}%
\pgfpathlineto{\pgfqpoint{2.961870in}{0.637273in}}%
\pgfpathlineto{\pgfqpoint{2.961884in}{0.637273in}}%
\pgfpathlineto{\pgfqpoint{2.963410in}{3.382727in}}%
\pgfpathlineto{\pgfqpoint{2.963424in}{0.637273in}}%
\pgfpathlineto{\pgfqpoint{2.963646in}{0.637273in}}%
\pgfpathlineto{\pgfqpoint{2.965124in}{3.382727in}}%
\pgfpathlineto{\pgfqpoint{2.965185in}{0.637273in}}%
\pgfpathlineto{\pgfqpoint{2.965213in}{0.637273in}}%
\pgfpathlineto{\pgfqpoint{2.966758in}{3.382727in}}%
\pgfpathlineto{\pgfqpoint{2.967820in}{3.382727in}}%
\pgfpathlineto{\pgfqpoint{2.969350in}{0.637273in}}%
\pgfpathlineto{\pgfqpoint{2.969360in}{3.382727in}}%
\pgfpathlineto{\pgfqpoint{2.969364in}{3.382727in}}%
\pgfpathlineto{\pgfqpoint{2.970909in}{0.637273in}}%
\pgfpathlineto{\pgfqpoint{2.970956in}{0.637273in}}%
\pgfpathlineto{\pgfqpoint{2.972481in}{3.382727in}}%
\pgfpathlineto{\pgfqpoint{2.972495in}{0.637273in}}%
\pgfpathlineto{\pgfqpoint{2.972750in}{0.637273in}}%
\pgfpathlineto{\pgfqpoint{2.974224in}{3.382727in}}%
\pgfpathlineto{\pgfqpoint{2.974290in}{0.637273in}}%
\pgfpathlineto{\pgfqpoint{2.974304in}{0.637273in}}%
\pgfpathlineto{\pgfqpoint{2.975848in}{3.382727in}}%
\pgfpathlineto{\pgfqpoint{2.976892in}{3.382727in}}%
\pgfpathlineto{\pgfqpoint{2.978426in}{0.637273in}}%
\pgfpathlineto{\pgfqpoint{2.978431in}{3.382727in}}%
\pgfpathlineto{\pgfqpoint{2.978441in}{3.382727in}}%
\pgfpathlineto{\pgfqpoint{2.979985in}{0.637273in}}%
\pgfpathlineto{\pgfqpoint{2.983918in}{0.637273in}}%
\pgfpathlineto{\pgfqpoint{2.985463in}{3.382727in}}%
\pgfpathlineto{\pgfqpoint{2.986653in}{3.382727in}}%
\pgfpathlineto{\pgfqpoint{2.988183in}{0.637273in}}%
\pgfpathlineto{\pgfqpoint{2.988192in}{3.382727in}}%
\pgfpathlineto{\pgfqpoint{2.988197in}{3.382727in}}%
\pgfpathlineto{\pgfqpoint{2.989618in}{0.637273in}}%
\pgfpathlineto{\pgfqpoint{2.989736in}{3.382727in}}%
\pgfpathlineto{\pgfqpoint{2.991059in}{3.382727in}}%
\pgfpathlineto{\pgfqpoint{2.992546in}{0.637273in}}%
\pgfpathlineto{\pgfqpoint{2.992598in}{3.382727in}}%
\pgfpathlineto{\pgfqpoint{2.992617in}{3.382727in}}%
\pgfpathlineto{\pgfqpoint{2.993977in}{0.637273in}}%
\pgfpathlineto{\pgfqpoint{2.994156in}{3.382727in}}%
\pgfpathlineto{\pgfqpoint{2.995417in}{3.382727in}}%
\pgfpathlineto{\pgfqpoint{2.996947in}{0.637273in}}%
\pgfpathlineto{\pgfqpoint{2.996957in}{3.382727in}}%
\pgfpathlineto{\pgfqpoint{2.996962in}{3.382727in}}%
\pgfpathlineto{\pgfqpoint{2.998336in}{0.637273in}}%
\pgfpathlineto{\pgfqpoint{2.998501in}{3.382727in}}%
\pgfpathlineto{\pgfqpoint{2.999776in}{3.382727in}}%
\pgfpathlineto{\pgfqpoint{3.001306in}{0.637273in}}%
\pgfpathlineto{\pgfqpoint{3.001315in}{3.382727in}}%
\pgfpathlineto{\pgfqpoint{3.001320in}{3.382727in}}%
\pgfpathlineto{\pgfqpoint{3.002751in}{0.637273in}}%
\pgfpathlineto{\pgfqpoint{3.002860in}{3.382727in}}%
\pgfpathlineto{\pgfqpoint{3.004191in}{3.382727in}}%
\pgfpathlineto{\pgfqpoint{3.005679in}{0.637273in}}%
\pgfpathlineto{\pgfqpoint{3.005731in}{3.382727in}}%
\pgfpathlineto{\pgfqpoint{3.005750in}{3.382727in}}%
\pgfpathlineto{\pgfqpoint{3.007247in}{0.637273in}}%
\pgfpathlineto{\pgfqpoint{3.007289in}{3.382727in}}%
\pgfpathlineto{\pgfqpoint{3.007393in}{3.382727in}}%
\pgfpathlineto{\pgfqpoint{3.008928in}{0.637273in}}%
\pgfpathlineto{\pgfqpoint{3.008933in}{3.382727in}}%
\pgfpathlineto{\pgfqpoint{3.008942in}{3.382727in}}%
\pgfpathlineto{\pgfqpoint{3.010458in}{0.637273in}}%
\pgfpathlineto{\pgfqpoint{3.010481in}{3.382727in}}%
\pgfpathlineto{\pgfqpoint{3.010784in}{3.382727in}}%
\pgfpathlineto{\pgfqpoint{3.012328in}{0.637273in}}%
\pgfpathlineto{\pgfqpoint{3.013971in}{0.637273in}}%
\pgfpathlineto{\pgfqpoint{3.015515in}{3.382727in}}%
\pgfpathlineto{\pgfqpoint{3.015520in}{3.382727in}}%
\pgfpathlineto{\pgfqpoint{3.016956in}{0.637273in}}%
\pgfpathlineto{\pgfqpoint{3.017060in}{3.382727in}}%
\pgfpathlineto{\pgfqpoint{3.018396in}{3.382727in}}%
\pgfpathlineto{\pgfqpoint{3.019931in}{0.637273in}}%
\pgfpathlineto{\pgfqpoint{3.019936in}{3.382727in}}%
\pgfpathlineto{\pgfqpoint{3.019945in}{3.382727in}}%
\pgfpathlineto{\pgfqpoint{3.021489in}{0.637273in}}%
\pgfpathlineto{\pgfqpoint{3.032251in}{0.637273in}}%
\pgfpathlineto{\pgfqpoint{3.032299in}{3.382727in}}%
\pgfpathlineto{\pgfqpoint{3.033791in}{0.637273in}}%
\pgfpathlineto{\pgfqpoint{3.045153in}{0.637273in}}%
\pgfpathlineto{\pgfqpoint{3.045200in}{3.382727in}}%
\pgfpathlineto{\pgfqpoint{3.046692in}{0.637273in}}%
\pgfpathlineto{\pgfqpoint{3.056066in}{0.637273in}}%
\pgfpathlineto{\pgfqpoint{3.056094in}{3.382727in}}%
\pgfpathlineto{\pgfqpoint{3.057605in}{0.637273in}}%
\pgfpathlineto{\pgfqpoint{3.066885in}{0.637273in}}%
\pgfpathlineto{\pgfqpoint{3.066913in}{3.382727in}}%
\pgfpathlineto{\pgfqpoint{3.068424in}{0.637273in}}%
\pgfpathlineto{\pgfqpoint{3.078015in}{0.637273in}}%
\pgfpathlineto{\pgfqpoint{3.078025in}{3.382727in}}%
\pgfpathlineto{\pgfqpoint{3.079555in}{0.637273in}}%
\pgfpathlineto{\pgfqpoint{3.088933in}{0.637273in}}%
\pgfpathlineto{\pgfqpoint{3.088962in}{3.382727in}}%
\pgfpathlineto{\pgfqpoint{3.090473in}{0.637273in}}%
\pgfpathlineto{\pgfqpoint{3.094732in}{0.637273in}}%
\pgfpathlineto{\pgfqpoint{3.094760in}{3.382727in}}%
\pgfpathlineto{\pgfqpoint{3.096272in}{0.637273in}}%
\pgfpathlineto{\pgfqpoint{3.105556in}{0.637273in}}%
\pgfpathlineto{\pgfqpoint{3.105584in}{3.382727in}}%
\pgfpathlineto{\pgfqpoint{3.107095in}{0.637273in}}%
\pgfpathlineto{\pgfqpoint{3.116370in}{0.637273in}}%
\pgfpathlineto{\pgfqpoint{3.116398in}{3.382727in}}%
\pgfpathlineto{\pgfqpoint{3.117909in}{0.637273in}}%
\pgfpathlineto{\pgfqpoint{3.127377in}{0.637273in}}%
\pgfpathlineto{\pgfqpoint{3.127406in}{3.382727in}}%
\pgfpathlineto{\pgfqpoint{3.128917in}{0.637273in}}%
\pgfpathlineto{\pgfqpoint{3.138201in}{0.637273in}}%
\pgfpathlineto{\pgfqpoint{3.138229in}{3.382727in}}%
\pgfpathlineto{\pgfqpoint{3.139740in}{0.637273in}}%
\pgfpathlineto{\pgfqpoint{3.150281in}{0.637273in}}%
\pgfpathlineto{\pgfqpoint{3.150309in}{3.382727in}}%
\pgfpathlineto{\pgfqpoint{3.151820in}{0.637273in}}%
\pgfpathlineto{\pgfqpoint{3.253109in}{0.637273in}}%
\pgfpathlineto{\pgfqpoint{3.254653in}{3.382727in}}%
\pgfpathlineto{\pgfqpoint{3.255328in}{3.382727in}}%
\pgfpathlineto{\pgfqpoint{3.256873in}{0.637273in}}%
\pgfpathlineto{\pgfqpoint{3.263630in}{0.637273in}}%
\pgfpathlineto{\pgfqpoint{3.265174in}{3.382727in}}%
\pgfpathlineto{\pgfqpoint{3.265269in}{3.382727in}}%
\pgfpathlineto{\pgfqpoint{3.266813in}{0.637273in}}%
\pgfpathlineto{\pgfqpoint{3.500940in}{0.637273in}}%
\pgfpathlineto{\pgfqpoint{3.502234in}{3.382727in}}%
\pgfpathlineto{\pgfqpoint{3.502480in}{0.637273in}}%
\pgfpathlineto{\pgfqpoint{3.771717in}{0.637273in}}%
\pgfpathlineto{\pgfqpoint{3.772992in}{3.382727in}}%
\pgfpathlineto{\pgfqpoint{3.773257in}{0.637273in}}%
\pgfpathlineto{\pgfqpoint{3.812547in}{0.637273in}}%
\pgfpathlineto{\pgfqpoint{3.813822in}{3.382727in}}%
\pgfpathlineto{\pgfqpoint{3.814086in}{0.637273in}}%
\pgfpathlineto{\pgfqpoint{4.219114in}{0.637273in}}%
\pgfpathlineto{\pgfqpoint{4.220658in}{3.382727in}}%
\pgfpathlineto{\pgfqpoint{4.220662in}{3.382727in}}%
\pgfpathlineto{\pgfqpoint{4.222207in}{0.637273in}}%
\pgfpathlineto{\pgfqpoint{4.361741in}{0.637273in}}%
\pgfpathlineto{\pgfqpoint{4.363262in}{3.382727in}}%
\pgfpathlineto{\pgfqpoint{4.363281in}{0.637273in}}%
\pgfpathlineto{\pgfqpoint{4.363300in}{0.637273in}}%
\pgfpathlineto{\pgfqpoint{4.363371in}{3.382727in}}%
\pgfpathlineto{\pgfqpoint{4.364839in}{0.637273in}}%
\pgfpathlineto{\pgfqpoint{4.459002in}{0.637273in}}%
\pgfpathlineto{\pgfqpoint{4.460508in}{3.382727in}}%
\pgfpathlineto{\pgfqpoint{4.460542in}{0.637273in}}%
\pgfpathlineto{\pgfqpoint{4.493371in}{0.637273in}}%
\pgfpathlineto{\pgfqpoint{4.494778in}{3.382727in}}%
\pgfpathlineto{\pgfqpoint{4.494911in}{0.637273in}}%
\pgfpathlineto{\pgfqpoint{4.522276in}{0.637273in}}%
\pgfpathlineto{\pgfqpoint{4.523650in}{3.382727in}}%
\pgfpathlineto{\pgfqpoint{4.523816in}{0.637273in}}%
\pgfpathlineto{\pgfqpoint{4.523858in}{0.637273in}}%
\pgfpathlineto{\pgfqpoint{4.523953in}{3.382727in}}%
\pgfpathlineto{\pgfqpoint{4.525398in}{0.637273in}}%
\pgfpathlineto{\pgfqpoint{4.562468in}{0.637273in}}%
\pgfpathlineto{\pgfqpoint{4.564012in}{3.382727in}}%
\pgfpathlineto{\pgfqpoint{4.565358in}{3.382727in}}%
\pgfpathlineto{\pgfqpoint{4.566902in}{0.637273in}}%
\pgfpathlineto{\pgfqpoint{4.570916in}{0.637273in}}%
\pgfpathlineto{\pgfqpoint{4.572319in}{3.382727in}}%
\pgfpathlineto{\pgfqpoint{4.572456in}{0.637273in}}%
\pgfpathlineto{\pgfqpoint{4.572550in}{0.637273in}}%
\pgfpathlineto{\pgfqpoint{4.574033in}{3.382727in}}%
\pgfpathlineto{\pgfqpoint{4.574089in}{0.637273in}}%
\pgfpathlineto{\pgfqpoint{4.574264in}{0.637273in}}%
\pgfpathlineto{\pgfqpoint{4.574292in}{3.382727in}}%
\pgfpathlineto{\pgfqpoint{4.575804in}{0.637273in}}%
\pgfpathlineto{\pgfqpoint{4.580082in}{0.637273in}}%
\pgfpathlineto{\pgfqpoint{4.581485in}{3.382727in}}%
\pgfpathlineto{\pgfqpoint{4.581622in}{0.637273in}}%
\pgfpathlineto{\pgfqpoint{4.581716in}{0.637273in}}%
\pgfpathlineto{\pgfqpoint{4.583260in}{3.382727in}}%
\pgfpathlineto{\pgfqpoint{4.584545in}{3.382727in}}%
\pgfpathlineto{\pgfqpoint{4.584752in}{0.637273in}}%
\pgfpathlineto{\pgfqpoint{4.586084in}{3.382727in}}%
\pgfpathlineto{\pgfqpoint{4.586127in}{3.382727in}}%
\pgfpathlineto{\pgfqpoint{4.587671in}{0.637273in}}%
\pgfpathlineto{\pgfqpoint{4.591826in}{0.637273in}}%
\pgfpathlineto{\pgfqpoint{4.593342in}{3.382727in}}%
\pgfpathlineto{\pgfqpoint{4.593366in}{0.637273in}}%
\pgfpathlineto{\pgfqpoint{4.593380in}{0.637273in}}%
\pgfpathlineto{\pgfqpoint{4.594882in}{3.382727in}}%
\pgfpathlineto{\pgfqpoint{4.594920in}{0.637273in}}%
\pgfpathlineto{\pgfqpoint{4.595113in}{0.637273in}}%
\pgfpathlineto{\pgfqpoint{4.595141in}{3.382727in}}%
\pgfpathlineto{\pgfqpoint{4.596653in}{0.637273in}}%
\pgfpathlineto{\pgfqpoint{4.600662in}{0.637273in}}%
\pgfpathlineto{\pgfqpoint{4.602187in}{3.382727in}}%
\pgfpathlineto{\pgfqpoint{4.602201in}{0.637273in}}%
\pgfpathlineto{\pgfqpoint{4.602291in}{0.637273in}}%
\pgfpathlineto{\pgfqpoint{4.603774in}{3.382727in}}%
\pgfpathlineto{\pgfqpoint{4.603831in}{0.637273in}}%
\pgfpathlineto{\pgfqpoint{4.604005in}{0.637273in}}%
\pgfpathlineto{\pgfqpoint{4.605526in}{3.382727in}}%
\pgfpathlineto{\pgfqpoint{4.605545in}{0.637273in}}%
\pgfpathlineto{\pgfqpoint{4.605668in}{0.637273in}}%
\pgfpathlineto{\pgfqpoint{4.605757in}{3.382727in}}%
\pgfpathlineto{\pgfqpoint{4.607207in}{0.637273in}}%
\pgfpathlineto{\pgfqpoint{4.611221in}{0.637273in}}%
\pgfpathlineto{\pgfqpoint{4.612600in}{3.382727in}}%
\pgfpathlineto{\pgfqpoint{4.612760in}{0.637273in}}%
\pgfpathlineto{\pgfqpoint{4.612770in}{0.637273in}}%
\pgfpathlineto{\pgfqpoint{4.614314in}{3.382727in}}%
\pgfpathlineto{\pgfqpoint{4.614361in}{3.382727in}}%
\pgfpathlineto{\pgfqpoint{4.614682in}{0.637273in}}%
\pgfpathlineto{\pgfqpoint{4.615901in}{3.382727in}}%
\pgfpathlineto{\pgfqpoint{4.615948in}{3.382727in}}%
\pgfpathlineto{\pgfqpoint{4.616227in}{0.637273in}}%
\pgfpathlineto{\pgfqpoint{4.617487in}{3.382727in}}%
\pgfpathlineto{\pgfqpoint{4.617605in}{3.382727in}}%
\pgfpathlineto{\pgfqpoint{4.619150in}{0.637273in}}%
\pgfpathlineto{\pgfqpoint{4.623371in}{0.637273in}}%
\pgfpathlineto{\pgfqpoint{4.624906in}{3.382727in}}%
\pgfpathlineto{\pgfqpoint{4.624911in}{0.637273in}}%
\pgfpathlineto{\pgfqpoint{4.624949in}{0.637273in}}%
\pgfpathlineto{\pgfqpoint{4.626441in}{3.382727in}}%
\pgfpathlineto{\pgfqpoint{4.626488in}{0.637273in}}%
\pgfpathlineto{\pgfqpoint{4.626601in}{0.637273in}}%
\pgfpathlineto{\pgfqpoint{4.628146in}{3.382727in}}%
\pgfpathlineto{\pgfqpoint{4.628240in}{3.382727in}}%
\pgfpathlineto{\pgfqpoint{4.629784in}{0.637273in}}%
\pgfpathlineto{\pgfqpoint{4.629912in}{0.637273in}}%
\pgfpathlineto{\pgfqpoint{4.631447in}{3.382727in}}%
\pgfpathlineto{\pgfqpoint{4.631451in}{0.637273in}}%
\pgfpathlineto{\pgfqpoint{4.631456in}{0.637273in}}%
\pgfpathlineto{\pgfqpoint{4.632977in}{3.382727in}}%
\pgfpathlineto{\pgfqpoint{4.632995in}{0.637273in}}%
\pgfpathlineto{\pgfqpoint{4.633033in}{0.637273in}}%
\pgfpathlineto{\pgfqpoint{4.634559in}{3.382727in}}%
\pgfpathlineto{\pgfqpoint{4.634573in}{0.637273in}}%
\pgfpathlineto{\pgfqpoint{4.634705in}{0.637273in}}%
\pgfpathlineto{\pgfqpoint{4.636235in}{3.382727in}}%
\pgfpathlineto{\pgfqpoint{4.636244in}{0.637273in}}%
\pgfpathlineto{\pgfqpoint{4.636296in}{0.637273in}}%
\pgfpathlineto{\pgfqpoint{4.637822in}{3.382727in}}%
\pgfpathlineto{\pgfqpoint{4.637836in}{0.637273in}}%
\pgfpathlineto{\pgfqpoint{4.638067in}{0.637273in}}%
\pgfpathlineto{\pgfqpoint{4.639611in}{3.382727in}}%
\pgfpathlineto{\pgfqpoint{4.640886in}{3.382727in}}%
\pgfpathlineto{\pgfqpoint{4.642431in}{0.637273in}}%
\pgfpathlineto{\pgfqpoint{4.642435in}{0.637273in}}%
\pgfpathlineto{\pgfqpoint{4.643980in}{3.382727in}}%
\pgfpathlineto{\pgfqpoint{4.645340in}{3.382727in}}%
\pgfpathlineto{\pgfqpoint{4.646884in}{0.637273in}}%
\pgfpathlineto{\pgfqpoint{4.646903in}{0.637273in}}%
\pgfpathlineto{\pgfqpoint{4.648447in}{3.382727in}}%
\pgfpathlineto{\pgfqpoint{4.649363in}{3.382727in}}%
\pgfpathlineto{\pgfqpoint{4.650827in}{0.637273in}}%
\pgfpathlineto{\pgfqpoint{4.650902in}{3.382727in}}%
\pgfpathlineto{\pgfqpoint{4.652248in}{3.382727in}}%
\pgfpathlineto{\pgfqpoint{4.652739in}{0.637273in}}%
\pgfpathlineto{\pgfqpoint{4.653788in}{3.382727in}}%
\pgfpathlineto{\pgfqpoint{4.654085in}{3.382727in}}%
\pgfpathlineto{\pgfqpoint{4.655630in}{0.637273in}}%
\pgfpathlineto{\pgfqpoint{4.655936in}{0.637273in}}%
\pgfpathlineto{\pgfqpoint{4.657481in}{3.382727in}}%
\pgfpathlineto{\pgfqpoint{4.657547in}{3.382727in}}%
\pgfpathlineto{\pgfqpoint{4.659091in}{0.637273in}}%
\pgfpathlineto{\pgfqpoint{4.661211in}{0.637273in}}%
\pgfpathlineto{\pgfqpoint{4.662755in}{3.382727in}}%
\pgfpathlineto{\pgfqpoint{4.663247in}{3.382727in}}%
\pgfpathlineto{\pgfqpoint{4.664791in}{0.637273in}}%
\pgfpathlineto{\pgfqpoint{4.666335in}{3.382727in}}%
\pgfpathlineto{\pgfqpoint{4.666642in}{3.382727in}}%
\pgfpathlineto{\pgfqpoint{4.668186in}{0.637273in}}%
\pgfpathlineto{\pgfqpoint{4.668191in}{0.637273in}}%
\pgfpathlineto{\pgfqpoint{4.669735in}{3.382727in}}%
\pgfpathlineto{\pgfqpoint{4.670292in}{3.382727in}}%
\pgfpathlineto{\pgfqpoint{4.671836in}{0.637273in}}%
\pgfpathlineto{\pgfqpoint{4.675765in}{0.637273in}}%
\pgfpathlineto{\pgfqpoint{4.676101in}{3.382727in}}%
\pgfpathlineto{\pgfqpoint{4.677305in}{0.637273in}}%
\pgfpathlineto{\pgfqpoint{4.680162in}{0.637273in}}%
\pgfpathlineto{\pgfqpoint{4.681706in}{3.382727in}}%
\pgfpathlineto{\pgfqpoint{4.690461in}{3.382727in}}%
\pgfpathlineto{\pgfqpoint{4.690900in}{0.637273in}}%
\pgfpathlineto{\pgfqpoint{4.692001in}{3.382727in}}%
\pgfpathlineto{\pgfqpoint{4.694716in}{3.382727in}}%
\pgfpathlineto{\pgfqpoint{4.694735in}{0.637273in}}%
\pgfpathlineto{\pgfqpoint{4.696256in}{3.382727in}}%
\pgfpathlineto{\pgfqpoint{4.697752in}{3.382727in}}%
\pgfpathlineto{\pgfqpoint{4.698428in}{0.637273in}}%
\pgfpathlineto{\pgfqpoint{4.699292in}{3.382727in}}%
\pgfpathlineto{\pgfqpoint{4.699774in}{3.382727in}}%
\pgfpathlineto{\pgfqpoint{4.701313in}{0.637273in}}%
\pgfpathlineto{\pgfqpoint{4.702848in}{3.382727in}}%
\pgfpathlineto{\pgfqpoint{4.702853in}{0.637273in}}%
\pgfpathlineto{\pgfqpoint{4.702867in}{0.637273in}}%
\pgfpathlineto{\pgfqpoint{4.704411in}{3.382727in}}%
\pgfpathlineto{\pgfqpoint{4.763190in}{3.382727in}}%
\pgfpathlineto{\pgfqpoint{4.763232in}{0.637273in}}%
\pgfpathlineto{\pgfqpoint{4.764729in}{3.382727in}}%
\pgfpathlineto{\pgfqpoint{4.768450in}{3.382727in}}%
\pgfpathlineto{\pgfqpoint{4.769801in}{0.637273in}}%
\pgfpathlineto{\pgfqpoint{4.769990in}{3.382727in}}%
\pgfpathlineto{\pgfqpoint{4.770060in}{3.382727in}}%
\pgfpathlineto{\pgfqpoint{4.771605in}{0.637273in}}%
\pgfpathlineto{\pgfqpoint{4.771973in}{0.637273in}}%
\pgfpathlineto{\pgfqpoint{4.773347in}{3.382727in}}%
\pgfpathlineto{\pgfqpoint{4.773512in}{0.637273in}}%
\pgfpathlineto{\pgfqpoint{4.773621in}{0.637273in}}%
\pgfpathlineto{\pgfqpoint{4.773654in}{3.382727in}}%
\pgfpathlineto{\pgfqpoint{4.775161in}{0.637273in}}%
\pgfpathlineto{\pgfqpoint{4.779269in}{0.637273in}}%
\pgfpathlineto{\pgfqpoint{4.780671in}{3.382727in}}%
\pgfpathlineto{\pgfqpoint{4.780808in}{0.637273in}}%
\pgfpathlineto{\pgfqpoint{4.780912in}{0.637273in}}%
\pgfpathlineto{\pgfqpoint{4.782452in}{3.382727in}}%
\pgfpathlineto{\pgfqpoint{4.782631in}{0.637273in}}%
\pgfpathlineto{\pgfqpoint{4.783991in}{3.382727in}}%
\pgfpathlineto{\pgfqpoint{4.784005in}{3.382727in}}%
\pgfpathlineto{\pgfqpoint{4.785550in}{0.637273in}}%
\pgfpathlineto{\pgfqpoint{4.932725in}{0.637273in}}%
\pgfpathlineto{\pgfqpoint{4.934000in}{3.382727in}}%
\pgfpathlineto{\pgfqpoint{4.934265in}{0.637273in}}%
\pgfpathlineto{\pgfqpoint{4.939742in}{0.637273in}}%
\pgfpathlineto{\pgfqpoint{4.941017in}{3.382727in}}%
\pgfpathlineto{\pgfqpoint{4.941282in}{0.637273in}}%
\pgfpathlineto{\pgfqpoint{5.074933in}{0.637273in}}%
\pgfpathlineto{\pgfqpoint{5.074942in}{3.382727in}}%
\pgfpathlineto{\pgfqpoint{5.076472in}{0.637273in}}%
\pgfpathlineto{\pgfqpoint{5.078267in}{0.637273in}}%
\pgfpathlineto{\pgfqpoint{5.078271in}{3.382727in}}%
\pgfpathlineto{\pgfqpoint{5.079806in}{0.637273in}}%
\pgfpathlineto{\pgfqpoint{5.087532in}{0.637273in}}%
\pgfpathlineto{\pgfqpoint{5.087928in}{3.382727in}}%
\pgfpathlineto{\pgfqpoint{5.089071in}{0.637273in}}%
\pgfpathlineto{\pgfqpoint{5.089780in}{0.637273in}}%
\pgfpathlineto{\pgfqpoint{5.089865in}{3.382727in}}%
\pgfpathlineto{\pgfqpoint{5.091319in}{0.637273in}}%
\pgfpathlineto{\pgfqpoint{5.101000in}{0.637273in}}%
\pgfpathlineto{\pgfqpoint{5.101151in}{3.382727in}}%
\pgfpathlineto{\pgfqpoint{5.102539in}{0.637273in}}%
\pgfpathlineto{\pgfqpoint{5.105495in}{0.637273in}}%
\pgfpathlineto{\pgfqpoint{5.106818in}{3.382727in}}%
\pgfpathlineto{\pgfqpoint{5.107035in}{0.637273in}}%
\pgfpathlineto{\pgfqpoint{5.113911in}{0.637273in}}%
\pgfpathlineto{\pgfqpoint{5.113953in}{3.382727in}}%
\pgfpathlineto{\pgfqpoint{5.115450in}{0.637273in}}%
\pgfpathlineto{\pgfqpoint{5.119488in}{0.637273in}}%
\pgfpathlineto{\pgfqpoint{5.121032in}{3.382727in}}%
\pgfpathlineto{\pgfqpoint{5.122491in}{3.382727in}}%
\pgfpathlineto{\pgfqpoint{5.123247in}{0.637273in}}%
\pgfpathlineto{\pgfqpoint{5.124030in}{3.382727in}}%
\pgfpathlineto{\pgfqpoint{5.124238in}{3.382727in}}%
\pgfpathlineto{\pgfqpoint{5.125735in}{0.637273in}}%
\pgfpathlineto{\pgfqpoint{5.125778in}{3.382727in}}%
\pgfpathlineto{\pgfqpoint{5.125806in}{3.382727in}}%
\pgfpathlineto{\pgfqpoint{5.127298in}{0.637273in}}%
\pgfpathlineto{\pgfqpoint{5.127346in}{3.382727in}}%
\pgfpathlineto{\pgfqpoint{5.127369in}{3.382727in}}%
\pgfpathlineto{\pgfqpoint{5.128913in}{0.637273in}}%
\pgfpathlineto{\pgfqpoint{5.134287in}{0.637273in}}%
\pgfpathlineto{\pgfqpoint{5.134297in}{3.382727in}}%
\pgfpathlineto{\pgfqpoint{5.135827in}{0.637273in}}%
\pgfpathlineto{\pgfqpoint{5.148941in}{0.637273in}}%
\pgfpathlineto{\pgfqpoint{5.149196in}{3.382727in}}%
\pgfpathlineto{\pgfqpoint{5.150480in}{0.637273in}}%
\pgfpathlineto{\pgfqpoint{5.151363in}{0.637273in}}%
\pgfpathlineto{\pgfqpoint{5.151510in}{3.382727in}}%
\pgfpathlineto{\pgfqpoint{5.152903in}{0.637273in}}%
\pgfpathlineto{\pgfqpoint{5.162574in}{0.637273in}}%
\pgfpathlineto{\pgfqpoint{5.163183in}{3.382727in}}%
\pgfpathlineto{\pgfqpoint{5.164113in}{0.637273in}}%
\pgfpathlineto{\pgfqpoint{5.170021in}{0.637273in}}%
\pgfpathlineto{\pgfqpoint{5.170026in}{3.382727in}}%
\pgfpathlineto{\pgfqpoint{5.171560in}{0.637273in}}%
\pgfpathlineto{\pgfqpoint{5.177317in}{0.637273in}}%
\pgfpathlineto{\pgfqpoint{5.177355in}{3.382727in}}%
\pgfpathlineto{\pgfqpoint{5.178856in}{0.637273in}}%
\pgfpathlineto{\pgfqpoint{5.188636in}{0.637273in}}%
\pgfpathlineto{\pgfqpoint{5.188636in}{0.637273in}}%
\pgfusepath{stroke}%
\end{pgfscope}%
\begin{pgfscope}%
\pgfpathrectangle{\pgfqpoint{0.750000in}{0.500000in}}{\pgfqpoint{4.650000in}{3.020000in}}%
\pgfusepath{clip}%
\pgfsetrectcap%
\pgfsetroundjoin%
\pgfsetlinewidth{1.505625pt}%
\definecolor{currentstroke}{rgb}{1.000000,0.000000,0.000000}%
\pgfsetstrokecolor{currentstroke}%
\pgfsetdash{}{0pt}%
\pgfpathmoveto{\pgfqpoint{1.013309in}{0.500000in}}%
\pgfpathlineto{\pgfqpoint{1.013309in}{3.520000in}}%
\pgfusepath{stroke}%
\end{pgfscope}%
\begin{pgfscope}%
\pgfpathrectangle{\pgfqpoint{0.750000in}{0.500000in}}{\pgfqpoint{4.650000in}{3.020000in}}%
\pgfusepath{clip}%
\pgfsetrectcap%
\pgfsetroundjoin%
\pgfsetlinewidth{1.505625pt}%
\definecolor{currentstroke}{rgb}{1.000000,0.000000,0.000000}%
\pgfsetstrokecolor{currentstroke}%
\pgfsetdash{}{0pt}%
\pgfpathmoveto{\pgfqpoint{1.967640in}{0.500000in}}%
\pgfpathlineto{\pgfqpoint{1.967640in}{3.520000in}}%
\pgfusepath{stroke}%
\end{pgfscope}%
\begin{pgfscope}%
\pgfpathrectangle{\pgfqpoint{0.750000in}{0.500000in}}{\pgfqpoint{4.650000in}{3.020000in}}%
\pgfusepath{clip}%
\pgfsetrectcap%
\pgfsetroundjoin%
\pgfsetlinewidth{1.505625pt}%
\definecolor{currentstroke}{rgb}{1.000000,0.000000,0.000000}%
\pgfsetstrokecolor{currentstroke}%
\pgfsetdash{}{0pt}%
\pgfpathmoveto{\pgfqpoint{2.081514in}{0.500000in}}%
\pgfpathlineto{\pgfqpoint{2.081514in}{3.520000in}}%
\pgfusepath{stroke}%
\end{pgfscope}%
\begin{pgfscope}%
\pgfpathrectangle{\pgfqpoint{0.750000in}{0.500000in}}{\pgfqpoint{4.650000in}{3.020000in}}%
\pgfusepath{clip}%
\pgfsetrectcap%
\pgfsetroundjoin%
\pgfsetlinewidth{1.505625pt}%
\definecolor{currentstroke}{rgb}{1.000000,0.000000,0.000000}%
\pgfsetstrokecolor{currentstroke}%
\pgfsetdash{}{0pt}%
\pgfpathmoveto{\pgfqpoint{3.253354in}{0.500000in}}%
\pgfpathlineto{\pgfqpoint{3.253354in}{3.520000in}}%
\pgfusepath{stroke}%
\end{pgfscope}%
\begin{pgfscope}%
\pgfpathrectangle{\pgfqpoint{0.750000in}{0.500000in}}{\pgfqpoint{4.650000in}{3.020000in}}%
\pgfusepath{clip}%
\pgfsetrectcap%
\pgfsetroundjoin%
\pgfsetlinewidth{1.505625pt}%
\definecolor{currentstroke}{rgb}{1.000000,0.000000,0.000000}%
\pgfsetstrokecolor{currentstroke}%
\pgfsetdash{}{0pt}%
\pgfpathmoveto{\pgfqpoint{4.680738in}{0.500000in}}%
\pgfpathlineto{\pgfqpoint{4.680738in}{3.520000in}}%
\pgfusepath{stroke}%
\end{pgfscope}%
\begin{pgfscope}%
\pgfpathrectangle{\pgfqpoint{0.750000in}{0.500000in}}{\pgfqpoint{4.650000in}{3.020000in}}%
\pgfusepath{clip}%
\pgfsetrectcap%
\pgfsetroundjoin%
\pgfsetlinewidth{1.505625pt}%
\definecolor{currentstroke}{rgb}{1.000000,0.000000,0.000000}%
\pgfsetstrokecolor{currentstroke}%
\pgfsetdash{}{0pt}%
\pgfpathmoveto{\pgfqpoint{4.817326in}{0.500000in}}%
\pgfpathlineto{\pgfqpoint{4.817326in}{3.520000in}}%
\pgfusepath{stroke}%
\end{pgfscope}%
\begin{pgfscope}%
\pgfsetrectcap%
\pgfsetmiterjoin%
\pgfsetlinewidth{0.803000pt}%
\definecolor{currentstroke}{rgb}{0.000000,0.000000,0.000000}%
\pgfsetstrokecolor{currentstroke}%
\pgfsetdash{}{0pt}%
\pgfpathmoveto{\pgfqpoint{0.750000in}{0.500000in}}%
\pgfpathlineto{\pgfqpoint{0.750000in}{3.520000in}}%
\pgfusepath{stroke}%
\end{pgfscope}%
\begin{pgfscope}%
\pgfsetrectcap%
\pgfsetmiterjoin%
\pgfsetlinewidth{0.803000pt}%
\definecolor{currentstroke}{rgb}{0.000000,0.000000,0.000000}%
\pgfsetstrokecolor{currentstroke}%
\pgfsetdash{}{0pt}%
\pgfpathmoveto{\pgfqpoint{5.400000in}{0.500000in}}%
\pgfpathlineto{\pgfqpoint{5.400000in}{3.520000in}}%
\pgfusepath{stroke}%
\end{pgfscope}%
\begin{pgfscope}%
\pgfsetrectcap%
\pgfsetmiterjoin%
\pgfsetlinewidth{0.803000pt}%
\definecolor{currentstroke}{rgb}{0.000000,0.000000,0.000000}%
\pgfsetstrokecolor{currentstroke}%
\pgfsetdash{}{0pt}%
\pgfpathmoveto{\pgfqpoint{0.750000in}{0.500000in}}%
\pgfpathlineto{\pgfqpoint{5.400000in}{0.500000in}}%
\pgfusepath{stroke}%
\end{pgfscope}%
\begin{pgfscope}%
\pgfsetrectcap%
\pgfsetmiterjoin%
\pgfsetlinewidth{0.803000pt}%
\definecolor{currentstroke}{rgb}{0.000000,0.000000,0.000000}%
\pgfsetstrokecolor{currentstroke}%
\pgfsetdash{}{0pt}%
\pgfpathmoveto{\pgfqpoint{0.750000in}{3.520000in}}%
\pgfpathlineto{\pgfqpoint{5.400000in}{3.520000in}}%
\pgfusepath{stroke}%
\end{pgfscope}%
\end{pgfpicture}%
\makeatother%
\endgroup%

    \caption{BETH Suspicious Outliers}
    \label{fig:beth_sus_outliers}
\end{figure}

Figure \ref{fig:beth_evil_outliers} shows the algorithms detection results compared to the values considered evil by the creators of the BETH dataset. The algorithm was able to detect the starts of the outliers within the debounce threshold and was able to detect the end of the faults with some delay. There are 2 false positives from the detector, which correspond to high UserID values in the dataset. These two locations are marked as suspicious so in this case the false positive would still be worth investigating even if it is deemed not evil.
 
\begin{figure}[H]
    %%\centering
    %% Creator: Matplotlib, PGF backend
%%
%% To include the figure in your LaTeX document, write
%%   \input{<filename>.pgf}
%%
%% Make sure the required packages are loaded in your preamble
%%   \usepackage{pgf}
%%
%% Also ensure that all the required font packages are loaded; for instance,
%% the lmodern package is sometimes necessary when using math font.
%%   \usepackage{lmodern}
%%
%% Figures using additional raster images can only be included by \input if
%% they are in the same directory as the main LaTeX file. For loading figures
%% from other directories you can use the `import` package
%%   \usepackage{import}
%%
%% and then include the figures with
%%   \import{<path to file>}{<filename>.pgf}
%%
%% Matplotlib used the following preamble
%%
\begingroup%
\makeatletter%
\begin{pgfpicture}%
\pgfpathrectangle{\pgfpointorigin}{\pgfqpoint{6.000000in}{4.000000in}}%
\pgfusepath{use as bounding box, clip}%
\begin{pgfscope}%
\pgfsetbuttcap%
\pgfsetmiterjoin%
\pgfsetlinewidth{0.000000pt}%
\definecolor{currentstroke}{rgb}{1.000000,1.000000,1.000000}%
\pgfsetstrokecolor{currentstroke}%
\pgfsetstrokeopacity{0.000000}%
\pgfsetdash{}{0pt}%
\pgfpathmoveto{\pgfqpoint{0.000000in}{0.000000in}}%
\pgfpathlineto{\pgfqpoint{6.000000in}{0.000000in}}%
\pgfpathlineto{\pgfqpoint{6.000000in}{4.000000in}}%
\pgfpathlineto{\pgfqpoint{0.000000in}{4.000000in}}%
\pgfpathlineto{\pgfqpoint{0.000000in}{0.000000in}}%
\pgfpathclose%
\pgfusepath{}%
\end{pgfscope}%
\begin{pgfscope}%
\pgfsetbuttcap%
\pgfsetmiterjoin%
\definecolor{currentfill}{rgb}{1.000000,1.000000,1.000000}%
\pgfsetfillcolor{currentfill}%
\pgfsetlinewidth{0.000000pt}%
\definecolor{currentstroke}{rgb}{0.000000,0.000000,0.000000}%
\pgfsetstrokecolor{currentstroke}%
\pgfsetstrokeopacity{0.000000}%
\pgfsetdash{}{0pt}%
\pgfpathmoveto{\pgfqpoint{0.750000in}{0.500000in}}%
\pgfpathlineto{\pgfqpoint{5.400000in}{0.500000in}}%
\pgfpathlineto{\pgfqpoint{5.400000in}{3.520000in}}%
\pgfpathlineto{\pgfqpoint{0.750000in}{3.520000in}}%
\pgfpathlineto{\pgfqpoint{0.750000in}{0.500000in}}%
\pgfpathclose%
\pgfusepath{fill}%
\end{pgfscope}%
\begin{pgfscope}%
\pgfsetbuttcap%
\pgfsetroundjoin%
\definecolor{currentfill}{rgb}{0.000000,0.000000,0.000000}%
\pgfsetfillcolor{currentfill}%
\pgfsetlinewidth{0.803000pt}%
\definecolor{currentstroke}{rgb}{0.000000,0.000000,0.000000}%
\pgfsetstrokecolor{currentstroke}%
\pgfsetdash{}{0pt}%
\pgfsys@defobject{currentmarker}{\pgfqpoint{0.000000in}{-0.048611in}}{\pgfqpoint{0.000000in}{0.000000in}}{%
\pgfpathmoveto{\pgfqpoint{0.000000in}{0.000000in}}%
\pgfpathlineto{\pgfqpoint{0.000000in}{-0.048611in}}%
\pgfusepath{stroke,fill}%
}%
\begin{pgfscope}%
\pgfsys@transformshift{0.961364in}{0.500000in}%
\pgfsys@useobject{currentmarker}{}%
\end{pgfscope}%
\end{pgfscope}%
\begin{pgfscope}%
\definecolor{textcolor}{rgb}{0.000000,0.000000,0.000000}%
\pgfsetstrokecolor{textcolor}%
\pgfsetfillcolor{textcolor}%
\pgftext[x=0.961364in,y=0.402778in,,top]{\color{textcolor}\rmfamily\fontsize{10.000000}{12.000000}\selectfont \(\displaystyle {0}\)}%
\end{pgfscope}%
\begin{pgfscope}%
\pgfsetbuttcap%
\pgfsetroundjoin%
\definecolor{currentfill}{rgb}{0.000000,0.000000,0.000000}%
\pgfsetfillcolor{currentfill}%
\pgfsetlinewidth{0.803000pt}%
\definecolor{currentstroke}{rgb}{0.000000,0.000000,0.000000}%
\pgfsetstrokecolor{currentstroke}%
\pgfsetdash{}{0pt}%
\pgfsys@defobject{currentmarker}{\pgfqpoint{0.000000in}{-0.048611in}}{\pgfqpoint{0.000000in}{0.000000in}}{%
\pgfpathmoveto{\pgfqpoint{0.000000in}{0.000000in}}%
\pgfpathlineto{\pgfqpoint{0.000000in}{-0.048611in}}%
\pgfusepath{stroke,fill}%
}%
\begin{pgfscope}%
\pgfsys@transformshift{1.905825in}{0.500000in}%
\pgfsys@useobject{currentmarker}{}%
\end{pgfscope}%
\end{pgfscope}%
\begin{pgfscope}%
\definecolor{textcolor}{rgb}{0.000000,0.000000,0.000000}%
\pgfsetstrokecolor{textcolor}%
\pgfsetfillcolor{textcolor}%
\pgftext[x=1.905825in,y=0.402778in,,top]{\color{textcolor}\rmfamily\fontsize{10.000000}{12.000000}\selectfont \(\displaystyle {200000}\)}%
\end{pgfscope}%
\begin{pgfscope}%
\pgfsetbuttcap%
\pgfsetroundjoin%
\definecolor{currentfill}{rgb}{0.000000,0.000000,0.000000}%
\pgfsetfillcolor{currentfill}%
\pgfsetlinewidth{0.803000pt}%
\definecolor{currentstroke}{rgb}{0.000000,0.000000,0.000000}%
\pgfsetstrokecolor{currentstroke}%
\pgfsetdash{}{0pt}%
\pgfsys@defobject{currentmarker}{\pgfqpoint{0.000000in}{-0.048611in}}{\pgfqpoint{0.000000in}{0.000000in}}{%
\pgfpathmoveto{\pgfqpoint{0.000000in}{0.000000in}}%
\pgfpathlineto{\pgfqpoint{0.000000in}{-0.048611in}}%
\pgfusepath{stroke,fill}%
}%
\begin{pgfscope}%
\pgfsys@transformshift{2.850287in}{0.500000in}%
\pgfsys@useobject{currentmarker}{}%
\end{pgfscope}%
\end{pgfscope}%
\begin{pgfscope}%
\definecolor{textcolor}{rgb}{0.000000,0.000000,0.000000}%
\pgfsetstrokecolor{textcolor}%
\pgfsetfillcolor{textcolor}%
\pgftext[x=2.850287in,y=0.402778in,,top]{\color{textcolor}\rmfamily\fontsize{10.000000}{12.000000}\selectfont \(\displaystyle {400000}\)}%
\end{pgfscope}%
\begin{pgfscope}%
\pgfsetbuttcap%
\pgfsetroundjoin%
\definecolor{currentfill}{rgb}{0.000000,0.000000,0.000000}%
\pgfsetfillcolor{currentfill}%
\pgfsetlinewidth{0.803000pt}%
\definecolor{currentstroke}{rgb}{0.000000,0.000000,0.000000}%
\pgfsetstrokecolor{currentstroke}%
\pgfsetdash{}{0pt}%
\pgfsys@defobject{currentmarker}{\pgfqpoint{0.000000in}{-0.048611in}}{\pgfqpoint{0.000000in}{0.000000in}}{%
\pgfpathmoveto{\pgfqpoint{0.000000in}{0.000000in}}%
\pgfpathlineto{\pgfqpoint{0.000000in}{-0.048611in}}%
\pgfusepath{stroke,fill}%
}%
\begin{pgfscope}%
\pgfsys@transformshift{3.794748in}{0.500000in}%
\pgfsys@useobject{currentmarker}{}%
\end{pgfscope}%
\end{pgfscope}%
\begin{pgfscope}%
\definecolor{textcolor}{rgb}{0.000000,0.000000,0.000000}%
\pgfsetstrokecolor{textcolor}%
\pgfsetfillcolor{textcolor}%
\pgftext[x=3.794748in,y=0.402778in,,top]{\color{textcolor}\rmfamily\fontsize{10.000000}{12.000000}\selectfont \(\displaystyle {600000}\)}%
\end{pgfscope}%
\begin{pgfscope}%
\pgfsetbuttcap%
\pgfsetroundjoin%
\definecolor{currentfill}{rgb}{0.000000,0.000000,0.000000}%
\pgfsetfillcolor{currentfill}%
\pgfsetlinewidth{0.803000pt}%
\definecolor{currentstroke}{rgb}{0.000000,0.000000,0.000000}%
\pgfsetstrokecolor{currentstroke}%
\pgfsetdash{}{0pt}%
\pgfsys@defobject{currentmarker}{\pgfqpoint{0.000000in}{-0.048611in}}{\pgfqpoint{0.000000in}{0.000000in}}{%
\pgfpathmoveto{\pgfqpoint{0.000000in}{0.000000in}}%
\pgfpathlineto{\pgfqpoint{0.000000in}{-0.048611in}}%
\pgfusepath{stroke,fill}%
}%
\begin{pgfscope}%
\pgfsys@transformshift{4.739210in}{0.500000in}%
\pgfsys@useobject{currentmarker}{}%
\end{pgfscope}%
\end{pgfscope}%
\begin{pgfscope}%
\definecolor{textcolor}{rgb}{0.000000,0.000000,0.000000}%
\pgfsetstrokecolor{textcolor}%
\pgfsetfillcolor{textcolor}%
\pgftext[x=4.739210in,y=0.402778in,,top]{\color{textcolor}\rmfamily\fontsize{10.000000}{12.000000}\selectfont \(\displaystyle {800000}\)}%
\end{pgfscope}%
\begin{pgfscope}%
\definecolor{textcolor}{rgb}{0.000000,0.000000,0.000000}%
\pgfsetstrokecolor{textcolor}%
\pgfsetfillcolor{textcolor}%
\pgftext[x=3.075000in,y=0.223766in,,top]{\color{textcolor}\rmfamily\fontsize{10.000000}{12.000000}\selectfont time}%
\end{pgfscope}%
\begin{pgfscope}%
\pgfsetbuttcap%
\pgfsetroundjoin%
\definecolor{currentfill}{rgb}{0.000000,0.000000,0.000000}%
\pgfsetfillcolor{currentfill}%
\pgfsetlinewidth{0.803000pt}%
\definecolor{currentstroke}{rgb}{0.000000,0.000000,0.000000}%
\pgfsetstrokecolor{currentstroke}%
\pgfsetdash{}{0pt}%
\pgfsys@defobject{currentmarker}{\pgfqpoint{-0.048611in}{0.000000in}}{\pgfqpoint{-0.000000in}{0.000000in}}{%
\pgfpathmoveto{\pgfqpoint{-0.000000in}{0.000000in}}%
\pgfpathlineto{\pgfqpoint{-0.048611in}{0.000000in}}%
\pgfusepath{stroke,fill}%
}%
\begin{pgfscope}%
\pgfsys@transformshift{0.750000in}{0.637273in}%
\pgfsys@useobject{currentmarker}{}%
\end{pgfscope}%
\end{pgfscope}%
\begin{pgfscope}%
\definecolor{textcolor}{rgb}{0.000000,0.000000,0.000000}%
\pgfsetstrokecolor{textcolor}%
\pgfsetfillcolor{textcolor}%
\pgftext[x=0.475308in, y=0.589047in, left, base]{\color{textcolor}\rmfamily\fontsize{10.000000}{12.000000}\selectfont \(\displaystyle {0.0}\)}%
\end{pgfscope}%
\begin{pgfscope}%
\pgfsetbuttcap%
\pgfsetroundjoin%
\definecolor{currentfill}{rgb}{0.000000,0.000000,0.000000}%
\pgfsetfillcolor{currentfill}%
\pgfsetlinewidth{0.803000pt}%
\definecolor{currentstroke}{rgb}{0.000000,0.000000,0.000000}%
\pgfsetstrokecolor{currentstroke}%
\pgfsetdash{}{0pt}%
\pgfsys@defobject{currentmarker}{\pgfqpoint{-0.048611in}{0.000000in}}{\pgfqpoint{-0.000000in}{0.000000in}}{%
\pgfpathmoveto{\pgfqpoint{-0.000000in}{0.000000in}}%
\pgfpathlineto{\pgfqpoint{-0.048611in}{0.000000in}}%
\pgfusepath{stroke,fill}%
}%
\begin{pgfscope}%
\pgfsys@transformshift{0.750000in}{1.186364in}%
\pgfsys@useobject{currentmarker}{}%
\end{pgfscope}%
\end{pgfscope}%
\begin{pgfscope}%
\definecolor{textcolor}{rgb}{0.000000,0.000000,0.000000}%
\pgfsetstrokecolor{textcolor}%
\pgfsetfillcolor{textcolor}%
\pgftext[x=0.475308in, y=1.138138in, left, base]{\color{textcolor}\rmfamily\fontsize{10.000000}{12.000000}\selectfont \(\displaystyle {0.2}\)}%
\end{pgfscope}%
\begin{pgfscope}%
\pgfsetbuttcap%
\pgfsetroundjoin%
\definecolor{currentfill}{rgb}{0.000000,0.000000,0.000000}%
\pgfsetfillcolor{currentfill}%
\pgfsetlinewidth{0.803000pt}%
\definecolor{currentstroke}{rgb}{0.000000,0.000000,0.000000}%
\pgfsetstrokecolor{currentstroke}%
\pgfsetdash{}{0pt}%
\pgfsys@defobject{currentmarker}{\pgfqpoint{-0.048611in}{0.000000in}}{\pgfqpoint{-0.000000in}{0.000000in}}{%
\pgfpathmoveto{\pgfqpoint{-0.000000in}{0.000000in}}%
\pgfpathlineto{\pgfqpoint{-0.048611in}{0.000000in}}%
\pgfusepath{stroke,fill}%
}%
\begin{pgfscope}%
\pgfsys@transformshift{0.750000in}{1.735455in}%
\pgfsys@useobject{currentmarker}{}%
\end{pgfscope}%
\end{pgfscope}%
\begin{pgfscope}%
\definecolor{textcolor}{rgb}{0.000000,0.000000,0.000000}%
\pgfsetstrokecolor{textcolor}%
\pgfsetfillcolor{textcolor}%
\pgftext[x=0.475308in, y=1.687229in, left, base]{\color{textcolor}\rmfamily\fontsize{10.000000}{12.000000}\selectfont \(\displaystyle {0.4}\)}%
\end{pgfscope}%
\begin{pgfscope}%
\pgfsetbuttcap%
\pgfsetroundjoin%
\definecolor{currentfill}{rgb}{0.000000,0.000000,0.000000}%
\pgfsetfillcolor{currentfill}%
\pgfsetlinewidth{0.803000pt}%
\definecolor{currentstroke}{rgb}{0.000000,0.000000,0.000000}%
\pgfsetstrokecolor{currentstroke}%
\pgfsetdash{}{0pt}%
\pgfsys@defobject{currentmarker}{\pgfqpoint{-0.048611in}{0.000000in}}{\pgfqpoint{-0.000000in}{0.000000in}}{%
\pgfpathmoveto{\pgfqpoint{-0.000000in}{0.000000in}}%
\pgfpathlineto{\pgfqpoint{-0.048611in}{0.000000in}}%
\pgfusepath{stroke,fill}%
}%
\begin{pgfscope}%
\pgfsys@transformshift{0.750000in}{2.284545in}%
\pgfsys@useobject{currentmarker}{}%
\end{pgfscope}%
\end{pgfscope}%
\begin{pgfscope}%
\definecolor{textcolor}{rgb}{0.000000,0.000000,0.000000}%
\pgfsetstrokecolor{textcolor}%
\pgfsetfillcolor{textcolor}%
\pgftext[x=0.475308in, y=2.236320in, left, base]{\color{textcolor}\rmfamily\fontsize{10.000000}{12.000000}\selectfont \(\displaystyle {0.6}\)}%
\end{pgfscope}%
\begin{pgfscope}%
\pgfsetbuttcap%
\pgfsetroundjoin%
\definecolor{currentfill}{rgb}{0.000000,0.000000,0.000000}%
\pgfsetfillcolor{currentfill}%
\pgfsetlinewidth{0.803000pt}%
\definecolor{currentstroke}{rgb}{0.000000,0.000000,0.000000}%
\pgfsetstrokecolor{currentstroke}%
\pgfsetdash{}{0pt}%
\pgfsys@defobject{currentmarker}{\pgfqpoint{-0.048611in}{0.000000in}}{\pgfqpoint{-0.000000in}{0.000000in}}{%
\pgfpathmoveto{\pgfqpoint{-0.000000in}{0.000000in}}%
\pgfpathlineto{\pgfqpoint{-0.048611in}{0.000000in}}%
\pgfusepath{stroke,fill}%
}%
\begin{pgfscope}%
\pgfsys@transformshift{0.750000in}{2.833636in}%
\pgfsys@useobject{currentmarker}{}%
\end{pgfscope}%
\end{pgfscope}%
\begin{pgfscope}%
\definecolor{textcolor}{rgb}{0.000000,0.000000,0.000000}%
\pgfsetstrokecolor{textcolor}%
\pgfsetfillcolor{textcolor}%
\pgftext[x=0.475308in, y=2.785411in, left, base]{\color{textcolor}\rmfamily\fontsize{10.000000}{12.000000}\selectfont \(\displaystyle {0.8}\)}%
\end{pgfscope}%
\begin{pgfscope}%
\pgfsetbuttcap%
\pgfsetroundjoin%
\definecolor{currentfill}{rgb}{0.000000,0.000000,0.000000}%
\pgfsetfillcolor{currentfill}%
\pgfsetlinewidth{0.803000pt}%
\definecolor{currentstroke}{rgb}{0.000000,0.000000,0.000000}%
\pgfsetstrokecolor{currentstroke}%
\pgfsetdash{}{0pt}%
\pgfsys@defobject{currentmarker}{\pgfqpoint{-0.048611in}{0.000000in}}{\pgfqpoint{-0.000000in}{0.000000in}}{%
\pgfpathmoveto{\pgfqpoint{-0.000000in}{0.000000in}}%
\pgfpathlineto{\pgfqpoint{-0.048611in}{0.000000in}}%
\pgfusepath{stroke,fill}%
}%
\begin{pgfscope}%
\pgfsys@transformshift{0.750000in}{3.382727in}%
\pgfsys@useobject{currentmarker}{}%
\end{pgfscope}%
\end{pgfscope}%
\begin{pgfscope}%
\definecolor{textcolor}{rgb}{0.000000,0.000000,0.000000}%
\pgfsetstrokecolor{textcolor}%
\pgfsetfillcolor{textcolor}%
\pgftext[x=0.475308in, y=3.334502in, left, base]{\color{textcolor}\rmfamily\fontsize{10.000000}{12.000000}\selectfont \(\displaystyle {1.0}\)}%
\end{pgfscope}%
\begin{pgfscope}%
\definecolor{textcolor}{rgb}{0.000000,0.000000,0.000000}%
\pgfsetstrokecolor{textcolor}%
\pgfsetfillcolor{textcolor}%
\pgftext[x=0.419753in,y=2.010000in,,bottom,rotate=90.000000]{\color{textcolor}\rmfamily\fontsize{10.000000}{12.000000}\selectfont evil}%
\end{pgfscope}%
\begin{pgfscope}%
\pgfpathrectangle{\pgfqpoint{0.750000in}{0.500000in}}{\pgfqpoint{4.650000in}{3.020000in}}%
\pgfusepath{clip}%
\pgfsetrectcap%
\pgfsetroundjoin%
\pgfsetlinewidth{1.505625pt}%
\definecolor{currentstroke}{rgb}{0.121569,0.466667,0.705882}%
\pgfsetstrokecolor{currentstroke}%
\pgfsetdash{}{0pt}%
\pgfpathmoveto{\pgfqpoint{0.961364in}{0.637273in}}%
\pgfpathlineto{\pgfqpoint{1.967352in}{0.637273in}}%
\pgfpathlineto{\pgfqpoint{1.968896in}{3.382727in}}%
\pgfpathlineto{\pgfqpoint{1.981387in}{3.382727in}}%
\pgfpathlineto{\pgfqpoint{1.981590in}{0.637273in}}%
\pgfpathlineto{\pgfqpoint{1.982926in}{3.382727in}}%
\pgfpathlineto{\pgfqpoint{1.985368in}{3.382727in}}%
\pgfpathlineto{\pgfqpoint{1.986237in}{0.637273in}}%
\pgfpathlineto{\pgfqpoint{1.986907in}{3.382727in}}%
\pgfpathlineto{\pgfqpoint{1.986950in}{3.382727in}}%
\pgfpathlineto{\pgfqpoint{1.988494in}{0.637273in}}%
\pgfpathlineto{\pgfqpoint{2.081419in}{0.637273in}}%
\pgfpathlineto{\pgfqpoint{2.082949in}{3.382727in}}%
\pgfpathlineto{\pgfqpoint{2.082959in}{0.637273in}}%
\pgfpathlineto{\pgfqpoint{2.083006in}{0.637273in}}%
\pgfpathlineto{\pgfqpoint{2.083190in}{3.382727in}}%
\pgfpathlineto{\pgfqpoint{2.084546in}{0.637273in}}%
\pgfpathlineto{\pgfqpoint{4.671020in}{0.637273in}}%
\pgfpathlineto{\pgfqpoint{4.671024in}{3.382727in}}%
\pgfpathlineto{\pgfqpoint{4.672559in}{0.637273in}}%
\pgfpathlineto{\pgfqpoint{4.680162in}{0.637273in}}%
\pgfpathlineto{\pgfqpoint{4.681706in}{3.382727in}}%
\pgfpathlineto{\pgfqpoint{4.690423in}{3.382727in}}%
\pgfpathlineto{\pgfqpoint{4.690900in}{0.637273in}}%
\pgfpathlineto{\pgfqpoint{4.691963in}{3.382727in}}%
\pgfpathlineto{\pgfqpoint{4.694617in}{3.382727in}}%
\pgfpathlineto{\pgfqpoint{4.694749in}{0.637273in}}%
\pgfpathlineto{\pgfqpoint{4.696156in}{3.382727in}}%
\pgfpathlineto{\pgfqpoint{4.697710in}{3.382727in}}%
\pgfpathlineto{\pgfqpoint{4.698428in}{0.637273in}}%
\pgfpathlineto{\pgfqpoint{4.699249in}{3.382727in}}%
\pgfpathlineto{\pgfqpoint{4.699774in}{3.382727in}}%
\pgfpathlineto{\pgfqpoint{4.701313in}{0.637273in}}%
\pgfpathlineto{\pgfqpoint{4.701346in}{3.382727in}}%
\pgfpathlineto{\pgfqpoint{4.702853in}{0.637273in}}%
\pgfpathlineto{\pgfqpoint{4.763308in}{0.637273in}}%
\pgfpathlineto{\pgfqpoint{4.764852in}{3.382727in}}%
\pgfpathlineto{\pgfqpoint{4.768450in}{3.382727in}}%
\pgfpathlineto{\pgfqpoint{4.769801in}{0.637273in}}%
\pgfpathlineto{\pgfqpoint{4.769990in}{3.382727in}}%
\pgfpathlineto{\pgfqpoint{4.770060in}{3.382727in}}%
\pgfpathlineto{\pgfqpoint{4.771605in}{0.637273in}}%
\pgfpathlineto{\pgfqpoint{5.188636in}{0.637273in}}%
\pgfpathlineto{\pgfqpoint{5.188636in}{0.637273in}}%
\pgfusepath{stroke}%
\end{pgfscope}%
\begin{pgfscope}%
\pgfpathrectangle{\pgfqpoint{0.750000in}{0.500000in}}{\pgfqpoint{4.650000in}{3.020000in}}%
\pgfusepath{clip}%
\pgfsetrectcap%
\pgfsetroundjoin%
\pgfsetlinewidth{1.505625pt}%
\definecolor{currentstroke}{rgb}{1.000000,0.000000,0.000000}%
\pgfsetstrokecolor{currentstroke}%
\pgfsetdash{}{0pt}%
\pgfpathmoveto{\pgfqpoint{1.013309in}{0.500000in}}%
\pgfpathlineto{\pgfqpoint{1.013309in}{3.520000in}}%
\pgfusepath{stroke}%
\end{pgfscope}%
\begin{pgfscope}%
\pgfpathrectangle{\pgfqpoint{0.750000in}{0.500000in}}{\pgfqpoint{4.650000in}{3.020000in}}%
\pgfusepath{clip}%
\pgfsetrectcap%
\pgfsetroundjoin%
\pgfsetlinewidth{1.505625pt}%
\definecolor{currentstroke}{rgb}{1.000000,0.000000,0.000000}%
\pgfsetstrokecolor{currentstroke}%
\pgfsetdash{}{0pt}%
\pgfpathmoveto{\pgfqpoint{1.967640in}{0.500000in}}%
\pgfpathlineto{\pgfqpoint{1.967640in}{3.520000in}}%
\pgfusepath{stroke}%
\end{pgfscope}%
\begin{pgfscope}%
\pgfpathrectangle{\pgfqpoint{0.750000in}{0.500000in}}{\pgfqpoint{4.650000in}{3.020000in}}%
\pgfusepath{clip}%
\pgfsetrectcap%
\pgfsetroundjoin%
\pgfsetlinewidth{1.505625pt}%
\definecolor{currentstroke}{rgb}{1.000000,0.000000,0.000000}%
\pgfsetstrokecolor{currentstroke}%
\pgfsetdash{}{0pt}%
\pgfpathmoveto{\pgfqpoint{2.081514in}{0.500000in}}%
\pgfpathlineto{\pgfqpoint{2.081514in}{3.520000in}}%
\pgfusepath{stroke}%
\end{pgfscope}%
\begin{pgfscope}%
\pgfpathrectangle{\pgfqpoint{0.750000in}{0.500000in}}{\pgfqpoint{4.650000in}{3.020000in}}%
\pgfusepath{clip}%
\pgfsetrectcap%
\pgfsetroundjoin%
\pgfsetlinewidth{1.505625pt}%
\definecolor{currentstroke}{rgb}{1.000000,0.000000,0.000000}%
\pgfsetstrokecolor{currentstroke}%
\pgfsetdash{}{0pt}%
\pgfpathmoveto{\pgfqpoint{3.253354in}{0.500000in}}%
\pgfpathlineto{\pgfqpoint{3.253354in}{3.520000in}}%
\pgfusepath{stroke}%
\end{pgfscope}%
\begin{pgfscope}%
\pgfpathrectangle{\pgfqpoint{0.750000in}{0.500000in}}{\pgfqpoint{4.650000in}{3.020000in}}%
\pgfusepath{clip}%
\pgfsetrectcap%
\pgfsetroundjoin%
\pgfsetlinewidth{1.505625pt}%
\definecolor{currentstroke}{rgb}{1.000000,0.000000,0.000000}%
\pgfsetstrokecolor{currentstroke}%
\pgfsetdash{}{0pt}%
\pgfpathmoveto{\pgfqpoint{4.680738in}{0.500000in}}%
\pgfpathlineto{\pgfqpoint{4.680738in}{3.520000in}}%
\pgfusepath{stroke}%
\end{pgfscope}%
\begin{pgfscope}%
\pgfpathrectangle{\pgfqpoint{0.750000in}{0.500000in}}{\pgfqpoint{4.650000in}{3.020000in}}%
\pgfusepath{clip}%
\pgfsetrectcap%
\pgfsetroundjoin%
\pgfsetlinewidth{1.505625pt}%
\definecolor{currentstroke}{rgb}{1.000000,0.000000,0.000000}%
\pgfsetstrokecolor{currentstroke}%
\pgfsetdash{}{0pt}%
\pgfpathmoveto{\pgfqpoint{4.817326in}{0.500000in}}%
\pgfpathlineto{\pgfqpoint{4.817326in}{3.520000in}}%
\pgfusepath{stroke}%
\end{pgfscope}%
\begin{pgfscope}%
\pgfsetrectcap%
\pgfsetmiterjoin%
\pgfsetlinewidth{0.803000pt}%
\definecolor{currentstroke}{rgb}{0.000000,0.000000,0.000000}%
\pgfsetstrokecolor{currentstroke}%
\pgfsetdash{}{0pt}%
\pgfpathmoveto{\pgfqpoint{0.750000in}{0.500000in}}%
\pgfpathlineto{\pgfqpoint{0.750000in}{3.520000in}}%
\pgfusepath{stroke}%
\end{pgfscope}%
\begin{pgfscope}%
\pgfsetrectcap%
\pgfsetmiterjoin%
\pgfsetlinewidth{0.803000pt}%
\definecolor{currentstroke}{rgb}{0.000000,0.000000,0.000000}%
\pgfsetstrokecolor{currentstroke}%
\pgfsetdash{}{0pt}%
\pgfpathmoveto{\pgfqpoint{5.400000in}{0.500000in}}%
\pgfpathlineto{\pgfqpoint{5.400000in}{3.520000in}}%
\pgfusepath{stroke}%
\end{pgfscope}%
\begin{pgfscope}%
\pgfsetrectcap%
\pgfsetmiterjoin%
\pgfsetlinewidth{0.803000pt}%
\definecolor{currentstroke}{rgb}{0.000000,0.000000,0.000000}%
\pgfsetstrokecolor{currentstroke}%
\pgfsetdash{}{0pt}%
\pgfpathmoveto{\pgfqpoint{0.750000in}{0.500000in}}%
\pgfpathlineto{\pgfqpoint{5.400000in}{0.500000in}}%
\pgfusepath{stroke}%
\end{pgfscope}%
\begin{pgfscope}%
\pgfsetrectcap%
\pgfsetmiterjoin%
\pgfsetlinewidth{0.803000pt}%
\definecolor{currentstroke}{rgb}{0.000000,0.000000,0.000000}%
\pgfsetstrokecolor{currentstroke}%
\pgfsetdash{}{0pt}%
\pgfpathmoveto{\pgfqpoint{0.750000in}{3.520000in}}%
\pgfpathlineto{\pgfqpoint{5.400000in}{3.520000in}}%
\pgfusepath{stroke}%
\end{pgfscope}%
\end{pgfpicture}%
\makeatother%
\endgroup%

    \caption{BETH Evil Outliers}
    \label{fig:beth_evil_outliers}
\end{figure}

