\section{Results}
\label{ref_results}

The results from the the two literature reviews and the development of the anomaly detector are utilized here. The anomaly detector was created in Python with the STUMPY library implementation of the Iterative Matrix Profile detector. In this section, the datasets from Section \ref{ref_datasets} are tested using the developed matrix profile anomaly detection technique against different properties and anomaly types. The parameters for each trail are listed and the results are explain. Further analysis is performed in Section \ref{sec:discussion}.

\subsection{Hydraulic Simulation Dataset}
\label{ref_results_hydraulic_sim}
The hydraulic simulation dataset presented in Section \ref{ref_hydraulic_dataset} is used to perform this simulation. Table \ref{tab:hydraulic_sim_params} specifies the parameters used with the anomaly detector to obtain the results in this section. This is the smallest dataset tested in this study and it outlines the performance and characteristics of the anomaly detector at small window sizes. The standard deviation multiplier is the default for the detector. If the data follows a standard Gaussian distribution, the detected points would fall outside 99.7\% of the data present in the window, which would represent a significant outlier comparative to the rest of the data. The same principal applies to the rolling range multiplier. The recent range detection debounce multiplier is also standard given the small window size and ensures that there are not over-noisy or duplicate detections.

This experiment demonstrate that the developed detector is robust to changes in scale. Figure \ref{fig:hydraulic_result_fp} shows that the pressure and force signals in the system have similar shape but different magnitude. 

\begin{equation}
    \label{eqn:cylinder_force}
    F = (A_1 \cdot p_A) - \dot{x}  | A_1 \cdot p_A | (1 - \eta)
\end{equation}

This relationship in the shape of the pressure and force signals is derived from the force pressure relationship described in Equation \ref{eqn:cylinder_force}. As the pressure increases, the force exerted by the hydraulic cylinder increase proportionally. Since the boom arm and end effector in the system are fixed, the end effector moves proportionally to the change in force. Figure \ref{fig:hydraulic_result_fp} shows small differences between the force and pressure which is a result of the modeling of the hydraulic boom arm as a flexible body, which introduces additional oscillation which affects the pressure.

\begin{table}[H]
%%\centering
\caption{Hydraulic Simulation Detector Parameters}
\begin{tabular}{|l|c|l|}
    \hline
	\textbf{Parameter} & \textbf{Value} & \textbf{Description} \\ \hline
	m & 15 & Window Size \\ \hline
	ts$\_$size & 30 & Time Series Size \\ \hline
	std$\_$dev & 3 & Standard Deviation Multiplier \\ \hline
	range & 3 & Rolling Range Multiplier\\ \hline
	recent & 2 & Recent Detection Debounce\\ \hline
\end{tabular}
\label{tab:hydraulic_sim_params}
\end{table}

Figure \ref{fig:hydraulic_result_fp} shows the force and pressure with the anomaly detections from the experiment highlighted with red lines. In this case, the detection points represent motif changes in the signals. The detector was able to locate all of these inflection points without false positives. This is a 100\% accuracy rate for the detector. The amplitude and waveform of the signals presented here are identical to the force signal presented above. The primary difference is the different amplitudes. The pressure signal is two orders of magnitude larger in amplitude, which illustrates the detector is robust to scale variations.

\begin{figure}[H]
    %\centering
    %% Creator: Matplotlib, PGF backend
%%
%% To include the figure in your LaTeX document, write
%%   \input{<filename>.pgf}
%%
%% Make sure the required packages are loaded in your preamble
%%   \usepackage{pgf}
%%
%% Also ensure that all the required font packages are loaded; for instance,
%% the lmodern package is sometimes necessary when using math font.
%%   \usepackage{lmodern}
%%
%% Figures using additional raster images can only be included by \input if
%% they are in the same directory as the main LaTeX file. For loading figures
%% from other directories you can use the `import` package
%%   \usepackage{import}
%%
%% and then include the figures with
%%   \import{<path to file>}{<filename>.pgf}
%%
%% Matplotlib used the following preamble
%%
\begingroup%
\makeatletter%
\begin{pgfpicture}%
\pgfpathrectangle{\pgfpointorigin}{\pgfqpoint{6.000000in}{4.000000in}}%
\pgfusepath{use as bounding box, clip}%
\begin{pgfscope}%
\pgfsetbuttcap%
\pgfsetmiterjoin%
\pgfsetlinewidth{0.000000pt}%
\definecolor{currentstroke}{rgb}{1.000000,1.000000,1.000000}%
\pgfsetstrokecolor{currentstroke}%
\pgfsetstrokeopacity{0.000000}%
\pgfsetdash{}{0pt}%
\pgfpathmoveto{\pgfqpoint{0.000000in}{0.000000in}}%
\pgfpathlineto{\pgfqpoint{6.000000in}{0.000000in}}%
\pgfpathlineto{\pgfqpoint{6.000000in}{4.000000in}}%
\pgfpathlineto{\pgfqpoint{0.000000in}{4.000000in}}%
\pgfpathlineto{\pgfqpoint{0.000000in}{0.000000in}}%
\pgfpathclose%
\pgfusepath{}%
\end{pgfscope}%
\begin{pgfscope}%
\pgfsetbuttcap%
\pgfsetmiterjoin%
\definecolor{currentfill}{rgb}{1.000000,1.000000,1.000000}%
\pgfsetfillcolor{currentfill}%
\pgfsetlinewidth{0.000000pt}%
\definecolor{currentstroke}{rgb}{0.000000,0.000000,0.000000}%
\pgfsetstrokecolor{currentstroke}%
\pgfsetstrokeopacity{0.000000}%
\pgfsetdash{}{0pt}%
\pgfpathmoveto{\pgfqpoint{0.750000in}{0.500000in}}%
\pgfpathlineto{\pgfqpoint{5.400000in}{0.500000in}}%
\pgfpathlineto{\pgfqpoint{5.400000in}{3.520000in}}%
\pgfpathlineto{\pgfqpoint{0.750000in}{3.520000in}}%
\pgfpathlineto{\pgfqpoint{0.750000in}{0.500000in}}%
\pgfpathclose%
\pgfusepath{fill}%
\end{pgfscope}%
\begin{pgfscope}%
\pgfsetbuttcap%
\pgfsetroundjoin%
\definecolor{currentfill}{rgb}{0.000000,0.000000,0.000000}%
\pgfsetfillcolor{currentfill}%
\pgfsetlinewidth{0.803000pt}%
\definecolor{currentstroke}{rgb}{0.000000,0.000000,0.000000}%
\pgfsetstrokecolor{currentstroke}%
\pgfsetdash{}{0pt}%
\pgfsys@defobject{currentmarker}{\pgfqpoint{0.000000in}{-0.048611in}}{\pgfqpoint{0.000000in}{0.000000in}}{%
\pgfpathmoveto{\pgfqpoint{0.000000in}{0.000000in}}%
\pgfpathlineto{\pgfqpoint{0.000000in}{-0.048611in}}%
\pgfusepath{stroke,fill}%
}%
\begin{pgfscope}%
\pgfsys@transformshift{0.750000in}{0.500000in}%
\pgfsys@useobject{currentmarker}{}%
\end{pgfscope}%
\end{pgfscope}%
\begin{pgfscope}%
\definecolor{textcolor}{rgb}{0.000000,0.000000,0.000000}%
\pgfsetstrokecolor{textcolor}%
\pgfsetfillcolor{textcolor}%
\pgftext[x=0.750000in,y=0.402778in,,top]{\color{textcolor}\rmfamily\fontsize{10.000000}{12.000000}\selectfont \(\displaystyle {0.0}\)}%
\end{pgfscope}%
\begin{pgfscope}%
\pgfsetbuttcap%
\pgfsetroundjoin%
\definecolor{currentfill}{rgb}{0.000000,0.000000,0.000000}%
\pgfsetfillcolor{currentfill}%
\pgfsetlinewidth{0.803000pt}%
\definecolor{currentstroke}{rgb}{0.000000,0.000000,0.000000}%
\pgfsetstrokecolor{currentstroke}%
\pgfsetdash{}{0pt}%
\pgfsys@defobject{currentmarker}{\pgfqpoint{0.000000in}{-0.048611in}}{\pgfqpoint{0.000000in}{0.000000in}}{%
\pgfpathmoveto{\pgfqpoint{0.000000in}{0.000000in}}%
\pgfpathlineto{\pgfqpoint{0.000000in}{-0.048611in}}%
\pgfusepath{stroke,fill}%
}%
\begin{pgfscope}%
\pgfsys@transformshift{1.340476in}{0.500000in}%
\pgfsys@useobject{currentmarker}{}%
\end{pgfscope}%
\end{pgfscope}%
\begin{pgfscope}%
\definecolor{textcolor}{rgb}{0.000000,0.000000,0.000000}%
\pgfsetstrokecolor{textcolor}%
\pgfsetfillcolor{textcolor}%
\pgftext[x=1.340476in,y=0.402778in,,top]{\color{textcolor}\rmfamily\fontsize{10.000000}{12.000000}\selectfont \(\displaystyle {0.2}\)}%
\end{pgfscope}%
\begin{pgfscope}%
\pgfsetbuttcap%
\pgfsetroundjoin%
\definecolor{currentfill}{rgb}{0.000000,0.000000,0.000000}%
\pgfsetfillcolor{currentfill}%
\pgfsetlinewidth{0.803000pt}%
\definecolor{currentstroke}{rgb}{0.000000,0.000000,0.000000}%
\pgfsetstrokecolor{currentstroke}%
\pgfsetdash{}{0pt}%
\pgfsys@defobject{currentmarker}{\pgfqpoint{0.000000in}{-0.048611in}}{\pgfqpoint{0.000000in}{0.000000in}}{%
\pgfpathmoveto{\pgfqpoint{0.000000in}{0.000000in}}%
\pgfpathlineto{\pgfqpoint{0.000000in}{-0.048611in}}%
\pgfusepath{stroke,fill}%
}%
\begin{pgfscope}%
\pgfsys@transformshift{1.930952in}{0.500000in}%
\pgfsys@useobject{currentmarker}{}%
\end{pgfscope}%
\end{pgfscope}%
\begin{pgfscope}%
\definecolor{textcolor}{rgb}{0.000000,0.000000,0.000000}%
\pgfsetstrokecolor{textcolor}%
\pgfsetfillcolor{textcolor}%
\pgftext[x=1.930952in,y=0.402778in,,top]{\color{textcolor}\rmfamily\fontsize{10.000000}{12.000000}\selectfont \(\displaystyle {0.4}\)}%
\end{pgfscope}%
\begin{pgfscope}%
\pgfsetbuttcap%
\pgfsetroundjoin%
\definecolor{currentfill}{rgb}{0.000000,0.000000,0.000000}%
\pgfsetfillcolor{currentfill}%
\pgfsetlinewidth{0.803000pt}%
\definecolor{currentstroke}{rgb}{0.000000,0.000000,0.000000}%
\pgfsetstrokecolor{currentstroke}%
\pgfsetdash{}{0pt}%
\pgfsys@defobject{currentmarker}{\pgfqpoint{0.000000in}{-0.048611in}}{\pgfqpoint{0.000000in}{0.000000in}}{%
\pgfpathmoveto{\pgfqpoint{0.000000in}{0.000000in}}%
\pgfpathlineto{\pgfqpoint{0.000000in}{-0.048611in}}%
\pgfusepath{stroke,fill}%
}%
\begin{pgfscope}%
\pgfsys@transformshift{2.521429in}{0.500000in}%
\pgfsys@useobject{currentmarker}{}%
\end{pgfscope}%
\end{pgfscope}%
\begin{pgfscope}%
\definecolor{textcolor}{rgb}{0.000000,0.000000,0.000000}%
\pgfsetstrokecolor{textcolor}%
\pgfsetfillcolor{textcolor}%
\pgftext[x=2.521429in,y=0.402778in,,top]{\color{textcolor}\rmfamily\fontsize{10.000000}{12.000000}\selectfont \(\displaystyle {0.6}\)}%
\end{pgfscope}%
\begin{pgfscope}%
\pgfsetbuttcap%
\pgfsetroundjoin%
\definecolor{currentfill}{rgb}{0.000000,0.000000,0.000000}%
\pgfsetfillcolor{currentfill}%
\pgfsetlinewidth{0.803000pt}%
\definecolor{currentstroke}{rgb}{0.000000,0.000000,0.000000}%
\pgfsetstrokecolor{currentstroke}%
\pgfsetdash{}{0pt}%
\pgfsys@defobject{currentmarker}{\pgfqpoint{0.000000in}{-0.048611in}}{\pgfqpoint{0.000000in}{0.000000in}}{%
\pgfpathmoveto{\pgfqpoint{0.000000in}{0.000000in}}%
\pgfpathlineto{\pgfqpoint{0.000000in}{-0.048611in}}%
\pgfusepath{stroke,fill}%
}%
\begin{pgfscope}%
\pgfsys@transformshift{3.111905in}{0.500000in}%
\pgfsys@useobject{currentmarker}{}%
\end{pgfscope}%
\end{pgfscope}%
\begin{pgfscope}%
\definecolor{textcolor}{rgb}{0.000000,0.000000,0.000000}%
\pgfsetstrokecolor{textcolor}%
\pgfsetfillcolor{textcolor}%
\pgftext[x=3.111905in,y=0.402778in,,top]{\color{textcolor}\rmfamily\fontsize{10.000000}{12.000000}\selectfont \(\displaystyle {0.8}\)}%
\end{pgfscope}%
\begin{pgfscope}%
\pgfsetbuttcap%
\pgfsetroundjoin%
\definecolor{currentfill}{rgb}{0.000000,0.000000,0.000000}%
\pgfsetfillcolor{currentfill}%
\pgfsetlinewidth{0.803000pt}%
\definecolor{currentstroke}{rgb}{0.000000,0.000000,0.000000}%
\pgfsetstrokecolor{currentstroke}%
\pgfsetdash{}{0pt}%
\pgfsys@defobject{currentmarker}{\pgfqpoint{0.000000in}{-0.048611in}}{\pgfqpoint{0.000000in}{0.000000in}}{%
\pgfpathmoveto{\pgfqpoint{0.000000in}{0.000000in}}%
\pgfpathlineto{\pgfqpoint{0.000000in}{-0.048611in}}%
\pgfusepath{stroke,fill}%
}%
\begin{pgfscope}%
\pgfsys@transformshift{3.702381in}{0.500000in}%
\pgfsys@useobject{currentmarker}{}%
\end{pgfscope}%
\end{pgfscope}%
\begin{pgfscope}%
\definecolor{textcolor}{rgb}{0.000000,0.000000,0.000000}%
\pgfsetstrokecolor{textcolor}%
\pgfsetfillcolor{textcolor}%
\pgftext[x=3.702381in,y=0.402778in,,top]{\color{textcolor}\rmfamily\fontsize{10.000000}{12.000000}\selectfont \(\displaystyle {1.0}\)}%
\end{pgfscope}%
\begin{pgfscope}%
\pgfsetbuttcap%
\pgfsetroundjoin%
\definecolor{currentfill}{rgb}{0.000000,0.000000,0.000000}%
\pgfsetfillcolor{currentfill}%
\pgfsetlinewidth{0.803000pt}%
\definecolor{currentstroke}{rgb}{0.000000,0.000000,0.000000}%
\pgfsetstrokecolor{currentstroke}%
\pgfsetdash{}{0pt}%
\pgfsys@defobject{currentmarker}{\pgfqpoint{0.000000in}{-0.048611in}}{\pgfqpoint{0.000000in}{0.000000in}}{%
\pgfpathmoveto{\pgfqpoint{0.000000in}{0.000000in}}%
\pgfpathlineto{\pgfqpoint{0.000000in}{-0.048611in}}%
\pgfusepath{stroke,fill}%
}%
\begin{pgfscope}%
\pgfsys@transformshift{4.292857in}{0.500000in}%
\pgfsys@useobject{currentmarker}{}%
\end{pgfscope}%
\end{pgfscope}%
\begin{pgfscope}%
\definecolor{textcolor}{rgb}{0.000000,0.000000,0.000000}%
\pgfsetstrokecolor{textcolor}%
\pgfsetfillcolor{textcolor}%
\pgftext[x=4.292857in,y=0.402778in,,top]{\color{textcolor}\rmfamily\fontsize{10.000000}{12.000000}\selectfont \(\displaystyle {1.2}\)}%
\end{pgfscope}%
\begin{pgfscope}%
\pgfsetbuttcap%
\pgfsetroundjoin%
\definecolor{currentfill}{rgb}{0.000000,0.000000,0.000000}%
\pgfsetfillcolor{currentfill}%
\pgfsetlinewidth{0.803000pt}%
\definecolor{currentstroke}{rgb}{0.000000,0.000000,0.000000}%
\pgfsetstrokecolor{currentstroke}%
\pgfsetdash{}{0pt}%
\pgfsys@defobject{currentmarker}{\pgfqpoint{0.000000in}{-0.048611in}}{\pgfqpoint{0.000000in}{0.000000in}}{%
\pgfpathmoveto{\pgfqpoint{0.000000in}{0.000000in}}%
\pgfpathlineto{\pgfqpoint{0.000000in}{-0.048611in}}%
\pgfusepath{stroke,fill}%
}%
\begin{pgfscope}%
\pgfsys@transformshift{4.883333in}{0.500000in}%
\pgfsys@useobject{currentmarker}{}%
\end{pgfscope}%
\end{pgfscope}%
\begin{pgfscope}%
\definecolor{textcolor}{rgb}{0.000000,0.000000,0.000000}%
\pgfsetstrokecolor{textcolor}%
\pgfsetfillcolor{textcolor}%
\pgftext[x=4.883333in,y=0.402778in,,top]{\color{textcolor}\rmfamily\fontsize{10.000000}{12.000000}\selectfont \(\displaystyle {1.4}\)}%
\end{pgfscope}%
\begin{pgfscope}%
\definecolor{textcolor}{rgb}{0.000000,0.000000,0.000000}%
\pgfsetstrokecolor{textcolor}%
\pgfsetfillcolor{textcolor}%
\pgftext[x=3.075000in,y=0.223766in,,top]{\color{textcolor}\rmfamily\fontsize{10.000000}{12.000000}\selectfont Time (s)}%
\end{pgfscope}%
\begin{pgfscope}%
\pgfsetbuttcap%
\pgfsetroundjoin%
\definecolor{currentfill}{rgb}{0.000000,0.000000,0.000000}%
\pgfsetfillcolor{currentfill}%
\pgfsetlinewidth{0.803000pt}%
\definecolor{currentstroke}{rgb}{0.000000,0.000000,0.000000}%
\pgfsetstrokecolor{currentstroke}%
\pgfsetdash{}{0pt}%
\pgfsys@defobject{currentmarker}{\pgfqpoint{-0.048611in}{0.000000in}}{\pgfqpoint{-0.000000in}{0.000000in}}{%
\pgfpathmoveto{\pgfqpoint{-0.000000in}{0.000000in}}%
\pgfpathlineto{\pgfqpoint{-0.048611in}{0.000000in}}%
\pgfusepath{stroke,fill}%
}%
\begin{pgfscope}%
\pgfsys@transformshift{0.750000in}{0.757003in}%
\pgfsys@useobject{currentmarker}{}%
\end{pgfscope}%
\end{pgfscope}%
\begin{pgfscope}%
\definecolor{textcolor}{rgb}{0.000000,0.000000,0.000000}%
\pgfsetstrokecolor{textcolor}%
\pgfsetfillcolor{textcolor}%
\pgftext[x=0.374999in, y=0.708777in, left, base]{\color{textcolor}\rmfamily\fontsize{10.000000}{12.000000}\selectfont \(\displaystyle {4000}\)}%
\end{pgfscope}%
\begin{pgfscope}%
\pgfsetbuttcap%
\pgfsetroundjoin%
\definecolor{currentfill}{rgb}{0.000000,0.000000,0.000000}%
\pgfsetfillcolor{currentfill}%
\pgfsetlinewidth{0.803000pt}%
\definecolor{currentstroke}{rgb}{0.000000,0.000000,0.000000}%
\pgfsetstrokecolor{currentstroke}%
\pgfsetdash{}{0pt}%
\pgfsys@defobject{currentmarker}{\pgfqpoint{-0.048611in}{0.000000in}}{\pgfqpoint{-0.000000in}{0.000000in}}{%
\pgfpathmoveto{\pgfqpoint{-0.000000in}{0.000000in}}%
\pgfpathlineto{\pgfqpoint{-0.048611in}{0.000000in}}%
\pgfusepath{stroke,fill}%
}%
\begin{pgfscope}%
\pgfsys@transformshift{0.750000in}{1.125213in}%
\pgfsys@useobject{currentmarker}{}%
\end{pgfscope}%
\end{pgfscope}%
\begin{pgfscope}%
\definecolor{textcolor}{rgb}{0.000000,0.000000,0.000000}%
\pgfsetstrokecolor{textcolor}%
\pgfsetfillcolor{textcolor}%
\pgftext[x=0.374999in, y=1.076988in, left, base]{\color{textcolor}\rmfamily\fontsize{10.000000}{12.000000}\selectfont \(\displaystyle {6000}\)}%
\end{pgfscope}%
\begin{pgfscope}%
\pgfsetbuttcap%
\pgfsetroundjoin%
\definecolor{currentfill}{rgb}{0.000000,0.000000,0.000000}%
\pgfsetfillcolor{currentfill}%
\pgfsetlinewidth{0.803000pt}%
\definecolor{currentstroke}{rgb}{0.000000,0.000000,0.000000}%
\pgfsetstrokecolor{currentstroke}%
\pgfsetdash{}{0pt}%
\pgfsys@defobject{currentmarker}{\pgfqpoint{-0.048611in}{0.000000in}}{\pgfqpoint{-0.000000in}{0.000000in}}{%
\pgfpathmoveto{\pgfqpoint{-0.000000in}{0.000000in}}%
\pgfpathlineto{\pgfqpoint{-0.048611in}{0.000000in}}%
\pgfusepath{stroke,fill}%
}%
\begin{pgfscope}%
\pgfsys@transformshift{0.750000in}{1.493423in}%
\pgfsys@useobject{currentmarker}{}%
\end{pgfscope}%
\end{pgfscope}%
\begin{pgfscope}%
\definecolor{textcolor}{rgb}{0.000000,0.000000,0.000000}%
\pgfsetstrokecolor{textcolor}%
\pgfsetfillcolor{textcolor}%
\pgftext[x=0.374999in, y=1.445198in, left, base]{\color{textcolor}\rmfamily\fontsize{10.000000}{12.000000}\selectfont \(\displaystyle {8000}\)}%
\end{pgfscope}%
\begin{pgfscope}%
\pgfsetbuttcap%
\pgfsetroundjoin%
\definecolor{currentfill}{rgb}{0.000000,0.000000,0.000000}%
\pgfsetfillcolor{currentfill}%
\pgfsetlinewidth{0.803000pt}%
\definecolor{currentstroke}{rgb}{0.000000,0.000000,0.000000}%
\pgfsetstrokecolor{currentstroke}%
\pgfsetdash{}{0pt}%
\pgfsys@defobject{currentmarker}{\pgfqpoint{-0.048611in}{0.000000in}}{\pgfqpoint{-0.000000in}{0.000000in}}{%
\pgfpathmoveto{\pgfqpoint{-0.000000in}{0.000000in}}%
\pgfpathlineto{\pgfqpoint{-0.048611in}{0.000000in}}%
\pgfusepath{stroke,fill}%
}%
\begin{pgfscope}%
\pgfsys@transformshift{0.750000in}{1.861634in}%
\pgfsys@useobject{currentmarker}{}%
\end{pgfscope}%
\end{pgfscope}%
\begin{pgfscope}%
\definecolor{textcolor}{rgb}{0.000000,0.000000,0.000000}%
\pgfsetstrokecolor{textcolor}%
\pgfsetfillcolor{textcolor}%
\pgftext[x=0.305554in, y=1.813408in, left, base]{\color{textcolor}\rmfamily\fontsize{10.000000}{12.000000}\selectfont \(\displaystyle {10000}\)}%
\end{pgfscope}%
\begin{pgfscope}%
\pgfsetbuttcap%
\pgfsetroundjoin%
\definecolor{currentfill}{rgb}{0.000000,0.000000,0.000000}%
\pgfsetfillcolor{currentfill}%
\pgfsetlinewidth{0.803000pt}%
\definecolor{currentstroke}{rgb}{0.000000,0.000000,0.000000}%
\pgfsetstrokecolor{currentstroke}%
\pgfsetdash{}{0pt}%
\pgfsys@defobject{currentmarker}{\pgfqpoint{-0.048611in}{0.000000in}}{\pgfqpoint{-0.000000in}{0.000000in}}{%
\pgfpathmoveto{\pgfqpoint{-0.000000in}{0.000000in}}%
\pgfpathlineto{\pgfqpoint{-0.048611in}{0.000000in}}%
\pgfusepath{stroke,fill}%
}%
\begin{pgfscope}%
\pgfsys@transformshift{0.750000in}{2.229844in}%
\pgfsys@useobject{currentmarker}{}%
\end{pgfscope}%
\end{pgfscope}%
\begin{pgfscope}%
\definecolor{textcolor}{rgb}{0.000000,0.000000,0.000000}%
\pgfsetstrokecolor{textcolor}%
\pgfsetfillcolor{textcolor}%
\pgftext[x=0.305554in, y=2.181619in, left, base]{\color{textcolor}\rmfamily\fontsize{10.000000}{12.000000}\selectfont \(\displaystyle {12000}\)}%
\end{pgfscope}%
\begin{pgfscope}%
\pgfsetbuttcap%
\pgfsetroundjoin%
\definecolor{currentfill}{rgb}{0.000000,0.000000,0.000000}%
\pgfsetfillcolor{currentfill}%
\pgfsetlinewidth{0.803000pt}%
\definecolor{currentstroke}{rgb}{0.000000,0.000000,0.000000}%
\pgfsetstrokecolor{currentstroke}%
\pgfsetdash{}{0pt}%
\pgfsys@defobject{currentmarker}{\pgfqpoint{-0.048611in}{0.000000in}}{\pgfqpoint{-0.000000in}{0.000000in}}{%
\pgfpathmoveto{\pgfqpoint{-0.000000in}{0.000000in}}%
\pgfpathlineto{\pgfqpoint{-0.048611in}{0.000000in}}%
\pgfusepath{stroke,fill}%
}%
\begin{pgfscope}%
\pgfsys@transformshift{0.750000in}{2.598054in}%
\pgfsys@useobject{currentmarker}{}%
\end{pgfscope}%
\end{pgfscope}%
\begin{pgfscope}%
\definecolor{textcolor}{rgb}{0.000000,0.000000,0.000000}%
\pgfsetstrokecolor{textcolor}%
\pgfsetfillcolor{textcolor}%
\pgftext[x=0.305554in, y=2.549829in, left, base]{\color{textcolor}\rmfamily\fontsize{10.000000}{12.000000}\selectfont \(\displaystyle {14000}\)}%
\end{pgfscope}%
\begin{pgfscope}%
\pgfsetbuttcap%
\pgfsetroundjoin%
\definecolor{currentfill}{rgb}{0.000000,0.000000,0.000000}%
\pgfsetfillcolor{currentfill}%
\pgfsetlinewidth{0.803000pt}%
\definecolor{currentstroke}{rgb}{0.000000,0.000000,0.000000}%
\pgfsetstrokecolor{currentstroke}%
\pgfsetdash{}{0pt}%
\pgfsys@defobject{currentmarker}{\pgfqpoint{-0.048611in}{0.000000in}}{\pgfqpoint{-0.000000in}{0.000000in}}{%
\pgfpathmoveto{\pgfqpoint{-0.000000in}{0.000000in}}%
\pgfpathlineto{\pgfqpoint{-0.048611in}{0.000000in}}%
\pgfusepath{stroke,fill}%
}%
\begin{pgfscope}%
\pgfsys@transformshift{0.750000in}{2.966265in}%
\pgfsys@useobject{currentmarker}{}%
\end{pgfscope}%
\end{pgfscope}%
\begin{pgfscope}%
\definecolor{textcolor}{rgb}{0.000000,0.000000,0.000000}%
\pgfsetstrokecolor{textcolor}%
\pgfsetfillcolor{textcolor}%
\pgftext[x=0.305554in, y=2.918040in, left, base]{\color{textcolor}\rmfamily\fontsize{10.000000}{12.000000}\selectfont \(\displaystyle {16000}\)}%
\end{pgfscope}%
\begin{pgfscope}%
\pgfsetbuttcap%
\pgfsetroundjoin%
\definecolor{currentfill}{rgb}{0.000000,0.000000,0.000000}%
\pgfsetfillcolor{currentfill}%
\pgfsetlinewidth{0.803000pt}%
\definecolor{currentstroke}{rgb}{0.000000,0.000000,0.000000}%
\pgfsetstrokecolor{currentstroke}%
\pgfsetdash{}{0pt}%
\pgfsys@defobject{currentmarker}{\pgfqpoint{-0.048611in}{0.000000in}}{\pgfqpoint{-0.000000in}{0.000000in}}{%
\pgfpathmoveto{\pgfqpoint{-0.000000in}{0.000000in}}%
\pgfpathlineto{\pgfqpoint{-0.048611in}{0.000000in}}%
\pgfusepath{stroke,fill}%
}%
\begin{pgfscope}%
\pgfsys@transformshift{0.750000in}{3.334475in}%
\pgfsys@useobject{currentmarker}{}%
\end{pgfscope}%
\end{pgfscope}%
\begin{pgfscope}%
\definecolor{textcolor}{rgb}{0.000000,0.000000,0.000000}%
\pgfsetstrokecolor{textcolor}%
\pgfsetfillcolor{textcolor}%
\pgftext[x=0.305554in, y=3.286250in, left, base]{\color{textcolor}\rmfamily\fontsize{10.000000}{12.000000}\selectfont \(\displaystyle {18000}\)}%
\end{pgfscope}%
\begin{pgfscope}%
\definecolor{textcolor}{rgb}{0.000000,0.000000,0.000000}%
\pgfsetstrokecolor{textcolor}%
\pgfsetfillcolor{textcolor}%
\pgftext[x=0.249999in,y=2.010000in,,bottom,rotate=90.000000]{\color{textcolor}\rmfamily\fontsize{10.000000}{12.000000}\selectfont Force (\(\displaystyle N\))}%
\end{pgfscope}%
\begin{pgfscope}%
\pgfpathrectangle{\pgfqpoint{0.750000in}{0.500000in}}{\pgfqpoint{4.650000in}{3.020000in}}%
\pgfusepath{clip}%
\pgfsetrectcap%
\pgfsetroundjoin%
\pgfsetlinewidth{1.505625pt}%
\definecolor{currentstroke}{rgb}{1.000000,0.000000,0.000000}%
\pgfsetstrokecolor{currentstroke}%
\pgfsetdash{}{0pt}%
\pgfpathmoveto{\pgfqpoint{1.351542in}{0.500000in}}%
\pgfpathlineto{\pgfqpoint{1.351542in}{3.520000in}}%
\pgfusepath{stroke}%
\end{pgfscope}%
\begin{pgfscope}%
\pgfpathrectangle{\pgfqpoint{0.750000in}{0.500000in}}{\pgfqpoint{4.650000in}{3.020000in}}%
\pgfusepath{clip}%
\pgfsetrectcap%
\pgfsetroundjoin%
\pgfsetlinewidth{1.505625pt}%
\definecolor{currentstroke}{rgb}{1.000000,0.000000,0.000000}%
\pgfsetstrokecolor{currentstroke}%
\pgfsetdash{}{0pt}%
\pgfpathmoveto{\pgfqpoint{1.849521in}{0.500000in}}%
\pgfpathlineto{\pgfqpoint{1.849521in}{3.520000in}}%
\pgfusepath{stroke}%
\end{pgfscope}%
\begin{pgfscope}%
\pgfpathrectangle{\pgfqpoint{0.750000in}{0.500000in}}{\pgfqpoint{4.650000in}{3.020000in}}%
\pgfusepath{clip}%
\pgfsetrectcap%
\pgfsetroundjoin%
\pgfsetlinewidth{1.505625pt}%
\definecolor{currentstroke}{rgb}{1.000000,0.000000,0.000000}%
\pgfsetstrokecolor{currentstroke}%
\pgfsetdash{}{0pt}%
\pgfpathmoveto{\pgfqpoint{2.585536in}{0.500000in}}%
\pgfpathlineto{\pgfqpoint{2.585536in}{3.520000in}}%
\pgfusepath{stroke}%
\end{pgfscope}%
\begin{pgfscope}%
\pgfpathrectangle{\pgfqpoint{0.750000in}{0.500000in}}{\pgfqpoint{4.650000in}{3.020000in}}%
\pgfusepath{clip}%
\pgfsetrectcap%
\pgfsetroundjoin%
\pgfsetlinewidth{1.505625pt}%
\definecolor{currentstroke}{rgb}{1.000000,0.000000,0.000000}%
\pgfsetstrokecolor{currentstroke}%
\pgfsetdash{}{0pt}%
\pgfpathmoveto{\pgfqpoint{3.262566in}{0.500000in}}%
\pgfpathlineto{\pgfqpoint{3.262566in}{3.520000in}}%
\pgfusepath{stroke}%
\end{pgfscope}%
\begin{pgfscope}%
\pgfpathrectangle{\pgfqpoint{0.750000in}{0.500000in}}{\pgfqpoint{4.650000in}{3.020000in}}%
\pgfusepath{clip}%
\pgfsetrectcap%
\pgfsetroundjoin%
\pgfsetlinewidth{1.505625pt}%
\definecolor{currentstroke}{rgb}{1.000000,0.000000,0.000000}%
\pgfsetstrokecolor{currentstroke}%
\pgfsetdash{}{0pt}%
\pgfpathmoveto{\pgfqpoint{4.140887in}{0.500000in}}%
\pgfpathlineto{\pgfqpoint{4.140887in}{3.520000in}}%
\pgfusepath{stroke}%
\end{pgfscope}%
\begin{pgfscope}%
\pgfpathrectangle{\pgfqpoint{0.750000in}{0.500000in}}{\pgfqpoint{4.650000in}{3.020000in}}%
\pgfusepath{clip}%
\pgfsetbuttcap%
\pgfsetroundjoin%
\pgfsetlinewidth{1.505625pt}%
\definecolor{currentstroke}{rgb}{0.000000,0.000000,1.000000}%
\pgfsetstrokecolor{currentstroke}%
\pgfsetdash{{9.600000pt}{2.400000pt}{1.500000pt}{2.400000pt}}{0.000000pt}%
\pgfpathmoveto{\pgfqpoint{0.750000in}{2.125526in}}%
\pgfpathlineto{\pgfqpoint{1.343944in}{2.126990in}}%
\pgfpathlineto{\pgfqpoint{1.347477in}{2.132024in}}%
\pgfpathlineto{\pgfqpoint{1.353575in}{2.149064in}}%
\pgfpathlineto{\pgfqpoint{1.359672in}{2.176390in}}%
\pgfpathlineto{\pgfqpoint{1.367802in}{2.227946in}}%
\pgfpathlineto{\pgfqpoint{1.378868in}{2.323034in}}%
\pgfpathlineto{\pgfqpoint{1.390957in}{2.453621in}}%
\pgfpathlineto{\pgfqpoint{1.407711in}{2.664126in}}%
\pgfpathlineto{\pgfqpoint{1.433440in}{2.996272in}}%
\pgfpathlineto{\pgfqpoint{1.450593in}{3.182818in}}%
\pgfpathlineto{\pgfqpoint{1.459170in}{3.256695in}}%
\pgfpathlineto{\pgfqpoint{1.467746in}{3.314796in}}%
\pgfpathlineto{\pgfqpoint{1.478278in}{3.362549in}}%
\pgfpathlineto{\pgfqpoint{1.488095in}{3.382710in}}%
\pgfpathlineto{\pgfqpoint{1.488095in}{3.382710in}}%
\pgfpathlineto{\pgfqpoint{1.489904in}{3.382727in}}%
\pgfpathlineto{\pgfqpoint{1.493521in}{3.379548in}}%
\pgfpathlineto{\pgfqpoint{1.497138in}{3.369959in}}%
\pgfpathlineto{\pgfqpoint{1.500755in}{3.354095in}}%
\pgfpathlineto{\pgfqpoint{1.505905in}{3.321090in}}%
\pgfpathlineto{\pgfqpoint{1.512153in}{3.265422in}}%
\pgfpathlineto{\pgfqpoint{1.520558in}{3.165830in}}%
\pgfpathlineto{\pgfqpoint{1.531360in}{3.001913in}}%
\pgfpathlineto{\pgfqpoint{1.552964in}{2.586767in}}%
\pgfpathlineto{\pgfqpoint{1.596171in}{1.689169in}}%
\pgfpathlineto{\pgfqpoint{1.606973in}{1.506941in}}%
\pgfpathlineto{\pgfqpoint{1.617774in}{1.357522in}}%
\pgfpathlineto{\pgfqpoint{1.631084in}{1.226802in}}%
\pgfpathlineto{\pgfqpoint{1.644393in}{1.160818in}}%
\pgfpathlineto{\pgfqpoint{1.657702in}{1.161993in}}%
\pgfpathlineto{\pgfqpoint{1.671011in}{1.228206in}}%
\pgfpathlineto{\pgfqpoint{1.684320in}{1.352934in}}%
\pgfpathlineto{\pgfqpoint{1.697629in}{1.525697in}}%
\pgfpathlineto{\pgfqpoint{1.724247in}{1.958686in}}%
\pgfpathlineto{\pgfqpoint{1.750866in}{2.400491in}}%
\pgfpathlineto{\pgfqpoint{1.764175in}{2.585950in}}%
\pgfpathlineto{\pgfqpoint{1.783333in}{2.780471in}}%
\pgfpathlineto{\pgfqpoint{1.783333in}{2.780471in}}%
\pgfpathlineto{\pgfqpoint{1.793901in}{2.841921in}}%
\pgfpathlineto{\pgfqpoint{1.800945in}{2.865365in}}%
\pgfpathlineto{\pgfqpoint{1.811500in}{2.872410in}}%
\pgfpathlineto{\pgfqpoint{1.824174in}{2.838129in}}%
\pgfpathlineto{\pgfqpoint{1.836847in}{2.761396in}}%
\pgfpathlineto{\pgfqpoint{1.849521in}{2.648867in}}%
\pgfpathlineto{\pgfqpoint{1.862194in}{2.508972in}}%
\pgfpathlineto{\pgfqpoint{1.912889in}{1.872972in}}%
\pgfpathlineto{\pgfqpoint{1.925562in}{1.743231in}}%
\pgfpathlineto{\pgfqpoint{1.938236in}{1.641254in}}%
\pgfpathlineto{\pgfqpoint{1.953944in}{1.561051in}}%
\pgfpathlineto{\pgfqpoint{1.969652in}{1.536337in}}%
\pgfpathlineto{\pgfqpoint{1.985360in}{1.567141in}}%
\pgfpathlineto{\pgfqpoint{2.001068in}{1.648295in}}%
\pgfpathlineto{\pgfqpoint{2.016776in}{1.770023in}}%
\pgfpathlineto{\pgfqpoint{2.048191in}{2.079763in}}%
\pgfpathlineto{\pgfqpoint{2.063899in}{2.236384in}}%
\pgfpathlineto{\pgfqpoint{2.079607in}{2.374076in}}%
\pgfpathlineto{\pgfqpoint{2.095315in}{2.480657in}}%
\pgfpathlineto{\pgfqpoint{2.111023in}{2.547624in}}%
\pgfpathlineto{\pgfqpoint{2.126731in}{2.570766in}}%
\pgfpathlineto{\pgfqpoint{2.142439in}{2.550276in}}%
\pgfpathlineto{\pgfqpoint{2.158147in}{2.490425in}}%
\pgfpathlineto{\pgfqpoint{2.173855in}{2.398866in}}%
\pgfpathlineto{\pgfqpoint{2.205270in}{2.162462in}}%
\pgfpathlineto{\pgfqpoint{2.220978in}{2.040931in}}%
\pgfpathlineto{\pgfqpoint{2.236686in}{1.932184in}}%
\pgfpathlineto{\pgfqpoint{2.252394in}{1.845624in}}%
\pgfpathlineto{\pgfqpoint{2.268102in}{1.788225in}}%
\pgfpathlineto{\pgfqpoint{2.283810in}{1.764004in}}%
\pgfpathlineto{\pgfqpoint{2.299518in}{1.773751in}}%
\pgfpathlineto{\pgfqpoint{2.315226in}{1.815048in}}%
\pgfpathlineto{\pgfqpoint{2.330934in}{1.882604in}}%
\pgfpathlineto{\pgfqpoint{2.346642in}{1.968839in}}%
\pgfpathlineto{\pgfqpoint{2.389070in}{2.222236in}}%
\pgfpathlineto{\pgfqpoint{2.399244in}{2.273442in}}%
\pgfpathlineto{\pgfqpoint{2.416777in}{2.341500in}}%
\pgfpathlineto{\pgfqpoint{2.434309in}{2.379357in}}%
\pgfpathlineto{\pgfqpoint{2.451841in}{2.384319in}}%
\pgfpathlineto{\pgfqpoint{2.469374in}{2.357686in}}%
\pgfpathlineto{\pgfqpoint{2.486906in}{2.304319in}}%
\pgfpathlineto{\pgfqpoint{2.504438in}{2.231848in}}%
\pgfpathlineto{\pgfqpoint{2.526808in}{2.122689in}}%
\pgfpathlineto{\pgfqpoint{2.536248in}{2.052587in}}%
\pgfpathlineto{\pgfqpoint{2.544046in}{1.978939in}}%
\pgfpathlineto{\pgfqpoint{2.557876in}{1.819964in}}%
\pgfpathlineto{\pgfqpoint{2.585536in}{1.434187in}}%
\pgfpathlineto{\pgfqpoint{2.613196in}{1.045449in}}%
\pgfpathlineto{\pgfqpoint{2.627026in}{0.882140in}}%
\pgfpathlineto{\pgfqpoint{2.640856in}{0.754440in}}%
\pgfpathlineto{\pgfqpoint{2.654686in}{0.670968in}}%
\pgfpathlineto{\pgfqpoint{2.670226in}{0.637284in}}%
\pgfpathlineto{\pgfqpoint{2.671617in}{0.638022in}}%
\pgfpathlineto{\pgfqpoint{2.674980in}{0.643918in}}%
\pgfpathlineto{\pgfqpoint{2.678343in}{0.655607in}}%
\pgfpathlineto{\pgfqpoint{2.681706in}{0.673024in}}%
\pgfpathlineto{\pgfqpoint{2.690547in}{0.745212in}}%
\pgfpathlineto{\pgfqpoint{2.702014in}{0.891142in}}%
\pgfpathlineto{\pgfqpoint{2.710189in}{1.025750in}}%
\pgfpathlineto{\pgfqpoint{2.732471in}{1.478115in}}%
\pgfpathlineto{\pgfqpoint{2.765894in}{2.186643in}}%
\pgfpathlineto{\pgfqpoint{2.777036in}{2.375921in}}%
\pgfpathlineto{\pgfqpoint{2.788177in}{2.524005in}}%
\pgfpathlineto{\pgfqpoint{2.799318in}{2.624409in}}%
\pgfpathlineto{\pgfqpoint{2.810459in}{2.673885in}}%
\pgfpathlineto{\pgfqpoint{2.816667in}{2.679361in}}%
\pgfpathlineto{\pgfqpoint{2.816667in}{2.679361in}}%
\pgfpathlineto{\pgfqpoint{2.819546in}{2.676142in}}%
\pgfpathlineto{\pgfqpoint{2.825305in}{2.666018in}}%
\pgfpathlineto{\pgfqpoint{2.833727in}{2.638496in}}%
\pgfpathlineto{\pgfqpoint{2.840995in}{2.603716in}}%
\pgfpathlineto{\pgfqpoint{2.848264in}{2.559885in}}%
\pgfpathlineto{\pgfqpoint{2.859596in}{2.476867in}}%
\pgfpathlineto{\pgfqpoint{2.891139in}{2.194584in}}%
\pgfpathlineto{\pgfqpoint{2.906911in}{2.052785in}}%
\pgfpathlineto{\pgfqpoint{2.922682in}{1.928876in}}%
\pgfpathlineto{\pgfqpoint{2.938454in}{1.833295in}}%
\pgfpathlineto{\pgfqpoint{2.954225in}{1.772785in}}%
\pgfpathlineto{\pgfqpoint{2.964286in}{1.753815in}}%
\pgfpathlineto{\pgfqpoint{2.964286in}{1.753815in}}%
\pgfpathlineto{\pgfqpoint{2.969006in}{1.751984in}}%
\pgfpathlineto{\pgfqpoint{2.978448in}{1.754831in}}%
\pgfpathlineto{\pgfqpoint{2.987889in}{1.770179in}}%
\pgfpathlineto{\pgfqpoint{3.004141in}{1.821625in}}%
\pgfpathlineto{\pgfqpoint{3.020393in}{1.897870in}}%
\pgfpathlineto{\pgfqpoint{3.069149in}{2.174878in}}%
\pgfpathlineto{\pgfqpoint{3.085402in}{2.250962in}}%
\pgfpathlineto{\pgfqpoint{3.101654in}{2.306539in}}%
\pgfpathlineto{\pgfqpoint{3.111905in}{2.329352in}}%
\pgfpathlineto{\pgfqpoint{3.111905in}{2.329352in}}%
\pgfpathlineto{\pgfqpoint{3.117996in}{2.336262in}}%
\pgfpathlineto{\pgfqpoint{3.130179in}{2.342949in}}%
\pgfpathlineto{\pgfqpoint{3.142361in}{2.335868in}}%
\pgfpathlineto{\pgfqpoint{3.162396in}{2.298647in}}%
\pgfpathlineto{\pgfqpoint{3.182430in}{2.237575in}}%
\pgfpathlineto{\pgfqpoint{3.222498in}{2.091396in}}%
\pgfpathlineto{\pgfqpoint{3.242532in}{2.029633in}}%
\pgfpathlineto{\pgfqpoint{3.262566in}{1.987228in}}%
\pgfpathlineto{\pgfqpoint{3.282600in}{1.968515in}}%
\pgfpathlineto{\pgfqpoint{3.302634in}{1.973789in}}%
\pgfpathlineto{\pgfqpoint{3.322668in}{1.999708in}}%
\pgfpathlineto{\pgfqpoint{3.342702in}{2.040199in}}%
\pgfpathlineto{\pgfqpoint{3.392004in}{2.152982in}}%
\pgfpathlineto{\pgfqpoint{3.416654in}{2.192488in}}%
\pgfpathlineto{\pgfqpoint{3.441305in}{2.210127in}}%
\pgfpathlineto{\pgfqpoint{3.465956in}{2.204786in}}%
\pgfpathlineto{\pgfqpoint{3.490607in}{2.180559in}}%
\pgfpathlineto{\pgfqpoint{3.545116in}{2.100981in}}%
\pgfpathlineto{\pgfqpoint{3.574975in}{2.068124in}}%
\pgfpathlineto{\pgfqpoint{3.604833in}{2.055236in}}%
\pgfpathlineto{\pgfqpoint{3.634692in}{2.063435in}}%
\pgfpathlineto{\pgfqpoint{3.664550in}{2.087115in}}%
\pgfpathlineto{\pgfqpoint{3.702381in}{2.124609in}}%
\pgfpathlineto{\pgfqpoint{3.702381in}{2.124609in}}%
\pgfpathlineto{\pgfqpoint{3.715445in}{2.134993in}}%
\pgfpathlineto{\pgfqpoint{3.741572in}{2.150998in}}%
\pgfpathlineto{\pgfqpoint{3.767699in}{2.156751in}}%
\pgfpathlineto{\pgfqpoint{3.809253in}{2.146487in}}%
\pgfpathlineto{\pgfqpoint{3.855230in}{2.121974in}}%
\pgfpathlineto{\pgfqpoint{3.861068in}{2.125180in}}%
\pgfpathlineto{\pgfqpoint{3.865423in}{2.130812in}}%
\pgfpathlineto{\pgfqpoint{3.870700in}{2.141189in}}%
\pgfpathlineto{\pgfqpoint{3.878279in}{2.162491in}}%
\pgfpathlineto{\pgfqpoint{3.889206in}{2.205136in}}%
\pgfpathlineto{\pgfqpoint{3.905161in}{2.287927in}}%
\pgfpathlineto{\pgfqpoint{3.940579in}{2.519295in}}%
\pgfpathlineto{\pgfqpoint{3.960043in}{2.645972in}}%
\pgfpathlineto{\pgfqpoint{3.979506in}{2.752553in}}%
\pgfpathlineto{\pgfqpoint{4.001145in}{2.831195in}}%
\pgfpathlineto{\pgfqpoint{4.004808in}{2.836628in}}%
\pgfpathlineto{\pgfqpoint{4.012489in}{2.835380in}}%
\pgfpathlineto{\pgfqpoint{4.025329in}{2.795995in}}%
\pgfpathlineto{\pgfqpoint{4.038169in}{2.712902in}}%
\pgfpathlineto{\pgfqpoint{4.051009in}{2.591933in}}%
\pgfpathlineto{\pgfqpoint{4.063848in}{2.441228in}}%
\pgfpathlineto{\pgfqpoint{4.115208in}{1.751801in}}%
\pgfpathlineto{\pgfqpoint{4.128048in}{1.613318in}}%
\pgfpathlineto{\pgfqpoint{4.140887in}{1.508021in}}%
\pgfpathlineto{\pgfqpoint{4.153727in}{1.442718in}}%
\pgfpathlineto{\pgfqpoint{4.166567in}{1.421697in}}%
\pgfpathlineto{\pgfqpoint{4.179407in}{1.446395in}}%
\pgfpathlineto{\pgfqpoint{4.192247in}{1.515251in}}%
\pgfpathlineto{\pgfqpoint{4.205087in}{1.623749in}}%
\pgfpathlineto{\pgfqpoint{4.217926in}{1.764674in}}%
\pgfpathlineto{\pgfqpoint{4.243606in}{2.104434in}}%
\pgfpathlineto{\pgfqpoint{4.269286in}{2.444703in}}%
\pgfpathlineto{\pgfqpoint{4.282126in}{2.586434in}}%
\pgfpathlineto{\pgfqpoint{4.294966in}{2.696350in}}%
\pgfpathlineto{\pgfqpoint{4.307805in}{2.767476in}}%
\pgfpathlineto{\pgfqpoint{4.320645in}{2.795532in}}%
\pgfpathlineto{\pgfqpoint{4.333485in}{2.779155in}}%
\pgfpathlineto{\pgfqpoint{4.346325in}{2.719905in}}%
\pgfpathlineto{\pgfqpoint{4.359165in}{2.622079in}}%
\pgfpathlineto{\pgfqpoint{4.372005in}{2.492365in}}%
\pgfpathlineto{\pgfqpoint{4.397684in}{2.173048in}}%
\pgfpathlineto{\pgfqpoint{4.423364in}{1.843347in}}%
\pgfpathlineto{\pgfqpoint{4.436204in}{1.701026in}}%
\pgfpathlineto{\pgfqpoint{4.449044in}{1.586248in}}%
\pgfpathlineto{\pgfqpoint{4.461883in}{1.506401in}}%
\pgfpathlineto{\pgfqpoint{4.474723in}{1.466672in}}%
\pgfpathlineto{\pgfqpoint{4.487563in}{1.469692in}}%
\pgfpathlineto{\pgfqpoint{4.500403in}{1.515308in}}%
\pgfpathlineto{\pgfqpoint{4.513243in}{1.600546in}}%
\pgfpathlineto{\pgfqpoint{4.526083in}{1.719761in}}%
\pgfpathlineto{\pgfqpoint{4.551762in}{2.026474in}}%
\pgfpathlineto{\pgfqpoint{4.577442in}{2.354449in}}%
\pgfpathlineto{\pgfqpoint{4.590282in}{2.499018in}}%
\pgfpathlineto{\pgfqpoint{4.603122in}{2.617558in}}%
\pgfpathlineto{\pgfqpoint{4.615961in}{2.702435in}}%
\pgfpathlineto{\pgfqpoint{4.628801in}{2.748375in}}%
\pgfpathlineto{\pgfqpoint{4.641641in}{2.752771in}}%
\pgfpathlineto{\pgfqpoint{4.654481in}{2.715782in}}%
\pgfpathlineto{\pgfqpoint{4.667321in}{2.640251in}}%
\pgfpathlineto{\pgfqpoint{4.680161in}{2.531446in}}%
\pgfpathlineto{\pgfqpoint{4.693000in}{2.396674in}}%
\pgfpathlineto{\pgfqpoint{4.739267in}{1.841944in}}%
\pgfpathlineto{\pgfqpoint{4.754689in}{1.687660in}}%
\pgfpathlineto{\pgfqpoint{4.770112in}{1.574042in}}%
\pgfpathlineto{\pgfqpoint{4.785534in}{1.511592in}}%
\pgfpathlineto{\pgfqpoint{4.800956in}{1.506150in}}%
\pgfpathlineto{\pgfqpoint{4.816378in}{1.558277in}}%
\pgfpathlineto{\pgfqpoint{4.831801in}{1.663080in}}%
\pgfpathlineto{\pgfqpoint{4.847223in}{1.810586in}}%
\pgfpathlineto{\pgfqpoint{4.908912in}{2.513247in}}%
\pgfpathlineto{\pgfqpoint{4.924334in}{2.632436in}}%
\pgfpathlineto{\pgfqpoint{4.939756in}{2.702314in}}%
\pgfpathlineto{\pgfqpoint{4.955178in}{2.716890in}}%
\pgfpathlineto{\pgfqpoint{4.970601in}{2.675478in}}%
\pgfpathlineto{\pgfqpoint{4.986023in}{2.582610in}}%
\pgfpathlineto{\pgfqpoint{5.001445in}{2.447460in}}%
\pgfpathlineto{\pgfqpoint{5.032290in}{2.104286in}}%
\pgfpathlineto{\pgfqpoint{5.047712in}{1.928093in}}%
\pgfpathlineto{\pgfqpoint{5.063134in}{1.770478in}}%
\pgfpathlineto{\pgfqpoint{5.078556in}{1.645892in}}%
\pgfpathlineto{\pgfqpoint{5.093979in}{1.565808in}}%
\pgfpathlineto{\pgfqpoint{5.109401in}{1.537659in}}%
\pgfpathlineto{\pgfqpoint{5.124823in}{1.564100in}}%
\pgfpathlineto{\pgfqpoint{5.140245in}{1.642659in}}%
\pgfpathlineto{\pgfqpoint{5.155668in}{1.765882in}}%
\pgfpathlineto{\pgfqpoint{5.178571in}{2.005687in}}%
\pgfpathlineto{\pgfqpoint{5.178571in}{2.005687in}}%
\pgfusepath{stroke}%
\end{pgfscope}%
\begin{pgfscope}%
\pgfsetrectcap%
\pgfsetmiterjoin%
\pgfsetlinewidth{0.803000pt}%
\definecolor{currentstroke}{rgb}{0.000000,0.000000,0.000000}%
\pgfsetstrokecolor{currentstroke}%
\pgfsetdash{}{0pt}%
\pgfpathmoveto{\pgfqpoint{0.750000in}{0.500000in}}%
\pgfpathlineto{\pgfqpoint{0.750000in}{3.520000in}}%
\pgfusepath{stroke}%
\end{pgfscope}%
\begin{pgfscope}%
\pgfsetrectcap%
\pgfsetmiterjoin%
\pgfsetlinewidth{0.803000pt}%
\definecolor{currentstroke}{rgb}{0.000000,0.000000,0.000000}%
\pgfsetstrokecolor{currentstroke}%
\pgfsetdash{}{0pt}%
\pgfpathmoveto{\pgfqpoint{5.400000in}{0.500000in}}%
\pgfpathlineto{\pgfqpoint{5.400000in}{3.520000in}}%
\pgfusepath{stroke}%
\end{pgfscope}%
\begin{pgfscope}%
\pgfsetrectcap%
\pgfsetmiterjoin%
\pgfsetlinewidth{0.803000pt}%
\definecolor{currentstroke}{rgb}{0.000000,0.000000,0.000000}%
\pgfsetstrokecolor{currentstroke}%
\pgfsetdash{}{0pt}%
\pgfpathmoveto{\pgfqpoint{0.750000in}{0.500000in}}%
\pgfpathlineto{\pgfqpoint{5.400000in}{0.500000in}}%
\pgfusepath{stroke}%
\end{pgfscope}%
\begin{pgfscope}%
\pgfsetrectcap%
\pgfsetmiterjoin%
\pgfsetlinewidth{0.803000pt}%
\definecolor{currentstroke}{rgb}{0.000000,0.000000,0.000000}%
\pgfsetstrokecolor{currentstroke}%
\pgfsetdash{}{0pt}%
\pgfpathmoveto{\pgfqpoint{0.750000in}{3.520000in}}%
\pgfpathlineto{\pgfqpoint{5.400000in}{3.520000in}}%
\pgfusepath{stroke}%
\end{pgfscope}%
\begin{pgfscope}%
\pgfsetbuttcap%
\pgfsetroundjoin%
\definecolor{currentfill}{rgb}{0.000000,0.000000,0.000000}%
\pgfsetfillcolor{currentfill}%
\pgfsetlinewidth{0.803000pt}%
\definecolor{currentstroke}{rgb}{0.000000,0.000000,0.000000}%
\pgfsetstrokecolor{currentstroke}%
\pgfsetdash{}{0pt}%
\pgfsys@defobject{currentmarker}{\pgfqpoint{0.000000in}{0.000000in}}{\pgfqpoint{0.048611in}{0.000000in}}{%
\pgfpathmoveto{\pgfqpoint{0.000000in}{0.000000in}}%
\pgfpathlineto{\pgfqpoint{0.048611in}{0.000000in}}%
\pgfusepath{stroke,fill}%
}%
\begin{pgfscope}%
\pgfsys@transformshift{5.400000in}{0.543195in}%
\pgfsys@useobject{currentmarker}{}%
\end{pgfscope}%
\end{pgfscope}%
\begin{pgfscope}%
\definecolor{textcolor}{rgb}{0.000000,0.000000,0.000000}%
\pgfsetstrokecolor{textcolor}%
\pgfsetfillcolor{textcolor}%
\pgftext[x=5.497222in, y=0.494970in, left, base]{\color{textcolor}\rmfamily\fontsize{10.000000}{12.000000}\selectfont \(\displaystyle {1}\)}%
\end{pgfscope}%
\begin{pgfscope}%
\pgfsetbuttcap%
\pgfsetroundjoin%
\definecolor{currentfill}{rgb}{0.000000,0.000000,0.000000}%
\pgfsetfillcolor{currentfill}%
\pgfsetlinewidth{0.803000pt}%
\definecolor{currentstroke}{rgb}{0.000000,0.000000,0.000000}%
\pgfsetstrokecolor{currentstroke}%
\pgfsetdash{}{0pt}%
\pgfsys@defobject{currentmarker}{\pgfqpoint{0.000000in}{0.000000in}}{\pgfqpoint{0.048611in}{0.000000in}}{%
\pgfpathmoveto{\pgfqpoint{0.000000in}{0.000000in}}%
\pgfpathlineto{\pgfqpoint{0.048611in}{0.000000in}}%
\pgfusepath{stroke,fill}%
}%
\begin{pgfscope}%
\pgfsys@transformshift{5.400000in}{1.052595in}%
\pgfsys@useobject{currentmarker}{}%
\end{pgfscope}%
\end{pgfscope}%
\begin{pgfscope}%
\definecolor{textcolor}{rgb}{0.000000,0.000000,0.000000}%
\pgfsetstrokecolor{textcolor}%
\pgfsetfillcolor{textcolor}%
\pgftext[x=5.497222in, y=1.004369in, left, base]{\color{textcolor}\rmfamily\fontsize{10.000000}{12.000000}\selectfont \(\displaystyle {2}\)}%
\end{pgfscope}%
\begin{pgfscope}%
\pgfsetbuttcap%
\pgfsetroundjoin%
\definecolor{currentfill}{rgb}{0.000000,0.000000,0.000000}%
\pgfsetfillcolor{currentfill}%
\pgfsetlinewidth{0.803000pt}%
\definecolor{currentstroke}{rgb}{0.000000,0.000000,0.000000}%
\pgfsetstrokecolor{currentstroke}%
\pgfsetdash{}{0pt}%
\pgfsys@defobject{currentmarker}{\pgfqpoint{0.000000in}{0.000000in}}{\pgfqpoint{0.048611in}{0.000000in}}{%
\pgfpathmoveto{\pgfqpoint{0.000000in}{0.000000in}}%
\pgfpathlineto{\pgfqpoint{0.048611in}{0.000000in}}%
\pgfusepath{stroke,fill}%
}%
\begin{pgfscope}%
\pgfsys@transformshift{5.400000in}{1.561994in}%
\pgfsys@useobject{currentmarker}{}%
\end{pgfscope}%
\end{pgfscope}%
\begin{pgfscope}%
\definecolor{textcolor}{rgb}{0.000000,0.000000,0.000000}%
\pgfsetstrokecolor{textcolor}%
\pgfsetfillcolor{textcolor}%
\pgftext[x=5.497222in, y=1.513769in, left, base]{\color{textcolor}\rmfamily\fontsize{10.000000}{12.000000}\selectfont \(\displaystyle {3}\)}%
\end{pgfscope}%
\begin{pgfscope}%
\pgfsetbuttcap%
\pgfsetroundjoin%
\definecolor{currentfill}{rgb}{0.000000,0.000000,0.000000}%
\pgfsetfillcolor{currentfill}%
\pgfsetlinewidth{0.803000pt}%
\definecolor{currentstroke}{rgb}{0.000000,0.000000,0.000000}%
\pgfsetstrokecolor{currentstroke}%
\pgfsetdash{}{0pt}%
\pgfsys@defobject{currentmarker}{\pgfqpoint{0.000000in}{0.000000in}}{\pgfqpoint{0.048611in}{0.000000in}}{%
\pgfpathmoveto{\pgfqpoint{0.000000in}{0.000000in}}%
\pgfpathlineto{\pgfqpoint{0.048611in}{0.000000in}}%
\pgfusepath{stroke,fill}%
}%
\begin{pgfscope}%
\pgfsys@transformshift{5.400000in}{2.071394in}%
\pgfsys@useobject{currentmarker}{}%
\end{pgfscope}%
\end{pgfscope}%
\begin{pgfscope}%
\definecolor{textcolor}{rgb}{0.000000,0.000000,0.000000}%
\pgfsetstrokecolor{textcolor}%
\pgfsetfillcolor{textcolor}%
\pgftext[x=5.497222in, y=2.023169in, left, base]{\color{textcolor}\rmfamily\fontsize{10.000000}{12.000000}\selectfont \(\displaystyle {4}\)}%
\end{pgfscope}%
\begin{pgfscope}%
\pgfsetbuttcap%
\pgfsetroundjoin%
\definecolor{currentfill}{rgb}{0.000000,0.000000,0.000000}%
\pgfsetfillcolor{currentfill}%
\pgfsetlinewidth{0.803000pt}%
\definecolor{currentstroke}{rgb}{0.000000,0.000000,0.000000}%
\pgfsetstrokecolor{currentstroke}%
\pgfsetdash{}{0pt}%
\pgfsys@defobject{currentmarker}{\pgfqpoint{0.000000in}{0.000000in}}{\pgfqpoint{0.048611in}{0.000000in}}{%
\pgfpathmoveto{\pgfqpoint{0.000000in}{0.000000in}}%
\pgfpathlineto{\pgfqpoint{0.048611in}{0.000000in}}%
\pgfusepath{stroke,fill}%
}%
\begin{pgfscope}%
\pgfsys@transformshift{5.400000in}{2.580794in}%
\pgfsys@useobject{currentmarker}{}%
\end{pgfscope}%
\end{pgfscope}%
\begin{pgfscope}%
\definecolor{textcolor}{rgb}{0.000000,0.000000,0.000000}%
\pgfsetstrokecolor{textcolor}%
\pgfsetfillcolor{textcolor}%
\pgftext[x=5.497222in, y=2.532569in, left, base]{\color{textcolor}\rmfamily\fontsize{10.000000}{12.000000}\selectfont \(\displaystyle {5}\)}%
\end{pgfscope}%
\begin{pgfscope}%
\pgfsetbuttcap%
\pgfsetroundjoin%
\definecolor{currentfill}{rgb}{0.000000,0.000000,0.000000}%
\pgfsetfillcolor{currentfill}%
\pgfsetlinewidth{0.803000pt}%
\definecolor{currentstroke}{rgb}{0.000000,0.000000,0.000000}%
\pgfsetstrokecolor{currentstroke}%
\pgfsetdash{}{0pt}%
\pgfsys@defobject{currentmarker}{\pgfqpoint{0.000000in}{0.000000in}}{\pgfqpoint{0.048611in}{0.000000in}}{%
\pgfpathmoveto{\pgfqpoint{0.000000in}{0.000000in}}%
\pgfpathlineto{\pgfqpoint{0.048611in}{0.000000in}}%
\pgfusepath{stroke,fill}%
}%
\begin{pgfscope}%
\pgfsys@transformshift{5.400000in}{3.090194in}%
\pgfsys@useobject{currentmarker}{}%
\end{pgfscope}%
\end{pgfscope}%
\begin{pgfscope}%
\definecolor{textcolor}{rgb}{0.000000,0.000000,0.000000}%
\pgfsetstrokecolor{textcolor}%
\pgfsetfillcolor{textcolor}%
\pgftext[x=5.497222in, y=3.041969in, left, base]{\color{textcolor}\rmfamily\fontsize{10.000000}{12.000000}\selectfont \(\displaystyle {6}\)}%
\end{pgfscope}%
\begin{pgfscope}%
\definecolor{textcolor}{rgb}{0.000000,0.000000,0.000000}%
\pgfsetstrokecolor{textcolor}%
\pgfsetfillcolor{textcolor}%
\pgftext[x=5.622223in,y=2.010000in,,top,rotate=90.000000]{\color{textcolor}\rmfamily\fontsize{10.000000}{12.000000}\selectfont Pressure (\(\displaystyle Pa\))}%
\end{pgfscope}%
\begin{pgfscope}%
\definecolor{textcolor}{rgb}{0.000000,0.000000,0.000000}%
\pgfsetstrokecolor{textcolor}%
\pgfsetfillcolor{textcolor}%
\pgftext[x=5.400000in,y=3.561667in,right,base]{\color{textcolor}\rmfamily\fontsize{10.000000}{12.000000}\selectfont \(\displaystyle \times{10^{6}}{}\)}%
\end{pgfscope}%
\begin{pgfscope}%
\pgfpathrectangle{\pgfqpoint{0.750000in}{0.500000in}}{\pgfqpoint{4.650000in}{3.020000in}}%
\pgfusepath{clip}%
\pgfsetbuttcap%
\pgfsetroundjoin%
\pgfsetlinewidth{1.505625pt}%
\definecolor{currentstroke}{rgb}{0.000000,0.500000,0.000000}%
\pgfsetstrokecolor{currentstroke}%
\pgfsetdash{{5.550000pt}{2.400000pt}}{0.000000pt}%
\pgfpathmoveto{\pgfqpoint{0.750000in}{2.093670in}}%
\pgfpathlineto{\pgfqpoint{1.343944in}{2.095103in}}%
\pgfpathlineto{\pgfqpoint{1.347477in}{2.100035in}}%
\pgfpathlineto{\pgfqpoint{1.353575in}{2.116744in}}%
\pgfpathlineto{\pgfqpoint{1.359672in}{2.143572in}}%
\pgfpathlineto{\pgfqpoint{1.367802in}{2.194274in}}%
\pgfpathlineto{\pgfqpoint{1.378868in}{2.288028in}}%
\pgfpathlineto{\pgfqpoint{1.390957in}{2.417280in}}%
\pgfpathlineto{\pgfqpoint{1.407711in}{2.626987in}}%
\pgfpathlineto{\pgfqpoint{1.442017in}{3.064543in}}%
\pgfpathlineto{\pgfqpoint{1.459170in}{3.234615in}}%
\pgfpathlineto{\pgfqpoint{1.467746in}{3.298191in}}%
\pgfpathlineto{\pgfqpoint{1.478278in}{3.353503in}}%
\pgfpathlineto{\pgfqpoint{1.489904in}{3.382727in}}%
\pgfpathlineto{\pgfqpoint{1.493521in}{3.382442in}}%
\pgfpathlineto{\pgfqpoint{1.497138in}{3.375747in}}%
\pgfpathlineto{\pgfqpoint{1.500755in}{3.362745in}}%
\pgfpathlineto{\pgfqpoint{1.505905in}{3.333699in}}%
\pgfpathlineto{\pgfqpoint{1.512153in}{3.282552in}}%
\pgfpathlineto{\pgfqpoint{1.520558in}{3.188352in}}%
\pgfpathlineto{\pgfqpoint{1.531360in}{3.029877in}}%
\pgfpathlineto{\pgfqpoint{1.552964in}{2.619905in}}%
\pgfpathlineto{\pgfqpoint{1.596171in}{1.715417in}}%
\pgfpathlineto{\pgfqpoint{1.606973in}{1.530141in}}%
\pgfpathlineto{\pgfqpoint{1.617774in}{1.377798in}}%
\pgfpathlineto{\pgfqpoint{1.631084in}{1.243829in}}%
\pgfpathlineto{\pgfqpoint{1.644393in}{1.174943in}}%
\pgfpathlineto{\pgfqpoint{1.657702in}{1.173341in}}%
\pgfpathlineto{\pgfqpoint{1.671011in}{1.236624in}}%
\pgfpathlineto{\pgfqpoint{1.684320in}{1.358075in}}%
\pgfpathlineto{\pgfqpoint{1.697629in}{1.527222in}}%
\pgfpathlineto{\pgfqpoint{1.724247in}{1.953138in}}%
\pgfpathlineto{\pgfqpoint{1.750866in}{2.391395in}}%
\pgfpathlineto{\pgfqpoint{1.764175in}{2.577492in}}%
\pgfpathlineto{\pgfqpoint{1.783333in}{2.775994in}}%
\pgfpathlineto{\pgfqpoint{1.783333in}{2.775994in}}%
\pgfpathlineto{\pgfqpoint{1.793901in}{2.840957in}}%
\pgfpathlineto{\pgfqpoint{1.800945in}{2.867058in}}%
\pgfpathlineto{\pgfqpoint{1.811500in}{2.878341in}}%
\pgfpathlineto{\pgfqpoint{1.824174in}{2.849135in}}%
\pgfpathlineto{\pgfqpoint{1.836847in}{2.776964in}}%
\pgfpathlineto{\pgfqpoint{1.849521in}{2.668068in}}%
\pgfpathlineto{\pgfqpoint{1.862194in}{2.530632in}}%
\pgfpathlineto{\pgfqpoint{1.925562in}{1.762139in}}%
\pgfpathlineto{\pgfqpoint{1.938236in}{1.658143in}}%
\pgfpathlineto{\pgfqpoint{1.953944in}{1.575359in}}%
\pgfpathlineto{\pgfqpoint{1.969652in}{1.548037in}}%
\pgfpathlineto{\pgfqpoint{1.985360in}{1.576191in}}%
\pgfpathlineto{\pgfqpoint{2.001068in}{1.654650in}}%
\pgfpathlineto{\pgfqpoint{2.016776in}{1.773737in}}%
\pgfpathlineto{\pgfqpoint{2.048191in}{2.079401in}}%
\pgfpathlineto{\pgfqpoint{2.063899in}{2.235254in}}%
\pgfpathlineto{\pgfqpoint{2.079607in}{2.373363in}}%
\pgfpathlineto{\pgfqpoint{2.095315in}{2.481577in}}%
\pgfpathlineto{\pgfqpoint{2.111023in}{2.551194in}}%
\pgfpathlineto{\pgfqpoint{2.126731in}{2.577610in}}%
\pgfpathlineto{\pgfqpoint{2.142439in}{2.560530in}}%
\pgfpathlineto{\pgfqpoint{2.158147in}{2.503739in}}%
\pgfpathlineto{\pgfqpoint{2.173855in}{2.414524in}}%
\pgfpathlineto{\pgfqpoint{2.205270in}{2.180033in}}%
\pgfpathlineto{\pgfqpoint{2.220978in}{2.058168in}}%
\pgfpathlineto{\pgfqpoint{2.236686in}{1.948461in}}%
\pgfpathlineto{\pgfqpoint{2.252394in}{1.860516in}}%
\pgfpathlineto{\pgfqpoint{2.268102in}{1.801477in}}%
\pgfpathlineto{\pgfqpoint{2.283810in}{1.775481in}}%
\pgfpathlineto{\pgfqpoint{2.299518in}{1.783399in}}%
\pgfpathlineto{\pgfqpoint{2.315226in}{1.822893in}}%
\pgfpathlineto{\pgfqpoint{2.330934in}{1.888767in}}%
\pgfpathlineto{\pgfqpoint{2.346642in}{1.973579in}}%
\pgfpathlineto{\pgfqpoint{2.389070in}{2.225600in}}%
\pgfpathlineto{\pgfqpoint{2.399244in}{2.277160in}}%
\pgfpathlineto{\pgfqpoint{2.416777in}{2.346427in}}%
\pgfpathlineto{\pgfqpoint{2.434309in}{2.386090in}}%
\pgfpathlineto{\pgfqpoint{2.451841in}{2.393159in}}%
\pgfpathlineto{\pgfqpoint{2.469374in}{2.368591in}}%
\pgfpathlineto{\pgfqpoint{2.486906in}{2.316946in}}%
\pgfpathlineto{\pgfqpoint{2.504438in}{2.245647in}}%
\pgfpathlineto{\pgfqpoint{2.526808in}{2.137056in}}%
\pgfpathlineto{\pgfqpoint{2.536248in}{2.066830in}}%
\pgfpathlineto{\pgfqpoint{2.544046in}{1.992862in}}%
\pgfpathlineto{\pgfqpoint{2.557876in}{1.832815in}}%
\pgfpathlineto{\pgfqpoint{2.585536in}{1.443520in}}%
\pgfpathlineto{\pgfqpoint{2.613196in}{1.051143in}}%
\pgfpathlineto{\pgfqpoint{2.627026in}{0.886458in}}%
\pgfpathlineto{\pgfqpoint{2.640856in}{0.757621in}}%
\pgfpathlineto{\pgfqpoint{2.654686in}{0.672952in}}%
\pgfpathlineto{\pgfqpoint{2.670226in}{0.637273in}}%
\pgfpathlineto{\pgfqpoint{2.671617in}{0.637762in}}%
\pgfpathlineto{\pgfqpoint{2.674980in}{0.642965in}}%
\pgfpathlineto{\pgfqpoint{2.678343in}{0.653818in}}%
\pgfpathlineto{\pgfqpoint{2.681706in}{0.670244in}}%
\pgfpathlineto{\pgfqpoint{2.690547in}{0.739025in}}%
\pgfpathlineto{\pgfqpoint{2.702014in}{0.878704in}}%
\pgfpathlineto{\pgfqpoint{2.710189in}{1.007638in}}%
\pgfpathlineto{\pgfqpoint{2.732471in}{1.440867in}}%
\pgfpathlineto{\pgfqpoint{2.765894in}{2.121369in}}%
\pgfpathlineto{\pgfqpoint{2.777036in}{2.304580in}}%
\pgfpathlineto{\pgfqpoint{2.788177in}{2.449026in}}%
\pgfpathlineto{\pgfqpoint{2.799318in}{2.548286in}}%
\pgfpathlineto{\pgfqpoint{2.810459in}{2.598903in}}%
\pgfpathlineto{\pgfqpoint{2.816667in}{2.605855in}}%
\pgfpathlineto{\pgfqpoint{2.816667in}{2.605855in}}%
\pgfpathlineto{\pgfqpoint{2.819546in}{2.603513in}}%
\pgfpathlineto{\pgfqpoint{2.825305in}{2.595244in}}%
\pgfpathlineto{\pgfqpoint{2.833727in}{2.570739in}}%
\pgfpathlineto{\pgfqpoint{2.840995in}{2.538754in}}%
\pgfpathlineto{\pgfqpoint{2.848264in}{2.497809in}}%
\pgfpathlineto{\pgfqpoint{2.859596in}{2.419292in}}%
\pgfpathlineto{\pgfqpoint{2.891139in}{2.147943in}}%
\pgfpathlineto{\pgfqpoint{2.906911in}{2.010110in}}%
\pgfpathlineto{\pgfqpoint{2.922682in}{1.888944in}}%
\pgfpathlineto{\pgfqpoint{2.938454in}{1.794825in}}%
\pgfpathlineto{\pgfqpoint{2.954225in}{1.734494in}}%
\pgfpathlineto{\pgfqpoint{2.964286in}{1.714992in}}%
\pgfpathlineto{\pgfqpoint{2.964286in}{1.714992in}}%
\pgfpathlineto{\pgfqpoint{2.969006in}{1.712684in}}%
\pgfpathlineto{\pgfqpoint{2.978448in}{1.714384in}}%
\pgfpathlineto{\pgfqpoint{2.987889in}{1.728216in}}%
\pgfpathlineto{\pgfqpoint{3.004141in}{1.776373in}}%
\pgfpathlineto{\pgfqpoint{3.020393in}{1.848757in}}%
\pgfpathlineto{\pgfqpoint{3.085402in}{2.189273in}}%
\pgfpathlineto{\pgfqpoint{3.101654in}{2.244079in}}%
\pgfpathlineto{\pgfqpoint{3.111905in}{2.266987in}}%
\pgfpathlineto{\pgfqpoint{3.111905in}{2.266987in}}%
\pgfpathlineto{\pgfqpoint{3.117996in}{2.274233in}}%
\pgfpathlineto{\pgfqpoint{3.130179in}{2.281846in}}%
\pgfpathlineto{\pgfqpoint{3.142361in}{2.276131in}}%
\pgfpathlineto{\pgfqpoint{3.162396in}{2.241748in}}%
\pgfpathlineto{\pgfqpoint{3.182430in}{2.183784in}}%
\pgfpathlineto{\pgfqpoint{3.222498in}{2.042768in}}%
\pgfpathlineto{\pgfqpoint{3.242532in}{1.982456in}}%
\pgfpathlineto{\pgfqpoint{3.262566in}{1.940570in}}%
\pgfpathlineto{\pgfqpoint{3.282600in}{1.921465in}}%
\pgfpathlineto{\pgfqpoint{3.302634in}{1.925564in}}%
\pgfpathlineto{\pgfqpoint{3.322668in}{1.949754in}}%
\pgfpathlineto{\pgfqpoint{3.342702in}{1.988262in}}%
\pgfpathlineto{\pgfqpoint{3.392004in}{2.097067in}}%
\pgfpathlineto{\pgfqpoint{3.416654in}{2.135839in}}%
\pgfpathlineto{\pgfqpoint{3.441305in}{2.153737in}}%
\pgfpathlineto{\pgfqpoint{3.465956in}{2.149449in}}%
\pgfpathlineto{\pgfqpoint{3.490607in}{2.126713in}}%
\pgfpathlineto{\pgfqpoint{3.545116in}{2.050108in}}%
\pgfpathlineto{\pgfqpoint{3.574975in}{2.017910in}}%
\pgfpathlineto{\pgfqpoint{3.604833in}{2.004820in}}%
\pgfpathlineto{\pgfqpoint{3.634692in}{2.012131in}}%
\pgfpathlineto{\pgfqpoint{3.664550in}{2.034600in}}%
\pgfpathlineto{\pgfqpoint{3.702381in}{2.070732in}}%
\pgfpathlineto{\pgfqpoint{3.702381in}{2.070732in}}%
\pgfpathlineto{\pgfqpoint{3.715445in}{2.080871in}}%
\pgfpathlineto{\pgfqpoint{3.741572in}{2.096619in}}%
\pgfpathlineto{\pgfqpoint{3.767699in}{2.102550in}}%
\pgfpathlineto{\pgfqpoint{3.809253in}{2.093141in}}%
\pgfpathlineto{\pgfqpoint{3.855230in}{2.069656in}}%
\pgfpathlineto{\pgfqpoint{3.861068in}{2.072773in}}%
\pgfpathlineto{\pgfqpoint{3.865423in}{2.078245in}}%
\pgfpathlineto{\pgfqpoint{3.870700in}{2.088333in}}%
\pgfpathlineto{\pgfqpoint{3.878279in}{2.109069in}}%
\pgfpathlineto{\pgfqpoint{3.889206in}{2.150666in}}%
\pgfpathlineto{\pgfqpoint{3.905161in}{2.231722in}}%
\pgfpathlineto{\pgfqpoint{3.921116in}{2.330096in}}%
\pgfpathlineto{\pgfqpoint{3.960043in}{2.587806in}}%
\pgfpathlineto{\pgfqpoint{3.979506in}{2.697076in}}%
\pgfpathlineto{\pgfqpoint{4.001145in}{2.781200in}}%
\pgfpathlineto{\pgfqpoint{4.004808in}{2.787857in}}%
\pgfpathlineto{\pgfqpoint{4.012489in}{2.789471in}}%
\pgfpathlineto{\pgfqpoint{4.025329in}{2.755464in}}%
\pgfpathlineto{\pgfqpoint{4.038169in}{2.678019in}}%
\pgfpathlineto{\pgfqpoint{4.051009in}{2.562459in}}%
\pgfpathlineto{\pgfqpoint{4.063848in}{2.416540in}}%
\pgfpathlineto{\pgfqpoint{4.128048in}{1.599557in}}%
\pgfpathlineto{\pgfqpoint{4.140887in}{1.494077in}}%
\pgfpathlineto{\pgfqpoint{4.153727in}{1.427807in}}%
\pgfpathlineto{\pgfqpoint{4.166567in}{1.404938in}}%
\pgfpathlineto{\pgfqpoint{4.179407in}{1.426819in}}%
\pgfpathlineto{\pgfqpoint{4.192247in}{1.491863in}}%
\pgfpathlineto{\pgfqpoint{4.205087in}{1.595651in}}%
\pgfpathlineto{\pgfqpoint{4.217926in}{1.731222in}}%
\pgfpathlineto{\pgfqpoint{4.243606in}{2.060080in}}%
\pgfpathlineto{\pgfqpoint{4.269286in}{2.392732in}}%
\pgfpathlineto{\pgfqpoint{4.282126in}{2.533015in}}%
\pgfpathlineto{\pgfqpoint{4.294966in}{2.643333in}}%
\pgfpathlineto{\pgfqpoint{4.307805in}{2.716629in}}%
\pgfpathlineto{\pgfqpoint{4.320645in}{2.748337in}}%
\pgfpathlineto{\pgfqpoint{4.333485in}{2.736648in}}%
\pgfpathlineto{\pgfqpoint{4.346325in}{2.682607in}}%
\pgfpathlineto{\pgfqpoint{4.359165in}{2.590006in}}%
\pgfpathlineto{\pgfqpoint{4.372005in}{2.465117in}}%
\pgfpathlineto{\pgfqpoint{4.397684in}{2.153246in}}%
\pgfpathlineto{\pgfqpoint{4.423364in}{1.827625in}}%
\pgfpathlineto{\pgfqpoint{4.436204in}{1.686181in}}%
\pgfpathlineto{\pgfqpoint{4.449044in}{1.571558in}}%
\pgfpathlineto{\pgfqpoint{4.461883in}{1.491123in}}%
\pgfpathlineto{\pgfqpoint{4.474723in}{1.450003in}}%
\pgfpathlineto{\pgfqpoint{4.487563in}{1.450746in}}%
\pgfpathlineto{\pgfqpoint{4.500403in}{1.493158in}}%
\pgfpathlineto{\pgfqpoint{4.513243in}{1.574310in}}%
\pgfpathlineto{\pgfqpoint{4.526083in}{1.688733in}}%
\pgfpathlineto{\pgfqpoint{4.551762in}{1.985131in}}%
\pgfpathlineto{\pgfqpoint{4.577442in}{2.304977in}}%
\pgfpathlineto{\pgfqpoint{4.590282in}{2.447434in}}%
\pgfpathlineto{\pgfqpoint{4.603122in}{2.565531in}}%
\pgfpathlineto{\pgfqpoint{4.615961in}{2.651660in}}%
\pgfpathlineto{\pgfqpoint{4.628801in}{2.700374in}}%
\pgfpathlineto{\pgfqpoint{4.641641in}{2.708720in}}%
\pgfpathlineto{\pgfqpoint{4.654481in}{2.676408in}}%
\pgfpathlineto{\pgfqpoint{4.667321in}{2.605807in}}%
\pgfpathlineto{\pgfqpoint{4.680161in}{2.501755in}}%
\pgfpathlineto{\pgfqpoint{4.693000in}{2.371230in}}%
\pgfpathlineto{\pgfqpoint{4.754689in}{1.672064in}}%
\pgfpathlineto{\pgfqpoint{4.770112in}{1.558256in}}%
\pgfpathlineto{\pgfqpoint{4.785534in}{1.494539in}}%
\pgfpathlineto{\pgfqpoint{4.800956in}{1.486651in}}%
\pgfpathlineto{\pgfqpoint{4.816378in}{1.535092in}}%
\pgfpathlineto{\pgfqpoint{4.831801in}{1.635074in}}%
\pgfpathlineto{\pgfqpoint{4.847223in}{1.776966in}}%
\pgfpathlineto{\pgfqpoint{4.878067in}{2.129639in}}%
\pgfpathlineto{\pgfqpoint{4.893489in}{2.307003in}}%
\pgfpathlineto{\pgfqpoint{4.908912in}{2.462655in}}%
\pgfpathlineto{\pgfqpoint{4.924334in}{2.582111in}}%
\pgfpathlineto{\pgfqpoint{4.939756in}{2.654419in}}%
\pgfpathlineto{\pgfqpoint{4.955178in}{2.673148in}}%
\pgfpathlineto{\pgfqpoint{4.970601in}{2.636942in}}%
\pgfpathlineto{\pgfqpoint{4.986023in}{2.549588in}}%
\pgfpathlineto{\pgfqpoint{5.001445in}{2.419610in}}%
\pgfpathlineto{\pgfqpoint{5.032290in}{2.084147in}}%
\pgfpathlineto{\pgfqpoint{5.047712in}{1.910215in}}%
\pgfpathlineto{\pgfqpoint{5.063134in}{1.753825in}}%
\pgfpathlineto{\pgfqpoint{5.078556in}{1.629469in}}%
\pgfpathlineto{\pgfqpoint{5.093979in}{1.548598in}}%
\pgfpathlineto{\pgfqpoint{5.109401in}{1.518572in}}%
\pgfpathlineto{\pgfqpoint{5.124823in}{1.541980in}}%
\pgfpathlineto{\pgfqpoint{5.140245in}{1.616392in}}%
\pgfpathlineto{\pgfqpoint{5.155668in}{1.734590in}}%
\pgfpathlineto{\pgfqpoint{5.178571in}{1.966337in}}%
\pgfpathlineto{\pgfqpoint{5.178571in}{1.966337in}}%
\pgfusepath{stroke}%
\end{pgfscope}%
\begin{pgfscope}%
\pgfpathrectangle{\pgfqpoint{0.750000in}{0.500000in}}{\pgfqpoint{4.650000in}{3.020000in}}%
\pgfusepath{clip}%
\pgfsetrectcap%
\pgfsetroundjoin%
\pgfsetlinewidth{1.505625pt}%
\definecolor{currentstroke}{rgb}{1.000000,0.000000,0.000000}%
\pgfsetstrokecolor{currentstroke}%
\pgfsetdash{}{0pt}%
\pgfpathmoveto{\pgfqpoint{1.351542in}{0.500000in}}%
\pgfpathlineto{\pgfqpoint{1.351542in}{3.520000in}}%
\pgfusepath{stroke}%
\end{pgfscope}%
\begin{pgfscope}%
\pgfpathrectangle{\pgfqpoint{0.750000in}{0.500000in}}{\pgfqpoint{4.650000in}{3.020000in}}%
\pgfusepath{clip}%
\pgfsetrectcap%
\pgfsetroundjoin%
\pgfsetlinewidth{1.505625pt}%
\definecolor{currentstroke}{rgb}{1.000000,0.000000,0.000000}%
\pgfsetstrokecolor{currentstroke}%
\pgfsetdash{}{0pt}%
\pgfpathmoveto{\pgfqpoint{1.849521in}{0.500000in}}%
\pgfpathlineto{\pgfqpoint{1.849521in}{3.520000in}}%
\pgfusepath{stroke}%
\end{pgfscope}%
\begin{pgfscope}%
\pgfpathrectangle{\pgfqpoint{0.750000in}{0.500000in}}{\pgfqpoint{4.650000in}{3.020000in}}%
\pgfusepath{clip}%
\pgfsetrectcap%
\pgfsetroundjoin%
\pgfsetlinewidth{1.505625pt}%
\definecolor{currentstroke}{rgb}{1.000000,0.000000,0.000000}%
\pgfsetstrokecolor{currentstroke}%
\pgfsetdash{}{0pt}%
\pgfpathmoveto{\pgfqpoint{2.585536in}{0.500000in}}%
\pgfpathlineto{\pgfqpoint{2.585536in}{3.520000in}}%
\pgfusepath{stroke}%
\end{pgfscope}%
\begin{pgfscope}%
\pgfpathrectangle{\pgfqpoint{0.750000in}{0.500000in}}{\pgfqpoint{4.650000in}{3.020000in}}%
\pgfusepath{clip}%
\pgfsetrectcap%
\pgfsetroundjoin%
\pgfsetlinewidth{1.505625pt}%
\definecolor{currentstroke}{rgb}{1.000000,0.000000,0.000000}%
\pgfsetstrokecolor{currentstroke}%
\pgfsetdash{}{0pt}%
\pgfpathmoveto{\pgfqpoint{3.262566in}{0.500000in}}%
\pgfpathlineto{\pgfqpoint{3.262566in}{3.520000in}}%
\pgfusepath{stroke}%
\end{pgfscope}%
\begin{pgfscope}%
\pgfpathrectangle{\pgfqpoint{0.750000in}{0.500000in}}{\pgfqpoint{4.650000in}{3.020000in}}%
\pgfusepath{clip}%
\pgfsetrectcap%
\pgfsetroundjoin%
\pgfsetlinewidth{1.505625pt}%
\definecolor{currentstroke}{rgb}{1.000000,0.000000,0.000000}%
\pgfsetstrokecolor{currentstroke}%
\pgfsetdash{}{0pt}%
\pgfpathmoveto{\pgfqpoint{4.140887in}{0.500000in}}%
\pgfpathlineto{\pgfqpoint{4.140887in}{3.520000in}}%
\pgfusepath{stroke}%
\end{pgfscope}%
\begin{pgfscope}%
\pgfsetrectcap%
\pgfsetmiterjoin%
\pgfsetlinewidth{0.803000pt}%
\definecolor{currentstroke}{rgb}{0.000000,0.000000,0.000000}%
\pgfsetstrokecolor{currentstroke}%
\pgfsetdash{}{0pt}%
\pgfpathmoveto{\pgfqpoint{0.750000in}{0.500000in}}%
\pgfpathlineto{\pgfqpoint{0.750000in}{3.520000in}}%
\pgfusepath{stroke}%
\end{pgfscope}%
\begin{pgfscope}%
\pgfsetrectcap%
\pgfsetmiterjoin%
\pgfsetlinewidth{0.803000pt}%
\definecolor{currentstroke}{rgb}{0.000000,0.000000,0.000000}%
\pgfsetstrokecolor{currentstroke}%
\pgfsetdash{}{0pt}%
\pgfpathmoveto{\pgfqpoint{5.400000in}{0.500000in}}%
\pgfpathlineto{\pgfqpoint{5.400000in}{3.520000in}}%
\pgfusepath{stroke}%
\end{pgfscope}%
\begin{pgfscope}%
\pgfsetrectcap%
\pgfsetmiterjoin%
\pgfsetlinewidth{0.803000pt}%
\definecolor{currentstroke}{rgb}{0.000000,0.000000,0.000000}%
\pgfsetstrokecolor{currentstroke}%
\pgfsetdash{}{0pt}%
\pgfpathmoveto{\pgfqpoint{0.750000in}{0.500000in}}%
\pgfpathlineto{\pgfqpoint{5.400000in}{0.500000in}}%
\pgfusepath{stroke}%
\end{pgfscope}%
\begin{pgfscope}%
\pgfsetrectcap%
\pgfsetmiterjoin%
\pgfsetlinewidth{0.803000pt}%
\definecolor{currentstroke}{rgb}{0.000000,0.000000,0.000000}%
\pgfsetstrokecolor{currentstroke}%
\pgfsetdash{}{0pt}%
\pgfpathmoveto{\pgfqpoint{0.750000in}{3.520000in}}%
\pgfpathlineto{\pgfqpoint{5.400000in}{3.520000in}}%
\pgfusepath{stroke}%
\end{pgfscope}%
\begin{pgfscope}%
\pgfsetbuttcap%
\pgfsetmiterjoin%
\definecolor{currentfill}{rgb}{1.000000,1.000000,1.000000}%
\pgfsetfillcolor{currentfill}%
\pgfsetfillopacity{0.800000}%
\pgfsetlinewidth{1.003750pt}%
\definecolor{currentstroke}{rgb}{0.800000,0.800000,0.800000}%
\pgfsetstrokecolor{currentstroke}%
\pgfsetstrokeopacity{0.800000}%
\pgfsetdash{}{0pt}%
\pgfpathmoveto{\pgfqpoint{4.302777in}{0.569444in}}%
\pgfpathlineto{\pgfqpoint{5.302778in}{0.569444in}}%
\pgfpathquadraticcurveto{\pgfqpoint{5.330556in}{0.569444in}}{\pgfqpoint{5.330556in}{0.597222in}}%
\pgfpathlineto{\pgfqpoint{5.330556in}{1.164352in}}%
\pgfpathquadraticcurveto{\pgfqpoint{5.330556in}{1.192129in}}{\pgfqpoint{5.302778in}{1.192129in}}%
\pgfpathlineto{\pgfqpoint{4.302777in}{1.192129in}}%
\pgfpathquadraticcurveto{\pgfqpoint{4.274999in}{1.192129in}}{\pgfqpoint{4.274999in}{1.164352in}}%
\pgfpathlineto{\pgfqpoint{4.274999in}{0.597222in}}%
\pgfpathquadraticcurveto{\pgfqpoint{4.274999in}{0.569444in}}{\pgfqpoint{4.302777in}{0.569444in}}%
\pgfpathlineto{\pgfqpoint{4.302777in}{0.569444in}}%
\pgfpathclose%
\pgfusepath{stroke,fill}%
\end{pgfscope}%
\begin{pgfscope}%
\pgfsetrectcap%
\pgfsetroundjoin%
\pgfsetlinewidth{1.505625pt}%
\definecolor{currentstroke}{rgb}{1.000000,0.000000,0.000000}%
\pgfsetstrokecolor{currentstroke}%
\pgfsetdash{}{0pt}%
\pgfpathmoveto{\pgfqpoint{4.330554in}{1.087963in}}%
\pgfpathlineto{\pgfqpoint{4.469443in}{1.087963in}}%
\pgfpathlineto{\pgfqpoint{4.608332in}{1.087963in}}%
\pgfusepath{stroke}%
\end{pgfscope}%
\begin{pgfscope}%
\definecolor{textcolor}{rgb}{0.000000,0.000000,0.000000}%
\pgfsetstrokecolor{textcolor}%
\pgfsetfillcolor{textcolor}%
\pgftext[x=4.719443in,y=1.039352in,left,base]{\color{textcolor}\rmfamily\fontsize{10.000000}{12.000000}\selectfont detection}%
\end{pgfscope}%
\begin{pgfscope}%
\pgfsetbuttcap%
\pgfsetroundjoin%
\pgfsetlinewidth{1.505625pt}%
\definecolor{currentstroke}{rgb}{0.000000,0.000000,1.000000}%
\pgfsetstrokecolor{currentstroke}%
\pgfsetdash{{9.600000pt}{2.400000pt}{1.500000pt}{2.400000pt}}{0.000000pt}%
\pgfpathmoveto{\pgfqpoint{4.330554in}{0.894290in}}%
\pgfpathlineto{\pgfqpoint{4.469443in}{0.894290in}}%
\pgfpathlineto{\pgfqpoint{4.608332in}{0.894290in}}%
\pgfusepath{stroke}%
\end{pgfscope}%
\begin{pgfscope}%
\definecolor{textcolor}{rgb}{0.000000,0.000000,0.000000}%
\pgfsetstrokecolor{textcolor}%
\pgfsetfillcolor{textcolor}%
\pgftext[x=4.719443in,y=0.845679in,left,base]{\color{textcolor}\rmfamily\fontsize{10.000000}{12.000000}\selectfont force}%
\end{pgfscope}%
\begin{pgfscope}%
\pgfsetbuttcap%
\pgfsetroundjoin%
\pgfsetlinewidth{1.505625pt}%
\definecolor{currentstroke}{rgb}{0.000000,0.500000,0.000000}%
\pgfsetstrokecolor{currentstroke}%
\pgfsetdash{{5.550000pt}{2.400000pt}}{0.000000pt}%
\pgfpathmoveto{\pgfqpoint{4.330554in}{0.700617in}}%
\pgfpathlineto{\pgfqpoint{4.469443in}{0.700617in}}%
\pgfpathlineto{\pgfqpoint{4.608332in}{0.700617in}}%
\pgfusepath{stroke}%
\end{pgfscope}%
\begin{pgfscope}%
\definecolor{textcolor}{rgb}{0.000000,0.000000,0.000000}%
\pgfsetstrokecolor{textcolor}%
\pgfsetfillcolor{textcolor}%
\pgftext[x=4.719443in,y=0.652006in,left,base]{\color{textcolor}\rmfamily\fontsize{10.000000}{12.000000}\selectfont pressure}%
\end{pgfscope}%
\end{pgfpicture}%
\makeatother%
\endgroup%

    \caption{Hydraulic Crane Results [Red Line = Motif Change]}
    \label{fig:hydraulic_result_fp}
\end{figure}

Figure \ref{fig:hydraulic_mp_hist_fp} shows the values calculated during the matrix profile for the force and pressure of the system with the control signal in Figure \ref{fig:hydraulic_cs}. The shape of the matrix profile value signals is nearly identical for the force and pressure values. The two signals have dramatically different magintudes but similar shapes. This shows the algorithm is robust to different macro data scales and explains the results obtained in Figure \ref{fig:hydraulic_result_fp}.
\begin{figure}[H]
    %\centering
    %% Creator: Matplotlib, PGF backend
%%
%% To include the figure in your LaTeX document, write
%%   \input{<filename>.pgf}
%%
%% Make sure the required packages are loaded in your preamble
%%   \usepackage{pgf}
%%
%% Also ensure that all the required font packages are loaded; for instance,
%% the lmodern package is sometimes necessary when using math font.
%%   \usepackage{lmodern}
%%
%% Figures using additional raster images can only be included by \input if
%% they are in the same directory as the main LaTeX file. For loading figures
%% from other directories you can use the `import` package
%%   \usepackage{import}
%%
%% and then include the figures with
%%   \import{<path to file>}{<filename>.pgf}
%%
%% Matplotlib used the following preamble
%%
\begingroup%
\makeatletter%
\begin{pgfpicture}%
\pgfpathrectangle{\pgfpointorigin}{\pgfqpoint{6.000000in}{4.000000in}}%
\pgfusepath{use as bounding box, clip}%
\begin{pgfscope}%
\pgfsetbuttcap%
\pgfsetmiterjoin%
\pgfsetlinewidth{0.000000pt}%
\definecolor{currentstroke}{rgb}{1.000000,1.000000,1.000000}%
\pgfsetstrokecolor{currentstroke}%
\pgfsetstrokeopacity{0.000000}%
\pgfsetdash{}{0pt}%
\pgfpathmoveto{\pgfqpoint{0.000000in}{0.000000in}}%
\pgfpathlineto{\pgfqpoint{6.000000in}{0.000000in}}%
\pgfpathlineto{\pgfqpoint{6.000000in}{4.000000in}}%
\pgfpathlineto{\pgfqpoint{0.000000in}{4.000000in}}%
\pgfpathlineto{\pgfqpoint{0.000000in}{0.000000in}}%
\pgfpathclose%
\pgfusepath{}%
\end{pgfscope}%
\begin{pgfscope}%
\pgfsetbuttcap%
\pgfsetmiterjoin%
\definecolor{currentfill}{rgb}{1.000000,1.000000,1.000000}%
\pgfsetfillcolor{currentfill}%
\pgfsetlinewidth{0.000000pt}%
\definecolor{currentstroke}{rgb}{0.000000,0.000000,0.000000}%
\pgfsetstrokecolor{currentstroke}%
\pgfsetstrokeopacity{0.000000}%
\pgfsetdash{}{0pt}%
\pgfpathmoveto{\pgfqpoint{0.750000in}{0.500000in}}%
\pgfpathlineto{\pgfqpoint{5.400000in}{0.500000in}}%
\pgfpathlineto{\pgfqpoint{5.400000in}{3.520000in}}%
\pgfpathlineto{\pgfqpoint{0.750000in}{3.520000in}}%
\pgfpathlineto{\pgfqpoint{0.750000in}{0.500000in}}%
\pgfpathclose%
\pgfusepath{fill}%
\end{pgfscope}%
\begin{pgfscope}%
\pgfsetbuttcap%
\pgfsetroundjoin%
\definecolor{currentfill}{rgb}{0.000000,0.000000,0.000000}%
\pgfsetfillcolor{currentfill}%
\pgfsetlinewidth{0.803000pt}%
\definecolor{currentstroke}{rgb}{0.000000,0.000000,0.000000}%
\pgfsetstrokecolor{currentstroke}%
\pgfsetdash{}{0pt}%
\pgfsys@defobject{currentmarker}{\pgfqpoint{0.000000in}{-0.048611in}}{\pgfqpoint{0.000000in}{0.000000in}}{%
\pgfpathmoveto{\pgfqpoint{0.000000in}{0.000000in}}%
\pgfpathlineto{\pgfqpoint{0.000000in}{-0.048611in}}%
\pgfusepath{stroke,fill}%
}%
\begin{pgfscope}%
\pgfsys@transformshift{0.750000in}{0.500000in}%
\pgfsys@useobject{currentmarker}{}%
\end{pgfscope}%
\end{pgfscope}%
\begin{pgfscope}%
\definecolor{textcolor}{rgb}{0.000000,0.000000,0.000000}%
\pgfsetstrokecolor{textcolor}%
\pgfsetfillcolor{textcolor}%
\pgftext[x=0.750000in,y=0.402778in,,top]{\color{textcolor}\rmfamily\fontsize{10.000000}{12.000000}\selectfont \(\displaystyle {0.0}\)}%
\end{pgfscope}%
\begin{pgfscope}%
\pgfsetbuttcap%
\pgfsetroundjoin%
\definecolor{currentfill}{rgb}{0.000000,0.000000,0.000000}%
\pgfsetfillcolor{currentfill}%
\pgfsetlinewidth{0.803000pt}%
\definecolor{currentstroke}{rgb}{0.000000,0.000000,0.000000}%
\pgfsetstrokecolor{currentstroke}%
\pgfsetdash{}{0pt}%
\pgfsys@defobject{currentmarker}{\pgfqpoint{0.000000in}{-0.048611in}}{\pgfqpoint{0.000000in}{0.000000in}}{%
\pgfpathmoveto{\pgfqpoint{0.000000in}{0.000000in}}%
\pgfpathlineto{\pgfqpoint{0.000000in}{-0.048611in}}%
\pgfusepath{stroke,fill}%
}%
\begin{pgfscope}%
\pgfsys@transformshift{1.344253in}{0.500000in}%
\pgfsys@useobject{currentmarker}{}%
\end{pgfscope}%
\end{pgfscope}%
\begin{pgfscope}%
\definecolor{textcolor}{rgb}{0.000000,0.000000,0.000000}%
\pgfsetstrokecolor{textcolor}%
\pgfsetfillcolor{textcolor}%
\pgftext[x=1.344253in,y=0.402778in,,top]{\color{textcolor}\rmfamily\fontsize{10.000000}{12.000000}\selectfont \(\displaystyle {0.2}\)}%
\end{pgfscope}%
\begin{pgfscope}%
\pgfsetbuttcap%
\pgfsetroundjoin%
\definecolor{currentfill}{rgb}{0.000000,0.000000,0.000000}%
\pgfsetfillcolor{currentfill}%
\pgfsetlinewidth{0.803000pt}%
\definecolor{currentstroke}{rgb}{0.000000,0.000000,0.000000}%
\pgfsetstrokecolor{currentstroke}%
\pgfsetdash{}{0pt}%
\pgfsys@defobject{currentmarker}{\pgfqpoint{0.000000in}{-0.048611in}}{\pgfqpoint{0.000000in}{0.000000in}}{%
\pgfpathmoveto{\pgfqpoint{0.000000in}{0.000000in}}%
\pgfpathlineto{\pgfqpoint{0.000000in}{-0.048611in}}%
\pgfusepath{stroke,fill}%
}%
\begin{pgfscope}%
\pgfsys@transformshift{1.938506in}{0.500000in}%
\pgfsys@useobject{currentmarker}{}%
\end{pgfscope}%
\end{pgfscope}%
\begin{pgfscope}%
\definecolor{textcolor}{rgb}{0.000000,0.000000,0.000000}%
\pgfsetstrokecolor{textcolor}%
\pgfsetfillcolor{textcolor}%
\pgftext[x=1.938506in,y=0.402778in,,top]{\color{textcolor}\rmfamily\fontsize{10.000000}{12.000000}\selectfont \(\displaystyle {0.4}\)}%
\end{pgfscope}%
\begin{pgfscope}%
\pgfsetbuttcap%
\pgfsetroundjoin%
\definecolor{currentfill}{rgb}{0.000000,0.000000,0.000000}%
\pgfsetfillcolor{currentfill}%
\pgfsetlinewidth{0.803000pt}%
\definecolor{currentstroke}{rgb}{0.000000,0.000000,0.000000}%
\pgfsetstrokecolor{currentstroke}%
\pgfsetdash{}{0pt}%
\pgfsys@defobject{currentmarker}{\pgfqpoint{0.000000in}{-0.048611in}}{\pgfqpoint{0.000000in}{0.000000in}}{%
\pgfpathmoveto{\pgfqpoint{0.000000in}{0.000000in}}%
\pgfpathlineto{\pgfqpoint{0.000000in}{-0.048611in}}%
\pgfusepath{stroke,fill}%
}%
\begin{pgfscope}%
\pgfsys@transformshift{2.532758in}{0.500000in}%
\pgfsys@useobject{currentmarker}{}%
\end{pgfscope}%
\end{pgfscope}%
\begin{pgfscope}%
\definecolor{textcolor}{rgb}{0.000000,0.000000,0.000000}%
\pgfsetstrokecolor{textcolor}%
\pgfsetfillcolor{textcolor}%
\pgftext[x=2.532758in,y=0.402778in,,top]{\color{textcolor}\rmfamily\fontsize{10.000000}{12.000000}\selectfont \(\displaystyle {0.6}\)}%
\end{pgfscope}%
\begin{pgfscope}%
\pgfsetbuttcap%
\pgfsetroundjoin%
\definecolor{currentfill}{rgb}{0.000000,0.000000,0.000000}%
\pgfsetfillcolor{currentfill}%
\pgfsetlinewidth{0.803000pt}%
\definecolor{currentstroke}{rgb}{0.000000,0.000000,0.000000}%
\pgfsetstrokecolor{currentstroke}%
\pgfsetdash{}{0pt}%
\pgfsys@defobject{currentmarker}{\pgfqpoint{0.000000in}{-0.048611in}}{\pgfqpoint{0.000000in}{0.000000in}}{%
\pgfpathmoveto{\pgfqpoint{0.000000in}{0.000000in}}%
\pgfpathlineto{\pgfqpoint{0.000000in}{-0.048611in}}%
\pgfusepath{stroke,fill}%
}%
\begin{pgfscope}%
\pgfsys@transformshift{3.127011in}{0.500000in}%
\pgfsys@useobject{currentmarker}{}%
\end{pgfscope}%
\end{pgfscope}%
\begin{pgfscope}%
\definecolor{textcolor}{rgb}{0.000000,0.000000,0.000000}%
\pgfsetstrokecolor{textcolor}%
\pgfsetfillcolor{textcolor}%
\pgftext[x=3.127011in,y=0.402778in,,top]{\color{textcolor}\rmfamily\fontsize{10.000000}{12.000000}\selectfont \(\displaystyle {0.8}\)}%
\end{pgfscope}%
\begin{pgfscope}%
\pgfsetbuttcap%
\pgfsetroundjoin%
\definecolor{currentfill}{rgb}{0.000000,0.000000,0.000000}%
\pgfsetfillcolor{currentfill}%
\pgfsetlinewidth{0.803000pt}%
\definecolor{currentstroke}{rgb}{0.000000,0.000000,0.000000}%
\pgfsetstrokecolor{currentstroke}%
\pgfsetdash{}{0pt}%
\pgfsys@defobject{currentmarker}{\pgfqpoint{0.000000in}{-0.048611in}}{\pgfqpoint{0.000000in}{0.000000in}}{%
\pgfpathmoveto{\pgfqpoint{0.000000in}{0.000000in}}%
\pgfpathlineto{\pgfqpoint{0.000000in}{-0.048611in}}%
\pgfusepath{stroke,fill}%
}%
\begin{pgfscope}%
\pgfsys@transformshift{3.721264in}{0.500000in}%
\pgfsys@useobject{currentmarker}{}%
\end{pgfscope}%
\end{pgfscope}%
\begin{pgfscope}%
\definecolor{textcolor}{rgb}{0.000000,0.000000,0.000000}%
\pgfsetstrokecolor{textcolor}%
\pgfsetfillcolor{textcolor}%
\pgftext[x=3.721264in,y=0.402778in,,top]{\color{textcolor}\rmfamily\fontsize{10.000000}{12.000000}\selectfont \(\displaystyle {1.0}\)}%
\end{pgfscope}%
\begin{pgfscope}%
\pgfsetbuttcap%
\pgfsetroundjoin%
\definecolor{currentfill}{rgb}{0.000000,0.000000,0.000000}%
\pgfsetfillcolor{currentfill}%
\pgfsetlinewidth{0.803000pt}%
\definecolor{currentstroke}{rgb}{0.000000,0.000000,0.000000}%
\pgfsetstrokecolor{currentstroke}%
\pgfsetdash{}{0pt}%
\pgfsys@defobject{currentmarker}{\pgfqpoint{0.000000in}{-0.048611in}}{\pgfqpoint{0.000000in}{0.000000in}}{%
\pgfpathmoveto{\pgfqpoint{0.000000in}{0.000000in}}%
\pgfpathlineto{\pgfqpoint{0.000000in}{-0.048611in}}%
\pgfusepath{stroke,fill}%
}%
\begin{pgfscope}%
\pgfsys@transformshift{4.315517in}{0.500000in}%
\pgfsys@useobject{currentmarker}{}%
\end{pgfscope}%
\end{pgfscope}%
\begin{pgfscope}%
\definecolor{textcolor}{rgb}{0.000000,0.000000,0.000000}%
\pgfsetstrokecolor{textcolor}%
\pgfsetfillcolor{textcolor}%
\pgftext[x=4.315517in,y=0.402778in,,top]{\color{textcolor}\rmfamily\fontsize{10.000000}{12.000000}\selectfont \(\displaystyle {1.2}\)}%
\end{pgfscope}%
\begin{pgfscope}%
\pgfsetbuttcap%
\pgfsetroundjoin%
\definecolor{currentfill}{rgb}{0.000000,0.000000,0.000000}%
\pgfsetfillcolor{currentfill}%
\pgfsetlinewidth{0.803000pt}%
\definecolor{currentstroke}{rgb}{0.000000,0.000000,0.000000}%
\pgfsetstrokecolor{currentstroke}%
\pgfsetdash{}{0pt}%
\pgfsys@defobject{currentmarker}{\pgfqpoint{0.000000in}{-0.048611in}}{\pgfqpoint{0.000000in}{0.000000in}}{%
\pgfpathmoveto{\pgfqpoint{0.000000in}{0.000000in}}%
\pgfpathlineto{\pgfqpoint{0.000000in}{-0.048611in}}%
\pgfusepath{stroke,fill}%
}%
\begin{pgfscope}%
\pgfsys@transformshift{4.909769in}{0.500000in}%
\pgfsys@useobject{currentmarker}{}%
\end{pgfscope}%
\end{pgfscope}%
\begin{pgfscope}%
\definecolor{textcolor}{rgb}{0.000000,0.000000,0.000000}%
\pgfsetstrokecolor{textcolor}%
\pgfsetfillcolor{textcolor}%
\pgftext[x=4.909769in,y=0.402778in,,top]{\color{textcolor}\rmfamily\fontsize{10.000000}{12.000000}\selectfont \(\displaystyle {1.4}\)}%
\end{pgfscope}%
\begin{pgfscope}%
\definecolor{textcolor}{rgb}{0.000000,0.000000,0.000000}%
\pgfsetstrokecolor{textcolor}%
\pgfsetfillcolor{textcolor}%
\pgftext[x=3.075000in,y=0.223766in,,top]{\color{textcolor}\rmfamily\fontsize{10.000000}{12.000000}\selectfont Time (s)}%
\end{pgfscope}%
\begin{pgfscope}%
\pgfsetbuttcap%
\pgfsetroundjoin%
\definecolor{currentfill}{rgb}{0.000000,0.000000,0.000000}%
\pgfsetfillcolor{currentfill}%
\pgfsetlinewidth{0.803000pt}%
\definecolor{currentstroke}{rgb}{0.000000,0.000000,0.000000}%
\pgfsetstrokecolor{currentstroke}%
\pgfsetdash{}{0pt}%
\pgfsys@defobject{currentmarker}{\pgfqpoint{-0.048611in}{0.000000in}}{\pgfqpoint{-0.000000in}{0.000000in}}{%
\pgfpathmoveto{\pgfqpoint{-0.000000in}{0.000000in}}%
\pgfpathlineto{\pgfqpoint{-0.048611in}{0.000000in}}%
\pgfusepath{stroke,fill}%
}%
\begin{pgfscope}%
\pgfsys@transformshift{0.750000in}{0.637268in}%
\pgfsys@useobject{currentmarker}{}%
\end{pgfscope}%
\end{pgfscope}%
\begin{pgfscope}%
\definecolor{textcolor}{rgb}{0.000000,0.000000,0.000000}%
\pgfsetstrokecolor{textcolor}%
\pgfsetfillcolor{textcolor}%
\pgftext[x=0.583333in, y=0.589043in, left, base]{\color{textcolor}\rmfamily\fontsize{10.000000}{12.000000}\selectfont \(\displaystyle {0}\)}%
\end{pgfscope}%
\begin{pgfscope}%
\pgfsetbuttcap%
\pgfsetroundjoin%
\definecolor{currentfill}{rgb}{0.000000,0.000000,0.000000}%
\pgfsetfillcolor{currentfill}%
\pgfsetlinewidth{0.803000pt}%
\definecolor{currentstroke}{rgb}{0.000000,0.000000,0.000000}%
\pgfsetstrokecolor{currentstroke}%
\pgfsetdash{}{0pt}%
\pgfsys@defobject{currentmarker}{\pgfqpoint{-0.048611in}{0.000000in}}{\pgfqpoint{-0.000000in}{0.000000in}}{%
\pgfpathmoveto{\pgfqpoint{-0.000000in}{0.000000in}}%
\pgfpathlineto{\pgfqpoint{-0.048611in}{0.000000in}}%
\pgfusepath{stroke,fill}%
}%
\begin{pgfscope}%
\pgfsys@transformshift{0.750000in}{1.026858in}%
\pgfsys@useobject{currentmarker}{}%
\end{pgfscope}%
\end{pgfscope}%
\begin{pgfscope}%
\definecolor{textcolor}{rgb}{0.000000,0.000000,0.000000}%
\pgfsetstrokecolor{textcolor}%
\pgfsetfillcolor{textcolor}%
\pgftext[x=0.374999in, y=0.978633in, left, base]{\color{textcolor}\rmfamily\fontsize{10.000000}{12.000000}\selectfont \(\displaystyle {2500}\)}%
\end{pgfscope}%
\begin{pgfscope}%
\pgfsetbuttcap%
\pgfsetroundjoin%
\definecolor{currentfill}{rgb}{0.000000,0.000000,0.000000}%
\pgfsetfillcolor{currentfill}%
\pgfsetlinewidth{0.803000pt}%
\definecolor{currentstroke}{rgb}{0.000000,0.000000,0.000000}%
\pgfsetstrokecolor{currentstroke}%
\pgfsetdash{}{0pt}%
\pgfsys@defobject{currentmarker}{\pgfqpoint{-0.048611in}{0.000000in}}{\pgfqpoint{-0.000000in}{0.000000in}}{%
\pgfpathmoveto{\pgfqpoint{-0.000000in}{0.000000in}}%
\pgfpathlineto{\pgfqpoint{-0.048611in}{0.000000in}}%
\pgfusepath{stroke,fill}%
}%
\begin{pgfscope}%
\pgfsys@transformshift{0.750000in}{1.416447in}%
\pgfsys@useobject{currentmarker}{}%
\end{pgfscope}%
\end{pgfscope}%
\begin{pgfscope}%
\definecolor{textcolor}{rgb}{0.000000,0.000000,0.000000}%
\pgfsetstrokecolor{textcolor}%
\pgfsetfillcolor{textcolor}%
\pgftext[x=0.374999in, y=1.368222in, left, base]{\color{textcolor}\rmfamily\fontsize{10.000000}{12.000000}\selectfont \(\displaystyle {5000}\)}%
\end{pgfscope}%
\begin{pgfscope}%
\pgfsetbuttcap%
\pgfsetroundjoin%
\definecolor{currentfill}{rgb}{0.000000,0.000000,0.000000}%
\pgfsetfillcolor{currentfill}%
\pgfsetlinewidth{0.803000pt}%
\definecolor{currentstroke}{rgb}{0.000000,0.000000,0.000000}%
\pgfsetstrokecolor{currentstroke}%
\pgfsetdash{}{0pt}%
\pgfsys@defobject{currentmarker}{\pgfqpoint{-0.048611in}{0.000000in}}{\pgfqpoint{-0.000000in}{0.000000in}}{%
\pgfpathmoveto{\pgfqpoint{-0.000000in}{0.000000in}}%
\pgfpathlineto{\pgfqpoint{-0.048611in}{0.000000in}}%
\pgfusepath{stroke,fill}%
}%
\begin{pgfscope}%
\pgfsys@transformshift{0.750000in}{1.806037in}%
\pgfsys@useobject{currentmarker}{}%
\end{pgfscope}%
\end{pgfscope}%
\begin{pgfscope}%
\definecolor{textcolor}{rgb}{0.000000,0.000000,0.000000}%
\pgfsetstrokecolor{textcolor}%
\pgfsetfillcolor{textcolor}%
\pgftext[x=0.374999in, y=1.757812in, left, base]{\color{textcolor}\rmfamily\fontsize{10.000000}{12.000000}\selectfont \(\displaystyle {7500}\)}%
\end{pgfscope}%
\begin{pgfscope}%
\pgfsetbuttcap%
\pgfsetroundjoin%
\definecolor{currentfill}{rgb}{0.000000,0.000000,0.000000}%
\pgfsetfillcolor{currentfill}%
\pgfsetlinewidth{0.803000pt}%
\definecolor{currentstroke}{rgb}{0.000000,0.000000,0.000000}%
\pgfsetstrokecolor{currentstroke}%
\pgfsetdash{}{0pt}%
\pgfsys@defobject{currentmarker}{\pgfqpoint{-0.048611in}{0.000000in}}{\pgfqpoint{-0.000000in}{0.000000in}}{%
\pgfpathmoveto{\pgfqpoint{-0.000000in}{0.000000in}}%
\pgfpathlineto{\pgfqpoint{-0.048611in}{0.000000in}}%
\pgfusepath{stroke,fill}%
}%
\begin{pgfscope}%
\pgfsys@transformshift{0.750000in}{2.195626in}%
\pgfsys@useobject{currentmarker}{}%
\end{pgfscope}%
\end{pgfscope}%
\begin{pgfscope}%
\definecolor{textcolor}{rgb}{0.000000,0.000000,0.000000}%
\pgfsetstrokecolor{textcolor}%
\pgfsetfillcolor{textcolor}%
\pgftext[x=0.305554in, y=2.147401in, left, base]{\color{textcolor}\rmfamily\fontsize{10.000000}{12.000000}\selectfont \(\displaystyle {10000}\)}%
\end{pgfscope}%
\begin{pgfscope}%
\pgfsetbuttcap%
\pgfsetroundjoin%
\definecolor{currentfill}{rgb}{0.000000,0.000000,0.000000}%
\pgfsetfillcolor{currentfill}%
\pgfsetlinewidth{0.803000pt}%
\definecolor{currentstroke}{rgb}{0.000000,0.000000,0.000000}%
\pgfsetstrokecolor{currentstroke}%
\pgfsetdash{}{0pt}%
\pgfsys@defobject{currentmarker}{\pgfqpoint{-0.048611in}{0.000000in}}{\pgfqpoint{-0.000000in}{0.000000in}}{%
\pgfpathmoveto{\pgfqpoint{-0.000000in}{0.000000in}}%
\pgfpathlineto{\pgfqpoint{-0.048611in}{0.000000in}}%
\pgfusepath{stroke,fill}%
}%
\begin{pgfscope}%
\pgfsys@transformshift{0.750000in}{2.585216in}%
\pgfsys@useobject{currentmarker}{}%
\end{pgfscope}%
\end{pgfscope}%
\begin{pgfscope}%
\definecolor{textcolor}{rgb}{0.000000,0.000000,0.000000}%
\pgfsetstrokecolor{textcolor}%
\pgfsetfillcolor{textcolor}%
\pgftext[x=0.305554in, y=2.536991in, left, base]{\color{textcolor}\rmfamily\fontsize{10.000000}{12.000000}\selectfont \(\displaystyle {12500}\)}%
\end{pgfscope}%
\begin{pgfscope}%
\pgfsetbuttcap%
\pgfsetroundjoin%
\definecolor{currentfill}{rgb}{0.000000,0.000000,0.000000}%
\pgfsetfillcolor{currentfill}%
\pgfsetlinewidth{0.803000pt}%
\definecolor{currentstroke}{rgb}{0.000000,0.000000,0.000000}%
\pgfsetstrokecolor{currentstroke}%
\pgfsetdash{}{0pt}%
\pgfsys@defobject{currentmarker}{\pgfqpoint{-0.048611in}{0.000000in}}{\pgfqpoint{-0.000000in}{0.000000in}}{%
\pgfpathmoveto{\pgfqpoint{-0.000000in}{0.000000in}}%
\pgfpathlineto{\pgfqpoint{-0.048611in}{0.000000in}}%
\pgfusepath{stroke,fill}%
}%
\begin{pgfscope}%
\pgfsys@transformshift{0.750000in}{2.974805in}%
\pgfsys@useobject{currentmarker}{}%
\end{pgfscope}%
\end{pgfscope}%
\begin{pgfscope}%
\definecolor{textcolor}{rgb}{0.000000,0.000000,0.000000}%
\pgfsetstrokecolor{textcolor}%
\pgfsetfillcolor{textcolor}%
\pgftext[x=0.305554in, y=2.926580in, left, base]{\color{textcolor}\rmfamily\fontsize{10.000000}{12.000000}\selectfont \(\displaystyle {15000}\)}%
\end{pgfscope}%
\begin{pgfscope}%
\pgfsetbuttcap%
\pgfsetroundjoin%
\definecolor{currentfill}{rgb}{0.000000,0.000000,0.000000}%
\pgfsetfillcolor{currentfill}%
\pgfsetlinewidth{0.803000pt}%
\definecolor{currentstroke}{rgb}{0.000000,0.000000,0.000000}%
\pgfsetstrokecolor{currentstroke}%
\pgfsetdash{}{0pt}%
\pgfsys@defobject{currentmarker}{\pgfqpoint{-0.048611in}{0.000000in}}{\pgfqpoint{-0.000000in}{0.000000in}}{%
\pgfpathmoveto{\pgfqpoint{-0.000000in}{0.000000in}}%
\pgfpathlineto{\pgfqpoint{-0.048611in}{0.000000in}}%
\pgfusepath{stroke,fill}%
}%
\begin{pgfscope}%
\pgfsys@transformshift{0.750000in}{3.364395in}%
\pgfsys@useobject{currentmarker}{}%
\end{pgfscope}%
\end{pgfscope}%
\begin{pgfscope}%
\definecolor{textcolor}{rgb}{0.000000,0.000000,0.000000}%
\pgfsetstrokecolor{textcolor}%
\pgfsetfillcolor{textcolor}%
\pgftext[x=0.305554in, y=3.316170in, left, base]{\color{textcolor}\rmfamily\fontsize{10.000000}{12.000000}\selectfont \(\displaystyle {17500}\)}%
\end{pgfscope}%
\begin{pgfscope}%
\definecolor{textcolor}{rgb}{0.000000,0.000000,0.000000}%
\pgfsetstrokecolor{textcolor}%
\pgfsetfillcolor{textcolor}%
\pgftext[x=0.249999in,y=2.010000in,,bottom,rotate=90.000000]{\color{textcolor}\rmfamily\fontsize{10.000000}{12.000000}\selectfont Force (\(\displaystyle N\))}%
\end{pgfscope}%
\begin{pgfscope}%
\pgfpathrectangle{\pgfqpoint{0.750000in}{0.500000in}}{\pgfqpoint{4.650000in}{3.020000in}}%
\pgfusepath{clip}%
\pgfsetbuttcap%
\pgfsetroundjoin%
\definecolor{currentfill}{rgb}{0.000000,0.000000,1.000000}%
\pgfsetfillcolor{currentfill}%
\pgfsetlinewidth{0.000000pt}%
\definecolor{currentstroke}{rgb}{0.000000,0.000000,1.000000}%
\pgfsetstrokecolor{currentstroke}%
\pgfsetdash{}{0pt}%
\pgfsys@defobject{currentmarker}{\pgfqpoint{-0.006944in}{-0.006944in}}{\pgfqpoint{0.006945in}{0.006945in}}{%
\pgfpathmoveto{\pgfqpoint{-0.006944in}{-0.006944in}}%
\pgfpathlineto{\pgfqpoint{0.006945in}{-0.006944in}}%
\pgfpathlineto{\pgfqpoint{0.006945in}{0.006945in}}%
\pgfpathlineto{\pgfqpoint{-0.006944in}{0.006945in}}%
\pgfpathlineto{\pgfqpoint{-0.006944in}{-0.006944in}}%
\pgfpathclose%
\pgfusepath{fill}%
}%
\begin{pgfscope}%
\pgfsys@transformshift{1.344812in}{0.637287in}%
\pgfsys@useobject{currentmarker}{}%
\end{pgfscope}%
\begin{pgfscope}%
\pgfsys@transformshift{1.344894in}{0.637295in}%
\pgfsys@useobject{currentmarker}{}%
\end{pgfscope}%
\begin{pgfscope}%
\pgfsys@transformshift{1.344977in}{0.637305in}%
\pgfsys@useobject{currentmarker}{}%
\end{pgfscope}%
\begin{pgfscope}%
\pgfsys@transformshift{1.345086in}{0.637320in}%
\pgfsys@useobject{currentmarker}{}%
\end{pgfscope}%
\begin{pgfscope}%
\pgfsys@transformshift{1.345195in}{0.637339in}%
\pgfsys@useobject{currentmarker}{}%
\end{pgfscope}%
\begin{pgfscope}%
\pgfsys@transformshift{1.345347in}{0.637369in}%
\pgfsys@useobject{currentmarker}{}%
\end{pgfscope}%
\begin{pgfscope}%
\pgfsys@transformshift{1.345498in}{0.637408in}%
\pgfsys@useobject{currentmarker}{}%
\end{pgfscope}%
\begin{pgfscope}%
\pgfsys@transformshift{1.345723in}{0.637470in}%
\pgfsys@useobject{currentmarker}{}%
\end{pgfscope}%
\begin{pgfscope}%
\pgfsys@transformshift{1.345999in}{0.637567in}%
\pgfsys@useobject{currentmarker}{}%
\end{pgfscope}%
\begin{pgfscope}%
\pgfsys@transformshift{1.346380in}{0.637733in}%
\pgfsys@useobject{currentmarker}{}%
\end{pgfscope}%
\begin{pgfscope}%
\pgfsys@transformshift{1.346908in}{0.638024in}%
\pgfsys@useobject{currentmarker}{}%
\end{pgfscope}%
\begin{pgfscope}%
\pgfsys@transformshift{1.347743in}{0.638622in}%
\pgfsys@useobject{currentmarker}{}%
\end{pgfscope}%
\begin{pgfscope}%
\pgfsys@transformshift{1.349253in}{0.640116in}%
\pgfsys@useobject{currentmarker}{}%
\end{pgfscope}%
\begin{pgfscope}%
\pgfsys@transformshift{1.351299in}{0.643232in}%
\pgfsys@useobject{currentmarker}{}%
\end{pgfscope}%
\begin{pgfscope}%
\pgfsys@transformshift{1.353344in}{0.647989in}%
\pgfsys@useobject{currentmarker}{}%
\end{pgfscope}%
\begin{pgfscope}%
\pgfsys@transformshift{1.355390in}{0.654471in}%
\pgfsys@useobject{currentmarker}{}%
\end{pgfscope}%
\begin{pgfscope}%
\pgfsys@transformshift{1.357435in}{0.662666in}%
\pgfsys@useobject{currentmarker}{}%
\end{pgfscope}%
\begin{pgfscope}%
\pgfsys@transformshift{1.359480in}{0.672216in}%
\pgfsys@useobject{currentmarker}{}%
\end{pgfscope}%
\begin{pgfscope}%
\pgfsys@transformshift{1.361526in}{0.682617in}%
\pgfsys@useobject{currentmarker}{}%
\end{pgfscope}%
\begin{pgfscope}%
\pgfsys@transformshift{1.363571in}{0.693787in}%
\pgfsys@useobject{currentmarker}{}%
\end{pgfscope}%
\begin{pgfscope}%
\pgfsys@transformshift{1.365617in}{0.705653in}%
\pgfsys@useobject{currentmarker}{}%
\end{pgfscope}%
\begin{pgfscope}%
\pgfsys@transformshift{1.367662in}{0.718148in}%
\pgfsys@useobject{currentmarker}{}%
\end{pgfscope}%
\begin{pgfscope}%
\pgfsys@transformshift{1.369708in}{0.731211in}%
\pgfsys@useobject{currentmarker}{}%
\end{pgfscope}%
\begin{pgfscope}%
\pgfsys@transformshift{1.371753in}{0.744785in}%
\pgfsys@useobject{currentmarker}{}%
\end{pgfscope}%
\begin{pgfscope}%
\pgfsys@transformshift{1.374261in}{0.760277in}%
\pgfsys@useobject{currentmarker}{}%
\end{pgfscope}%
\begin{pgfscope}%
\pgfsys@transformshift{1.376769in}{0.777585in}%
\pgfsys@useobject{currentmarker}{}%
\end{pgfscope}%
\begin{pgfscope}%
\pgfsys@transformshift{1.379830in}{0.798656in}%
\pgfsys@useobject{currentmarker}{}%
\end{pgfscope}%
\begin{pgfscope}%
\pgfsys@transformshift{1.382890in}{0.823372in}%
\pgfsys@useobject{currentmarker}{}%
\end{pgfscope}%
\begin{pgfscope}%
\pgfsys@transformshift{1.386634in}{0.854563in}%
\pgfsys@useobject{currentmarker}{}%
\end{pgfscope}%
\begin{pgfscope}%
\pgfsys@transformshift{1.390378in}{0.890343in}%
\pgfsys@useobject{currentmarker}{}%
\end{pgfscope}%
\begin{pgfscope}%
\pgfsys@transformshift{1.395057in}{0.935085in}%
\pgfsys@useobject{currentmarker}{}%
\end{pgfscope}%
\begin{pgfscope}%
\pgfsys@transformshift{1.399736in}{0.985870in}%
\pgfsys@useobject{currentmarker}{}%
\end{pgfscope}%
\begin{pgfscope}%
\pgfsys@transformshift{1.405827in}{1.049951in}%
\pgfsys@useobject{currentmarker}{}%
\end{pgfscope}%
\begin{pgfscope}%
\pgfsys@transformshift{1.411918in}{1.122527in}%
\pgfsys@useobject{currentmarker}{}%
\end{pgfscope}%
\begin{pgfscope}%
\pgfsys@transformshift{1.420549in}{1.216896in}%
\pgfsys@useobject{currentmarker}{}%
\end{pgfscope}%
\begin{pgfscope}%
\pgfsys@transformshift{1.429180in}{1.323128in}%
\pgfsys@useobject{currentmarker}{}%
\end{pgfscope}%
\begin{pgfscope}%
\pgfsys@transformshift{1.437812in}{1.436217in}%
\pgfsys@useobject{currentmarker}{}%
\end{pgfscope}%
\begin{pgfscope}%
\pgfsys@transformshift{1.446443in}{1.543987in}%
\pgfsys@useobject{currentmarker}{}%
\end{pgfscope}%
\begin{pgfscope}%
\pgfsys@transformshift{1.455074in}{1.643333in}%
\pgfsys@useobject{currentmarker}{}%
\end{pgfscope}%
\begin{pgfscope}%
\pgfsys@transformshift{1.463705in}{1.720455in}%
\pgfsys@useobject{currentmarker}{}%
\end{pgfscope}%
\begin{pgfscope}%
\pgfsys@transformshift{1.472337in}{1.776799in}%
\pgfsys@useobject{currentmarker}{}%
\end{pgfscope}%
\begin{pgfscope}%
\pgfsys@transformshift{1.482936in}{1.815527in}%
\pgfsys@useobject{currentmarker}{}%
\end{pgfscope}%
\begin{pgfscope}%
\pgfsys@transformshift{1.492816in}{1.836827in}%
\pgfsys@useobject{currentmarker}{}%
\end{pgfscope}%
\begin{pgfscope}%
\pgfsys@transformshift{1.492816in}{1.843485in}%
\pgfsys@useobject{currentmarker}{}%
\end{pgfscope}%
\begin{pgfscope}%
\pgfsys@transformshift{1.492816in}{1.843485in}%
\pgfsys@useobject{currentmarker}{}%
\end{pgfscope}%
\begin{pgfscope}%
\pgfsys@transformshift{1.494636in}{1.843485in}%
\pgfsys@useobject{currentmarker}{}%
\end{pgfscope}%
\begin{pgfscope}%
\pgfsys@transformshift{1.498276in}{1.843485in}%
\pgfsys@useobject{currentmarker}{}%
\end{pgfscope}%
\begin{pgfscope}%
\pgfsys@transformshift{1.501916in}{1.843485in}%
\pgfsys@useobject{currentmarker}{}%
\end{pgfscope}%
\begin{pgfscope}%
\pgfsys@transformshift{1.505557in}{1.841190in}%
\pgfsys@useobject{currentmarker}{}%
\end{pgfscope}%
\begin{pgfscope}%
\pgfsys@transformshift{1.510739in}{1.832218in}%
\pgfsys@useobject{currentmarker}{}%
\end{pgfscope}%
\begin{pgfscope}%
\pgfsys@transformshift{1.517027in}{1.818342in}%
\pgfsys@useobject{currentmarker}{}%
\end{pgfscope}%
\begin{pgfscope}%
\pgfsys@transformshift{1.525486in}{1.815527in}%
\pgfsys@useobject{currentmarker}{}%
\end{pgfscope}%
\begin{pgfscope}%
\pgfsys@transformshift{1.536357in}{1.815527in}%
\pgfsys@useobject{currentmarker}{}%
\end{pgfscope}%
\begin{pgfscope}%
\pgfsys@transformshift{1.547228in}{1.815527in}%
\pgfsys@useobject{currentmarker}{}%
\end{pgfscope}%
\begin{pgfscope}%
\pgfsys@transformshift{1.558099in}{1.815527in}%
\pgfsys@useobject{currentmarker}{}%
\end{pgfscope}%
\begin{pgfscope}%
\pgfsys@transformshift{1.568970in}{1.877000in}%
\pgfsys@useobject{currentmarker}{}%
\end{pgfscope}%
\begin{pgfscope}%
\pgfsys@transformshift{1.579841in}{2.127557in}%
\pgfsys@useobject{currentmarker}{}%
\end{pgfscope}%
\begin{pgfscope}%
\pgfsys@transformshift{1.590712in}{2.379757in}%
\pgfsys@useobject{currentmarker}{}%
\end{pgfscope}%
\begin{pgfscope}%
\pgfsys@transformshift{1.601583in}{2.612293in}%
\pgfsys@useobject{currentmarker}{}%
\end{pgfscope}%
\begin{pgfscope}%
\pgfsys@transformshift{1.612454in}{2.810923in}%
\pgfsys@useobject{currentmarker}{}%
\end{pgfscope}%
\begin{pgfscope}%
\pgfsys@transformshift{1.623325in}{2.968609in}%
\pgfsys@useobject{currentmarker}{}%
\end{pgfscope}%
\begin{pgfscope}%
\pgfsys@transformshift{1.636719in}{3.088707in}%
\pgfsys@useobject{currentmarker}{}%
\end{pgfscope}%
\begin{pgfscope}%
\pgfsys@transformshift{1.650113in}{3.166664in}%
\pgfsys@useobject{currentmarker}{}%
\end{pgfscope}%
\begin{pgfscope}%
\pgfsys@transformshift{1.663507in}{3.205609in}%
\pgfsys@useobject{currentmarker}{}%
\end{pgfscope}%
\begin{pgfscope}%
\pgfsys@transformshift{1.676902in}{3.215896in}%
\pgfsys@useobject{currentmarker}{}%
\end{pgfscope}%
\begin{pgfscope}%
\pgfsys@transformshift{1.690296in}{3.215896in}%
\pgfsys@useobject{currentmarker}{}%
\end{pgfscope}%
\begin{pgfscope}%
\pgfsys@transformshift{1.703690in}{3.220049in}%
\pgfsys@useobject{currentmarker}{}%
\end{pgfscope}%
\begin{pgfscope}%
\pgfsys@transformshift{1.717084in}{3.246513in}%
\pgfsys@useobject{currentmarker}{}%
\end{pgfscope}%
\begin{pgfscope}%
\pgfsys@transformshift{1.730479in}{3.291962in}%
\pgfsys@useobject{currentmarker}{}%
\end{pgfscope}%
\begin{pgfscope}%
\pgfsys@transformshift{1.743873in}{3.342661in}%
\pgfsys@useobject{currentmarker}{}%
\end{pgfscope}%
\begin{pgfscope}%
\pgfsys@transformshift{1.757267in}{3.378121in}%
\pgfsys@useobject{currentmarker}{}%
\end{pgfscope}%
\begin{pgfscope}%
\pgfsys@transformshift{1.770661in}{3.382727in}%
\pgfsys@useobject{currentmarker}{}%
\end{pgfscope}%
\begin{pgfscope}%
\pgfsys@transformshift{1.784056in}{3.382727in}%
\pgfsys@useobject{currentmarker}{}%
\end{pgfscope}%
\begin{pgfscope}%
\pgfsys@transformshift{1.789942in}{3.382727in}%
\pgfsys@useobject{currentmarker}{}%
\end{pgfscope}%
\begin{pgfscope}%
\pgfsys@transformshift{1.789942in}{3.382727in}%
\pgfsys@useobject{currentmarker}{}%
\end{pgfscope}%
\begin{pgfscope}%
\pgfsys@transformshift{1.789942in}{3.382727in}%
\pgfsys@useobject{currentmarker}{}%
\end{pgfscope}%
\begin{pgfscope}%
\pgfsys@transformshift{1.793487in}{3.353378in}%
\pgfsys@useobject{currentmarker}{}%
\end{pgfscope}%
\begin{pgfscope}%
\pgfsys@transformshift{1.800577in}{3.276970in}%
\pgfsys@useobject{currentmarker}{}%
\end{pgfscope}%
\begin{pgfscope}%
\pgfsys@transformshift{1.807667in}{3.276970in}%
\pgfsys@useobject{currentmarker}{}%
\end{pgfscope}%
\begin{pgfscope}%
\pgfsys@transformshift{1.818289in}{3.276970in}%
\pgfsys@useobject{currentmarker}{}%
\end{pgfscope}%
\begin{pgfscope}%
\pgfsys@transformshift{1.831044in}{3.276970in}%
\pgfsys@useobject{currentmarker}{}%
\end{pgfscope}%
\begin{pgfscope}%
\pgfsys@transformshift{1.843798in}{3.276970in}%
\pgfsys@useobject{currentmarker}{}%
\end{pgfscope}%
\begin{pgfscope}%
\pgfsys@transformshift{1.856553in}{3.276970in}%
\pgfsys@useobject{currentmarker}{}%
\end{pgfscope}%
\begin{pgfscope}%
\pgfsys@transformshift{1.869308in}{3.276970in}%
\pgfsys@useobject{currentmarker}{}%
\end{pgfscope}%
\begin{pgfscope}%
\pgfsys@transformshift{1.882062in}{3.164181in}%
\pgfsys@useobject{currentmarker}{}%
\end{pgfscope}%
\begin{pgfscope}%
\pgfsys@transformshift{1.894817in}{3.040402in}%
\pgfsys@useobject{currentmarker}{}%
\end{pgfscope}%
\begin{pgfscope}%
\pgfsys@transformshift{1.907572in}{2.925136in}%
\pgfsys@useobject{currentmarker}{}%
\end{pgfscope}%
\begin{pgfscope}%
\pgfsys@transformshift{1.920326in}{2.840606in}%
\pgfsys@useobject{currentmarker}{}%
\end{pgfscope}%
\begin{pgfscope}%
\pgfsys@transformshift{1.933081in}{2.796143in}%
\pgfsys@useobject{currentmarker}{}%
\end{pgfscope}%
\begin{pgfscope}%
\pgfsys@transformshift{1.945836in}{2.785174in}%
\pgfsys@useobject{currentmarker}{}%
\end{pgfscope}%
\begin{pgfscope}%
\pgfsys@transformshift{1.961644in}{2.785174in}%
\pgfsys@useobject{currentmarker}{}%
\end{pgfscope}%
\begin{pgfscope}%
\pgfsys@transformshift{1.977452in}{2.770519in}%
\pgfsys@useobject{currentmarker}{}%
\end{pgfscope}%
\begin{pgfscope}%
\pgfsys@transformshift{1.993261in}{2.721257in}%
\pgfsys@useobject{currentmarker}{}%
\end{pgfscope}%
\begin{pgfscope}%
\pgfsys@transformshift{2.009069in}{2.632074in}%
\pgfsys@useobject{currentmarker}{}%
\end{pgfscope}%
\begin{pgfscope}%
\pgfsys@transformshift{2.024878in}{2.512288in}%
\pgfsys@useobject{currentmarker}{}%
\end{pgfscope}%
\begin{pgfscope}%
\pgfsys@transformshift{2.040686in}{2.379127in}%
\pgfsys@useobject{currentmarker}{}%
\end{pgfscope}%
\begin{pgfscope}%
\pgfsys@transformshift{2.056494in}{2.386089in}%
\pgfsys@useobject{currentmarker}{}%
\end{pgfscope}%
\begin{pgfscope}%
\pgfsys@transformshift{2.072303in}{2.468271in}%
\pgfsys@useobject{currentmarker}{}%
\end{pgfscope}%
\begin{pgfscope}%
\pgfsys@transformshift{2.088111in}{2.544879in}%
\pgfsys@useobject{currentmarker}{}%
\end{pgfscope}%
\begin{pgfscope}%
\pgfsys@transformshift{2.103919in}{2.588230in}%
\pgfsys@useobject{currentmarker}{}%
\end{pgfscope}%
\begin{pgfscope}%
\pgfsys@transformshift{2.119728in}{2.588230in}%
\pgfsys@useobject{currentmarker}{}%
\end{pgfscope}%
\begin{pgfscope}%
\pgfsys@transformshift{2.135536in}{2.588230in}%
\pgfsys@useobject{currentmarker}{}%
\end{pgfscope}%
\begin{pgfscope}%
\pgfsys@transformshift{2.151345in}{2.588230in}%
\pgfsys@useobject{currentmarker}{}%
\end{pgfscope}%
\begin{pgfscope}%
\pgfsys@transformshift{2.167153in}{2.588230in}%
\pgfsys@useobject{currentmarker}{}%
\end{pgfscope}%
\begin{pgfscope}%
\pgfsys@transformshift{2.182961in}{2.582716in}%
\pgfsys@useobject{currentmarker}{}%
\end{pgfscope}%
\begin{pgfscope}%
\pgfsys@transformshift{2.198770in}{2.526002in}%
\pgfsys@useobject{currentmarker}{}%
\end{pgfscope}%
\begin{pgfscope}%
\pgfsys@transformshift{2.214578in}{2.427961in}%
\pgfsys@useobject{currentmarker}{}%
\end{pgfscope}%
\begin{pgfscope}%
\pgfsys@transformshift{2.230386in}{2.427961in}%
\pgfsys@useobject{currentmarker}{}%
\end{pgfscope}%
\begin{pgfscope}%
\pgfsys@transformshift{2.246195in}{2.427961in}%
\pgfsys@useobject{currentmarker}{}%
\end{pgfscope}%
\begin{pgfscope}%
\pgfsys@transformshift{2.262003in}{2.427961in}%
\pgfsys@useobject{currentmarker}{}%
\end{pgfscope}%
\begin{pgfscope}%
\pgfsys@transformshift{2.277812in}{2.427961in}%
\pgfsys@useobject{currentmarker}{}%
\end{pgfscope}%
\begin{pgfscope}%
\pgfsys@transformshift{2.293620in}{2.414324in}%
\pgfsys@useobject{currentmarker}{}%
\end{pgfscope}%
\begin{pgfscope}%
\pgfsys@transformshift{2.309428in}{2.309442in}%
\pgfsys@useobject{currentmarker}{}%
\end{pgfscope}%
\begin{pgfscope}%
\pgfsys@transformshift{2.325237in}{2.309442in}%
\pgfsys@useobject{currentmarker}{}%
\end{pgfscope}%
\begin{pgfscope}%
\pgfsys@transformshift{2.341045in}{2.306710in}%
\pgfsys@useobject{currentmarker}{}%
\end{pgfscope}%
\begin{pgfscope}%
\pgfsys@transformshift{2.356853in}{2.236838in}%
\pgfsys@useobject{currentmarker}{}%
\end{pgfscope}%
\begin{pgfscope}%
\pgfsys@transformshift{2.376266in}{2.216538in}%
\pgfsys@useobject{currentmarker}{}%
\end{pgfscope}%
\begin{pgfscope}%
\pgfsys@transformshift{2.384195in}{2.216538in}%
\pgfsys@useobject{currentmarker}{}%
\end{pgfscope}%
\begin{pgfscope}%
\pgfsys@transformshift{2.384195in}{2.216538in}%
\pgfsys@useobject{currentmarker}{}%
\end{pgfscope}%
\begin{pgfscope}%
\pgfsys@transformshift{2.384195in}{2.146332in}%
\pgfsys@useobject{currentmarker}{}%
\end{pgfscope}%
\begin{pgfscope}%
\pgfsys@transformshift{2.389315in}{2.124148in}%
\pgfsys@useobject{currentmarker}{}%
\end{pgfscope}%
\begin{pgfscope}%
\pgfsys@transformshift{2.399554in}{2.069703in}%
\pgfsys@useobject{currentmarker}{}%
\end{pgfscope}%
\begin{pgfscope}%
\pgfsys@transformshift{2.409792in}{1.983987in}%
\pgfsys@useobject{currentmarker}{}%
\end{pgfscope}%
\begin{pgfscope}%
\pgfsys@transformshift{2.427437in}{1.937664in}%
\pgfsys@useobject{currentmarker}{}%
\end{pgfscope}%
\begin{pgfscope}%
\pgfsys@transformshift{2.445081in}{1.937664in}%
\pgfsys@useobject{currentmarker}{}%
\end{pgfscope}%
\begin{pgfscope}%
\pgfsys@transformshift{2.462726in}{1.845611in}%
\pgfsys@useobject{currentmarker}{}%
\end{pgfscope}%
\begin{pgfscope}%
\pgfsys@transformshift{2.480371in}{1.845611in}%
\pgfsys@useobject{currentmarker}{}%
\end{pgfscope}%
\begin{pgfscope}%
\pgfsys@transformshift{2.498015in}{1.845611in}%
\pgfsys@useobject{currentmarker}{}%
\end{pgfscope}%
\begin{pgfscope}%
\pgfsys@transformshift{2.515660in}{1.845611in}%
\pgfsys@useobject{currentmarker}{}%
\end{pgfscope}%
\begin{pgfscope}%
\pgfsys@transformshift{2.532758in}{1.845611in}%
\pgfsys@useobject{currentmarker}{}%
\end{pgfscope}%
\begin{pgfscope}%
\pgfsys@transformshift{2.532758in}{1.833624in}%
\pgfsys@useobject{currentmarker}{}%
\end{pgfscope}%
\begin{pgfscope}%
\pgfsys@transformshift{2.532758in}{1.791487in}%
\pgfsys@useobject{currentmarker}{}%
\end{pgfscope}%
\begin{pgfscope}%
\pgfsys@transformshift{2.533532in}{1.714359in}%
\pgfsys@useobject{currentmarker}{}%
\end{pgfscope}%
\begin{pgfscope}%
\pgfsys@transformshift{2.535078in}{1.620870in}%
\pgfsys@useobject{currentmarker}{}%
\end{pgfscope}%
\begin{pgfscope}%
\pgfsys@transformshift{2.536625in}{1.526674in}%
\pgfsys@useobject{currentmarker}{}%
\end{pgfscope}%
\begin{pgfscope}%
\pgfsys@transformshift{2.538172in}{1.442410in}%
\pgfsys@useobject{currentmarker}{}%
\end{pgfscope}%
\begin{pgfscope}%
\pgfsys@transformshift{2.540260in}{1.377298in}%
\pgfsys@useobject{currentmarker}{}%
\end{pgfscope}%
\begin{pgfscope}%
\pgfsys@transformshift{2.543158in}{1.376938in}%
\pgfsys@useobject{currentmarker}{}%
\end{pgfscope}%
\begin{pgfscope}%
\pgfsys@transformshift{2.547673in}{1.376938in}%
\pgfsys@useobject{currentmarker}{}%
\end{pgfscope}%
\begin{pgfscope}%
\pgfsys@transformshift{2.555520in}{1.376938in}%
\pgfsys@useobject{currentmarker}{}%
\end{pgfscope}%
\begin{pgfscope}%
\pgfsys@transformshift{2.569439in}{1.376938in}%
\pgfsys@useobject{currentmarker}{}%
\end{pgfscope}%
\begin{pgfscope}%
\pgfsys@transformshift{2.583357in}{1.346384in}%
\pgfsys@useobject{currentmarker}{}%
\end{pgfscope}%
\begin{pgfscope}%
\pgfsys@transformshift{2.597276in}{1.473417in}%
\pgfsys@useobject{currentmarker}{}%
\end{pgfscope}%
\begin{pgfscope}%
\pgfsys@transformshift{2.611194in}{1.719457in}%
\pgfsys@useobject{currentmarker}{}%
\end{pgfscope}%
\begin{pgfscope}%
\pgfsys@transformshift{2.625113in}{1.963368in}%
\pgfsys@useobject{currentmarker}{}%
\end{pgfscope}%
\begin{pgfscope}%
\pgfsys@transformshift{2.639031in}{2.170280in}%
\pgfsys@useobject{currentmarker}{}%
\end{pgfscope}%
\begin{pgfscope}%
\pgfsys@transformshift{2.652950in}{2.332651in}%
\pgfsys@useobject{currentmarker}{}%
\end{pgfscope}%
\begin{pgfscope}%
\pgfsys@transformshift{2.666868in}{2.446764in}%
\pgfsys@useobject{currentmarker}{}%
\end{pgfscope}%
\begin{pgfscope}%
\pgfsys@transformshift{2.680787in}{2.514174in}%
\pgfsys@useobject{currentmarker}{}%
\end{pgfscope}%
\begin{pgfscope}%
\pgfsys@transformshift{2.681322in}{2.545622in}%
\pgfsys@useobject{currentmarker}{}%
\end{pgfscope}%
\begin{pgfscope}%
\pgfsys@transformshift{2.681322in}{2.556676in}%
\pgfsys@useobject{currentmarker}{}%
\end{pgfscope}%
\begin{pgfscope}%
\pgfsys@transformshift{2.681322in}{2.558887in}%
\pgfsys@useobject{currentmarker}{}%
\end{pgfscope}%
\begin{pgfscope}%
\pgfsys@transformshift{2.681618in}{2.558887in}%
\pgfsys@useobject{currentmarker}{}%
\end{pgfscope}%
\begin{pgfscope}%
\pgfsys@transformshift{2.681618in}{2.558887in}%
\pgfsys@useobject{currentmarker}{}%
\end{pgfscope}%
\begin{pgfscope}%
\pgfsys@transformshift{2.681914in}{2.558887in}%
\pgfsys@useobject{currentmarker}{}%
\end{pgfscope}%
\begin{pgfscope}%
\pgfsys@transformshift{2.682507in}{2.558887in}%
\pgfsys@useobject{currentmarker}{}%
\end{pgfscope}%
\begin{pgfscope}%
\pgfsys@transformshift{2.683907in}{2.558432in}%
\pgfsys@useobject{currentmarker}{}%
\end{pgfscope}%
\begin{pgfscope}%
\pgfsys@transformshift{2.687292in}{2.556989in}%
\pgfsys@useobject{currentmarker}{}%
\end{pgfscope}%
\begin{pgfscope}%
\pgfsys@transformshift{2.690676in}{2.552616in}%
\pgfsys@useobject{currentmarker}{}%
\end{pgfscope}%
\begin{pgfscope}%
\pgfsys@transformshift{2.694061in}{2.545622in}%
\pgfsys@useobject{currentmarker}{}%
\end{pgfscope}%
\begin{pgfscope}%
\pgfsys@transformshift{2.698509in}{2.545622in}%
\pgfsys@useobject{currentmarker}{}%
\end{pgfscope}%
\begin{pgfscope}%
\pgfsys@transformshift{2.702958in}{2.545622in}%
\pgfsys@useobject{currentmarker}{}%
\end{pgfscope}%
\begin{pgfscope}%
\pgfsys@transformshift{2.708728in}{2.545622in}%
\pgfsys@useobject{currentmarker}{}%
\end{pgfscope}%
\begin{pgfscope}%
\pgfsys@transformshift{2.714499in}{2.545622in}%
\pgfsys@useobject{currentmarker}{}%
\end{pgfscope}%
\begin{pgfscope}%
\pgfsys@transformshift{2.722726in}{2.545622in}%
\pgfsys@useobject{currentmarker}{}%
\end{pgfscope}%
\begin{pgfscope}%
\pgfsys@transformshift{2.733938in}{2.535399in}%
\pgfsys@useobject{currentmarker}{}%
\end{pgfscope}%
\begin{pgfscope}%
\pgfsys@transformshift{2.745150in}{2.492305in}%
\pgfsys@useobject{currentmarker}{}%
\end{pgfscope}%
\begin{pgfscope}%
\pgfsys@transformshift{2.756363in}{2.407659in}%
\pgfsys@useobject{currentmarker}{}%
\end{pgfscope}%
\begin{pgfscope}%
\pgfsys@transformshift{2.767575in}{2.266354in}%
\pgfsys@useobject{currentmarker}{}%
\end{pgfscope}%
\begin{pgfscope}%
\pgfsys@transformshift{2.778788in}{2.410078in}%
\pgfsys@useobject{currentmarker}{}%
\end{pgfscope}%
\begin{pgfscope}%
\pgfsys@transformshift{2.790000in}{2.653971in}%
\pgfsys@useobject{currentmarker}{}%
\end{pgfscope}%
\begin{pgfscope}%
\pgfsys@transformshift{2.801213in}{2.839729in}%
\pgfsys@useobject{currentmarker}{}%
\end{pgfscope}%
\begin{pgfscope}%
\pgfsys@transformshift{2.812425in}{2.967625in}%
\pgfsys@useobject{currentmarker}{}%
\end{pgfscope}%
\begin{pgfscope}%
\pgfsys@transformshift{2.823637in}{3.043452in}%
\pgfsys@useobject{currentmarker}{}%
\end{pgfscope}%
\begin{pgfscope}%
\pgfsys@transformshift{2.829885in}{3.079142in}%
\pgfsys@useobject{currentmarker}{}%
\end{pgfscope}%
\begin{pgfscope}%
\pgfsys@transformshift{2.829885in}{3.091955in}%
\pgfsys@useobject{currentmarker}{}%
\end{pgfscope}%
\begin{pgfscope}%
\pgfsys@transformshift{2.829885in}{3.093800in}%
\pgfsys@useobject{currentmarker}{}%
\end{pgfscope}%
\begin{pgfscope}%
\pgfsys@transformshift{2.832783in}{3.093800in}%
\pgfsys@useobject{currentmarker}{}%
\end{pgfscope}%
\begin{pgfscope}%
\pgfsys@transformshift{2.838578in}{3.093800in}%
\pgfsys@useobject{currentmarker}{}%
\end{pgfscope}%
\begin{pgfscope}%
\pgfsys@transformshift{2.842816in}{3.093800in}%
\pgfsys@useobject{currentmarker}{}%
\end{pgfscope}%
\begin{pgfscope}%
\pgfsys@transformshift{2.847054in}{3.093800in}%
\pgfsys@useobject{currentmarker}{}%
\end{pgfscope}%
\begin{pgfscope}%
\pgfsys@transformshift{2.854369in}{3.090102in}%
\pgfsys@useobject{currentmarker}{}%
\end{pgfscope}%
\begin{pgfscope}%
\pgfsys@transformshift{2.861684in}{3.081996in}%
\pgfsys@useobject{currentmarker}{}%
\end{pgfscope}%
\begin{pgfscope}%
\pgfsys@transformshift{2.873089in}{3.063790in}%
\pgfsys@useobject{currentmarker}{}%
\end{pgfscope}%
\begin{pgfscope}%
\pgfsys@transformshift{2.888961in}{3.063790in}%
\pgfsys@useobject{currentmarker}{}%
\end{pgfscope}%
\begin{pgfscope}%
\pgfsys@transformshift{2.904834in}{3.063790in}%
\pgfsys@useobject{currentmarker}{}%
\end{pgfscope}%
\begin{pgfscope}%
\pgfsys@transformshift{2.920706in}{3.063790in}%
\pgfsys@useobject{currentmarker}{}%
\end{pgfscope}%
\begin{pgfscope}%
\pgfsys@transformshift{2.936578in}{3.063790in}%
\pgfsys@useobject{currentmarker}{}%
\end{pgfscope}%
\begin{pgfscope}%
\pgfsys@transformshift{2.952451in}{3.063790in}%
\pgfsys@useobject{currentmarker}{}%
\end{pgfscope}%
\begin{pgfscope}%
\pgfsys@transformshift{2.968323in}{3.063790in}%
\pgfsys@useobject{currentmarker}{}%
\end{pgfscope}%
\begin{pgfscope}%
\pgfsys@transformshift{2.978448in}{3.021295in}%
\pgfsys@useobject{currentmarker}{}%
\end{pgfscope}%
\begin{pgfscope}%
\pgfsys@transformshift{2.978448in}{2.940056in}%
\pgfsys@useobject{currentmarker}{}%
\end{pgfscope}%
\begin{pgfscope}%
\pgfsys@transformshift{2.978448in}{2.808562in}%
\pgfsys@useobject{currentmarker}{}%
\end{pgfscope}%
\begin{pgfscope}%
\pgfsys@transformshift{2.983199in}{2.615387in}%
\pgfsys@useobject{currentmarker}{}%
\end{pgfscope}%
\begin{pgfscope}%
\pgfsys@transformshift{2.992700in}{2.374399in}%
\pgfsys@useobject{currentmarker}{}%
\end{pgfscope}%
\begin{pgfscope}%
\pgfsys@transformshift{3.002202in}{2.132118in}%
\pgfsys@useobject{currentmarker}{}%
\end{pgfscope}%
\begin{pgfscope}%
\pgfsys@transformshift{3.018558in}{1.928896in}%
\pgfsys@useobject{currentmarker}{}%
\end{pgfscope}%
\begin{pgfscope}%
\pgfsys@transformshift{3.034914in}{1.868682in}%
\pgfsys@useobject{currentmarker}{}%
\end{pgfscope}%
\begin{pgfscope}%
\pgfsys@transformshift{3.051270in}{1.874609in}%
\pgfsys@useobject{currentmarker}{}%
\end{pgfscope}%
\begin{pgfscope}%
\pgfsys@transformshift{3.067626in}{1.877727in}%
\pgfsys@useobject{currentmarker}{}%
\end{pgfscope}%
\begin{pgfscope}%
\pgfsys@transformshift{3.083982in}{1.877727in}%
\pgfsys@useobject{currentmarker}{}%
\end{pgfscope}%
\begin{pgfscope}%
\pgfsys@transformshift{3.100338in}{1.877727in}%
\pgfsys@useobject{currentmarker}{}%
\end{pgfscope}%
\begin{pgfscope}%
\pgfsys@transformshift{3.116694in}{1.877727in}%
\pgfsys@useobject{currentmarker}{}%
\end{pgfscope}%
\begin{pgfscope}%
\pgfsys@transformshift{3.127011in}{1.877727in}%
\pgfsys@useobject{currentmarker}{}%
\end{pgfscope}%
\begin{pgfscope}%
\pgfsys@transformshift{3.127011in}{1.868079in}%
\pgfsys@useobject{currentmarker}{}%
\end{pgfscope}%
\begin{pgfscope}%
\pgfsys@transformshift{3.127011in}{1.867403in}%
\pgfsys@useobject{currentmarker}{}%
\end{pgfscope}%
\begin{pgfscope}%
\pgfsys@transformshift{3.133141in}{1.867403in}%
\pgfsys@useobject{currentmarker}{}%
\end{pgfscope}%
\begin{pgfscope}%
\pgfsys@transformshift{3.145402in}{1.867403in}%
\pgfsys@useobject{currentmarker}{}%
\end{pgfscope}%
\begin{pgfscope}%
\pgfsys@transformshift{3.157663in}{1.867403in}%
\pgfsys@useobject{currentmarker}{}%
\end{pgfscope}%
\begin{pgfscope}%
\pgfsys@transformshift{3.177825in}{1.867311in}%
\pgfsys@useobject{currentmarker}{}%
\end{pgfscope}%
\begin{pgfscope}%
\pgfsys@transformshift{3.197987in}{1.866902in}%
\pgfsys@useobject{currentmarker}{}%
\end{pgfscope}%
\begin{pgfscope}%
\pgfsys@transformshift{3.218149in}{1.832883in}%
\pgfsys@useobject{currentmarker}{}%
\end{pgfscope}%
\begin{pgfscope}%
\pgfsys@transformshift{3.238311in}{1.761489in}%
\pgfsys@useobject{currentmarker}{}%
\end{pgfscope}%
\begin{pgfscope}%
\pgfsys@transformshift{3.258473in}{1.662006in}%
\pgfsys@useobject{currentmarker}{}%
\end{pgfscope}%
\begin{pgfscope}%
\pgfsys@transformshift{3.278636in}{1.565533in}%
\pgfsys@useobject{currentmarker}{}%
\end{pgfscope}%
\begin{pgfscope}%
\pgfsys@transformshift{3.298798in}{1.497255in}%
\pgfsys@useobject{currentmarker}{}%
\end{pgfscope}%
\begin{pgfscope}%
\pgfsys@transformshift{3.318960in}{1.495516in}%
\pgfsys@useobject{currentmarker}{}%
\end{pgfscope}%
\begin{pgfscope}%
\pgfsys@transformshift{3.339122in}{1.495516in}%
\pgfsys@useobject{currentmarker}{}%
\end{pgfscope}%
\begin{pgfscope}%
\pgfsys@transformshift{3.359284in}{1.490558in}%
\pgfsys@useobject{currentmarker}{}%
\end{pgfscope}%
\begin{pgfscope}%
\pgfsys@transformshift{3.384093in}{1.461402in}%
\pgfsys@useobject{currentmarker}{}%
\end{pgfscope}%
\begin{pgfscope}%
\pgfsys@transformshift{3.408901in}{1.461402in}%
\pgfsys@useobject{currentmarker}{}%
\end{pgfscope}%
\begin{pgfscope}%
\pgfsys@transformshift{3.433710in}{1.461402in}%
\pgfsys@useobject{currentmarker}{}%
\end{pgfscope}%
\begin{pgfscope}%
\pgfsys@transformshift{3.458518in}{1.450486in}%
\pgfsys@useobject{currentmarker}{}%
\end{pgfscope}%
\begin{pgfscope}%
\pgfsys@transformshift{3.483327in}{1.450486in}%
\pgfsys@useobject{currentmarker}{}%
\end{pgfscope}%
\begin{pgfscope}%
\pgfsys@transformshift{3.508136in}{1.411020in}%
\pgfsys@useobject{currentmarker}{}%
\end{pgfscope}%
\begin{pgfscope}%
\pgfsys@transformshift{3.532944in}{1.314609in}%
\pgfsys@useobject{currentmarker}{}%
\end{pgfscope}%
\begin{pgfscope}%
\pgfsys@transformshift{3.562994in}{1.282780in}%
\pgfsys@useobject{currentmarker}{}%
\end{pgfscope}%
\begin{pgfscope}%
\pgfsys@transformshift{3.593043in}{1.282780in}%
\pgfsys@useobject{currentmarker}{}%
\end{pgfscope}%
\begin{pgfscope}%
\pgfsys@transformshift{3.623092in}{1.282780in}%
\pgfsys@useobject{currentmarker}{}%
\end{pgfscope}%
\begin{pgfscope}%
\pgfsys@transformshift{3.653142in}{1.278122in}%
\pgfsys@useobject{currentmarker}{}%
\end{pgfscope}%
\begin{pgfscope}%
\pgfsys@transformshift{3.683191in}{1.243068in}%
\pgfsys@useobject{currentmarker}{}%
\end{pgfscope}%
\begin{pgfscope}%
\pgfsys@transformshift{3.713241in}{1.220833in}%
\pgfsys@useobject{currentmarker}{}%
\end{pgfscope}%
\begin{pgfscope}%
\pgfsys@transformshift{3.721264in}{1.197151in}%
\pgfsys@useobject{currentmarker}{}%
\end{pgfscope}%
\begin{pgfscope}%
\pgfsys@transformshift{3.721264in}{1.141065in}%
\pgfsys@useobject{currentmarker}{}%
\end{pgfscope}%
\begin{pgfscope}%
\pgfsys@transformshift{3.721264in}{1.089330in}%
\pgfsys@useobject{currentmarker}{}%
\end{pgfscope}%
\begin{pgfscope}%
\pgfsys@transformshift{3.734411in}{1.059774in}%
\pgfsys@useobject{currentmarker}{}%
\end{pgfscope}%
\begin{pgfscope}%
\pgfsys@transformshift{3.760705in}{1.055327in}%
\pgfsys@useobject{currentmarker}{}%
\end{pgfscope}%
\begin{pgfscope}%
\pgfsys@transformshift{3.787000in}{1.015734in}%
\pgfsys@useobject{currentmarker}{}%
\end{pgfscope}%
\begin{pgfscope}%
\pgfsys@transformshift{3.828820in}{0.965427in}%
\pgfsys@useobject{currentmarker}{}%
\end{pgfscope}%
\begin{pgfscope}%
\pgfsys@transformshift{3.869827in}{0.913688in}%
\pgfsys@useobject{currentmarker}{}%
\end{pgfscope}%
\begin{pgfscope}%
\pgfsys@transformshift{3.869827in}{0.890400in}%
\pgfsys@useobject{currentmarker}{}%
\end{pgfscope}%
\begin{pgfscope}%
\pgfsys@transformshift{3.869827in}{0.884126in}%
\pgfsys@useobject{currentmarker}{}%
\end{pgfscope}%
\begin{pgfscope}%
\pgfsys@transformshift{3.871582in}{0.880218in}%
\pgfsys@useobject{currentmarker}{}%
\end{pgfscope}%
\begin{pgfscope}%
\pgfsys@transformshift{3.875091in}{0.860825in}%
\pgfsys@useobject{currentmarker}{}%
\end{pgfscope}%
\begin{pgfscope}%
\pgfsys@transformshift{3.877759in}{0.843604in}%
\pgfsys@useobject{currentmarker}{}%
\end{pgfscope}%
\begin{pgfscope}%
\pgfsys@transformshift{3.880966in}{0.832464in}%
\pgfsys@useobject{currentmarker}{}%
\end{pgfscope}%
\begin{pgfscope}%
\pgfsys@transformshift{3.885349in}{0.815453in}%
\pgfsys@useobject{currentmarker}{}%
\end{pgfscope}%
\begin{pgfscope}%
\pgfsys@transformshift{3.890659in}{0.799534in}%
\pgfsys@useobject{currentmarker}{}%
\end{pgfscope}%
\begin{pgfscope}%
\pgfsys@transformshift{3.898287in}{0.778882in}%
\pgfsys@useobject{currentmarker}{}%
\end{pgfscope}%
\begin{pgfscope}%
\pgfsys@transformshift{3.909283in}{0.778096in}%
\pgfsys@useobject{currentmarker}{}%
\end{pgfscope}%
\begin{pgfscope}%
\pgfsys@transformshift{3.925341in}{0.801327in}%
\pgfsys@useobject{currentmarker}{}%
\end{pgfscope}%
\begin{pgfscope}%
\pgfsys@transformshift{3.941398in}{0.895319in}%
\pgfsys@useobject{currentmarker}{}%
\end{pgfscope}%
\begin{pgfscope}%
\pgfsys@transformshift{3.960986in}{1.045231in}%
\pgfsys@useobject{currentmarker}{}%
\end{pgfscope}%
\begin{pgfscope}%
\pgfsys@transformshift{3.980574in}{1.224196in}%
\pgfsys@useobject{currentmarker}{}%
\end{pgfscope}%
\begin{pgfscope}%
\pgfsys@transformshift{4.000162in}{1.385056in}%
\pgfsys@useobject{currentmarker}{}%
\end{pgfscope}%
\begin{pgfscope}%
\pgfsys@transformshift{4.018390in}{1.511237in}%
\pgfsys@useobject{currentmarker}{}%
\end{pgfscope}%
\begin{pgfscope}%
\pgfsys@transformshift{4.018390in}{1.585214in}%
\pgfsys@useobject{currentmarker}{}%
\end{pgfscope}%
\begin{pgfscope}%
\pgfsys@transformshift{4.018390in}{1.618911in}%
\pgfsys@useobject{currentmarker}{}%
\end{pgfscope}%
\begin{pgfscope}%
\pgfsys@transformshift{4.019146in}{1.630269in}%
\pgfsys@useobject{currentmarker}{}%
\end{pgfscope}%
\begin{pgfscope}%
\pgfsys@transformshift{4.019519in}{1.632192in}%
\pgfsys@useobject{currentmarker}{}%
\end{pgfscope}%
\begin{pgfscope}%
\pgfsys@transformshift{4.019519in}{1.632194in}%
\pgfsys@useobject{currentmarker}{}%
\end{pgfscope}%
\begin{pgfscope}%
\pgfsys@transformshift{4.020325in}{1.632194in}%
\pgfsys@useobject{currentmarker}{}%
\end{pgfscope}%
\begin{pgfscope}%
\pgfsys@transformshift{4.021939in}{1.632194in}%
\pgfsys@useobject{currentmarker}{}%
\end{pgfscope}%
\begin{pgfscope}%
\pgfsys@transformshift{4.025625in}{1.632194in}%
\pgfsys@useobject{currentmarker}{}%
\end{pgfscope}%
\begin{pgfscope}%
\pgfsys@transformshift{4.033355in}{1.632194in}%
\pgfsys@useobject{currentmarker}{}%
\end{pgfscope}%
\begin{pgfscope}%
\pgfsys@transformshift{4.046277in}{1.632183in}%
\pgfsys@useobject{currentmarker}{}%
\end{pgfscope}%
\begin{pgfscope}%
\pgfsys@transformshift{4.059199in}{1.632085in}%
\pgfsys@useobject{currentmarker}{}%
\end{pgfscope}%
\begin{pgfscope}%
\pgfsys@transformshift{4.072121in}{1.631546in}%
\pgfsys@useobject{currentmarker}{}%
\end{pgfscope}%
\begin{pgfscope}%
\pgfsys@transformshift{4.085043in}{1.630269in}%
\pgfsys@useobject{currentmarker}{}%
\end{pgfscope}%
\begin{pgfscope}%
\pgfsys@transformshift{4.097965in}{1.630269in}%
\pgfsys@useobject{currentmarker}{}%
\end{pgfscope}%
\begin{pgfscope}%
\pgfsys@transformshift{4.110887in}{1.630269in}%
\pgfsys@useobject{currentmarker}{}%
\end{pgfscope}%
\begin{pgfscope}%
\pgfsys@transformshift{4.123809in}{1.776021in}%
\pgfsys@useobject{currentmarker}{}%
\end{pgfscope}%
\begin{pgfscope}%
\pgfsys@transformshift{4.136731in}{1.955070in}%
\pgfsys@useobject{currentmarker}{}%
\end{pgfscope}%
\begin{pgfscope}%
\pgfsys@transformshift{4.149653in}{2.122140in}%
\pgfsys@useobject{currentmarker}{}%
\end{pgfscope}%
\begin{pgfscope}%
\pgfsys@transformshift{4.162575in}{2.255217in}%
\pgfsys@useobject{currentmarker}{}%
\end{pgfscope}%
\begin{pgfscope}%
\pgfsys@transformshift{4.175497in}{2.345820in}%
\pgfsys@useobject{currentmarker}{}%
\end{pgfscope}%
\begin{pgfscope}%
\pgfsys@transformshift{4.188419in}{2.396016in}%
\pgfsys@useobject{currentmarker}{}%
\end{pgfscope}%
\begin{pgfscope}%
\pgfsys@transformshift{4.201341in}{2.414898in}%
\pgfsys@useobject{currentmarker}{}%
\end{pgfscope}%
\begin{pgfscope}%
\pgfsys@transformshift{4.214263in}{2.416806in}%
\pgfsys@useobject{currentmarker}{}%
\end{pgfscope}%
\begin{pgfscope}%
\pgfsys@transformshift{4.227185in}{2.419481in}%
\pgfsys@useobject{currentmarker}{}%
\end{pgfscope}%
\begin{pgfscope}%
\pgfsys@transformshift{4.240107in}{2.440022in}%
\pgfsys@useobject{currentmarker}{}%
\end{pgfscope}%
\begin{pgfscope}%
\pgfsys@transformshift{4.253029in}{2.487830in}%
\pgfsys@useobject{currentmarker}{}%
\end{pgfscope}%
\begin{pgfscope}%
\pgfsys@transformshift{4.265951in}{2.559197in}%
\pgfsys@useobject{currentmarker}{}%
\end{pgfscope}%
\begin{pgfscope}%
\pgfsys@transformshift{4.278873in}{2.637584in}%
\pgfsys@useobject{currentmarker}{}%
\end{pgfscope}%
\begin{pgfscope}%
\pgfsys@transformshift{4.291795in}{2.700196in}%
\pgfsys@useobject{currentmarker}{}%
\end{pgfscope}%
\begin{pgfscope}%
\pgfsys@transformshift{4.304717in}{2.730998in}%
\pgfsys@useobject{currentmarker}{}%
\end{pgfscope}%
\begin{pgfscope}%
\pgfsys@transformshift{4.317638in}{2.730998in}%
\pgfsys@useobject{currentmarker}{}%
\end{pgfscope}%
\begin{pgfscope}%
\pgfsys@transformshift{4.330560in}{2.730998in}%
\pgfsys@useobject{currentmarker}{}%
\end{pgfscope}%
\begin{pgfscope}%
\pgfsys@transformshift{4.343482in}{2.730998in}%
\pgfsys@useobject{currentmarker}{}%
\end{pgfscope}%
\begin{pgfscope}%
\pgfsys@transformshift{4.356404in}{2.730998in}%
\pgfsys@useobject{currentmarker}{}%
\end{pgfscope}%
\begin{pgfscope}%
\pgfsys@transformshift{4.369326in}{2.722799in}%
\pgfsys@useobject{currentmarker}{}%
\end{pgfscope}%
\begin{pgfscope}%
\pgfsys@transformshift{4.382248in}{2.676560in}%
\pgfsys@useobject{currentmarker}{}%
\end{pgfscope}%
\begin{pgfscope}%
\pgfsys@transformshift{4.395170in}{2.601380in}%
\pgfsys@useobject{currentmarker}{}%
\end{pgfscope}%
\begin{pgfscope}%
\pgfsys@transformshift{4.408092in}{2.601380in}%
\pgfsys@useobject{currentmarker}{}%
\end{pgfscope}%
\begin{pgfscope}%
\pgfsys@transformshift{4.421014in}{2.601380in}%
\pgfsys@useobject{currentmarker}{}%
\end{pgfscope}%
\begin{pgfscope}%
\pgfsys@transformshift{4.433936in}{2.601380in}%
\pgfsys@useobject{currentmarker}{}%
\end{pgfscope}%
\begin{pgfscope}%
\pgfsys@transformshift{4.446858in}{2.634477in}%
\pgfsys@useobject{currentmarker}{}%
\end{pgfscope}%
\begin{pgfscope}%
\pgfsys@transformshift{4.459780in}{2.668836in}%
\pgfsys@useobject{currentmarker}{}%
\end{pgfscope}%
\begin{pgfscope}%
\pgfsys@transformshift{4.472702in}{2.668836in}%
\pgfsys@useobject{currentmarker}{}%
\end{pgfscope}%
\begin{pgfscope}%
\pgfsys@transformshift{4.485624in}{2.668836in}%
\pgfsys@useobject{currentmarker}{}%
\end{pgfscope}%
\begin{pgfscope}%
\pgfsys@transformshift{4.498546in}{2.668836in}%
\pgfsys@useobject{currentmarker}{}%
\end{pgfscope}%
\begin{pgfscope}%
\pgfsys@transformshift{4.511468in}{2.668836in}%
\pgfsys@useobject{currentmarker}{}%
\end{pgfscope}%
\begin{pgfscope}%
\pgfsys@transformshift{4.524390in}{2.665612in}%
\pgfsys@useobject{currentmarker}{}%
\end{pgfscope}%
\begin{pgfscope}%
\pgfsys@transformshift{4.537312in}{2.624656in}%
\pgfsys@useobject{currentmarker}{}%
\end{pgfscope}%
\begin{pgfscope}%
\pgfsys@transformshift{4.550234in}{2.553730in}%
\pgfsys@useobject{currentmarker}{}%
\end{pgfscope}%
\begin{pgfscope}%
\pgfsys@transformshift{4.563156in}{2.553730in}%
\pgfsys@useobject{currentmarker}{}%
\end{pgfscope}%
\begin{pgfscope}%
\pgfsys@transformshift{4.576078in}{2.553730in}%
\pgfsys@useobject{currentmarker}{}%
\end{pgfscope}%
\begin{pgfscope}%
\pgfsys@transformshift{4.589000in}{2.553730in}%
\pgfsys@useobject{currentmarker}{}%
\end{pgfscope}%
\begin{pgfscope}%
\pgfsys@transformshift{4.601922in}{2.554330in}%
\pgfsys@useobject{currentmarker}{}%
\end{pgfscope}%
\begin{pgfscope}%
\pgfsys@transformshift{4.614844in}{2.598643in}%
\pgfsys@useobject{currentmarker}{}%
\end{pgfscope}%
\begin{pgfscope}%
\pgfsys@transformshift{4.627765in}{2.608480in}%
\pgfsys@useobject{currentmarker}{}%
\end{pgfscope}%
\begin{pgfscope}%
\pgfsys@transformshift{4.640687in}{2.608480in}%
\pgfsys@useobject{currentmarker}{}%
\end{pgfscope}%
\begin{pgfscope}%
\pgfsys@transformshift{4.653609in}{2.608480in}%
\pgfsys@useobject{currentmarker}{}%
\end{pgfscope}%
\begin{pgfscope}%
\pgfsys@transformshift{4.666531in}{2.608480in}%
\pgfsys@useobject{currentmarker}{}%
\end{pgfscope}%
\begin{pgfscope}%
\pgfsys@transformshift{4.679453in}{2.608480in}%
\pgfsys@useobject{currentmarker}{}%
\end{pgfscope}%
\begin{pgfscope}%
\pgfsys@transformshift{4.692375in}{2.581121in}%
\pgfsys@useobject{currentmarker}{}%
\end{pgfscope}%
\begin{pgfscope}%
\pgfsys@transformshift{4.705297in}{2.521571in}%
\pgfsys@useobject{currentmarker}{}%
\end{pgfscope}%
\begin{pgfscope}%
\pgfsys@transformshift{4.718219in}{2.485778in}%
\pgfsys@useobject{currentmarker}{}%
\end{pgfscope}%
\begin{pgfscope}%
\pgfsys@transformshift{4.733740in}{2.485778in}%
\pgfsys@useobject{currentmarker}{}%
\end{pgfscope}%
\begin{pgfscope}%
\pgfsys@transformshift{4.749261in}{2.485778in}%
\pgfsys@useobject{currentmarker}{}%
\end{pgfscope}%
\begin{pgfscope}%
\pgfsys@transformshift{4.764782in}{2.538340in}%
\pgfsys@useobject{currentmarker}{}%
\end{pgfscope}%
\begin{pgfscope}%
\pgfsys@transformshift{4.780303in}{2.612635in}%
\pgfsys@useobject{currentmarker}{}%
\end{pgfscope}%
\begin{pgfscope}%
\pgfsys@transformshift{4.795824in}{2.651599in}%
\pgfsys@useobject{currentmarker}{}%
\end{pgfscope}%
\begin{pgfscope}%
\pgfsys@transformshift{4.811344in}{2.651599in}%
\pgfsys@useobject{currentmarker}{}%
\end{pgfscope}%
\begin{pgfscope}%
\pgfsys@transformshift{4.826865in}{2.651599in}%
\pgfsys@useobject{currentmarker}{}%
\end{pgfscope}%
\begin{pgfscope}%
\pgfsys@transformshift{4.842386in}{2.651599in}%
\pgfsys@useobject{currentmarker}{}%
\end{pgfscope}%
\begin{pgfscope}%
\pgfsys@transformshift{4.857907in}{2.651599in}%
\pgfsys@useobject{currentmarker}{}%
\end{pgfscope}%
\begin{pgfscope}%
\pgfsys@transformshift{4.873428in}{2.638089in}%
\pgfsys@useobject{currentmarker}{}%
\end{pgfscope}%
\begin{pgfscope}%
\pgfsys@transformshift{4.888949in}{2.576188in}%
\pgfsys@useobject{currentmarker}{}%
\end{pgfscope}%
\begin{pgfscope}%
\pgfsys@transformshift{4.904470in}{2.484232in}%
\pgfsys@useobject{currentmarker}{}%
\end{pgfscope}%
\begin{pgfscope}%
\pgfsys@transformshift{4.919991in}{2.559008in}%
\pgfsys@useobject{currentmarker}{}%
\end{pgfscope}%
\begin{pgfscope}%
\pgfsys@transformshift{4.935511in}{2.684216in}%
\pgfsys@useobject{currentmarker}{}%
\end{pgfscope}%
\begin{pgfscope}%
\pgfsys@transformshift{4.951032in}{2.778601in}%
\pgfsys@useobject{currentmarker}{}%
\end{pgfscope}%
\begin{pgfscope}%
\pgfsys@transformshift{4.966553in}{2.815317in}%
\pgfsys@useobject{currentmarker}{}%
\end{pgfscope}%
\begin{pgfscope}%
\pgfsys@transformshift{4.982074in}{2.815317in}%
\pgfsys@useobject{currentmarker}{}%
\end{pgfscope}%
\begin{pgfscope}%
\pgfsys@transformshift{4.997595in}{2.815317in}%
\pgfsys@useobject{currentmarker}{}%
\end{pgfscope}%
\begin{pgfscope}%
\pgfsys@transformshift{5.013116in}{2.815317in}%
\pgfsys@useobject{currentmarker}{}%
\end{pgfscope}%
\begin{pgfscope}%
\pgfsys@transformshift{5.028637in}{2.815317in}%
\pgfsys@useobject{currentmarker}{}%
\end{pgfscope}%
\begin{pgfscope}%
\pgfsys@transformshift{5.044158in}{2.784516in}%
\pgfsys@useobject{currentmarker}{}%
\end{pgfscope}%
\begin{pgfscope}%
\pgfsys@transformshift{5.059678in}{2.693325in}%
\pgfsys@useobject{currentmarker}{}%
\end{pgfscope}%
\begin{pgfscope}%
\pgfsys@transformshift{5.075199in}{2.566181in}%
\pgfsys@useobject{currentmarker}{}%
\end{pgfscope}%
\begin{pgfscope}%
\pgfsys@transformshift{5.090720in}{2.618629in}%
\pgfsys@useobject{currentmarker}{}%
\end{pgfscope}%
\begin{pgfscope}%
\pgfsys@transformshift{5.106241in}{2.618629in}%
\pgfsys@useobject{currentmarker}{}%
\end{pgfscope}%
\begin{pgfscope}%
\pgfsys@transformshift{5.121762in}{2.618629in}%
\pgfsys@useobject{currentmarker}{}%
\end{pgfscope}%
\begin{pgfscope}%
\pgfsys@transformshift{5.137283in}{2.618629in}%
\pgfsys@useobject{currentmarker}{}%
\end{pgfscope}%
\begin{pgfscope}%
\pgfsys@transformshift{5.152804in}{2.618629in}%
\pgfsys@useobject{currentmarker}{}%
\end{pgfscope}%
\begin{pgfscope}%
\pgfsys@transformshift{5.168325in}{2.599199in}%
\pgfsys@useobject{currentmarker}{}%
\end{pgfscope}%
\begin{pgfscope}%
\pgfsys@transformshift{5.183845in}{2.599199in}%
\pgfsys@useobject{currentmarker}{}%
\end{pgfscope}%
\begin{pgfscope}%
\pgfsys@transformshift{5.199366in}{2.599199in}%
\pgfsys@useobject{currentmarker}{}%
\end{pgfscope}%
\begin{pgfscope}%
\pgfsys@transformshift{5.206896in}{2.599199in}%
\pgfsys@useobject{currentmarker}{}%
\end{pgfscope}%
\end{pgfscope}%
\begin{pgfscope}%
\pgfpathrectangle{\pgfqpoint{0.750000in}{0.500000in}}{\pgfqpoint{4.650000in}{3.020000in}}%
\pgfusepath{clip}%
\pgfsetbuttcap%
\pgfsetroundjoin%
\definecolor{currentfill}{rgb}{0.750000,0.750000,0.000000}%
\pgfsetfillcolor{currentfill}%
\pgfsetlinewidth{0.000000pt}%
\definecolor{currentstroke}{rgb}{0.750000,0.750000,0.000000}%
\pgfsetstrokecolor{currentstroke}%
\pgfsetdash{}{0pt}%
\pgfsys@defobject{currentmarker}{\pgfqpoint{-0.006944in}{-0.006944in}}{\pgfqpoint{0.006945in}{0.006945in}}{%
\pgfpathmoveto{\pgfqpoint{-0.006944in}{-0.006944in}}%
\pgfpathlineto{\pgfqpoint{0.006945in}{-0.006944in}}%
\pgfpathlineto{\pgfqpoint{0.006945in}{0.006945in}}%
\pgfpathlineto{\pgfqpoint{-0.006944in}{0.006945in}}%
\pgfpathlineto{\pgfqpoint{-0.006944in}{-0.006944in}}%
\pgfpathclose%
\pgfusepath{fill}%
}%
\begin{pgfscope}%
\pgfsys@transformshift{1.344812in}{0.637274in}%
\pgfsys@useobject{currentmarker}{}%
\end{pgfscope}%
\begin{pgfscope}%
\pgfsys@transformshift{1.344894in}{0.637275in}%
\pgfsys@useobject{currentmarker}{}%
\end{pgfscope}%
\begin{pgfscope}%
\pgfsys@transformshift{1.344977in}{0.637277in}%
\pgfsys@useobject{currentmarker}{}%
\end{pgfscope}%
\begin{pgfscope}%
\pgfsys@transformshift{1.345086in}{0.637280in}%
\pgfsys@useobject{currentmarker}{}%
\end{pgfscope}%
\begin{pgfscope}%
\pgfsys@transformshift{1.345195in}{0.637285in}%
\pgfsys@useobject{currentmarker}{}%
\end{pgfscope}%
\begin{pgfscope}%
\pgfsys@transformshift{1.345347in}{0.637291in}%
\pgfsys@useobject{currentmarker}{}%
\end{pgfscope}%
\begin{pgfscope}%
\pgfsys@transformshift{1.345498in}{0.637299in}%
\pgfsys@useobject{currentmarker}{}%
\end{pgfscope}%
\begin{pgfscope}%
\pgfsys@transformshift{1.345723in}{0.637312in}%
\pgfsys@useobject{currentmarker}{}%
\end{pgfscope}%
\begin{pgfscope}%
\pgfsys@transformshift{1.345999in}{0.637330in}%
\pgfsys@useobject{currentmarker}{}%
\end{pgfscope}%
\begin{pgfscope}%
\pgfsys@transformshift{1.346380in}{0.637359in}%
\pgfsys@useobject{currentmarker}{}%
\end{pgfscope}%
\begin{pgfscope}%
\pgfsys@transformshift{1.346908in}{0.637406in}%
\pgfsys@useobject{currentmarker}{}%
\end{pgfscope}%
\begin{pgfscope}%
\pgfsys@transformshift{1.347743in}{0.637490in}%
\pgfsys@useobject{currentmarker}{}%
\end{pgfscope}%
\begin{pgfscope}%
\pgfsys@transformshift{1.349253in}{0.637668in}%
\pgfsys@useobject{currentmarker}{}%
\end{pgfscope}%
\begin{pgfscope}%
\pgfsys@transformshift{1.351299in}{0.638040in}%
\pgfsys@useobject{currentmarker}{}%
\end{pgfscope}%
\begin{pgfscope}%
\pgfsys@transformshift{1.353344in}{0.638710in}%
\pgfsys@useobject{currentmarker}{}%
\end{pgfscope}%
\begin{pgfscope}%
\pgfsys@transformshift{1.355390in}{0.639784in}%
\pgfsys@useobject{currentmarker}{}%
\end{pgfscope}%
\begin{pgfscope}%
\pgfsys@transformshift{1.357435in}{0.641370in}%
\pgfsys@useobject{currentmarker}{}%
\end{pgfscope}%
\begin{pgfscope}%
\pgfsys@transformshift{1.359480in}{0.643553in}%
\pgfsys@useobject{currentmarker}{}%
\end{pgfscope}%
\begin{pgfscope}%
\pgfsys@transformshift{1.361526in}{0.646385in}%
\pgfsys@useobject{currentmarker}{}%
\end{pgfscope}%
\begin{pgfscope}%
\pgfsys@transformshift{1.363571in}{0.649914in}%
\pgfsys@useobject{currentmarker}{}%
\end{pgfscope}%
\begin{pgfscope}%
\pgfsys@transformshift{1.365617in}{0.654184in}%
\pgfsys@useobject{currentmarker}{}%
\end{pgfscope}%
\begin{pgfscope}%
\pgfsys@transformshift{1.367662in}{0.659232in}%
\pgfsys@useobject{currentmarker}{}%
\end{pgfscope}%
\begin{pgfscope}%
\pgfsys@transformshift{1.369708in}{0.665095in}%
\pgfsys@useobject{currentmarker}{}%
\end{pgfscope}%
\begin{pgfscope}%
\pgfsys@transformshift{1.371753in}{0.671802in}%
\pgfsys@useobject{currentmarker}{}%
\end{pgfscope}%
\begin{pgfscope}%
\pgfsys@transformshift{1.374261in}{0.679472in}%
\pgfsys@useobject{currentmarker}{}%
\end{pgfscope}%
\begin{pgfscope}%
\pgfsys@transformshift{1.376769in}{0.688212in}%
\pgfsys@useobject{currentmarker}{}%
\end{pgfscope}%
\begin{pgfscope}%
\pgfsys@transformshift{1.379830in}{0.698252in}%
\pgfsys@useobject{currentmarker}{}%
\end{pgfscope}%
\begin{pgfscope}%
\pgfsys@transformshift{1.382890in}{0.709799in}%
\pgfsys@useobject{currentmarker}{}%
\end{pgfscope}%
\begin{pgfscope}%
\pgfsys@transformshift{1.386634in}{0.723202in}%
\pgfsys@useobject{currentmarker}{}%
\end{pgfscope}%
\begin{pgfscope}%
\pgfsys@transformshift{1.390378in}{0.738646in}%
\pgfsys@useobject{currentmarker}{}%
\end{pgfscope}%
\begin{pgfscope}%
\pgfsys@transformshift{1.395057in}{0.756590in}%
\pgfsys@useobject{currentmarker}{}%
\end{pgfscope}%
\begin{pgfscope}%
\pgfsys@transformshift{1.399736in}{0.777302in}%
\pgfsys@useobject{currentmarker}{}%
\end{pgfscope}%
\begin{pgfscope}%
\pgfsys@transformshift{1.405827in}{0.801507in}%
\pgfsys@useobject{currentmarker}{}%
\end{pgfscope}%
\begin{pgfscope}%
\pgfsys@transformshift{1.411918in}{0.829652in}%
\pgfsys@useobject{currentmarker}{}%
\end{pgfscope}%
\begin{pgfscope}%
\pgfsys@transformshift{1.420549in}{0.863044in}%
\pgfsys@useobject{currentmarker}{}%
\end{pgfscope}%
\begin{pgfscope}%
\pgfsys@transformshift{1.429180in}{0.902378in}%
\pgfsys@useobject{currentmarker}{}%
\end{pgfscope}%
\begin{pgfscope}%
\pgfsys@transformshift{1.437812in}{0.948038in}%
\pgfsys@useobject{currentmarker}{}%
\end{pgfscope}%
\begin{pgfscope}%
\pgfsys@transformshift{1.446443in}{0.999653in}%
\pgfsys@useobject{currentmarker}{}%
\end{pgfscope}%
\begin{pgfscope}%
\pgfsys@transformshift{1.455074in}{1.056661in}%
\pgfsys@useobject{currentmarker}{}%
\end{pgfscope}%
\begin{pgfscope}%
\pgfsys@transformshift{1.463705in}{1.117640in}%
\pgfsys@useobject{currentmarker}{}%
\end{pgfscope}%
\begin{pgfscope}%
\pgfsys@transformshift{1.472337in}{1.181173in}%
\pgfsys@useobject{currentmarker}{}%
\end{pgfscope}%
\begin{pgfscope}%
\pgfsys@transformshift{1.482936in}{1.246044in}%
\pgfsys@useobject{currentmarker}{}%
\end{pgfscope}%
\begin{pgfscope}%
\pgfsys@transformshift{1.492816in}{1.310930in}%
\pgfsys@useobject{currentmarker}{}%
\end{pgfscope}%
\begin{pgfscope}%
\pgfsys@transformshift{1.492816in}{1.374687in}%
\pgfsys@useobject{currentmarker}{}%
\end{pgfscope}%
\begin{pgfscope}%
\pgfsys@transformshift{1.492816in}{1.436351in}%
\pgfsys@useobject{currentmarker}{}%
\end{pgfscope}%
\begin{pgfscope}%
\pgfsys@transformshift{1.494636in}{1.495218in}%
\pgfsys@useobject{currentmarker}{}%
\end{pgfscope}%
\begin{pgfscope}%
\pgfsys@transformshift{1.498276in}{1.550422in}%
\pgfsys@useobject{currentmarker}{}%
\end{pgfscope}%
\begin{pgfscope}%
\pgfsys@transformshift{1.501916in}{1.598667in}%
\pgfsys@useobject{currentmarker}{}%
\end{pgfscope}%
\begin{pgfscope}%
\pgfsys@transformshift{1.505557in}{1.638922in}%
\pgfsys@useobject{currentmarker}{}%
\end{pgfscope}%
\begin{pgfscope}%
\pgfsys@transformshift{1.510739in}{1.669212in}%
\pgfsys@useobject{currentmarker}{}%
\end{pgfscope}%
\begin{pgfscope}%
\pgfsys@transformshift{1.517027in}{1.686711in}%
\pgfsys@useobject{currentmarker}{}%
\end{pgfscope}%
\begin{pgfscope}%
\pgfsys@transformshift{1.525486in}{1.689809in}%
\pgfsys@useobject{currentmarker}{}%
\end{pgfscope}%
\begin{pgfscope}%
\pgfsys@transformshift{1.536357in}{1.681198in}%
\pgfsys@useobject{currentmarker}{}%
\end{pgfscope}%
\begin{pgfscope}%
\pgfsys@transformshift{1.547228in}{1.669187in}%
\pgfsys@useobject{currentmarker}{}%
\end{pgfscope}%
\begin{pgfscope}%
\pgfsys@transformshift{1.558099in}{1.667895in}%
\pgfsys@useobject{currentmarker}{}%
\end{pgfscope}%
\begin{pgfscope}%
\pgfsys@transformshift{1.568970in}{1.677679in}%
\pgfsys@useobject{currentmarker}{}%
\end{pgfscope}%
\begin{pgfscope}%
\pgfsys@transformshift{1.579841in}{1.699602in}%
\pgfsys@useobject{currentmarker}{}%
\end{pgfscope}%
\begin{pgfscope}%
\pgfsys@transformshift{1.590712in}{1.734866in}%
\pgfsys@useobject{currentmarker}{}%
\end{pgfscope}%
\begin{pgfscope}%
\pgfsys@transformshift{1.601583in}{1.785802in}%
\pgfsys@useobject{currentmarker}{}%
\end{pgfscope}%
\begin{pgfscope}%
\pgfsys@transformshift{1.612454in}{1.850938in}%
\pgfsys@useobject{currentmarker}{}%
\end{pgfscope}%
\begin{pgfscope}%
\pgfsys@transformshift{1.623325in}{1.928715in}%
\pgfsys@useobject{currentmarker}{}%
\end{pgfscope}%
\begin{pgfscope}%
\pgfsys@transformshift{1.636719in}{2.017725in}%
\pgfsys@useobject{currentmarker}{}%
\end{pgfscope}%
\begin{pgfscope}%
\pgfsys@transformshift{1.650113in}{2.115921in}%
\pgfsys@useobject{currentmarker}{}%
\end{pgfscope}%
\begin{pgfscope}%
\pgfsys@transformshift{1.663507in}{2.219530in}%
\pgfsys@useobject{currentmarker}{}%
\end{pgfscope}%
\begin{pgfscope}%
\pgfsys@transformshift{1.676902in}{2.323005in}%
\pgfsys@useobject{currentmarker}{}%
\end{pgfscope}%
\begin{pgfscope}%
\pgfsys@transformshift{1.690296in}{2.419291in}%
\pgfsys@useobject{currentmarker}{}%
\end{pgfscope}%
\begin{pgfscope}%
\pgfsys@transformshift{1.703690in}{2.516509in}%
\pgfsys@useobject{currentmarker}{}%
\end{pgfscope}%
\begin{pgfscope}%
\pgfsys@transformshift{1.717084in}{2.619696in}%
\pgfsys@useobject{currentmarker}{}%
\end{pgfscope}%
\begin{pgfscope}%
\pgfsys@transformshift{1.730479in}{2.728702in}%
\pgfsys@useobject{currentmarker}{}%
\end{pgfscope}%
\begin{pgfscope}%
\pgfsys@transformshift{1.743873in}{2.840100in}%
\pgfsys@useobject{currentmarker}{}%
\end{pgfscope}%
\begin{pgfscope}%
\pgfsys@transformshift{1.757267in}{2.946644in}%
\pgfsys@useobject{currentmarker}{}%
\end{pgfscope}%
\begin{pgfscope}%
\pgfsys@transformshift{1.770661in}{3.040752in}%
\pgfsys@useobject{currentmarker}{}%
\end{pgfscope}%
\begin{pgfscope}%
\pgfsys@transformshift{1.784056in}{3.117366in}%
\pgfsys@useobject{currentmarker}{}%
\end{pgfscope}%
\begin{pgfscope}%
\pgfsys@transformshift{1.789942in}{3.172505in}%
\pgfsys@useobject{currentmarker}{}%
\end{pgfscope}%
\begin{pgfscope}%
\pgfsys@transformshift{1.789942in}{3.195843in}%
\pgfsys@useobject{currentmarker}{}%
\end{pgfscope}%
\begin{pgfscope}%
\pgfsys@transformshift{1.789942in}{3.189078in}%
\pgfsys@useobject{currentmarker}{}%
\end{pgfscope}%
\begin{pgfscope}%
\pgfsys@transformshift{1.793487in}{3.157761in}%
\pgfsys@useobject{currentmarker}{}%
\end{pgfscope}%
\begin{pgfscope}%
\pgfsys@transformshift{1.800577in}{3.110207in}%
\pgfsys@useobject{currentmarker}{}%
\end{pgfscope}%
\begin{pgfscope}%
\pgfsys@transformshift{1.807667in}{3.056997in}%
\pgfsys@useobject{currentmarker}{}%
\end{pgfscope}%
\begin{pgfscope}%
\pgfsys@transformshift{1.818289in}{3.006964in}%
\pgfsys@useobject{currentmarker}{}%
\end{pgfscope}%
\begin{pgfscope}%
\pgfsys@transformshift{1.831044in}{2.964092in}%
\pgfsys@useobject{currentmarker}{}%
\end{pgfscope}%
\begin{pgfscope}%
\pgfsys@transformshift{1.843798in}{2.926711in}%
\pgfsys@useobject{currentmarker}{}%
\end{pgfscope}%
\begin{pgfscope}%
\pgfsys@transformshift{1.856553in}{2.888077in}%
\pgfsys@useobject{currentmarker}{}%
\end{pgfscope}%
\begin{pgfscope}%
\pgfsys@transformshift{1.869308in}{2.839421in}%
\pgfsys@useobject{currentmarker}{}%
\end{pgfscope}%
\begin{pgfscope}%
\pgfsys@transformshift{1.882062in}{2.774607in}%
\pgfsys@useobject{currentmarker}{}%
\end{pgfscope}%
\begin{pgfscope}%
\pgfsys@transformshift{1.894817in}{2.700163in}%
\pgfsys@useobject{currentmarker}{}%
\end{pgfscope}%
\begin{pgfscope}%
\pgfsys@transformshift{1.907572in}{2.618924in}%
\pgfsys@useobject{currentmarker}{}%
\end{pgfscope}%
\begin{pgfscope}%
\pgfsys@transformshift{1.920326in}{2.536729in}%
\pgfsys@useobject{currentmarker}{}%
\end{pgfscope}%
\begin{pgfscope}%
\pgfsys@transformshift{1.933081in}{2.462767in}%
\pgfsys@useobject{currentmarker}{}%
\end{pgfscope}%
\begin{pgfscope}%
\pgfsys@transformshift{1.945836in}{2.403518in}%
\pgfsys@useobject{currentmarker}{}%
\end{pgfscope}%
\begin{pgfscope}%
\pgfsys@transformshift{1.961644in}{2.360332in}%
\pgfsys@useobject{currentmarker}{}%
\end{pgfscope}%
\begin{pgfscope}%
\pgfsys@transformshift{1.977452in}{2.328658in}%
\pgfsys@useobject{currentmarker}{}%
\end{pgfscope}%
\begin{pgfscope}%
\pgfsys@transformshift{1.993261in}{2.299003in}%
\pgfsys@useobject{currentmarker}{}%
\end{pgfscope}%
\begin{pgfscope}%
\pgfsys@transformshift{2.009069in}{2.272451in}%
\pgfsys@useobject{currentmarker}{}%
\end{pgfscope}%
\begin{pgfscope}%
\pgfsys@transformshift{2.024878in}{2.251583in}%
\pgfsys@useobject{currentmarker}{}%
\end{pgfscope}%
\begin{pgfscope}%
\pgfsys@transformshift{2.040686in}{2.239870in}%
\pgfsys@useobject{currentmarker}{}%
\end{pgfscope}%
\begin{pgfscope}%
\pgfsys@transformshift{2.056494in}{2.240305in}%
\pgfsys@useobject{currentmarker}{}%
\end{pgfscope}%
\begin{pgfscope}%
\pgfsys@transformshift{2.072303in}{2.253601in}%
\pgfsys@useobject{currentmarker}{}%
\end{pgfscope}%
\begin{pgfscope}%
\pgfsys@transformshift{2.088111in}{2.277303in}%
\pgfsys@useobject{currentmarker}{}%
\end{pgfscope}%
\begin{pgfscope}%
\pgfsys@transformshift{2.103919in}{2.305028in}%
\pgfsys@useobject{currentmarker}{}%
\end{pgfscope}%
\begin{pgfscope}%
\pgfsys@transformshift{2.119728in}{2.330183in}%
\pgfsys@useobject{currentmarker}{}%
\end{pgfscope}%
\begin{pgfscope}%
\pgfsys@transformshift{2.135536in}{2.348284in}%
\pgfsys@useobject{currentmarker}{}%
\end{pgfscope}%
\begin{pgfscope}%
\pgfsys@transformshift{2.151345in}{2.356541in}%
\pgfsys@useobject{currentmarker}{}%
\end{pgfscope}%
\begin{pgfscope}%
\pgfsys@transformshift{2.167153in}{2.350814in}%
\pgfsys@useobject{currentmarker}{}%
\end{pgfscope}%
\begin{pgfscope}%
\pgfsys@transformshift{2.182961in}{2.329913in}%
\pgfsys@useobject{currentmarker}{}%
\end{pgfscope}%
\begin{pgfscope}%
\pgfsys@transformshift{2.198770in}{2.297670in}%
\pgfsys@useobject{currentmarker}{}%
\end{pgfscope}%
\begin{pgfscope}%
\pgfsys@transformshift{2.214578in}{2.263323in}%
\pgfsys@useobject{currentmarker}{}%
\end{pgfscope}%
\begin{pgfscope}%
\pgfsys@transformshift{2.230386in}{2.237468in}%
\pgfsys@useobject{currentmarker}{}%
\end{pgfscope}%
\begin{pgfscope}%
\pgfsys@transformshift{2.246195in}{2.226693in}%
\pgfsys@useobject{currentmarker}{}%
\end{pgfscope}%
\begin{pgfscope}%
\pgfsys@transformshift{2.262003in}{2.225722in}%
\pgfsys@useobject{currentmarker}{}%
\end{pgfscope}%
\begin{pgfscope}%
\pgfsys@transformshift{2.277812in}{2.225735in}%
\pgfsys@useobject{currentmarker}{}%
\end{pgfscope}%
\begin{pgfscope}%
\pgfsys@transformshift{2.293620in}{2.210571in}%
\pgfsys@useobject{currentmarker}{}%
\end{pgfscope}%
\begin{pgfscope}%
\pgfsys@transformshift{2.309428in}{2.197334in}%
\pgfsys@useobject{currentmarker}{}%
\end{pgfscope}%
\begin{pgfscope}%
\pgfsys@transformshift{2.325237in}{2.177098in}%
\pgfsys@useobject{currentmarker}{}%
\end{pgfscope}%
\begin{pgfscope}%
\pgfsys@transformshift{2.341045in}{2.132410in}%
\pgfsys@useobject{currentmarker}{}%
\end{pgfscope}%
\begin{pgfscope}%
\pgfsys@transformshift{2.356853in}{2.079992in}%
\pgfsys@useobject{currentmarker}{}%
\end{pgfscope}%
\begin{pgfscope}%
\pgfsys@transformshift{2.376266in}{2.031448in}%
\pgfsys@useobject{currentmarker}{}%
\end{pgfscope}%
\begin{pgfscope}%
\pgfsys@transformshift{2.384195in}{1.987959in}%
\pgfsys@useobject{currentmarker}{}%
\end{pgfscope}%
\begin{pgfscope}%
\pgfsys@transformshift{2.384195in}{1.953035in}%
\pgfsys@useobject{currentmarker}{}%
\end{pgfscope}%
\begin{pgfscope}%
\pgfsys@transformshift{2.384195in}{1.920630in}%
\pgfsys@useobject{currentmarker}{}%
\end{pgfscope}%
\begin{pgfscope}%
\pgfsys@transformshift{2.389315in}{1.891514in}%
\pgfsys@useobject{currentmarker}{}%
\end{pgfscope}%
\begin{pgfscope}%
\pgfsys@transformshift{2.399554in}{1.858998in}%
\pgfsys@useobject{currentmarker}{}%
\end{pgfscope}%
\begin{pgfscope}%
\pgfsys@transformshift{2.409792in}{1.827657in}%
\pgfsys@useobject{currentmarker}{}%
\end{pgfscope}%
\begin{pgfscope}%
\pgfsys@transformshift{2.427437in}{1.793549in}%
\pgfsys@useobject{currentmarker}{}%
\end{pgfscope}%
\begin{pgfscope}%
\pgfsys@transformshift{2.445081in}{1.751003in}%
\pgfsys@useobject{currentmarker}{}%
\end{pgfscope}%
\begin{pgfscope}%
\pgfsys@transformshift{2.462726in}{1.695411in}%
\pgfsys@useobject{currentmarker}{}%
\end{pgfscope}%
\begin{pgfscope}%
\pgfsys@transformshift{2.480371in}{1.641004in}%
\pgfsys@useobject{currentmarker}{}%
\end{pgfscope}%
\begin{pgfscope}%
\pgfsys@transformshift{2.498015in}{1.589664in}%
\pgfsys@useobject{currentmarker}{}%
\end{pgfscope}%
\begin{pgfscope}%
\pgfsys@transformshift{2.515660in}{1.543446in}%
\pgfsys@useobject{currentmarker}{}%
\end{pgfscope}%
\begin{pgfscope}%
\pgfsys@transformshift{2.532758in}{1.504564in}%
\pgfsys@useobject{currentmarker}{}%
\end{pgfscope}%
\begin{pgfscope}%
\pgfsys@transformshift{2.532758in}{1.471180in}%
\pgfsys@useobject{currentmarker}{}%
\end{pgfscope}%
\begin{pgfscope}%
\pgfsys@transformshift{2.532758in}{1.438795in}%
\pgfsys@useobject{currentmarker}{}%
\end{pgfscope}%
\begin{pgfscope}%
\pgfsys@transformshift{2.533532in}{1.404257in}%
\pgfsys@useobject{currentmarker}{}%
\end{pgfscope}%
\begin{pgfscope}%
\pgfsys@transformshift{2.535078in}{1.367883in}%
\pgfsys@useobject{currentmarker}{}%
\end{pgfscope}%
\begin{pgfscope}%
\pgfsys@transformshift{2.536625in}{1.330910in}%
\pgfsys@useobject{currentmarker}{}%
\end{pgfscope}%
\begin{pgfscope}%
\pgfsys@transformshift{2.538172in}{1.296114in}%
\pgfsys@useobject{currentmarker}{}%
\end{pgfscope}%
\begin{pgfscope}%
\pgfsys@transformshift{2.540260in}{1.267095in}%
\pgfsys@useobject{currentmarker}{}%
\end{pgfscope}%
\begin{pgfscope}%
\pgfsys@transformshift{2.543158in}{1.244281in}%
\pgfsys@useobject{currentmarker}{}%
\end{pgfscope}%
\begin{pgfscope}%
\pgfsys@transformshift{2.547673in}{1.223879in}%
\pgfsys@useobject{currentmarker}{}%
\end{pgfscope}%
\begin{pgfscope}%
\pgfsys@transformshift{2.555520in}{1.202538in}%
\pgfsys@useobject{currentmarker}{}%
\end{pgfscope}%
\begin{pgfscope}%
\pgfsys@transformshift{2.569439in}{1.185230in}%
\pgfsys@useobject{currentmarker}{}%
\end{pgfscope}%
\begin{pgfscope}%
\pgfsys@transformshift{2.583357in}{1.178178in}%
\pgfsys@useobject{currentmarker}{}%
\end{pgfscope}%
\begin{pgfscope}%
\pgfsys@transformshift{2.597276in}{1.186117in}%
\pgfsys@useobject{currentmarker}{}%
\end{pgfscope}%
\begin{pgfscope}%
\pgfsys@transformshift{2.611194in}{1.212920in}%
\pgfsys@useobject{currentmarker}{}%
\end{pgfscope}%
\begin{pgfscope}%
\pgfsys@transformshift{2.625113in}{1.258321in}%
\pgfsys@useobject{currentmarker}{}%
\end{pgfscope}%
\begin{pgfscope}%
\pgfsys@transformshift{2.639031in}{1.319091in}%
\pgfsys@useobject{currentmarker}{}%
\end{pgfscope}%
\begin{pgfscope}%
\pgfsys@transformshift{2.652950in}{1.391542in}%
\pgfsys@useobject{currentmarker}{}%
\end{pgfscope}%
\begin{pgfscope}%
\pgfsys@transformshift{2.666868in}{1.471825in}%
\pgfsys@useobject{currentmarker}{}%
\end{pgfscope}%
\begin{pgfscope}%
\pgfsys@transformshift{2.680787in}{1.556995in}%
\pgfsys@useobject{currentmarker}{}%
\end{pgfscope}%
\begin{pgfscope}%
\pgfsys@transformshift{2.681322in}{1.645693in}%
\pgfsys@useobject{currentmarker}{}%
\end{pgfscope}%
\begin{pgfscope}%
\pgfsys@transformshift{2.681322in}{1.736798in}%
\pgfsys@useobject{currentmarker}{}%
\end{pgfscope}%
\begin{pgfscope}%
\pgfsys@transformshift{2.681322in}{1.826006in}%
\pgfsys@useobject{currentmarker}{}%
\end{pgfscope}%
\begin{pgfscope}%
\pgfsys@transformshift{2.681618in}{1.913268in}%
\pgfsys@useobject{currentmarker}{}%
\end{pgfscope}%
\begin{pgfscope}%
\pgfsys@transformshift{2.681618in}{2.001114in}%
\pgfsys@useobject{currentmarker}{}%
\end{pgfscope}%
\begin{pgfscope}%
\pgfsys@transformshift{2.681914in}{2.090250in}%
\pgfsys@useobject{currentmarker}{}%
\end{pgfscope}%
\begin{pgfscope}%
\pgfsys@transformshift{2.682507in}{2.178695in}%
\pgfsys@useobject{currentmarker}{}%
\end{pgfscope}%
\begin{pgfscope}%
\pgfsys@transformshift{2.683907in}{2.259580in}%
\pgfsys@useobject{currentmarker}{}%
\end{pgfscope}%
\begin{pgfscope}%
\pgfsys@transformshift{2.687292in}{2.321629in}%
\pgfsys@useobject{currentmarker}{}%
\end{pgfscope}%
\begin{pgfscope}%
\pgfsys@transformshift{2.690676in}{2.353023in}%
\pgfsys@useobject{currentmarker}{}%
\end{pgfscope}%
\begin{pgfscope}%
\pgfsys@transformshift{2.694061in}{2.343796in}%
\pgfsys@useobject{currentmarker}{}%
\end{pgfscope}%
\begin{pgfscope}%
\pgfsys@transformshift{2.698509in}{2.290372in}%
\pgfsys@useobject{currentmarker}{}%
\end{pgfscope}%
\begin{pgfscope}%
\pgfsys@transformshift{2.702958in}{2.194364in}%
\pgfsys@useobject{currentmarker}{}%
\end{pgfscope}%
\begin{pgfscope}%
\pgfsys@transformshift{2.708728in}{2.062542in}%
\pgfsys@useobject{currentmarker}{}%
\end{pgfscope}%
\begin{pgfscope}%
\pgfsys@transformshift{2.714499in}{1.909550in}%
\pgfsys@useobject{currentmarker}{}%
\end{pgfscope}%
\begin{pgfscope}%
\pgfsys@transformshift{2.722726in}{1.761169in}%
\pgfsys@useobject{currentmarker}{}%
\end{pgfscope}%
\begin{pgfscope}%
\pgfsys@transformshift{2.733938in}{1.644690in}%
\pgfsys@useobject{currentmarker}{}%
\end{pgfscope}%
\begin{pgfscope}%
\pgfsys@transformshift{2.745150in}{1.576647in}%
\pgfsys@useobject{currentmarker}{}%
\end{pgfscope}%
\begin{pgfscope}%
\pgfsys@transformshift{2.756363in}{1.533328in}%
\pgfsys@useobject{currentmarker}{}%
\end{pgfscope}%
\begin{pgfscope}%
\pgfsys@transformshift{2.767575in}{1.514901in}%
\pgfsys@useobject{currentmarker}{}%
\end{pgfscope}%
\begin{pgfscope}%
\pgfsys@transformshift{2.778788in}{1.523883in}%
\pgfsys@useobject{currentmarker}{}%
\end{pgfscope}%
\begin{pgfscope}%
\pgfsys@transformshift{2.790000in}{1.560868in}%
\pgfsys@useobject{currentmarker}{}%
\end{pgfscope}%
\begin{pgfscope}%
\pgfsys@transformshift{2.801213in}{1.624476in}%
\pgfsys@useobject{currentmarker}{}%
\end{pgfscope}%
\begin{pgfscope}%
\pgfsys@transformshift{2.812425in}{1.712251in}%
\pgfsys@useobject{currentmarker}{}%
\end{pgfscope}%
\begin{pgfscope}%
\pgfsys@transformshift{2.823637in}{1.820243in}%
\pgfsys@useobject{currentmarker}{}%
\end{pgfscope}%
\begin{pgfscope}%
\pgfsys@transformshift{2.829885in}{1.941901in}%
\pgfsys@useobject{currentmarker}{}%
\end{pgfscope}%
\begin{pgfscope}%
\pgfsys@transformshift{2.829885in}{2.066326in}%
\pgfsys@useobject{currentmarker}{}%
\end{pgfscope}%
\begin{pgfscope}%
\pgfsys@transformshift{2.829885in}{2.181440in}%
\pgfsys@useobject{currentmarker}{}%
\end{pgfscope}%
\begin{pgfscope}%
\pgfsys@transformshift{2.832783in}{2.280511in}%
\pgfsys@useobject{currentmarker}{}%
\end{pgfscope}%
\begin{pgfscope}%
\pgfsys@transformshift{2.838578in}{2.390912in}%
\pgfsys@useobject{currentmarker}{}%
\end{pgfscope}%
\begin{pgfscope}%
\pgfsys@transformshift{2.842816in}{2.510650in}%
\pgfsys@useobject{currentmarker}{}%
\end{pgfscope}%
\begin{pgfscope}%
\pgfsys@transformshift{2.847054in}{2.626243in}%
\pgfsys@useobject{currentmarker}{}%
\end{pgfscope}%
\begin{pgfscope}%
\pgfsys@transformshift{2.854369in}{2.722139in}%
\pgfsys@useobject{currentmarker}{}%
\end{pgfscope}%
\begin{pgfscope}%
\pgfsys@transformshift{2.861684in}{2.786018in}%
\pgfsys@useobject{currentmarker}{}%
\end{pgfscope}%
\begin{pgfscope}%
\pgfsys@transformshift{2.873089in}{2.807866in}%
\pgfsys@useobject{currentmarker}{}%
\end{pgfscope}%
\begin{pgfscope}%
\pgfsys@transformshift{2.888961in}{2.781128in}%
\pgfsys@useobject{currentmarker}{}%
\end{pgfscope}%
\begin{pgfscope}%
\pgfsys@transformshift{2.904834in}{2.708182in}%
\pgfsys@useobject{currentmarker}{}%
\end{pgfscope}%
\begin{pgfscope}%
\pgfsys@transformshift{2.920706in}{2.599667in}%
\pgfsys@useobject{currentmarker}{}%
\end{pgfscope}%
\begin{pgfscope}%
\pgfsys@transformshift{2.936578in}{2.472820in}%
\pgfsys@useobject{currentmarker}{}%
\end{pgfscope}%
\begin{pgfscope}%
\pgfsys@transformshift{2.952451in}{2.346673in}%
\pgfsys@useobject{currentmarker}{}%
\end{pgfscope}%
\begin{pgfscope}%
\pgfsys@transformshift{2.968323in}{2.233946in}%
\pgfsys@useobject{currentmarker}{}%
\end{pgfscope}%
\begin{pgfscope}%
\pgfsys@transformshift{2.978448in}{2.138952in}%
\pgfsys@useobject{currentmarker}{}%
\end{pgfscope}%
\begin{pgfscope}%
\pgfsys@transformshift{2.978448in}{2.061728in}%
\pgfsys@useobject{currentmarker}{}%
\end{pgfscope}%
\begin{pgfscope}%
\pgfsys@transformshift{2.978448in}{1.994568in}%
\pgfsys@useobject{currentmarker}{}%
\end{pgfscope}%
\begin{pgfscope}%
\pgfsys@transformshift{2.983199in}{1.935746in}%
\pgfsys@useobject{currentmarker}{}%
\end{pgfscope}%
\begin{pgfscope}%
\pgfsys@transformshift{2.992700in}{1.888991in}%
\pgfsys@useobject{currentmarker}{}%
\end{pgfscope}%
\begin{pgfscope}%
\pgfsys@transformshift{3.002202in}{1.857273in}%
\pgfsys@useobject{currentmarker}{}%
\end{pgfscope}%
\begin{pgfscope}%
\pgfsys@transformshift{3.018558in}{1.840676in}%
\pgfsys@useobject{currentmarker}{}%
\end{pgfscope}%
\begin{pgfscope}%
\pgfsys@transformshift{3.034914in}{1.836913in}%
\pgfsys@useobject{currentmarker}{}%
\end{pgfscope}%
\begin{pgfscope}%
\pgfsys@transformshift{3.051270in}{1.840990in}%
\pgfsys@useobject{currentmarker}{}%
\end{pgfscope}%
\begin{pgfscope}%
\pgfsys@transformshift{3.067626in}{1.846952in}%
\pgfsys@useobject{currentmarker}{}%
\end{pgfscope}%
\begin{pgfscope}%
\pgfsys@transformshift{3.083982in}{1.850763in}%
\pgfsys@useobject{currentmarker}{}%
\end{pgfscope}%
\begin{pgfscope}%
\pgfsys@transformshift{3.100338in}{1.848338in}%
\pgfsys@useobject{currentmarker}{}%
\end{pgfscope}%
\begin{pgfscope}%
\pgfsys@transformshift{3.116694in}{1.835649in}%
\pgfsys@useobject{currentmarker}{}%
\end{pgfscope}%
\begin{pgfscope}%
\pgfsys@transformshift{3.127011in}{1.813151in}%
\pgfsys@useobject{currentmarker}{}%
\end{pgfscope}%
\begin{pgfscope}%
\pgfsys@transformshift{3.127011in}{1.780089in}%
\pgfsys@useobject{currentmarker}{}%
\end{pgfscope}%
\begin{pgfscope}%
\pgfsys@transformshift{3.127011in}{1.737547in}%
\pgfsys@useobject{currentmarker}{}%
\end{pgfscope}%
\begin{pgfscope}%
\pgfsys@transformshift{3.133141in}{1.690910in}%
\pgfsys@useobject{currentmarker}{}%
\end{pgfscope}%
\begin{pgfscope}%
\pgfsys@transformshift{3.145402in}{1.645924in}%
\pgfsys@useobject{currentmarker}{}%
\end{pgfscope}%
\begin{pgfscope}%
\pgfsys@transformshift{3.157663in}{1.606282in}%
\pgfsys@useobject{currentmarker}{}%
\end{pgfscope}%
\begin{pgfscope}%
\pgfsys@transformshift{3.177825in}{1.572584in}%
\pgfsys@useobject{currentmarker}{}%
\end{pgfscope}%
\begin{pgfscope}%
\pgfsys@transformshift{3.197987in}{1.543609in}%
\pgfsys@useobject{currentmarker}{}%
\end{pgfscope}%
\begin{pgfscope}%
\pgfsys@transformshift{3.218149in}{1.518206in}%
\pgfsys@useobject{currentmarker}{}%
\end{pgfscope}%
\begin{pgfscope}%
\pgfsys@transformshift{3.238311in}{1.496132in}%
\pgfsys@useobject{currentmarker}{}%
\end{pgfscope}%
\begin{pgfscope}%
\pgfsys@transformshift{3.258473in}{1.479508in}%
\pgfsys@useobject{currentmarker}{}%
\end{pgfscope}%
\begin{pgfscope}%
\pgfsys@transformshift{3.278636in}{1.468793in}%
\pgfsys@useobject{currentmarker}{}%
\end{pgfscope}%
\begin{pgfscope}%
\pgfsys@transformshift{3.298798in}{1.461652in}%
\pgfsys@useobject{currentmarker}{}%
\end{pgfscope}%
\begin{pgfscope}%
\pgfsys@transformshift{3.318960in}{1.449651in}%
\pgfsys@useobject{currentmarker}{}%
\end{pgfscope}%
\begin{pgfscope}%
\pgfsys@transformshift{3.339122in}{1.428625in}%
\pgfsys@useobject{currentmarker}{}%
\end{pgfscope}%
\begin{pgfscope}%
\pgfsys@transformshift{3.359284in}{1.399226in}%
\pgfsys@useobject{currentmarker}{}%
\end{pgfscope}%
\begin{pgfscope}%
\pgfsys@transformshift{3.384093in}{1.366685in}%
\pgfsys@useobject{currentmarker}{}%
\end{pgfscope}%
\begin{pgfscope}%
\pgfsys@transformshift{3.408901in}{1.337374in}%
\pgfsys@useobject{currentmarker}{}%
\end{pgfscope}%
\begin{pgfscope}%
\pgfsys@transformshift{3.433710in}{1.315167in}%
\pgfsys@useobject{currentmarker}{}%
\end{pgfscope}%
\begin{pgfscope}%
\pgfsys@transformshift{3.458518in}{1.300439in}%
\pgfsys@useobject{currentmarker}{}%
\end{pgfscope}%
\begin{pgfscope}%
\pgfsys@transformshift{3.483327in}{1.289816in}%
\pgfsys@useobject{currentmarker}{}%
\end{pgfscope}%
\begin{pgfscope}%
\pgfsys@transformshift{3.508136in}{1.274209in}%
\pgfsys@useobject{currentmarker}{}%
\end{pgfscope}%
\begin{pgfscope}%
\pgfsys@transformshift{3.532944in}{1.255755in}%
\pgfsys@useobject{currentmarker}{}%
\end{pgfscope}%
\begin{pgfscope}%
\pgfsys@transformshift{3.562994in}{1.245623in}%
\pgfsys@useobject{currentmarker}{}%
\end{pgfscope}%
\begin{pgfscope}%
\pgfsys@transformshift{3.593043in}{1.239339in}%
\pgfsys@useobject{currentmarker}{}%
\end{pgfscope}%
\begin{pgfscope}%
\pgfsys@transformshift{3.623092in}{1.231640in}%
\pgfsys@useobject{currentmarker}{}%
\end{pgfscope}%
\begin{pgfscope}%
\pgfsys@transformshift{3.653142in}{1.217836in}%
\pgfsys@useobject{currentmarker}{}%
\end{pgfscope}%
\begin{pgfscope}%
\pgfsys@transformshift{3.683191in}{1.193122in}%
\pgfsys@useobject{currentmarker}{}%
\end{pgfscope}%
\begin{pgfscope}%
\pgfsys@transformshift{3.713241in}{1.141904in}%
\pgfsys@useobject{currentmarker}{}%
\end{pgfscope}%
\begin{pgfscope}%
\pgfsys@transformshift{3.721264in}{1.088976in}%
\pgfsys@useobject{currentmarker}{}%
\end{pgfscope}%
\begin{pgfscope}%
\pgfsys@transformshift{3.721264in}{1.041419in}%
\pgfsys@useobject{currentmarker}{}%
\end{pgfscope}%
\begin{pgfscope}%
\pgfsys@transformshift{3.721264in}{1.006251in}%
\pgfsys@useobject{currentmarker}{}%
\end{pgfscope}%
\begin{pgfscope}%
\pgfsys@transformshift{3.734411in}{0.979212in}%
\pgfsys@useobject{currentmarker}{}%
\end{pgfscope}%
\begin{pgfscope}%
\pgfsys@transformshift{3.760705in}{0.954339in}%
\pgfsys@useobject{currentmarker}{}%
\end{pgfscope}%
\begin{pgfscope}%
\pgfsys@transformshift{3.787000in}{0.923575in}%
\pgfsys@useobject{currentmarker}{}%
\end{pgfscope}%
\begin{pgfscope}%
\pgfsys@transformshift{3.828820in}{0.895581in}%
\pgfsys@useobject{currentmarker}{}%
\end{pgfscope}%
\begin{pgfscope}%
\pgfsys@transformshift{3.869827in}{0.873693in}%
\pgfsys@useobject{currentmarker}{}%
\end{pgfscope}%
\begin{pgfscope}%
\pgfsys@transformshift{3.869827in}{0.859584in}%
\pgfsys@useobject{currentmarker}{}%
\end{pgfscope}%
\begin{pgfscope}%
\pgfsys@transformshift{3.869827in}{0.850169in}%
\pgfsys@useobject{currentmarker}{}%
\end{pgfscope}%
\begin{pgfscope}%
\pgfsys@transformshift{3.871582in}{0.842217in}%
\pgfsys@useobject{currentmarker}{}%
\end{pgfscope}%
\begin{pgfscope}%
\pgfsys@transformshift{3.875091in}{0.829499in}%
\pgfsys@useobject{currentmarker}{}%
\end{pgfscope}%
\begin{pgfscope}%
\pgfsys@transformshift{3.877759in}{0.812086in}%
\pgfsys@useobject{currentmarker}{}%
\end{pgfscope}%
\begin{pgfscope}%
\pgfsys@transformshift{3.880966in}{0.796247in}%
\pgfsys@useobject{currentmarker}{}%
\end{pgfscope}%
\begin{pgfscope}%
\pgfsys@transformshift{3.885349in}{0.783564in}%
\pgfsys@useobject{currentmarker}{}%
\end{pgfscope}%
\begin{pgfscope}%
\pgfsys@transformshift{3.890659in}{0.771864in}%
\pgfsys@useobject{currentmarker}{}%
\end{pgfscope}%
\begin{pgfscope}%
\pgfsys@transformshift{3.898287in}{0.761082in}%
\pgfsys@useobject{currentmarker}{}%
\end{pgfscope}%
\begin{pgfscope}%
\pgfsys@transformshift{3.909283in}{0.756600in}%
\pgfsys@useobject{currentmarker}{}%
\end{pgfscope}%
\begin{pgfscope}%
\pgfsys@transformshift{3.925341in}{0.758301in}%
\pgfsys@useobject{currentmarker}{}%
\end{pgfscope}%
\begin{pgfscope}%
\pgfsys@transformshift{3.941398in}{0.765627in}%
\pgfsys@useobject{currentmarker}{}%
\end{pgfscope}%
\begin{pgfscope}%
\pgfsys@transformshift{3.960986in}{0.782594in}%
\pgfsys@useobject{currentmarker}{}%
\end{pgfscope}%
\begin{pgfscope}%
\pgfsys@transformshift{3.980574in}{0.811251in}%
\pgfsys@useobject{currentmarker}{}%
\end{pgfscope}%
\begin{pgfscope}%
\pgfsys@transformshift{4.000162in}{0.849708in}%
\pgfsys@useobject{currentmarker}{}%
\end{pgfscope}%
\begin{pgfscope}%
\pgfsys@transformshift{4.018390in}{0.896302in}%
\pgfsys@useobject{currentmarker}{}%
\end{pgfscope}%
\begin{pgfscope}%
\pgfsys@transformshift{4.018390in}{0.948131in}%
\pgfsys@useobject{currentmarker}{}%
\end{pgfscope}%
\begin{pgfscope}%
\pgfsys@transformshift{4.018390in}{1.002101in}%
\pgfsys@useobject{currentmarker}{}%
\end{pgfscope}%
\begin{pgfscope}%
\pgfsys@transformshift{4.019146in}{1.057341in}%
\pgfsys@useobject{currentmarker}{}%
\end{pgfscope}%
\begin{pgfscope}%
\pgfsys@transformshift{4.019519in}{1.113386in}%
\pgfsys@useobject{currentmarker}{}%
\end{pgfscope}%
\begin{pgfscope}%
\pgfsys@transformshift{4.019519in}{1.168774in}%
\pgfsys@useobject{currentmarker}{}%
\end{pgfscope}%
\begin{pgfscope}%
\pgfsys@transformshift{4.020325in}{1.223538in}%
\pgfsys@useobject{currentmarker}{}%
\end{pgfscope}%
\begin{pgfscope}%
\pgfsys@transformshift{4.021939in}{1.278331in}%
\pgfsys@useobject{currentmarker}{}%
\end{pgfscope}%
\begin{pgfscope}%
\pgfsys@transformshift{4.025625in}{1.333651in}%
\pgfsys@useobject{currentmarker}{}%
\end{pgfscope}%
\begin{pgfscope}%
\pgfsys@transformshift{4.033355in}{1.389314in}%
\pgfsys@useobject{currentmarker}{}%
\end{pgfscope}%
\begin{pgfscope}%
\pgfsys@transformshift{4.046277in}{1.443158in}%
\pgfsys@useobject{currentmarker}{}%
\end{pgfscope}%
\begin{pgfscope}%
\pgfsys@transformshift{4.059199in}{1.491107in}%
\pgfsys@useobject{currentmarker}{}%
\end{pgfscope}%
\begin{pgfscope}%
\pgfsys@transformshift{4.072121in}{1.529135in}%
\pgfsys@useobject{currentmarker}{}%
\end{pgfscope}%
\begin{pgfscope}%
\pgfsys@transformshift{4.085043in}{1.554040in}%
\pgfsys@useobject{currentmarker}{}%
\end{pgfscope}%
\begin{pgfscope}%
\pgfsys@transformshift{4.097965in}{1.568862in}%
\pgfsys@useobject{currentmarker}{}%
\end{pgfscope}%
\begin{pgfscope}%
\pgfsys@transformshift{4.110887in}{1.583854in}%
\pgfsys@useobject{currentmarker}{}%
\end{pgfscope}%
\begin{pgfscope}%
\pgfsys@transformshift{4.123809in}{1.600403in}%
\pgfsys@useobject{currentmarker}{}%
\end{pgfscope}%
\begin{pgfscope}%
\pgfsys@transformshift{4.136731in}{1.623519in}%
\pgfsys@useobject{currentmarker}{}%
\end{pgfscope}%
\begin{pgfscope}%
\pgfsys@transformshift{4.149653in}{1.654971in}%
\pgfsys@useobject{currentmarker}{}%
\end{pgfscope}%
\begin{pgfscope}%
\pgfsys@transformshift{4.162575in}{1.694030in}%
\pgfsys@useobject{currentmarker}{}%
\end{pgfscope}%
\begin{pgfscope}%
\pgfsys@transformshift{4.175497in}{1.738823in}%
\pgfsys@useobject{currentmarker}{}%
\end{pgfscope}%
\begin{pgfscope}%
\pgfsys@transformshift{4.188419in}{1.787334in}%
\pgfsys@useobject{currentmarker}{}%
\end{pgfscope}%
\begin{pgfscope}%
\pgfsys@transformshift{4.201341in}{1.838243in}%
\pgfsys@useobject{currentmarker}{}%
\end{pgfscope}%
\begin{pgfscope}%
\pgfsys@transformshift{4.214263in}{1.891298in}%
\pgfsys@useobject{currentmarker}{}%
\end{pgfscope}%
\begin{pgfscope}%
\pgfsys@transformshift{4.227185in}{1.946414in}%
\pgfsys@useobject{currentmarker}{}%
\end{pgfscope}%
\begin{pgfscope}%
\pgfsys@transformshift{4.240107in}{2.002338in}%
\pgfsys@useobject{currentmarker}{}%
\end{pgfscope}%
\begin{pgfscope}%
\pgfsys@transformshift{4.253029in}{2.056587in}%
\pgfsys@useobject{currentmarker}{}%
\end{pgfscope}%
\begin{pgfscope}%
\pgfsys@transformshift{4.265951in}{2.116515in}%
\pgfsys@useobject{currentmarker}{}%
\end{pgfscope}%
\begin{pgfscope}%
\pgfsys@transformshift{4.278873in}{2.183369in}%
\pgfsys@useobject{currentmarker}{}%
\end{pgfscope}%
\begin{pgfscope}%
\pgfsys@transformshift{4.291795in}{2.256029in}%
\pgfsys@useobject{currentmarker}{}%
\end{pgfscope}%
\begin{pgfscope}%
\pgfsys@transformshift{4.304717in}{2.330139in}%
\pgfsys@useobject{currentmarker}{}%
\end{pgfscope}%
\begin{pgfscope}%
\pgfsys@transformshift{4.317638in}{2.398756in}%
\pgfsys@useobject{currentmarker}{}%
\end{pgfscope}%
\begin{pgfscope}%
\pgfsys@transformshift{4.330560in}{2.455040in}%
\pgfsys@useobject{currentmarker}{}%
\end{pgfscope}%
\begin{pgfscope}%
\pgfsys@transformshift{4.343482in}{2.493171in}%
\pgfsys@useobject{currentmarker}{}%
\end{pgfscope}%
\begin{pgfscope}%
\pgfsys@transformshift{4.356404in}{2.505993in}%
\pgfsys@useobject{currentmarker}{}%
\end{pgfscope}%
\begin{pgfscope}%
\pgfsys@transformshift{4.369326in}{2.498752in}%
\pgfsys@useobject{currentmarker}{}%
\end{pgfscope}%
\begin{pgfscope}%
\pgfsys@transformshift{4.382248in}{2.480368in}%
\pgfsys@useobject{currentmarker}{}%
\end{pgfscope}%
\begin{pgfscope}%
\pgfsys@transformshift{4.395170in}{2.461287in}%
\pgfsys@useobject{currentmarker}{}%
\end{pgfscope}%
\begin{pgfscope}%
\pgfsys@transformshift{4.408092in}{2.450755in}%
\pgfsys@useobject{currentmarker}{}%
\end{pgfscope}%
\begin{pgfscope}%
\pgfsys@transformshift{4.421014in}{2.454284in}%
\pgfsys@useobject{currentmarker}{}%
\end{pgfscope}%
\begin{pgfscope}%
\pgfsys@transformshift{4.433936in}{2.463739in}%
\pgfsys@useobject{currentmarker}{}%
\end{pgfscope}%
\begin{pgfscope}%
\pgfsys@transformshift{4.446858in}{2.475893in}%
\pgfsys@useobject{currentmarker}{}%
\end{pgfscope}%
\begin{pgfscope}%
\pgfsys@transformshift{4.459780in}{2.487206in}%
\pgfsys@useobject{currentmarker}{}%
\end{pgfscope}%
\begin{pgfscope}%
\pgfsys@transformshift{4.472702in}{2.493857in}%
\pgfsys@useobject{currentmarker}{}%
\end{pgfscope}%
\begin{pgfscope}%
\pgfsys@transformshift{4.485624in}{2.495311in}%
\pgfsys@useobject{currentmarker}{}%
\end{pgfscope}%
\begin{pgfscope}%
\pgfsys@transformshift{4.498546in}{2.496746in}%
\pgfsys@useobject{currentmarker}{}%
\end{pgfscope}%
\begin{pgfscope}%
\pgfsys@transformshift{4.511468in}{2.488355in}%
\pgfsys@useobject{currentmarker}{}%
\end{pgfscope}%
\begin{pgfscope}%
\pgfsys@transformshift{4.524390in}{2.470679in}%
\pgfsys@useobject{currentmarker}{}%
\end{pgfscope}%
\begin{pgfscope}%
\pgfsys@transformshift{4.537312in}{2.446533in}%
\pgfsys@useobject{currentmarker}{}%
\end{pgfscope}%
\begin{pgfscope}%
\pgfsys@transformshift{4.550234in}{2.420671in}%
\pgfsys@useobject{currentmarker}{}%
\end{pgfscope}%
\begin{pgfscope}%
\pgfsys@transformshift{4.563156in}{2.398611in}%
\pgfsys@useobject{currentmarker}{}%
\end{pgfscope}%
\begin{pgfscope}%
\pgfsys@transformshift{4.576078in}{2.393403in}%
\pgfsys@useobject{currentmarker}{}%
\end{pgfscope}%
\begin{pgfscope}%
\pgfsys@transformshift{4.589000in}{2.399562in}%
\pgfsys@useobject{currentmarker}{}%
\end{pgfscope}%
\begin{pgfscope}%
\pgfsys@transformshift{4.601922in}{2.410230in}%
\pgfsys@useobject{currentmarker}{}%
\end{pgfscope}%
\begin{pgfscope}%
\pgfsys@transformshift{4.614844in}{2.421153in}%
\pgfsys@useobject{currentmarker}{}%
\end{pgfscope}%
\begin{pgfscope}%
\pgfsys@transformshift{4.627765in}{2.428339in}%
\pgfsys@useobject{currentmarker}{}%
\end{pgfscope}%
\begin{pgfscope}%
\pgfsys@transformshift{4.640687in}{2.430051in}%
\pgfsys@useobject{currentmarker}{}%
\end{pgfscope}%
\begin{pgfscope}%
\pgfsys@transformshift{4.653609in}{2.433417in}%
\pgfsys@useobject{currentmarker}{}%
\end{pgfscope}%
\begin{pgfscope}%
\pgfsys@transformshift{4.666531in}{2.429886in}%
\pgfsys@useobject{currentmarker}{}%
\end{pgfscope}%
\begin{pgfscope}%
\pgfsys@transformshift{4.679453in}{2.416949in}%
\pgfsys@useobject{currentmarker}{}%
\end{pgfscope}%
\begin{pgfscope}%
\pgfsys@transformshift{4.692375in}{2.396654in}%
\pgfsys@useobject{currentmarker}{}%
\end{pgfscope}%
\begin{pgfscope}%
\pgfsys@transformshift{4.705297in}{2.373216in}%
\pgfsys@useobject{currentmarker}{}%
\end{pgfscope}%
\begin{pgfscope}%
\pgfsys@transformshift{4.718219in}{2.352007in}%
\pgfsys@useobject{currentmarker}{}%
\end{pgfscope}%
\begin{pgfscope}%
\pgfsys@transformshift{4.733740in}{2.344906in}%
\pgfsys@useobject{currentmarker}{}%
\end{pgfscope}%
\begin{pgfscope}%
\pgfsys@transformshift{4.749261in}{2.352059in}%
\pgfsys@useobject{currentmarker}{}%
\end{pgfscope}%
\begin{pgfscope}%
\pgfsys@transformshift{4.764782in}{2.365869in}%
\pgfsys@useobject{currentmarker}{}%
\end{pgfscope}%
\begin{pgfscope}%
\pgfsys@transformshift{4.780303in}{2.382463in}%
\pgfsys@useobject{currentmarker}{}%
\end{pgfscope}%
\begin{pgfscope}%
\pgfsys@transformshift{4.795824in}{2.397576in}%
\pgfsys@useobject{currentmarker}{}%
\end{pgfscope}%
\begin{pgfscope}%
\pgfsys@transformshift{4.811344in}{2.407096in}%
\pgfsys@useobject{currentmarker}{}%
\end{pgfscope}%
\begin{pgfscope}%
\pgfsys@transformshift{4.826865in}{2.415468in}%
\pgfsys@useobject{currentmarker}{}%
\end{pgfscope}%
\begin{pgfscope}%
\pgfsys@transformshift{4.842386in}{2.419754in}%
\pgfsys@useobject{currentmarker}{}%
\end{pgfscope}%
\begin{pgfscope}%
\pgfsys@transformshift{4.857907in}{2.411865in}%
\pgfsys@useobject{currentmarker}{}%
\end{pgfscope}%
\begin{pgfscope}%
\pgfsys@transformshift{4.873428in}{2.396514in}%
\pgfsys@useobject{currentmarker}{}%
\end{pgfscope}%
\begin{pgfscope}%
\pgfsys@transformshift{4.888949in}{2.382549in}%
\pgfsys@useobject{currentmarker}{}%
\end{pgfscope}%
\begin{pgfscope}%
\pgfsys@transformshift{4.904470in}{2.378773in}%
\pgfsys@useobject{currentmarker}{}%
\end{pgfscope}%
\begin{pgfscope}%
\pgfsys@transformshift{4.919991in}{2.391114in}%
\pgfsys@useobject{currentmarker}{}%
\end{pgfscope}%
\begin{pgfscope}%
\pgfsys@transformshift{4.935511in}{2.415092in}%
\pgfsys@useobject{currentmarker}{}%
\end{pgfscope}%
\begin{pgfscope}%
\pgfsys@transformshift{4.951032in}{2.446466in}%
\pgfsys@useobject{currentmarker}{}%
\end{pgfscope}%
\begin{pgfscope}%
\pgfsys@transformshift{4.966553in}{2.478923in}%
\pgfsys@useobject{currentmarker}{}%
\end{pgfscope}%
\begin{pgfscope}%
\pgfsys@transformshift{4.982074in}{2.505604in}%
\pgfsys@useobject{currentmarker}{}%
\end{pgfscope}%
\begin{pgfscope}%
\pgfsys@transformshift{4.997595in}{2.521068in}%
\pgfsys@useobject{currentmarker}{}%
\end{pgfscope}%
\begin{pgfscope}%
\pgfsys@transformshift{5.013116in}{2.518813in}%
\pgfsys@useobject{currentmarker}{}%
\end{pgfscope}%
\begin{pgfscope}%
\pgfsys@transformshift{5.028637in}{2.500921in}%
\pgfsys@useobject{currentmarker}{}%
\end{pgfscope}%
\begin{pgfscope}%
\pgfsys@transformshift{5.044158in}{2.476000in}%
\pgfsys@useobject{currentmarker}{}%
\end{pgfscope}%
\begin{pgfscope}%
\pgfsys@transformshift{5.059678in}{2.454897in}%
\pgfsys@useobject{currentmarker}{}%
\end{pgfscope}%
\begin{pgfscope}%
\pgfsys@transformshift{5.075199in}{2.446230in}%
\pgfsys@useobject{currentmarker}{}%
\end{pgfscope}%
\begin{pgfscope}%
\pgfsys@transformshift{5.090720in}{2.454630in}%
\pgfsys@useobject{currentmarker}{}%
\end{pgfscope}%
\begin{pgfscope}%
\pgfsys@transformshift{5.106241in}{2.464624in}%
\pgfsys@useobject{currentmarker}{}%
\end{pgfscope}%
\begin{pgfscope}%
\pgfsys@transformshift{5.121762in}{2.472993in}%
\pgfsys@useobject{currentmarker}{}%
\end{pgfscope}%
\begin{pgfscope}%
\pgfsys@transformshift{5.137283in}{2.481843in}%
\pgfsys@useobject{currentmarker}{}%
\end{pgfscope}%
\begin{pgfscope}%
\pgfsys@transformshift{5.152804in}{2.491999in}%
\pgfsys@useobject{currentmarker}{}%
\end{pgfscope}%
\begin{pgfscope}%
\pgfsys@transformshift{5.168325in}{2.485101in}%
\pgfsys@useobject{currentmarker}{}%
\end{pgfscope}%
\begin{pgfscope}%
\pgfsys@transformshift{5.183845in}{2.465671in}%
\pgfsys@useobject{currentmarker}{}%
\end{pgfscope}%
\begin{pgfscope}%
\pgfsys@transformshift{5.199366in}{2.448843in}%
\pgfsys@useobject{currentmarker}{}%
\end{pgfscope}%
\begin{pgfscope}%
\pgfsys@transformshift{5.206896in}{2.432646in}%
\pgfsys@useobject{currentmarker}{}%
\end{pgfscope}%
\end{pgfscope}%
\begin{pgfscope}%
\pgfpathrectangle{\pgfqpoint{0.750000in}{0.500000in}}{\pgfqpoint{4.650000in}{3.020000in}}%
\pgfusepath{clip}%
\pgfsetbuttcap%
\pgfsetroundjoin%
\definecolor{currentfill}{rgb}{0.000000,0.500000,0.000000}%
\pgfsetfillcolor{currentfill}%
\pgfsetlinewidth{0.000000pt}%
\definecolor{currentstroke}{rgb}{0.000000,0.500000,0.000000}%
\pgfsetstrokecolor{currentstroke}%
\pgfsetdash{}{0pt}%
\pgfsys@defobject{currentmarker}{\pgfqpoint{-0.006944in}{-0.006944in}}{\pgfqpoint{0.006945in}{0.006945in}}{%
\pgfpathmoveto{\pgfqpoint{-0.006944in}{-0.006944in}}%
\pgfpathlineto{\pgfqpoint{0.006945in}{-0.006944in}}%
\pgfpathlineto{\pgfqpoint{0.006945in}{0.006945in}}%
\pgfpathlineto{\pgfqpoint{-0.006944in}{0.006945in}}%
\pgfpathlineto{\pgfqpoint{-0.006944in}{-0.006944in}}%
\pgfpathclose%
\pgfusepath{fill}%
}%
\begin{pgfscope}%
\pgfsys@transformshift{1.344812in}{0.637273in}%
\pgfsys@useobject{currentmarker}{}%
\end{pgfscope}%
\begin{pgfscope}%
\pgfsys@transformshift{1.344894in}{0.637275in}%
\pgfsys@useobject{currentmarker}{}%
\end{pgfscope}%
\begin{pgfscope}%
\pgfsys@transformshift{1.344977in}{0.637278in}%
\pgfsys@useobject{currentmarker}{}%
\end{pgfscope}%
\begin{pgfscope}%
\pgfsys@transformshift{1.345086in}{0.637282in}%
\pgfsys@useobject{currentmarker}{}%
\end{pgfscope}%
\begin{pgfscope}%
\pgfsys@transformshift{1.345195in}{0.637288in}%
\pgfsys@useobject{currentmarker}{}%
\end{pgfscope}%
\begin{pgfscope}%
\pgfsys@transformshift{1.345347in}{0.637296in}%
\pgfsys@useobject{currentmarker}{}%
\end{pgfscope}%
\begin{pgfscope}%
\pgfsys@transformshift{1.345498in}{0.637308in}%
\pgfsys@useobject{currentmarker}{}%
\end{pgfscope}%
\begin{pgfscope}%
\pgfsys@transformshift{1.345723in}{0.637325in}%
\pgfsys@useobject{currentmarker}{}%
\end{pgfscope}%
\begin{pgfscope}%
\pgfsys@transformshift{1.345999in}{0.637351in}%
\pgfsys@useobject{currentmarker}{}%
\end{pgfscope}%
\begin{pgfscope}%
\pgfsys@transformshift{1.346380in}{0.637394in}%
\pgfsys@useobject{currentmarker}{}%
\end{pgfscope}%
\begin{pgfscope}%
\pgfsys@transformshift{1.346908in}{0.637470in}%
\pgfsys@useobject{currentmarker}{}%
\end{pgfscope}%
\begin{pgfscope}%
\pgfsys@transformshift{1.347743in}{0.637622in}%
\pgfsys@useobject{currentmarker}{}%
\end{pgfscope}%
\begin{pgfscope}%
\pgfsys@transformshift{1.349253in}{0.637990in}%
\pgfsys@useobject{currentmarker}{}%
\end{pgfscope}%
\begin{pgfscope}%
\pgfsys@transformshift{1.351299in}{0.638787in}%
\pgfsys@useobject{currentmarker}{}%
\end{pgfscope}%
\begin{pgfscope}%
\pgfsys@transformshift{1.353344in}{0.640098in}%
\pgfsys@useobject{currentmarker}{}%
\end{pgfscope}%
\begin{pgfscope}%
\pgfsys@transformshift{1.355390in}{0.641986in}%
\pgfsys@useobject{currentmarker}{}%
\end{pgfscope}%
\begin{pgfscope}%
\pgfsys@transformshift{1.357435in}{0.644484in}%
\pgfsys@useobject{currentmarker}{}%
\end{pgfscope}%
\begin{pgfscope}%
\pgfsys@transformshift{1.359480in}{0.647551in}%
\pgfsys@useobject{currentmarker}{}%
\end{pgfscope}%
\begin{pgfscope}%
\pgfsys@transformshift{1.361526in}{0.651076in}%
\pgfsys@useobject{currentmarker}{}%
\end{pgfscope}%
\begin{pgfscope}%
\pgfsys@transformshift{1.363571in}{0.654974in}%
\pgfsys@useobject{currentmarker}{}%
\end{pgfscope}%
\begin{pgfscope}%
\pgfsys@transformshift{1.365617in}{0.659167in}%
\pgfsys@useobject{currentmarker}{}%
\end{pgfscope}%
\begin{pgfscope}%
\pgfsys@transformshift{1.367662in}{0.663576in}%
\pgfsys@useobject{currentmarker}{}%
\end{pgfscope}%
\begin{pgfscope}%
\pgfsys@transformshift{1.369708in}{0.668119in}%
\pgfsys@useobject{currentmarker}{}%
\end{pgfscope}%
\begin{pgfscope}%
\pgfsys@transformshift{1.371753in}{0.672708in}%
\pgfsys@useobject{currentmarker}{}%
\end{pgfscope}%
\begin{pgfscope}%
\pgfsys@transformshift{1.374261in}{0.677432in}%
\pgfsys@useobject{currentmarker}{}%
\end{pgfscope}%
\begin{pgfscope}%
\pgfsys@transformshift{1.376769in}{0.682318in}%
\pgfsys@useobject{currentmarker}{}%
\end{pgfscope}%
\begin{pgfscope}%
\pgfsys@transformshift{1.379830in}{0.687604in}%
\pgfsys@useobject{currentmarker}{}%
\end{pgfscope}%
\begin{pgfscope}%
\pgfsys@transformshift{1.382890in}{0.693451in}%
\pgfsys@useobject{currentmarker}{}%
\end{pgfscope}%
\begin{pgfscope}%
\pgfsys@transformshift{1.386634in}{0.700381in}%
\pgfsys@useobject{currentmarker}{}%
\end{pgfscope}%
\begin{pgfscope}%
\pgfsys@transformshift{1.390378in}{0.708619in}%
\pgfsys@useobject{currentmarker}{}%
\end{pgfscope}%
\begin{pgfscope}%
\pgfsys@transformshift{1.395057in}{0.718921in}%
\pgfsys@useobject{currentmarker}{}%
\end{pgfscope}%
\begin{pgfscope}%
\pgfsys@transformshift{1.399736in}{0.731459in}%
\pgfsys@useobject{currentmarker}{}%
\end{pgfscope}%
\begin{pgfscope}%
\pgfsys@transformshift{1.405827in}{0.747317in}%
\pgfsys@useobject{currentmarker}{}%
\end{pgfscope}%
\begin{pgfscope}%
\pgfsys@transformshift{1.411918in}{0.766554in}%
\pgfsys@useobject{currentmarker}{}%
\end{pgfscope}%
\begin{pgfscope}%
\pgfsys@transformshift{1.420549in}{0.790959in}%
\pgfsys@useobject{currentmarker}{}%
\end{pgfscope}%
\begin{pgfscope}%
\pgfsys@transformshift{1.429180in}{0.820334in}%
\pgfsys@useobject{currentmarker}{}%
\end{pgfscope}%
\begin{pgfscope}%
\pgfsys@transformshift{1.437812in}{0.853650in}%
\pgfsys@useobject{currentmarker}{}%
\end{pgfscope}%
\begin{pgfscope}%
\pgfsys@transformshift{1.446443in}{0.888368in}%
\pgfsys@useobject{currentmarker}{}%
\end{pgfscope}%
\begin{pgfscope}%
\pgfsys@transformshift{1.455074in}{0.922211in}%
\pgfsys@useobject{currentmarker}{}%
\end{pgfscope}%
\begin{pgfscope}%
\pgfsys@transformshift{1.463705in}{0.951805in}%
\pgfsys@useobject{currentmarker}{}%
\end{pgfscope}%
\begin{pgfscope}%
\pgfsys@transformshift{1.472337in}{0.975012in}%
\pgfsys@useobject{currentmarker}{}%
\end{pgfscope}%
\begin{pgfscope}%
\pgfsys@transformshift{1.482936in}{0.990585in}%
\pgfsys@useobject{currentmarker}{}%
\end{pgfscope}%
\begin{pgfscope}%
\pgfsys@transformshift{1.492816in}{0.997722in}%
\pgfsys@useobject{currentmarker}{}%
\end{pgfscope}%
\begin{pgfscope}%
\pgfsys@transformshift{1.492816in}{0.996059in}%
\pgfsys@useobject{currentmarker}{}%
\end{pgfscope}%
\begin{pgfscope}%
\pgfsys@transformshift{1.492816in}{0.986012in}%
\pgfsys@useobject{currentmarker}{}%
\end{pgfscope}%
\begin{pgfscope}%
\pgfsys@transformshift{1.494636in}{0.967901in}%
\pgfsys@useobject{currentmarker}{}%
\end{pgfscope}%
\begin{pgfscope}%
\pgfsys@transformshift{1.498276in}{0.942534in}%
\pgfsys@useobject{currentmarker}{}%
\end{pgfscope}%
\begin{pgfscope}%
\pgfsys@transformshift{1.501916in}{0.908251in}%
\pgfsys@useobject{currentmarker}{}%
\end{pgfscope}%
\begin{pgfscope}%
\pgfsys@transformshift{1.505557in}{0.867189in}%
\pgfsys@useobject{currentmarker}{}%
\end{pgfscope}%
\begin{pgfscope}%
\pgfsys@transformshift{1.510739in}{0.820657in}%
\pgfsys@useobject{currentmarker}{}%
\end{pgfscope}%
\begin{pgfscope}%
\pgfsys@transformshift{1.517027in}{0.774674in}%
\pgfsys@useobject{currentmarker}{}%
\end{pgfscope}%
\begin{pgfscope}%
\pgfsys@transformshift{1.525486in}{0.740055in}%
\pgfsys@useobject{currentmarker}{}%
\end{pgfscope}%
\begin{pgfscope}%
\pgfsys@transformshift{1.536357in}{0.728195in}%
\pgfsys@useobject{currentmarker}{}%
\end{pgfscope}%
\begin{pgfscope}%
\pgfsys@transformshift{1.547228in}{0.728345in}%
\pgfsys@useobject{currentmarker}{}%
\end{pgfscope}%
\begin{pgfscope}%
\pgfsys@transformshift{1.558099in}{0.727072in}%
\pgfsys@useobject{currentmarker}{}%
\end{pgfscope}%
\begin{pgfscope}%
\pgfsys@transformshift{1.568970in}{0.739880in}%
\pgfsys@useobject{currentmarker}{}%
\end{pgfscope}%
\begin{pgfscope}%
\pgfsys@transformshift{1.579841in}{0.785875in}%
\pgfsys@useobject{currentmarker}{}%
\end{pgfscope}%
\begin{pgfscope}%
\pgfsys@transformshift{1.590712in}{0.858433in}%
\pgfsys@useobject{currentmarker}{}%
\end{pgfscope}%
\begin{pgfscope}%
\pgfsys@transformshift{1.601583in}{0.944177in}%
\pgfsys@useobject{currentmarker}{}%
\end{pgfscope}%
\begin{pgfscope}%
\pgfsys@transformshift{1.612454in}{1.031745in}%
\pgfsys@useobject{currentmarker}{}%
\end{pgfscope}%
\begin{pgfscope}%
\pgfsys@transformshift{1.623325in}{1.113327in}%
\pgfsys@useobject{currentmarker}{}%
\end{pgfscope}%
\begin{pgfscope}%
\pgfsys@transformshift{1.636719in}{1.183571in}%
\pgfsys@useobject{currentmarker}{}%
\end{pgfscope}%
\begin{pgfscope}%
\pgfsys@transformshift{1.650113in}{1.237408in}%
\pgfsys@useobject{currentmarker}{}%
\end{pgfscope}%
\begin{pgfscope}%
\pgfsys@transformshift{1.663507in}{1.272468in}%
\pgfsys@useobject{currentmarker}{}%
\end{pgfscope}%
\begin{pgfscope}%
\pgfsys@transformshift{1.676902in}{1.291222in}%
\pgfsys@useobject{currentmarker}{}%
\end{pgfscope}%
\begin{pgfscope}%
\pgfsys@transformshift{1.690296in}{1.301819in}%
\pgfsys@useobject{currentmarker}{}%
\end{pgfscope}%
\begin{pgfscope}%
\pgfsys@transformshift{1.703690in}{1.298063in}%
\pgfsys@useobject{currentmarker}{}%
\end{pgfscope}%
\begin{pgfscope}%
\pgfsys@transformshift{1.717084in}{1.274681in}%
\pgfsys@useobject{currentmarker}{}%
\end{pgfscope}%
\begin{pgfscope}%
\pgfsys@transformshift{1.730479in}{1.229602in}%
\pgfsys@useobject{currentmarker}{}%
\end{pgfscope}%
\begin{pgfscope}%
\pgfsys@transformshift{1.743873in}{1.163276in}%
\pgfsys@useobject{currentmarker}{}%
\end{pgfscope}%
\begin{pgfscope}%
\pgfsys@transformshift{1.757267in}{1.082636in}%
\pgfsys@useobject{currentmarker}{}%
\end{pgfscope}%
\begin{pgfscope}%
\pgfsys@transformshift{1.770661in}{0.997647in}%
\pgfsys@useobject{currentmarker}{}%
\end{pgfscope}%
\begin{pgfscope}%
\pgfsys@transformshift{1.784056in}{0.916537in}%
\pgfsys@useobject{currentmarker}{}%
\end{pgfscope}%
\begin{pgfscope}%
\pgfsys@transformshift{1.789942in}{0.842757in}%
\pgfsys@useobject{currentmarker}{}%
\end{pgfscope}%
\begin{pgfscope}%
\pgfsys@transformshift{1.789942in}{0.779625in}%
\pgfsys@useobject{currentmarker}{}%
\end{pgfscope}%
\begin{pgfscope}%
\pgfsys@transformshift{1.789942in}{0.746369in}%
\pgfsys@useobject{currentmarker}{}%
\end{pgfscope}%
\begin{pgfscope}%
\pgfsys@transformshift{1.793487in}{0.756637in}%
\pgfsys@useobject{currentmarker}{}%
\end{pgfscope}%
\begin{pgfscope}%
\pgfsys@transformshift{1.800577in}{0.785157in}%
\pgfsys@useobject{currentmarker}{}%
\end{pgfscope}%
\begin{pgfscope}%
\pgfsys@transformshift{1.807667in}{0.808826in}%
\pgfsys@useobject{currentmarker}{}%
\end{pgfscope}%
\begin{pgfscope}%
\pgfsys@transformshift{1.818289in}{0.821940in}%
\pgfsys@useobject{currentmarker}{}%
\end{pgfscope}%
\begin{pgfscope}%
\pgfsys@transformshift{1.831044in}{0.825991in}%
\pgfsys@useobject{currentmarker}{}%
\end{pgfscope}%
\begin{pgfscope}%
\pgfsys@transformshift{1.843798in}{0.824097in}%
\pgfsys@useobject{currentmarker}{}%
\end{pgfscope}%
\begin{pgfscope}%
\pgfsys@transformshift{1.856553in}{0.818508in}%
\pgfsys@useobject{currentmarker}{}%
\end{pgfscope}%
\begin{pgfscope}%
\pgfsys@transformshift{1.869308in}{0.810803in}%
\pgfsys@useobject{currentmarker}{}%
\end{pgfscope}%
\begin{pgfscope}%
\pgfsys@transformshift{1.882062in}{0.802002in}%
\pgfsys@useobject{currentmarker}{}%
\end{pgfscope}%
\begin{pgfscope}%
\pgfsys@transformshift{1.894817in}{0.815183in}%
\pgfsys@useobject{currentmarker}{}%
\end{pgfscope}%
\begin{pgfscope}%
\pgfsys@transformshift{1.907572in}{0.842742in}%
\pgfsys@useobject{currentmarker}{}%
\end{pgfscope}%
\begin{pgfscope}%
\pgfsys@transformshift{1.920326in}{0.871004in}%
\pgfsys@useobject{currentmarker}{}%
\end{pgfscope}%
\begin{pgfscope}%
\pgfsys@transformshift{1.933081in}{0.886345in}%
\pgfsys@useobject{currentmarker}{}%
\end{pgfscope}%
\begin{pgfscope}%
\pgfsys@transformshift{1.945836in}{0.885075in}%
\pgfsys@useobject{currentmarker}{}%
\end{pgfscope}%
\begin{pgfscope}%
\pgfsys@transformshift{1.961644in}{0.869475in}%
\pgfsys@useobject{currentmarker}{}%
\end{pgfscope}%
\begin{pgfscope}%
\pgfsys@transformshift{1.977452in}{0.842343in}%
\pgfsys@useobject{currentmarker}{}%
\end{pgfscope}%
\begin{pgfscope}%
\pgfsys@transformshift{1.993261in}{0.807680in}%
\pgfsys@useobject{currentmarker}{}%
\end{pgfscope}%
\begin{pgfscope}%
\pgfsys@transformshift{2.009069in}{0.768387in}%
\pgfsys@useobject{currentmarker}{}%
\end{pgfscope}%
\begin{pgfscope}%
\pgfsys@transformshift{2.024878in}{0.730622in}%
\pgfsys@useobject{currentmarker}{}%
\end{pgfscope}%
\begin{pgfscope}%
\pgfsys@transformshift{2.040686in}{0.705573in}%
\pgfsys@useobject{currentmarker}{}%
\end{pgfscope}%
\begin{pgfscope}%
\pgfsys@transformshift{2.056494in}{0.706475in}%
\pgfsys@useobject{currentmarker}{}%
\end{pgfscope}%
\begin{pgfscope}%
\pgfsys@transformshift{2.072303in}{0.725848in}%
\pgfsys@useobject{currentmarker}{}%
\end{pgfscope}%
\begin{pgfscope}%
\pgfsys@transformshift{2.088111in}{0.747285in}%
\pgfsys@useobject{currentmarker}{}%
\end{pgfscope}%
\begin{pgfscope}%
\pgfsys@transformshift{2.103919in}{0.764850in}%
\pgfsys@useobject{currentmarker}{}%
\end{pgfscope}%
\begin{pgfscope}%
\pgfsys@transformshift{2.119728in}{0.776876in}%
\pgfsys@useobject{currentmarker}{}%
\end{pgfscope}%
\begin{pgfscope}%
\pgfsys@transformshift{2.135536in}{0.782214in}%
\pgfsys@useobject{currentmarker}{}%
\end{pgfscope}%
\begin{pgfscope}%
\pgfsys@transformshift{2.151345in}{0.779737in}%
\pgfsys@useobject{currentmarker}{}%
\end{pgfscope}%
\begin{pgfscope}%
\pgfsys@transformshift{2.167153in}{0.763946in}%
\pgfsys@useobject{currentmarker}{}%
\end{pgfscope}%
\begin{pgfscope}%
\pgfsys@transformshift{2.182961in}{0.748187in}%
\pgfsys@useobject{currentmarker}{}%
\end{pgfscope}%
\begin{pgfscope}%
\pgfsys@transformshift{2.198770in}{0.743066in}%
\pgfsys@useobject{currentmarker}{}%
\end{pgfscope}%
\begin{pgfscope}%
\pgfsys@transformshift{2.214578in}{0.742588in}%
\pgfsys@useobject{currentmarker}{}%
\end{pgfscope}%
\begin{pgfscope}%
\pgfsys@transformshift{2.230386in}{0.739377in}%
\pgfsys@useobject{currentmarker}{}%
\end{pgfscope}%
\begin{pgfscope}%
\pgfsys@transformshift{2.246195in}{0.736398in}%
\pgfsys@useobject{currentmarker}{}%
\end{pgfscope}%
\begin{pgfscope}%
\pgfsys@transformshift{2.262003in}{0.735785in}%
\pgfsys@useobject{currentmarker}{}%
\end{pgfscope}%
\begin{pgfscope}%
\pgfsys@transformshift{2.277812in}{0.735699in}%
\pgfsys@useobject{currentmarker}{}%
\end{pgfscope}%
\begin{pgfscope}%
\pgfsys@transformshift{2.293620in}{0.721042in}%
\pgfsys@useobject{currentmarker}{}%
\end{pgfscope}%
\begin{pgfscope}%
\pgfsys@transformshift{2.309428in}{0.702890in}%
\pgfsys@useobject{currentmarker}{}%
\end{pgfscope}%
\begin{pgfscope}%
\pgfsys@transformshift{2.325237in}{0.713149in}%
\pgfsys@useobject{currentmarker}{}%
\end{pgfscope}%
\begin{pgfscope}%
\pgfsys@transformshift{2.341045in}{0.732971in}%
\pgfsys@useobject{currentmarker}{}%
\end{pgfscope}%
\begin{pgfscope}%
\pgfsys@transformshift{2.356853in}{0.761335in}%
\pgfsys@useobject{currentmarker}{}%
\end{pgfscope}%
\begin{pgfscope}%
\pgfsys@transformshift{2.376266in}{0.784779in}%
\pgfsys@useobject{currentmarker}{}%
\end{pgfscope}%
\begin{pgfscope}%
\pgfsys@transformshift{2.384195in}{0.793767in}%
\pgfsys@useobject{currentmarker}{}%
\end{pgfscope}%
\begin{pgfscope}%
\pgfsys@transformshift{2.384195in}{0.793666in}%
\pgfsys@useobject{currentmarker}{}%
\end{pgfscope}%
\begin{pgfscope}%
\pgfsys@transformshift{2.384195in}{0.775046in}%
\pgfsys@useobject{currentmarker}{}%
\end{pgfscope}%
\begin{pgfscope}%
\pgfsys@transformshift{2.389315in}{0.747886in}%
\pgfsys@useobject{currentmarker}{}%
\end{pgfscope}%
\begin{pgfscope}%
\pgfsys@transformshift{2.399554in}{0.714657in}%
\pgfsys@useobject{currentmarker}{}%
\end{pgfscope}%
\begin{pgfscope}%
\pgfsys@transformshift{2.409792in}{0.702871in}%
\pgfsys@useobject{currentmarker}{}%
\end{pgfscope}%
\begin{pgfscope}%
\pgfsys@transformshift{2.427437in}{0.721587in}%
\pgfsys@useobject{currentmarker}{}%
\end{pgfscope}%
\begin{pgfscope}%
\pgfsys@transformshift{2.445081in}{0.753327in}%
\pgfsys@useobject{currentmarker}{}%
\end{pgfscope}%
\begin{pgfscope}%
\pgfsys@transformshift{2.462726in}{0.777436in}%
\pgfsys@useobject{currentmarker}{}%
\end{pgfscope}%
\begin{pgfscope}%
\pgfsys@transformshift{2.480371in}{0.803297in}%
\pgfsys@useobject{currentmarker}{}%
\end{pgfscope}%
\begin{pgfscope}%
\pgfsys@transformshift{2.498015in}{0.822522in}%
\pgfsys@useobject{currentmarker}{}%
\end{pgfscope}%
\begin{pgfscope}%
\pgfsys@transformshift{2.515660in}{0.829648in}%
\pgfsys@useobject{currentmarker}{}%
\end{pgfscope}%
\begin{pgfscope}%
\pgfsys@transformshift{2.532758in}{0.822490in}%
\pgfsys@useobject{currentmarker}{}%
\end{pgfscope}%
\begin{pgfscope}%
\pgfsys@transformshift{2.532758in}{0.803755in}%
\pgfsys@useobject{currentmarker}{}%
\end{pgfscope}%
\begin{pgfscope}%
\pgfsys@transformshift{2.532758in}{0.778174in}%
\pgfsys@useobject{currentmarker}{}%
\end{pgfscope}%
\begin{pgfscope}%
\pgfsys@transformshift{2.533532in}{0.751507in}%
\pgfsys@useobject{currentmarker}{}%
\end{pgfscope}%
\begin{pgfscope}%
\pgfsys@transformshift{2.535078in}{0.731979in}%
\pgfsys@useobject{currentmarker}{}%
\end{pgfscope}%
\begin{pgfscope}%
\pgfsys@transformshift{2.536625in}{0.722196in}%
\pgfsys@useobject{currentmarker}{}%
\end{pgfscope}%
\begin{pgfscope}%
\pgfsys@transformshift{2.538172in}{0.718347in}%
\pgfsys@useobject{currentmarker}{}%
\end{pgfscope}%
\begin{pgfscope}%
\pgfsys@transformshift{2.540260in}{0.717744in}%
\pgfsys@useobject{currentmarker}{}%
\end{pgfscope}%
\begin{pgfscope}%
\pgfsys@transformshift{2.543158in}{0.719936in}%
\pgfsys@useobject{currentmarker}{}%
\end{pgfscope}%
\begin{pgfscope}%
\pgfsys@transformshift{2.547673in}{0.724150in}%
\pgfsys@useobject{currentmarker}{}%
\end{pgfscope}%
\begin{pgfscope}%
\pgfsys@transformshift{2.555520in}{0.727465in}%
\pgfsys@useobject{currentmarker}{}%
\end{pgfscope}%
\begin{pgfscope}%
\pgfsys@transformshift{2.569439in}{0.719308in}%
\pgfsys@useobject{currentmarker}{}%
\end{pgfscope}%
\begin{pgfscope}%
\pgfsys@transformshift{2.583357in}{0.706352in}%
\pgfsys@useobject{currentmarker}{}%
\end{pgfscope}%
\begin{pgfscope}%
\pgfsys@transformshift{2.597276in}{0.728860in}%
\pgfsys@useobject{currentmarker}{}%
\end{pgfscope}%
\begin{pgfscope}%
\pgfsys@transformshift{2.611194in}{0.794641in}%
\pgfsys@useobject{currentmarker}{}%
\end{pgfscope}%
\begin{pgfscope}%
\pgfsys@transformshift{2.625113in}{0.877824in}%
\pgfsys@useobject{currentmarker}{}%
\end{pgfscope}%
\begin{pgfscope}%
\pgfsys@transformshift{2.639031in}{0.962731in}%
\pgfsys@useobject{currentmarker}{}%
\end{pgfscope}%
\begin{pgfscope}%
\pgfsys@transformshift{2.652950in}{1.041690in}%
\pgfsys@useobject{currentmarker}{}%
\end{pgfscope}%
\begin{pgfscope}%
\pgfsys@transformshift{2.666868in}{1.109941in}%
\pgfsys@useobject{currentmarker}{}%
\end{pgfscope}%
\begin{pgfscope}%
\pgfsys@transformshift{2.680787in}{1.164199in}%
\pgfsys@useobject{currentmarker}{}%
\end{pgfscope}%
\begin{pgfscope}%
\pgfsys@transformshift{2.681322in}{1.202325in}%
\pgfsys@useobject{currentmarker}{}%
\end{pgfscope}%
\begin{pgfscope}%
\pgfsys@transformshift{2.681322in}{1.223935in}%
\pgfsys@useobject{currentmarker}{}%
\end{pgfscope}%
\begin{pgfscope}%
\pgfsys@transformshift{2.681322in}{1.233561in}%
\pgfsys@useobject{currentmarker}{}%
\end{pgfscope}%
\begin{pgfscope}%
\pgfsys@transformshift{2.681618in}{1.232198in}%
\pgfsys@useobject{currentmarker}{}%
\end{pgfscope}%
\begin{pgfscope}%
\pgfsys@transformshift{2.681618in}{1.216794in}%
\pgfsys@useobject{currentmarker}{}%
\end{pgfscope}%
\begin{pgfscope}%
\pgfsys@transformshift{2.681914in}{1.184168in}%
\pgfsys@useobject{currentmarker}{}%
\end{pgfscope}%
\begin{pgfscope}%
\pgfsys@transformshift{2.682507in}{1.128225in}%
\pgfsys@useobject{currentmarker}{}%
\end{pgfscope}%
\begin{pgfscope}%
\pgfsys@transformshift{2.683907in}{1.048639in}%
\pgfsys@useobject{currentmarker}{}%
\end{pgfscope}%
\begin{pgfscope}%
\pgfsys@transformshift{2.687292in}{0.954440in}%
\pgfsys@useobject{currentmarker}{}%
\end{pgfscope}%
\begin{pgfscope}%
\pgfsys@transformshift{2.690676in}{0.864349in}%
\pgfsys@useobject{currentmarker}{}%
\end{pgfscope}%
\begin{pgfscope}%
\pgfsys@transformshift{2.694061in}{0.822938in}%
\pgfsys@useobject{currentmarker}{}%
\end{pgfscope}%
\begin{pgfscope}%
\pgfsys@transformshift{2.698509in}{0.870322in}%
\pgfsys@useobject{currentmarker}{}%
\end{pgfscope}%
\begin{pgfscope}%
\pgfsys@transformshift{2.702958in}{0.965506in}%
\pgfsys@useobject{currentmarker}{}%
\end{pgfscope}%
\begin{pgfscope}%
\pgfsys@transformshift{2.708728in}{1.064886in}%
\pgfsys@useobject{currentmarker}{}%
\end{pgfscope}%
\begin{pgfscope}%
\pgfsys@transformshift{2.714499in}{1.142956in}%
\pgfsys@useobject{currentmarker}{}%
\end{pgfscope}%
\begin{pgfscope}%
\pgfsys@transformshift{2.722726in}{1.176794in}%
\pgfsys@useobject{currentmarker}{}%
\end{pgfscope}%
\begin{pgfscope}%
\pgfsys@transformshift{2.733938in}{1.158074in}%
\pgfsys@useobject{currentmarker}{}%
\end{pgfscope}%
\begin{pgfscope}%
\pgfsys@transformshift{2.745150in}{1.105295in}%
\pgfsys@useobject{currentmarker}{}%
\end{pgfscope}%
\begin{pgfscope}%
\pgfsys@transformshift{2.756363in}{1.046983in}%
\pgfsys@useobject{currentmarker}{}%
\end{pgfscope}%
\begin{pgfscope}%
\pgfsys@transformshift{2.767575in}{1.012417in}%
\pgfsys@useobject{currentmarker}{}%
\end{pgfscope}%
\begin{pgfscope}%
\pgfsys@transformshift{2.778788in}{1.031536in}%
\pgfsys@useobject{currentmarker}{}%
\end{pgfscope}%
\begin{pgfscope}%
\pgfsys@transformshift{2.790000in}{1.101795in}%
\pgfsys@useobject{currentmarker}{}%
\end{pgfscope}%
\begin{pgfscope}%
\pgfsys@transformshift{2.801213in}{1.193771in}%
\pgfsys@useobject{currentmarker}{}%
\end{pgfscope}%
\begin{pgfscope}%
\pgfsys@transformshift{2.812425in}{1.281092in}%
\pgfsys@useobject{currentmarker}{}%
\end{pgfscope}%
\begin{pgfscope}%
\pgfsys@transformshift{2.823637in}{1.347035in}%
\pgfsys@useobject{currentmarker}{}%
\end{pgfscope}%
\begin{pgfscope}%
\pgfsys@transformshift{2.829885in}{1.384576in}%
\pgfsys@useobject{currentmarker}{}%
\end{pgfscope}%
\begin{pgfscope}%
\pgfsys@transformshift{2.829885in}{1.399811in}%
\pgfsys@useobject{currentmarker}{}%
\end{pgfscope}%
\begin{pgfscope}%
\pgfsys@transformshift{2.829885in}{1.407174in}%
\pgfsys@useobject{currentmarker}{}%
\end{pgfscope}%
\begin{pgfscope}%
\pgfsys@transformshift{2.832783in}{1.415691in}%
\pgfsys@useobject{currentmarker}{}%
\end{pgfscope}%
\begin{pgfscope}%
\pgfsys@transformshift{2.838578in}{1.396024in}%
\pgfsys@useobject{currentmarker}{}%
\end{pgfscope}%
\begin{pgfscope}%
\pgfsys@transformshift{2.842816in}{1.336846in}%
\pgfsys@useobject{currentmarker}{}%
\end{pgfscope}%
\begin{pgfscope}%
\pgfsys@transformshift{2.847054in}{1.240057in}%
\pgfsys@useobject{currentmarker}{}%
\end{pgfscope}%
\begin{pgfscope}%
\pgfsys@transformshift{2.854369in}{1.121796in}%
\pgfsys@useobject{currentmarker}{}%
\end{pgfscope}%
\begin{pgfscope}%
\pgfsys@transformshift{2.861684in}{0.998603in}%
\pgfsys@useobject{currentmarker}{}%
\end{pgfscope}%
\begin{pgfscope}%
\pgfsys@transformshift{2.873089in}{0.896857in}%
\pgfsys@useobject{currentmarker}{}%
\end{pgfscope}%
\begin{pgfscope}%
\pgfsys@transformshift{2.888961in}{0.873641in}%
\pgfsys@useobject{currentmarker}{}%
\end{pgfscope}%
\begin{pgfscope}%
\pgfsys@transformshift{2.904834in}{0.938918in}%
\pgfsys@useobject{currentmarker}{}%
\end{pgfscope}%
\begin{pgfscope}%
\pgfsys@transformshift{2.920706in}{1.026606in}%
\pgfsys@useobject{currentmarker}{}%
\end{pgfscope}%
\begin{pgfscope}%
\pgfsys@transformshift{2.936578in}{1.092500in}%
\pgfsys@useobject{currentmarker}{}%
\end{pgfscope}%
\begin{pgfscope}%
\pgfsys@transformshift{2.952451in}{1.121363in}%
\pgfsys@useobject{currentmarker}{}%
\end{pgfscope}%
\begin{pgfscope}%
\pgfsys@transformshift{2.968323in}{1.114384in}%
\pgfsys@useobject{currentmarker}{}%
\end{pgfscope}%
\begin{pgfscope}%
\pgfsys@transformshift{2.978448in}{1.078317in}%
\pgfsys@useobject{currentmarker}{}%
\end{pgfscope}%
\begin{pgfscope}%
\pgfsys@transformshift{2.978448in}{1.020752in}%
\pgfsys@useobject{currentmarker}{}%
\end{pgfscope}%
\begin{pgfscope}%
\pgfsys@transformshift{2.978448in}{0.948299in}%
\pgfsys@useobject{currentmarker}{}%
\end{pgfscope}%
\begin{pgfscope}%
\pgfsys@transformshift{2.983199in}{0.867223in}%
\pgfsys@useobject{currentmarker}{}%
\end{pgfscope}%
\begin{pgfscope}%
\pgfsys@transformshift{2.992700in}{0.785984in}%
\pgfsys@useobject{currentmarker}{}%
\end{pgfscope}%
\begin{pgfscope}%
\pgfsys@transformshift{3.002202in}{0.717358in}%
\pgfsys@useobject{currentmarker}{}%
\end{pgfscope}%
\begin{pgfscope}%
\pgfsys@transformshift{3.018558in}{0.674993in}%
\pgfsys@useobject{currentmarker}{}%
\end{pgfscope}%
\begin{pgfscope}%
\pgfsys@transformshift{3.034914in}{0.668438in}%
\pgfsys@useobject{currentmarker}{}%
\end{pgfscope}%
\begin{pgfscope}%
\pgfsys@transformshift{3.051270in}{0.668833in}%
\pgfsys@useobject{currentmarker}{}%
\end{pgfscope}%
\begin{pgfscope}%
\pgfsys@transformshift{3.067626in}{0.666080in}%
\pgfsys@useobject{currentmarker}{}%
\end{pgfscope}%
\begin{pgfscope}%
\pgfsys@transformshift{3.083982in}{0.664549in}%
\pgfsys@useobject{currentmarker}{}%
\end{pgfscope}%
\begin{pgfscope}%
\pgfsys@transformshift{3.100338in}{0.664701in}%
\pgfsys@useobject{currentmarker}{}%
\end{pgfscope}%
\begin{pgfscope}%
\pgfsys@transformshift{3.116694in}{0.676043in}%
\pgfsys@useobject{currentmarker}{}%
\end{pgfscope}%
\begin{pgfscope}%
\pgfsys@transformshift{3.127011in}{0.705300in}%
\pgfsys@useobject{currentmarker}{}%
\end{pgfscope}%
\begin{pgfscope}%
\pgfsys@transformshift{3.127011in}{0.741573in}%
\pgfsys@useobject{currentmarker}{}%
\end{pgfscope}%
\begin{pgfscope}%
\pgfsys@transformshift{3.127011in}{0.773428in}%
\pgfsys@useobject{currentmarker}{}%
\end{pgfscope}%
\begin{pgfscope}%
\pgfsys@transformshift{3.133141in}{0.794217in}%
\pgfsys@useobject{currentmarker}{}%
\end{pgfscope}%
\begin{pgfscope}%
\pgfsys@transformshift{3.145402in}{0.804156in}%
\pgfsys@useobject{currentmarker}{}%
\end{pgfscope}%
\begin{pgfscope}%
\pgfsys@transformshift{3.157663in}{0.805245in}%
\pgfsys@useobject{currentmarker}{}%
\end{pgfscope}%
\begin{pgfscope}%
\pgfsys@transformshift{3.177825in}{0.798176in}%
\pgfsys@useobject{currentmarker}{}%
\end{pgfscope}%
\begin{pgfscope}%
\pgfsys@transformshift{3.197987in}{0.782635in}%
\pgfsys@useobject{currentmarker}{}%
\end{pgfscope}%
\begin{pgfscope}%
\pgfsys@transformshift{3.218149in}{0.757209in}%
\pgfsys@useobject{currentmarker}{}%
\end{pgfscope}%
\begin{pgfscope}%
\pgfsys@transformshift{3.238311in}{0.725600in}%
\pgfsys@useobject{currentmarker}{}%
\end{pgfscope}%
\begin{pgfscope}%
\pgfsys@transformshift{3.258473in}{0.693172in}%
\pgfsys@useobject{currentmarker}{}%
\end{pgfscope}%
\begin{pgfscope}%
\pgfsys@transformshift{3.278636in}{0.667870in}%
\pgfsys@useobject{currentmarker}{}%
\end{pgfscope}%
\begin{pgfscope}%
\pgfsys@transformshift{3.298798in}{0.655149in}%
\pgfsys@useobject{currentmarker}{}%
\end{pgfscope}%
\begin{pgfscope}%
\pgfsys@transformshift{3.318960in}{0.666655in}%
\pgfsys@useobject{currentmarker}{}%
\end{pgfscope}%
\begin{pgfscope}%
\pgfsys@transformshift{3.339122in}{0.692498in}%
\pgfsys@useobject{currentmarker}{}%
\end{pgfscope}%
\begin{pgfscope}%
\pgfsys@transformshift{3.359284in}{0.717604in}%
\pgfsys@useobject{currentmarker}{}%
\end{pgfscope}%
\begin{pgfscope}%
\pgfsys@transformshift{3.384093in}{0.733662in}%
\pgfsys@useobject{currentmarker}{}%
\end{pgfscope}%
\begin{pgfscope}%
\pgfsys@transformshift{3.408901in}{0.739613in}%
\pgfsys@useobject{currentmarker}{}%
\end{pgfscope}%
\begin{pgfscope}%
\pgfsys@transformshift{3.433710in}{0.738528in}%
\pgfsys@useobject{currentmarker}{}%
\end{pgfscope}%
\begin{pgfscope}%
\pgfsys@transformshift{3.458518in}{0.732212in}%
\pgfsys@useobject{currentmarker}{}%
\end{pgfscope}%
\begin{pgfscope}%
\pgfsys@transformshift{3.483327in}{0.724100in}%
\pgfsys@useobject{currentmarker}{}%
\end{pgfscope}%
\begin{pgfscope}%
\pgfsys@transformshift{3.508136in}{0.702773in}%
\pgfsys@useobject{currentmarker}{}%
\end{pgfscope}%
\begin{pgfscope}%
\pgfsys@transformshift{3.532944in}{0.672045in}%
\pgfsys@useobject{currentmarker}{}%
\end{pgfscope}%
\begin{pgfscope}%
\pgfsys@transformshift{3.562994in}{0.662461in}%
\pgfsys@useobject{currentmarker}{}%
\end{pgfscope}%
\begin{pgfscope}%
\pgfsys@transformshift{3.593043in}{0.661630in}%
\pgfsys@useobject{currentmarker}{}%
\end{pgfscope}%
\begin{pgfscope}%
\pgfsys@transformshift{3.623092in}{0.662897in}%
\pgfsys@useobject{currentmarker}{}%
\end{pgfscope}%
\begin{pgfscope}%
\pgfsys@transformshift{3.653142in}{0.673602in}%
\pgfsys@useobject{currentmarker}{}%
\end{pgfscope}%
\begin{pgfscope}%
\pgfsys@transformshift{3.683191in}{0.686059in}%
\pgfsys@useobject{currentmarker}{}%
\end{pgfscope}%
\begin{pgfscope}%
\pgfsys@transformshift{3.713241in}{0.681759in}%
\pgfsys@useobject{currentmarker}{}%
\end{pgfscope}%
\begin{pgfscope}%
\pgfsys@transformshift{3.721264in}{0.673236in}%
\pgfsys@useobject{currentmarker}{}%
\end{pgfscope}%
\begin{pgfscope}%
\pgfsys@transformshift{3.721264in}{0.678739in}%
\pgfsys@useobject{currentmarker}{}%
\end{pgfscope}%
\begin{pgfscope}%
\pgfsys@transformshift{3.721264in}{0.693217in}%
\pgfsys@useobject{currentmarker}{}%
\end{pgfscope}%
\begin{pgfscope}%
\pgfsys@transformshift{3.734411in}{0.704348in}%
\pgfsys@useobject{currentmarker}{}%
\end{pgfscope}%
\begin{pgfscope}%
\pgfsys@transformshift{3.760705in}{0.708213in}%
\pgfsys@useobject{currentmarker}{}%
\end{pgfscope}%
\begin{pgfscope}%
\pgfsys@transformshift{3.787000in}{0.696946in}%
\pgfsys@useobject{currentmarker}{}%
\end{pgfscope}%
\begin{pgfscope}%
\pgfsys@transformshift{3.828820in}{0.681423in}%
\pgfsys@useobject{currentmarker}{}%
\end{pgfscope}%
\begin{pgfscope}%
\pgfsys@transformshift{3.869827in}{0.664612in}%
\pgfsys@useobject{currentmarker}{}%
\end{pgfscope}%
\begin{pgfscope}%
\pgfsys@transformshift{3.869827in}{0.654491in}%
\pgfsys@useobject{currentmarker}{}%
\end{pgfscope}%
\begin{pgfscope}%
\pgfsys@transformshift{3.869827in}{0.653789in}%
\pgfsys@useobject{currentmarker}{}%
\end{pgfscope}%
\begin{pgfscope}%
\pgfsys@transformshift{3.871582in}{0.655217in}%
\pgfsys@useobject{currentmarker}{}%
\end{pgfscope}%
\begin{pgfscope}%
\pgfsys@transformshift{3.875091in}{0.656095in}%
\pgfsys@useobject{currentmarker}{}%
\end{pgfscope}%
\begin{pgfscope}%
\pgfsys@transformshift{3.877759in}{0.660630in}%
\pgfsys@useobject{currentmarker}{}%
\end{pgfscope}%
\begin{pgfscope}%
\pgfsys@transformshift{3.880966in}{0.660830in}%
\pgfsys@useobject{currentmarker}{}%
\end{pgfscope}%
\begin{pgfscope}%
\pgfsys@transformshift{3.885349in}{0.658337in}%
\pgfsys@useobject{currentmarker}{}%
\end{pgfscope}%
\begin{pgfscope}%
\pgfsys@transformshift{3.890659in}{0.655007in}%
\pgfsys@useobject{currentmarker}{}%
\end{pgfscope}%
\begin{pgfscope}%
\pgfsys@transformshift{3.898287in}{0.651797in}%
\pgfsys@useobject{currentmarker}{}%
\end{pgfscope}%
\begin{pgfscope}%
\pgfsys@transformshift{3.909283in}{0.650332in}%
\pgfsys@useobject{currentmarker}{}%
\end{pgfscope}%
\begin{pgfscope}%
\pgfsys@transformshift{3.925341in}{0.653810in}%
\pgfsys@useobject{currentmarker}{}%
\end{pgfscope}%
\begin{pgfscope}%
\pgfsys@transformshift{3.941398in}{0.674266in}%
\pgfsys@useobject{currentmarker}{}%
\end{pgfscope}%
\begin{pgfscope}%
\pgfsys@transformshift{3.960986in}{0.714489in}%
\pgfsys@useobject{currentmarker}{}%
\end{pgfscope}%
\begin{pgfscope}%
\pgfsys@transformshift{3.980574in}{0.768844in}%
\pgfsys@useobject{currentmarker}{}%
\end{pgfscope}%
\begin{pgfscope}%
\pgfsys@transformshift{4.000162in}{0.827804in}%
\pgfsys@useobject{currentmarker}{}%
\end{pgfscope}%
\begin{pgfscope}%
\pgfsys@transformshift{4.018390in}{0.884338in}%
\pgfsys@useobject{currentmarker}{}%
\end{pgfscope}%
\begin{pgfscope}%
\pgfsys@transformshift{4.018390in}{0.931867in}%
\pgfsys@useobject{currentmarker}{}%
\end{pgfscope}%
\begin{pgfscope}%
\pgfsys@transformshift{4.018390in}{0.968441in}%
\pgfsys@useobject{currentmarker}{}%
\end{pgfscope}%
\begin{pgfscope}%
\pgfsys@transformshift{4.019146in}{0.993921in}%
\pgfsys@useobject{currentmarker}{}%
\end{pgfscope}%
\begin{pgfscope}%
\pgfsys@transformshift{4.019519in}{1.009072in}%
\pgfsys@useobject{currentmarker}{}%
\end{pgfscope}%
\begin{pgfscope}%
\pgfsys@transformshift{4.019519in}{1.016157in}%
\pgfsys@useobject{currentmarker}{}%
\end{pgfscope}%
\begin{pgfscope}%
\pgfsys@transformshift{4.020325in}{1.015856in}%
\pgfsys@useobject{currentmarker}{}%
\end{pgfscope}%
\begin{pgfscope}%
\pgfsys@transformshift{4.021939in}{1.007485in}%
\pgfsys@useobject{currentmarker}{}%
\end{pgfscope}%
\begin{pgfscope}%
\pgfsys@transformshift{4.025625in}{0.989568in}%
\pgfsys@useobject{currentmarker}{}%
\end{pgfscope}%
\begin{pgfscope}%
\pgfsys@transformshift{4.033355in}{0.959753in}%
\pgfsys@useobject{currentmarker}{}%
\end{pgfscope}%
\begin{pgfscope}%
\pgfsys@transformshift{4.046277in}{0.916886in}%
\pgfsys@useobject{currentmarker}{}%
\end{pgfscope}%
\begin{pgfscope}%
\pgfsys@transformshift{4.059199in}{0.863091in}%
\pgfsys@useobject{currentmarker}{}%
\end{pgfscope}%
\begin{pgfscope}%
\pgfsys@transformshift{4.072121in}{0.801158in}%
\pgfsys@useobject{currentmarker}{}%
\end{pgfscope}%
\begin{pgfscope}%
\pgfsys@transformshift{4.085043in}{0.741948in}%
\pgfsys@useobject{currentmarker}{}%
\end{pgfscope}%
\begin{pgfscope}%
\pgfsys@transformshift{4.097965in}{0.697232in}%
\pgfsys@useobject{currentmarker}{}%
\end{pgfscope}%
\begin{pgfscope}%
\pgfsys@transformshift{4.110887in}{0.675424in}%
\pgfsys@useobject{currentmarker}{}%
\end{pgfscope}%
\begin{pgfscope}%
\pgfsys@transformshift{4.123809in}{0.693486in}%
\pgfsys@useobject{currentmarker}{}%
\end{pgfscope}%
\begin{pgfscope}%
\pgfsys@transformshift{4.136731in}{0.739608in}%
\pgfsys@useobject{currentmarker}{}%
\end{pgfscope}%
\begin{pgfscope}%
\pgfsys@transformshift{4.149653in}{0.795451in}%
\pgfsys@useobject{currentmarker}{}%
\end{pgfscope}%
\begin{pgfscope}%
\pgfsys@transformshift{4.162575in}{0.851689in}%
\pgfsys@useobject{currentmarker}{}%
\end{pgfscope}%
\begin{pgfscope}%
\pgfsys@transformshift{4.175497in}{0.902332in}%
\pgfsys@useobject{currentmarker}{}%
\end{pgfscope}%
\begin{pgfscope}%
\pgfsys@transformshift{4.188419in}{0.943886in}%
\pgfsys@useobject{currentmarker}{}%
\end{pgfscope}%
\begin{pgfscope}%
\pgfsys@transformshift{4.201341in}{0.974688in}%
\pgfsys@useobject{currentmarker}{}%
\end{pgfscope}%
\begin{pgfscope}%
\pgfsys@transformshift{4.214263in}{0.994187in}%
\pgfsys@useobject{currentmarker}{}%
\end{pgfscope}%
\begin{pgfscope}%
\pgfsys@transformshift{4.227185in}{1.003290in}%
\pgfsys@useobject{currentmarker}{}%
\end{pgfscope}%
\begin{pgfscope}%
\pgfsys@transformshift{4.240107in}{1.006069in}%
\pgfsys@useobject{currentmarker}{}%
\end{pgfscope}%
\begin{pgfscope}%
\pgfsys@transformshift{4.253029in}{1.009637in}%
\pgfsys@useobject{currentmarker}{}%
\end{pgfscope}%
\begin{pgfscope}%
\pgfsys@transformshift{4.265951in}{1.008545in}%
\pgfsys@useobject{currentmarker}{}%
\end{pgfscope}%
\begin{pgfscope}%
\pgfsys@transformshift{4.278873in}{0.999948in}%
\pgfsys@useobject{currentmarker}{}%
\end{pgfscope}%
\begin{pgfscope}%
\pgfsys@transformshift{4.291795in}{0.979160in}%
\pgfsys@useobject{currentmarker}{}%
\end{pgfscope}%
\begin{pgfscope}%
\pgfsys@transformshift{4.304717in}{0.943734in}%
\pgfsys@useobject{currentmarker}{}%
\end{pgfscope}%
\begin{pgfscope}%
\pgfsys@transformshift{4.317638in}{0.897590in}%
\pgfsys@useobject{currentmarker}{}%
\end{pgfscope}%
\begin{pgfscope}%
\pgfsys@transformshift{4.330560in}{0.849837in}%
\pgfsys@useobject{currentmarker}{}%
\end{pgfscope}%
\begin{pgfscope}%
\pgfsys@transformshift{4.343482in}{0.806686in}%
\pgfsys@useobject{currentmarker}{}%
\end{pgfscope}%
\begin{pgfscope}%
\pgfsys@transformshift{4.356404in}{0.768829in}%
\pgfsys@useobject{currentmarker}{}%
\end{pgfscope}%
\begin{pgfscope}%
\pgfsys@transformshift{4.369326in}{0.741132in}%
\pgfsys@useobject{currentmarker}{}%
\end{pgfscope}%
\begin{pgfscope}%
\pgfsys@transformshift{4.382248in}{0.723321in}%
\pgfsys@useobject{currentmarker}{}%
\end{pgfscope}%
\begin{pgfscope}%
\pgfsys@transformshift{4.395170in}{0.710482in}%
\pgfsys@useobject{currentmarker}{}%
\end{pgfscope}%
\begin{pgfscope}%
\pgfsys@transformshift{4.408092in}{0.701011in}%
\pgfsys@useobject{currentmarker}{}%
\end{pgfscope}%
\begin{pgfscope}%
\pgfsys@transformshift{4.421014in}{0.700190in}%
\pgfsys@useobject{currentmarker}{}%
\end{pgfscope}%
\begin{pgfscope}%
\pgfsys@transformshift{4.433936in}{0.705401in}%
\pgfsys@useobject{currentmarker}{}%
\end{pgfscope}%
\begin{pgfscope}%
\pgfsys@transformshift{4.446858in}{0.716522in}%
\pgfsys@useobject{currentmarker}{}%
\end{pgfscope}%
\begin{pgfscope}%
\pgfsys@transformshift{4.459780in}{0.729306in}%
\pgfsys@useobject{currentmarker}{}%
\end{pgfscope}%
\begin{pgfscope}%
\pgfsys@transformshift{4.472702in}{0.737728in}%
\pgfsys@useobject{currentmarker}{}%
\end{pgfscope}%
\begin{pgfscope}%
\pgfsys@transformshift{4.485624in}{0.739429in}%
\pgfsys@useobject{currentmarker}{}%
\end{pgfscope}%
\begin{pgfscope}%
\pgfsys@transformshift{4.498546in}{0.739660in}%
\pgfsys@useobject{currentmarker}{}%
\end{pgfscope}%
\begin{pgfscope}%
\pgfsys@transformshift{4.511468in}{0.732645in}%
\pgfsys@useobject{currentmarker}{}%
\end{pgfscope}%
\begin{pgfscope}%
\pgfsys@transformshift{4.524390in}{0.723309in}%
\pgfsys@useobject{currentmarker}{}%
\end{pgfscope}%
\begin{pgfscope}%
\pgfsys@transformshift{4.537312in}{0.717233in}%
\pgfsys@useobject{currentmarker}{}%
\end{pgfscope}%
\begin{pgfscope}%
\pgfsys@transformshift{4.550234in}{0.713398in}%
\pgfsys@useobject{currentmarker}{}%
\end{pgfscope}%
\begin{pgfscope}%
\pgfsys@transformshift{4.563156in}{0.704699in}%
\pgfsys@useobject{currentmarker}{}%
\end{pgfscope}%
\begin{pgfscope}%
\pgfsys@transformshift{4.576078in}{0.701715in}%
\pgfsys@useobject{currentmarker}{}%
\end{pgfscope}%
\begin{pgfscope}%
\pgfsys@transformshift{4.589000in}{0.705433in}%
\pgfsys@useobject{currentmarker}{}%
\end{pgfscope}%
\begin{pgfscope}%
\pgfsys@transformshift{4.601922in}{0.714817in}%
\pgfsys@useobject{currentmarker}{}%
\end{pgfscope}%
\begin{pgfscope}%
\pgfsys@transformshift{4.614844in}{0.727277in}%
\pgfsys@useobject{currentmarker}{}%
\end{pgfscope}%
\begin{pgfscope}%
\pgfsys@transformshift{4.627765in}{0.736847in}%
\pgfsys@useobject{currentmarker}{}%
\end{pgfscope}%
\begin{pgfscope}%
\pgfsys@transformshift{4.640687in}{0.739196in}%
\pgfsys@useobject{currentmarker}{}%
\end{pgfscope}%
\begin{pgfscope}%
\pgfsys@transformshift{4.653609in}{0.741253in}%
\pgfsys@useobject{currentmarker}{}%
\end{pgfscope}%
\begin{pgfscope}%
\pgfsys@transformshift{4.666531in}{0.736058in}%
\pgfsys@useobject{currentmarker}{}%
\end{pgfscope}%
\begin{pgfscope}%
\pgfsys@transformshift{4.679453in}{0.725532in}%
\pgfsys@useobject{currentmarker}{}%
\end{pgfscope}%
\begin{pgfscope}%
\pgfsys@transformshift{4.692375in}{0.715748in}%
\pgfsys@useobject{currentmarker}{}%
\end{pgfscope}%
\begin{pgfscope}%
\pgfsys@transformshift{4.705297in}{0.708897in}%
\pgfsys@useobject{currentmarker}{}%
\end{pgfscope}%
\begin{pgfscope}%
\pgfsys@transformshift{4.718219in}{0.700252in}%
\pgfsys@useobject{currentmarker}{}%
\end{pgfscope}%
\begin{pgfscope}%
\pgfsys@transformshift{4.733740in}{0.695033in}%
\pgfsys@useobject{currentmarker}{}%
\end{pgfscope}%
\begin{pgfscope}%
\pgfsys@transformshift{4.749261in}{0.699812in}%
\pgfsys@useobject{currentmarker}{}%
\end{pgfscope}%
\begin{pgfscope}%
\pgfsys@transformshift{4.764782in}{0.713522in}%
\pgfsys@useobject{currentmarker}{}%
\end{pgfscope}%
\begin{pgfscope}%
\pgfsys@transformshift{4.780303in}{0.733825in}%
\pgfsys@useobject{currentmarker}{}%
\end{pgfscope}%
\begin{pgfscope}%
\pgfsys@transformshift{4.795824in}{0.753781in}%
\pgfsys@useobject{currentmarker}{}%
\end{pgfscope}%
\begin{pgfscope}%
\pgfsys@transformshift{4.811344in}{0.766163in}%
\pgfsys@useobject{currentmarker}{}%
\end{pgfscope}%
\begin{pgfscope}%
\pgfsys@transformshift{4.826865in}{0.772374in}%
\pgfsys@useobject{currentmarker}{}%
\end{pgfscope}%
\begin{pgfscope}%
\pgfsys@transformshift{4.842386in}{0.770313in}%
\pgfsys@useobject{currentmarker}{}%
\end{pgfscope}%
\begin{pgfscope}%
\pgfsys@transformshift{4.857907in}{0.757734in}%
\pgfsys@useobject{currentmarker}{}%
\end{pgfscope}%
\begin{pgfscope}%
\pgfsys@transformshift{4.873428in}{0.738083in}%
\pgfsys@useobject{currentmarker}{}%
\end{pgfscope}%
\begin{pgfscope}%
\pgfsys@transformshift{4.888949in}{0.714091in}%
\pgfsys@useobject{currentmarker}{}%
\end{pgfscope}%
\begin{pgfscope}%
\pgfsys@transformshift{4.904470in}{0.699491in}%
\pgfsys@useobject{currentmarker}{}%
\end{pgfscope}%
\begin{pgfscope}%
\pgfsys@transformshift{4.919991in}{0.712972in}%
\pgfsys@useobject{currentmarker}{}%
\end{pgfscope}%
\begin{pgfscope}%
\pgfsys@transformshift{4.935511in}{0.737334in}%
\pgfsys@useobject{currentmarker}{}%
\end{pgfscope}%
\begin{pgfscope}%
\pgfsys@transformshift{4.951032in}{0.764111in}%
\pgfsys@useobject{currentmarker}{}%
\end{pgfscope}%
\begin{pgfscope}%
\pgfsys@transformshift{4.966553in}{0.786009in}%
\pgfsys@useobject{currentmarker}{}%
\end{pgfscope}%
\begin{pgfscope}%
\pgfsys@transformshift{4.982074in}{0.799531in}%
\pgfsys@useobject{currentmarker}{}%
\end{pgfscope}%
\begin{pgfscope}%
\pgfsys@transformshift{4.997595in}{0.804808in}%
\pgfsys@useobject{currentmarker}{}%
\end{pgfscope}%
\begin{pgfscope}%
\pgfsys@transformshift{5.013116in}{0.800273in}%
\pgfsys@useobject{currentmarker}{}%
\end{pgfscope}%
\begin{pgfscope}%
\pgfsys@transformshift{5.028637in}{0.784818in}%
\pgfsys@useobject{currentmarker}{}%
\end{pgfscope}%
\begin{pgfscope}%
\pgfsys@transformshift{5.044158in}{0.760716in}%
\pgfsys@useobject{currentmarker}{}%
\end{pgfscope}%
\begin{pgfscope}%
\pgfsys@transformshift{5.059678in}{0.730200in}%
\pgfsys@useobject{currentmarker}{}%
\end{pgfscope}%
\begin{pgfscope}%
\pgfsys@transformshift{5.075199in}{0.709347in}%
\pgfsys@useobject{currentmarker}{}%
\end{pgfscope}%
\begin{pgfscope}%
\pgfsys@transformshift{5.090720in}{0.720287in}%
\pgfsys@useobject{currentmarker}{}%
\end{pgfscope}%
\begin{pgfscope}%
\pgfsys@transformshift{5.106241in}{0.721935in}%
\pgfsys@useobject{currentmarker}{}%
\end{pgfscope}%
\begin{pgfscope}%
\pgfsys@transformshift{5.121762in}{0.715721in}%
\pgfsys@useobject{currentmarker}{}%
\end{pgfscope}%
\begin{pgfscope}%
\pgfsys@transformshift{5.137283in}{0.709705in}%
\pgfsys@useobject{currentmarker}{}%
\end{pgfscope}%
\begin{pgfscope}%
\pgfsys@transformshift{5.152804in}{0.713932in}%
\pgfsys@useobject{currentmarker}{}%
\end{pgfscope}%
\begin{pgfscope}%
\pgfsys@transformshift{5.168325in}{0.706287in}%
\pgfsys@useobject{currentmarker}{}%
\end{pgfscope}%
\begin{pgfscope}%
\pgfsys@transformshift{5.183845in}{0.704340in}%
\pgfsys@useobject{currentmarker}{}%
\end{pgfscope}%
\begin{pgfscope}%
\pgfsys@transformshift{5.199366in}{0.718944in}%
\pgfsys@useobject{currentmarker}{}%
\end{pgfscope}%
\begin{pgfscope}%
\pgfsys@transformshift{5.206896in}{0.729059in}%
\pgfsys@useobject{currentmarker}{}%
\end{pgfscope}%
\end{pgfscope}%
\begin{pgfscope}%
\pgfpathrectangle{\pgfqpoint{0.750000in}{0.500000in}}{\pgfqpoint{4.650000in}{3.020000in}}%
\pgfusepath{clip}%
\pgfsetrectcap%
\pgfsetroundjoin%
\pgfsetlinewidth{1.505625pt}%
\definecolor{currentstroke}{rgb}{1.000000,0.000000,0.000000}%
\pgfsetstrokecolor{currentstroke}%
\pgfsetdash{}{0pt}%
\pgfpathmoveto{\pgfqpoint{1.355390in}{0.500000in}}%
\pgfpathlineto{\pgfqpoint{1.355390in}{3.520000in}}%
\pgfusepath{stroke}%
\end{pgfscope}%
\begin{pgfscope}%
\pgfpathrectangle{\pgfqpoint{0.750000in}{0.500000in}}{\pgfqpoint{4.650000in}{3.020000in}}%
\pgfusepath{clip}%
\pgfsetrectcap%
\pgfsetroundjoin%
\pgfsetlinewidth{1.505625pt}%
\definecolor{currentstroke}{rgb}{1.000000,0.000000,0.000000}%
\pgfsetstrokecolor{currentstroke}%
\pgfsetdash{}{0pt}%
\pgfpathmoveto{\pgfqpoint{1.856553in}{0.500000in}}%
\pgfpathlineto{\pgfqpoint{1.856553in}{3.520000in}}%
\pgfusepath{stroke}%
\end{pgfscope}%
\begin{pgfscope}%
\pgfpathrectangle{\pgfqpoint{0.750000in}{0.500000in}}{\pgfqpoint{4.650000in}{3.020000in}}%
\pgfusepath{clip}%
\pgfsetrectcap%
\pgfsetroundjoin%
\pgfsetlinewidth{1.505625pt}%
\definecolor{currentstroke}{rgb}{1.000000,0.000000,0.000000}%
\pgfsetstrokecolor{currentstroke}%
\pgfsetdash{}{0pt}%
\pgfpathmoveto{\pgfqpoint{2.597276in}{0.500000in}}%
\pgfpathlineto{\pgfqpoint{2.597276in}{3.520000in}}%
\pgfusepath{stroke}%
\end{pgfscope}%
\begin{pgfscope}%
\pgfpathrectangle{\pgfqpoint{0.750000in}{0.500000in}}{\pgfqpoint{4.650000in}{3.020000in}}%
\pgfusepath{clip}%
\pgfsetrectcap%
\pgfsetroundjoin%
\pgfsetlinewidth{1.505625pt}%
\definecolor{currentstroke}{rgb}{1.000000,0.000000,0.000000}%
\pgfsetstrokecolor{currentstroke}%
\pgfsetdash{}{0pt}%
\pgfpathmoveto{\pgfqpoint{3.278636in}{0.500000in}}%
\pgfpathlineto{\pgfqpoint{3.278636in}{3.520000in}}%
\pgfusepath{stroke}%
\end{pgfscope}%
\begin{pgfscope}%
\pgfpathrectangle{\pgfqpoint{0.750000in}{0.500000in}}{\pgfqpoint{4.650000in}{3.020000in}}%
\pgfusepath{clip}%
\pgfsetrectcap%
\pgfsetroundjoin%
\pgfsetlinewidth{1.505625pt}%
\definecolor{currentstroke}{rgb}{1.000000,0.000000,0.000000}%
\pgfsetstrokecolor{currentstroke}%
\pgfsetdash{}{0pt}%
\pgfpathmoveto{\pgfqpoint{4.162575in}{0.500000in}}%
\pgfpathlineto{\pgfqpoint{4.162575in}{3.520000in}}%
\pgfusepath{stroke}%
\end{pgfscope}%
\begin{pgfscope}%
\pgfsetrectcap%
\pgfsetmiterjoin%
\pgfsetlinewidth{0.803000pt}%
\definecolor{currentstroke}{rgb}{0.000000,0.000000,0.000000}%
\pgfsetstrokecolor{currentstroke}%
\pgfsetdash{}{0pt}%
\pgfpathmoveto{\pgfqpoint{0.750000in}{0.500000in}}%
\pgfpathlineto{\pgfqpoint{0.750000in}{3.520000in}}%
\pgfusepath{stroke}%
\end{pgfscope}%
\begin{pgfscope}%
\pgfsetrectcap%
\pgfsetmiterjoin%
\pgfsetlinewidth{0.803000pt}%
\definecolor{currentstroke}{rgb}{0.000000,0.000000,0.000000}%
\pgfsetstrokecolor{currentstroke}%
\pgfsetdash{}{0pt}%
\pgfpathmoveto{\pgfqpoint{5.400000in}{0.500000in}}%
\pgfpathlineto{\pgfqpoint{5.400000in}{3.520000in}}%
\pgfusepath{stroke}%
\end{pgfscope}%
\begin{pgfscope}%
\pgfsetrectcap%
\pgfsetmiterjoin%
\pgfsetlinewidth{0.803000pt}%
\definecolor{currentstroke}{rgb}{0.000000,0.000000,0.000000}%
\pgfsetstrokecolor{currentstroke}%
\pgfsetdash{}{0pt}%
\pgfpathmoveto{\pgfqpoint{0.750000in}{0.500000in}}%
\pgfpathlineto{\pgfqpoint{5.400000in}{0.500000in}}%
\pgfusepath{stroke}%
\end{pgfscope}%
\begin{pgfscope}%
\pgfsetrectcap%
\pgfsetmiterjoin%
\pgfsetlinewidth{0.803000pt}%
\definecolor{currentstroke}{rgb}{0.000000,0.000000,0.000000}%
\pgfsetstrokecolor{currentstroke}%
\pgfsetdash{}{0pt}%
\pgfpathmoveto{\pgfqpoint{0.750000in}{3.520000in}}%
\pgfpathlineto{\pgfqpoint{5.400000in}{3.520000in}}%
\pgfusepath{stroke}%
\end{pgfscope}%
\begin{pgfscope}%
\pgfsetbuttcap%
\pgfsetroundjoin%
\definecolor{currentfill}{rgb}{0.000000,0.000000,0.000000}%
\pgfsetfillcolor{currentfill}%
\pgfsetlinewidth{0.803000pt}%
\definecolor{currentstroke}{rgb}{0.000000,0.000000,0.000000}%
\pgfsetstrokecolor{currentstroke}%
\pgfsetdash{}{0pt}%
\pgfsys@defobject{currentmarker}{\pgfqpoint{0.000000in}{0.000000in}}{\pgfqpoint{0.048611in}{0.000000in}}{%
\pgfpathmoveto{\pgfqpoint{0.000000in}{0.000000in}}%
\pgfpathlineto{\pgfqpoint{0.048611in}{0.000000in}}%
\pgfusepath{stroke,fill}%
}%
\begin{pgfscope}%
\pgfsys@transformshift{5.400000in}{0.637268in}%
\pgfsys@useobject{currentmarker}{}%
\end{pgfscope}%
\end{pgfscope}%
\begin{pgfscope}%
\definecolor{textcolor}{rgb}{0.000000,0.000000,0.000000}%
\pgfsetstrokecolor{textcolor}%
\pgfsetfillcolor{textcolor}%
\pgftext[x=5.497222in, y=0.589043in, left, base]{\color{textcolor}\rmfamily\fontsize{10.000000}{12.000000}\selectfont \(\displaystyle {0}\)}%
\end{pgfscope}%
\begin{pgfscope}%
\pgfsetbuttcap%
\pgfsetroundjoin%
\definecolor{currentfill}{rgb}{0.000000,0.000000,0.000000}%
\pgfsetfillcolor{currentfill}%
\pgfsetlinewidth{0.803000pt}%
\definecolor{currentstroke}{rgb}{0.000000,0.000000,0.000000}%
\pgfsetstrokecolor{currentstroke}%
\pgfsetdash{}{0pt}%
\pgfsys@defobject{currentmarker}{\pgfqpoint{0.000000in}{0.000000in}}{\pgfqpoint{0.048611in}{0.000000in}}{%
\pgfpathmoveto{\pgfqpoint{0.000000in}{0.000000in}}%
\pgfpathlineto{\pgfqpoint{0.048611in}{0.000000in}}%
\pgfusepath{stroke,fill}%
}%
\begin{pgfscope}%
\pgfsys@transformshift{5.400000in}{1.068578in}%
\pgfsys@useobject{currentmarker}{}%
\end{pgfscope}%
\end{pgfscope}%
\begin{pgfscope}%
\definecolor{textcolor}{rgb}{0.000000,0.000000,0.000000}%
\pgfsetstrokecolor{textcolor}%
\pgfsetfillcolor{textcolor}%
\pgftext[x=5.497222in, y=1.020353in, left, base]{\color{textcolor}\rmfamily\fontsize{10.000000}{12.000000}\selectfont \(\displaystyle {1}\)}%
\end{pgfscope}%
\begin{pgfscope}%
\pgfsetbuttcap%
\pgfsetroundjoin%
\definecolor{currentfill}{rgb}{0.000000,0.000000,0.000000}%
\pgfsetfillcolor{currentfill}%
\pgfsetlinewidth{0.803000pt}%
\definecolor{currentstroke}{rgb}{0.000000,0.000000,0.000000}%
\pgfsetstrokecolor{currentstroke}%
\pgfsetdash{}{0pt}%
\pgfsys@defobject{currentmarker}{\pgfqpoint{0.000000in}{0.000000in}}{\pgfqpoint{0.048611in}{0.000000in}}{%
\pgfpathmoveto{\pgfqpoint{0.000000in}{0.000000in}}%
\pgfpathlineto{\pgfqpoint{0.048611in}{0.000000in}}%
\pgfusepath{stroke,fill}%
}%
\begin{pgfscope}%
\pgfsys@transformshift{5.400000in}{1.499888in}%
\pgfsys@useobject{currentmarker}{}%
\end{pgfscope}%
\end{pgfscope}%
\begin{pgfscope}%
\definecolor{textcolor}{rgb}{0.000000,0.000000,0.000000}%
\pgfsetstrokecolor{textcolor}%
\pgfsetfillcolor{textcolor}%
\pgftext[x=5.497222in, y=1.451663in, left, base]{\color{textcolor}\rmfamily\fontsize{10.000000}{12.000000}\selectfont \(\displaystyle {2}\)}%
\end{pgfscope}%
\begin{pgfscope}%
\pgfsetbuttcap%
\pgfsetroundjoin%
\definecolor{currentfill}{rgb}{0.000000,0.000000,0.000000}%
\pgfsetfillcolor{currentfill}%
\pgfsetlinewidth{0.803000pt}%
\definecolor{currentstroke}{rgb}{0.000000,0.000000,0.000000}%
\pgfsetstrokecolor{currentstroke}%
\pgfsetdash{}{0pt}%
\pgfsys@defobject{currentmarker}{\pgfqpoint{0.000000in}{0.000000in}}{\pgfqpoint{0.048611in}{0.000000in}}{%
\pgfpathmoveto{\pgfqpoint{0.000000in}{0.000000in}}%
\pgfpathlineto{\pgfqpoint{0.048611in}{0.000000in}}%
\pgfusepath{stroke,fill}%
}%
\begin{pgfscope}%
\pgfsys@transformshift{5.400000in}{1.931198in}%
\pgfsys@useobject{currentmarker}{}%
\end{pgfscope}%
\end{pgfscope}%
\begin{pgfscope}%
\definecolor{textcolor}{rgb}{0.000000,0.000000,0.000000}%
\pgfsetstrokecolor{textcolor}%
\pgfsetfillcolor{textcolor}%
\pgftext[x=5.497222in, y=1.882972in, left, base]{\color{textcolor}\rmfamily\fontsize{10.000000}{12.000000}\selectfont \(\displaystyle {3}\)}%
\end{pgfscope}%
\begin{pgfscope}%
\pgfsetbuttcap%
\pgfsetroundjoin%
\definecolor{currentfill}{rgb}{0.000000,0.000000,0.000000}%
\pgfsetfillcolor{currentfill}%
\pgfsetlinewidth{0.803000pt}%
\definecolor{currentstroke}{rgb}{0.000000,0.000000,0.000000}%
\pgfsetstrokecolor{currentstroke}%
\pgfsetdash{}{0pt}%
\pgfsys@defobject{currentmarker}{\pgfqpoint{0.000000in}{0.000000in}}{\pgfqpoint{0.048611in}{0.000000in}}{%
\pgfpathmoveto{\pgfqpoint{0.000000in}{0.000000in}}%
\pgfpathlineto{\pgfqpoint{0.048611in}{0.000000in}}%
\pgfusepath{stroke,fill}%
}%
\begin{pgfscope}%
\pgfsys@transformshift{5.400000in}{2.362507in}%
\pgfsys@useobject{currentmarker}{}%
\end{pgfscope}%
\end{pgfscope}%
\begin{pgfscope}%
\definecolor{textcolor}{rgb}{0.000000,0.000000,0.000000}%
\pgfsetstrokecolor{textcolor}%
\pgfsetfillcolor{textcolor}%
\pgftext[x=5.497222in, y=2.314282in, left, base]{\color{textcolor}\rmfamily\fontsize{10.000000}{12.000000}\selectfont \(\displaystyle {4}\)}%
\end{pgfscope}%
\begin{pgfscope}%
\pgfsetbuttcap%
\pgfsetroundjoin%
\definecolor{currentfill}{rgb}{0.000000,0.000000,0.000000}%
\pgfsetfillcolor{currentfill}%
\pgfsetlinewidth{0.803000pt}%
\definecolor{currentstroke}{rgb}{0.000000,0.000000,0.000000}%
\pgfsetstrokecolor{currentstroke}%
\pgfsetdash{}{0pt}%
\pgfsys@defobject{currentmarker}{\pgfqpoint{0.000000in}{0.000000in}}{\pgfqpoint{0.048611in}{0.000000in}}{%
\pgfpathmoveto{\pgfqpoint{0.000000in}{0.000000in}}%
\pgfpathlineto{\pgfqpoint{0.048611in}{0.000000in}}%
\pgfusepath{stroke,fill}%
}%
\begin{pgfscope}%
\pgfsys@transformshift{5.400000in}{2.793817in}%
\pgfsys@useobject{currentmarker}{}%
\end{pgfscope}%
\end{pgfscope}%
\begin{pgfscope}%
\definecolor{textcolor}{rgb}{0.000000,0.000000,0.000000}%
\pgfsetstrokecolor{textcolor}%
\pgfsetfillcolor{textcolor}%
\pgftext[x=5.497222in, y=2.745592in, left, base]{\color{textcolor}\rmfamily\fontsize{10.000000}{12.000000}\selectfont \(\displaystyle {5}\)}%
\end{pgfscope}%
\begin{pgfscope}%
\pgfsetbuttcap%
\pgfsetroundjoin%
\definecolor{currentfill}{rgb}{0.000000,0.000000,0.000000}%
\pgfsetfillcolor{currentfill}%
\pgfsetlinewidth{0.803000pt}%
\definecolor{currentstroke}{rgb}{0.000000,0.000000,0.000000}%
\pgfsetstrokecolor{currentstroke}%
\pgfsetdash{}{0pt}%
\pgfsys@defobject{currentmarker}{\pgfqpoint{0.000000in}{0.000000in}}{\pgfqpoint{0.048611in}{0.000000in}}{%
\pgfpathmoveto{\pgfqpoint{0.000000in}{0.000000in}}%
\pgfpathlineto{\pgfqpoint{0.048611in}{0.000000in}}%
\pgfusepath{stroke,fill}%
}%
\begin{pgfscope}%
\pgfsys@transformshift{5.400000in}{3.225127in}%
\pgfsys@useobject{currentmarker}{}%
\end{pgfscope}%
\end{pgfscope}%
\begin{pgfscope}%
\definecolor{textcolor}{rgb}{0.000000,0.000000,0.000000}%
\pgfsetstrokecolor{textcolor}%
\pgfsetfillcolor{textcolor}%
\pgftext[x=5.497222in, y=3.176902in, left, base]{\color{textcolor}\rmfamily\fontsize{10.000000}{12.000000}\selectfont \(\displaystyle {6}\)}%
\end{pgfscope}%
\begin{pgfscope}%
\definecolor{textcolor}{rgb}{0.000000,0.000000,0.000000}%
\pgfsetstrokecolor{textcolor}%
\pgfsetfillcolor{textcolor}%
\pgftext[x=5.622223in,y=2.010000in,,top,rotate=90.000000]{\color{textcolor}\rmfamily\fontsize{10.000000}{12.000000}\selectfont Pressure (\(\displaystyle Pa\))}%
\end{pgfscope}%
\begin{pgfscope}%
\definecolor{textcolor}{rgb}{0.000000,0.000000,0.000000}%
\pgfsetstrokecolor{textcolor}%
\pgfsetfillcolor{textcolor}%
\pgftext[x=5.400000in,y=3.561667in,right,base]{\color{textcolor}\rmfamily\fontsize{10.000000}{12.000000}\selectfont \(\displaystyle \times{10^{6}}{}\)}%
\end{pgfscope}%
\begin{pgfscope}%
\pgfpathrectangle{\pgfqpoint{0.750000in}{0.500000in}}{\pgfqpoint{4.650000in}{3.020000in}}%
\pgfusepath{clip}%
\pgfsetbuttcap%
\pgfsetroundjoin%
\pgfsetlinewidth{1.505625pt}%
\definecolor{currentstroke}{rgb}{0.000000,0.000000,1.000000}%
\pgfsetstrokecolor{currentstroke}%
\pgfsetdash{{5.550000pt}{2.400000pt}}{0.000000pt}%
\pgfpathmoveto{\pgfqpoint{1.344812in}{0.637287in}}%
\pgfpathlineto{\pgfqpoint{1.347743in}{0.638594in}}%
\pgfpathlineto{\pgfqpoint{1.351299in}{0.643111in}}%
\pgfpathlineto{\pgfqpoint{1.355390in}{0.654130in}}%
\pgfpathlineto{\pgfqpoint{1.361526in}{0.681753in}}%
\pgfpathlineto{\pgfqpoint{1.374261in}{0.758167in}}%
\pgfpathlineto{\pgfqpoint{1.386634in}{0.851306in}}%
\pgfpathlineto{\pgfqpoint{1.405827in}{1.045485in}}%
\pgfpathlineto{\pgfqpoint{1.463705in}{1.730415in}}%
\pgfpathlineto{\pgfqpoint{1.472337in}{1.791497in}}%
\pgfpathlineto{\pgfqpoint{1.482936in}{1.835409in}}%
\pgfpathlineto{\pgfqpoint{1.492816in}{1.861646in}}%
\pgfpathlineto{\pgfqpoint{1.492816in}{1.871444in}}%
\pgfpathlineto{\pgfqpoint{1.492816in}{1.871444in}}%
\pgfpathlineto{\pgfqpoint{1.505557in}{1.871005in}}%
\pgfpathlineto{\pgfqpoint{1.510739in}{1.863211in}}%
\pgfpathlineto{\pgfqpoint{1.525486in}{1.835409in}}%
\pgfpathlineto{\pgfqpoint{1.558099in}{1.835409in}}%
\pgfpathlineto{\pgfqpoint{1.568970in}{1.860774in}}%
\pgfpathlineto{\pgfqpoint{1.601583in}{2.592150in}}%
\pgfpathlineto{\pgfqpoint{1.612454in}{2.795399in}}%
\pgfpathlineto{\pgfqpoint{1.623325in}{2.957807in}}%
\pgfpathlineto{\pgfqpoint{1.636719in}{3.082279in}}%
\pgfpathlineto{\pgfqpoint{1.650113in}{3.163675in}}%
\pgfpathlineto{\pgfqpoint{1.663507in}{3.204981in}}%
\pgfpathlineto{\pgfqpoint{1.676902in}{3.216665in}}%
\pgfpathlineto{\pgfqpoint{1.690296in}{3.216665in}}%
\pgfpathlineto{\pgfqpoint{1.703690in}{3.221246in}}%
\pgfpathlineto{\pgfqpoint{1.717084in}{3.247362in}}%
\pgfpathlineto{\pgfqpoint{1.757267in}{3.378414in}}%
\pgfpathlineto{\pgfqpoint{1.770661in}{3.382727in}}%
\pgfpathlineto{\pgfqpoint{1.789942in}{3.382727in}}%
\pgfpathlineto{\pgfqpoint{1.789942in}{3.382727in}}%
\pgfpathlineto{\pgfqpoint{1.793487in}{3.352694in}}%
\pgfpathlineto{\pgfqpoint{1.800577in}{3.274860in}}%
\pgfpathlineto{\pgfqpoint{1.869308in}{3.274860in}}%
\pgfpathlineto{\pgfqpoint{1.894817in}{3.032038in}}%
\pgfpathlineto{\pgfqpoint{1.907572in}{2.912496in}}%
\pgfpathlineto{\pgfqpoint{1.920326in}{2.823951in}}%
\pgfpathlineto{\pgfqpoint{1.933081in}{2.776630in}}%
\pgfpathlineto{\pgfqpoint{1.945836in}{2.764748in}}%
\pgfpathlineto{\pgfqpoint{1.961644in}{2.764748in}}%
\pgfpathlineto{\pgfqpoint{1.977452in}{2.751264in}}%
\pgfpathlineto{\pgfqpoint{1.993261in}{2.703424in}}%
\pgfpathlineto{\pgfqpoint{2.009069in}{2.615703in}}%
\pgfpathlineto{\pgfqpoint{2.024878in}{2.497458in}}%
\pgfpathlineto{\pgfqpoint{2.040686in}{2.366005in}}%
\pgfpathlineto{\pgfqpoint{2.056494in}{2.376826in}}%
\pgfpathlineto{\pgfqpoint{2.088111in}{2.536006in}}%
\pgfpathlineto{\pgfqpoint{2.103919in}{2.582056in}}%
\pgfpathlineto{\pgfqpoint{2.167153in}{2.582056in}}%
\pgfpathlineto{\pgfqpoint{2.182961in}{2.579894in}}%
\pgfpathlineto{\pgfqpoint{2.198770in}{2.526261in}}%
\pgfpathlineto{\pgfqpoint{2.214578in}{2.430108in}}%
\pgfpathlineto{\pgfqpoint{2.277812in}{2.430108in}}%
\pgfpathlineto{\pgfqpoint{2.293620in}{2.405504in}}%
\pgfpathlineto{\pgfqpoint{2.309428in}{2.311505in}}%
\pgfpathlineto{\pgfqpoint{2.325237in}{2.297913in}}%
\pgfpathlineto{\pgfqpoint{2.341045in}{2.297856in}}%
\pgfpathlineto{\pgfqpoint{2.356853in}{2.232938in}}%
\pgfpathlineto{\pgfqpoint{2.376266in}{2.203236in}}%
\pgfpathlineto{\pgfqpoint{2.384195in}{2.203236in}}%
\pgfpathlineto{\pgfqpoint{2.384195in}{2.203236in}}%
\pgfpathlineto{\pgfqpoint{2.384195in}{2.147116in}}%
\pgfpathlineto{\pgfqpoint{2.399554in}{2.071207in}}%
\pgfpathlineto{\pgfqpoint{2.409792in}{1.976877in}}%
\pgfpathlineto{\pgfqpoint{2.427437in}{1.940172in}}%
\pgfpathlineto{\pgfqpoint{2.445081in}{1.940172in}}%
\pgfpathlineto{\pgfqpoint{2.462726in}{1.837760in}}%
\pgfpathlineto{\pgfqpoint{2.532758in}{1.837760in}}%
\pgfpathlineto{\pgfqpoint{2.533532in}{1.712847in}}%
\pgfpathlineto{\pgfqpoint{2.540260in}{1.375159in}}%
\pgfpathlineto{\pgfqpoint{2.543158in}{1.370053in}}%
\pgfpathlineto{\pgfqpoint{2.569439in}{1.370053in}}%
\pgfpathlineto{\pgfqpoint{2.583357in}{1.341037in}}%
\pgfpathlineto{\pgfqpoint{2.597276in}{1.474016in}}%
\pgfpathlineto{\pgfqpoint{2.625113in}{1.972586in}}%
\pgfpathlineto{\pgfqpoint{2.639031in}{2.182694in}}%
\pgfpathlineto{\pgfqpoint{2.652950in}{2.347258in}}%
\pgfpathlineto{\pgfqpoint{2.666868in}{2.462841in}}%
\pgfpathlineto{\pgfqpoint{2.680787in}{2.531357in}}%
\pgfpathlineto{\pgfqpoint{2.682507in}{2.577046in}}%
\pgfpathlineto{\pgfqpoint{2.687292in}{2.575185in}}%
\pgfpathlineto{\pgfqpoint{2.690676in}{2.570825in}}%
\pgfpathlineto{\pgfqpoint{2.694061in}{2.563385in}}%
\pgfpathlineto{\pgfqpoint{2.722726in}{2.563385in}}%
\pgfpathlineto{\pgfqpoint{2.733938in}{2.553589in}}%
\pgfpathlineto{\pgfqpoint{2.745150in}{2.510319in}}%
\pgfpathlineto{\pgfqpoint{2.756363in}{2.425154in}}%
\pgfpathlineto{\pgfqpoint{2.767575in}{2.282799in}}%
\pgfpathlineto{\pgfqpoint{2.778788in}{2.336524in}}%
\pgfpathlineto{\pgfqpoint{2.790000in}{2.571625in}}%
\pgfpathlineto{\pgfqpoint{2.801213in}{2.751597in}}%
\pgfpathlineto{\pgfqpoint{2.812425in}{2.876502in}}%
\pgfpathlineto{\pgfqpoint{2.832783in}{3.002793in}}%
\pgfpathlineto{\pgfqpoint{2.847054in}{3.002793in}}%
\pgfpathlineto{\pgfqpoint{2.854369in}{2.999409in}}%
\pgfpathlineto{\pgfqpoint{2.861684in}{2.991724in}}%
\pgfpathlineto{\pgfqpoint{2.873089in}{2.974397in}}%
\pgfpathlineto{\pgfqpoint{2.968323in}{2.974397in}}%
\pgfpathlineto{\pgfqpoint{2.978448in}{2.933893in}}%
\pgfpathlineto{\pgfqpoint{2.978448in}{2.730964in}}%
\pgfpathlineto{\pgfqpoint{2.983199in}{2.546584in}}%
\pgfpathlineto{\pgfqpoint{3.002202in}{2.083895in}}%
\pgfpathlineto{\pgfqpoint{3.018558in}{1.888039in}}%
\pgfpathlineto{\pgfqpoint{3.034914in}{1.828302in}}%
\pgfpathlineto{\pgfqpoint{3.051270in}{1.833972in}}%
\pgfpathlineto{\pgfqpoint{3.067626in}{1.837256in}}%
\pgfpathlineto{\pgfqpoint{3.127011in}{1.837256in}}%
\pgfpathlineto{\pgfqpoint{3.127011in}{1.826831in}}%
\pgfpathlineto{\pgfqpoint{3.197987in}{1.826500in}}%
\pgfpathlineto{\pgfqpoint{3.218149in}{1.795289in}}%
\pgfpathlineto{\pgfqpoint{3.238311in}{1.726892in}}%
\pgfpathlineto{\pgfqpoint{3.278636in}{1.535389in}}%
\pgfpathlineto{\pgfqpoint{3.298798in}{1.467192in}}%
\pgfpathlineto{\pgfqpoint{3.318960in}{1.462580in}}%
\pgfpathlineto{\pgfqpoint{3.339122in}{1.462580in}}%
\pgfpathlineto{\pgfqpoint{3.359284in}{1.459274in}}%
\pgfpathlineto{\pgfqpoint{3.384093in}{1.429035in}}%
\pgfpathlineto{\pgfqpoint{3.433710in}{1.429035in}}%
\pgfpathlineto{\pgfqpoint{3.458518in}{1.418976in}}%
\pgfpathlineto{\pgfqpoint{3.483327in}{1.418976in}}%
\pgfpathlineto{\pgfqpoint{3.508136in}{1.386660in}}%
\pgfpathlineto{\pgfqpoint{3.532944in}{1.290419in}}%
\pgfpathlineto{\pgfqpoint{3.562994in}{1.259390in}}%
\pgfpathlineto{\pgfqpoint{3.623092in}{1.259390in}}%
\pgfpathlineto{\pgfqpoint{3.653142in}{1.255399in}}%
\pgfpathlineto{\pgfqpoint{3.683191in}{1.221284in}}%
\pgfpathlineto{\pgfqpoint{3.713241in}{1.197265in}}%
\pgfpathlineto{\pgfqpoint{3.721264in}{1.179708in}}%
\pgfpathlineto{\pgfqpoint{3.721264in}{1.075136in}}%
\pgfpathlineto{\pgfqpoint{3.734411in}{1.045357in}}%
\pgfpathlineto{\pgfqpoint{3.760705in}{1.040175in}}%
\pgfpathlineto{\pgfqpoint{3.787000in}{1.003923in}}%
\pgfpathlineto{\pgfqpoint{3.828820in}{0.955633in}}%
\pgfpathlineto{\pgfqpoint{3.869827in}{0.905779in}}%
\pgfpathlineto{\pgfqpoint{3.869827in}{0.875040in}}%
\pgfpathlineto{\pgfqpoint{3.871582in}{0.871798in}}%
\pgfpathlineto{\pgfqpoint{3.880966in}{0.827265in}}%
\pgfpathlineto{\pgfqpoint{3.890659in}{0.794326in}}%
\pgfpathlineto{\pgfqpoint{3.898287in}{0.775463in}}%
\pgfpathlineto{\pgfqpoint{3.909283in}{0.774678in}}%
\pgfpathlineto{\pgfqpoint{3.925341in}{0.798799in}}%
\pgfpathlineto{\pgfqpoint{3.941398in}{0.890929in}}%
\pgfpathlineto{\pgfqpoint{3.960986in}{1.039340in}}%
\pgfpathlineto{\pgfqpoint{3.980574in}{1.215965in}}%
\pgfpathlineto{\pgfqpoint{4.000162in}{1.378167in}}%
\pgfpathlineto{\pgfqpoint{4.018390in}{1.507709in}}%
\pgfpathlineto{\pgfqpoint{4.019519in}{1.634309in}}%
\pgfpathlineto{\pgfqpoint{4.072121in}{1.633730in}}%
\pgfpathlineto{\pgfqpoint{4.085043in}{1.632129in}}%
\pgfpathlineto{\pgfqpoint{4.110887in}{1.632129in}}%
\pgfpathlineto{\pgfqpoint{4.123809in}{1.740177in}}%
\pgfpathlineto{\pgfqpoint{4.149653in}{2.082517in}}%
\pgfpathlineto{\pgfqpoint{4.162575in}{2.216503in}}%
\pgfpathlineto{\pgfqpoint{4.175497in}{2.308474in}}%
\pgfpathlineto{\pgfqpoint{4.188419in}{2.360022in}}%
\pgfpathlineto{\pgfqpoint{4.201341in}{2.379968in}}%
\pgfpathlineto{\pgfqpoint{4.227185in}{2.384385in}}%
\pgfpathlineto{\pgfqpoint{4.240107in}{2.403096in}}%
\pgfpathlineto{\pgfqpoint{4.253029in}{2.448197in}}%
\pgfpathlineto{\pgfqpoint{4.291795in}{2.655402in}}%
\pgfpathlineto{\pgfqpoint{4.304717in}{2.686754in}}%
\pgfpathlineto{\pgfqpoint{4.356404in}{2.686754in}}%
\pgfpathlineto{\pgfqpoint{4.369326in}{2.679974in}}%
\pgfpathlineto{\pgfqpoint{4.382248in}{2.635708in}}%
\pgfpathlineto{\pgfqpoint{4.395170in}{2.562547in}}%
\pgfpathlineto{\pgfqpoint{4.433936in}{2.562547in}}%
\pgfpathlineto{\pgfqpoint{4.446858in}{2.585043in}}%
\pgfpathlineto{\pgfqpoint{4.459780in}{2.622541in}}%
\pgfpathlineto{\pgfqpoint{4.485624in}{2.623878in}}%
\pgfpathlineto{\pgfqpoint{4.524390in}{2.623878in}}%
\pgfpathlineto{\pgfqpoint{4.537312in}{2.588021in}}%
\pgfpathlineto{\pgfqpoint{4.550234in}{2.521704in}}%
\pgfpathlineto{\pgfqpoint{4.576078in}{2.520218in}}%
\pgfpathlineto{\pgfqpoint{4.601922in}{2.520218in}}%
\pgfpathlineto{\pgfqpoint{4.614844in}{2.556544in}}%
\pgfpathlineto{\pgfqpoint{4.627765in}{2.567455in}}%
\pgfpathlineto{\pgfqpoint{4.679453in}{2.567455in}}%
\pgfpathlineto{\pgfqpoint{4.692375in}{2.541850in}}%
\pgfpathlineto{\pgfqpoint{4.705297in}{2.484333in}}%
\pgfpathlineto{\pgfqpoint{4.718219in}{2.444492in}}%
\pgfpathlineto{\pgfqpoint{4.749261in}{2.444492in}}%
\pgfpathlineto{\pgfqpoint{4.764782in}{2.490565in}}%
\pgfpathlineto{\pgfqpoint{4.780303in}{2.566903in}}%
\pgfpathlineto{\pgfqpoint{4.795824in}{2.609684in}}%
\pgfpathlineto{\pgfqpoint{4.857907in}{2.609684in}}%
\pgfpathlineto{\pgfqpoint{4.873428in}{2.601094in}}%
\pgfpathlineto{\pgfqpoint{4.888949in}{2.544136in}}%
\pgfpathlineto{\pgfqpoint{4.904470in}{2.455915in}}%
\pgfpathlineto{\pgfqpoint{4.919991in}{2.513412in}}%
\pgfpathlineto{\pgfqpoint{4.935511in}{2.635969in}}%
\pgfpathlineto{\pgfqpoint{4.951032in}{2.730562in}}%
\pgfpathlineto{\pgfqpoint{4.966553in}{2.762550in}}%
\pgfpathlineto{\pgfqpoint{5.028637in}{2.762550in}}%
\pgfpathlineto{\pgfqpoint{5.044158in}{2.743737in}}%
\pgfpathlineto{\pgfqpoint{5.059678in}{2.656864in}}%
\pgfpathlineto{\pgfqpoint{5.075199in}{2.532176in}}%
\pgfpathlineto{\pgfqpoint{5.090720in}{2.567647in}}%
\pgfpathlineto{\pgfqpoint{5.152804in}{2.567647in}}%
\pgfpathlineto{\pgfqpoint{5.168325in}{2.551902in}}%
\pgfpathlineto{\pgfqpoint{5.206896in}{2.551902in}}%
\pgfpathlineto{\pgfqpoint{5.206896in}{2.551902in}}%
\pgfusepath{stroke}%
\end{pgfscope}%
\begin{pgfscope}%
\pgfpathrectangle{\pgfqpoint{0.750000in}{0.500000in}}{\pgfqpoint{4.650000in}{3.020000in}}%
\pgfusepath{clip}%
\pgfsetbuttcap%
\pgfsetroundjoin%
\pgfsetlinewidth{1.505625pt}%
\definecolor{currentstroke}{rgb}{0.750000,0.750000,0.000000}%
\pgfsetstrokecolor{currentstroke}%
\pgfsetdash{{5.550000pt}{2.400000pt}}{0.000000pt}%
\pgfpathmoveto{\pgfqpoint{1.344812in}{0.637274in}}%
\pgfpathlineto{\pgfqpoint{1.353344in}{0.638681in}}%
\pgfpathlineto{\pgfqpoint{1.359480in}{0.643429in}}%
\pgfpathlineto{\pgfqpoint{1.365617in}{0.653861in}}%
\pgfpathlineto{\pgfqpoint{1.374261in}{0.678704in}}%
\pgfpathlineto{\pgfqpoint{1.395057in}{0.754681in}}%
\pgfpathlineto{\pgfqpoint{1.437812in}{0.945095in}}%
\pgfpathlineto{\pgfqpoint{1.455074in}{1.054375in}}%
\pgfpathlineto{\pgfqpoint{1.492816in}{1.313539in}}%
\pgfpathlineto{\pgfqpoint{1.492816in}{1.442957in}}%
\pgfpathlineto{\pgfqpoint{1.494636in}{1.503986in}}%
\pgfpathlineto{\pgfqpoint{1.505557in}{1.655437in}}%
\pgfpathlineto{\pgfqpoint{1.510739in}{1.688465in}}%
\pgfpathlineto{\pgfqpoint{1.517027in}{1.708653in}}%
\pgfpathlineto{\pgfqpoint{1.525486in}{1.714122in}}%
\pgfpathlineto{\pgfqpoint{1.536357in}{1.706847in}}%
\pgfpathlineto{\pgfqpoint{1.547228in}{1.694210in}}%
\pgfpathlineto{\pgfqpoint{1.558099in}{1.689935in}}%
\pgfpathlineto{\pgfqpoint{1.568970in}{1.698083in}}%
\pgfpathlineto{\pgfqpoint{1.579841in}{1.717685in}}%
\pgfpathlineto{\pgfqpoint{1.590712in}{1.750255in}}%
\pgfpathlineto{\pgfqpoint{1.601583in}{1.797895in}}%
\pgfpathlineto{\pgfqpoint{1.612454in}{1.859963in}}%
\pgfpathlineto{\pgfqpoint{1.663507in}{2.219166in}}%
\pgfpathlineto{\pgfqpoint{1.690296in}{2.418504in}}%
\pgfpathlineto{\pgfqpoint{1.717084in}{2.614766in}}%
\pgfpathlineto{\pgfqpoint{1.770661in}{3.034706in}}%
\pgfpathlineto{\pgfqpoint{1.784056in}{3.112679in}}%
\pgfpathlineto{\pgfqpoint{1.789942in}{3.168969in}}%
\pgfpathlineto{\pgfqpoint{1.789942in}{3.192949in}}%
\pgfpathlineto{\pgfqpoint{1.789942in}{3.186090in}}%
\pgfpathlineto{\pgfqpoint{1.793487in}{3.153869in}}%
\pgfpathlineto{\pgfqpoint{1.807667in}{3.049347in}}%
\pgfpathlineto{\pgfqpoint{1.818289in}{2.997060in}}%
\pgfpathlineto{\pgfqpoint{1.831044in}{2.952109in}}%
\pgfpathlineto{\pgfqpoint{1.856553in}{2.873132in}}%
\pgfpathlineto{\pgfqpoint{1.869308in}{2.823595in}}%
\pgfpathlineto{\pgfqpoint{1.882062in}{2.758354in}}%
\pgfpathlineto{\pgfqpoint{1.907572in}{2.602871in}}%
\pgfpathlineto{\pgfqpoint{1.933081in}{2.447941in}}%
\pgfpathlineto{\pgfqpoint{1.945836in}{2.389656in}}%
\pgfpathlineto{\pgfqpoint{1.961644in}{2.347656in}}%
\pgfpathlineto{\pgfqpoint{1.993261in}{2.288214in}}%
\pgfpathlineto{\pgfqpoint{2.009069in}{2.262544in}}%
\pgfpathlineto{\pgfqpoint{2.024878in}{2.242395in}}%
\pgfpathlineto{\pgfqpoint{2.040686in}{2.231164in}}%
\pgfpathlineto{\pgfqpoint{2.056494in}{2.231840in}}%
\pgfpathlineto{\pgfqpoint{2.072303in}{2.245239in}}%
\pgfpathlineto{\pgfqpoint{2.088111in}{2.269086in}}%
\pgfpathlineto{\pgfqpoint{2.119728in}{2.322667in}}%
\pgfpathlineto{\pgfqpoint{2.135536in}{2.341249in}}%
\pgfpathlineto{\pgfqpoint{2.151345in}{2.351128in}}%
\pgfpathlineto{\pgfqpoint{2.167153in}{2.346912in}}%
\pgfpathlineto{\pgfqpoint{2.182961in}{2.327028in}}%
\pgfpathlineto{\pgfqpoint{2.230386in}{2.232671in}}%
\pgfpathlineto{\pgfqpoint{2.246195in}{2.220357in}}%
\pgfpathlineto{\pgfqpoint{2.262003in}{2.218878in}}%
\pgfpathlineto{\pgfqpoint{2.277812in}{2.219161in}}%
\pgfpathlineto{\pgfqpoint{2.293620in}{2.203330in}}%
\pgfpathlineto{\pgfqpoint{2.309428in}{2.190182in}}%
\pgfpathlineto{\pgfqpoint{2.325237in}{2.170736in}}%
\pgfpathlineto{\pgfqpoint{2.341045in}{2.127822in}}%
\pgfpathlineto{\pgfqpoint{2.356853in}{2.076094in}}%
\pgfpathlineto{\pgfqpoint{2.376266in}{2.027733in}}%
\pgfpathlineto{\pgfqpoint{2.384195in}{1.984110in}}%
\pgfpathlineto{\pgfqpoint{2.384195in}{1.916372in}}%
\pgfpathlineto{\pgfqpoint{2.389315in}{1.888289in}}%
\pgfpathlineto{\pgfqpoint{2.409792in}{1.823459in}}%
\pgfpathlineto{\pgfqpoint{2.427437in}{1.790012in}}%
\pgfpathlineto{\pgfqpoint{2.445081in}{1.748394in}}%
\pgfpathlineto{\pgfqpoint{2.498015in}{1.586854in}}%
\pgfpathlineto{\pgfqpoint{2.515660in}{1.540299in}}%
\pgfpathlineto{\pgfqpoint{2.532758in}{1.501028in}}%
\pgfpathlineto{\pgfqpoint{2.533532in}{1.400070in}}%
\pgfpathlineto{\pgfqpoint{2.540260in}{1.263835in}}%
\pgfpathlineto{\pgfqpoint{2.543158in}{1.241203in}}%
\pgfpathlineto{\pgfqpoint{2.547673in}{1.220951in}}%
\pgfpathlineto{\pgfqpoint{2.555520in}{1.199672in}}%
\pgfpathlineto{\pgfqpoint{2.569439in}{1.182371in}}%
\pgfpathlineto{\pgfqpoint{2.583357in}{1.175577in}}%
\pgfpathlineto{\pgfqpoint{2.597276in}{1.183888in}}%
\pgfpathlineto{\pgfqpoint{2.611194in}{1.211226in}}%
\pgfpathlineto{\pgfqpoint{2.625113in}{1.257349in}}%
\pgfpathlineto{\pgfqpoint{2.639031in}{1.318996in}}%
\pgfpathlineto{\pgfqpoint{2.666868in}{1.473812in}}%
\pgfpathlineto{\pgfqpoint{2.680787in}{1.560146in}}%
\pgfpathlineto{\pgfqpoint{2.681618in}{1.921608in}}%
\pgfpathlineto{\pgfqpoint{2.682507in}{2.190879in}}%
\pgfpathlineto{\pgfqpoint{2.683907in}{2.273145in}}%
\pgfpathlineto{\pgfqpoint{2.687292in}{2.336416in}}%
\pgfpathlineto{\pgfqpoint{2.690676in}{2.368691in}}%
\pgfpathlineto{\pgfqpoint{2.694061in}{2.359863in}}%
\pgfpathlineto{\pgfqpoint{2.698509in}{2.306299in}}%
\pgfpathlineto{\pgfqpoint{2.708728in}{2.076570in}}%
\pgfpathlineto{\pgfqpoint{2.714499in}{1.921363in}}%
\pgfpathlineto{\pgfqpoint{2.722726in}{1.769305in}}%
\pgfpathlineto{\pgfqpoint{2.733938in}{1.647729in}}%
\pgfpathlineto{\pgfqpoint{2.745150in}{1.573284in}}%
\pgfpathlineto{\pgfqpoint{2.756363in}{1.525758in}}%
\pgfpathlineto{\pgfqpoint{2.767575in}{1.502364in}}%
\pgfpathlineto{\pgfqpoint{2.778788in}{1.505722in}}%
\pgfpathlineto{\pgfqpoint{2.790000in}{1.536637in}}%
\pgfpathlineto{\pgfqpoint{2.801213in}{1.593951in}}%
\pgfpathlineto{\pgfqpoint{2.812425in}{1.675408in}}%
\pgfpathlineto{\pgfqpoint{2.823637in}{1.777271in}}%
\pgfpathlineto{\pgfqpoint{2.829885in}{1.893287in}}%
\pgfpathlineto{\pgfqpoint{2.829885in}{2.123870in}}%
\pgfpathlineto{\pgfqpoint{2.832783in}{2.219594in}}%
\pgfpathlineto{\pgfqpoint{2.838578in}{2.323971in}}%
\pgfpathlineto{\pgfqpoint{2.847054in}{2.549576in}}%
\pgfpathlineto{\pgfqpoint{2.854369in}{2.642304in}}%
\pgfpathlineto{\pgfqpoint{2.861684in}{2.704472in}}%
\pgfpathlineto{\pgfqpoint{2.873089in}{2.726443in}}%
\pgfpathlineto{\pgfqpoint{2.888961in}{2.701891in}}%
\pgfpathlineto{\pgfqpoint{2.904834in}{2.632977in}}%
\pgfpathlineto{\pgfqpoint{2.920706in}{2.529730in}}%
\pgfpathlineto{\pgfqpoint{2.952451in}{2.287525in}}%
\pgfpathlineto{\pgfqpoint{2.968323in}{2.179186in}}%
\pgfpathlineto{\pgfqpoint{2.978448in}{2.087718in}}%
\pgfpathlineto{\pgfqpoint{2.978448in}{1.948942in}}%
\pgfpathlineto{\pgfqpoint{2.983199in}{1.892433in}}%
\pgfpathlineto{\pgfqpoint{2.992700in}{1.847448in}}%
\pgfpathlineto{\pgfqpoint{3.002202in}{1.816841in}}%
\pgfpathlineto{\pgfqpoint{3.018558in}{1.800740in}}%
\pgfpathlineto{\pgfqpoint{3.034914in}{1.797007in}}%
\pgfpathlineto{\pgfqpoint{3.051270in}{1.800914in}}%
\pgfpathlineto{\pgfqpoint{3.067626in}{1.806759in}}%
\pgfpathlineto{\pgfqpoint{3.083982in}{1.810597in}}%
\pgfpathlineto{\pgfqpoint{3.100338in}{1.808528in}}%
\pgfpathlineto{\pgfqpoint{3.116694in}{1.796586in}}%
\pgfpathlineto{\pgfqpoint{3.127011in}{1.775044in}}%
\pgfpathlineto{\pgfqpoint{3.127011in}{1.701826in}}%
\pgfpathlineto{\pgfqpoint{3.133141in}{1.656275in}}%
\pgfpathlineto{\pgfqpoint{3.157663in}{1.573185in}}%
\pgfpathlineto{\pgfqpoint{3.177825in}{1.540031in}}%
\pgfpathlineto{\pgfqpoint{3.197987in}{1.511527in}}%
\pgfpathlineto{\pgfqpoint{3.218149in}{1.486504in}}%
\pgfpathlineto{\pgfqpoint{3.238311in}{1.464712in}}%
\pgfpathlineto{\pgfqpoint{3.258473in}{1.448192in}}%
\pgfpathlineto{\pgfqpoint{3.278636in}{1.437498in}}%
\pgfpathlineto{\pgfqpoint{3.298798in}{1.430487in}}%
\pgfpathlineto{\pgfqpoint{3.318960in}{1.419339in}}%
\pgfpathlineto{\pgfqpoint{3.339122in}{1.399609in}}%
\pgfpathlineto{\pgfqpoint{3.359284in}{1.371682in}}%
\pgfpathlineto{\pgfqpoint{3.384093in}{1.340472in}}%
\pgfpathlineto{\pgfqpoint{3.408901in}{1.312139in}}%
\pgfpathlineto{\pgfqpoint{3.433710in}{1.290553in}}%
\pgfpathlineto{\pgfqpoint{3.458518in}{1.276226in}}%
\pgfpathlineto{\pgfqpoint{3.483327in}{1.266038in}}%
\pgfpathlineto{\pgfqpoint{3.508136in}{1.251734in}}%
\pgfpathlineto{\pgfqpoint{3.532944in}{1.233539in}}%
\pgfpathlineto{\pgfqpoint{3.562994in}{1.223259in}}%
\pgfpathlineto{\pgfqpoint{3.623092in}{1.210082in}}%
\pgfpathlineto{\pgfqpoint{3.653142in}{1.197328in}}%
\pgfpathlineto{\pgfqpoint{3.683191in}{1.173836in}}%
\pgfpathlineto{\pgfqpoint{3.713241in}{1.125436in}}%
\pgfpathlineto{\pgfqpoint{3.721264in}{1.074461in}}%
\pgfpathlineto{\pgfqpoint{3.721264in}{0.994025in}}%
\pgfpathlineto{\pgfqpoint{3.734411in}{0.967759in}}%
\pgfpathlineto{\pgfqpoint{3.760705in}{0.943931in}}%
\pgfpathlineto{\pgfqpoint{3.787000in}{0.914448in}}%
\pgfpathlineto{\pgfqpoint{3.828820in}{0.887283in}}%
\pgfpathlineto{\pgfqpoint{3.869827in}{0.865909in}}%
\pgfpathlineto{\pgfqpoint{3.869827in}{0.843058in}}%
\pgfpathlineto{\pgfqpoint{3.885349in}{0.778679in}}%
\pgfpathlineto{\pgfqpoint{3.890659in}{0.767186in}}%
\pgfpathlineto{\pgfqpoint{3.898287in}{0.756765in}}%
\pgfpathlineto{\pgfqpoint{3.909283in}{0.752582in}}%
\pgfpathlineto{\pgfqpoint{3.925341in}{0.754340in}}%
\pgfpathlineto{\pgfqpoint{3.941398in}{0.761606in}}%
\pgfpathlineto{\pgfqpoint{3.960986in}{0.778365in}}%
\pgfpathlineto{\pgfqpoint{3.980574in}{0.806673in}}%
\pgfpathlineto{\pgfqpoint{4.000162in}{0.844930in}}%
\pgfpathlineto{\pgfqpoint{4.018390in}{0.891613in}}%
\pgfpathlineto{\pgfqpoint{4.019519in}{1.165823in}}%
\pgfpathlineto{\pgfqpoint{4.021939in}{1.276315in}}%
\pgfpathlineto{\pgfqpoint{4.025625in}{1.332046in}}%
\pgfpathlineto{\pgfqpoint{4.033355in}{1.388087in}}%
\pgfpathlineto{\pgfqpoint{4.059199in}{1.490608in}}%
\pgfpathlineto{\pgfqpoint{4.072121in}{1.528850in}}%
\pgfpathlineto{\pgfqpoint{4.085043in}{1.553366in}}%
\pgfpathlineto{\pgfqpoint{4.123809in}{1.593214in}}%
\pgfpathlineto{\pgfqpoint{4.136731in}{1.613944in}}%
\pgfpathlineto{\pgfqpoint{4.149653in}{1.642864in}}%
\pgfpathlineto{\pgfqpoint{4.162575in}{1.679387in}}%
\pgfpathlineto{\pgfqpoint{4.188419in}{1.767795in}}%
\pgfpathlineto{\pgfqpoint{4.214263in}{1.867233in}}%
\pgfpathlineto{\pgfqpoint{4.265951in}{2.085386in}}%
\pgfpathlineto{\pgfqpoint{4.291795in}{2.219607in}}%
\pgfpathlineto{\pgfqpoint{4.317638in}{2.359615in}}%
\pgfpathlineto{\pgfqpoint{4.330560in}{2.415586in}}%
\pgfpathlineto{\pgfqpoint{4.343482in}{2.454076in}}%
\pgfpathlineto{\pgfqpoint{4.356404in}{2.467489in}}%
\pgfpathlineto{\pgfqpoint{4.369326in}{2.460744in}}%
\pgfpathlineto{\pgfqpoint{4.395170in}{2.422821in}}%
\pgfpathlineto{\pgfqpoint{4.408092in}{2.411078in}}%
\pgfpathlineto{\pgfqpoint{4.421014in}{2.412920in}}%
\pgfpathlineto{\pgfqpoint{4.433936in}{2.421409in}}%
\pgfpathlineto{\pgfqpoint{4.459780in}{2.443677in}}%
\pgfpathlineto{\pgfqpoint{4.472702in}{2.450370in}}%
\pgfpathlineto{\pgfqpoint{4.498546in}{2.454801in}}%
\pgfpathlineto{\pgfqpoint{4.511468in}{2.448234in}}%
\pgfpathlineto{\pgfqpoint{4.524390in}{2.432420in}}%
\pgfpathlineto{\pgfqpoint{4.563156in}{2.363718in}}%
\pgfpathlineto{\pgfqpoint{4.576078in}{2.357766in}}%
\pgfpathlineto{\pgfqpoint{4.589000in}{2.363806in}}%
\pgfpathlineto{\pgfqpoint{4.614844in}{2.385710in}}%
\pgfpathlineto{\pgfqpoint{4.627765in}{2.393424in}}%
\pgfpathlineto{\pgfqpoint{4.653609in}{2.397621in}}%
\pgfpathlineto{\pgfqpoint{4.666531in}{2.394037in}}%
\pgfpathlineto{\pgfqpoint{4.679453in}{2.381208in}}%
\pgfpathlineto{\pgfqpoint{4.692375in}{2.361029in}}%
\pgfpathlineto{\pgfqpoint{4.718219in}{2.315823in}}%
\pgfpathlineto{\pgfqpoint{4.733740in}{2.307009in}}%
\pgfpathlineto{\pgfqpoint{4.749261in}{2.313016in}}%
\pgfpathlineto{\pgfqpoint{4.764782in}{2.325864in}}%
\pgfpathlineto{\pgfqpoint{4.795824in}{2.356791in}}%
\pgfpathlineto{\pgfqpoint{4.826865in}{2.375167in}}%
\pgfpathlineto{\pgfqpoint{4.842386in}{2.381077in}}%
\pgfpathlineto{\pgfqpoint{4.857907in}{2.374898in}}%
\pgfpathlineto{\pgfqpoint{4.888949in}{2.347598in}}%
\pgfpathlineto{\pgfqpoint{4.904470in}{2.343605in}}%
\pgfpathlineto{\pgfqpoint{4.919991in}{2.355274in}}%
\pgfpathlineto{\pgfqpoint{4.935511in}{2.378472in}}%
\pgfpathlineto{\pgfqpoint{4.982074in}{2.467993in}}%
\pgfpathlineto{\pgfqpoint{4.997595in}{2.484156in}}%
\pgfpathlineto{\pgfqpoint{5.013116in}{2.482435in}}%
\pgfpathlineto{\pgfqpoint{5.028637in}{2.464885in}}%
\pgfpathlineto{\pgfqpoint{5.059678in}{2.418192in}}%
\pgfpathlineto{\pgfqpoint{5.075199in}{2.408188in}}%
\pgfpathlineto{\pgfqpoint{5.090720in}{2.415172in}}%
\pgfpathlineto{\pgfqpoint{5.121762in}{2.433118in}}%
\pgfpathlineto{\pgfqpoint{5.152804in}{2.451325in}}%
\pgfpathlineto{\pgfqpoint{5.168325in}{2.446619in}}%
\pgfpathlineto{\pgfqpoint{5.183845in}{2.427643in}}%
\pgfpathlineto{\pgfqpoint{5.199366in}{2.411326in}}%
\pgfpathlineto{\pgfqpoint{5.206896in}{2.395773in}}%
\pgfpathlineto{\pgfqpoint{5.206896in}{2.395773in}}%
\pgfusepath{stroke}%
\end{pgfscope}%
\begin{pgfscope}%
\pgfpathrectangle{\pgfqpoint{0.750000in}{0.500000in}}{\pgfqpoint{4.650000in}{3.020000in}}%
\pgfusepath{clip}%
\pgfsetbuttcap%
\pgfsetroundjoin%
\pgfsetlinewidth{1.505625pt}%
\definecolor{currentstroke}{rgb}{0.000000,0.500000,0.000000}%
\pgfsetstrokecolor{currentstroke}%
\pgfsetdash{{5.550000pt}{2.400000pt}}{0.000000pt}%
\pgfpathmoveto{\pgfqpoint{1.344812in}{0.637273in}}%
\pgfpathlineto{\pgfqpoint{1.351299in}{0.638757in}}%
\pgfpathlineto{\pgfqpoint{1.357435in}{0.644343in}}%
\pgfpathlineto{\pgfqpoint{1.365617in}{0.658758in}}%
\pgfpathlineto{\pgfqpoint{1.382890in}{0.692556in}}%
\pgfpathlineto{\pgfqpoint{1.395057in}{0.717839in}}%
\pgfpathlineto{\pgfqpoint{1.420549in}{0.789913in}}%
\pgfpathlineto{\pgfqpoint{1.437812in}{0.853390in}}%
\pgfpathlineto{\pgfqpoint{1.463705in}{0.954765in}}%
\pgfpathlineto{\pgfqpoint{1.472337in}{0.979521in}}%
\pgfpathlineto{\pgfqpoint{1.482936in}{0.996802in}}%
\pgfpathlineto{\pgfqpoint{1.492816in}{1.005695in}}%
\pgfpathlineto{\pgfqpoint{1.492816in}{1.005618in}}%
\pgfpathlineto{\pgfqpoint{1.492816in}{0.996874in}}%
\pgfpathlineto{\pgfqpoint{1.510739in}{0.833225in}}%
\pgfpathlineto{\pgfqpoint{1.517027in}{0.785456in}}%
\pgfpathlineto{\pgfqpoint{1.525486in}{0.747003in}}%
\pgfpathlineto{\pgfqpoint{1.536357in}{0.730703in}}%
\pgfpathlineto{\pgfqpoint{1.558099in}{0.728313in}}%
\pgfpathlineto{\pgfqpoint{1.568970in}{0.736990in}}%
\pgfpathlineto{\pgfqpoint{1.579841in}{0.776446in}}%
\pgfpathlineto{\pgfqpoint{1.590712in}{0.844516in}}%
\pgfpathlineto{\pgfqpoint{1.623325in}{1.097694in}}%
\pgfpathlineto{\pgfqpoint{1.636719in}{1.169565in}}%
\pgfpathlineto{\pgfqpoint{1.650113in}{1.225548in}}%
\pgfpathlineto{\pgfqpoint{1.663507in}{1.262737in}}%
\pgfpathlineto{\pgfqpoint{1.676902in}{1.282827in}}%
\pgfpathlineto{\pgfqpoint{1.690296in}{1.293553in}}%
\pgfpathlineto{\pgfqpoint{1.703690in}{1.292732in}}%
\pgfpathlineto{\pgfqpoint{1.717084in}{1.272979in}}%
\pgfpathlineto{\pgfqpoint{1.730479in}{1.231955in}}%
\pgfpathlineto{\pgfqpoint{1.743873in}{1.169215in}}%
\pgfpathlineto{\pgfqpoint{1.770661in}{1.006179in}}%
\pgfpathlineto{\pgfqpoint{1.784056in}{0.924197in}}%
\pgfpathlineto{\pgfqpoint{1.789942in}{0.848671in}}%
\pgfpathlineto{\pgfqpoint{1.789942in}{0.749517in}}%
\pgfpathlineto{\pgfqpoint{1.793487in}{0.760737in}}%
\pgfpathlineto{\pgfqpoint{1.807667in}{0.816160in}}%
\pgfpathlineto{\pgfqpoint{1.818289in}{0.830092in}}%
\pgfpathlineto{\pgfqpoint{1.831044in}{0.834286in}}%
\pgfpathlineto{\pgfqpoint{1.843798in}{0.831955in}}%
\pgfpathlineto{\pgfqpoint{1.856553in}{0.825336in}}%
\pgfpathlineto{\pgfqpoint{1.882062in}{0.804461in}}%
\pgfpathlineto{\pgfqpoint{1.894817in}{0.814994in}}%
\pgfpathlineto{\pgfqpoint{1.920326in}{0.867924in}}%
\pgfpathlineto{\pgfqpoint{1.933081in}{0.882828in}}%
\pgfpathlineto{\pgfqpoint{1.945836in}{0.881401in}}%
\pgfpathlineto{\pgfqpoint{1.961644in}{0.865907in}}%
\pgfpathlineto{\pgfqpoint{1.977452in}{0.839275in}}%
\pgfpathlineto{\pgfqpoint{1.993261in}{0.805287in}}%
\pgfpathlineto{\pgfqpoint{2.024878in}{0.730306in}}%
\pgfpathlineto{\pgfqpoint{2.040686in}{0.706685in}}%
\pgfpathlineto{\pgfqpoint{2.056494in}{0.708035in}}%
\pgfpathlineto{\pgfqpoint{2.103919in}{0.765405in}}%
\pgfpathlineto{\pgfqpoint{2.119728in}{0.777860in}}%
\pgfpathlineto{\pgfqpoint{2.135536in}{0.783769in}}%
\pgfpathlineto{\pgfqpoint{2.151345in}{0.781848in}}%
\pgfpathlineto{\pgfqpoint{2.182961in}{0.749426in}}%
\pgfpathlineto{\pgfqpoint{2.198770in}{0.744260in}}%
\pgfpathlineto{\pgfqpoint{2.214578in}{0.744503in}}%
\pgfpathlineto{\pgfqpoint{2.262003in}{0.737730in}}%
\pgfpathlineto{\pgfqpoint{2.277812in}{0.737946in}}%
\pgfpathlineto{\pgfqpoint{2.309428in}{0.703851in}}%
\pgfpathlineto{\pgfqpoint{2.325237in}{0.711322in}}%
\pgfpathlineto{\pgfqpoint{2.341045in}{0.730394in}}%
\pgfpathlineto{\pgfqpoint{2.356853in}{0.758200in}}%
\pgfpathlineto{\pgfqpoint{2.384195in}{0.792356in}}%
\pgfpathlineto{\pgfqpoint{2.384195in}{0.775253in}}%
\pgfpathlineto{\pgfqpoint{2.389315in}{0.750536in}}%
\pgfpathlineto{\pgfqpoint{2.399554in}{0.716773in}}%
\pgfpathlineto{\pgfqpoint{2.409792in}{0.702542in}}%
\pgfpathlineto{\pgfqpoint{2.427437in}{0.720566in}}%
\pgfpathlineto{\pgfqpoint{2.445081in}{0.752209in}}%
\pgfpathlineto{\pgfqpoint{2.498015in}{0.821222in}}%
\pgfpathlineto{\pgfqpoint{2.515660in}{0.829069in}}%
\pgfpathlineto{\pgfqpoint{2.532758in}{0.822799in}}%
\pgfpathlineto{\pgfqpoint{2.533532in}{0.753098in}}%
\pgfpathlineto{\pgfqpoint{2.536625in}{0.722088in}}%
\pgfpathlineto{\pgfqpoint{2.538172in}{0.717301in}}%
\pgfpathlineto{\pgfqpoint{2.540260in}{0.716063in}}%
\pgfpathlineto{\pgfqpoint{2.555520in}{0.725930in}}%
\pgfpathlineto{\pgfqpoint{2.569439in}{0.718350in}}%
\pgfpathlineto{\pgfqpoint{2.583357in}{0.705945in}}%
\pgfpathlineto{\pgfqpoint{2.597276in}{0.729479in}}%
\pgfpathlineto{\pgfqpoint{2.611194in}{0.796484in}}%
\pgfpathlineto{\pgfqpoint{2.652950in}{1.047440in}}%
\pgfpathlineto{\pgfqpoint{2.666868in}{1.116726in}}%
\pgfpathlineto{\pgfqpoint{2.680787in}{1.171846in}}%
\pgfpathlineto{\pgfqpoint{2.681322in}{1.242305in}}%
\pgfpathlineto{\pgfqpoint{2.681914in}{1.192654in}}%
\pgfpathlineto{\pgfqpoint{2.683907in}{1.055309in}}%
\pgfpathlineto{\pgfqpoint{2.690676in}{0.867865in}}%
\pgfpathlineto{\pgfqpoint{2.694061in}{0.824977in}}%
\pgfpathlineto{\pgfqpoint{2.698509in}{0.872080in}}%
\pgfpathlineto{\pgfqpoint{2.708728in}{1.068753in}}%
\pgfpathlineto{\pgfqpoint{2.714499in}{1.148962in}}%
\pgfpathlineto{\pgfqpoint{2.722726in}{1.185732in}}%
\pgfpathlineto{\pgfqpoint{2.733938in}{1.169436in}}%
\pgfpathlineto{\pgfqpoint{2.745150in}{1.116927in}}%
\pgfpathlineto{\pgfqpoint{2.756363in}{1.055462in}}%
\pgfpathlineto{\pgfqpoint{2.767575in}{1.012787in}}%
\pgfpathlineto{\pgfqpoint{2.778788in}{1.019923in}}%
\pgfpathlineto{\pgfqpoint{2.790000in}{1.080080in}}%
\pgfpathlineto{\pgfqpoint{2.812425in}{1.250174in}}%
\pgfpathlineto{\pgfqpoint{2.829885in}{1.352259in}}%
\pgfpathlineto{\pgfqpoint{2.829885in}{1.374329in}}%
\pgfpathlineto{\pgfqpoint{2.832783in}{1.382323in}}%
\pgfpathlineto{\pgfqpoint{2.838578in}{1.366024in}}%
\pgfpathlineto{\pgfqpoint{2.842816in}{1.311396in}}%
\pgfpathlineto{\pgfqpoint{2.847054in}{1.219504in}}%
\pgfpathlineto{\pgfqpoint{2.861684in}{0.987747in}}%
\pgfpathlineto{\pgfqpoint{2.873089in}{0.888985in}}%
\pgfpathlineto{\pgfqpoint{2.888961in}{0.864042in}}%
\pgfpathlineto{\pgfqpoint{2.904834in}{0.924861in}}%
\pgfpathlineto{\pgfqpoint{2.920706in}{1.008834in}}%
\pgfpathlineto{\pgfqpoint{2.936578in}{1.072732in}}%
\pgfpathlineto{\pgfqpoint{2.952451in}{1.101331in}}%
\pgfpathlineto{\pgfqpoint{2.968323in}{1.095374in}}%
\pgfpathlineto{\pgfqpoint{2.978448in}{1.061089in}}%
\pgfpathlineto{\pgfqpoint{2.978448in}{0.936774in}}%
\pgfpathlineto{\pgfqpoint{2.983199in}{0.859125in}}%
\pgfpathlineto{\pgfqpoint{3.002202in}{0.715230in}}%
\pgfpathlineto{\pgfqpoint{3.018558in}{0.674242in}}%
\pgfpathlineto{\pgfqpoint{3.034914in}{0.667671in}}%
\pgfpathlineto{\pgfqpoint{3.051270in}{0.668150in}}%
\pgfpathlineto{\pgfqpoint{3.083982in}{0.663997in}}%
\pgfpathlineto{\pgfqpoint{3.100338in}{0.663974in}}%
\pgfpathlineto{\pgfqpoint{3.116694in}{0.674352in}}%
\pgfpathlineto{\pgfqpoint{3.127011in}{0.702361in}}%
\pgfpathlineto{\pgfqpoint{3.127011in}{0.769238in}}%
\pgfpathlineto{\pgfqpoint{3.133141in}{0.790006in}}%
\pgfpathlineto{\pgfqpoint{3.145402in}{0.800097in}}%
\pgfpathlineto{\pgfqpoint{3.157663in}{0.801489in}}%
\pgfpathlineto{\pgfqpoint{3.177825in}{0.794915in}}%
\pgfpathlineto{\pgfqpoint{3.197987in}{0.780096in}}%
\pgfpathlineto{\pgfqpoint{3.218149in}{0.755708in}}%
\pgfpathlineto{\pgfqpoint{3.258473in}{0.693164in}}%
\pgfpathlineto{\pgfqpoint{3.278636in}{0.667976in}}%
\pgfpathlineto{\pgfqpoint{3.298798in}{0.654808in}}%
\pgfpathlineto{\pgfqpoint{3.318960in}{0.664454in}}%
\pgfpathlineto{\pgfqpoint{3.359284in}{0.713260in}}%
\pgfpathlineto{\pgfqpoint{3.384093in}{0.729052in}}%
\pgfpathlineto{\pgfqpoint{3.408901in}{0.735049in}}%
\pgfpathlineto{\pgfqpoint{3.433710in}{0.734191in}}%
\pgfpathlineto{\pgfqpoint{3.458518in}{0.728354in}}%
\pgfpathlineto{\pgfqpoint{3.483327in}{0.720622in}}%
\pgfpathlineto{\pgfqpoint{3.508136in}{0.701617in}}%
\pgfpathlineto{\pgfqpoint{3.532944in}{0.671238in}}%
\pgfpathlineto{\pgfqpoint{3.562994in}{0.661375in}}%
\pgfpathlineto{\pgfqpoint{3.623092in}{0.661715in}}%
\pgfpathlineto{\pgfqpoint{3.653142in}{0.671324in}}%
\pgfpathlineto{\pgfqpoint{3.683191in}{0.683343in}}%
\pgfpathlineto{\pgfqpoint{3.713241in}{0.680065in}}%
\pgfpathlineto{\pgfqpoint{3.721264in}{0.672484in}}%
\pgfpathlineto{\pgfqpoint{3.721264in}{0.690655in}}%
\pgfpathlineto{\pgfqpoint{3.734411in}{0.701418in}}%
\pgfpathlineto{\pgfqpoint{3.760705in}{0.705553in}}%
\pgfpathlineto{\pgfqpoint{3.787000in}{0.695231in}}%
\pgfpathlineto{\pgfqpoint{3.828820in}{0.680474in}}%
\pgfpathlineto{\pgfqpoint{3.869827in}{0.664210in}}%
\pgfpathlineto{\pgfqpoint{3.869827in}{0.652741in}}%
\pgfpathlineto{\pgfqpoint{3.880966in}{0.660855in}}%
\pgfpathlineto{\pgfqpoint{3.898287in}{0.651955in}}%
\pgfpathlineto{\pgfqpoint{3.909283in}{0.650502in}}%
\pgfpathlineto{\pgfqpoint{3.925341in}{0.654153in}}%
\pgfpathlineto{\pgfqpoint{3.941398in}{0.674316in}}%
\pgfpathlineto{\pgfqpoint{3.960986in}{0.714125in}}%
\pgfpathlineto{\pgfqpoint{3.980574in}{0.767879in}}%
\pgfpathlineto{\pgfqpoint{4.018390in}{0.883720in}}%
\pgfpathlineto{\pgfqpoint{4.019519in}{1.018090in}}%
\pgfpathlineto{\pgfqpoint{4.020325in}{1.017939in}}%
\pgfpathlineto{\pgfqpoint{4.025625in}{0.991815in}}%
\pgfpathlineto{\pgfqpoint{4.033355in}{0.962166in}}%
\pgfpathlineto{\pgfqpoint{4.046277in}{0.919519in}}%
\pgfpathlineto{\pgfqpoint{4.059199in}{0.866105in}}%
\pgfpathlineto{\pgfqpoint{4.085043in}{0.745130in}}%
\pgfpathlineto{\pgfqpoint{4.097965in}{0.700770in}}%
\pgfpathlineto{\pgfqpoint{4.110887in}{0.677106in}}%
\pgfpathlineto{\pgfqpoint{4.123809in}{0.689144in}}%
\pgfpathlineto{\pgfqpoint{4.136731in}{0.730896in}}%
\pgfpathlineto{\pgfqpoint{4.175497in}{0.889610in}}%
\pgfpathlineto{\pgfqpoint{4.188419in}{0.931211in}}%
\pgfpathlineto{\pgfqpoint{4.201341in}{0.962480in}}%
\pgfpathlineto{\pgfqpoint{4.214263in}{0.982745in}}%
\pgfpathlineto{\pgfqpoint{4.227185in}{0.992571in}}%
\pgfpathlineto{\pgfqpoint{4.253029in}{0.998513in}}%
\pgfpathlineto{\pgfqpoint{4.265951in}{0.998982in}}%
\pgfpathlineto{\pgfqpoint{4.278873in}{0.992459in}}%
\pgfpathlineto{\pgfqpoint{4.291795in}{0.974108in}}%
\pgfpathlineto{\pgfqpoint{4.304717in}{0.940782in}}%
\pgfpathlineto{\pgfqpoint{4.343482in}{0.804756in}}%
\pgfpathlineto{\pgfqpoint{4.356404in}{0.766775in}}%
\pgfpathlineto{\pgfqpoint{4.369326in}{0.739219in}}%
\pgfpathlineto{\pgfqpoint{4.382248in}{0.722118in}}%
\pgfpathlineto{\pgfqpoint{4.395170in}{0.710380in}}%
\pgfpathlineto{\pgfqpoint{4.408092in}{0.701543in}}%
\pgfpathlineto{\pgfqpoint{4.421014in}{0.700087in}}%
\pgfpathlineto{\pgfqpoint{4.433936in}{0.704669in}}%
\pgfpathlineto{\pgfqpoint{4.446858in}{0.715153in}}%
\pgfpathlineto{\pgfqpoint{4.459780in}{0.727728in}}%
\pgfpathlineto{\pgfqpoint{4.472702in}{0.736432in}}%
\pgfpathlineto{\pgfqpoint{4.498546in}{0.739678in}}%
\pgfpathlineto{\pgfqpoint{4.511468in}{0.733249in}}%
\pgfpathlineto{\pgfqpoint{4.524390in}{0.723500in}}%
\pgfpathlineto{\pgfqpoint{4.537312in}{0.716229in}}%
\pgfpathlineto{\pgfqpoint{4.550234in}{0.711596in}}%
\pgfpathlineto{\pgfqpoint{4.563156in}{0.703359in}}%
\pgfpathlineto{\pgfqpoint{4.576078in}{0.699830in}}%
\pgfpathlineto{\pgfqpoint{4.589000in}{0.703158in}}%
\pgfpathlineto{\pgfqpoint{4.601922in}{0.711889in}}%
\pgfpathlineto{\pgfqpoint{4.614844in}{0.723948in}}%
\pgfpathlineto{\pgfqpoint{4.627765in}{0.733735in}}%
\pgfpathlineto{\pgfqpoint{4.653609in}{0.737508in}}%
\pgfpathlineto{\pgfqpoint{4.666531in}{0.732829in}}%
\pgfpathlineto{\pgfqpoint{4.692375in}{0.714053in}}%
\pgfpathlineto{\pgfqpoint{4.733740in}{0.694751in}}%
\pgfpathlineto{\pgfqpoint{4.749261in}{0.698807in}}%
\pgfpathlineto{\pgfqpoint{4.764782in}{0.711721in}}%
\pgfpathlineto{\pgfqpoint{4.795824in}{0.751784in}}%
\pgfpathlineto{\pgfqpoint{4.811344in}{0.764813in}}%
\pgfpathlineto{\pgfqpoint{4.826865in}{0.771669in}}%
\pgfpathlineto{\pgfqpoint{4.842386in}{0.770927in}}%
\pgfpathlineto{\pgfqpoint{4.857907in}{0.759129in}}%
\pgfpathlineto{\pgfqpoint{4.873428in}{0.739908in}}%
\pgfpathlineto{\pgfqpoint{4.888949in}{0.716247in}}%
\pgfpathlineto{\pgfqpoint{4.904470in}{0.700708in}}%
\pgfpathlineto{\pgfqpoint{4.919991in}{0.712586in}}%
\pgfpathlineto{\pgfqpoint{4.935511in}{0.734979in}}%
\pgfpathlineto{\pgfqpoint{4.951032in}{0.760230in}}%
\pgfpathlineto{\pgfqpoint{4.966553in}{0.780120in}}%
\pgfpathlineto{\pgfqpoint{4.982074in}{0.793374in}}%
\pgfpathlineto{\pgfqpoint{4.997595in}{0.798617in}}%
\pgfpathlineto{\pgfqpoint{5.013116in}{0.794838in}}%
\pgfpathlineto{\pgfqpoint{5.028637in}{0.780524in}}%
\pgfpathlineto{\pgfqpoint{5.044158in}{0.759214in}}%
\pgfpathlineto{\pgfqpoint{5.059678in}{0.730110in}}%
\pgfpathlineto{\pgfqpoint{5.075199in}{0.708186in}}%
\pgfpathlineto{\pgfqpoint{5.090720in}{0.717439in}}%
\pgfpathlineto{\pgfqpoint{5.106241in}{0.720447in}}%
\pgfpathlineto{\pgfqpoint{5.137283in}{0.709738in}}%
\pgfpathlineto{\pgfqpoint{5.152804in}{0.713354in}}%
\pgfpathlineto{\pgfqpoint{5.183845in}{0.703933in}}%
\pgfpathlineto{\pgfqpoint{5.199366in}{0.717003in}}%
\pgfpathlineto{\pgfqpoint{5.206896in}{0.726883in}}%
\pgfpathlineto{\pgfqpoint{5.206896in}{0.726883in}}%
\pgfusepath{stroke}%
\end{pgfscope}%
\begin{pgfscope}%
\pgfpathrectangle{\pgfqpoint{0.750000in}{0.500000in}}{\pgfqpoint{4.650000in}{3.020000in}}%
\pgfusepath{clip}%
\pgfsetrectcap%
\pgfsetroundjoin%
\pgfsetlinewidth{1.505625pt}%
\definecolor{currentstroke}{rgb}{1.000000,0.000000,0.000000}%
\pgfsetstrokecolor{currentstroke}%
\pgfsetdash{}{0pt}%
\pgfpathmoveto{\pgfqpoint{1.355390in}{0.500000in}}%
\pgfpathlineto{\pgfqpoint{1.355390in}{3.520000in}}%
\pgfusepath{stroke}%
\end{pgfscope}%
\begin{pgfscope}%
\pgfpathrectangle{\pgfqpoint{0.750000in}{0.500000in}}{\pgfqpoint{4.650000in}{3.020000in}}%
\pgfusepath{clip}%
\pgfsetrectcap%
\pgfsetroundjoin%
\pgfsetlinewidth{1.505625pt}%
\definecolor{currentstroke}{rgb}{1.000000,0.000000,0.000000}%
\pgfsetstrokecolor{currentstroke}%
\pgfsetdash{}{0pt}%
\pgfpathmoveto{\pgfqpoint{1.856553in}{0.500000in}}%
\pgfpathlineto{\pgfqpoint{1.856553in}{3.520000in}}%
\pgfusepath{stroke}%
\end{pgfscope}%
\begin{pgfscope}%
\pgfpathrectangle{\pgfqpoint{0.750000in}{0.500000in}}{\pgfqpoint{4.650000in}{3.020000in}}%
\pgfusepath{clip}%
\pgfsetrectcap%
\pgfsetroundjoin%
\pgfsetlinewidth{1.505625pt}%
\definecolor{currentstroke}{rgb}{1.000000,0.000000,0.000000}%
\pgfsetstrokecolor{currentstroke}%
\pgfsetdash{}{0pt}%
\pgfpathmoveto{\pgfqpoint{2.597276in}{0.500000in}}%
\pgfpathlineto{\pgfqpoint{2.597276in}{3.520000in}}%
\pgfusepath{stroke}%
\end{pgfscope}%
\begin{pgfscope}%
\pgfpathrectangle{\pgfqpoint{0.750000in}{0.500000in}}{\pgfqpoint{4.650000in}{3.020000in}}%
\pgfusepath{clip}%
\pgfsetrectcap%
\pgfsetroundjoin%
\pgfsetlinewidth{1.505625pt}%
\definecolor{currentstroke}{rgb}{1.000000,0.000000,0.000000}%
\pgfsetstrokecolor{currentstroke}%
\pgfsetdash{}{0pt}%
\pgfpathmoveto{\pgfqpoint{3.278636in}{0.500000in}}%
\pgfpathlineto{\pgfqpoint{3.278636in}{3.520000in}}%
\pgfusepath{stroke}%
\end{pgfscope}%
\begin{pgfscope}%
\pgfpathrectangle{\pgfqpoint{0.750000in}{0.500000in}}{\pgfqpoint{4.650000in}{3.020000in}}%
\pgfusepath{clip}%
\pgfsetrectcap%
\pgfsetroundjoin%
\pgfsetlinewidth{1.505625pt}%
\definecolor{currentstroke}{rgb}{1.000000,0.000000,0.000000}%
\pgfsetstrokecolor{currentstroke}%
\pgfsetdash{}{0pt}%
\pgfpathmoveto{\pgfqpoint{4.162575in}{0.500000in}}%
\pgfpathlineto{\pgfqpoint{4.162575in}{3.520000in}}%
\pgfusepath{stroke}%
\end{pgfscope}%
\begin{pgfscope}%
\pgfsetrectcap%
\pgfsetmiterjoin%
\pgfsetlinewidth{0.803000pt}%
\definecolor{currentstroke}{rgb}{0.000000,0.000000,0.000000}%
\pgfsetstrokecolor{currentstroke}%
\pgfsetdash{}{0pt}%
\pgfpathmoveto{\pgfqpoint{0.750000in}{0.500000in}}%
\pgfpathlineto{\pgfqpoint{0.750000in}{3.520000in}}%
\pgfusepath{stroke}%
\end{pgfscope}%
\begin{pgfscope}%
\pgfsetrectcap%
\pgfsetmiterjoin%
\pgfsetlinewidth{0.803000pt}%
\definecolor{currentstroke}{rgb}{0.000000,0.000000,0.000000}%
\pgfsetstrokecolor{currentstroke}%
\pgfsetdash{}{0pt}%
\pgfpathmoveto{\pgfqpoint{5.400000in}{0.500000in}}%
\pgfpathlineto{\pgfqpoint{5.400000in}{3.520000in}}%
\pgfusepath{stroke}%
\end{pgfscope}%
\begin{pgfscope}%
\pgfsetrectcap%
\pgfsetmiterjoin%
\pgfsetlinewidth{0.803000pt}%
\definecolor{currentstroke}{rgb}{0.000000,0.000000,0.000000}%
\pgfsetstrokecolor{currentstroke}%
\pgfsetdash{}{0pt}%
\pgfpathmoveto{\pgfqpoint{0.750000in}{0.500000in}}%
\pgfpathlineto{\pgfqpoint{5.400000in}{0.500000in}}%
\pgfusepath{stroke}%
\end{pgfscope}%
\begin{pgfscope}%
\pgfsetrectcap%
\pgfsetmiterjoin%
\pgfsetlinewidth{0.803000pt}%
\definecolor{currentstroke}{rgb}{0.000000,0.000000,0.000000}%
\pgfsetstrokecolor{currentstroke}%
\pgfsetdash{}{0pt}%
\pgfpathmoveto{\pgfqpoint{0.750000in}{3.520000in}}%
\pgfpathlineto{\pgfqpoint{5.400000in}{3.520000in}}%
\pgfusepath{stroke}%
\end{pgfscope}%
\begin{pgfscope}%
\pgfsetbuttcap%
\pgfsetmiterjoin%
\definecolor{currentfill}{rgb}{1.000000,1.000000,1.000000}%
\pgfsetfillcolor{currentfill}%
\pgfsetfillopacity{0.800000}%
\pgfsetlinewidth{1.003750pt}%
\definecolor{currentstroke}{rgb}{0.800000,0.800000,0.800000}%
\pgfsetstrokecolor{currentstroke}%
\pgfsetstrokeopacity{0.800000}%
\pgfsetdash{}{0pt}%
\pgfpathmoveto{\pgfqpoint{4.302777in}{1.601821in}}%
\pgfpathlineto{\pgfqpoint{5.302778in}{1.601821in}}%
\pgfpathquadraticcurveto{\pgfqpoint{5.330556in}{1.601821in}}{\pgfqpoint{5.330556in}{1.629599in}}%
\pgfpathlineto{\pgfqpoint{5.330556in}{2.390401in}}%
\pgfpathquadraticcurveto{\pgfqpoint{5.330556in}{2.418179in}}{\pgfqpoint{5.302778in}{2.418179in}}%
\pgfpathlineto{\pgfqpoint{4.302777in}{2.418179in}}%
\pgfpathquadraticcurveto{\pgfqpoint{4.274999in}{2.418179in}}{\pgfqpoint{4.274999in}{2.390401in}}%
\pgfpathlineto{\pgfqpoint{4.274999in}{1.629599in}}%
\pgfpathquadraticcurveto{\pgfqpoint{4.274999in}{1.601821in}}{\pgfqpoint{4.302777in}{1.601821in}}%
\pgfpathlineto{\pgfqpoint{4.302777in}{1.601821in}}%
\pgfpathclose%
\pgfusepath{stroke,fill}%
\end{pgfscope}%
\begin{pgfscope}%
\pgfsetrectcap%
\pgfsetroundjoin%
\pgfsetlinewidth{1.505625pt}%
\definecolor{currentstroke}{rgb}{1.000000,0.000000,0.000000}%
\pgfsetstrokecolor{currentstroke}%
\pgfsetdash{}{0pt}%
\pgfpathmoveto{\pgfqpoint{4.330554in}{2.314012in}}%
\pgfpathlineto{\pgfqpoint{4.469443in}{2.314012in}}%
\pgfpathlineto{\pgfqpoint{4.608332in}{2.314012in}}%
\pgfusepath{stroke}%
\end{pgfscope}%
\begin{pgfscope}%
\definecolor{textcolor}{rgb}{0.000000,0.000000,0.000000}%
\pgfsetstrokecolor{textcolor}%
\pgfsetfillcolor{textcolor}%
\pgftext[x=4.719443in,y=2.265401in,left,base]{\color{textcolor}\rmfamily\fontsize{10.000000}{12.000000}\selectfont detection}%
\end{pgfscope}%
\begin{pgfscope}%
\pgfsetbuttcap%
\pgfsetroundjoin%
\pgfsetlinewidth{1.505625pt}%
\definecolor{currentstroke}{rgb}{0.000000,0.000000,1.000000}%
\pgfsetstrokecolor{currentstroke}%
\pgfsetdash{{5.550000pt}{2.400000pt}}{0.000000pt}%
\pgfpathmoveto{\pgfqpoint{4.330554in}{2.120339in}}%
\pgfpathlineto{\pgfqpoint{4.469443in}{2.120339in}}%
\pgfpathlineto{\pgfqpoint{4.608332in}{2.120339in}}%
\pgfusepath{stroke}%
\end{pgfscope}%
\begin{pgfscope}%
\definecolor{textcolor}{rgb}{0.000000,0.000000,0.000000}%
\pgfsetstrokecolor{textcolor}%
\pgfsetfillcolor{textcolor}%
\pgftext[x=4.719443in,y=2.071728in,left,base]{\color{textcolor}\rmfamily\fontsize{10.000000}{12.000000}\selectfont max}%
\end{pgfscope}%
\begin{pgfscope}%
\pgfsetbuttcap%
\pgfsetroundjoin%
\pgfsetlinewidth{1.505625pt}%
\definecolor{currentstroke}{rgb}{0.750000,0.750000,0.000000}%
\pgfsetstrokecolor{currentstroke}%
\pgfsetdash{{5.550000pt}{2.400000pt}}{0.000000pt}%
\pgfpathmoveto{\pgfqpoint{4.330554in}{1.926667in}}%
\pgfpathlineto{\pgfqpoint{4.469443in}{1.926667in}}%
\pgfpathlineto{\pgfqpoint{4.608332in}{1.926667in}}%
\pgfusepath{stroke}%
\end{pgfscope}%
\begin{pgfscope}%
\definecolor{textcolor}{rgb}{0.000000,0.000000,0.000000}%
\pgfsetstrokecolor{textcolor}%
\pgfsetfillcolor{textcolor}%
\pgftext[x=4.719443in,y=1.878056in,left,base]{\color{textcolor}\rmfamily\fontsize{10.000000}{12.000000}\selectfont \(\displaystyle \mu\)}%
\end{pgfscope}%
\begin{pgfscope}%
\pgfsetbuttcap%
\pgfsetroundjoin%
\pgfsetlinewidth{1.505625pt}%
\definecolor{currentstroke}{rgb}{0.000000,0.500000,0.000000}%
\pgfsetstrokecolor{currentstroke}%
\pgfsetdash{{5.550000pt}{2.400000pt}}{0.000000pt}%
\pgfpathmoveto{\pgfqpoint{4.330554in}{1.732994in}}%
\pgfpathlineto{\pgfqpoint{4.469443in}{1.732994in}}%
\pgfpathlineto{\pgfqpoint{4.608332in}{1.732994in}}%
\pgfusepath{stroke}%
\end{pgfscope}%
\begin{pgfscope}%
\definecolor{textcolor}{rgb}{0.000000,0.000000,0.000000}%
\pgfsetstrokecolor{textcolor}%
\pgfsetfillcolor{textcolor}%
\pgftext[x=4.719443in,y=1.684383in,left,base]{\color{textcolor}\rmfamily\fontsize{10.000000}{12.000000}\selectfont \(\displaystyle \sigma\)}%
\end{pgfscope}%
\end{pgfpicture}%
\makeatother%
\endgroup%

    \caption{Hydraulic Simulation Matrix Profile Values}
    \label{fig:hydraulic_mp_hist_fp}
\end{figure}


\subsection{Power Electronic Converter Dataset}
\label{ref_results_pec_sim}
The Power Electronic Converter (PEC) dataset presented in Section \ref{ref_pec_dataset} is used to perform this simulation. Table \ref{tab:pec_sim_params} specifies the parameters used with the anomaly detector to obtain the results in this section. This dataset is significantly larger than the previous, containing approximately 300,000 time steps (~83 hour of signal data). This dataset outlines the scenarios at which this detector excels and its performance for medium window sizes. The standard deviation multiplier is significant. If the data follows a standard Gaussian distribution, the detected points would fall outside 99.999\% of the data present in the window, which would represent a significant outlier comparative to the rest of the data. The rolling range multiplier is disabled for this experiment since the outliers are not relational to each other. There are 4 different types of outliers which all have radically different behaviors and signal shapes, so it is not desired to compare or remember them in the context of the next outlier. The recent range detection debounce multiplier is only one due to the larger size of the time series window and overall window size.

\begin{table}[H]
%%\centering
\caption{PEC Dataset Detector Parameters}
\begin{tabular}{|l|c|l|}
    \hline
	\textbf{Parameter} & \textbf{Value} & \textbf{Description} \\ \hline
	m & 250 & Window Size \\ \hline
	ts$\_$size & 5000 & Time Series Size \\ \hline
	std$\_$dev & 4 & Standard Deviation Multiplier \\ \hline
	range & 0 & Rolling Range Multiplier\\ \hline
	recent & 1 & Recent Detection Debounce\\ \hline
\end{tabular}
\label{tab:pec_sim_params}
\end{table}

Figure \ref{fig:pec_mp_hist} shows the computed matrix profile values during the experiment. Because of the scale variance of the anomalies, only 2 of the 4 detected anomalies are visually represented in the figure. The 2 anomalies that are not depicted show similar patterns, but at much smaller scale. This shows the detector is able to perform on signals with widely differing anomaly characteristics and patterns.

\begin{figure}[H]
    %%\centering
    %% Creator: Matplotlib, PGF backend
%%
%% To include the figure in your LaTeX document, write
%%   \input{<filename>.pgf}
%%
%% Make sure the required packages are loaded in your preamble
%%   \usepackage{pgf}
%%
%% Also ensure that all the required font packages are loaded; for instance,
%% the lmodern package is sometimes necessary when using math font.
%%   \usepackage{lmodern}
%%
%% Figures using additional raster images can only be included by \input if
%% they are in the same directory as the main LaTeX file. For loading figures
%% from other directories you can use the `import` package
%%   \usepackage{import}
%%
%% and then include the figures with
%%   \import{<path to file>}{<filename>.pgf}
%%
%% Matplotlib used the following preamble
%%
\begingroup%
\makeatletter%
\begin{pgfpicture}%
\pgfpathrectangle{\pgfpointorigin}{\pgfqpoint{6.000000in}{4.000000in}}%
\pgfusepath{use as bounding box, clip}%
\begin{pgfscope}%
\pgfsetbuttcap%
\pgfsetmiterjoin%
\pgfsetlinewidth{0.000000pt}%
\definecolor{currentstroke}{rgb}{1.000000,1.000000,1.000000}%
\pgfsetstrokecolor{currentstroke}%
\pgfsetstrokeopacity{0.000000}%
\pgfsetdash{}{0pt}%
\pgfpathmoveto{\pgfqpoint{0.000000in}{0.000000in}}%
\pgfpathlineto{\pgfqpoint{6.000000in}{0.000000in}}%
\pgfpathlineto{\pgfqpoint{6.000000in}{4.000000in}}%
\pgfpathlineto{\pgfqpoint{0.000000in}{4.000000in}}%
\pgfpathlineto{\pgfqpoint{0.000000in}{0.000000in}}%
\pgfpathclose%
\pgfusepath{}%
\end{pgfscope}%
\begin{pgfscope}%
\pgfsetbuttcap%
\pgfsetmiterjoin%
\definecolor{currentfill}{rgb}{1.000000,1.000000,1.000000}%
\pgfsetfillcolor{currentfill}%
\pgfsetlinewidth{0.000000pt}%
\definecolor{currentstroke}{rgb}{0.000000,0.000000,0.000000}%
\pgfsetstrokecolor{currentstroke}%
\pgfsetstrokeopacity{0.000000}%
\pgfsetdash{}{0pt}%
\pgfpathmoveto{\pgfqpoint{0.750000in}{0.500000in}}%
\pgfpathlineto{\pgfqpoint{5.400000in}{0.500000in}}%
\pgfpathlineto{\pgfqpoint{5.400000in}{3.520000in}}%
\pgfpathlineto{\pgfqpoint{0.750000in}{3.520000in}}%
\pgfpathlineto{\pgfqpoint{0.750000in}{0.500000in}}%
\pgfpathclose%
\pgfusepath{fill}%
\end{pgfscope}%
\begin{pgfscope}%
\pgfsetbuttcap%
\pgfsetroundjoin%
\definecolor{currentfill}{rgb}{0.000000,0.000000,0.000000}%
\pgfsetfillcolor{currentfill}%
\pgfsetlinewidth{0.803000pt}%
\definecolor{currentstroke}{rgb}{0.000000,0.000000,0.000000}%
\pgfsetstrokecolor{currentstroke}%
\pgfsetdash{}{0pt}%
\pgfsys@defobject{currentmarker}{\pgfqpoint{0.000000in}{-0.048611in}}{\pgfqpoint{0.000000in}{0.000000in}}{%
\pgfpathmoveto{\pgfqpoint{0.000000in}{0.000000in}}%
\pgfpathlineto{\pgfqpoint{0.000000in}{-0.048611in}}%
\pgfusepath{stroke,fill}%
}%
\begin{pgfscope}%
\pgfsys@transformshift{0.961364in}{0.500000in}%
\pgfsys@useobject{currentmarker}{}%
\end{pgfscope}%
\end{pgfscope}%
\begin{pgfscope}%
\definecolor{textcolor}{rgb}{0.000000,0.000000,0.000000}%
\pgfsetstrokecolor{textcolor}%
\pgfsetfillcolor{textcolor}%
\pgftext[x=0.961364in,y=0.402778in,,top]{\color{textcolor}\rmfamily\fontsize{10.000000}{12.000000}\selectfont \(\displaystyle {0}\)}%
\end{pgfscope}%
\begin{pgfscope}%
\pgfsetbuttcap%
\pgfsetroundjoin%
\definecolor{currentfill}{rgb}{0.000000,0.000000,0.000000}%
\pgfsetfillcolor{currentfill}%
\pgfsetlinewidth{0.803000pt}%
\definecolor{currentstroke}{rgb}{0.000000,0.000000,0.000000}%
\pgfsetstrokecolor{currentstroke}%
\pgfsetdash{}{0pt}%
\pgfsys@defobject{currentmarker}{\pgfqpoint{0.000000in}{-0.048611in}}{\pgfqpoint{0.000000in}{0.000000in}}{%
\pgfpathmoveto{\pgfqpoint{0.000000in}{0.000000in}}%
\pgfpathlineto{\pgfqpoint{0.000000in}{-0.048611in}}%
\pgfusepath{stroke,fill}%
}%
\begin{pgfscope}%
\pgfsys@transformshift{1.632244in}{0.500000in}%
\pgfsys@useobject{currentmarker}{}%
\end{pgfscope}%
\end{pgfscope}%
\begin{pgfscope}%
\definecolor{textcolor}{rgb}{0.000000,0.000000,0.000000}%
\pgfsetstrokecolor{textcolor}%
\pgfsetfillcolor{textcolor}%
\pgftext[x=1.632244in,y=0.402778in,,top]{\color{textcolor}\rmfamily\fontsize{10.000000}{12.000000}\selectfont \(\displaystyle {50000}\)}%
\end{pgfscope}%
\begin{pgfscope}%
\pgfsetbuttcap%
\pgfsetroundjoin%
\definecolor{currentfill}{rgb}{0.000000,0.000000,0.000000}%
\pgfsetfillcolor{currentfill}%
\pgfsetlinewidth{0.803000pt}%
\definecolor{currentstroke}{rgb}{0.000000,0.000000,0.000000}%
\pgfsetstrokecolor{currentstroke}%
\pgfsetdash{}{0pt}%
\pgfsys@defobject{currentmarker}{\pgfqpoint{0.000000in}{-0.048611in}}{\pgfqpoint{0.000000in}{0.000000in}}{%
\pgfpathmoveto{\pgfqpoint{0.000000in}{0.000000in}}%
\pgfpathlineto{\pgfqpoint{0.000000in}{-0.048611in}}%
\pgfusepath{stroke,fill}%
}%
\begin{pgfscope}%
\pgfsys@transformshift{2.303125in}{0.500000in}%
\pgfsys@useobject{currentmarker}{}%
\end{pgfscope}%
\end{pgfscope}%
\begin{pgfscope}%
\definecolor{textcolor}{rgb}{0.000000,0.000000,0.000000}%
\pgfsetstrokecolor{textcolor}%
\pgfsetfillcolor{textcolor}%
\pgftext[x=2.303125in,y=0.402778in,,top]{\color{textcolor}\rmfamily\fontsize{10.000000}{12.000000}\selectfont \(\displaystyle {100000}\)}%
\end{pgfscope}%
\begin{pgfscope}%
\pgfsetbuttcap%
\pgfsetroundjoin%
\definecolor{currentfill}{rgb}{0.000000,0.000000,0.000000}%
\pgfsetfillcolor{currentfill}%
\pgfsetlinewidth{0.803000pt}%
\definecolor{currentstroke}{rgb}{0.000000,0.000000,0.000000}%
\pgfsetstrokecolor{currentstroke}%
\pgfsetdash{}{0pt}%
\pgfsys@defobject{currentmarker}{\pgfqpoint{0.000000in}{-0.048611in}}{\pgfqpoint{0.000000in}{0.000000in}}{%
\pgfpathmoveto{\pgfqpoint{0.000000in}{0.000000in}}%
\pgfpathlineto{\pgfqpoint{0.000000in}{-0.048611in}}%
\pgfusepath{stroke,fill}%
}%
\begin{pgfscope}%
\pgfsys@transformshift{2.974006in}{0.500000in}%
\pgfsys@useobject{currentmarker}{}%
\end{pgfscope}%
\end{pgfscope}%
\begin{pgfscope}%
\definecolor{textcolor}{rgb}{0.000000,0.000000,0.000000}%
\pgfsetstrokecolor{textcolor}%
\pgfsetfillcolor{textcolor}%
\pgftext[x=2.974006in,y=0.402778in,,top]{\color{textcolor}\rmfamily\fontsize{10.000000}{12.000000}\selectfont \(\displaystyle {150000}\)}%
\end{pgfscope}%
\begin{pgfscope}%
\pgfsetbuttcap%
\pgfsetroundjoin%
\definecolor{currentfill}{rgb}{0.000000,0.000000,0.000000}%
\pgfsetfillcolor{currentfill}%
\pgfsetlinewidth{0.803000pt}%
\definecolor{currentstroke}{rgb}{0.000000,0.000000,0.000000}%
\pgfsetstrokecolor{currentstroke}%
\pgfsetdash{}{0pt}%
\pgfsys@defobject{currentmarker}{\pgfqpoint{0.000000in}{-0.048611in}}{\pgfqpoint{0.000000in}{0.000000in}}{%
\pgfpathmoveto{\pgfqpoint{0.000000in}{0.000000in}}%
\pgfpathlineto{\pgfqpoint{0.000000in}{-0.048611in}}%
\pgfusepath{stroke,fill}%
}%
\begin{pgfscope}%
\pgfsys@transformshift{3.644886in}{0.500000in}%
\pgfsys@useobject{currentmarker}{}%
\end{pgfscope}%
\end{pgfscope}%
\begin{pgfscope}%
\definecolor{textcolor}{rgb}{0.000000,0.000000,0.000000}%
\pgfsetstrokecolor{textcolor}%
\pgfsetfillcolor{textcolor}%
\pgftext[x=3.644886in,y=0.402778in,,top]{\color{textcolor}\rmfamily\fontsize{10.000000}{12.000000}\selectfont \(\displaystyle {200000}\)}%
\end{pgfscope}%
\begin{pgfscope}%
\pgfsetbuttcap%
\pgfsetroundjoin%
\definecolor{currentfill}{rgb}{0.000000,0.000000,0.000000}%
\pgfsetfillcolor{currentfill}%
\pgfsetlinewidth{0.803000pt}%
\definecolor{currentstroke}{rgb}{0.000000,0.000000,0.000000}%
\pgfsetstrokecolor{currentstroke}%
\pgfsetdash{}{0pt}%
\pgfsys@defobject{currentmarker}{\pgfqpoint{0.000000in}{-0.048611in}}{\pgfqpoint{0.000000in}{0.000000in}}{%
\pgfpathmoveto{\pgfqpoint{0.000000in}{0.000000in}}%
\pgfpathlineto{\pgfqpoint{0.000000in}{-0.048611in}}%
\pgfusepath{stroke,fill}%
}%
\begin{pgfscope}%
\pgfsys@transformshift{4.315767in}{0.500000in}%
\pgfsys@useobject{currentmarker}{}%
\end{pgfscope}%
\end{pgfscope}%
\begin{pgfscope}%
\definecolor{textcolor}{rgb}{0.000000,0.000000,0.000000}%
\pgfsetstrokecolor{textcolor}%
\pgfsetfillcolor{textcolor}%
\pgftext[x=4.315767in,y=0.402778in,,top]{\color{textcolor}\rmfamily\fontsize{10.000000}{12.000000}\selectfont \(\displaystyle {250000}\)}%
\end{pgfscope}%
\begin{pgfscope}%
\pgfsetbuttcap%
\pgfsetroundjoin%
\definecolor{currentfill}{rgb}{0.000000,0.000000,0.000000}%
\pgfsetfillcolor{currentfill}%
\pgfsetlinewidth{0.803000pt}%
\definecolor{currentstroke}{rgb}{0.000000,0.000000,0.000000}%
\pgfsetstrokecolor{currentstroke}%
\pgfsetdash{}{0pt}%
\pgfsys@defobject{currentmarker}{\pgfqpoint{0.000000in}{-0.048611in}}{\pgfqpoint{0.000000in}{0.000000in}}{%
\pgfpathmoveto{\pgfqpoint{0.000000in}{0.000000in}}%
\pgfpathlineto{\pgfqpoint{0.000000in}{-0.048611in}}%
\pgfusepath{stroke,fill}%
}%
\begin{pgfscope}%
\pgfsys@transformshift{4.986648in}{0.500000in}%
\pgfsys@useobject{currentmarker}{}%
\end{pgfscope}%
\end{pgfscope}%
\begin{pgfscope}%
\definecolor{textcolor}{rgb}{0.000000,0.000000,0.000000}%
\pgfsetstrokecolor{textcolor}%
\pgfsetfillcolor{textcolor}%
\pgftext[x=4.986648in,y=0.402778in,,top]{\color{textcolor}\rmfamily\fontsize{10.000000}{12.000000}\selectfont \(\displaystyle {300000}\)}%
\end{pgfscope}%
\begin{pgfscope}%
\definecolor{textcolor}{rgb}{0.000000,0.000000,0.000000}%
\pgfsetstrokecolor{textcolor}%
\pgfsetfillcolor{textcolor}%
\pgftext[x=3.075000in,y=0.223766in,,top]{\color{textcolor}\rmfamily\fontsize{10.000000}{12.000000}\selectfont time}%
\end{pgfscope}%
\begin{pgfscope}%
\pgfsetbuttcap%
\pgfsetroundjoin%
\definecolor{currentfill}{rgb}{0.000000,0.000000,0.000000}%
\pgfsetfillcolor{currentfill}%
\pgfsetlinewidth{0.803000pt}%
\definecolor{currentstroke}{rgb}{0.000000,0.000000,0.000000}%
\pgfsetstrokecolor{currentstroke}%
\pgfsetdash{}{0pt}%
\pgfsys@defobject{currentmarker}{\pgfqpoint{-0.048611in}{0.000000in}}{\pgfqpoint{-0.000000in}{0.000000in}}{%
\pgfpathmoveto{\pgfqpoint{-0.000000in}{0.000000in}}%
\pgfpathlineto{\pgfqpoint{-0.048611in}{0.000000in}}%
\pgfusepath{stroke,fill}%
}%
\begin{pgfscope}%
\pgfsys@transformshift{0.750000in}{0.637273in}%
\pgfsys@useobject{currentmarker}{}%
\end{pgfscope}%
\end{pgfscope}%
\begin{pgfscope}%
\definecolor{textcolor}{rgb}{0.000000,0.000000,0.000000}%
\pgfsetstrokecolor{textcolor}%
\pgfsetfillcolor{textcolor}%
\pgftext[x=0.583333in, y=0.589047in, left, base]{\color{textcolor}\rmfamily\fontsize{10.000000}{12.000000}\selectfont \(\displaystyle {0}\)}%
\end{pgfscope}%
\begin{pgfscope}%
\pgfsetbuttcap%
\pgfsetroundjoin%
\definecolor{currentfill}{rgb}{0.000000,0.000000,0.000000}%
\pgfsetfillcolor{currentfill}%
\pgfsetlinewidth{0.803000pt}%
\definecolor{currentstroke}{rgb}{0.000000,0.000000,0.000000}%
\pgfsetstrokecolor{currentstroke}%
\pgfsetdash{}{0pt}%
\pgfsys@defobject{currentmarker}{\pgfqpoint{-0.048611in}{0.000000in}}{\pgfqpoint{-0.000000in}{0.000000in}}{%
\pgfpathmoveto{\pgfqpoint{-0.000000in}{0.000000in}}%
\pgfpathlineto{\pgfqpoint{-0.048611in}{0.000000in}}%
\pgfusepath{stroke,fill}%
}%
\begin{pgfscope}%
\pgfsys@transformshift{0.750000in}{1.112883in}%
\pgfsys@useobject{currentmarker}{}%
\end{pgfscope}%
\end{pgfscope}%
\begin{pgfscope}%
\definecolor{textcolor}{rgb}{0.000000,0.000000,0.000000}%
\pgfsetstrokecolor{textcolor}%
\pgfsetfillcolor{textcolor}%
\pgftext[x=0.305554in, y=1.064658in, left, base]{\color{textcolor}\rmfamily\fontsize{10.000000}{12.000000}\selectfont \(\displaystyle {10000}\)}%
\end{pgfscope}%
\begin{pgfscope}%
\pgfsetbuttcap%
\pgfsetroundjoin%
\definecolor{currentfill}{rgb}{0.000000,0.000000,0.000000}%
\pgfsetfillcolor{currentfill}%
\pgfsetlinewidth{0.803000pt}%
\definecolor{currentstroke}{rgb}{0.000000,0.000000,0.000000}%
\pgfsetstrokecolor{currentstroke}%
\pgfsetdash{}{0pt}%
\pgfsys@defobject{currentmarker}{\pgfqpoint{-0.048611in}{0.000000in}}{\pgfqpoint{-0.000000in}{0.000000in}}{%
\pgfpathmoveto{\pgfqpoint{-0.000000in}{0.000000in}}%
\pgfpathlineto{\pgfqpoint{-0.048611in}{0.000000in}}%
\pgfusepath{stroke,fill}%
}%
\begin{pgfscope}%
\pgfsys@transformshift{0.750000in}{1.588494in}%
\pgfsys@useobject{currentmarker}{}%
\end{pgfscope}%
\end{pgfscope}%
\begin{pgfscope}%
\definecolor{textcolor}{rgb}{0.000000,0.000000,0.000000}%
\pgfsetstrokecolor{textcolor}%
\pgfsetfillcolor{textcolor}%
\pgftext[x=0.305554in, y=1.540268in, left, base]{\color{textcolor}\rmfamily\fontsize{10.000000}{12.000000}\selectfont \(\displaystyle {20000}\)}%
\end{pgfscope}%
\begin{pgfscope}%
\pgfsetbuttcap%
\pgfsetroundjoin%
\definecolor{currentfill}{rgb}{0.000000,0.000000,0.000000}%
\pgfsetfillcolor{currentfill}%
\pgfsetlinewidth{0.803000pt}%
\definecolor{currentstroke}{rgb}{0.000000,0.000000,0.000000}%
\pgfsetstrokecolor{currentstroke}%
\pgfsetdash{}{0pt}%
\pgfsys@defobject{currentmarker}{\pgfqpoint{-0.048611in}{0.000000in}}{\pgfqpoint{-0.000000in}{0.000000in}}{%
\pgfpathmoveto{\pgfqpoint{-0.000000in}{0.000000in}}%
\pgfpathlineto{\pgfqpoint{-0.048611in}{0.000000in}}%
\pgfusepath{stroke,fill}%
}%
\begin{pgfscope}%
\pgfsys@transformshift{0.750000in}{2.064104in}%
\pgfsys@useobject{currentmarker}{}%
\end{pgfscope}%
\end{pgfscope}%
\begin{pgfscope}%
\definecolor{textcolor}{rgb}{0.000000,0.000000,0.000000}%
\pgfsetstrokecolor{textcolor}%
\pgfsetfillcolor{textcolor}%
\pgftext[x=0.305554in, y=2.015879in, left, base]{\color{textcolor}\rmfamily\fontsize{10.000000}{12.000000}\selectfont \(\displaystyle {30000}\)}%
\end{pgfscope}%
\begin{pgfscope}%
\pgfsetbuttcap%
\pgfsetroundjoin%
\definecolor{currentfill}{rgb}{0.000000,0.000000,0.000000}%
\pgfsetfillcolor{currentfill}%
\pgfsetlinewidth{0.803000pt}%
\definecolor{currentstroke}{rgb}{0.000000,0.000000,0.000000}%
\pgfsetstrokecolor{currentstroke}%
\pgfsetdash{}{0pt}%
\pgfsys@defobject{currentmarker}{\pgfqpoint{-0.048611in}{0.000000in}}{\pgfqpoint{-0.000000in}{0.000000in}}{%
\pgfpathmoveto{\pgfqpoint{-0.000000in}{0.000000in}}%
\pgfpathlineto{\pgfqpoint{-0.048611in}{0.000000in}}%
\pgfusepath{stroke,fill}%
}%
\begin{pgfscope}%
\pgfsys@transformshift{0.750000in}{2.539714in}%
\pgfsys@useobject{currentmarker}{}%
\end{pgfscope}%
\end{pgfscope}%
\begin{pgfscope}%
\definecolor{textcolor}{rgb}{0.000000,0.000000,0.000000}%
\pgfsetstrokecolor{textcolor}%
\pgfsetfillcolor{textcolor}%
\pgftext[x=0.305554in, y=2.491489in, left, base]{\color{textcolor}\rmfamily\fontsize{10.000000}{12.000000}\selectfont \(\displaystyle {40000}\)}%
\end{pgfscope}%
\begin{pgfscope}%
\pgfsetbuttcap%
\pgfsetroundjoin%
\definecolor{currentfill}{rgb}{0.000000,0.000000,0.000000}%
\pgfsetfillcolor{currentfill}%
\pgfsetlinewidth{0.803000pt}%
\definecolor{currentstroke}{rgb}{0.000000,0.000000,0.000000}%
\pgfsetstrokecolor{currentstroke}%
\pgfsetdash{}{0pt}%
\pgfsys@defobject{currentmarker}{\pgfqpoint{-0.048611in}{0.000000in}}{\pgfqpoint{-0.000000in}{0.000000in}}{%
\pgfpathmoveto{\pgfqpoint{-0.000000in}{0.000000in}}%
\pgfpathlineto{\pgfqpoint{-0.048611in}{0.000000in}}%
\pgfusepath{stroke,fill}%
}%
\begin{pgfscope}%
\pgfsys@transformshift{0.750000in}{3.015325in}%
\pgfsys@useobject{currentmarker}{}%
\end{pgfscope}%
\end{pgfscope}%
\begin{pgfscope}%
\definecolor{textcolor}{rgb}{0.000000,0.000000,0.000000}%
\pgfsetstrokecolor{textcolor}%
\pgfsetfillcolor{textcolor}%
\pgftext[x=0.305554in, y=2.967100in, left, base]{\color{textcolor}\rmfamily\fontsize{10.000000}{12.000000}\selectfont \(\displaystyle {50000}\)}%
\end{pgfscope}%
\begin{pgfscope}%
\pgfsetbuttcap%
\pgfsetroundjoin%
\definecolor{currentfill}{rgb}{0.000000,0.000000,0.000000}%
\pgfsetfillcolor{currentfill}%
\pgfsetlinewidth{0.803000pt}%
\definecolor{currentstroke}{rgb}{0.000000,0.000000,0.000000}%
\pgfsetstrokecolor{currentstroke}%
\pgfsetdash{}{0pt}%
\pgfsys@defobject{currentmarker}{\pgfqpoint{-0.048611in}{0.000000in}}{\pgfqpoint{-0.000000in}{0.000000in}}{%
\pgfpathmoveto{\pgfqpoint{-0.000000in}{0.000000in}}%
\pgfpathlineto{\pgfqpoint{-0.048611in}{0.000000in}}%
\pgfusepath{stroke,fill}%
}%
\begin{pgfscope}%
\pgfsys@transformshift{0.750000in}{3.490935in}%
\pgfsys@useobject{currentmarker}{}%
\end{pgfscope}%
\end{pgfscope}%
\begin{pgfscope}%
\definecolor{textcolor}{rgb}{0.000000,0.000000,0.000000}%
\pgfsetstrokecolor{textcolor}%
\pgfsetfillcolor{textcolor}%
\pgftext[x=0.305554in, y=3.442710in, left, base]{\color{textcolor}\rmfamily\fontsize{10.000000}{12.000000}\selectfont \(\displaystyle {60000}\)}%
\end{pgfscope}%
\begin{pgfscope}%
\pgfpathrectangle{\pgfqpoint{0.750000in}{0.500000in}}{\pgfqpoint{4.650000in}{3.020000in}}%
\pgfusepath{clip}%
\pgfsetrectcap%
\pgfsetroundjoin%
\pgfsetlinewidth{1.505625pt}%
\definecolor{currentstroke}{rgb}{0.000000,0.000000,1.000000}%
\pgfsetstrokecolor{currentstroke}%
\pgfsetdash{}{0pt}%
\pgfpathmoveto{\pgfqpoint{0.961364in}{0.637273in}}%
\pgfpathlineto{\pgfqpoint{1.435542in}{0.637273in}}%
\pgfpathlineto{\pgfqpoint{1.437125in}{0.911208in}}%
\pgfpathlineto{\pgfqpoint{1.439742in}{0.912562in}}%
\pgfpathlineto{\pgfqpoint{1.441298in}{0.908808in}}%
\pgfpathlineto{\pgfqpoint{1.441620in}{0.908683in}}%
\pgfpathlineto{\pgfqpoint{1.444988in}{2.989772in}}%
\pgfpathlineto{\pgfqpoint{1.446249in}{3.096074in}}%
\pgfpathlineto{\pgfqpoint{1.446424in}{2.964231in}}%
\pgfpathlineto{\pgfqpoint{1.448181in}{2.855428in}}%
\pgfpathlineto{\pgfqpoint{1.508561in}{2.855428in}}%
\pgfpathlineto{\pgfqpoint{1.510520in}{2.645898in}}%
\pgfpathlineto{\pgfqpoint{1.510775in}{2.645861in}}%
\pgfpathlineto{\pgfqpoint{1.514116in}{0.646721in}}%
\pgfpathlineto{\pgfqpoint{1.522850in}{0.645183in}}%
\pgfpathlineto{\pgfqpoint{1.527426in}{0.641315in}}%
\pgfpathlineto{\pgfqpoint{1.534671in}{0.639773in}}%
\pgfpathlineto{\pgfqpoint{1.546412in}{0.638308in}}%
\pgfpathlineto{\pgfqpoint{3.583112in}{0.639850in}}%
\pgfpathlineto{\pgfqpoint{3.589552in}{0.639139in}}%
\pgfpathlineto{\pgfqpoint{3.620413in}{0.637865in}}%
\pgfpathlineto{\pgfqpoint{4.656494in}{0.637273in}}%
\pgfpathlineto{\pgfqpoint{4.658144in}{3.382714in}}%
\pgfpathlineto{\pgfqpoint{4.722723in}{3.381537in}}%
\pgfpathlineto{\pgfqpoint{4.724280in}{3.379132in}}%
\pgfpathlineto{\pgfqpoint{4.775857in}{3.379125in}}%
\pgfpathlineto{\pgfqpoint{4.776031in}{3.106482in}}%
\pgfpathlineto{\pgfqpoint{4.778956in}{0.639340in}}%
\pgfpathlineto{\pgfqpoint{4.782203in}{0.638979in}}%
\pgfpathlineto{\pgfqpoint{4.815949in}{0.637436in}}%
\pgfpathlineto{\pgfqpoint{4.858362in}{0.637281in}}%
\pgfpathlineto{\pgfqpoint{5.188636in}{0.637273in}}%
\pgfpathlineto{\pgfqpoint{5.188636in}{0.637273in}}%
\pgfusepath{stroke}%
\end{pgfscope}%
\begin{pgfscope}%
\pgfpathrectangle{\pgfqpoint{0.750000in}{0.500000in}}{\pgfqpoint{4.650000in}{3.020000in}}%
\pgfusepath{clip}%
\pgfsetrectcap%
\pgfsetroundjoin%
\pgfsetlinewidth{1.505625pt}%
\definecolor{currentstroke}{rgb}{1.000000,0.000000,0.000000}%
\pgfsetstrokecolor{currentstroke}%
\pgfsetdash{}{0pt}%
\pgfpathmoveto{\pgfqpoint{0.961364in}{0.637273in}}%
\pgfpathlineto{\pgfqpoint{1.435904in}{0.638780in}}%
\pgfpathlineto{\pgfqpoint{1.439782in}{0.652044in}}%
\pgfpathlineto{\pgfqpoint{1.440480in}{0.651649in}}%
\pgfpathlineto{\pgfqpoint{1.441620in}{0.651577in}}%
\pgfpathlineto{\pgfqpoint{1.441808in}{0.652641in}}%
\pgfpathlineto{\pgfqpoint{1.442895in}{0.679423in}}%
\pgfpathlineto{\pgfqpoint{1.448785in}{0.827662in}}%
\pgfpathlineto{\pgfqpoint{1.449282in}{0.827620in}}%
\pgfpathlineto{\pgfqpoint{1.457641in}{0.828529in}}%
\pgfpathlineto{\pgfqpoint{1.481994in}{0.829362in}}%
\pgfpathlineto{\pgfqpoint{1.499665in}{0.828023in}}%
\pgfpathlineto{\pgfqpoint{1.504227in}{0.815387in}}%
\pgfpathlineto{\pgfqpoint{1.505582in}{0.813928in}}%
\pgfpathlineto{\pgfqpoint{1.506843in}{0.781270in}}%
\pgfpathlineto{\pgfqpoint{1.513029in}{0.639236in}}%
\pgfpathlineto{\pgfqpoint{1.554503in}{0.637388in}}%
\pgfpathlineto{\pgfqpoint{1.777316in}{0.637273in}}%
\pgfpathlineto{\pgfqpoint{4.656615in}{0.638667in}}%
\pgfpathlineto{\pgfqpoint{4.657930in}{0.687051in}}%
\pgfpathlineto{\pgfqpoint{4.660492in}{0.784806in}}%
\pgfpathlineto{\pgfqpoint{4.661230in}{0.763086in}}%
\pgfpathlineto{\pgfqpoint{4.661552in}{0.757392in}}%
\pgfpathlineto{\pgfqpoint{4.662975in}{0.757688in}}%
\pgfpathlineto{\pgfqpoint{4.678928in}{0.759260in}}%
\pgfpathlineto{\pgfqpoint{4.698867in}{0.759935in}}%
\pgfpathlineto{\pgfqpoint{4.709762in}{0.761407in}}%
\pgfpathlineto{\pgfqpoint{4.711077in}{0.809722in}}%
\pgfpathlineto{\pgfqpoint{4.714243in}{0.879746in}}%
\pgfpathlineto{\pgfqpoint{4.720335in}{0.879079in}}%
\pgfpathlineto{\pgfqpoint{4.720362in}{0.878658in}}%
\pgfpathlineto{\pgfqpoint{4.721677in}{0.830265in}}%
\pgfpathlineto{\pgfqpoint{4.724843in}{0.760013in}}%
\pgfpathlineto{\pgfqpoint{4.741682in}{0.758743in}}%
\pgfpathlineto{\pgfqpoint{4.773549in}{0.755339in}}%
\pgfpathlineto{\pgfqpoint{4.775119in}{0.695046in}}%
\pgfpathlineto{\pgfqpoint{4.777990in}{0.637778in}}%
\pgfpathlineto{\pgfqpoint{4.820631in}{0.637298in}}%
\pgfpathlineto{\pgfqpoint{5.188636in}{0.637273in}}%
\pgfpathlineto{\pgfqpoint{5.188636in}{0.637273in}}%
\pgfusepath{stroke}%
\end{pgfscope}%
\begin{pgfscope}%
\pgfpathrectangle{\pgfqpoint{0.750000in}{0.500000in}}{\pgfqpoint{4.650000in}{3.020000in}}%
\pgfusepath{clip}%
\pgfsetrectcap%
\pgfsetroundjoin%
\pgfsetlinewidth{1.505625pt}%
\definecolor{currentstroke}{rgb}{0.000000,0.500000,0.000000}%
\pgfsetstrokecolor{currentstroke}%
\pgfsetdash{}{0pt}%
\pgfpathmoveto{\pgfqpoint{0.961364in}{0.637273in}}%
\pgfpathlineto{\pgfqpoint{1.435556in}{0.638439in}}%
\pgfpathlineto{\pgfqpoint{1.437300in}{0.681870in}}%
\pgfpathlineto{\pgfqpoint{1.438923in}{0.698448in}}%
\pgfpathlineto{\pgfqpoint{1.440104in}{0.697684in}}%
\pgfpathlineto{\pgfqpoint{1.441754in}{0.698904in}}%
\pgfpathlineto{\pgfqpoint{1.442170in}{0.740756in}}%
\pgfpathlineto{\pgfqpoint{1.446303in}{1.133484in}}%
\pgfpathlineto{\pgfqpoint{1.449859in}{1.184509in}}%
\pgfpathlineto{\pgfqpoint{1.449886in}{1.184508in}}%
\pgfpathlineto{\pgfqpoint{1.499665in}{1.184020in}}%
\pgfpathlineto{\pgfqpoint{1.505555in}{1.185168in}}%
\pgfpathlineto{\pgfqpoint{1.505622in}{1.184899in}}%
\pgfpathlineto{\pgfqpoint{1.506172in}{1.172972in}}%
\pgfpathlineto{\pgfqpoint{1.509097in}{0.996424in}}%
\pgfpathlineto{\pgfqpoint{1.511432in}{0.680005in}}%
\pgfpathlineto{\pgfqpoint{1.513190in}{0.639392in}}%
\pgfpathlineto{\pgfqpoint{1.677220in}{0.637273in}}%
\pgfpathlineto{\pgfqpoint{4.656494in}{0.637273in}}%
\pgfpathlineto{\pgfqpoint{4.659486in}{1.175948in}}%
\pgfpathlineto{\pgfqpoint{4.660251in}{1.228988in}}%
\pgfpathlineto{\pgfqpoint{4.661069in}{1.180007in}}%
\pgfpathlineto{\pgfqpoint{4.662854in}{1.166981in}}%
\pgfpathlineto{\pgfqpoint{4.709453in}{1.166411in}}%
\pgfpathlineto{\pgfqpoint{4.709802in}{1.167734in}}%
\pgfpathlineto{\pgfqpoint{4.710513in}{1.208294in}}%
\pgfpathlineto{\pgfqpoint{4.712969in}{1.365780in}}%
\pgfpathlineto{\pgfqpoint{4.713760in}{1.365592in}}%
\pgfpathlineto{\pgfqpoint{4.720496in}{1.363548in}}%
\pgfpathlineto{\pgfqpoint{4.721301in}{1.322938in}}%
\pgfpathlineto{\pgfqpoint{4.723582in}{1.165572in}}%
\pgfpathlineto{\pgfqpoint{4.724991in}{1.165627in}}%
\pgfpathlineto{\pgfqpoint{4.773469in}{1.165884in}}%
\pgfpathlineto{\pgfqpoint{4.773952in}{1.147630in}}%
\pgfpathlineto{\pgfqpoint{4.775589in}{0.925642in}}%
\pgfpathlineto{\pgfqpoint{4.778567in}{0.637822in}}%
\pgfpathlineto{\pgfqpoint{4.819276in}{0.637303in}}%
\pgfpathlineto{\pgfqpoint{5.188636in}{0.637273in}}%
\pgfpathlineto{\pgfqpoint{5.188636in}{0.637273in}}%
\pgfusepath{stroke}%
\end{pgfscope}%
\begin{pgfscope}%
\pgfsetrectcap%
\pgfsetmiterjoin%
\pgfsetlinewidth{0.803000pt}%
\definecolor{currentstroke}{rgb}{0.000000,0.000000,0.000000}%
\pgfsetstrokecolor{currentstroke}%
\pgfsetdash{}{0pt}%
\pgfpathmoveto{\pgfqpoint{0.750000in}{0.500000in}}%
\pgfpathlineto{\pgfqpoint{0.750000in}{3.520000in}}%
\pgfusepath{stroke}%
\end{pgfscope}%
\begin{pgfscope}%
\pgfsetrectcap%
\pgfsetmiterjoin%
\pgfsetlinewidth{0.803000pt}%
\definecolor{currentstroke}{rgb}{0.000000,0.000000,0.000000}%
\pgfsetstrokecolor{currentstroke}%
\pgfsetdash{}{0pt}%
\pgfpathmoveto{\pgfqpoint{5.400000in}{0.500000in}}%
\pgfpathlineto{\pgfqpoint{5.400000in}{3.520000in}}%
\pgfusepath{stroke}%
\end{pgfscope}%
\begin{pgfscope}%
\pgfsetrectcap%
\pgfsetmiterjoin%
\pgfsetlinewidth{0.803000pt}%
\definecolor{currentstroke}{rgb}{0.000000,0.000000,0.000000}%
\pgfsetstrokecolor{currentstroke}%
\pgfsetdash{}{0pt}%
\pgfpathmoveto{\pgfqpoint{0.750000in}{0.500000in}}%
\pgfpathlineto{\pgfqpoint{5.400000in}{0.500000in}}%
\pgfusepath{stroke}%
\end{pgfscope}%
\begin{pgfscope}%
\pgfsetrectcap%
\pgfsetmiterjoin%
\pgfsetlinewidth{0.803000pt}%
\definecolor{currentstroke}{rgb}{0.000000,0.000000,0.000000}%
\pgfsetstrokecolor{currentstroke}%
\pgfsetdash{}{0pt}%
\pgfpathmoveto{\pgfqpoint{0.750000in}{3.520000in}}%
\pgfpathlineto{\pgfqpoint{5.400000in}{3.520000in}}%
\pgfusepath{stroke}%
\end{pgfscope}%
\begin{pgfscope}%
\pgfsetbuttcap%
\pgfsetmiterjoin%
\definecolor{currentfill}{rgb}{1.000000,1.000000,1.000000}%
\pgfsetfillcolor{currentfill}%
\pgfsetfillopacity{0.800000}%
\pgfsetlinewidth{1.003750pt}%
\definecolor{currentstroke}{rgb}{0.800000,0.800000,0.800000}%
\pgfsetstrokecolor{currentstroke}%
\pgfsetstrokeopacity{0.800000}%
\pgfsetdash{}{0pt}%
\pgfpathmoveto{\pgfqpoint{2.723533in}{2.827871in}}%
\pgfpathlineto{\pgfqpoint{3.426467in}{2.827871in}}%
\pgfpathquadraticcurveto{\pgfqpoint{3.454244in}{2.827871in}}{\pgfqpoint{3.454244in}{2.855648in}}%
\pgfpathlineto{\pgfqpoint{3.454244in}{3.422778in}}%
\pgfpathquadraticcurveto{\pgfqpoint{3.454244in}{3.450556in}}{\pgfqpoint{3.426467in}{3.450556in}}%
\pgfpathlineto{\pgfqpoint{2.723533in}{3.450556in}}%
\pgfpathquadraticcurveto{\pgfqpoint{2.695756in}{3.450556in}}{\pgfqpoint{2.695756in}{3.422778in}}%
\pgfpathlineto{\pgfqpoint{2.695756in}{2.855648in}}%
\pgfpathquadraticcurveto{\pgfqpoint{2.695756in}{2.827871in}}{\pgfqpoint{2.723533in}{2.827871in}}%
\pgfpathlineto{\pgfqpoint{2.723533in}{2.827871in}}%
\pgfpathclose%
\pgfusepath{stroke,fill}%
\end{pgfscope}%
\begin{pgfscope}%
\pgfsetrectcap%
\pgfsetroundjoin%
\pgfsetlinewidth{1.505625pt}%
\definecolor{currentstroke}{rgb}{0.000000,0.000000,1.000000}%
\pgfsetstrokecolor{currentstroke}%
\pgfsetdash{}{0pt}%
\pgfpathmoveto{\pgfqpoint{2.751311in}{3.346389in}}%
\pgfpathlineto{\pgfqpoint{2.890200in}{3.346389in}}%
\pgfpathlineto{\pgfqpoint{3.029089in}{3.346389in}}%
\pgfusepath{stroke}%
\end{pgfscope}%
\begin{pgfscope}%
\definecolor{textcolor}{rgb}{0.000000,0.000000,0.000000}%
\pgfsetstrokecolor{textcolor}%
\pgfsetfillcolor{textcolor}%
\pgftext[x=3.140200in,y=3.297778in,left,base]{\color{textcolor}\rmfamily\fontsize{10.000000}{12.000000}\selectfont max}%
\end{pgfscope}%
\begin{pgfscope}%
\pgfsetrectcap%
\pgfsetroundjoin%
\pgfsetlinewidth{1.505625pt}%
\definecolor{currentstroke}{rgb}{1.000000,0.000000,0.000000}%
\pgfsetstrokecolor{currentstroke}%
\pgfsetdash{}{0pt}%
\pgfpathmoveto{\pgfqpoint{2.751311in}{3.152716in}}%
\pgfpathlineto{\pgfqpoint{2.890200in}{3.152716in}}%
\pgfpathlineto{\pgfqpoint{3.029089in}{3.152716in}}%
\pgfusepath{stroke}%
\end{pgfscope}%
\begin{pgfscope}%
\definecolor{textcolor}{rgb}{0.000000,0.000000,0.000000}%
\pgfsetstrokecolor{textcolor}%
\pgfsetfillcolor{textcolor}%
\pgftext[x=3.140200in,y=3.104105in,left,base]{\color{textcolor}\rmfamily\fontsize{10.000000}{12.000000}\selectfont \(\displaystyle \mu\)}%
\end{pgfscope}%
\begin{pgfscope}%
\pgfsetrectcap%
\pgfsetroundjoin%
\pgfsetlinewidth{1.505625pt}%
\definecolor{currentstroke}{rgb}{0.000000,0.500000,0.000000}%
\pgfsetstrokecolor{currentstroke}%
\pgfsetdash{}{0pt}%
\pgfpathmoveto{\pgfqpoint{2.751311in}{2.959043in}}%
\pgfpathlineto{\pgfqpoint{2.890200in}{2.959043in}}%
\pgfpathlineto{\pgfqpoint{3.029089in}{2.959043in}}%
\pgfusepath{stroke}%
\end{pgfscope}%
\begin{pgfscope}%
\definecolor{textcolor}{rgb}{0.000000,0.000000,0.000000}%
\pgfsetstrokecolor{textcolor}%
\pgfsetfillcolor{textcolor}%
\pgftext[x=3.140200in,y=2.910432in,left,base]{\color{textcolor}\rmfamily\fontsize{10.000000}{12.000000}\selectfont \(\displaystyle \sigma\)}%
\end{pgfscope}%
\end{pgfpicture}%
\makeatother%
\endgroup%

    \caption{PEC Dataset Matrix Profile Values}
    \label{fig:pec_mp_hist}
\end{figure}

Figure \ref{fig:pec_outliers} shows the anomalies detected by the algorithm in red, over the manually labeled anomaly (true/false values) start and end periods. The detector accurately determined the start and end of each anomaly (100\% detection rate) with no false positives (0\% error rate). It detected the start of each anomaly almost immediately quickly detected the end of the corresponding anomaly once it was over, although this takes more time in certain cases. By nature, anomalies are rare so interpreting the detection and false positive rates in a traditional way is not possible. The interpretation of the results and significance are provided in Section \ref{sec:discussion}.
 
\begin{figure}[H]
    %%\centering
    %% Creator: Matplotlib, PGF backend
%%
%% To include the figure in your LaTeX document, write
%%   \input{<filename>.pgf}
%%
%% Make sure the required packages are loaded in your preamble
%%   \usepackage{pgf}
%%
%% Also ensure that all the required font packages are loaded; for instance,
%% the lmodern package is sometimes necessary when using math font.
%%   \usepackage{lmodern}
%%
%% Figures using additional raster images can only be included by \input if
%% they are in the same directory as the main LaTeX file. For loading figures
%% from other directories you can use the `import` package
%%   \usepackage{import}
%%
%% and then include the figures with
%%   \import{<path to file>}{<filename>.pgf}
%%
%% Matplotlib used the following preamble
%%
\begingroup%
\makeatletter%
\begin{pgfpicture}%
\pgfpathrectangle{\pgfpointorigin}{\pgfqpoint{5.500000in}{2.500000in}}%
\pgfusepath{use as bounding box, clip}%
\begin{pgfscope}%
\pgfsetbuttcap%
\pgfsetmiterjoin%
\pgfsetlinewidth{0.000000pt}%
\definecolor{currentstroke}{rgb}{1.000000,1.000000,1.000000}%
\pgfsetstrokecolor{currentstroke}%
\pgfsetstrokeopacity{0.000000}%
\pgfsetdash{}{0pt}%
\pgfpathmoveto{\pgfqpoint{0.000000in}{0.000000in}}%
\pgfpathlineto{\pgfqpoint{5.500000in}{0.000000in}}%
\pgfpathlineto{\pgfqpoint{5.500000in}{2.500000in}}%
\pgfpathlineto{\pgfqpoint{0.000000in}{2.500000in}}%
\pgfpathlineto{\pgfqpoint{0.000000in}{0.000000in}}%
\pgfpathclose%
\pgfusepath{}%
\end{pgfscope}%
\begin{pgfscope}%
\pgfsetbuttcap%
\pgfsetmiterjoin%
\definecolor{currentfill}{rgb}{1.000000,1.000000,1.000000}%
\pgfsetfillcolor{currentfill}%
\pgfsetlinewidth{0.000000pt}%
\definecolor{currentstroke}{rgb}{0.000000,0.000000,0.000000}%
\pgfsetstrokecolor{currentstroke}%
\pgfsetstrokeopacity{0.000000}%
\pgfsetdash{}{0pt}%
\pgfpathmoveto{\pgfqpoint{0.329012in}{0.565123in}}%
\pgfpathlineto{\pgfqpoint{5.350000in}{0.565123in}}%
\pgfpathlineto{\pgfqpoint{5.350000in}{2.350000in}}%
\pgfpathlineto{\pgfqpoint{0.329012in}{2.350000in}}%
\pgfpathlineto{\pgfqpoint{0.329012in}{0.565123in}}%
\pgfpathclose%
\pgfusepath{fill}%
\end{pgfscope}%
\begin{pgfscope}%
\pgfpathrectangle{\pgfqpoint{0.329012in}{0.565123in}}{\pgfqpoint{5.020988in}{1.784877in}}%
\pgfusepath{clip}%
\pgfsetbuttcap%
\pgfsetroundjoin%
\definecolor{currentfill}{rgb}{0.564706,0.933333,0.564706}%
\pgfsetfillcolor{currentfill}%
\pgfsetlinewidth{1.003750pt}%
\definecolor{currentstroke}{rgb}{0.564706,0.933333,0.564706}%
\pgfsetstrokecolor{currentstroke}%
\pgfsetdash{}{0pt}%
\pgfpathmoveto{\pgfqpoint{0.329012in}{0.565123in}}%
\pgfpathlineto{\pgfqpoint{0.329012in}{2.350000in}}%
\pgfpathlineto{\pgfqpoint{0.961785in}{2.350000in}}%
\pgfpathlineto{\pgfqpoint{0.961785in}{0.565123in}}%
\pgfpathlineto{\pgfqpoint{0.329012in}{0.565123in}}%
\pgfpathlineto{\pgfqpoint{0.329012in}{0.565123in}}%
\pgfpathclose%
\pgfusepath{stroke,fill}%
\end{pgfscope}%
\begin{pgfscope}%
\pgfpathrectangle{\pgfqpoint{0.329012in}{0.565123in}}{\pgfqpoint{5.020988in}{1.784877in}}%
\pgfusepath{clip}%
\pgfsetbuttcap%
\pgfsetroundjoin%
\definecolor{currentfill}{rgb}{0.564706,0.933333,0.564706}%
\pgfsetfillcolor{currentfill}%
\pgfsetlinewidth{1.003750pt}%
\definecolor{currentstroke}{rgb}{0.564706,0.933333,0.564706}%
\pgfsetstrokecolor{currentstroke}%
\pgfsetdash{}{0pt}%
\pgfpathmoveto{\pgfqpoint{1.011453in}{0.565123in}}%
\pgfpathlineto{\pgfqpoint{1.011453in}{2.350000in}}%
\pgfpathlineto{\pgfqpoint{2.212177in}{2.350000in}}%
\pgfpathlineto{\pgfqpoint{2.212177in}{0.565123in}}%
\pgfpathlineto{\pgfqpoint{1.011453in}{0.565123in}}%
\pgfpathlineto{\pgfqpoint{1.011453in}{0.565123in}}%
\pgfpathclose%
\pgfusepath{stroke,fill}%
\end{pgfscope}%
\begin{pgfscope}%
\pgfpathrectangle{\pgfqpoint{0.329012in}{0.565123in}}{\pgfqpoint{5.020988in}{1.784877in}}%
\pgfusepath{clip}%
\pgfsetbuttcap%
\pgfsetroundjoin%
\definecolor{currentfill}{rgb}{0.564706,0.933333,0.564706}%
\pgfsetfillcolor{currentfill}%
\pgfsetlinewidth{1.003750pt}%
\definecolor{currentstroke}{rgb}{0.564706,0.933333,0.564706}%
\pgfsetstrokecolor{currentstroke}%
\pgfsetdash{}{0pt}%
\pgfpathmoveto{\pgfqpoint{2.839914in}{0.565123in}}%
\pgfpathlineto{\pgfqpoint{2.839914in}{2.350000in}}%
\pgfpathlineto{\pgfqpoint{3.467432in}{2.350000in}}%
\pgfpathlineto{\pgfqpoint{3.467432in}{0.565123in}}%
\pgfpathlineto{\pgfqpoint{2.839914in}{0.565123in}}%
\pgfpathlineto{\pgfqpoint{2.839914in}{0.565123in}}%
\pgfpathclose%
\pgfusepath{stroke,fill}%
\end{pgfscope}%
\begin{pgfscope}%
\pgfpathrectangle{\pgfqpoint{0.329012in}{0.565123in}}{\pgfqpoint{5.020988in}{1.784877in}}%
\pgfusepath{clip}%
\pgfsetbuttcap%
\pgfsetroundjoin%
\definecolor{currentfill}{rgb}{0.564706,0.933333,0.564706}%
\pgfsetfillcolor{currentfill}%
\pgfsetlinewidth{1.003750pt}%
\definecolor{currentstroke}{rgb}{0.564706,0.933333,0.564706}%
\pgfsetstrokecolor{currentstroke}%
\pgfsetdash{}{0pt}%
\pgfpathmoveto{\pgfqpoint{4.094965in}{0.565123in}}%
\pgfpathlineto{\pgfqpoint{4.094965in}{2.350000in}}%
\pgfpathlineto{\pgfqpoint{4.727707in}{2.350000in}}%
\pgfpathlineto{\pgfqpoint{4.727707in}{0.565123in}}%
\pgfpathlineto{\pgfqpoint{4.094965in}{0.565123in}}%
\pgfpathlineto{\pgfqpoint{4.094965in}{0.565123in}}%
\pgfpathclose%
\pgfusepath{stroke,fill}%
\end{pgfscope}%
\begin{pgfscope}%
\pgfpathrectangle{\pgfqpoint{0.329012in}{0.565123in}}{\pgfqpoint{5.020988in}{1.784877in}}%
\pgfusepath{clip}%
\pgfsetbuttcap%
\pgfsetroundjoin%
\definecolor{currentfill}{rgb}{0.564706,0.933333,0.564706}%
\pgfsetfillcolor{currentfill}%
\pgfsetlinewidth{1.003750pt}%
\definecolor{currentstroke}{rgb}{0.564706,0.933333,0.564706}%
\pgfsetstrokecolor{currentstroke}%
\pgfsetdash{}{0pt}%
\pgfpathmoveto{\pgfqpoint{4.748650in}{0.565123in}}%
\pgfpathlineto{\pgfqpoint{4.748650in}{2.350000in}}%
\pgfpathlineto{\pgfqpoint{5.350000in}{2.350000in}}%
\pgfpathlineto{\pgfqpoint{5.350000in}{0.565123in}}%
\pgfpathlineto{\pgfqpoint{4.748650in}{0.565123in}}%
\pgfpathlineto{\pgfqpoint{4.748650in}{0.565123in}}%
\pgfpathclose%
\pgfusepath{stroke,fill}%
\end{pgfscope}%
\begin{pgfscope}%
\pgfpathrectangle{\pgfqpoint{0.329012in}{0.565123in}}{\pgfqpoint{5.020988in}{1.784877in}}%
\pgfusepath{clip}%
\pgfsetbuttcap%
\pgfsetroundjoin%
\definecolor{currentfill}{rgb}{1.000000,0.627451,0.478431}%
\pgfsetfillcolor{currentfill}%
\pgfsetlinewidth{1.003750pt}%
\definecolor{currentstroke}{rgb}{1.000000,0.627451,0.478431}%
\pgfsetstrokecolor{currentstroke}%
\pgfsetdash{}{0pt}%
\pgfpathmoveto{\pgfqpoint{0.961785in}{0.565123in}}%
\pgfpathlineto{\pgfqpoint{0.961785in}{2.350000in}}%
\pgfpathlineto{\pgfqpoint{1.011453in}{2.350000in}}%
\pgfpathlineto{\pgfqpoint{1.011453in}{0.565123in}}%
\pgfpathlineto{\pgfqpoint{0.961785in}{0.565123in}}%
\pgfpathlineto{\pgfqpoint{0.961785in}{0.565123in}}%
\pgfpathclose%
\pgfusepath{stroke,fill}%
\end{pgfscope}%
\begin{pgfscope}%
\pgfpathrectangle{\pgfqpoint{0.329012in}{0.565123in}}{\pgfqpoint{5.020988in}{1.784877in}}%
\pgfusepath{clip}%
\pgfsetbuttcap%
\pgfsetroundjoin%
\definecolor{currentfill}{rgb}{1.000000,0.627451,0.478431}%
\pgfsetfillcolor{currentfill}%
\pgfsetlinewidth{1.003750pt}%
\definecolor{currentstroke}{rgb}{1.000000,0.627451,0.478431}%
\pgfsetstrokecolor{currentstroke}%
\pgfsetdash{}{0pt}%
\pgfpathmoveto{\pgfqpoint{2.212177in}{0.565123in}}%
\pgfpathlineto{\pgfqpoint{2.212177in}{2.350000in}}%
\pgfpathlineto{\pgfqpoint{2.839914in}{2.350000in}}%
\pgfpathlineto{\pgfqpoint{2.839914in}{0.565123in}}%
\pgfpathlineto{\pgfqpoint{2.212177in}{0.565123in}}%
\pgfpathlineto{\pgfqpoint{2.212177in}{0.565123in}}%
\pgfpathclose%
\pgfusepath{stroke,fill}%
\end{pgfscope}%
\begin{pgfscope}%
\pgfpathrectangle{\pgfqpoint{0.329012in}{0.565123in}}{\pgfqpoint{5.020988in}{1.784877in}}%
\pgfusepath{clip}%
\pgfsetbuttcap%
\pgfsetroundjoin%
\definecolor{currentfill}{rgb}{1.000000,0.627451,0.478431}%
\pgfsetfillcolor{currentfill}%
\pgfsetlinewidth{1.003750pt}%
\definecolor{currentstroke}{rgb}{1.000000,0.627451,0.478431}%
\pgfsetstrokecolor{currentstroke}%
\pgfsetdash{}{0pt}%
\pgfpathmoveto{\pgfqpoint{3.467432in}{0.565123in}}%
\pgfpathlineto{\pgfqpoint{3.467432in}{2.350000in}}%
\pgfpathlineto{\pgfqpoint{4.094965in}{2.350000in}}%
\pgfpathlineto{\pgfqpoint{4.094965in}{0.565123in}}%
\pgfpathlineto{\pgfqpoint{3.467432in}{0.565123in}}%
\pgfpathlineto{\pgfqpoint{3.467432in}{0.565123in}}%
\pgfpathclose%
\pgfusepath{stroke,fill}%
\end{pgfscope}%
\begin{pgfscope}%
\pgfpathrectangle{\pgfqpoint{0.329012in}{0.565123in}}{\pgfqpoint{5.020988in}{1.784877in}}%
\pgfusepath{clip}%
\pgfsetbuttcap%
\pgfsetroundjoin%
\definecolor{currentfill}{rgb}{1.000000,0.627451,0.478431}%
\pgfsetfillcolor{currentfill}%
\pgfsetlinewidth{1.003750pt}%
\definecolor{currentstroke}{rgb}{1.000000,0.627451,0.478431}%
\pgfsetstrokecolor{currentstroke}%
\pgfsetdash{}{0pt}%
\pgfpathmoveto{\pgfqpoint{4.727707in}{0.565123in}}%
\pgfpathlineto{\pgfqpoint{4.727707in}{2.350000in}}%
\pgfpathlineto{\pgfqpoint{4.748650in}{2.350000in}}%
\pgfpathlineto{\pgfqpoint{4.748650in}{0.565123in}}%
\pgfpathlineto{\pgfqpoint{4.727707in}{0.565123in}}%
\pgfpathlineto{\pgfqpoint{4.727707in}{0.565123in}}%
\pgfpathclose%
\pgfusepath{stroke,fill}%
\end{pgfscope}%
\begin{pgfscope}%
\pgfsetbuttcap%
\pgfsetroundjoin%
\definecolor{currentfill}{rgb}{0.000000,0.000000,0.000000}%
\pgfsetfillcolor{currentfill}%
\pgfsetlinewidth{0.803000pt}%
\definecolor{currentstroke}{rgb}{0.000000,0.000000,0.000000}%
\pgfsetstrokecolor{currentstroke}%
\pgfsetdash{}{0pt}%
\pgfsys@defobject{currentmarker}{\pgfqpoint{0.000000in}{-0.048611in}}{\pgfqpoint{0.000000in}{0.000000in}}{%
\pgfpathmoveto{\pgfqpoint{0.000000in}{0.000000in}}%
\pgfpathlineto{\pgfqpoint{0.000000in}{-0.048611in}}%
\pgfusepath{stroke,fill}%
}%
\begin{pgfscope}%
\pgfsys@transformshift{0.329012in}{0.565123in}%
\pgfsys@useobject{currentmarker}{}%
\end{pgfscope}%
\end{pgfscope}%
\begin{pgfscope}%
\definecolor{textcolor}{rgb}{0.000000,0.000000,0.000000}%
\pgfsetstrokecolor{textcolor}%
\pgfsetfillcolor{textcolor}%
\pgftext[x=0.329012in,y=0.467901in,,top]{\color{textcolor}\rmfamily\fontsize{10.000000}{12.000000}\selectfont \(\displaystyle {0}\)}%
\end{pgfscope}%
\begin{pgfscope}%
\pgfsetbuttcap%
\pgfsetroundjoin%
\definecolor{currentfill}{rgb}{0.000000,0.000000,0.000000}%
\pgfsetfillcolor{currentfill}%
\pgfsetlinewidth{0.803000pt}%
\definecolor{currentstroke}{rgb}{0.000000,0.000000,0.000000}%
\pgfsetstrokecolor{currentstroke}%
\pgfsetdash{}{0pt}%
\pgfsys@defobject{currentmarker}{\pgfqpoint{0.000000in}{-0.048611in}}{\pgfqpoint{0.000000in}{0.000000in}}{%
\pgfpathmoveto{\pgfqpoint{0.000000in}{0.000000in}}%
\pgfpathlineto{\pgfqpoint{0.000000in}{-0.048611in}}%
\pgfusepath{stroke,fill}%
}%
\begin{pgfscope}%
\pgfsys@transformshift{1.113409in}{0.565123in}%
\pgfsys@useobject{currentmarker}{}%
\end{pgfscope}%
\end{pgfscope}%
\begin{pgfscope}%
\definecolor{textcolor}{rgb}{0.000000,0.000000,0.000000}%
\pgfsetstrokecolor{textcolor}%
\pgfsetfillcolor{textcolor}%
\pgftext[x=1.113409in,y=0.467901in,,top]{\color{textcolor}\rmfamily\fontsize{10.000000}{12.000000}\selectfont \(\displaystyle {50000}\)}%
\end{pgfscope}%
\begin{pgfscope}%
\pgfsetbuttcap%
\pgfsetroundjoin%
\definecolor{currentfill}{rgb}{0.000000,0.000000,0.000000}%
\pgfsetfillcolor{currentfill}%
\pgfsetlinewidth{0.803000pt}%
\definecolor{currentstroke}{rgb}{0.000000,0.000000,0.000000}%
\pgfsetstrokecolor{currentstroke}%
\pgfsetdash{}{0pt}%
\pgfsys@defobject{currentmarker}{\pgfqpoint{0.000000in}{-0.048611in}}{\pgfqpoint{0.000000in}{0.000000in}}{%
\pgfpathmoveto{\pgfqpoint{0.000000in}{0.000000in}}%
\pgfpathlineto{\pgfqpoint{0.000000in}{-0.048611in}}%
\pgfusepath{stroke,fill}%
}%
\begin{pgfscope}%
\pgfsys@transformshift{1.897806in}{0.565123in}%
\pgfsys@useobject{currentmarker}{}%
\end{pgfscope}%
\end{pgfscope}%
\begin{pgfscope}%
\definecolor{textcolor}{rgb}{0.000000,0.000000,0.000000}%
\pgfsetstrokecolor{textcolor}%
\pgfsetfillcolor{textcolor}%
\pgftext[x=1.897806in,y=0.467901in,,top]{\color{textcolor}\rmfamily\fontsize{10.000000}{12.000000}\selectfont \(\displaystyle {100000}\)}%
\end{pgfscope}%
\begin{pgfscope}%
\pgfsetbuttcap%
\pgfsetroundjoin%
\definecolor{currentfill}{rgb}{0.000000,0.000000,0.000000}%
\pgfsetfillcolor{currentfill}%
\pgfsetlinewidth{0.803000pt}%
\definecolor{currentstroke}{rgb}{0.000000,0.000000,0.000000}%
\pgfsetstrokecolor{currentstroke}%
\pgfsetdash{}{0pt}%
\pgfsys@defobject{currentmarker}{\pgfqpoint{0.000000in}{-0.048611in}}{\pgfqpoint{0.000000in}{0.000000in}}{%
\pgfpathmoveto{\pgfqpoint{0.000000in}{0.000000in}}%
\pgfpathlineto{\pgfqpoint{0.000000in}{-0.048611in}}%
\pgfusepath{stroke,fill}%
}%
\begin{pgfscope}%
\pgfsys@transformshift{2.682203in}{0.565123in}%
\pgfsys@useobject{currentmarker}{}%
\end{pgfscope}%
\end{pgfscope}%
\begin{pgfscope}%
\definecolor{textcolor}{rgb}{0.000000,0.000000,0.000000}%
\pgfsetstrokecolor{textcolor}%
\pgfsetfillcolor{textcolor}%
\pgftext[x=2.682203in,y=0.467901in,,top]{\color{textcolor}\rmfamily\fontsize{10.000000}{12.000000}\selectfont \(\displaystyle {150000}\)}%
\end{pgfscope}%
\begin{pgfscope}%
\pgfsetbuttcap%
\pgfsetroundjoin%
\definecolor{currentfill}{rgb}{0.000000,0.000000,0.000000}%
\pgfsetfillcolor{currentfill}%
\pgfsetlinewidth{0.803000pt}%
\definecolor{currentstroke}{rgb}{0.000000,0.000000,0.000000}%
\pgfsetstrokecolor{currentstroke}%
\pgfsetdash{}{0pt}%
\pgfsys@defobject{currentmarker}{\pgfqpoint{0.000000in}{-0.048611in}}{\pgfqpoint{0.000000in}{0.000000in}}{%
\pgfpathmoveto{\pgfqpoint{0.000000in}{0.000000in}}%
\pgfpathlineto{\pgfqpoint{0.000000in}{-0.048611in}}%
\pgfusepath{stroke,fill}%
}%
\begin{pgfscope}%
\pgfsys@transformshift{3.466600in}{0.565123in}%
\pgfsys@useobject{currentmarker}{}%
\end{pgfscope}%
\end{pgfscope}%
\begin{pgfscope}%
\definecolor{textcolor}{rgb}{0.000000,0.000000,0.000000}%
\pgfsetstrokecolor{textcolor}%
\pgfsetfillcolor{textcolor}%
\pgftext[x=3.466600in,y=0.467901in,,top]{\color{textcolor}\rmfamily\fontsize{10.000000}{12.000000}\selectfont \(\displaystyle {200000}\)}%
\end{pgfscope}%
\begin{pgfscope}%
\pgfsetbuttcap%
\pgfsetroundjoin%
\definecolor{currentfill}{rgb}{0.000000,0.000000,0.000000}%
\pgfsetfillcolor{currentfill}%
\pgfsetlinewidth{0.803000pt}%
\definecolor{currentstroke}{rgb}{0.000000,0.000000,0.000000}%
\pgfsetstrokecolor{currentstroke}%
\pgfsetdash{}{0pt}%
\pgfsys@defobject{currentmarker}{\pgfqpoint{0.000000in}{-0.048611in}}{\pgfqpoint{0.000000in}{0.000000in}}{%
\pgfpathmoveto{\pgfqpoint{0.000000in}{0.000000in}}%
\pgfpathlineto{\pgfqpoint{0.000000in}{-0.048611in}}%
\pgfusepath{stroke,fill}%
}%
\begin{pgfscope}%
\pgfsys@transformshift{4.250997in}{0.565123in}%
\pgfsys@useobject{currentmarker}{}%
\end{pgfscope}%
\end{pgfscope}%
\begin{pgfscope}%
\definecolor{textcolor}{rgb}{0.000000,0.000000,0.000000}%
\pgfsetstrokecolor{textcolor}%
\pgfsetfillcolor{textcolor}%
\pgftext[x=4.250997in,y=0.467901in,,top]{\color{textcolor}\rmfamily\fontsize{10.000000}{12.000000}\selectfont \(\displaystyle {250000}\)}%
\end{pgfscope}%
\begin{pgfscope}%
\pgfsetbuttcap%
\pgfsetroundjoin%
\definecolor{currentfill}{rgb}{0.000000,0.000000,0.000000}%
\pgfsetfillcolor{currentfill}%
\pgfsetlinewidth{0.803000pt}%
\definecolor{currentstroke}{rgb}{0.000000,0.000000,0.000000}%
\pgfsetstrokecolor{currentstroke}%
\pgfsetdash{}{0pt}%
\pgfsys@defobject{currentmarker}{\pgfqpoint{0.000000in}{-0.048611in}}{\pgfqpoint{0.000000in}{0.000000in}}{%
\pgfpathmoveto{\pgfqpoint{0.000000in}{0.000000in}}%
\pgfpathlineto{\pgfqpoint{0.000000in}{-0.048611in}}%
\pgfusepath{stroke,fill}%
}%
\begin{pgfscope}%
\pgfsys@transformshift{5.035394in}{0.565123in}%
\pgfsys@useobject{currentmarker}{}%
\end{pgfscope}%
\end{pgfscope}%
\begin{pgfscope}%
\definecolor{textcolor}{rgb}{0.000000,0.000000,0.000000}%
\pgfsetstrokecolor{textcolor}%
\pgfsetfillcolor{textcolor}%
\pgftext[x=5.035394in,y=0.467901in,,top]{\color{textcolor}\rmfamily\fontsize{10.000000}{12.000000}\selectfont \(\displaystyle {300000}\)}%
\end{pgfscope}%
\begin{pgfscope}%
\definecolor{textcolor}{rgb}{0.000000,0.000000,0.000000}%
\pgfsetstrokecolor{textcolor}%
\pgfsetfillcolor{textcolor}%
\pgftext[x=2.839506in,y=0.288889in,,top]{\color{textcolor}\rmfamily\fontsize{10.000000}{12.000000}\selectfont Time (s)}%
\end{pgfscope}%
\begin{pgfscope}%
\definecolor{textcolor}{rgb}{0.000000,0.000000,0.000000}%
\pgfsetstrokecolor{textcolor}%
\pgfsetfillcolor{textcolor}%
\pgftext[x=0.273457in,y=1.457562in,,bottom,rotate=90.000000]{\color{textcolor}\rmfamily\fontsize{10.000000}{12.000000}\selectfont Ground Truth}%
\end{pgfscope}%
\begin{pgfscope}%
\pgfpathrectangle{\pgfqpoint{0.329012in}{0.565123in}}{\pgfqpoint{5.020988in}{1.784877in}}%
\pgfusepath{clip}%
\pgfsetbuttcap%
\pgfsetroundjoin%
\pgfsetlinewidth{2.007500pt}%
\definecolor{currentstroke}{rgb}{1.000000,0.000000,0.000000}%
\pgfsetstrokecolor{currentstroke}%
\pgfsetdash{{7.400000pt}{3.200000pt}}{0.000000pt}%
\pgfpathmoveto{\pgfqpoint{0.961879in}{0.565123in}}%
\pgfpathlineto{\pgfqpoint{0.961879in}{2.350000in}}%
\pgfusepath{stroke}%
\end{pgfscope}%
\begin{pgfscope}%
\pgfpathrectangle{\pgfqpoint{0.329012in}{0.565123in}}{\pgfqpoint{5.020988in}{1.784877in}}%
\pgfusepath{clip}%
\pgfsetbuttcap%
\pgfsetroundjoin%
\pgfsetlinewidth{2.007500pt}%
\definecolor{currentstroke}{rgb}{1.000000,0.000000,0.000000}%
\pgfsetstrokecolor{currentstroke}%
\pgfsetdash{{7.400000pt}{3.200000pt}}{0.000000pt}%
\pgfpathmoveto{\pgfqpoint{1.047253in}{0.565123in}}%
\pgfpathlineto{\pgfqpoint{1.047253in}{2.350000in}}%
\pgfusepath{stroke}%
\end{pgfscope}%
\begin{pgfscope}%
\pgfpathrectangle{\pgfqpoint{0.329012in}{0.565123in}}{\pgfqpoint{5.020988in}{1.784877in}}%
\pgfusepath{clip}%
\pgfsetbuttcap%
\pgfsetroundjoin%
\pgfsetlinewidth{2.007500pt}%
\definecolor{currentstroke}{rgb}{1.000000,0.000000,0.000000}%
\pgfsetstrokecolor{currentstroke}%
\pgfsetdash{{7.400000pt}{3.200000pt}}{0.000000pt}%
\pgfpathmoveto{\pgfqpoint{2.212161in}{0.565123in}}%
\pgfpathlineto{\pgfqpoint{2.212161in}{2.350000in}}%
\pgfusepath{stroke}%
\end{pgfscope}%
\begin{pgfscope}%
\pgfpathrectangle{\pgfqpoint{0.329012in}{0.565123in}}{\pgfqpoint{5.020988in}{1.784877in}}%
\pgfusepath{clip}%
\pgfsetbuttcap%
\pgfsetroundjoin%
\pgfsetlinewidth{2.007500pt}%
\definecolor{currentstroke}{rgb}{1.000000,0.000000,0.000000}%
\pgfsetstrokecolor{currentstroke}%
\pgfsetdash{{7.400000pt}{3.200000pt}}{0.000000pt}%
\pgfpathmoveto{\pgfqpoint{2.839914in}{0.565123in}}%
\pgfpathlineto{\pgfqpoint{2.839914in}{2.350000in}}%
\pgfusepath{stroke}%
\end{pgfscope}%
\begin{pgfscope}%
\pgfpathrectangle{\pgfqpoint{0.329012in}{0.565123in}}{\pgfqpoint{5.020988in}{1.784877in}}%
\pgfusepath{clip}%
\pgfsetbuttcap%
\pgfsetroundjoin%
\pgfsetlinewidth{2.007500pt}%
\definecolor{currentstroke}{rgb}{1.000000,0.000000,0.000000}%
\pgfsetstrokecolor{currentstroke}%
\pgfsetdash{{7.400000pt}{3.200000pt}}{0.000000pt}%
\pgfpathmoveto{\pgfqpoint{3.468138in}{0.565123in}}%
\pgfpathlineto{\pgfqpoint{3.468138in}{2.350000in}}%
\pgfusepath{stroke}%
\end{pgfscope}%
\begin{pgfscope}%
\pgfpathrectangle{\pgfqpoint{0.329012in}{0.565123in}}{\pgfqpoint{5.020988in}{1.784877in}}%
\pgfusepath{clip}%
\pgfsetbuttcap%
\pgfsetroundjoin%
\pgfsetlinewidth{2.007500pt}%
\definecolor{currentstroke}{rgb}{1.000000,0.000000,0.000000}%
\pgfsetstrokecolor{currentstroke}%
\pgfsetdash{{7.400000pt}{3.200000pt}}{0.000000pt}%
\pgfpathmoveto{\pgfqpoint{4.094965in}{0.565123in}}%
\pgfpathlineto{\pgfqpoint{4.094965in}{2.350000in}}%
\pgfusepath{stroke}%
\end{pgfscope}%
\begin{pgfscope}%
\pgfpathrectangle{\pgfqpoint{0.329012in}{0.565123in}}{\pgfqpoint{5.020988in}{1.784877in}}%
\pgfusepath{clip}%
\pgfsetbuttcap%
\pgfsetroundjoin%
\pgfsetlinewidth{2.007500pt}%
\definecolor{currentstroke}{rgb}{1.000000,0.000000,0.000000}%
\pgfsetstrokecolor{currentstroke}%
\pgfsetdash{{7.400000pt}{3.200000pt}}{0.000000pt}%
\pgfpathmoveto{\pgfqpoint{4.727832in}{0.565123in}}%
\pgfpathlineto{\pgfqpoint{4.727832in}{2.350000in}}%
\pgfusepath{stroke}%
\end{pgfscope}%
\begin{pgfscope}%
\pgfpathrectangle{\pgfqpoint{0.329012in}{0.565123in}}{\pgfqpoint{5.020988in}{1.784877in}}%
\pgfusepath{clip}%
\pgfsetbuttcap%
\pgfsetroundjoin%
\pgfsetlinewidth{2.007500pt}%
\definecolor{currentstroke}{rgb}{1.000000,0.000000,0.000000}%
\pgfsetstrokecolor{currentstroke}%
\pgfsetdash{{7.400000pt}{3.200000pt}}{0.000000pt}%
\pgfpathmoveto{\pgfqpoint{4.806272in}{0.565123in}}%
\pgfpathlineto{\pgfqpoint{4.806272in}{2.350000in}}%
\pgfusepath{stroke}%
\end{pgfscope}%
\begin{pgfscope}%
\pgfsetrectcap%
\pgfsetmiterjoin%
\pgfsetlinewidth{0.803000pt}%
\definecolor{currentstroke}{rgb}{0.000000,0.000000,0.000000}%
\pgfsetstrokecolor{currentstroke}%
\pgfsetdash{}{0pt}%
\pgfpathmoveto{\pgfqpoint{0.329012in}{0.565123in}}%
\pgfpathlineto{\pgfqpoint{0.329012in}{2.350000in}}%
\pgfusepath{stroke}%
\end{pgfscope}%
\begin{pgfscope}%
\pgfsetrectcap%
\pgfsetmiterjoin%
\pgfsetlinewidth{0.803000pt}%
\definecolor{currentstroke}{rgb}{0.000000,0.000000,0.000000}%
\pgfsetstrokecolor{currentstroke}%
\pgfsetdash{}{0pt}%
\pgfpathmoveto{\pgfqpoint{5.350000in}{0.565123in}}%
\pgfpathlineto{\pgfqpoint{5.350000in}{2.350000in}}%
\pgfusepath{stroke}%
\end{pgfscope}%
\begin{pgfscope}%
\pgfsetrectcap%
\pgfsetmiterjoin%
\pgfsetlinewidth{0.803000pt}%
\definecolor{currentstroke}{rgb}{0.000000,0.000000,0.000000}%
\pgfsetstrokecolor{currentstroke}%
\pgfsetdash{}{0pt}%
\pgfpathmoveto{\pgfqpoint{0.329012in}{0.565123in}}%
\pgfpathlineto{\pgfqpoint{5.350000in}{0.565123in}}%
\pgfusepath{stroke}%
\end{pgfscope}%
\begin{pgfscope}%
\pgfsetrectcap%
\pgfsetmiterjoin%
\pgfsetlinewidth{0.803000pt}%
\definecolor{currentstroke}{rgb}{0.000000,0.000000,0.000000}%
\pgfsetstrokecolor{currentstroke}%
\pgfsetdash{}{0pt}%
\pgfpathmoveto{\pgfqpoint{0.329012in}{2.350000in}}%
\pgfpathlineto{\pgfqpoint{5.350000in}{2.350000in}}%
\pgfusepath{stroke}%
\end{pgfscope}%
\begin{pgfscope}%
\pgfsetbuttcap%
\pgfsetmiterjoin%
\definecolor{currentfill}{rgb}{1.000000,1.000000,1.000000}%
\pgfsetfillcolor{currentfill}%
\pgfsetfillopacity{0.800000}%
\pgfsetlinewidth{1.003750pt}%
\definecolor{currentstroke}{rgb}{0.800000,0.800000,0.800000}%
\pgfsetstrokecolor{currentstroke}%
\pgfsetstrokeopacity{0.800000}%
\pgfsetdash{}{0pt}%
\pgfpathmoveto{\pgfqpoint{4.252777in}{2.045216in}}%
\pgfpathlineto{\pgfqpoint{5.252778in}{2.045216in}}%
\pgfpathquadraticcurveto{\pgfqpoint{5.280556in}{2.045216in}}{\pgfqpoint{5.280556in}{2.072994in}}%
\pgfpathlineto{\pgfqpoint{5.280556in}{2.252778in}}%
\pgfpathquadraticcurveto{\pgfqpoint{5.280556in}{2.280556in}}{\pgfqpoint{5.252778in}{2.280556in}}%
\pgfpathlineto{\pgfqpoint{4.252777in}{2.280556in}}%
\pgfpathquadraticcurveto{\pgfqpoint{4.224999in}{2.280556in}}{\pgfqpoint{4.224999in}{2.252778in}}%
\pgfpathlineto{\pgfqpoint{4.224999in}{2.072994in}}%
\pgfpathquadraticcurveto{\pgfqpoint{4.224999in}{2.045216in}}{\pgfqpoint{4.252777in}{2.045216in}}%
\pgfpathlineto{\pgfqpoint{4.252777in}{2.045216in}}%
\pgfpathclose%
\pgfusepath{stroke,fill}%
\end{pgfscope}%
\begin{pgfscope}%
\pgfsetbuttcap%
\pgfsetroundjoin%
\pgfsetlinewidth{2.007500pt}%
\definecolor{currentstroke}{rgb}{1.000000,0.000000,0.000000}%
\pgfsetstrokecolor{currentstroke}%
\pgfsetdash{{7.400000pt}{3.200000pt}}{0.000000pt}%
\pgfpathmoveto{\pgfqpoint{4.280554in}{2.176389in}}%
\pgfpathlineto{\pgfqpoint{4.558332in}{2.176389in}}%
\pgfusepath{stroke}%
\end{pgfscope}%
\begin{pgfscope}%
\definecolor{textcolor}{rgb}{0.000000,0.000000,0.000000}%
\pgfsetstrokecolor{textcolor}%
\pgfsetfillcolor{textcolor}%
\pgftext[x=4.669443in,y=2.127778in,left,base]{\color{textcolor}\rmfamily\fontsize{10.000000}{12.000000}\selectfont detection}%
\end{pgfscope}%
\end{pgfpicture}%
\makeatother%
\endgroup%

    \caption{PEC Dataset Anomalies [Red Line = Anomaly]}
    \label{fig:pec_outliers}
\end{figure}

\subsection{Cyber Security BETH Dataset}
\label{ref_results_beth_sim}
The Cyber Security BETH dataset presented in Section \ref{ref_beth_dataset} is used to perform this simulation. Table \ref{tab:beth_sim_params} specifies the parameters used with the anomaly detector to obtain the results in this section. This dataset is the largest of all in this study, containing approximately 800,000 time steps (~9 days of signal data). This dataset outlines the most challenging of the test scenarios for the detector and its performance for large window sizes. The standard deviation multiplier is the default. If the data follows a standard Gaussian distribution, the detected points would fall outside 99.7\% of the data present in the window, which would represent a significant outlier comparative to the rest of the data. The rolling range multiplier is set to one below the standard deviation multiplier for this experiment which provides some recall of previous outliers, but provides more weight to the present outlier detection score. The attacks have been manually labelled and although there are different kinds of attacks, the baseline signal (userId) remains the same. The recent range detection debounce multiplier is two due to the nature and volatility of the data.

\begin{table}[H]
%%\centering
\caption{BETH Dataset Model Parameters}
\begin{tabular}{|l|c|l|}
    \hline
	\textbf{Parameter} & \textbf{Value} & \textbf{Description} \\ \hline
	m & 1000 & Window Size \\ \hline
	ts$\_$size & 10,000 & Time Series Size \\ \hline
	std$\_$dev & 3 & Standard Deviation Multiplier \\ \hline
	range & 2 & Rolling Range Multiplier\\ \hline
	recent & 2 & Recent Detection Debounce\\ \hline
\end{tabular}
\label{tab:beth_sim_params}
\end{table}

Figure \ref{fig:beth_mp_hist} shows the computed matrix profile values for this experiment. In this study, the outliers are only present when there is a significantly large range spike as shown near time step 200,000. The regular and periodic small peaks prior are not outliers and the rolling range technique ensures they do not trigger the detector. This provides an example of how the rolling range parameter can be used to tune the memory and ability of the detector. 

\begin{figure}[H]
    %%\centering
    %% Creator: Matplotlib, PGF backend
%%
%% To include the figure in your LaTeX document, write
%%   \input{<filename>.pgf}
%%
%% Make sure the required packages are loaded in your preamble
%%   \usepackage{pgf}
%%
%% Also ensure that all the required font packages are loaded; for instance,
%% the lmodern package is sometimes necessary when using math font.
%%   \usepackage{lmodern}
%%
%% Figures using additional raster images can only be included by \input if
%% they are in the same directory as the main LaTeX file. For loading figures
%% from other directories you can use the `import` package
%%   \usepackage{import}
%%
%% and then include the figures with
%%   \import{<path to file>}{<filename>.pgf}
%%
%% Matplotlib used the following preamble
%%
\begingroup%
\makeatletter%
\begin{pgfpicture}%
\pgfpathrectangle{\pgfpointorigin}{\pgfqpoint{6.000000in}{4.000000in}}%
\pgfusepath{use as bounding box, clip}%
\begin{pgfscope}%
\pgfsetbuttcap%
\pgfsetmiterjoin%
\pgfsetlinewidth{0.000000pt}%
\definecolor{currentstroke}{rgb}{1.000000,1.000000,1.000000}%
\pgfsetstrokecolor{currentstroke}%
\pgfsetstrokeopacity{0.000000}%
\pgfsetdash{}{0pt}%
\pgfpathmoveto{\pgfqpoint{0.000000in}{0.000000in}}%
\pgfpathlineto{\pgfqpoint{6.000000in}{0.000000in}}%
\pgfpathlineto{\pgfqpoint{6.000000in}{4.000000in}}%
\pgfpathlineto{\pgfqpoint{0.000000in}{4.000000in}}%
\pgfpathlineto{\pgfqpoint{0.000000in}{0.000000in}}%
\pgfpathclose%
\pgfusepath{}%
\end{pgfscope}%
\begin{pgfscope}%
\pgfsetbuttcap%
\pgfsetmiterjoin%
\definecolor{currentfill}{rgb}{1.000000,1.000000,1.000000}%
\pgfsetfillcolor{currentfill}%
\pgfsetlinewidth{0.000000pt}%
\definecolor{currentstroke}{rgb}{0.000000,0.000000,0.000000}%
\pgfsetstrokecolor{currentstroke}%
\pgfsetstrokeopacity{0.000000}%
\pgfsetdash{}{0pt}%
\pgfpathmoveto{\pgfqpoint{0.750000in}{0.500000in}}%
\pgfpathlineto{\pgfqpoint{5.400000in}{0.500000in}}%
\pgfpathlineto{\pgfqpoint{5.400000in}{3.520000in}}%
\pgfpathlineto{\pgfqpoint{0.750000in}{3.520000in}}%
\pgfpathlineto{\pgfqpoint{0.750000in}{0.500000in}}%
\pgfpathclose%
\pgfusepath{fill}%
\end{pgfscope}%
\begin{pgfscope}%
\pgfsetbuttcap%
\pgfsetroundjoin%
\definecolor{currentfill}{rgb}{0.000000,0.000000,0.000000}%
\pgfsetfillcolor{currentfill}%
\pgfsetlinewidth{0.803000pt}%
\definecolor{currentstroke}{rgb}{0.000000,0.000000,0.000000}%
\pgfsetstrokecolor{currentstroke}%
\pgfsetdash{}{0pt}%
\pgfsys@defobject{currentmarker}{\pgfqpoint{0.000000in}{-0.048611in}}{\pgfqpoint{0.000000in}{0.000000in}}{%
\pgfpathmoveto{\pgfqpoint{0.000000in}{0.000000in}}%
\pgfpathlineto{\pgfqpoint{0.000000in}{-0.048611in}}%
\pgfusepath{stroke,fill}%
}%
\begin{pgfscope}%
\pgfsys@transformshift{0.961364in}{0.500000in}%
\pgfsys@useobject{currentmarker}{}%
\end{pgfscope}%
\end{pgfscope}%
\begin{pgfscope}%
\definecolor{textcolor}{rgb}{0.000000,0.000000,0.000000}%
\pgfsetstrokecolor{textcolor}%
\pgfsetfillcolor{textcolor}%
\pgftext[x=0.961364in,y=0.402778in,,top]{\color{textcolor}\rmfamily\fontsize{10.000000}{12.000000}\selectfont \(\displaystyle {0}\)}%
\end{pgfscope}%
\begin{pgfscope}%
\pgfsetbuttcap%
\pgfsetroundjoin%
\definecolor{currentfill}{rgb}{0.000000,0.000000,0.000000}%
\pgfsetfillcolor{currentfill}%
\pgfsetlinewidth{0.803000pt}%
\definecolor{currentstroke}{rgb}{0.000000,0.000000,0.000000}%
\pgfsetstrokecolor{currentstroke}%
\pgfsetdash{}{0pt}%
\pgfsys@defobject{currentmarker}{\pgfqpoint{0.000000in}{-0.048611in}}{\pgfqpoint{0.000000in}{0.000000in}}{%
\pgfpathmoveto{\pgfqpoint{0.000000in}{0.000000in}}%
\pgfpathlineto{\pgfqpoint{0.000000in}{-0.048611in}}%
\pgfusepath{stroke,fill}%
}%
\begin{pgfscope}%
\pgfsys@transformshift{1.916495in}{0.500000in}%
\pgfsys@useobject{currentmarker}{}%
\end{pgfscope}%
\end{pgfscope}%
\begin{pgfscope}%
\definecolor{textcolor}{rgb}{0.000000,0.000000,0.000000}%
\pgfsetstrokecolor{textcolor}%
\pgfsetfillcolor{textcolor}%
\pgftext[x=1.916495in,y=0.402778in,,top]{\color{textcolor}\rmfamily\fontsize{10.000000}{12.000000}\selectfont \(\displaystyle {200000}\)}%
\end{pgfscope}%
\begin{pgfscope}%
\pgfsetbuttcap%
\pgfsetroundjoin%
\definecolor{currentfill}{rgb}{0.000000,0.000000,0.000000}%
\pgfsetfillcolor{currentfill}%
\pgfsetlinewidth{0.803000pt}%
\definecolor{currentstroke}{rgb}{0.000000,0.000000,0.000000}%
\pgfsetstrokecolor{currentstroke}%
\pgfsetdash{}{0pt}%
\pgfsys@defobject{currentmarker}{\pgfqpoint{0.000000in}{-0.048611in}}{\pgfqpoint{0.000000in}{0.000000in}}{%
\pgfpathmoveto{\pgfqpoint{0.000000in}{0.000000in}}%
\pgfpathlineto{\pgfqpoint{0.000000in}{-0.048611in}}%
\pgfusepath{stroke,fill}%
}%
\begin{pgfscope}%
\pgfsys@transformshift{2.871626in}{0.500000in}%
\pgfsys@useobject{currentmarker}{}%
\end{pgfscope}%
\end{pgfscope}%
\begin{pgfscope}%
\definecolor{textcolor}{rgb}{0.000000,0.000000,0.000000}%
\pgfsetstrokecolor{textcolor}%
\pgfsetfillcolor{textcolor}%
\pgftext[x=2.871626in,y=0.402778in,,top]{\color{textcolor}\rmfamily\fontsize{10.000000}{12.000000}\selectfont \(\displaystyle {400000}\)}%
\end{pgfscope}%
\begin{pgfscope}%
\pgfsetbuttcap%
\pgfsetroundjoin%
\definecolor{currentfill}{rgb}{0.000000,0.000000,0.000000}%
\pgfsetfillcolor{currentfill}%
\pgfsetlinewidth{0.803000pt}%
\definecolor{currentstroke}{rgb}{0.000000,0.000000,0.000000}%
\pgfsetstrokecolor{currentstroke}%
\pgfsetdash{}{0pt}%
\pgfsys@defobject{currentmarker}{\pgfqpoint{0.000000in}{-0.048611in}}{\pgfqpoint{0.000000in}{0.000000in}}{%
\pgfpathmoveto{\pgfqpoint{0.000000in}{0.000000in}}%
\pgfpathlineto{\pgfqpoint{0.000000in}{-0.048611in}}%
\pgfusepath{stroke,fill}%
}%
\begin{pgfscope}%
\pgfsys@transformshift{3.826758in}{0.500000in}%
\pgfsys@useobject{currentmarker}{}%
\end{pgfscope}%
\end{pgfscope}%
\begin{pgfscope}%
\definecolor{textcolor}{rgb}{0.000000,0.000000,0.000000}%
\pgfsetstrokecolor{textcolor}%
\pgfsetfillcolor{textcolor}%
\pgftext[x=3.826758in,y=0.402778in,,top]{\color{textcolor}\rmfamily\fontsize{10.000000}{12.000000}\selectfont \(\displaystyle {600000}\)}%
\end{pgfscope}%
\begin{pgfscope}%
\pgfsetbuttcap%
\pgfsetroundjoin%
\definecolor{currentfill}{rgb}{0.000000,0.000000,0.000000}%
\pgfsetfillcolor{currentfill}%
\pgfsetlinewidth{0.803000pt}%
\definecolor{currentstroke}{rgb}{0.000000,0.000000,0.000000}%
\pgfsetstrokecolor{currentstroke}%
\pgfsetdash{}{0pt}%
\pgfsys@defobject{currentmarker}{\pgfqpoint{0.000000in}{-0.048611in}}{\pgfqpoint{0.000000in}{0.000000in}}{%
\pgfpathmoveto{\pgfqpoint{0.000000in}{0.000000in}}%
\pgfpathlineto{\pgfqpoint{0.000000in}{-0.048611in}}%
\pgfusepath{stroke,fill}%
}%
\begin{pgfscope}%
\pgfsys@transformshift{4.781889in}{0.500000in}%
\pgfsys@useobject{currentmarker}{}%
\end{pgfscope}%
\end{pgfscope}%
\begin{pgfscope}%
\definecolor{textcolor}{rgb}{0.000000,0.000000,0.000000}%
\pgfsetstrokecolor{textcolor}%
\pgfsetfillcolor{textcolor}%
\pgftext[x=4.781889in,y=0.402778in,,top]{\color{textcolor}\rmfamily\fontsize{10.000000}{12.000000}\selectfont \(\displaystyle {800000}\)}%
\end{pgfscope}%
\begin{pgfscope}%
\definecolor{textcolor}{rgb}{0.000000,0.000000,0.000000}%
\pgfsetstrokecolor{textcolor}%
\pgfsetfillcolor{textcolor}%
\pgftext[x=3.075000in,y=0.223766in,,top]{\color{textcolor}\rmfamily\fontsize{10.000000}{12.000000}\selectfont time}%
\end{pgfscope}%
\begin{pgfscope}%
\pgfsetbuttcap%
\pgfsetroundjoin%
\definecolor{currentfill}{rgb}{0.000000,0.000000,0.000000}%
\pgfsetfillcolor{currentfill}%
\pgfsetlinewidth{0.803000pt}%
\definecolor{currentstroke}{rgb}{0.000000,0.000000,0.000000}%
\pgfsetstrokecolor{currentstroke}%
\pgfsetdash{}{0pt}%
\pgfsys@defobject{currentmarker}{\pgfqpoint{-0.048611in}{0.000000in}}{\pgfqpoint{-0.000000in}{0.000000in}}{%
\pgfpathmoveto{\pgfqpoint{-0.000000in}{0.000000in}}%
\pgfpathlineto{\pgfqpoint{-0.048611in}{0.000000in}}%
\pgfusepath{stroke,fill}%
}%
\begin{pgfscope}%
\pgfsys@transformshift{0.750000in}{0.637273in}%
\pgfsys@useobject{currentmarker}{}%
\end{pgfscope}%
\end{pgfscope}%
\begin{pgfscope}%
\definecolor{textcolor}{rgb}{0.000000,0.000000,0.000000}%
\pgfsetstrokecolor{textcolor}%
\pgfsetfillcolor{textcolor}%
\pgftext[x=0.583333in, y=0.589047in, left, base]{\color{textcolor}\rmfamily\fontsize{10.000000}{12.000000}\selectfont \(\displaystyle {0}\)}%
\end{pgfscope}%
\begin{pgfscope}%
\pgfsetbuttcap%
\pgfsetroundjoin%
\definecolor{currentfill}{rgb}{0.000000,0.000000,0.000000}%
\pgfsetfillcolor{currentfill}%
\pgfsetlinewidth{0.803000pt}%
\definecolor{currentstroke}{rgb}{0.000000,0.000000,0.000000}%
\pgfsetstrokecolor{currentstroke}%
\pgfsetdash{}{0pt}%
\pgfsys@defobject{currentmarker}{\pgfqpoint{-0.048611in}{0.000000in}}{\pgfqpoint{-0.000000in}{0.000000in}}{%
\pgfpathmoveto{\pgfqpoint{-0.000000in}{0.000000in}}%
\pgfpathlineto{\pgfqpoint{-0.048611in}{0.000000in}}%
\pgfusepath{stroke,fill}%
}%
\begin{pgfscope}%
\pgfsys@transformshift{0.750000in}{0.996115in}%
\pgfsys@useobject{currentmarker}{}%
\end{pgfscope}%
\end{pgfscope}%
\begin{pgfscope}%
\definecolor{textcolor}{rgb}{0.000000,0.000000,0.000000}%
\pgfsetstrokecolor{textcolor}%
\pgfsetfillcolor{textcolor}%
\pgftext[x=0.374999in, y=0.947890in, left, base]{\color{textcolor}\rmfamily\fontsize{10.000000}{12.000000}\selectfont \(\displaystyle {2500}\)}%
\end{pgfscope}%
\begin{pgfscope}%
\pgfsetbuttcap%
\pgfsetroundjoin%
\definecolor{currentfill}{rgb}{0.000000,0.000000,0.000000}%
\pgfsetfillcolor{currentfill}%
\pgfsetlinewidth{0.803000pt}%
\definecolor{currentstroke}{rgb}{0.000000,0.000000,0.000000}%
\pgfsetstrokecolor{currentstroke}%
\pgfsetdash{}{0pt}%
\pgfsys@defobject{currentmarker}{\pgfqpoint{-0.048611in}{0.000000in}}{\pgfqpoint{-0.000000in}{0.000000in}}{%
\pgfpathmoveto{\pgfqpoint{-0.000000in}{0.000000in}}%
\pgfpathlineto{\pgfqpoint{-0.048611in}{0.000000in}}%
\pgfusepath{stroke,fill}%
}%
\begin{pgfscope}%
\pgfsys@transformshift{0.750000in}{1.354958in}%
\pgfsys@useobject{currentmarker}{}%
\end{pgfscope}%
\end{pgfscope}%
\begin{pgfscope}%
\definecolor{textcolor}{rgb}{0.000000,0.000000,0.000000}%
\pgfsetstrokecolor{textcolor}%
\pgfsetfillcolor{textcolor}%
\pgftext[x=0.374999in, y=1.306733in, left, base]{\color{textcolor}\rmfamily\fontsize{10.000000}{12.000000}\selectfont \(\displaystyle {5000}\)}%
\end{pgfscope}%
\begin{pgfscope}%
\pgfsetbuttcap%
\pgfsetroundjoin%
\definecolor{currentfill}{rgb}{0.000000,0.000000,0.000000}%
\pgfsetfillcolor{currentfill}%
\pgfsetlinewidth{0.803000pt}%
\definecolor{currentstroke}{rgb}{0.000000,0.000000,0.000000}%
\pgfsetstrokecolor{currentstroke}%
\pgfsetdash{}{0pt}%
\pgfsys@defobject{currentmarker}{\pgfqpoint{-0.048611in}{0.000000in}}{\pgfqpoint{-0.000000in}{0.000000in}}{%
\pgfpathmoveto{\pgfqpoint{-0.000000in}{0.000000in}}%
\pgfpathlineto{\pgfqpoint{-0.048611in}{0.000000in}}%
\pgfusepath{stroke,fill}%
}%
\begin{pgfscope}%
\pgfsys@transformshift{0.750000in}{1.713801in}%
\pgfsys@useobject{currentmarker}{}%
\end{pgfscope}%
\end{pgfscope}%
\begin{pgfscope}%
\definecolor{textcolor}{rgb}{0.000000,0.000000,0.000000}%
\pgfsetstrokecolor{textcolor}%
\pgfsetfillcolor{textcolor}%
\pgftext[x=0.374999in, y=1.665576in, left, base]{\color{textcolor}\rmfamily\fontsize{10.000000}{12.000000}\selectfont \(\displaystyle {7500}\)}%
\end{pgfscope}%
\begin{pgfscope}%
\pgfsetbuttcap%
\pgfsetroundjoin%
\definecolor{currentfill}{rgb}{0.000000,0.000000,0.000000}%
\pgfsetfillcolor{currentfill}%
\pgfsetlinewidth{0.803000pt}%
\definecolor{currentstroke}{rgb}{0.000000,0.000000,0.000000}%
\pgfsetstrokecolor{currentstroke}%
\pgfsetdash{}{0pt}%
\pgfsys@defobject{currentmarker}{\pgfqpoint{-0.048611in}{0.000000in}}{\pgfqpoint{-0.000000in}{0.000000in}}{%
\pgfpathmoveto{\pgfqpoint{-0.000000in}{0.000000in}}%
\pgfpathlineto{\pgfqpoint{-0.048611in}{0.000000in}}%
\pgfusepath{stroke,fill}%
}%
\begin{pgfscope}%
\pgfsys@transformshift{0.750000in}{2.072644in}%
\pgfsys@useobject{currentmarker}{}%
\end{pgfscope}%
\end{pgfscope}%
\begin{pgfscope}%
\definecolor{textcolor}{rgb}{0.000000,0.000000,0.000000}%
\pgfsetstrokecolor{textcolor}%
\pgfsetfillcolor{textcolor}%
\pgftext[x=0.305554in, y=2.024418in, left, base]{\color{textcolor}\rmfamily\fontsize{10.000000}{12.000000}\selectfont \(\displaystyle {10000}\)}%
\end{pgfscope}%
\begin{pgfscope}%
\pgfsetbuttcap%
\pgfsetroundjoin%
\definecolor{currentfill}{rgb}{0.000000,0.000000,0.000000}%
\pgfsetfillcolor{currentfill}%
\pgfsetlinewidth{0.803000pt}%
\definecolor{currentstroke}{rgb}{0.000000,0.000000,0.000000}%
\pgfsetstrokecolor{currentstroke}%
\pgfsetdash{}{0pt}%
\pgfsys@defobject{currentmarker}{\pgfqpoint{-0.048611in}{0.000000in}}{\pgfqpoint{-0.000000in}{0.000000in}}{%
\pgfpathmoveto{\pgfqpoint{-0.000000in}{0.000000in}}%
\pgfpathlineto{\pgfqpoint{-0.048611in}{0.000000in}}%
\pgfusepath{stroke,fill}%
}%
\begin{pgfscope}%
\pgfsys@transformshift{0.750000in}{2.431486in}%
\pgfsys@useobject{currentmarker}{}%
\end{pgfscope}%
\end{pgfscope}%
\begin{pgfscope}%
\definecolor{textcolor}{rgb}{0.000000,0.000000,0.000000}%
\pgfsetstrokecolor{textcolor}%
\pgfsetfillcolor{textcolor}%
\pgftext[x=0.305554in, y=2.383261in, left, base]{\color{textcolor}\rmfamily\fontsize{10.000000}{12.000000}\selectfont \(\displaystyle {12500}\)}%
\end{pgfscope}%
\begin{pgfscope}%
\pgfsetbuttcap%
\pgfsetroundjoin%
\definecolor{currentfill}{rgb}{0.000000,0.000000,0.000000}%
\pgfsetfillcolor{currentfill}%
\pgfsetlinewidth{0.803000pt}%
\definecolor{currentstroke}{rgb}{0.000000,0.000000,0.000000}%
\pgfsetstrokecolor{currentstroke}%
\pgfsetdash{}{0pt}%
\pgfsys@defobject{currentmarker}{\pgfqpoint{-0.048611in}{0.000000in}}{\pgfqpoint{-0.000000in}{0.000000in}}{%
\pgfpathmoveto{\pgfqpoint{-0.000000in}{0.000000in}}%
\pgfpathlineto{\pgfqpoint{-0.048611in}{0.000000in}}%
\pgfusepath{stroke,fill}%
}%
\begin{pgfscope}%
\pgfsys@transformshift{0.750000in}{2.790329in}%
\pgfsys@useobject{currentmarker}{}%
\end{pgfscope}%
\end{pgfscope}%
\begin{pgfscope}%
\definecolor{textcolor}{rgb}{0.000000,0.000000,0.000000}%
\pgfsetstrokecolor{textcolor}%
\pgfsetfillcolor{textcolor}%
\pgftext[x=0.305554in, y=2.742104in, left, base]{\color{textcolor}\rmfamily\fontsize{10.000000}{12.000000}\selectfont \(\displaystyle {15000}\)}%
\end{pgfscope}%
\begin{pgfscope}%
\pgfsetbuttcap%
\pgfsetroundjoin%
\definecolor{currentfill}{rgb}{0.000000,0.000000,0.000000}%
\pgfsetfillcolor{currentfill}%
\pgfsetlinewidth{0.803000pt}%
\definecolor{currentstroke}{rgb}{0.000000,0.000000,0.000000}%
\pgfsetstrokecolor{currentstroke}%
\pgfsetdash{}{0pt}%
\pgfsys@defobject{currentmarker}{\pgfqpoint{-0.048611in}{0.000000in}}{\pgfqpoint{-0.000000in}{0.000000in}}{%
\pgfpathmoveto{\pgfqpoint{-0.000000in}{0.000000in}}%
\pgfpathlineto{\pgfqpoint{-0.048611in}{0.000000in}}%
\pgfusepath{stroke,fill}%
}%
\begin{pgfscope}%
\pgfsys@transformshift{0.750000in}{3.149172in}%
\pgfsys@useobject{currentmarker}{}%
\end{pgfscope}%
\end{pgfscope}%
\begin{pgfscope}%
\definecolor{textcolor}{rgb}{0.000000,0.000000,0.000000}%
\pgfsetstrokecolor{textcolor}%
\pgfsetfillcolor{textcolor}%
\pgftext[x=0.305554in, y=3.100946in, left, base]{\color{textcolor}\rmfamily\fontsize{10.000000}{12.000000}\selectfont \(\displaystyle {17500}\)}%
\end{pgfscope}%
\begin{pgfscope}%
\pgfsetbuttcap%
\pgfsetroundjoin%
\definecolor{currentfill}{rgb}{0.000000,0.000000,0.000000}%
\pgfsetfillcolor{currentfill}%
\pgfsetlinewidth{0.803000pt}%
\definecolor{currentstroke}{rgb}{0.000000,0.000000,0.000000}%
\pgfsetstrokecolor{currentstroke}%
\pgfsetdash{}{0pt}%
\pgfsys@defobject{currentmarker}{\pgfqpoint{-0.048611in}{0.000000in}}{\pgfqpoint{-0.000000in}{0.000000in}}{%
\pgfpathmoveto{\pgfqpoint{-0.000000in}{0.000000in}}%
\pgfpathlineto{\pgfqpoint{-0.048611in}{0.000000in}}%
\pgfusepath{stroke,fill}%
}%
\begin{pgfscope}%
\pgfsys@transformshift{0.750000in}{3.508014in}%
\pgfsys@useobject{currentmarker}{}%
\end{pgfscope}%
\end{pgfscope}%
\begin{pgfscope}%
\definecolor{textcolor}{rgb}{0.000000,0.000000,0.000000}%
\pgfsetstrokecolor{textcolor}%
\pgfsetfillcolor{textcolor}%
\pgftext[x=0.305554in, y=3.459789in, left, base]{\color{textcolor}\rmfamily\fontsize{10.000000}{12.000000}\selectfont \(\displaystyle {20000}\)}%
\end{pgfscope}%
\begin{pgfscope}%
\pgfpathrectangle{\pgfqpoint{0.750000in}{0.500000in}}{\pgfqpoint{4.650000in}{3.020000in}}%
\pgfusepath{clip}%
\pgfsetrectcap%
\pgfsetroundjoin%
\pgfsetlinewidth{1.505625pt}%
\definecolor{currentstroke}{rgb}{0.000000,0.000000,1.000000}%
\pgfsetstrokecolor{currentstroke}%
\pgfsetdash{}{0pt}%
\pgfpathmoveto{\pgfqpoint{0.961364in}{3.136814in}}%
\pgfpathlineto{\pgfqpoint{0.965643in}{3.136726in}}%
\pgfpathlineto{\pgfqpoint{0.968131in}{1.476377in}}%
\pgfpathlineto{\pgfqpoint{0.969969in}{0.703583in}}%
\pgfpathlineto{\pgfqpoint{0.992148in}{0.703552in}}%
\pgfpathlineto{\pgfqpoint{0.993814in}{0.662468in}}%
\pgfpathlineto{\pgfqpoint{1.004841in}{0.662468in}}%
\pgfpathlineto{\pgfqpoint{1.006389in}{0.657572in}}%
\pgfpathlineto{\pgfqpoint{1.016551in}{0.657572in}}%
\pgfpathlineto{\pgfqpoint{1.018137in}{0.700367in}}%
\pgfpathlineto{\pgfqpoint{1.064160in}{0.700367in}}%
\pgfpathlineto{\pgfqpoint{1.064704in}{0.657675in}}%
\pgfpathlineto{\pgfqpoint{1.065764in}{0.678249in}}%
\pgfpathlineto{\pgfqpoint{1.067340in}{0.678249in}}%
\pgfpathlineto{\pgfqpoint{1.068921in}{0.703651in}}%
\pgfpathlineto{\pgfqpoint{1.112074in}{0.703651in}}%
\pgfpathlineto{\pgfqpoint{1.113621in}{0.700400in}}%
\pgfpathlineto{\pgfqpoint{1.115407in}{0.701593in}}%
\pgfpathlineto{\pgfqpoint{1.117356in}{0.836741in}}%
\pgfpathlineto{\pgfqpoint{1.149066in}{0.835290in}}%
\pgfpathlineto{\pgfqpoint{1.150647in}{0.830837in}}%
\pgfpathlineto{\pgfqpoint{1.162958in}{0.829770in}}%
\pgfpathlineto{\pgfqpoint{1.165461in}{0.721757in}}%
\pgfpathlineto{\pgfqpoint{1.192028in}{0.720528in}}%
\pgfpathlineto{\pgfqpoint{1.193776in}{0.666053in}}%
\pgfpathlineto{\pgfqpoint{1.237621in}{0.666053in}}%
\pgfpathlineto{\pgfqpoint{1.239173in}{0.657775in}}%
\pgfpathlineto{\pgfqpoint{1.251523in}{0.657775in}}%
\pgfpathlineto{\pgfqpoint{1.253113in}{0.703583in}}%
\pgfpathlineto{\pgfqpoint{1.280053in}{0.704751in}}%
\pgfpathlineto{\pgfqpoint{1.282016in}{0.837771in}}%
\pgfpathlineto{\pgfqpoint{1.327556in}{0.836720in}}%
\pgfpathlineto{\pgfqpoint{1.330751in}{0.657775in}}%
\pgfpathlineto{\pgfqpoint{1.338507in}{0.657775in}}%
\pgfpathlineto{\pgfqpoint{1.340054in}{0.666053in}}%
\pgfpathlineto{\pgfqpoint{1.366459in}{0.666125in}}%
\pgfpathlineto{\pgfqpoint{1.368049in}{0.700400in}}%
\pgfpathlineto{\pgfqpoint{1.413752in}{0.700400in}}%
\pgfpathlineto{\pgfqpoint{1.415414in}{0.657775in}}%
\pgfpathlineto{\pgfqpoint{1.444779in}{0.657775in}}%
\pgfpathlineto{\pgfqpoint{1.446527in}{0.847513in}}%
\pgfpathlineto{\pgfqpoint{1.484632in}{0.846028in}}%
\pgfpathlineto{\pgfqpoint{1.486261in}{0.836751in}}%
\pgfpathlineto{\pgfqpoint{1.492460in}{0.835716in}}%
\pgfpathlineto{\pgfqpoint{1.495191in}{0.703583in}}%
\pgfpathlineto{\pgfqpoint{1.528478in}{0.703583in}}%
\pgfpathlineto{\pgfqpoint{1.530130in}{0.666053in}}%
\pgfpathlineto{\pgfqpoint{1.532035in}{0.666053in}}%
\pgfpathlineto{\pgfqpoint{1.533588in}{0.657573in}}%
\pgfpathlineto{\pgfqpoint{1.606555in}{0.657775in}}%
\pgfpathlineto{\pgfqpoint{1.608169in}{0.683412in}}%
\pgfpathlineto{\pgfqpoint{1.618045in}{0.683050in}}%
\pgfpathlineto{\pgfqpoint{1.619941in}{0.842994in}}%
\pgfpathlineto{\pgfqpoint{1.665730in}{0.841486in}}%
\pgfpathlineto{\pgfqpoint{1.668514in}{0.710184in}}%
\pgfpathlineto{\pgfqpoint{1.697994in}{0.710155in}}%
\pgfpathlineto{\pgfqpoint{1.699585in}{0.692992in}}%
\pgfpathlineto{\pgfqpoint{1.708635in}{0.692992in}}%
\pgfpathlineto{\pgfqpoint{1.710230in}{0.667300in}}%
\pgfpathlineto{\pgfqpoint{1.710416in}{0.667300in}}%
\pgfpathlineto{\pgfqpoint{1.711968in}{0.657775in}}%
\pgfpathlineto{\pgfqpoint{1.714891in}{0.657776in}}%
\pgfpathlineto{\pgfqpoint{1.716471in}{0.696978in}}%
\pgfpathlineto{\pgfqpoint{1.762628in}{0.696978in}}%
\pgfpathlineto{\pgfqpoint{1.764276in}{0.657775in}}%
\pgfpathlineto{\pgfqpoint{1.793235in}{0.657573in}}%
\pgfpathlineto{\pgfqpoint{1.794974in}{0.847513in}}%
\pgfpathlineto{\pgfqpoint{1.840935in}{0.846018in}}%
\pgfpathlineto{\pgfqpoint{1.844215in}{0.665980in}}%
\pgfpathlineto{\pgfqpoint{1.848919in}{0.665980in}}%
\pgfpathlineto{\pgfqpoint{1.850472in}{0.657775in}}%
\pgfpathlineto{\pgfqpoint{1.890143in}{0.657775in}}%
\pgfpathlineto{\pgfqpoint{1.891733in}{0.675203in}}%
\pgfpathlineto{\pgfqpoint{1.893710in}{0.675203in}}%
\pgfpathlineto{\pgfqpoint{1.895267in}{0.684209in}}%
\pgfpathlineto{\pgfqpoint{1.921404in}{0.684209in}}%
\pgfpathlineto{\pgfqpoint{1.923128in}{0.782640in}}%
\pgfpathlineto{\pgfqpoint{1.931214in}{0.781873in}}%
\pgfpathlineto{\pgfqpoint{1.933902in}{2.913650in}}%
\pgfpathlineto{\pgfqpoint{1.934017in}{2.913602in}}%
\pgfpathlineto{\pgfqpoint{1.952608in}{2.913602in}}%
\pgfpathlineto{\pgfqpoint{1.954672in}{3.382727in}}%
\pgfpathlineto{\pgfqpoint{1.955149in}{3.382727in}}%
\pgfpathlineto{\pgfqpoint{1.956945in}{3.220671in}}%
\pgfpathlineto{\pgfqpoint{1.995948in}{3.220671in}}%
\pgfpathlineto{\pgfqpoint{1.999014in}{1.637691in}}%
\pgfpathlineto{\pgfqpoint{2.001015in}{0.637273in}}%
\pgfpathlineto{\pgfqpoint{2.014620in}{0.637273in}}%
\pgfpathlineto{\pgfqpoint{2.016172in}{0.991692in}}%
\pgfpathlineto{\pgfqpoint{2.017161in}{0.993071in}}%
\pgfpathlineto{\pgfqpoint{2.019148in}{1.025090in}}%
\pgfpathlineto{\pgfqpoint{2.022997in}{1.025090in}}%
\pgfpathlineto{\pgfqpoint{2.024544in}{0.994885in}}%
\pgfpathlineto{\pgfqpoint{2.044139in}{0.994885in}}%
\pgfpathlineto{\pgfqpoint{2.045696in}{0.897707in}}%
\pgfpathlineto{\pgfqpoint{2.046340in}{0.897707in}}%
\pgfpathlineto{\pgfqpoint{2.048198in}{2.879490in}}%
\pgfpathlineto{\pgfqpoint{2.092979in}{2.879927in}}%
\pgfpathlineto{\pgfqpoint{2.096781in}{0.700400in}}%
\pgfpathlineto{\pgfqpoint{2.113257in}{0.701690in}}%
\pgfpathlineto{\pgfqpoint{2.115210in}{0.837802in}}%
\pgfpathlineto{\pgfqpoint{2.160808in}{0.836772in}}%
\pgfpathlineto{\pgfqpoint{2.163893in}{0.665980in}}%
\pgfpathlineto{\pgfqpoint{2.183999in}{0.665980in}}%
\pgfpathlineto{\pgfqpoint{2.185551in}{0.657572in}}%
\pgfpathlineto{\pgfqpoint{2.210499in}{0.657572in}}%
\pgfpathlineto{\pgfqpoint{2.212046in}{0.665980in}}%
\pgfpathlineto{\pgfqpoint{2.217251in}{0.665980in}}%
\pgfpathlineto{\pgfqpoint{2.218842in}{0.687432in}}%
\pgfpathlineto{\pgfqpoint{2.264922in}{0.687432in}}%
\pgfpathlineto{\pgfqpoint{2.266531in}{0.657573in}}%
\pgfpathlineto{\pgfqpoint{2.277979in}{0.657573in}}%
\pgfpathlineto{\pgfqpoint{2.279536in}{0.637273in}}%
\pgfpathlineto{\pgfqpoint{2.284120in}{0.637273in}}%
\pgfpathlineto{\pgfqpoint{2.285730in}{0.847513in}}%
\pgfpathlineto{\pgfqpoint{2.317259in}{0.846018in}}%
\pgfpathlineto{\pgfqpoint{2.318868in}{0.838837in}}%
\pgfpathlineto{\pgfqpoint{2.331920in}{0.837791in}}%
\pgfpathlineto{\pgfqpoint{2.334766in}{0.697047in}}%
\pgfpathlineto{\pgfqpoint{2.360908in}{0.697046in}}%
\pgfpathlineto{\pgfqpoint{2.362556in}{0.657775in}}%
\pgfpathlineto{\pgfqpoint{2.370923in}{0.657775in}}%
\pgfpathlineto{\pgfqpoint{2.372475in}{0.666124in}}%
\pgfpathlineto{\pgfqpoint{2.377761in}{0.666124in}}%
\pgfpathlineto{\pgfqpoint{2.379332in}{0.683001in}}%
\pgfpathlineto{\pgfqpoint{2.396802in}{0.684209in}}%
\pgfpathlineto{\pgfqpoint{2.398349in}{0.686748in}}%
\pgfpathlineto{\pgfqpoint{2.411267in}{0.686748in}}%
\pgfpathlineto{\pgfqpoint{2.412815in}{0.684209in}}%
\pgfpathlineto{\pgfqpoint{2.428441in}{0.684209in}}%
\pgfpathlineto{\pgfqpoint{2.429993in}{0.688492in}}%
\pgfpathlineto{\pgfqpoint{2.460901in}{0.688492in}}%
\pgfpathlineto{\pgfqpoint{2.462448in}{0.684209in}}%
\pgfpathlineto{\pgfqpoint{2.475242in}{0.683376in}}%
\pgfpathlineto{\pgfqpoint{2.476789in}{0.681525in}}%
\pgfpathlineto{\pgfqpoint{2.488270in}{0.681525in}}%
\pgfpathlineto{\pgfqpoint{2.489832in}{0.690828in}}%
\pgfpathlineto{\pgfqpoint{2.532884in}{0.690751in}}%
\pgfpathlineto{\pgfqpoint{2.534431in}{0.688411in}}%
\pgfpathlineto{\pgfqpoint{2.551681in}{0.688411in}}%
\pgfpathlineto{\pgfqpoint{2.553233in}{0.681525in}}%
\pgfpathlineto{\pgfqpoint{2.599037in}{0.681525in}}%
\pgfpathlineto{\pgfqpoint{2.600598in}{0.668564in}}%
\pgfpathlineto{\pgfqpoint{2.613072in}{0.668564in}}%
\pgfpathlineto{\pgfqpoint{2.614624in}{0.659399in}}%
\pgfpathlineto{\pgfqpoint{2.617432in}{0.659399in}}%
\pgfpathlineto{\pgfqpoint{2.618980in}{0.657572in}}%
\pgfpathlineto{\pgfqpoint{2.624839in}{0.657572in}}%
\pgfpathlineto{\pgfqpoint{2.626530in}{0.715548in}}%
\pgfpathlineto{\pgfqpoint{2.640222in}{0.715548in}}%
\pgfpathlineto{\pgfqpoint{2.641788in}{0.707599in}}%
\pgfpathlineto{\pgfqpoint{2.645322in}{0.707599in}}%
\pgfpathlineto{\pgfqpoint{2.646870in}{0.705525in}}%
\pgfpathlineto{\pgfqpoint{2.670991in}{0.705525in}}%
\pgfpathlineto{\pgfqpoint{2.672539in}{0.703707in}}%
\pgfpathlineto{\pgfqpoint{2.672596in}{0.702168in}}%
\pgfpathlineto{\pgfqpoint{2.674182in}{0.684209in}}%
\pgfpathlineto{\pgfqpoint{2.682090in}{0.684209in}}%
\pgfpathlineto{\pgfqpoint{2.683637in}{0.681525in}}%
\pgfpathlineto{\pgfqpoint{2.686398in}{0.681525in}}%
\pgfpathlineto{\pgfqpoint{2.687945in}{0.684209in}}%
\pgfpathlineto{\pgfqpoint{2.695113in}{0.685242in}}%
\pgfpathlineto{\pgfqpoint{2.696670in}{0.692368in}}%
\pgfpathlineto{\pgfqpoint{2.738366in}{0.692368in}}%
\pgfpathlineto{\pgfqpoint{2.739918in}{0.686748in}}%
\pgfpathlineto{\pgfqpoint{2.745167in}{0.686748in}}%
\pgfpathlineto{\pgfqpoint{2.746714in}{0.684209in}}%
\pgfpathlineto{\pgfqpoint{2.753386in}{0.684209in}}%
\pgfpathlineto{\pgfqpoint{2.754933in}{0.686748in}}%
\pgfpathlineto{\pgfqpoint{2.757703in}{0.687729in}}%
\pgfpathlineto{\pgfqpoint{2.759255in}{0.692368in}}%
\pgfpathlineto{\pgfqpoint{2.792489in}{0.693741in}}%
\pgfpathlineto{\pgfqpoint{2.794437in}{0.850079in}}%
\pgfpathlineto{\pgfqpoint{2.840145in}{0.848687in}}%
\pgfpathlineto{\pgfqpoint{2.840809in}{0.773401in}}%
\pgfpathlineto{\pgfqpoint{2.842901in}{0.684209in}}%
\pgfpathlineto{\pgfqpoint{2.854052in}{0.685242in}}%
\pgfpathlineto{\pgfqpoint{2.855604in}{0.690099in}}%
\pgfpathlineto{\pgfqpoint{2.897329in}{0.690099in}}%
\pgfpathlineto{\pgfqpoint{2.898881in}{0.684209in}}%
\pgfpathlineto{\pgfqpoint{2.966122in}{0.685242in}}%
\pgfpathlineto{\pgfqpoint{2.967674in}{0.690099in}}%
\pgfpathlineto{\pgfqpoint{2.979685in}{0.690099in}}%
\pgfpathlineto{\pgfqpoint{2.981237in}{0.685242in}}%
\pgfpathlineto{\pgfqpoint{3.022661in}{0.685242in}}%
\pgfpathlineto{\pgfqpoint{3.024232in}{0.668564in}}%
\pgfpathlineto{\pgfqpoint{3.040045in}{0.667300in}}%
\pgfpathlineto{\pgfqpoint{3.041592in}{0.665980in}}%
\pgfpathlineto{\pgfqpoint{3.085132in}{0.665980in}}%
\pgfpathlineto{\pgfqpoint{3.086708in}{0.637273in}}%
\pgfpathlineto{\pgfqpoint{3.087457in}{0.637273in}}%
\pgfpathlineto{\pgfqpoint{3.089009in}{0.666124in}}%
\pgfpathlineto{\pgfqpoint{3.120576in}{0.665980in}}%
\pgfpathlineto{\pgfqpoint{3.122372in}{0.846511in}}%
\pgfpathlineto{\pgfqpoint{3.168319in}{0.845019in}}%
\pgfpathlineto{\pgfqpoint{3.171022in}{0.721717in}}%
\pgfpathlineto{\pgfqpoint{3.174045in}{0.723027in}}%
\pgfpathlineto{\pgfqpoint{3.175912in}{0.810719in}}%
\pgfpathlineto{\pgfqpoint{3.175988in}{0.811999in}}%
\pgfpathlineto{\pgfqpoint{3.178009in}{0.875651in}}%
\pgfpathlineto{\pgfqpoint{3.178080in}{0.877050in}}%
\pgfpathlineto{\pgfqpoint{3.179771in}{0.891485in}}%
\pgfpathlineto{\pgfqpoint{3.182999in}{0.892796in}}%
\pgfpathlineto{\pgfqpoint{3.184589in}{0.897120in}}%
\pgfpathlineto{\pgfqpoint{3.186466in}{0.898404in}}%
\pgfpathlineto{\pgfqpoint{3.188042in}{0.901373in}}%
\pgfpathlineto{\pgfqpoint{3.191963in}{0.902636in}}%
\pgfpathlineto{\pgfqpoint{3.193940in}{0.941010in}}%
\pgfpathlineto{\pgfqpoint{3.194408in}{0.942473in}}%
\pgfpathlineto{\pgfqpoint{3.196008in}{0.946823in}}%
\pgfpathlineto{\pgfqpoint{3.196963in}{0.945380in}}%
\pgfpathlineto{\pgfqpoint{3.198601in}{0.938061in}}%
\pgfpathlineto{\pgfqpoint{3.200884in}{0.936575in}}%
\pgfpathlineto{\pgfqpoint{3.202450in}{0.934708in}}%
\pgfpathlineto{\pgfqpoint{3.209012in}{0.933206in}}%
\pgfpathlineto{\pgfqpoint{3.210583in}{0.930938in}}%
\pgfpathlineto{\pgfqpoint{3.210918in}{0.929417in}}%
\pgfpathlineto{\pgfqpoint{3.212522in}{0.924416in}}%
\pgfpathlineto{\pgfqpoint{3.231252in}{0.924416in}}%
\pgfpathlineto{\pgfqpoint{3.233177in}{2.752681in}}%
\pgfpathlineto{\pgfqpoint{3.233430in}{2.754207in}}%
\pgfpathlineto{\pgfqpoint{3.233549in}{2.772441in}}%
\pgfpathlineto{\pgfqpoint{3.235159in}{2.820186in}}%
\pgfpathlineto{\pgfqpoint{3.246362in}{2.819484in}}%
\pgfpathlineto{\pgfqpoint{3.248827in}{2.096380in}}%
\pgfpathlineto{\pgfqpoint{3.288188in}{2.094849in}}%
\pgfpathlineto{\pgfqpoint{3.289659in}{2.053371in}}%
\pgfpathlineto{\pgfqpoint{3.292085in}{0.702108in}}%
\pgfpathlineto{\pgfqpoint{3.303169in}{0.702108in}}%
\pgfpathlineto{\pgfqpoint{3.304716in}{0.700400in}}%
\pgfpathlineto{\pgfqpoint{3.308776in}{0.700400in}}%
\pgfpathlineto{\pgfqpoint{3.310170in}{0.662386in}}%
\pgfpathlineto{\pgfqpoint{3.310414in}{0.666124in}}%
\pgfpathlineto{\pgfqpoint{3.333576in}{0.666268in}}%
\pgfpathlineto{\pgfqpoint{3.335171in}{0.703583in}}%
\pgfpathlineto{\pgfqpoint{3.350066in}{0.703583in}}%
\pgfpathlineto{\pgfqpoint{3.351618in}{0.700400in}}%
\pgfpathlineto{\pgfqpoint{3.381194in}{0.700400in}}%
\pgfpathlineto{\pgfqpoint{3.382856in}{0.657775in}}%
\pgfpathlineto{\pgfqpoint{3.390267in}{0.657775in}}%
\pgfpathlineto{\pgfqpoint{3.391858in}{0.703614in}}%
\pgfpathlineto{\pgfqpoint{3.418716in}{0.704874in}}%
\pgfpathlineto{\pgfqpoint{3.420679in}{0.837791in}}%
\pgfpathlineto{\pgfqpoint{3.466277in}{0.836761in}}%
\pgfpathlineto{\pgfqpoint{3.469075in}{0.700400in}}%
\pgfpathlineto{\pgfqpoint{3.495408in}{0.700399in}}%
\pgfpathlineto{\pgfqpoint{3.497046in}{0.666053in}}%
\pgfpathlineto{\pgfqpoint{3.507276in}{0.666053in}}%
\pgfpathlineto{\pgfqpoint{3.508828in}{0.657775in}}%
\pgfpathlineto{\pgfqpoint{3.555796in}{0.657775in}}%
\pgfpathlineto{\pgfqpoint{3.557387in}{0.703614in}}%
\pgfpathlineto{\pgfqpoint{3.584250in}{0.704905in}}%
\pgfpathlineto{\pgfqpoint{3.586213in}{0.837791in}}%
\pgfpathlineto{\pgfqpoint{3.631811in}{0.836761in}}%
\pgfpathlineto{\pgfqpoint{3.635005in}{0.657572in}}%
\pgfpathlineto{\pgfqpoint{3.670737in}{0.657775in}}%
\pgfpathlineto{\pgfqpoint{3.672322in}{0.700400in}}%
\pgfpathlineto{\pgfqpoint{3.718326in}{0.700400in}}%
\pgfpathlineto{\pgfqpoint{3.719964in}{0.666124in}}%
\pgfpathlineto{\pgfqpoint{3.749034in}{0.666124in}}%
\pgfpathlineto{\pgfqpoint{3.750829in}{0.847513in}}%
\pgfpathlineto{\pgfqpoint{3.788853in}{0.846018in}}%
\pgfpathlineto{\pgfqpoint{3.790463in}{0.838837in}}%
\pgfpathlineto{\pgfqpoint{3.796752in}{0.837791in}}%
\pgfpathlineto{\pgfqpoint{3.799570in}{0.700400in}}%
\pgfpathlineto{\pgfqpoint{3.832746in}{0.700399in}}%
\pgfpathlineto{\pgfqpoint{3.834384in}{0.665980in}}%
\pgfpathlineto{\pgfqpoint{3.896009in}{0.666053in}}%
\pgfpathlineto{\pgfqpoint{3.897561in}{0.657775in}}%
\pgfpathlineto{\pgfqpoint{3.903990in}{0.657775in}}%
\pgfpathlineto{\pgfqpoint{3.905537in}{0.666124in}}%
\pgfpathlineto{\pgfqpoint{3.920327in}{0.666124in}}%
\pgfpathlineto{\pgfqpoint{3.922132in}{0.846511in}}%
\pgfpathlineto{\pgfqpoint{3.968064in}{0.845029in}}%
\pgfpathlineto{\pgfqpoint{3.971407in}{0.657775in}}%
\pgfpathlineto{\pgfqpoint{4.006762in}{0.657776in}}%
\pgfpathlineto{\pgfqpoint{4.008347in}{0.700367in}}%
\pgfpathlineto{\pgfqpoint{4.054475in}{0.700367in}}%
\pgfpathlineto{\pgfqpoint{4.056137in}{0.657775in}}%
\pgfpathlineto{\pgfqpoint{4.085144in}{0.657775in}}%
\pgfpathlineto{\pgfqpoint{4.086892in}{0.847513in}}%
\pgfpathlineto{\pgfqpoint{4.132892in}{0.846038in}}%
\pgfpathlineto{\pgfqpoint{4.136096in}{0.637273in}}%
\pgfpathlineto{\pgfqpoint{4.136254in}{0.657572in}}%
\pgfpathlineto{\pgfqpoint{4.226752in}{0.657776in}}%
\pgfpathlineto{\pgfqpoint{4.228333in}{0.697047in}}%
\pgfpathlineto{\pgfqpoint{4.255282in}{0.698374in}}%
\pgfpathlineto{\pgfqpoint{4.257231in}{0.839856in}}%
\pgfpathlineto{\pgfqpoint{4.302848in}{0.838325in}}%
\pgfpathlineto{\pgfqpoint{4.306071in}{0.657775in}}%
\pgfpathlineto{\pgfqpoint{4.353751in}{0.657775in}}%
\pgfpathlineto{\pgfqpoint{4.355375in}{0.687763in}}%
\pgfpathlineto{\pgfqpoint{4.401494in}{0.687763in}}%
\pgfpathlineto{\pgfqpoint{4.403093in}{0.657775in}}%
\pgfpathlineto{\pgfqpoint{4.420663in}{0.657775in}}%
\pgfpathlineto{\pgfqpoint{4.422411in}{0.846511in}}%
\pgfpathlineto{\pgfqpoint{4.453964in}{0.845029in}}%
\pgfpathlineto{\pgfqpoint{4.455573in}{0.837791in}}%
\pgfpathlineto{\pgfqpoint{4.468405in}{0.836761in}}%
\pgfpathlineto{\pgfqpoint{4.471204in}{0.700400in}}%
\pgfpathlineto{\pgfqpoint{4.497217in}{0.700400in}}%
\pgfpathlineto{\pgfqpoint{4.498879in}{0.657674in}}%
\pgfpathlineto{\pgfqpoint{4.516071in}{0.657674in}}%
\pgfpathlineto{\pgfqpoint{4.517680in}{0.683050in}}%
\pgfpathlineto{\pgfqpoint{4.561631in}{0.683050in}}%
\pgfpathlineto{\pgfqpoint{4.563178in}{0.678914in}}%
\pgfpathlineto{\pgfqpoint{4.567032in}{0.679835in}}%
\pgfpathlineto{\pgfqpoint{4.568584in}{0.686748in}}%
\pgfpathlineto{\pgfqpoint{4.577653in}{0.687729in}}%
\pgfpathlineto{\pgfqpoint{4.579205in}{0.692368in}}%
\pgfpathlineto{\pgfqpoint{4.579267in}{0.693684in}}%
\pgfpathlineto{\pgfqpoint{4.580824in}{0.701781in}}%
\pgfpathlineto{\pgfqpoint{4.598886in}{0.702536in}}%
\pgfpathlineto{\pgfqpoint{4.600433in}{0.703651in}}%
\pgfpathlineto{\pgfqpoint{4.634823in}{0.705021in}}%
\pgfpathlineto{\pgfqpoint{4.636819in}{0.851727in}}%
\pgfpathlineto{\pgfqpoint{4.674680in}{0.851336in}}%
\pgfpathlineto{\pgfqpoint{4.676452in}{2.913602in}}%
\pgfpathlineto{\pgfqpoint{4.696586in}{2.914518in}}%
\pgfpathlineto{\pgfqpoint{4.698239in}{3.242820in}}%
\pgfpathlineto{\pgfqpoint{4.698363in}{3.137786in}}%
\pgfpathlineto{\pgfqpoint{4.698845in}{3.137786in}}%
\pgfpathlineto{\pgfqpoint{4.700402in}{3.125371in}}%
\pgfpathlineto{\pgfqpoint{4.739920in}{3.125371in}}%
\pgfpathlineto{\pgfqpoint{4.745016in}{0.637273in}}%
\pgfpathlineto{\pgfqpoint{4.758503in}{0.637273in}}%
\pgfpathlineto{\pgfqpoint{4.760117in}{2.913602in}}%
\pgfpathlineto{\pgfqpoint{4.767295in}{2.913602in}}%
\pgfpathlineto{\pgfqpoint{4.769233in}{3.060643in}}%
\pgfpathlineto{\pgfqpoint{4.769673in}{3.060643in}}%
\pgfpathlineto{\pgfqpoint{4.771297in}{3.001850in}}%
\pgfpathlineto{\pgfqpoint{4.810829in}{3.001850in}}%
\pgfpathlineto{\pgfqpoint{4.814870in}{0.712957in}}%
\pgfpathlineto{\pgfqpoint{4.823719in}{0.712957in}}%
\pgfpathlineto{\pgfqpoint{4.825309in}{0.696645in}}%
\pgfpathlineto{\pgfqpoint{4.825457in}{0.696645in}}%
\pgfpathlineto{\pgfqpoint{4.827052in}{0.673632in}}%
\pgfpathlineto{\pgfqpoint{4.827119in}{0.673632in}}%
\pgfpathlineto{\pgfqpoint{4.828681in}{0.657573in}}%
\pgfpathlineto{\pgfqpoint{4.943932in}{0.657775in}}%
\pgfpathlineto{\pgfqpoint{4.945680in}{0.848491in}}%
\pgfpathlineto{\pgfqpoint{4.977672in}{0.847023in}}%
\pgfpathlineto{\pgfqpoint{4.979281in}{0.839856in}}%
\pgfpathlineto{\pgfqpoint{4.991684in}{0.838325in}}%
\pgfpathlineto{\pgfqpoint{4.994539in}{0.697047in}}%
\pgfpathlineto{\pgfqpoint{5.021574in}{0.696977in}}%
\pgfpathlineto{\pgfqpoint{5.023222in}{0.657674in}}%
\pgfpathlineto{\pgfqpoint{5.073491in}{0.657573in}}%
\pgfpathlineto{\pgfqpoint{5.075047in}{0.637273in}}%
\pgfpathlineto{\pgfqpoint{5.082999in}{0.637273in}}%
\pgfpathlineto{\pgfqpoint{5.084546in}{0.657573in}}%
\pgfpathlineto{\pgfqpoint{5.094413in}{0.657573in}}%
\pgfpathlineto{\pgfqpoint{5.095960in}{0.665981in}}%
\pgfpathlineto{\pgfqpoint{5.105970in}{0.666125in}}%
\pgfpathlineto{\pgfqpoint{5.107765in}{0.847513in}}%
\pgfpathlineto{\pgfqpoint{5.125158in}{0.846018in}}%
\pgfpathlineto{\pgfqpoint{5.126725in}{0.843507in}}%
\pgfpathlineto{\pgfqpoint{5.153612in}{0.841992in}}%
\pgfpathlineto{\pgfqpoint{5.156515in}{0.695199in}}%
\pgfpathlineto{\pgfqpoint{5.168464in}{0.695199in}}%
\pgfpathlineto{\pgfqpoint{5.170083in}{0.665981in}}%
\pgfpathlineto{\pgfqpoint{5.184568in}{0.667135in}}%
\pgfpathlineto{\pgfqpoint{5.186134in}{0.684479in}}%
\pgfpathlineto{\pgfqpoint{5.186325in}{0.684479in}}%
\pgfpathlineto{\pgfqpoint{5.187901in}{0.700340in}}%
\pgfpathlineto{\pgfqpoint{5.188216in}{0.700340in}}%
\pgfpathlineto{\pgfqpoint{5.188636in}{0.703785in}}%
\pgfpathlineto{\pgfqpoint{5.188636in}{0.703785in}}%
\pgfusepath{stroke}%
\end{pgfscope}%
\begin{pgfscope}%
\pgfpathrectangle{\pgfqpoint{0.750000in}{0.500000in}}{\pgfqpoint{4.650000in}{3.020000in}}%
\pgfusepath{clip}%
\pgfsetrectcap%
\pgfsetroundjoin%
\pgfsetlinewidth{1.505625pt}%
\definecolor{currentstroke}{rgb}{1.000000,0.000000,0.000000}%
\pgfsetstrokecolor{currentstroke}%
\pgfsetdash{}{0pt}%
\pgfpathmoveto{\pgfqpoint{0.961364in}{0.960794in}}%
\pgfpathlineto{\pgfqpoint{0.964821in}{0.793899in}}%
\pgfpathlineto{\pgfqpoint{0.968838in}{0.649979in}}%
\pgfpathlineto{\pgfqpoint{0.974220in}{0.651522in}}%
\pgfpathlineto{\pgfqpoint{0.982768in}{0.653853in}}%
\pgfpathlineto{\pgfqpoint{0.991909in}{0.647804in}}%
\pgfpathlineto{\pgfqpoint{0.994956in}{0.646804in}}%
\pgfpathlineto{\pgfqpoint{1.015725in}{0.645261in}}%
\pgfpathlineto{\pgfqpoint{1.017731in}{0.646016in}}%
\pgfpathlineto{\pgfqpoint{1.023103in}{0.649189in}}%
\pgfpathlineto{\pgfqpoint{1.027549in}{0.648797in}}%
\pgfpathlineto{\pgfqpoint{1.062125in}{0.647259in}}%
\pgfpathlineto{\pgfqpoint{1.064718in}{0.644003in}}%
\pgfpathlineto{\pgfqpoint{1.065631in}{0.644856in}}%
\pgfpathlineto{\pgfqpoint{1.080965in}{0.657335in}}%
\pgfpathlineto{\pgfqpoint{1.085827in}{0.655792in}}%
\pgfpathlineto{\pgfqpoint{1.087952in}{0.656488in}}%
\pgfpathlineto{\pgfqpoint{1.093664in}{0.661633in}}%
\pgfpathlineto{\pgfqpoint{1.109323in}{0.660093in}}%
\pgfpathlineto{\pgfqpoint{1.115469in}{0.654797in}}%
\pgfpathlineto{\pgfqpoint{1.115890in}{0.655946in}}%
\pgfpathlineto{\pgfqpoint{1.120871in}{0.673949in}}%
\pgfpathlineto{\pgfqpoint{1.122088in}{0.672471in}}%
\pgfpathlineto{\pgfqpoint{1.124586in}{0.672624in}}%
\pgfpathlineto{\pgfqpoint{1.130054in}{0.673532in}}%
\pgfpathlineto{\pgfqpoint{1.138550in}{0.664964in}}%
\pgfpathlineto{\pgfqpoint{1.146731in}{0.666503in}}%
\pgfpathlineto{\pgfqpoint{1.151568in}{0.672134in}}%
\pgfpathlineto{\pgfqpoint{1.156124in}{0.672579in}}%
\pgfpathlineto{\pgfqpoint{1.159467in}{0.669295in}}%
\pgfpathlineto{\pgfqpoint{1.165466in}{0.651512in}}%
\pgfpathlineto{\pgfqpoint{1.170943in}{0.653054in}}%
\pgfpathlineto{\pgfqpoint{1.176025in}{0.653745in}}%
\pgfpathlineto{\pgfqpoint{1.182200in}{0.655288in}}%
\pgfpathlineto{\pgfqpoint{1.187310in}{0.656008in}}%
\pgfpathlineto{\pgfqpoint{1.188460in}{0.654470in}}%
\pgfpathlineto{\pgfqpoint{1.193098in}{0.648300in}}%
\pgfpathlineto{\pgfqpoint{1.193274in}{0.648418in}}%
\pgfpathlineto{\pgfqpoint{1.196116in}{0.648468in}}%
\pgfpathlineto{\pgfqpoint{1.204301in}{0.646311in}}%
\pgfpathlineto{\pgfqpoint{1.214106in}{0.644801in}}%
\pgfpathlineto{\pgfqpoint{1.218552in}{0.642714in}}%
\pgfpathlineto{\pgfqpoint{1.226647in}{0.644255in}}%
\pgfpathlineto{\pgfqpoint{1.231718in}{0.644942in}}%
\pgfpathlineto{\pgfqpoint{1.237081in}{0.643400in}}%
\pgfpathlineto{\pgfqpoint{1.242015in}{0.644022in}}%
\pgfpathlineto{\pgfqpoint{1.247769in}{0.642479in}}%
\pgfpathlineto{\pgfqpoint{1.251776in}{0.642282in}}%
\pgfpathlineto{\pgfqpoint{1.258180in}{0.649326in}}%
\pgfpathlineto{\pgfqpoint{1.280912in}{0.650858in}}%
\pgfpathlineto{\pgfqpoint{1.287183in}{0.670369in}}%
\pgfpathlineto{\pgfqpoint{1.296896in}{0.668828in}}%
\pgfpathlineto{\pgfqpoint{1.301128in}{0.665146in}}%
\pgfpathlineto{\pgfqpoint{1.306424in}{0.663605in}}%
\pgfpathlineto{\pgfqpoint{1.310206in}{0.663523in}}%
\pgfpathlineto{\pgfqpoint{1.321759in}{0.667269in}}%
\pgfpathlineto{\pgfqpoint{1.323516in}{0.665731in}}%
\pgfpathlineto{\pgfqpoint{1.328631in}{0.645366in}}%
\pgfpathlineto{\pgfqpoint{1.330469in}{0.646094in}}%
\pgfpathlineto{\pgfqpoint{1.345427in}{0.647635in}}%
\pgfpathlineto{\pgfqpoint{1.352160in}{0.649125in}}%
\pgfpathlineto{\pgfqpoint{1.359720in}{0.643853in}}%
\pgfpathlineto{\pgfqpoint{1.363216in}{0.642651in}}%
\pgfpathlineto{\pgfqpoint{1.367509in}{0.644188in}}%
\pgfpathlineto{\pgfqpoint{1.371970in}{0.648003in}}%
\pgfpathlineto{\pgfqpoint{1.374773in}{0.648373in}}%
\pgfpathlineto{\pgfqpoint{1.383174in}{0.650632in}}%
\pgfpathlineto{\pgfqpoint{1.388723in}{0.648769in}}%
\pgfpathlineto{\pgfqpoint{1.392639in}{0.650311in}}%
\pgfpathlineto{\pgfqpoint{1.397706in}{0.650996in}}%
\pgfpathlineto{\pgfqpoint{1.403790in}{0.652538in}}%
\pgfpathlineto{\pgfqpoint{1.408867in}{0.653255in}}%
\pgfpathlineto{\pgfqpoint{1.410500in}{0.651713in}}%
\pgfpathlineto{\pgfqpoint{1.415013in}{0.647830in}}%
\pgfpathlineto{\pgfqpoint{1.417883in}{0.647494in}}%
\pgfpathlineto{\pgfqpoint{1.425715in}{0.645504in}}%
\pgfpathlineto{\pgfqpoint{1.445415in}{0.647092in}}%
\pgfpathlineto{\pgfqpoint{1.453361in}{0.668943in}}%
\pgfpathlineto{\pgfqpoint{1.457855in}{0.667401in}}%
\pgfpathlineto{\pgfqpoint{1.467602in}{0.664452in}}%
\pgfpathlineto{\pgfqpoint{1.480033in}{0.662910in}}%
\pgfpathlineto{\pgfqpoint{1.482240in}{0.663561in}}%
\pgfpathlineto{\pgfqpoint{1.487593in}{0.668157in}}%
\pgfpathlineto{\pgfqpoint{1.488429in}{0.666615in}}%
\pgfpathlineto{\pgfqpoint{1.493529in}{0.645139in}}%
\pgfpathlineto{\pgfqpoint{1.495487in}{0.646064in}}%
\pgfpathlineto{\pgfqpoint{1.501022in}{0.646985in}}%
\pgfpathlineto{\pgfqpoint{1.508353in}{0.648527in}}%
\pgfpathlineto{\pgfqpoint{1.513434in}{0.649219in}}%
\pgfpathlineto{\pgfqpoint{1.518616in}{0.650761in}}%
\pgfpathlineto{\pgfqpoint{1.523821in}{0.652395in}}%
\pgfpathlineto{\pgfqpoint{1.526176in}{0.648961in}}%
\pgfpathlineto{\pgfqpoint{1.530149in}{0.645668in}}%
\pgfpathlineto{\pgfqpoint{1.546095in}{0.648430in}}%
\pgfpathlineto{\pgfqpoint{1.550484in}{0.646887in}}%
\pgfpathlineto{\pgfqpoint{1.555775in}{0.644002in}}%
\pgfpathlineto{\pgfqpoint{1.575083in}{0.645544in}}%
\pgfpathlineto{\pgfqpoint{1.578173in}{0.645297in}}%
\pgfpathlineto{\pgfqpoint{1.585795in}{0.643407in}}%
\pgfpathlineto{\pgfqpoint{1.596144in}{0.644950in}}%
\pgfpathlineto{\pgfqpoint{1.605428in}{0.647677in}}%
\pgfpathlineto{\pgfqpoint{1.608226in}{0.648048in}}%
\pgfpathlineto{\pgfqpoint{1.619358in}{0.656464in}}%
\pgfpathlineto{\pgfqpoint{1.624798in}{0.675664in}}%
\pgfpathlineto{\pgfqpoint{1.625333in}{0.675418in}}%
\pgfpathlineto{\pgfqpoint{1.629507in}{0.671715in}}%
\pgfpathlineto{\pgfqpoint{1.632329in}{0.671151in}}%
\pgfpathlineto{\pgfqpoint{1.640381in}{0.669609in}}%
\pgfpathlineto{\pgfqpoint{1.652062in}{0.665736in}}%
\pgfpathlineto{\pgfqpoint{1.655056in}{0.667535in}}%
\pgfpathlineto{\pgfqpoint{1.659870in}{0.671615in}}%
\pgfpathlineto{\pgfqpoint{1.661547in}{0.670074in}}%
\pgfpathlineto{\pgfqpoint{1.668261in}{0.651928in}}%
\pgfpathlineto{\pgfqpoint{1.671895in}{0.653471in}}%
\pgfpathlineto{\pgfqpoint{1.676981in}{0.654193in}}%
\pgfpathlineto{\pgfqpoint{1.684618in}{0.655735in}}%
\pgfpathlineto{\pgfqpoint{1.689728in}{0.656455in}}%
\pgfpathlineto{\pgfqpoint{1.697789in}{0.654914in}}%
\pgfpathlineto{\pgfqpoint{1.701910in}{0.650958in}}%
\pgfpathlineto{\pgfqpoint{1.706987in}{0.649418in}}%
\pgfpathlineto{\pgfqpoint{1.711237in}{0.647278in}}%
\pgfpathlineto{\pgfqpoint{1.714489in}{0.647107in}}%
\pgfpathlineto{\pgfqpoint{1.716185in}{0.648004in}}%
\pgfpathlineto{\pgfqpoint{1.723721in}{0.652801in}}%
\pgfpathlineto{\pgfqpoint{1.755379in}{0.651259in}}%
\pgfpathlineto{\pgfqpoint{1.760278in}{0.647246in}}%
\pgfpathlineto{\pgfqpoint{1.764968in}{0.641775in}}%
\pgfpathlineto{\pgfqpoint{1.783947in}{0.640232in}}%
\pgfpathlineto{\pgfqpoint{1.789047in}{0.639504in}}%
\pgfpathlineto{\pgfqpoint{1.793837in}{0.641036in}}%
\pgfpathlineto{\pgfqpoint{1.801937in}{0.665392in}}%
\pgfpathlineto{\pgfqpoint{1.814477in}{0.669604in}}%
\pgfpathlineto{\pgfqpoint{1.819540in}{0.668919in}}%
\pgfpathlineto{\pgfqpoint{1.824841in}{0.669714in}}%
\pgfpathlineto{\pgfqpoint{1.830992in}{0.668172in}}%
\pgfpathlineto{\pgfqpoint{1.836073in}{0.667465in}}%
\pgfpathlineto{\pgfqpoint{1.836837in}{0.665922in}}%
\pgfpathlineto{\pgfqpoint{1.844168in}{0.644361in}}%
\pgfpathlineto{\pgfqpoint{1.848915in}{0.643827in}}%
\pgfpathlineto{\pgfqpoint{1.855882in}{0.644038in}}%
\pgfpathlineto{\pgfqpoint{1.864383in}{0.645579in}}%
\pgfpathlineto{\pgfqpoint{1.869331in}{0.646235in}}%
\pgfpathlineto{\pgfqpoint{1.875386in}{0.647778in}}%
\pgfpathlineto{\pgfqpoint{1.883428in}{0.649867in}}%
\pgfpathlineto{\pgfqpoint{1.891662in}{0.647687in}}%
\pgfpathlineto{\pgfqpoint{1.896991in}{0.648215in}}%
\pgfpathlineto{\pgfqpoint{1.899794in}{0.648270in}}%
\pgfpathlineto{\pgfqpoint{1.908529in}{0.649805in}}%
\pgfpathlineto{\pgfqpoint{1.910936in}{0.649796in}}%
\pgfpathlineto{\pgfqpoint{1.916290in}{0.651674in}}%
\pgfpathlineto{\pgfqpoint{1.921495in}{0.650921in}}%
\pgfpathlineto{\pgfqpoint{1.922422in}{0.653411in}}%
\pgfpathlineto{\pgfqpoint{1.928401in}{0.665985in}}%
\pgfpathlineto{\pgfqpoint{1.931366in}{0.669191in}}%
\pgfpathlineto{\pgfqpoint{1.932661in}{0.722594in}}%
\pgfpathlineto{\pgfqpoint{1.936877in}{0.924169in}}%
\pgfpathlineto{\pgfqpoint{1.937560in}{0.900420in}}%
\pgfpathlineto{\pgfqpoint{1.939666in}{0.875450in}}%
\pgfpathlineto{\pgfqpoint{1.945158in}{0.873188in}}%
\pgfpathlineto{\pgfqpoint{1.945330in}{0.875021in}}%
\pgfpathlineto{\pgfqpoint{1.951501in}{0.980128in}}%
\pgfpathlineto{\pgfqpoint{1.953358in}{1.071476in}}%
\pgfpathlineto{\pgfqpoint{1.956859in}{1.208554in}}%
\pgfpathlineto{\pgfqpoint{1.957088in}{1.208524in}}%
\pgfpathlineto{\pgfqpoint{1.964390in}{1.206253in}}%
\pgfpathlineto{\pgfqpoint{1.965827in}{1.202256in}}%
\pgfpathlineto{\pgfqpoint{1.971157in}{1.189739in}}%
\pgfpathlineto{\pgfqpoint{1.974323in}{1.186459in}}%
\pgfpathlineto{\pgfqpoint{1.975498in}{1.140387in}}%
\pgfpathlineto{\pgfqpoint{1.980088in}{0.978972in}}%
\pgfpathlineto{\pgfqpoint{1.988287in}{0.977441in}}%
\pgfpathlineto{\pgfqpoint{1.994520in}{0.867439in}}%
\pgfpathlineto{\pgfqpoint{1.996459in}{0.771149in}}%
\pgfpathlineto{\pgfqpoint{2.000265in}{0.637273in}}%
\pgfpathlineto{\pgfqpoint{2.014850in}{0.638813in}}%
\pgfpathlineto{\pgfqpoint{2.023011in}{0.688204in}}%
\pgfpathlineto{\pgfqpoint{2.025170in}{0.689562in}}%
\pgfpathlineto{\pgfqpoint{2.033957in}{0.689604in}}%
\pgfpathlineto{\pgfqpoint{2.040796in}{0.691143in}}%
\pgfpathlineto{\pgfqpoint{2.046488in}{0.706783in}}%
\pgfpathlineto{\pgfqpoint{2.048012in}{0.767430in}}%
\pgfpathlineto{\pgfqpoint{2.052697in}{0.977259in}}%
\pgfpathlineto{\pgfqpoint{2.053580in}{0.977055in}}%
\pgfpathlineto{\pgfqpoint{2.058399in}{0.975522in}}%
\pgfpathlineto{\pgfqpoint{2.067196in}{0.944025in}}%
\pgfpathlineto{\pgfqpoint{2.079593in}{0.942446in}}%
\pgfpathlineto{\pgfqpoint{2.082817in}{0.939237in}}%
\pgfpathlineto{\pgfqpoint{2.090076in}{0.892659in}}%
\pgfpathlineto{\pgfqpoint{2.098142in}{0.646302in}}%
\pgfpathlineto{\pgfqpoint{2.106046in}{0.644255in}}%
\pgfpathlineto{\pgfqpoint{2.110840in}{0.645797in}}%
\pgfpathlineto{\pgfqpoint{2.114026in}{0.649004in}}%
\pgfpathlineto{\pgfqpoint{2.120248in}{0.669735in}}%
\pgfpathlineto{\pgfqpoint{2.128596in}{0.671276in}}%
\pgfpathlineto{\pgfqpoint{2.133338in}{0.671823in}}%
\pgfpathlineto{\pgfqpoint{2.138830in}{0.666822in}}%
\pgfpathlineto{\pgfqpoint{2.143916in}{0.668360in}}%
\pgfpathlineto{\pgfqpoint{2.156768in}{0.666817in}}%
\pgfpathlineto{\pgfqpoint{2.161859in}{0.646445in}}%
\pgfpathlineto{\pgfqpoint{2.163549in}{0.647163in}}%
\pgfpathlineto{\pgfqpoint{2.181611in}{0.645621in}}%
\pgfpathlineto{\pgfqpoint{2.185154in}{0.645107in}}%
\pgfpathlineto{\pgfqpoint{2.190861in}{0.646097in}}%
\pgfpathlineto{\pgfqpoint{2.208436in}{0.639509in}}%
\pgfpathlineto{\pgfqpoint{2.212762in}{0.641052in}}%
\pgfpathlineto{\pgfqpoint{2.226106in}{0.648100in}}%
\pgfpathlineto{\pgfqpoint{2.231387in}{0.647313in}}%
\pgfpathlineto{\pgfqpoint{2.237109in}{0.648339in}}%
\pgfpathlineto{\pgfqpoint{2.243169in}{0.646796in}}%
\pgfpathlineto{\pgfqpoint{2.248279in}{0.646064in}}%
\pgfpathlineto{\pgfqpoint{2.260810in}{0.644523in}}%
\pgfpathlineto{\pgfqpoint{2.267377in}{0.639495in}}%
\pgfpathlineto{\pgfqpoint{2.276556in}{0.637954in}}%
\pgfpathlineto{\pgfqpoint{2.281613in}{0.637273in}}%
\pgfpathlineto{\pgfqpoint{2.284794in}{0.638809in}}%
\pgfpathlineto{\pgfqpoint{2.292922in}{0.663174in}}%
\pgfpathlineto{\pgfqpoint{2.299350in}{0.664536in}}%
\pgfpathlineto{\pgfqpoint{2.305176in}{0.662993in}}%
\pgfpathlineto{\pgfqpoint{2.310324in}{0.662255in}}%
\pgfpathlineto{\pgfqpoint{2.314708in}{0.663798in}}%
\pgfpathlineto{\pgfqpoint{2.320057in}{0.668082in}}%
\pgfpathlineto{\pgfqpoint{2.328133in}{0.666552in}}%
\pgfpathlineto{\pgfqpoint{2.334642in}{0.646243in}}%
\pgfpathlineto{\pgfqpoint{2.339246in}{0.644701in}}%
\pgfpathlineto{\pgfqpoint{2.344298in}{0.644010in}}%
\pgfpathlineto{\pgfqpoint{2.351944in}{0.645553in}}%
\pgfpathlineto{\pgfqpoint{2.356500in}{0.646011in}}%
\pgfpathlineto{\pgfqpoint{2.363344in}{0.641273in}}%
\pgfpathlineto{\pgfqpoint{2.368033in}{0.641792in}}%
\pgfpathlineto{\pgfqpoint{2.373029in}{0.643335in}}%
\pgfpathlineto{\pgfqpoint{2.389347in}{0.652461in}}%
\pgfpathlineto{\pgfqpoint{2.400030in}{0.654003in}}%
\pgfpathlineto{\pgfqpoint{2.406516in}{0.654987in}}%
\pgfpathlineto{\pgfqpoint{2.434506in}{0.652591in}}%
\pgfpathlineto{\pgfqpoint{2.436975in}{0.651858in}}%
\pgfpathlineto{\pgfqpoint{2.441870in}{0.649278in}}%
\pgfpathlineto{\pgfqpoint{2.448326in}{0.647735in}}%
\pgfpathlineto{\pgfqpoint{2.457491in}{0.646044in}}%
\pgfpathlineto{\pgfqpoint{2.467558in}{0.648685in}}%
\pgfpathlineto{\pgfqpoint{2.471297in}{0.648876in}}%
\pgfpathlineto{\pgfqpoint{2.479459in}{0.645770in}}%
\pgfpathlineto{\pgfqpoint{2.482659in}{0.645293in}}%
\pgfpathlineto{\pgfqpoint{2.485156in}{0.646025in}}%
\pgfpathlineto{\pgfqpoint{2.494999in}{0.649349in}}%
\pgfpathlineto{\pgfqpoint{2.500357in}{0.650892in}}%
\pgfpathlineto{\pgfqpoint{2.503891in}{0.649499in}}%
\pgfpathlineto{\pgfqpoint{2.506952in}{0.648770in}}%
\pgfpathlineto{\pgfqpoint{2.519976in}{0.652324in}}%
\pgfpathlineto{\pgfqpoint{2.522736in}{0.653446in}}%
\pgfpathlineto{\pgfqpoint{2.522793in}{0.653376in}}%
\pgfpathlineto{\pgfqpoint{2.525086in}{0.653095in}}%
\pgfpathlineto{\pgfqpoint{2.531046in}{0.651554in}}%
\pgfpathlineto{\pgfqpoint{2.536136in}{0.648978in}}%
\pgfpathlineto{\pgfqpoint{2.546777in}{0.650520in}}%
\pgfpathlineto{\pgfqpoint{2.553095in}{0.651176in}}%
\pgfpathlineto{\pgfqpoint{2.562412in}{0.650688in}}%
\pgfpathlineto{\pgfqpoint{2.569747in}{0.648115in}}%
\pgfpathlineto{\pgfqpoint{2.579122in}{0.649882in}}%
\pgfpathlineto{\pgfqpoint{2.588564in}{0.648342in}}%
\pgfpathlineto{\pgfqpoint{2.597083in}{0.644789in}}%
\pgfpathlineto{\pgfqpoint{2.604065in}{0.643161in}}%
\pgfpathlineto{\pgfqpoint{2.615498in}{0.641619in}}%
\pgfpathlineto{\pgfqpoint{2.621200in}{0.640423in}}%
\pgfpathlineto{\pgfqpoint{2.626797in}{0.641961in}}%
\pgfpathlineto{\pgfqpoint{2.632447in}{0.647600in}}%
\pgfpathlineto{\pgfqpoint{2.638632in}{0.649141in}}%
\pgfpathlineto{\pgfqpoint{2.643412in}{0.650977in}}%
\pgfpathlineto{\pgfqpoint{2.646755in}{0.649434in}}%
\pgfpathlineto{\pgfqpoint{2.651096in}{0.649613in}}%
\pgfpathlineto{\pgfqpoint{2.664907in}{0.655017in}}%
\pgfpathlineto{\pgfqpoint{2.668102in}{0.655109in}}%
\pgfpathlineto{\pgfqpoint{2.669038in}{0.653631in}}%
\pgfpathlineto{\pgfqpoint{2.674148in}{0.646473in}}%
\pgfpathlineto{\pgfqpoint{2.682954in}{0.644930in}}%
\pgfpathlineto{\pgfqpoint{2.686312in}{0.644032in}}%
\pgfpathlineto{\pgfqpoint{2.709732in}{0.650908in}}%
\pgfpathlineto{\pgfqpoint{2.716456in}{0.650104in}}%
\pgfpathlineto{\pgfqpoint{2.720935in}{0.651647in}}%
\pgfpathlineto{\pgfqpoint{2.729737in}{0.654439in}}%
\pgfpathlineto{\pgfqpoint{2.733538in}{0.654660in}}%
\pgfpathlineto{\pgfqpoint{2.739288in}{0.653120in}}%
\pgfpathlineto{\pgfqpoint{2.741509in}{0.653044in}}%
\pgfpathlineto{\pgfqpoint{2.744159in}{0.652366in}}%
\pgfpathlineto{\pgfqpoint{2.754312in}{0.648879in}}%
\pgfpathlineto{\pgfqpoint{2.763300in}{0.653508in}}%
\pgfpathlineto{\pgfqpoint{2.769876in}{0.654258in}}%
\pgfpathlineto{\pgfqpoint{2.775841in}{0.651826in}}%
\pgfpathlineto{\pgfqpoint{2.792852in}{0.653362in}}%
\pgfpathlineto{\pgfqpoint{2.796271in}{0.669068in}}%
\pgfpathlineto{\pgfqpoint{2.798239in}{0.675582in}}%
\pgfpathlineto{\pgfqpoint{2.799151in}{0.675028in}}%
\pgfpathlineto{\pgfqpoint{2.803931in}{0.673854in}}%
\pgfpathlineto{\pgfqpoint{2.806386in}{0.673388in}}%
\pgfpathlineto{\pgfqpoint{2.820789in}{0.669789in}}%
\pgfpathlineto{\pgfqpoint{2.832418in}{0.672377in}}%
\pgfpathlineto{\pgfqpoint{2.835422in}{0.670576in}}%
\pgfpathlineto{\pgfqpoint{2.836411in}{0.668096in}}%
\pgfpathlineto{\pgfqpoint{2.842996in}{0.646492in}}%
\pgfpathlineto{\pgfqpoint{2.850480in}{0.644379in}}%
\pgfpathlineto{\pgfqpoint{2.854128in}{0.645917in}}%
\pgfpathlineto{\pgfqpoint{2.857466in}{0.647578in}}%
\pgfpathlineto{\pgfqpoint{2.860327in}{0.648176in}}%
\pgfpathlineto{\pgfqpoint{2.863775in}{0.648219in}}%
\pgfpathlineto{\pgfqpoint{2.867696in}{0.646831in}}%
\pgfpathlineto{\pgfqpoint{2.871540in}{0.646587in}}%
\pgfpathlineto{\pgfqpoint{2.875366in}{0.646352in}}%
\pgfpathlineto{\pgfqpoint{2.895528in}{0.649020in}}%
\pgfpathlineto{\pgfqpoint{2.906780in}{0.642372in}}%
\pgfpathlineto{\pgfqpoint{2.911546in}{0.645746in}}%
\pgfpathlineto{\pgfqpoint{2.916011in}{0.647501in}}%
\pgfpathlineto{\pgfqpoint{2.931675in}{0.643445in}}%
\pgfpathlineto{\pgfqpoint{2.934679in}{0.644987in}}%
\pgfpathlineto{\pgfqpoint{2.948901in}{0.648518in}}%
\pgfpathlineto{\pgfqpoint{2.954899in}{0.644695in}}%
\pgfpathlineto{\pgfqpoint{2.960769in}{0.642813in}}%
\pgfpathlineto{\pgfqpoint{2.968582in}{0.647295in}}%
\pgfpathlineto{\pgfqpoint{2.973128in}{0.648827in}}%
\pgfpathlineto{\pgfqpoint{2.980974in}{0.650369in}}%
\pgfpathlineto{\pgfqpoint{2.989361in}{0.651780in}}%
\pgfpathlineto{\pgfqpoint{2.994872in}{0.649679in}}%
\pgfpathlineto{\pgfqpoint{2.998387in}{0.650030in}}%
\pgfpathlineto{\pgfqpoint{3.008100in}{0.654047in}}%
\pgfpathlineto{\pgfqpoint{3.017389in}{0.651626in}}%
\pgfpathlineto{\pgfqpoint{3.020627in}{0.651459in}}%
\pgfpathlineto{\pgfqpoint{3.024653in}{0.647656in}}%
\pgfpathlineto{\pgfqpoint{3.028153in}{0.646421in}}%
\pgfpathlineto{\pgfqpoint{3.039457in}{0.649576in}}%
\pgfpathlineto{\pgfqpoint{3.042662in}{0.649862in}}%
\pgfpathlineto{\pgfqpoint{3.051530in}{0.648278in}}%
\pgfpathlineto{\pgfqpoint{3.057012in}{0.649174in}}%
\pgfpathlineto{\pgfqpoint{3.060642in}{0.647631in}}%
\pgfpathlineto{\pgfqpoint{3.065766in}{0.644904in}}%
\pgfpathlineto{\pgfqpoint{3.071554in}{0.645948in}}%
\pgfpathlineto{\pgfqpoint{3.082729in}{0.638933in}}%
\pgfpathlineto{\pgfqpoint{3.087744in}{0.637441in}}%
\pgfpathlineto{\pgfqpoint{3.095027in}{0.640466in}}%
\pgfpathlineto{\pgfqpoint{3.101947in}{0.642008in}}%
\pgfpathlineto{\pgfqpoint{3.107004in}{0.642676in}}%
\pgfpathlineto{\pgfqpoint{3.112845in}{0.644217in}}%
\pgfpathlineto{\pgfqpoint{3.121493in}{0.650003in}}%
\pgfpathlineto{\pgfqpoint{3.127750in}{0.671895in}}%
\pgfpathlineto{\pgfqpoint{3.132788in}{0.670352in}}%
\pgfpathlineto{\pgfqpoint{3.135405in}{0.669910in}}%
\pgfpathlineto{\pgfqpoint{3.142989in}{0.675193in}}%
\pgfpathlineto{\pgfqpoint{3.163839in}{0.705298in}}%
\pgfpathlineto{\pgfqpoint{3.164966in}{0.702590in}}%
\pgfpathlineto{\pgfqpoint{3.169236in}{0.690508in}}%
\pgfpathlineto{\pgfqpoint{3.170119in}{0.691893in}}%
\pgfpathlineto{\pgfqpoint{3.176046in}{0.705019in}}%
\pgfpathlineto{\pgfqpoint{3.180836in}{0.728616in}}%
\pgfpathlineto{\pgfqpoint{3.188195in}{0.762171in}}%
\pgfpathlineto{\pgfqpoint{3.197465in}{0.804450in}}%
\pgfpathlineto{\pgfqpoint{3.203831in}{0.825726in}}%
\pgfpathlineto{\pgfqpoint{3.218559in}{0.855009in}}%
\pgfpathlineto{\pgfqpoint{3.220798in}{0.854397in}}%
\pgfpathlineto{\pgfqpoint{3.225507in}{0.850096in}}%
\pgfpathlineto{\pgfqpoint{3.230288in}{0.845053in}}%
\pgfpathlineto{\pgfqpoint{3.231248in}{0.843717in}}%
\pgfpathlineto{\pgfqpoint{3.231548in}{0.845881in}}%
\pgfpathlineto{\pgfqpoint{3.232828in}{0.890024in}}%
\pgfpathlineto{\pgfqpoint{3.238774in}{1.082387in}}%
\pgfpathlineto{\pgfqpoint{3.239524in}{1.082261in}}%
\pgfpathlineto{\pgfqpoint{3.238817in}{1.082146in}}%
\pgfpathlineto{\pgfqpoint{3.239925in}{1.081358in}}%
\pgfpathlineto{\pgfqpoint{3.240660in}{1.072661in}}%
\pgfpathlineto{\pgfqpoint{3.242876in}{1.051305in}}%
\pgfpathlineto{\pgfqpoint{3.243387in}{1.057458in}}%
\pgfpathlineto{\pgfqpoint{3.246797in}{1.099997in}}%
\pgfpathlineto{\pgfqpoint{3.247036in}{1.098873in}}%
\pgfpathlineto{\pgfqpoint{3.249061in}{1.090832in}}%
\pgfpathlineto{\pgfqpoint{3.269988in}{1.005808in}}%
\pgfpathlineto{\pgfqpoint{3.275332in}{0.968263in}}%
\pgfpathlineto{\pgfqpoint{3.278718in}{0.860943in}}%
\pgfpathlineto{\pgfqpoint{3.281612in}{0.819170in}}%
\pgfpathlineto{\pgfqpoint{3.283417in}{0.817720in}}%
\pgfpathlineto{\pgfqpoint{3.285017in}{0.816230in}}%
\pgfpathlineto{\pgfqpoint{3.286683in}{0.778109in}}%
\pgfpathlineto{\pgfqpoint{3.292261in}{0.651008in}}%
\pgfpathlineto{\pgfqpoint{3.305413in}{0.649545in}}%
\pgfpathlineto{\pgfqpoint{3.310853in}{0.644292in}}%
\pgfpathlineto{\pgfqpoint{3.317491in}{0.646904in}}%
\pgfpathlineto{\pgfqpoint{3.333829in}{0.643078in}}%
\pgfpathlineto{\pgfqpoint{3.340457in}{0.650047in}}%
\pgfpathlineto{\pgfqpoint{3.345357in}{0.648506in}}%
\pgfpathlineto{\pgfqpoint{3.348156in}{0.648876in}}%
\pgfpathlineto{\pgfqpoint{3.353595in}{0.649405in}}%
\pgfpathlineto{\pgfqpoint{3.358934in}{0.647669in}}%
\pgfpathlineto{\pgfqpoint{3.364928in}{0.648807in}}%
\pgfpathlineto{\pgfqpoint{3.371136in}{0.650350in}}%
\pgfpathlineto{\pgfqpoint{3.376174in}{0.651036in}}%
\pgfpathlineto{\pgfqpoint{3.377660in}{0.649494in}}%
\pgfpathlineto{\pgfqpoint{3.384785in}{0.641778in}}%
\pgfpathlineto{\pgfqpoint{3.392493in}{0.643317in}}%
\pgfpathlineto{\pgfqpoint{3.398023in}{0.646952in}}%
\pgfpathlineto{\pgfqpoint{3.405234in}{0.648493in}}%
\pgfpathlineto{\pgfqpoint{3.410354in}{0.649229in}}%
\pgfpathlineto{\pgfqpoint{3.419213in}{0.650756in}}%
\pgfpathlineto{\pgfqpoint{3.426787in}{0.672511in}}%
\pgfpathlineto{\pgfqpoint{3.435736in}{0.670973in}}%
\pgfpathlineto{\pgfqpoint{3.438979in}{0.668800in}}%
\pgfpathlineto{\pgfqpoint{3.443220in}{0.669095in}}%
\pgfpathlineto{\pgfqpoint{3.449586in}{0.667779in}}%
\pgfpathlineto{\pgfqpoint{3.456128in}{0.672100in}}%
\pgfpathlineto{\pgfqpoint{3.461592in}{0.671199in}}%
\pgfpathlineto{\pgfqpoint{3.462447in}{0.669290in}}%
\pgfpathlineto{\pgfqpoint{3.469333in}{0.649696in}}%
\pgfpathlineto{\pgfqpoint{3.473884in}{0.651238in}}%
\pgfpathlineto{\pgfqpoint{3.478937in}{0.651943in}}%
\pgfpathlineto{\pgfqpoint{3.493321in}{0.650402in}}%
\pgfpathlineto{\pgfqpoint{3.497323in}{0.648597in}}%
\pgfpathlineto{\pgfqpoint{3.504343in}{0.650244in}}%
\pgfpathlineto{\pgfqpoint{3.515294in}{0.649708in}}%
\pgfpathlineto{\pgfqpoint{3.532377in}{0.648165in}}%
\pgfpathlineto{\pgfqpoint{3.536689in}{0.647822in}}%
\pgfpathlineto{\pgfqpoint{3.539492in}{0.647453in}}%
\pgfpathlineto{\pgfqpoint{3.546861in}{0.645681in}}%
\pgfpathlineto{\pgfqpoint{3.557391in}{0.647220in}}%
\pgfpathlineto{\pgfqpoint{3.563466in}{0.651449in}}%
\pgfpathlineto{\pgfqpoint{3.567984in}{0.649907in}}%
\pgfpathlineto{\pgfqpoint{3.573323in}{0.646973in}}%
\pgfpathlineto{\pgfqpoint{3.584756in}{0.648512in}}%
\pgfpathlineto{\pgfqpoint{3.592273in}{0.670243in}}%
\pgfpathlineto{\pgfqpoint{3.601246in}{0.668706in}}%
\pgfpathlineto{\pgfqpoint{3.605334in}{0.665225in}}%
\pgfpathlineto{\pgfqpoint{3.610755in}{0.666767in}}%
\pgfpathlineto{\pgfqpoint{3.615779in}{0.667433in}}%
\pgfpathlineto{\pgfqpoint{3.621772in}{0.665894in}}%
\pgfpathlineto{\pgfqpoint{3.626032in}{0.663976in}}%
\pgfpathlineto{\pgfqpoint{3.627918in}{0.662430in}}%
\pgfpathlineto{\pgfqpoint{3.632775in}{0.643183in}}%
\pgfpathlineto{\pgfqpoint{3.634475in}{0.643917in}}%
\pgfpathlineto{\pgfqpoint{3.639356in}{0.643915in}}%
\pgfpathlineto{\pgfqpoint{3.642670in}{0.642373in}}%
\pgfpathlineto{\pgfqpoint{3.652351in}{0.639483in}}%
\pgfpathlineto{\pgfqpoint{3.670426in}{0.637941in}}%
\pgfpathlineto{\pgfqpoint{3.671902in}{0.638946in}}%
\pgfpathlineto{\pgfqpoint{3.677413in}{0.644304in}}%
\pgfpathlineto{\pgfqpoint{3.685752in}{0.645847in}}%
\pgfpathlineto{\pgfqpoint{3.690871in}{0.646585in}}%
\pgfpathlineto{\pgfqpoint{3.696788in}{0.648126in}}%
\pgfpathlineto{\pgfqpoint{3.701865in}{0.648842in}}%
\pgfpathlineto{\pgfqpoint{3.707973in}{0.650384in}}%
\pgfpathlineto{\pgfqpoint{3.713040in}{0.651096in}}%
\pgfpathlineto{\pgfqpoint{3.714797in}{0.649556in}}%
\pgfpathlineto{\pgfqpoint{3.718923in}{0.646227in}}%
\pgfpathlineto{\pgfqpoint{3.723231in}{0.647258in}}%
\pgfpathlineto{\pgfqpoint{3.749908in}{0.648799in}}%
\pgfpathlineto{\pgfqpoint{3.756804in}{0.669964in}}%
\pgfpathlineto{\pgfqpoint{3.760982in}{0.668423in}}%
\pgfpathlineto{\pgfqpoint{3.766197in}{0.666770in}}%
\pgfpathlineto{\pgfqpoint{3.777057in}{0.668312in}}%
\pgfpathlineto{\pgfqpoint{3.781518in}{0.668737in}}%
\pgfpathlineto{\pgfqpoint{3.786451in}{0.668089in}}%
\pgfpathlineto{\pgfqpoint{3.791747in}{0.672414in}}%
\pgfpathlineto{\pgfqpoint{3.792406in}{0.672116in}}%
\pgfpathlineto{\pgfqpoint{3.792741in}{0.671195in}}%
\pgfpathlineto{\pgfqpoint{3.800410in}{0.650920in}}%
\pgfpathlineto{\pgfqpoint{3.806179in}{0.652461in}}%
\pgfpathlineto{\pgfqpoint{3.812397in}{0.654772in}}%
\pgfpathlineto{\pgfqpoint{3.829981in}{0.653124in}}%
\pgfpathlineto{\pgfqpoint{3.837403in}{0.647914in}}%
\pgfpathlineto{\pgfqpoint{3.894706in}{0.646372in}}%
\pgfpathlineto{\pgfqpoint{3.900126in}{0.642509in}}%
\pgfpathlineto{\pgfqpoint{3.904486in}{0.642154in}}%
\pgfpathlineto{\pgfqpoint{3.911449in}{0.644964in}}%
\pgfpathlineto{\pgfqpoint{3.920934in}{0.646503in}}%
\pgfpathlineto{\pgfqpoint{3.926120in}{0.669190in}}%
\pgfpathlineto{\pgfqpoint{3.928369in}{0.668118in}}%
\pgfpathlineto{\pgfqpoint{3.932834in}{0.667692in}}%
\pgfpathlineto{\pgfqpoint{3.940132in}{0.670727in}}%
\pgfpathlineto{\pgfqpoint{3.958255in}{0.669187in}}%
\pgfpathlineto{\pgfqpoint{3.964803in}{0.660859in}}%
\pgfpathlineto{\pgfqpoint{3.969110in}{0.643048in}}%
\pgfpathlineto{\pgfqpoint{3.969951in}{0.643449in}}%
\pgfpathlineto{\pgfqpoint{3.974397in}{0.643864in}}%
\pgfpathlineto{\pgfqpoint{3.978867in}{0.642323in}}%
\pgfpathlineto{\pgfqpoint{3.983934in}{0.641636in}}%
\pgfpathlineto{\pgfqpoint{3.995763in}{0.640095in}}%
\pgfpathlineto{\pgfqpoint{4.000553in}{0.639539in}}%
\pgfpathlineto{\pgfqpoint{4.004135in}{0.641080in}}%
\pgfpathlineto{\pgfqpoint{4.009149in}{0.645151in}}%
\pgfpathlineto{\pgfqpoint{4.013061in}{0.646850in}}%
\pgfpathlineto{\pgfqpoint{4.017383in}{0.646494in}}%
\pgfpathlineto{\pgfqpoint{4.021791in}{0.648036in}}%
\pgfpathlineto{\pgfqpoint{4.026881in}{0.648759in}}%
\pgfpathlineto{\pgfqpoint{4.032775in}{0.650301in}}%
\pgfpathlineto{\pgfqpoint{4.042636in}{0.653306in}}%
\pgfpathlineto{\pgfqpoint{4.050736in}{0.651765in}}%
\pgfpathlineto{\pgfqpoint{4.058181in}{0.646551in}}%
\pgfpathlineto{\pgfqpoint{4.063797in}{0.647484in}}%
\pgfpathlineto{\pgfqpoint{4.069265in}{0.646577in}}%
\pgfpathlineto{\pgfqpoint{4.074772in}{0.647503in}}%
\pgfpathlineto{\pgfqpoint{4.082227in}{0.645670in}}%
\pgfpathlineto{\pgfqpoint{4.085584in}{0.647200in}}%
\pgfpathlineto{\pgfqpoint{4.088717in}{0.662028in}}%
\pgfpathlineto{\pgfqpoint{4.090937in}{0.670515in}}%
\pgfpathlineto{\pgfqpoint{4.091926in}{0.670045in}}%
\pgfpathlineto{\pgfqpoint{4.097580in}{0.669066in}}%
\pgfpathlineto{\pgfqpoint{4.103249in}{0.670054in}}%
\pgfpathlineto{\pgfqpoint{4.118641in}{0.664496in}}%
\pgfpathlineto{\pgfqpoint{4.125933in}{0.662954in}}%
\pgfpathlineto{\pgfqpoint{4.128971in}{0.659809in}}%
\pgfpathlineto{\pgfqpoint{4.135685in}{0.637273in}}%
\pgfpathlineto{\pgfqpoint{4.139367in}{0.638814in}}%
\pgfpathlineto{\pgfqpoint{4.149114in}{0.641708in}}%
\pgfpathlineto{\pgfqpoint{4.161393in}{0.643251in}}%
\pgfpathlineto{\pgfqpoint{4.171197in}{0.646171in}}%
\pgfpathlineto{\pgfqpoint{4.178241in}{0.647714in}}%
\pgfpathlineto{\pgfqpoint{4.181193in}{0.647400in}}%
\pgfpathlineto{\pgfqpoint{4.188523in}{0.645647in}}%
\pgfpathlineto{\pgfqpoint{4.199297in}{0.647190in}}%
\pgfpathlineto{\pgfqpoint{4.203266in}{0.647359in}}%
\pgfpathlineto{\pgfqpoint{4.211938in}{0.644973in}}%
\pgfpathlineto{\pgfqpoint{4.218018in}{0.646169in}}%
\pgfpathlineto{\pgfqpoint{4.227855in}{0.643229in}}%
\pgfpathlineto{\pgfqpoint{4.232316in}{0.646784in}}%
\pgfpathlineto{\pgfqpoint{4.235172in}{0.647133in}}%
\pgfpathlineto{\pgfqpoint{4.243114in}{0.649176in}}%
\pgfpathlineto{\pgfqpoint{4.255296in}{0.650791in}}%
\pgfpathlineto{\pgfqpoint{4.256576in}{0.655665in}}%
\pgfpathlineto{\pgfqpoint{4.262078in}{0.674177in}}%
\pgfpathlineto{\pgfqpoint{4.272517in}{0.672636in}}%
\pgfpathlineto{\pgfqpoint{4.276577in}{0.669629in}}%
\pgfpathlineto{\pgfqpoint{4.281352in}{0.669077in}}%
\pgfpathlineto{\pgfqpoint{4.284103in}{0.668682in}}%
\pgfpathlineto{\pgfqpoint{4.289767in}{0.667699in}}%
\pgfpathlineto{\pgfqpoint{4.295240in}{0.668607in}}%
\pgfpathlineto{\pgfqpoint{4.299022in}{0.665105in}}%
\pgfpathlineto{\pgfqpoint{4.303898in}{0.645564in}}%
\pgfpathlineto{\pgfqpoint{4.304911in}{0.646047in}}%
\pgfpathlineto{\pgfqpoint{4.311148in}{0.647302in}}%
\pgfpathlineto{\pgfqpoint{4.319944in}{0.648870in}}%
\pgfpathlineto{\pgfqpoint{4.324424in}{0.649289in}}%
\pgfpathlineto{\pgfqpoint{4.329668in}{0.648506in}}%
\pgfpathlineto{\pgfqpoint{4.344167in}{0.646966in}}%
\pgfpathlineto{\pgfqpoint{4.348660in}{0.644812in}}%
\pgfpathlineto{\pgfqpoint{4.353231in}{0.645275in}}%
\pgfpathlineto{\pgfqpoint{4.355299in}{0.645994in}}%
\pgfpathlineto{\pgfqpoint{4.360657in}{0.647851in}}%
\pgfpathlineto{\pgfqpoint{4.365036in}{0.647490in}}%
\pgfpathlineto{\pgfqpoint{4.368599in}{0.649033in}}%
\pgfpathlineto{\pgfqpoint{4.373475in}{0.649657in}}%
\pgfpathlineto{\pgfqpoint{4.378948in}{0.648749in}}%
\pgfpathlineto{\pgfqpoint{4.384645in}{0.649762in}}%
\pgfpathlineto{\pgfqpoint{4.396135in}{0.651304in}}%
\pgfpathlineto{\pgfqpoint{4.398084in}{0.650518in}}%
\pgfpathlineto{\pgfqpoint{4.402291in}{0.647964in}}%
\pgfpathlineto{\pgfqpoint{4.410395in}{0.650083in}}%
\pgfpathlineto{\pgfqpoint{4.421351in}{0.651511in}}%
\pgfpathlineto{\pgfqpoint{4.429565in}{0.674279in}}%
\pgfpathlineto{\pgfqpoint{4.432893in}{0.674146in}}%
\pgfpathlineto{\pgfqpoint{4.435329in}{0.674702in}}%
\pgfpathlineto{\pgfqpoint{4.438486in}{0.674957in}}%
\pgfpathlineto{\pgfqpoint{4.450625in}{0.670898in}}%
\pgfpathlineto{\pgfqpoint{4.456695in}{0.673747in}}%
\pgfpathlineto{\pgfqpoint{4.460989in}{0.673427in}}%
\pgfpathlineto{\pgfqpoint{4.464451in}{0.671897in}}%
\pgfpathlineto{\pgfqpoint{4.471643in}{0.652045in}}%
\pgfpathlineto{\pgfqpoint{4.486214in}{0.653586in}}%
\pgfpathlineto{\pgfqpoint{4.490378in}{0.653826in}}%
\pgfpathlineto{\pgfqpoint{4.497580in}{0.648698in}}%
\pgfpathlineto{\pgfqpoint{4.499328in}{0.648902in}}%
\pgfpathlineto{\pgfqpoint{4.506314in}{0.650514in}}%
\pgfpathlineto{\pgfqpoint{4.511907in}{0.649563in}}%
\pgfpathlineto{\pgfqpoint{4.522657in}{0.652962in}}%
\pgfpathlineto{\pgfqpoint{4.526597in}{0.649384in}}%
\pgfpathlineto{\pgfqpoint{4.530169in}{0.648127in}}%
\pgfpathlineto{\pgfqpoint{4.534882in}{0.647596in}}%
\pgfpathlineto{\pgfqpoint{4.538407in}{0.647573in}}%
\pgfpathlineto{\pgfqpoint{4.543784in}{0.646742in}}%
\pgfpathlineto{\pgfqpoint{4.555423in}{0.645199in}}%
\pgfpathlineto{\pgfqpoint{4.559071in}{0.643851in}}%
\pgfpathlineto{\pgfqpoint{4.562424in}{0.642510in}}%
\pgfpathlineto{\pgfqpoint{4.566044in}{0.642066in}}%
\pgfpathlineto{\pgfqpoint{4.569310in}{0.644928in}}%
\pgfpathlineto{\pgfqpoint{4.572997in}{0.646667in}}%
\pgfpathlineto{\pgfqpoint{4.575881in}{0.648209in}}%
\pgfpathlineto{\pgfqpoint{4.600104in}{0.666299in}}%
\pgfpathlineto{\pgfqpoint{4.602816in}{0.667653in}}%
\pgfpathlineto{\pgfqpoint{4.613237in}{0.667584in}}%
\pgfpathlineto{\pgfqpoint{4.617922in}{0.668460in}}%
\pgfpathlineto{\pgfqpoint{4.622005in}{0.667579in}}%
\pgfpathlineto{\pgfqpoint{4.633949in}{0.668025in}}%
\pgfpathlineto{\pgfqpoint{4.634808in}{0.667244in}}%
\pgfpathlineto{\pgfqpoint{4.635434in}{0.668558in}}%
\pgfpathlineto{\pgfqpoint{4.639794in}{0.687655in}}%
\pgfpathlineto{\pgfqpoint{4.640654in}{0.689185in}}%
\pgfpathlineto{\pgfqpoint{4.641647in}{0.688626in}}%
\pgfpathlineto{\pgfqpoint{4.647063in}{0.687818in}}%
\pgfpathlineto{\pgfqpoint{4.651709in}{0.690817in}}%
\pgfpathlineto{\pgfqpoint{4.659584in}{0.687561in}}%
\pgfpathlineto{\pgfqpoint{4.664155in}{0.688175in}}%
\pgfpathlineto{\pgfqpoint{4.665172in}{0.686693in}}%
\pgfpathlineto{\pgfqpoint{4.665960in}{0.688298in}}%
\pgfpathlineto{\pgfqpoint{4.673004in}{0.699271in}}%
\pgfpathlineto{\pgfqpoint{4.674924in}{0.701696in}}%
\pgfpathlineto{\pgfqpoint{4.676180in}{0.746713in}}%
\pgfpathlineto{\pgfqpoint{4.680435in}{0.940567in}}%
\pgfpathlineto{\pgfqpoint{4.681108in}{0.922216in}}%
\pgfpathlineto{\pgfqpoint{4.683439in}{0.882755in}}%
\pgfpathlineto{\pgfqpoint{4.684795in}{0.881871in}}%
\pgfpathlineto{\pgfqpoint{4.684924in}{0.883246in}}%
\pgfpathlineto{\pgfqpoint{4.688329in}{0.965753in}}%
\pgfpathlineto{\pgfqpoint{4.691686in}{1.010724in}}%
\pgfpathlineto{\pgfqpoint{4.693248in}{1.035005in}}%
\pgfpathlineto{\pgfqpoint{4.697102in}{1.149840in}}%
\pgfpathlineto{\pgfqpoint{4.700951in}{1.286982in}}%
\pgfpathlineto{\pgfqpoint{4.702957in}{1.285947in}}%
\pgfpathlineto{\pgfqpoint{4.706873in}{1.283052in}}%
\pgfpathlineto{\pgfqpoint{4.708855in}{1.280698in}}%
\pgfpathlineto{\pgfqpoint{4.715144in}{1.265153in}}%
\pgfpathlineto{\pgfqpoint{4.717890in}{1.260997in}}%
\pgfpathlineto{\pgfqpoint{4.719132in}{1.216301in}}%
\pgfpathlineto{\pgfqpoint{4.723736in}{1.058160in}}%
\pgfpathlineto{\pgfqpoint{4.727914in}{1.056643in}}%
\pgfpathlineto{\pgfqpoint{4.732322in}{0.962901in}}%
\pgfpathlineto{\pgfqpoint{4.737003in}{0.875899in}}%
\pgfpathlineto{\pgfqpoint{4.739467in}{0.810953in}}%
\pgfpathlineto{\pgfqpoint{4.744433in}{0.637273in}}%
\pgfpathlineto{\pgfqpoint{4.758631in}{0.638804in}}%
\pgfpathlineto{\pgfqpoint{4.759849in}{0.687554in}}%
\pgfpathlineto{\pgfqpoint{4.765480in}{0.904082in}}%
\pgfpathlineto{\pgfqpoint{4.767753in}{1.001283in}}%
\pgfpathlineto{\pgfqpoint{4.771702in}{1.136767in}}%
\pgfpathlineto{\pgfqpoint{4.774964in}{1.138053in}}%
\pgfpathlineto{\pgfqpoint{4.777763in}{1.139592in}}%
\pgfpathlineto{\pgfqpoint{4.786961in}{1.147344in}}%
\pgfpathlineto{\pgfqpoint{4.793102in}{1.148885in}}%
\pgfpathlineto{\pgfqpoint{4.798661in}{1.151452in}}%
\pgfpathlineto{\pgfqpoint{4.801503in}{1.152242in}}%
\pgfpathlineto{\pgfqpoint{4.801641in}{1.150492in}}%
\pgfpathlineto{\pgfqpoint{4.802907in}{1.098758in}}%
\pgfpathlineto{\pgfqpoint{4.808461in}{0.887378in}}%
\pgfpathlineto{\pgfqpoint{4.810767in}{0.787639in}}%
\pgfpathlineto{\pgfqpoint{4.814736in}{0.652845in}}%
\pgfpathlineto{\pgfqpoint{4.819292in}{0.649138in}}%
\pgfpathlineto{\pgfqpoint{4.830185in}{0.640771in}}%
\pgfpathlineto{\pgfqpoint{4.834894in}{0.641289in}}%
\pgfpathlineto{\pgfqpoint{4.842473in}{0.639397in}}%
\pgfpathlineto{\pgfqpoint{4.898682in}{0.640940in}}%
\pgfpathlineto{\pgfqpoint{4.904002in}{0.641745in}}%
\pgfpathlineto{\pgfqpoint{4.910082in}{0.643287in}}%
\pgfpathlineto{\pgfqpoint{4.915197in}{0.644023in}}%
\pgfpathlineto{\pgfqpoint{4.919194in}{0.642481in}}%
\pgfpathlineto{\pgfqpoint{4.924266in}{0.641794in}}%
\pgfpathlineto{\pgfqpoint{4.941888in}{0.640252in}}%
\pgfpathlineto{\pgfqpoint{4.944137in}{0.639845in}}%
\pgfpathlineto{\pgfqpoint{4.944452in}{0.640734in}}%
\pgfpathlineto{\pgfqpoint{4.948316in}{0.659330in}}%
\pgfpathlineto{\pgfqpoint{4.949830in}{0.664659in}}%
\pgfpathlineto{\pgfqpoint{4.950871in}{0.664165in}}%
\pgfpathlineto{\pgfqpoint{4.958182in}{0.662381in}}%
\pgfpathlineto{\pgfqpoint{4.975432in}{0.663921in}}%
\pgfpathlineto{\pgfqpoint{4.980499in}{0.667858in}}%
\pgfpathlineto{\pgfqpoint{4.987791in}{0.666319in}}%
\pgfpathlineto{\pgfqpoint{4.994640in}{0.646075in}}%
\pgfpathlineto{\pgfqpoint{5.000437in}{0.647618in}}%
\pgfpathlineto{\pgfqpoint{5.010223in}{0.650558in}}%
\pgfpathlineto{\pgfqpoint{5.017147in}{0.649023in}}%
\pgfpathlineto{\pgfqpoint{5.022787in}{0.641772in}}%
\pgfpathlineto{\pgfqpoint{5.043141in}{0.640230in}}%
\pgfpathlineto{\pgfqpoint{5.048519in}{0.639384in}}%
\pgfpathlineto{\pgfqpoint{5.072297in}{0.637842in}}%
\pgfpathlineto{\pgfqpoint{5.077115in}{0.637273in}}%
\pgfpathlineto{\pgfqpoint{5.086265in}{0.638815in}}%
\pgfpathlineto{\pgfqpoint{5.091113in}{0.639397in}}%
\pgfpathlineto{\pgfqpoint{5.096180in}{0.640939in}}%
\pgfpathlineto{\pgfqpoint{5.101676in}{0.642769in}}%
\pgfpathlineto{\pgfqpoint{5.106562in}{0.644306in}}%
\pgfpathlineto{\pgfqpoint{5.114509in}{0.667807in}}%
\pgfpathlineto{\pgfqpoint{5.122756in}{0.669348in}}%
\pgfpathlineto{\pgfqpoint{5.127298in}{0.672840in}}%
\pgfpathlineto{\pgfqpoint{5.131023in}{0.672785in}}%
\pgfpathlineto{\pgfqpoint{5.135904in}{0.673408in}}%
\pgfpathlineto{\pgfqpoint{5.139175in}{0.671867in}}%
\pgfpathlineto{\pgfqpoint{5.142556in}{0.671464in}}%
\pgfpathlineto{\pgfqpoint{5.149166in}{0.674834in}}%
\pgfpathlineto{\pgfqpoint{5.149447in}{0.674217in}}%
\pgfpathlineto{\pgfqpoint{5.152260in}{0.662021in}}%
\pgfpathlineto{\pgfqpoint{5.154724in}{0.653123in}}%
\pgfpathlineto{\pgfqpoint{5.155794in}{0.653614in}}%
\pgfpathlineto{\pgfqpoint{5.167413in}{0.652075in}}%
\pgfpathlineto{\pgfqpoint{5.170976in}{0.648980in}}%
\pgfpathlineto{\pgfqpoint{5.174558in}{0.647287in}}%
\pgfpathlineto{\pgfqpoint{5.184405in}{0.644291in}}%
\pgfpathlineto{\pgfqpoint{5.187638in}{0.645892in}}%
\pgfpathlineto{\pgfqpoint{5.188636in}{0.646834in}}%
\pgfpathlineto{\pgfqpoint{5.188636in}{0.646834in}}%
\pgfusepath{stroke}%
\end{pgfscope}%
\begin{pgfscope}%
\pgfpathrectangle{\pgfqpoint{0.750000in}{0.500000in}}{\pgfqpoint{4.650000in}{3.020000in}}%
\pgfusepath{clip}%
\pgfsetrectcap%
\pgfsetroundjoin%
\pgfsetlinewidth{1.505625pt}%
\definecolor{currentstroke}{rgb}{0.000000,0.500000,0.000000}%
\pgfsetstrokecolor{currentstroke}%
\pgfsetdash{}{0pt}%
\pgfpathmoveto{\pgfqpoint{0.961364in}{1.368563in}}%
\pgfpathlineto{\pgfqpoint{0.963188in}{1.304196in}}%
\pgfpathlineto{\pgfqpoint{0.965657in}{1.014953in}}%
\pgfpathlineto{\pgfqpoint{0.969286in}{0.658183in}}%
\pgfpathlineto{\pgfqpoint{0.988695in}{0.656642in}}%
\pgfpathlineto{\pgfqpoint{0.995935in}{0.647827in}}%
\pgfpathlineto{\pgfqpoint{1.017349in}{0.649369in}}%
\pgfpathlineto{\pgfqpoint{1.021819in}{0.657130in}}%
\pgfpathlineto{\pgfqpoint{1.050430in}{0.657219in}}%
\pgfpathlineto{\pgfqpoint{1.060497in}{0.655679in}}%
\pgfpathlineto{\pgfqpoint{1.064718in}{0.646828in}}%
\pgfpathlineto{\pgfqpoint{1.066557in}{0.648910in}}%
\pgfpathlineto{\pgfqpoint{1.069050in}{0.653521in}}%
\pgfpathlineto{\pgfqpoint{1.072460in}{0.656964in}}%
\pgfpathlineto{\pgfqpoint{1.090808in}{0.661672in}}%
\pgfpathlineto{\pgfqpoint{1.095158in}{0.661897in}}%
\pgfpathlineto{\pgfqpoint{1.112217in}{0.660354in}}%
\pgfpathlineto{\pgfqpoint{1.115407in}{0.658508in}}%
\pgfpathlineto{\pgfqpoint{1.115617in}{0.659137in}}%
\pgfpathlineto{\pgfqpoint{1.117069in}{0.674245in}}%
\pgfpathlineto{\pgfqpoint{1.120971in}{0.697073in}}%
\pgfpathlineto{\pgfqpoint{1.124873in}{0.696993in}}%
\pgfpathlineto{\pgfqpoint{1.132872in}{0.696956in}}%
\pgfpathlineto{\pgfqpoint{1.159210in}{0.695968in}}%
\pgfpathlineto{\pgfqpoint{1.162753in}{0.671063in}}%
\pgfpathlineto{\pgfqpoint{1.165098in}{0.662680in}}%
\pgfpathlineto{\pgfqpoint{1.188045in}{0.661139in}}%
\pgfpathlineto{\pgfqpoint{1.189664in}{0.657762in}}%
\pgfpathlineto{\pgfqpoint{1.194831in}{0.648504in}}%
\pgfpathlineto{\pgfqpoint{1.214249in}{0.648301in}}%
\pgfpathlineto{\pgfqpoint{1.223638in}{0.647657in}}%
\pgfpathlineto{\pgfqpoint{1.233538in}{0.648136in}}%
\pgfpathlineto{\pgfqpoint{1.240357in}{0.646847in}}%
\pgfpathlineto{\pgfqpoint{1.252172in}{0.648388in}}%
\pgfpathlineto{\pgfqpoint{1.256275in}{0.657960in}}%
\pgfpathlineto{\pgfqpoint{1.258352in}{0.658051in}}%
\pgfpathlineto{\pgfqpoint{1.280349in}{0.659584in}}%
\pgfpathlineto{\pgfqpoint{1.287317in}{0.698881in}}%
\pgfpathlineto{\pgfqpoint{1.318865in}{0.697338in}}%
\pgfpathlineto{\pgfqpoint{1.323869in}{0.695008in}}%
\pgfpathlineto{\pgfqpoint{1.327131in}{0.666809in}}%
\pgfpathlineto{\pgfqpoint{1.328626in}{0.647241in}}%
\pgfpathlineto{\pgfqpoint{1.330073in}{0.647366in}}%
\pgfpathlineto{\pgfqpoint{1.360298in}{0.647910in}}%
\pgfpathlineto{\pgfqpoint{1.366702in}{0.648318in}}%
\pgfpathlineto{\pgfqpoint{1.370976in}{0.657476in}}%
\pgfpathlineto{\pgfqpoint{1.373188in}{0.657845in}}%
\pgfpathlineto{\pgfqpoint{1.409707in}{0.656083in}}%
\pgfpathlineto{\pgfqpoint{1.413800in}{0.648515in}}%
\pgfpathlineto{\pgfqpoint{1.415930in}{0.647407in}}%
\pgfpathlineto{\pgfqpoint{1.445080in}{0.649124in}}%
\pgfpathlineto{\pgfqpoint{1.451155in}{0.700720in}}%
\pgfpathlineto{\pgfqpoint{1.478405in}{0.702262in}}%
\pgfpathlineto{\pgfqpoint{1.482827in}{0.701243in}}%
\pgfpathlineto{\pgfqpoint{1.488983in}{0.696107in}}%
\pgfpathlineto{\pgfqpoint{1.492374in}{0.667636in}}%
\pgfpathlineto{\pgfqpoint{1.494656in}{0.657656in}}%
\pgfpathlineto{\pgfqpoint{1.524304in}{0.657360in}}%
\pgfpathlineto{\pgfqpoint{1.527336in}{0.651882in}}%
\pgfpathlineto{\pgfqpoint{1.530732in}{0.647631in}}%
\pgfpathlineto{\pgfqpoint{1.538640in}{0.647384in}}%
\pgfpathlineto{\pgfqpoint{1.554319in}{0.646827in}}%
\pgfpathlineto{\pgfqpoint{1.607462in}{0.648369in}}%
\pgfpathlineto{\pgfqpoint{1.612290in}{0.651884in}}%
\pgfpathlineto{\pgfqpoint{1.618384in}{0.653383in}}%
\pgfpathlineto{\pgfqpoint{1.624573in}{0.697249in}}%
\pgfpathlineto{\pgfqpoint{1.629086in}{0.698790in}}%
\pgfpathlineto{\pgfqpoint{1.634402in}{0.699168in}}%
\pgfpathlineto{\pgfqpoint{1.647511in}{0.700710in}}%
\pgfpathlineto{\pgfqpoint{1.660658in}{0.700698in}}%
\pgfpathlineto{\pgfqpoint{1.661809in}{0.699168in}}%
\pgfpathlineto{\pgfqpoint{1.664746in}{0.678212in}}%
\pgfpathlineto{\pgfqpoint{1.668261in}{0.659717in}}%
\pgfpathlineto{\pgfqpoint{1.695143in}{0.658174in}}%
\pgfpathlineto{\pgfqpoint{1.697884in}{0.655282in}}%
\pgfpathlineto{\pgfqpoint{1.700540in}{0.653694in}}%
\pgfpathlineto{\pgfqpoint{1.706605in}{0.652011in}}%
\pgfpathlineto{\pgfqpoint{1.711309in}{0.647438in}}%
\pgfpathlineto{\pgfqpoint{1.715669in}{0.649006in}}%
\pgfpathlineto{\pgfqpoint{1.720607in}{0.655380in}}%
\pgfpathlineto{\pgfqpoint{1.759524in}{0.653841in}}%
\pgfpathlineto{\pgfqpoint{1.766535in}{0.645724in}}%
\pgfpathlineto{\pgfqpoint{1.784453in}{0.644183in}}%
\pgfpathlineto{\pgfqpoint{1.788889in}{0.643614in}}%
\pgfpathlineto{\pgfqpoint{1.793441in}{0.645141in}}%
\pgfpathlineto{\pgfqpoint{1.799343in}{0.702052in}}%
\pgfpathlineto{\pgfqpoint{1.809181in}{0.700509in}}%
\pgfpathlineto{\pgfqpoint{1.815690in}{0.700476in}}%
\pgfpathlineto{\pgfqpoint{1.823857in}{0.700238in}}%
\pgfpathlineto{\pgfqpoint{1.837186in}{0.698720in}}%
\pgfpathlineto{\pgfqpoint{1.840309in}{0.671736in}}%
\pgfpathlineto{\pgfqpoint{1.843657in}{0.648271in}}%
\pgfpathlineto{\pgfqpoint{1.864335in}{0.647277in}}%
\pgfpathlineto{\pgfqpoint{1.880009in}{0.647235in}}%
\pgfpathlineto{\pgfqpoint{1.890554in}{0.647737in}}%
\pgfpathlineto{\pgfqpoint{1.896069in}{0.649958in}}%
\pgfpathlineto{\pgfqpoint{1.902063in}{0.647736in}}%
\pgfpathlineto{\pgfqpoint{1.918591in}{0.648634in}}%
\pgfpathlineto{\pgfqpoint{1.921629in}{0.650954in}}%
\pgfpathlineto{\pgfqpoint{1.924957in}{0.674641in}}%
\pgfpathlineto{\pgfqpoint{1.927689in}{0.679179in}}%
\pgfpathlineto{\pgfqpoint{1.931218in}{0.677692in}}%
\pgfpathlineto{\pgfqpoint{1.931276in}{0.679706in}}%
\pgfpathlineto{\pgfqpoint{1.932126in}{0.838976in}}%
\pgfpathlineto{\pgfqpoint{1.935741in}{1.280681in}}%
\pgfpathlineto{\pgfqpoint{1.936510in}{1.315919in}}%
\pgfpathlineto{\pgfqpoint{1.937298in}{1.281746in}}%
\pgfpathlineto{\pgfqpoint{1.938329in}{1.249936in}}%
\pgfpathlineto{\pgfqpoint{1.939346in}{1.250112in}}%
\pgfpathlineto{\pgfqpoint{1.945932in}{1.252572in}}%
\pgfpathlineto{\pgfqpoint{1.948769in}{1.257368in}}%
\pgfpathlineto{\pgfqpoint{1.949003in}{1.255813in}}%
\pgfpathlineto{\pgfqpoint{1.949237in}{1.254420in}}%
\pgfpathlineto{\pgfqpoint{1.950135in}{1.258945in}}%
\pgfpathlineto{\pgfqpoint{1.951692in}{1.269093in}}%
\pgfpathlineto{\pgfqpoint{1.952852in}{1.323154in}}%
\pgfpathlineto{\pgfqpoint{1.955555in}{1.467954in}}%
\pgfpathlineto{\pgfqpoint{1.956338in}{1.463016in}}%
\pgfpathlineto{\pgfqpoint{1.958492in}{1.461017in}}%
\pgfpathlineto{\pgfqpoint{1.964696in}{1.462558in}}%
\pgfpathlineto{\pgfqpoint{1.966931in}{1.466904in}}%
\pgfpathlineto{\pgfqpoint{1.971339in}{1.472196in}}%
\pgfpathlineto{\pgfqpoint{1.974572in}{1.473983in}}%
\pgfpathlineto{\pgfqpoint{1.974844in}{1.471728in}}%
\pgfpathlineto{\pgfqpoint{1.975909in}{1.435431in}}%
\pgfpathlineto{\pgfqpoint{1.978865in}{1.315829in}}%
\pgfpathlineto{\pgfqpoint{1.980193in}{1.315972in}}%
\pgfpathlineto{\pgfqpoint{1.992065in}{1.314434in}}%
\pgfpathlineto{\pgfqpoint{1.992777in}{1.311292in}}%
\pgfpathlineto{\pgfqpoint{1.994954in}{1.285898in}}%
\pgfpathlineto{\pgfqpoint{1.996396in}{1.154499in}}%
\pgfpathlineto{\pgfqpoint{2.001922in}{0.637273in}}%
\pgfpathlineto{\pgfqpoint{2.014625in}{0.638787in}}%
\pgfpathlineto{\pgfqpoint{2.016459in}{0.707167in}}%
\pgfpathlineto{\pgfqpoint{2.019716in}{0.751076in}}%
\pgfpathlineto{\pgfqpoint{2.020246in}{0.750188in}}%
\pgfpathlineto{\pgfqpoint{2.028030in}{0.746308in}}%
\pgfpathlineto{\pgfqpoint{2.039669in}{0.744788in}}%
\pgfpathlineto{\pgfqpoint{2.043432in}{0.730866in}}%
\pgfpathlineto{\pgfqpoint{2.045294in}{0.729018in}}%
\pgfpathlineto{\pgfqpoint{2.045653in}{0.729749in}}%
\pgfpathlineto{\pgfqpoint{2.046460in}{0.734818in}}%
\pgfpathlineto{\pgfqpoint{2.047553in}{0.889573in}}%
\pgfpathlineto{\pgfqpoint{2.051336in}{1.283741in}}%
\pgfpathlineto{\pgfqpoint{2.052210in}{1.307293in}}%
\pgfpathlineto{\pgfqpoint{2.053303in}{1.305544in}}%
\pgfpathlineto{\pgfqpoint{2.059306in}{1.307083in}}%
\pgfpathlineto{\pgfqpoint{2.067415in}{1.316036in}}%
\pgfpathlineto{\pgfqpoint{2.082621in}{1.317624in}}%
\pgfpathlineto{\pgfqpoint{2.089398in}{1.325207in}}%
\pgfpathlineto{\pgfqpoint{2.089584in}{1.324400in}}%
\pgfpathlineto{\pgfqpoint{2.090324in}{1.305602in}}%
\pgfpathlineto{\pgfqpoint{2.092588in}{1.120962in}}%
\pgfpathlineto{\pgfqpoint{2.098467in}{0.657122in}}%
\pgfpathlineto{\pgfqpoint{2.113510in}{0.658166in}}%
\pgfpathlineto{\pgfqpoint{2.121060in}{0.698915in}}%
\pgfpathlineto{\pgfqpoint{2.140712in}{0.697372in}}%
\pgfpathlineto{\pgfqpoint{2.152431in}{0.696791in}}%
\pgfpathlineto{\pgfqpoint{2.156892in}{0.696032in}}%
\pgfpathlineto{\pgfqpoint{2.159285in}{0.679134in}}%
\pgfpathlineto{\pgfqpoint{2.164137in}{0.648605in}}%
\pgfpathlineto{\pgfqpoint{2.183827in}{0.647062in}}%
\pgfpathlineto{\pgfqpoint{2.187872in}{0.647307in}}%
\pgfpathlineto{\pgfqpoint{2.198049in}{0.645765in}}%
\pgfpathlineto{\pgfqpoint{2.210317in}{0.643617in}}%
\pgfpathlineto{\pgfqpoint{2.211898in}{0.645156in}}%
\pgfpathlineto{\pgfqpoint{2.216707in}{0.647573in}}%
\pgfpathlineto{\pgfqpoint{2.218173in}{0.649110in}}%
\pgfpathlineto{\pgfqpoint{2.222701in}{0.654214in}}%
\pgfpathlineto{\pgfqpoint{2.260887in}{0.652672in}}%
\pgfpathlineto{\pgfqpoint{2.264693in}{0.644904in}}%
\pgfpathlineto{\pgfqpoint{2.266680in}{0.643604in}}%
\pgfpathlineto{\pgfqpoint{2.275448in}{0.642064in}}%
\pgfpathlineto{\pgfqpoint{2.277964in}{0.637798in}}%
\pgfpathlineto{\pgfqpoint{2.279569in}{0.637273in}}%
\pgfpathlineto{\pgfqpoint{2.284249in}{0.638798in}}%
\pgfpathlineto{\pgfqpoint{2.288791in}{0.698643in}}%
\pgfpathlineto{\pgfqpoint{2.290935in}{0.702595in}}%
\pgfpathlineto{\pgfqpoint{2.315979in}{0.701053in}}%
\pgfpathlineto{\pgfqpoint{2.321791in}{0.699970in}}%
\pgfpathlineto{\pgfqpoint{2.327927in}{0.698437in}}%
\pgfpathlineto{\pgfqpoint{2.330038in}{0.684702in}}%
\pgfpathlineto{\pgfqpoint{2.335210in}{0.656116in}}%
\pgfpathlineto{\pgfqpoint{2.357345in}{0.654578in}}%
\pgfpathlineto{\pgfqpoint{2.364093in}{0.645638in}}%
\pgfpathlineto{\pgfqpoint{2.369256in}{0.645748in}}%
\pgfpathlineto{\pgfqpoint{2.373181in}{0.647290in}}%
\pgfpathlineto{\pgfqpoint{2.386076in}{0.652660in}}%
\pgfpathlineto{\pgfqpoint{2.427562in}{0.649350in}}%
\pgfpathlineto{\pgfqpoint{2.434892in}{0.649169in}}%
\pgfpathlineto{\pgfqpoint{2.446750in}{0.647974in}}%
\pgfpathlineto{\pgfqpoint{2.488752in}{0.649895in}}%
\pgfpathlineto{\pgfqpoint{2.500567in}{0.650907in}}%
\pgfpathlineto{\pgfqpoint{2.531580in}{0.650111in}}%
\pgfpathlineto{\pgfqpoint{2.536896in}{0.649281in}}%
\pgfpathlineto{\pgfqpoint{2.598712in}{0.647743in}}%
\pgfpathlineto{\pgfqpoint{2.601281in}{0.646668in}}%
\pgfpathlineto{\pgfqpoint{2.610460in}{0.646286in}}%
\pgfpathlineto{\pgfqpoint{2.616133in}{0.644745in}}%
\pgfpathlineto{\pgfqpoint{2.621391in}{0.643915in}}%
\pgfpathlineto{\pgfqpoint{2.625446in}{0.645445in}}%
\pgfpathlineto{\pgfqpoint{2.630169in}{0.661440in}}%
\pgfpathlineto{\pgfqpoint{2.634415in}{0.661510in}}%
\pgfpathlineto{\pgfqpoint{2.642060in}{0.659967in}}%
\pgfpathlineto{\pgfqpoint{2.649186in}{0.658117in}}%
\pgfpathlineto{\pgfqpoint{2.669043in}{0.656578in}}%
\pgfpathlineto{\pgfqpoint{2.672582in}{0.649344in}}%
\pgfpathlineto{\pgfqpoint{2.674406in}{0.648724in}}%
\pgfpathlineto{\pgfqpoint{2.695376in}{0.650260in}}%
\pgfpathlineto{\pgfqpoint{2.697449in}{0.650709in}}%
\pgfpathlineto{\pgfqpoint{2.749049in}{0.649478in}}%
\pgfpathlineto{\pgfqpoint{2.753386in}{0.649048in}}%
\pgfpathlineto{\pgfqpoint{2.756585in}{0.650805in}}%
\pgfpathlineto{\pgfqpoint{2.760330in}{0.651975in}}%
\pgfpathlineto{\pgfqpoint{2.792847in}{0.653491in}}%
\pgfpathlineto{\pgfqpoint{2.799189in}{0.699690in}}%
\pgfpathlineto{\pgfqpoint{2.828588in}{0.700704in}}%
\pgfpathlineto{\pgfqpoint{2.835341in}{0.700995in}}%
\pgfpathlineto{\pgfqpoint{2.836220in}{0.699484in}}%
\pgfpathlineto{\pgfqpoint{2.838631in}{0.681668in}}%
\pgfpathlineto{\pgfqpoint{2.843588in}{0.648991in}}%
\pgfpathlineto{\pgfqpoint{2.854095in}{0.649523in}}%
\pgfpathlineto{\pgfqpoint{2.856406in}{0.650498in}}%
\pgfpathlineto{\pgfqpoint{2.865418in}{0.650796in}}%
\pgfpathlineto{\pgfqpoint{2.867868in}{0.650480in}}%
\pgfpathlineto{\pgfqpoint{2.899015in}{0.647344in}}%
\pgfpathlineto{\pgfqpoint{2.905958in}{0.646018in}}%
\pgfpathlineto{\pgfqpoint{2.910094in}{0.648067in}}%
\pgfpathlineto{\pgfqpoint{2.913733in}{0.650249in}}%
\pgfpathlineto{\pgfqpoint{2.952860in}{0.650236in}}%
\pgfpathlineto{\pgfqpoint{2.962244in}{0.646496in}}%
\pgfpathlineto{\pgfqpoint{2.966901in}{0.648349in}}%
\pgfpathlineto{\pgfqpoint{2.973099in}{0.649883in}}%
\pgfpathlineto{\pgfqpoint{3.011042in}{0.648341in}}%
\pgfpathlineto{\pgfqpoint{3.016768in}{0.647527in}}%
\pgfpathlineto{\pgfqpoint{3.072562in}{0.648331in}}%
\pgfpathlineto{\pgfqpoint{3.080872in}{0.646061in}}%
\pgfpathlineto{\pgfqpoint{3.085136in}{0.638326in}}%
\pgfpathlineto{\pgfqpoint{3.086846in}{0.637273in}}%
\pgfpathlineto{\pgfqpoint{3.087634in}{0.638804in}}%
\pgfpathlineto{\pgfqpoint{3.090041in}{0.644058in}}%
\pgfpathlineto{\pgfqpoint{3.093771in}{0.646277in}}%
\pgfpathlineto{\pgfqpoint{3.110829in}{0.647820in}}%
\pgfpathlineto{\pgfqpoint{3.120978in}{0.651241in}}%
\pgfpathlineto{\pgfqpoint{3.126240in}{0.699449in}}%
\pgfpathlineto{\pgfqpoint{3.126417in}{0.699431in}}%
\pgfpathlineto{\pgfqpoint{3.153681in}{0.697888in}}%
\pgfpathlineto{\pgfqpoint{3.162688in}{0.695086in}}%
\pgfpathlineto{\pgfqpoint{3.164613in}{0.692605in}}%
\pgfpathlineto{\pgfqpoint{3.173137in}{0.664184in}}%
\pgfpathlineto{\pgfqpoint{3.175119in}{0.662428in}}%
\pgfpathlineto{\pgfqpoint{3.175630in}{0.663253in}}%
\pgfpathlineto{\pgfqpoint{3.177225in}{0.671076in}}%
\pgfpathlineto{\pgfqpoint{3.184819in}{0.703896in}}%
\pgfpathlineto{\pgfqpoint{3.196887in}{0.727077in}}%
\pgfpathlineto{\pgfqpoint{3.200325in}{0.727877in}}%
\pgfpathlineto{\pgfqpoint{3.200497in}{0.727562in}}%
\pgfpathlineto{\pgfqpoint{3.206037in}{0.713686in}}%
\pgfpathlineto{\pgfqpoint{3.213310in}{0.692233in}}%
\pgfpathlineto{\pgfqpoint{3.218945in}{0.675913in}}%
\pgfpathlineto{\pgfqpoint{3.231252in}{0.672194in}}%
\pgfpathlineto{\pgfqpoint{3.231372in}{0.673274in}}%
\pgfpathlineto{\pgfqpoint{3.231620in}{0.694660in}}%
\pgfpathlineto{\pgfqpoint{3.236988in}{1.243132in}}%
\pgfpathlineto{\pgfqpoint{3.237279in}{1.244558in}}%
\pgfpathlineto{\pgfqpoint{3.238573in}{1.243570in}}%
\pgfpathlineto{\pgfqpoint{3.240006in}{1.243031in}}%
\pgfpathlineto{\pgfqpoint{3.240235in}{1.244332in}}%
\pgfpathlineto{\pgfqpoint{3.241897in}{1.251970in}}%
\pgfpathlineto{\pgfqpoint{3.242088in}{1.246060in}}%
\pgfpathlineto{\pgfqpoint{3.244596in}{1.194529in}}%
\pgfpathlineto{\pgfqpoint{3.248292in}{1.103085in}}%
\pgfpathlineto{\pgfqpoint{3.249925in}{1.107621in}}%
\pgfpathlineto{\pgfqpoint{3.274787in}{1.168145in}}%
\pgfpathlineto{\pgfqpoint{3.275408in}{1.165260in}}%
\pgfpathlineto{\pgfqpoint{3.276382in}{1.146056in}}%
\pgfpathlineto{\pgfqpoint{3.281311in}{1.053865in}}%
\pgfpathlineto{\pgfqpoint{3.281812in}{1.054213in}}%
\pgfpathlineto{\pgfqpoint{3.285265in}{1.053692in}}%
\pgfpathlineto{\pgfqpoint{3.285298in}{1.053499in}}%
\pgfpathlineto{\pgfqpoint{3.286301in}{1.034477in}}%
\pgfpathlineto{\pgfqpoint{3.288828in}{0.884371in}}%
\pgfpathlineto{\pgfqpoint{3.292896in}{0.657584in}}%
\pgfpathlineto{\pgfqpoint{3.305361in}{0.656042in}}%
\pgfpathlineto{\pgfqpoint{3.312362in}{0.648281in}}%
\pgfpathlineto{\pgfqpoint{3.318532in}{0.649037in}}%
\pgfpathlineto{\pgfqpoint{3.333700in}{0.647994in}}%
\pgfpathlineto{\pgfqpoint{3.338227in}{0.658388in}}%
\pgfpathlineto{\pgfqpoint{3.340233in}{0.658555in}}%
\pgfpathlineto{\pgfqpoint{3.370382in}{0.657012in}}%
\pgfpathlineto{\pgfqpoint{3.377941in}{0.655015in}}%
\pgfpathlineto{\pgfqpoint{3.381327in}{0.646524in}}%
\pgfpathlineto{\pgfqpoint{3.383137in}{0.645918in}}%
\pgfpathlineto{\pgfqpoint{3.390325in}{0.645253in}}%
\pgfpathlineto{\pgfqpoint{3.390349in}{0.645348in}}%
\pgfpathlineto{\pgfqpoint{3.394460in}{0.657170in}}%
\pgfpathlineto{\pgfqpoint{3.396719in}{0.658146in}}%
\pgfpathlineto{\pgfqpoint{3.419045in}{0.659659in}}%
\pgfpathlineto{\pgfqpoint{3.425846in}{0.698130in}}%
\pgfpathlineto{\pgfqpoint{3.440584in}{0.696587in}}%
\pgfpathlineto{\pgfqpoint{3.445612in}{0.697262in}}%
\pgfpathlineto{\pgfqpoint{3.452795in}{0.697931in}}%
\pgfpathlineto{\pgfqpoint{3.462509in}{0.696410in}}%
\pgfpathlineto{\pgfqpoint{3.465856in}{0.670583in}}%
\pgfpathlineto{\pgfqpoint{3.468526in}{0.657531in}}%
\pgfpathlineto{\pgfqpoint{3.491626in}{0.655988in}}%
\pgfpathlineto{\pgfqpoint{3.498727in}{0.648517in}}%
\pgfpathlineto{\pgfqpoint{3.535223in}{0.647431in}}%
\pgfpathlineto{\pgfqpoint{3.556365in}{0.649035in}}%
\pgfpathlineto{\pgfqpoint{3.560763in}{0.657749in}}%
\pgfpathlineto{\pgfqpoint{3.584512in}{0.659282in}}%
\pgfpathlineto{\pgfqpoint{3.592115in}{0.699017in}}%
\pgfpathlineto{\pgfqpoint{3.611256in}{0.697474in}}%
\pgfpathlineto{\pgfqpoint{3.620607in}{0.697628in}}%
\pgfpathlineto{\pgfqpoint{3.627813in}{0.697153in}}%
\pgfpathlineto{\pgfqpoint{3.629886in}{0.682842in}}%
\pgfpathlineto{\pgfqpoint{3.635612in}{0.646782in}}%
\pgfpathlineto{\pgfqpoint{3.643998in}{0.645242in}}%
\pgfpathlineto{\pgfqpoint{3.648812in}{0.643588in}}%
\pgfpathlineto{\pgfqpoint{3.669309in}{0.642049in}}%
\pgfpathlineto{\pgfqpoint{3.670742in}{0.640510in}}%
\pgfpathlineto{\pgfqpoint{3.671014in}{0.642560in}}%
\pgfpathlineto{\pgfqpoint{3.674051in}{0.653973in}}%
\pgfpathlineto{\pgfqpoint{3.677017in}{0.657008in}}%
\pgfpathlineto{\pgfqpoint{3.714816in}{0.655469in}}%
\pgfpathlineto{\pgfqpoint{3.720996in}{0.648683in}}%
\pgfpathlineto{\pgfqpoint{3.749359in}{0.650193in}}%
\pgfpathlineto{\pgfqpoint{3.756240in}{0.700503in}}%
\pgfpathlineto{\pgfqpoint{3.788767in}{0.698960in}}%
\pgfpathlineto{\pgfqpoint{3.793347in}{0.694340in}}%
\pgfpathlineto{\pgfqpoint{3.796743in}{0.665219in}}%
\pgfpathlineto{\pgfqpoint{3.798901in}{0.656670in}}%
\pgfpathlineto{\pgfqpoint{3.829079in}{0.655131in}}%
\pgfpathlineto{\pgfqpoint{3.835693in}{0.648567in}}%
\pgfpathlineto{\pgfqpoint{3.864108in}{0.648475in}}%
\pgfpathlineto{\pgfqpoint{3.882614in}{0.648576in}}%
\pgfpathlineto{\pgfqpoint{3.896602in}{0.647033in}}%
\pgfpathlineto{\pgfqpoint{3.906201in}{0.647234in}}%
\pgfpathlineto{\pgfqpoint{3.911397in}{0.648400in}}%
\pgfpathlineto{\pgfqpoint{3.920633in}{0.649914in}}%
\pgfpathlineto{\pgfqpoint{3.927567in}{0.700666in}}%
\pgfpathlineto{\pgfqpoint{3.935748in}{0.700408in}}%
\pgfpathlineto{\pgfqpoint{3.947806in}{0.699660in}}%
\pgfpathlineto{\pgfqpoint{3.964230in}{0.699109in}}%
\pgfpathlineto{\pgfqpoint{3.967100in}{0.675293in}}%
\pgfpathlineto{\pgfqpoint{3.969106in}{0.646432in}}%
\pgfpathlineto{\pgfqpoint{3.971064in}{0.646781in}}%
\pgfpathlineto{\pgfqpoint{3.993609in}{0.645239in}}%
\pgfpathlineto{\pgfqpoint{4.001403in}{0.643978in}}%
\pgfpathlineto{\pgfqpoint{4.007793in}{0.649165in}}%
\pgfpathlineto{\pgfqpoint{4.011934in}{0.657151in}}%
\pgfpathlineto{\pgfqpoint{4.050588in}{0.655496in}}%
\pgfpathlineto{\pgfqpoint{4.057914in}{0.647417in}}%
\pgfpathlineto{\pgfqpoint{4.085555in}{0.650105in}}%
\pgfpathlineto{\pgfqpoint{4.091009in}{0.700163in}}%
\pgfpathlineto{\pgfqpoint{4.100274in}{0.700212in}}%
\pgfpathlineto{\pgfqpoint{4.106678in}{0.700680in}}%
\pgfpathlineto{\pgfqpoint{4.124138in}{0.702223in}}%
\pgfpathlineto{\pgfqpoint{4.128890in}{0.701242in}}%
\pgfpathlineto{\pgfqpoint{4.131048in}{0.685039in}}%
\pgfpathlineto{\pgfqpoint{4.133488in}{0.647126in}}%
\pgfpathlineto{\pgfqpoint{4.135470in}{0.637273in}}%
\pgfpathlineto{\pgfqpoint{4.136354in}{0.638804in}}%
\pgfpathlineto{\pgfqpoint{4.139176in}{0.642498in}}%
\pgfpathlineto{\pgfqpoint{4.145003in}{0.645123in}}%
\pgfpathlineto{\pgfqpoint{4.150614in}{0.645644in}}%
\pgfpathlineto{\pgfqpoint{4.166765in}{0.647187in}}%
\pgfpathlineto{\pgfqpoint{4.183647in}{0.647311in}}%
\pgfpathlineto{\pgfqpoint{4.212626in}{0.647202in}}%
\pgfpathlineto{\pgfqpoint{4.222201in}{0.646658in}}%
\pgfpathlineto{\pgfqpoint{4.226795in}{0.645734in}}%
\pgfpathlineto{\pgfqpoint{4.226967in}{0.646396in}}%
\pgfpathlineto{\pgfqpoint{4.230907in}{0.655389in}}%
\pgfpathlineto{\pgfqpoint{4.233328in}{0.656310in}}%
\pgfpathlineto{\pgfqpoint{4.255640in}{0.657779in}}%
\pgfpathlineto{\pgfqpoint{4.261911in}{0.697460in}}%
\pgfpathlineto{\pgfqpoint{4.299075in}{0.695933in}}%
\pgfpathlineto{\pgfqpoint{4.302136in}{0.670528in}}%
\pgfpathlineto{\pgfqpoint{4.305536in}{0.647410in}}%
\pgfpathlineto{\pgfqpoint{4.354988in}{0.648947in}}%
\pgfpathlineto{\pgfqpoint{4.359640in}{0.653531in}}%
\pgfpathlineto{\pgfqpoint{4.396871in}{0.652995in}}%
\pgfpathlineto{\pgfqpoint{4.399655in}{0.650211in}}%
\pgfpathlineto{\pgfqpoint{4.403805in}{0.647403in}}%
\pgfpathlineto{\pgfqpoint{4.421016in}{0.648903in}}%
\pgfpathlineto{\pgfqpoint{4.427187in}{0.698647in}}%
\pgfpathlineto{\pgfqpoint{4.464313in}{0.697107in}}%
\pgfpathlineto{\pgfqpoint{4.465965in}{0.687754in}}%
\pgfpathlineto{\pgfqpoint{4.472006in}{0.657172in}}%
\pgfpathlineto{\pgfqpoint{4.493487in}{0.655634in}}%
\pgfpathlineto{\pgfqpoint{4.501386in}{0.647121in}}%
\pgfpathlineto{\pgfqpoint{4.516128in}{0.646946in}}%
\pgfpathlineto{\pgfqpoint{4.526272in}{0.651916in}}%
\pgfpathlineto{\pgfqpoint{4.538569in}{0.652131in}}%
\pgfpathlineto{\pgfqpoint{4.560977in}{0.650659in}}%
\pgfpathlineto{\pgfqpoint{4.565494in}{0.648257in}}%
\pgfpathlineto{\pgfqpoint{4.568164in}{0.649800in}}%
\pgfpathlineto{\pgfqpoint{4.571679in}{0.652261in}}%
\pgfpathlineto{\pgfqpoint{4.579482in}{0.654298in}}%
\pgfpathlineto{\pgfqpoint{4.583628in}{0.655754in}}%
\pgfpathlineto{\pgfqpoint{4.590013in}{0.656223in}}%
\pgfpathlineto{\pgfqpoint{4.593188in}{0.656554in}}%
\pgfpathlineto{\pgfqpoint{4.601923in}{0.655441in}}%
\pgfpathlineto{\pgfqpoint{4.616164in}{0.655051in}}%
\pgfpathlineto{\pgfqpoint{4.617773in}{0.655948in}}%
\pgfpathlineto{\pgfqpoint{4.621193in}{0.656214in}}%
\pgfpathlineto{\pgfqpoint{4.635033in}{0.654718in}}%
\pgfpathlineto{\pgfqpoint{4.635047in}{0.654787in}}%
\pgfpathlineto{\pgfqpoint{4.636250in}{0.668660in}}%
\pgfpathlineto{\pgfqpoint{4.640066in}{0.694958in}}%
\pgfpathlineto{\pgfqpoint{4.642129in}{0.695512in}}%
\pgfpathlineto{\pgfqpoint{4.648863in}{0.694412in}}%
\pgfpathlineto{\pgfqpoint{4.652488in}{0.694151in}}%
\pgfpathlineto{\pgfqpoint{4.658496in}{0.695640in}}%
\pgfpathlineto{\pgfqpoint{4.666509in}{0.697540in}}%
\pgfpathlineto{\pgfqpoint{4.671094in}{0.701480in}}%
\pgfpathlineto{\pgfqpoint{4.674852in}{0.703110in}}%
\pgfpathlineto{\pgfqpoint{4.675535in}{0.792163in}}%
\pgfpathlineto{\pgfqpoint{4.679981in}{1.298202in}}%
\pgfpathlineto{\pgfqpoint{4.680120in}{1.299395in}}%
\pgfpathlineto{\pgfqpoint{4.680640in}{1.283084in}}%
\pgfpathlineto{\pgfqpoint{4.681915in}{1.235356in}}%
\pgfpathlineto{\pgfqpoint{4.682937in}{1.236772in}}%
\pgfpathlineto{\pgfqpoint{4.685382in}{1.241504in}}%
\pgfpathlineto{\pgfqpoint{4.688635in}{1.274946in}}%
\pgfpathlineto{\pgfqpoint{4.689485in}{1.273260in}}%
\pgfpathlineto{\pgfqpoint{4.691061in}{1.264894in}}%
\pgfpathlineto{\pgfqpoint{4.692603in}{1.257425in}}%
\pgfpathlineto{\pgfqpoint{4.693296in}{1.260616in}}%
\pgfpathlineto{\pgfqpoint{4.693907in}{1.262537in}}%
\pgfpathlineto{\pgfqpoint{4.695173in}{1.261613in}}%
\pgfpathlineto{\pgfqpoint{4.695865in}{1.265381in}}%
\pgfpathlineto{\pgfqpoint{4.696820in}{1.303950in}}%
\pgfpathlineto{\pgfqpoint{4.699623in}{1.409627in}}%
\pgfpathlineto{\pgfqpoint{4.699953in}{1.407899in}}%
\pgfpathlineto{\pgfqpoint{4.701811in}{1.403450in}}%
\pgfpathlineto{\pgfqpoint{4.701935in}{1.403490in}}%
\pgfpathlineto{\pgfqpoint{4.705445in}{1.406312in}}%
\pgfpathlineto{\pgfqpoint{4.709208in}{1.409582in}}%
\pgfpathlineto{\pgfqpoint{4.715178in}{1.420144in}}%
\pgfpathlineto{\pgfqpoint{4.718110in}{1.423785in}}%
\pgfpathlineto{\pgfqpoint{4.718239in}{1.423954in}}%
\pgfpathlineto{\pgfqpoint{4.718674in}{1.420814in}}%
\pgfpathlineto{\pgfqpoint{4.719863in}{1.380315in}}%
\pgfpathlineto{\pgfqpoint{4.722408in}{1.292827in}}%
\pgfpathlineto{\pgfqpoint{4.723430in}{1.294003in}}%
\pgfpathlineto{\pgfqpoint{4.728731in}{1.292474in}}%
\pgfpathlineto{\pgfqpoint{4.730971in}{1.290650in}}%
\pgfpathlineto{\pgfqpoint{4.731845in}{1.285331in}}%
\pgfpathlineto{\pgfqpoint{4.734529in}{1.276103in}}%
\pgfpathlineto{\pgfqpoint{4.735794in}{1.272504in}}%
\pgfpathlineto{\pgfqpoint{4.739319in}{1.224651in}}%
\pgfpathlineto{\pgfqpoint{4.741382in}{0.983389in}}%
\pgfpathlineto{\pgfqpoint{4.745097in}{0.637273in}}%
\pgfpathlineto{\pgfqpoint{4.758507in}{0.638787in}}%
\pgfpathlineto{\pgfqpoint{4.762767in}{1.269868in}}%
\pgfpathlineto{\pgfqpoint{4.763875in}{1.327333in}}%
\pgfpathlineto{\pgfqpoint{4.764663in}{1.297468in}}%
\pgfpathlineto{\pgfqpoint{4.765289in}{1.285687in}}%
\pgfpathlineto{\pgfqpoint{4.766163in}{1.298957in}}%
\pgfpathlineto{\pgfqpoint{4.767567in}{1.365778in}}%
\pgfpathlineto{\pgfqpoint{4.770107in}{1.473460in}}%
\pgfpathlineto{\pgfqpoint{4.770719in}{1.468588in}}%
\pgfpathlineto{\pgfqpoint{4.772724in}{1.464162in}}%
\pgfpathlineto{\pgfqpoint{4.775628in}{1.463670in}}%
\pgfpathlineto{\pgfqpoint{4.777806in}{1.462129in}}%
\pgfpathlineto{\pgfqpoint{4.787983in}{1.457500in}}%
\pgfpathlineto{\pgfqpoint{4.794339in}{1.455959in}}%
\pgfpathlineto{\pgfqpoint{4.800356in}{1.454697in}}%
\pgfpathlineto{\pgfqpoint{4.802085in}{1.452814in}}%
\pgfpathlineto{\pgfqpoint{4.803074in}{1.418424in}}%
\pgfpathlineto{\pgfqpoint{4.808064in}{1.257622in}}%
\pgfpathlineto{\pgfqpoint{4.809550in}{1.222676in}}%
\pgfpathlineto{\pgfqpoint{4.811340in}{1.049090in}}%
\pgfpathlineto{\pgfqpoint{4.816145in}{0.657732in}}%
\pgfpathlineto{\pgfqpoint{4.822496in}{0.655533in}}%
\pgfpathlineto{\pgfqpoint{4.827668in}{0.643716in}}%
\pgfpathlineto{\pgfqpoint{4.829149in}{0.644484in}}%
\pgfpathlineto{\pgfqpoint{4.834770in}{0.645400in}}%
\pgfpathlineto{\pgfqpoint{4.844789in}{0.643483in}}%
\pgfpathlineto{\pgfqpoint{4.872540in}{0.641942in}}%
\pgfpathlineto{\pgfqpoint{4.875095in}{0.641699in}}%
\pgfpathlineto{\pgfqpoint{4.879274in}{0.643618in}}%
\pgfpathlineto{\pgfqpoint{4.898888in}{0.645161in}}%
\pgfpathlineto{\pgfqpoint{4.904585in}{0.645684in}}%
\pgfpathlineto{\pgfqpoint{4.942580in}{0.644142in}}%
\pgfpathlineto{\pgfqpoint{4.943951in}{0.643742in}}%
\pgfpathlineto{\pgfqpoint{4.944142in}{0.645035in}}%
\pgfpathlineto{\pgfqpoint{4.950450in}{0.702396in}}%
\pgfpathlineto{\pgfqpoint{4.966974in}{0.702877in}}%
\pgfpathlineto{\pgfqpoint{4.976325in}{0.701336in}}%
\pgfpathlineto{\pgfqpoint{4.982371in}{0.700187in}}%
\pgfpathlineto{\pgfqpoint{4.987753in}{0.698662in}}%
\pgfpathlineto{\pgfqpoint{4.990170in}{0.681837in}}%
\pgfpathlineto{\pgfqpoint{4.994611in}{0.656111in}}%
\pgfpathlineto{\pgfqpoint{5.018040in}{0.654569in}}%
\pgfpathlineto{\pgfqpoint{5.021517in}{0.646138in}}%
\pgfpathlineto{\pgfqpoint{5.023212in}{0.645712in}}%
\pgfpathlineto{\pgfqpoint{5.043671in}{0.644170in}}%
\pgfpathlineto{\pgfqpoint{5.048394in}{0.643466in}}%
\pgfpathlineto{\pgfqpoint{5.071107in}{0.641924in}}%
\pgfpathlineto{\pgfqpoint{5.073481in}{0.637673in}}%
\pgfpathlineto{\pgfqpoint{5.075057in}{0.637273in}}%
\pgfpathlineto{\pgfqpoint{5.083247in}{0.638804in}}%
\pgfpathlineto{\pgfqpoint{5.086065in}{0.642495in}}%
\pgfpathlineto{\pgfqpoint{5.089809in}{0.643483in}}%
\pgfpathlineto{\pgfqpoint{5.095444in}{0.645024in}}%
\pgfpathlineto{\pgfqpoint{5.100201in}{0.647459in}}%
\pgfpathlineto{\pgfqpoint{5.106237in}{0.648953in}}%
\pgfpathlineto{\pgfqpoint{5.112770in}{0.701276in}}%
\pgfpathlineto{\pgfqpoint{5.124986in}{0.699735in}}%
\pgfpathlineto{\pgfqpoint{5.129179in}{0.699536in}}%
\pgfpathlineto{\pgfqpoint{5.149428in}{0.697997in}}%
\pgfpathlineto{\pgfqpoint{5.150536in}{0.692722in}}%
\pgfpathlineto{\pgfqpoint{5.153698in}{0.663834in}}%
\pgfpathlineto{\pgfqpoint{5.155837in}{0.654925in}}%
\pgfpathlineto{\pgfqpoint{5.165422in}{0.653386in}}%
\pgfpathlineto{\pgfqpoint{5.170646in}{0.648557in}}%
\pgfpathlineto{\pgfqpoint{5.186888in}{0.650780in}}%
\pgfpathlineto{\pgfqpoint{5.188636in}{0.654166in}}%
\pgfpathlineto{\pgfqpoint{5.188636in}{0.654166in}}%
\pgfusepath{stroke}%
\end{pgfscope}%
\begin{pgfscope}%
\pgfsetrectcap%
\pgfsetmiterjoin%
\pgfsetlinewidth{0.803000pt}%
\definecolor{currentstroke}{rgb}{0.000000,0.000000,0.000000}%
\pgfsetstrokecolor{currentstroke}%
\pgfsetdash{}{0pt}%
\pgfpathmoveto{\pgfqpoint{0.750000in}{0.500000in}}%
\pgfpathlineto{\pgfqpoint{0.750000in}{3.520000in}}%
\pgfusepath{stroke}%
\end{pgfscope}%
\begin{pgfscope}%
\pgfsetrectcap%
\pgfsetmiterjoin%
\pgfsetlinewidth{0.803000pt}%
\definecolor{currentstroke}{rgb}{0.000000,0.000000,0.000000}%
\pgfsetstrokecolor{currentstroke}%
\pgfsetdash{}{0pt}%
\pgfpathmoveto{\pgfqpoint{5.400000in}{0.500000in}}%
\pgfpathlineto{\pgfqpoint{5.400000in}{3.520000in}}%
\pgfusepath{stroke}%
\end{pgfscope}%
\begin{pgfscope}%
\pgfsetrectcap%
\pgfsetmiterjoin%
\pgfsetlinewidth{0.803000pt}%
\definecolor{currentstroke}{rgb}{0.000000,0.000000,0.000000}%
\pgfsetstrokecolor{currentstroke}%
\pgfsetdash{}{0pt}%
\pgfpathmoveto{\pgfqpoint{0.750000in}{0.500000in}}%
\pgfpathlineto{\pgfqpoint{5.400000in}{0.500000in}}%
\pgfusepath{stroke}%
\end{pgfscope}%
\begin{pgfscope}%
\pgfsetrectcap%
\pgfsetmiterjoin%
\pgfsetlinewidth{0.803000pt}%
\definecolor{currentstroke}{rgb}{0.000000,0.000000,0.000000}%
\pgfsetstrokecolor{currentstroke}%
\pgfsetdash{}{0pt}%
\pgfpathmoveto{\pgfqpoint{0.750000in}{3.520000in}}%
\pgfpathlineto{\pgfqpoint{5.400000in}{3.520000in}}%
\pgfusepath{stroke}%
\end{pgfscope}%
\begin{pgfscope}%
\pgfsetbuttcap%
\pgfsetmiterjoin%
\definecolor{currentfill}{rgb}{1.000000,1.000000,1.000000}%
\pgfsetfillcolor{currentfill}%
\pgfsetfillopacity{0.800000}%
\pgfsetlinewidth{1.003750pt}%
\definecolor{currentstroke}{rgb}{0.800000,0.800000,0.800000}%
\pgfsetstrokecolor{currentstroke}%
\pgfsetstrokeopacity{0.800000}%
\pgfsetdash{}{0pt}%
\pgfpathmoveto{\pgfqpoint{2.723533in}{2.827871in}}%
\pgfpathlineto{\pgfqpoint{3.426467in}{2.827871in}}%
\pgfpathquadraticcurveto{\pgfqpoint{3.454244in}{2.827871in}}{\pgfqpoint{3.454244in}{2.855648in}}%
\pgfpathlineto{\pgfqpoint{3.454244in}{3.422778in}}%
\pgfpathquadraticcurveto{\pgfqpoint{3.454244in}{3.450556in}}{\pgfqpoint{3.426467in}{3.450556in}}%
\pgfpathlineto{\pgfqpoint{2.723533in}{3.450556in}}%
\pgfpathquadraticcurveto{\pgfqpoint{2.695756in}{3.450556in}}{\pgfqpoint{2.695756in}{3.422778in}}%
\pgfpathlineto{\pgfqpoint{2.695756in}{2.855648in}}%
\pgfpathquadraticcurveto{\pgfqpoint{2.695756in}{2.827871in}}{\pgfqpoint{2.723533in}{2.827871in}}%
\pgfpathlineto{\pgfqpoint{2.723533in}{2.827871in}}%
\pgfpathclose%
\pgfusepath{stroke,fill}%
\end{pgfscope}%
\begin{pgfscope}%
\pgfsetrectcap%
\pgfsetroundjoin%
\pgfsetlinewidth{1.505625pt}%
\definecolor{currentstroke}{rgb}{0.000000,0.000000,1.000000}%
\pgfsetstrokecolor{currentstroke}%
\pgfsetdash{}{0pt}%
\pgfpathmoveto{\pgfqpoint{2.751311in}{3.346389in}}%
\pgfpathlineto{\pgfqpoint{2.890200in}{3.346389in}}%
\pgfpathlineto{\pgfqpoint{3.029089in}{3.346389in}}%
\pgfusepath{stroke}%
\end{pgfscope}%
\begin{pgfscope}%
\definecolor{textcolor}{rgb}{0.000000,0.000000,0.000000}%
\pgfsetstrokecolor{textcolor}%
\pgfsetfillcolor{textcolor}%
\pgftext[x=3.140200in,y=3.297778in,left,base]{\color{textcolor}\rmfamily\fontsize{10.000000}{12.000000}\selectfont max}%
\end{pgfscope}%
\begin{pgfscope}%
\pgfsetrectcap%
\pgfsetroundjoin%
\pgfsetlinewidth{1.505625pt}%
\definecolor{currentstroke}{rgb}{1.000000,0.000000,0.000000}%
\pgfsetstrokecolor{currentstroke}%
\pgfsetdash{}{0pt}%
\pgfpathmoveto{\pgfqpoint{2.751311in}{3.152716in}}%
\pgfpathlineto{\pgfqpoint{2.890200in}{3.152716in}}%
\pgfpathlineto{\pgfqpoint{3.029089in}{3.152716in}}%
\pgfusepath{stroke}%
\end{pgfscope}%
\begin{pgfscope}%
\definecolor{textcolor}{rgb}{0.000000,0.000000,0.000000}%
\pgfsetstrokecolor{textcolor}%
\pgfsetfillcolor{textcolor}%
\pgftext[x=3.140200in,y=3.104105in,left,base]{\color{textcolor}\rmfamily\fontsize{10.000000}{12.000000}\selectfont \(\displaystyle \mu\)}%
\end{pgfscope}%
\begin{pgfscope}%
\pgfsetrectcap%
\pgfsetroundjoin%
\pgfsetlinewidth{1.505625pt}%
\definecolor{currentstroke}{rgb}{0.000000,0.500000,0.000000}%
\pgfsetstrokecolor{currentstroke}%
\pgfsetdash{}{0pt}%
\pgfpathmoveto{\pgfqpoint{2.751311in}{2.959043in}}%
\pgfpathlineto{\pgfqpoint{2.890200in}{2.959043in}}%
\pgfpathlineto{\pgfqpoint{3.029089in}{2.959043in}}%
\pgfusepath{stroke}%
\end{pgfscope}%
\begin{pgfscope}%
\definecolor{textcolor}{rgb}{0.000000,0.000000,0.000000}%
\pgfsetstrokecolor{textcolor}%
\pgfsetfillcolor{textcolor}%
\pgftext[x=3.140200in,y=2.910432in,left,base]{\color{textcolor}\rmfamily\fontsize{10.000000}{12.000000}\selectfont \(\displaystyle \sigma\)}%
\end{pgfscope}%
\end{pgfpicture}%
\makeatother%
\endgroup%

    \caption{BETH Matrix Profile}
    \label{fig:beth_mp_hist}
\end{figure}

Figure \ref{fig:beth_sus_outliers} shows the algorithms detection results compared to the values considered suspicious by Authors \cite{beth-dataset} of the BETH dataset. They determined that many of the data points that are considered suspicious are not actually dangerous (evil in their taxonomy) as shown in Figure \ref{fig:beth_evil_outliers}. In this case, the detector is able to discern where the dangerous outliers lie among the very noisy suspicious data points. This is the most challenging dataset analyzed thus far and as such the results do not provide 100\% accuracy with no false positives.

% TODO: Redo axis labels
 \begin{figure}[H]
    %%\centering
    %% Creator: Matplotlib, PGF backend
%%
%% To include the figure in your LaTeX document, write
%%   \input{<filename>.pgf}
%%
%% Make sure the required packages are loaded in your preamble
%%   \usepackage{pgf}
%%
%% Also ensure that all the required font packages are loaded; for instance,
%% the lmodern package is sometimes necessary when using math font.
%%   \usepackage{lmodern}
%%
%% Figures using additional raster images can only be included by \input if
%% they are in the same directory as the main LaTeX file. For loading figures
%% from other directories you can use the `import` package
%%   \usepackage{import}
%%
%% and then include the figures with
%%   \import{<path to file>}{<filename>.pgf}
%%
%% Matplotlib used the following preamble
%%
\begingroup%
\makeatletter%
\begin{pgfpicture}%
\pgfpathrectangle{\pgfpointorigin}{\pgfqpoint{6.000000in}{4.000000in}}%
\pgfusepath{use as bounding box, clip}%
\begin{pgfscope}%
\pgfsetbuttcap%
\pgfsetmiterjoin%
\pgfsetlinewidth{0.000000pt}%
\definecolor{currentstroke}{rgb}{1.000000,1.000000,1.000000}%
\pgfsetstrokecolor{currentstroke}%
\pgfsetstrokeopacity{0.000000}%
\pgfsetdash{}{0pt}%
\pgfpathmoveto{\pgfqpoint{0.000000in}{0.000000in}}%
\pgfpathlineto{\pgfqpoint{6.000000in}{0.000000in}}%
\pgfpathlineto{\pgfqpoint{6.000000in}{4.000000in}}%
\pgfpathlineto{\pgfqpoint{0.000000in}{4.000000in}}%
\pgfpathlineto{\pgfqpoint{0.000000in}{0.000000in}}%
\pgfpathclose%
\pgfusepath{}%
\end{pgfscope}%
\begin{pgfscope}%
\pgfsetbuttcap%
\pgfsetmiterjoin%
\definecolor{currentfill}{rgb}{1.000000,1.000000,1.000000}%
\pgfsetfillcolor{currentfill}%
\pgfsetlinewidth{0.000000pt}%
\definecolor{currentstroke}{rgb}{0.000000,0.000000,0.000000}%
\pgfsetstrokecolor{currentstroke}%
\pgfsetstrokeopacity{0.000000}%
\pgfsetdash{}{0pt}%
\pgfpathmoveto{\pgfqpoint{0.750000in}{0.500000in}}%
\pgfpathlineto{\pgfqpoint{5.400000in}{0.500000in}}%
\pgfpathlineto{\pgfqpoint{5.400000in}{3.520000in}}%
\pgfpathlineto{\pgfqpoint{0.750000in}{3.520000in}}%
\pgfpathlineto{\pgfqpoint{0.750000in}{0.500000in}}%
\pgfpathclose%
\pgfusepath{fill}%
\end{pgfscope}%
\begin{pgfscope}%
\pgfsetbuttcap%
\pgfsetroundjoin%
\definecolor{currentfill}{rgb}{0.000000,0.000000,0.000000}%
\pgfsetfillcolor{currentfill}%
\pgfsetlinewidth{0.803000pt}%
\definecolor{currentstroke}{rgb}{0.000000,0.000000,0.000000}%
\pgfsetstrokecolor{currentstroke}%
\pgfsetdash{}{0pt}%
\pgfsys@defobject{currentmarker}{\pgfqpoint{0.000000in}{-0.048611in}}{\pgfqpoint{0.000000in}{0.000000in}}{%
\pgfpathmoveto{\pgfqpoint{0.000000in}{0.000000in}}%
\pgfpathlineto{\pgfqpoint{0.000000in}{-0.048611in}}%
\pgfusepath{stroke,fill}%
}%
\begin{pgfscope}%
\pgfsys@transformshift{0.961364in}{0.500000in}%
\pgfsys@useobject{currentmarker}{}%
\end{pgfscope}%
\end{pgfscope}%
\begin{pgfscope}%
\definecolor{textcolor}{rgb}{0.000000,0.000000,0.000000}%
\pgfsetstrokecolor{textcolor}%
\pgfsetfillcolor{textcolor}%
\pgftext[x=0.961364in,y=0.402778in,,top]{\color{textcolor}\rmfamily\fontsize{10.000000}{12.000000}\selectfont \(\displaystyle {0}\)}%
\end{pgfscope}%
\begin{pgfscope}%
\pgfsetbuttcap%
\pgfsetroundjoin%
\definecolor{currentfill}{rgb}{0.000000,0.000000,0.000000}%
\pgfsetfillcolor{currentfill}%
\pgfsetlinewidth{0.803000pt}%
\definecolor{currentstroke}{rgb}{0.000000,0.000000,0.000000}%
\pgfsetstrokecolor{currentstroke}%
\pgfsetdash{}{0pt}%
\pgfsys@defobject{currentmarker}{\pgfqpoint{0.000000in}{-0.048611in}}{\pgfqpoint{0.000000in}{0.000000in}}{%
\pgfpathmoveto{\pgfqpoint{0.000000in}{0.000000in}}%
\pgfpathlineto{\pgfqpoint{0.000000in}{-0.048611in}}%
\pgfusepath{stroke,fill}%
}%
\begin{pgfscope}%
\pgfsys@transformshift{1.905825in}{0.500000in}%
\pgfsys@useobject{currentmarker}{}%
\end{pgfscope}%
\end{pgfscope}%
\begin{pgfscope}%
\definecolor{textcolor}{rgb}{0.000000,0.000000,0.000000}%
\pgfsetstrokecolor{textcolor}%
\pgfsetfillcolor{textcolor}%
\pgftext[x=1.905825in,y=0.402778in,,top]{\color{textcolor}\rmfamily\fontsize{10.000000}{12.000000}\selectfont \(\displaystyle {200000}\)}%
\end{pgfscope}%
\begin{pgfscope}%
\pgfsetbuttcap%
\pgfsetroundjoin%
\definecolor{currentfill}{rgb}{0.000000,0.000000,0.000000}%
\pgfsetfillcolor{currentfill}%
\pgfsetlinewidth{0.803000pt}%
\definecolor{currentstroke}{rgb}{0.000000,0.000000,0.000000}%
\pgfsetstrokecolor{currentstroke}%
\pgfsetdash{}{0pt}%
\pgfsys@defobject{currentmarker}{\pgfqpoint{0.000000in}{-0.048611in}}{\pgfqpoint{0.000000in}{0.000000in}}{%
\pgfpathmoveto{\pgfqpoint{0.000000in}{0.000000in}}%
\pgfpathlineto{\pgfqpoint{0.000000in}{-0.048611in}}%
\pgfusepath{stroke,fill}%
}%
\begin{pgfscope}%
\pgfsys@transformshift{2.850287in}{0.500000in}%
\pgfsys@useobject{currentmarker}{}%
\end{pgfscope}%
\end{pgfscope}%
\begin{pgfscope}%
\definecolor{textcolor}{rgb}{0.000000,0.000000,0.000000}%
\pgfsetstrokecolor{textcolor}%
\pgfsetfillcolor{textcolor}%
\pgftext[x=2.850287in,y=0.402778in,,top]{\color{textcolor}\rmfamily\fontsize{10.000000}{12.000000}\selectfont \(\displaystyle {400000}\)}%
\end{pgfscope}%
\begin{pgfscope}%
\pgfsetbuttcap%
\pgfsetroundjoin%
\definecolor{currentfill}{rgb}{0.000000,0.000000,0.000000}%
\pgfsetfillcolor{currentfill}%
\pgfsetlinewidth{0.803000pt}%
\definecolor{currentstroke}{rgb}{0.000000,0.000000,0.000000}%
\pgfsetstrokecolor{currentstroke}%
\pgfsetdash{}{0pt}%
\pgfsys@defobject{currentmarker}{\pgfqpoint{0.000000in}{-0.048611in}}{\pgfqpoint{0.000000in}{0.000000in}}{%
\pgfpathmoveto{\pgfqpoint{0.000000in}{0.000000in}}%
\pgfpathlineto{\pgfqpoint{0.000000in}{-0.048611in}}%
\pgfusepath{stroke,fill}%
}%
\begin{pgfscope}%
\pgfsys@transformshift{3.794748in}{0.500000in}%
\pgfsys@useobject{currentmarker}{}%
\end{pgfscope}%
\end{pgfscope}%
\begin{pgfscope}%
\definecolor{textcolor}{rgb}{0.000000,0.000000,0.000000}%
\pgfsetstrokecolor{textcolor}%
\pgfsetfillcolor{textcolor}%
\pgftext[x=3.794748in,y=0.402778in,,top]{\color{textcolor}\rmfamily\fontsize{10.000000}{12.000000}\selectfont \(\displaystyle {600000}\)}%
\end{pgfscope}%
\begin{pgfscope}%
\pgfsetbuttcap%
\pgfsetroundjoin%
\definecolor{currentfill}{rgb}{0.000000,0.000000,0.000000}%
\pgfsetfillcolor{currentfill}%
\pgfsetlinewidth{0.803000pt}%
\definecolor{currentstroke}{rgb}{0.000000,0.000000,0.000000}%
\pgfsetstrokecolor{currentstroke}%
\pgfsetdash{}{0pt}%
\pgfsys@defobject{currentmarker}{\pgfqpoint{0.000000in}{-0.048611in}}{\pgfqpoint{0.000000in}{0.000000in}}{%
\pgfpathmoveto{\pgfqpoint{0.000000in}{0.000000in}}%
\pgfpathlineto{\pgfqpoint{0.000000in}{-0.048611in}}%
\pgfusepath{stroke,fill}%
}%
\begin{pgfscope}%
\pgfsys@transformshift{4.739210in}{0.500000in}%
\pgfsys@useobject{currentmarker}{}%
\end{pgfscope}%
\end{pgfscope}%
\begin{pgfscope}%
\definecolor{textcolor}{rgb}{0.000000,0.000000,0.000000}%
\pgfsetstrokecolor{textcolor}%
\pgfsetfillcolor{textcolor}%
\pgftext[x=4.739210in,y=0.402778in,,top]{\color{textcolor}\rmfamily\fontsize{10.000000}{12.000000}\selectfont \(\displaystyle {800000}\)}%
\end{pgfscope}%
\begin{pgfscope}%
\definecolor{textcolor}{rgb}{0.000000,0.000000,0.000000}%
\pgfsetstrokecolor{textcolor}%
\pgfsetfillcolor{textcolor}%
\pgftext[x=3.075000in,y=0.223766in,,top]{\color{textcolor}\rmfamily\fontsize{10.000000}{12.000000}\selectfont time}%
\end{pgfscope}%
\begin{pgfscope}%
\pgfsetbuttcap%
\pgfsetroundjoin%
\definecolor{currentfill}{rgb}{0.000000,0.000000,0.000000}%
\pgfsetfillcolor{currentfill}%
\pgfsetlinewidth{0.803000pt}%
\definecolor{currentstroke}{rgb}{0.000000,0.000000,0.000000}%
\pgfsetstrokecolor{currentstroke}%
\pgfsetdash{}{0pt}%
\pgfsys@defobject{currentmarker}{\pgfqpoint{-0.048611in}{0.000000in}}{\pgfqpoint{-0.000000in}{0.000000in}}{%
\pgfpathmoveto{\pgfqpoint{-0.000000in}{0.000000in}}%
\pgfpathlineto{\pgfqpoint{-0.048611in}{0.000000in}}%
\pgfusepath{stroke,fill}%
}%
\begin{pgfscope}%
\pgfsys@transformshift{0.750000in}{0.637273in}%
\pgfsys@useobject{currentmarker}{}%
\end{pgfscope}%
\end{pgfscope}%
\begin{pgfscope}%
\definecolor{textcolor}{rgb}{0.000000,0.000000,0.000000}%
\pgfsetstrokecolor{textcolor}%
\pgfsetfillcolor{textcolor}%
\pgftext[x=0.475308in, y=0.589047in, left, base]{\color{textcolor}\rmfamily\fontsize{10.000000}{12.000000}\selectfont \(\displaystyle {0.0}\)}%
\end{pgfscope}%
\begin{pgfscope}%
\pgfsetbuttcap%
\pgfsetroundjoin%
\definecolor{currentfill}{rgb}{0.000000,0.000000,0.000000}%
\pgfsetfillcolor{currentfill}%
\pgfsetlinewidth{0.803000pt}%
\definecolor{currentstroke}{rgb}{0.000000,0.000000,0.000000}%
\pgfsetstrokecolor{currentstroke}%
\pgfsetdash{}{0pt}%
\pgfsys@defobject{currentmarker}{\pgfqpoint{-0.048611in}{0.000000in}}{\pgfqpoint{-0.000000in}{0.000000in}}{%
\pgfpathmoveto{\pgfqpoint{-0.000000in}{0.000000in}}%
\pgfpathlineto{\pgfqpoint{-0.048611in}{0.000000in}}%
\pgfusepath{stroke,fill}%
}%
\begin{pgfscope}%
\pgfsys@transformshift{0.750000in}{1.186364in}%
\pgfsys@useobject{currentmarker}{}%
\end{pgfscope}%
\end{pgfscope}%
\begin{pgfscope}%
\definecolor{textcolor}{rgb}{0.000000,0.000000,0.000000}%
\pgfsetstrokecolor{textcolor}%
\pgfsetfillcolor{textcolor}%
\pgftext[x=0.475308in, y=1.138138in, left, base]{\color{textcolor}\rmfamily\fontsize{10.000000}{12.000000}\selectfont \(\displaystyle {0.2}\)}%
\end{pgfscope}%
\begin{pgfscope}%
\pgfsetbuttcap%
\pgfsetroundjoin%
\definecolor{currentfill}{rgb}{0.000000,0.000000,0.000000}%
\pgfsetfillcolor{currentfill}%
\pgfsetlinewidth{0.803000pt}%
\definecolor{currentstroke}{rgb}{0.000000,0.000000,0.000000}%
\pgfsetstrokecolor{currentstroke}%
\pgfsetdash{}{0pt}%
\pgfsys@defobject{currentmarker}{\pgfqpoint{-0.048611in}{0.000000in}}{\pgfqpoint{-0.000000in}{0.000000in}}{%
\pgfpathmoveto{\pgfqpoint{-0.000000in}{0.000000in}}%
\pgfpathlineto{\pgfqpoint{-0.048611in}{0.000000in}}%
\pgfusepath{stroke,fill}%
}%
\begin{pgfscope}%
\pgfsys@transformshift{0.750000in}{1.735455in}%
\pgfsys@useobject{currentmarker}{}%
\end{pgfscope}%
\end{pgfscope}%
\begin{pgfscope}%
\definecolor{textcolor}{rgb}{0.000000,0.000000,0.000000}%
\pgfsetstrokecolor{textcolor}%
\pgfsetfillcolor{textcolor}%
\pgftext[x=0.475308in, y=1.687229in, left, base]{\color{textcolor}\rmfamily\fontsize{10.000000}{12.000000}\selectfont \(\displaystyle {0.4}\)}%
\end{pgfscope}%
\begin{pgfscope}%
\pgfsetbuttcap%
\pgfsetroundjoin%
\definecolor{currentfill}{rgb}{0.000000,0.000000,0.000000}%
\pgfsetfillcolor{currentfill}%
\pgfsetlinewidth{0.803000pt}%
\definecolor{currentstroke}{rgb}{0.000000,0.000000,0.000000}%
\pgfsetstrokecolor{currentstroke}%
\pgfsetdash{}{0pt}%
\pgfsys@defobject{currentmarker}{\pgfqpoint{-0.048611in}{0.000000in}}{\pgfqpoint{-0.000000in}{0.000000in}}{%
\pgfpathmoveto{\pgfqpoint{-0.000000in}{0.000000in}}%
\pgfpathlineto{\pgfqpoint{-0.048611in}{0.000000in}}%
\pgfusepath{stroke,fill}%
}%
\begin{pgfscope}%
\pgfsys@transformshift{0.750000in}{2.284545in}%
\pgfsys@useobject{currentmarker}{}%
\end{pgfscope}%
\end{pgfscope}%
\begin{pgfscope}%
\definecolor{textcolor}{rgb}{0.000000,0.000000,0.000000}%
\pgfsetstrokecolor{textcolor}%
\pgfsetfillcolor{textcolor}%
\pgftext[x=0.475308in, y=2.236320in, left, base]{\color{textcolor}\rmfamily\fontsize{10.000000}{12.000000}\selectfont \(\displaystyle {0.6}\)}%
\end{pgfscope}%
\begin{pgfscope}%
\pgfsetbuttcap%
\pgfsetroundjoin%
\definecolor{currentfill}{rgb}{0.000000,0.000000,0.000000}%
\pgfsetfillcolor{currentfill}%
\pgfsetlinewidth{0.803000pt}%
\definecolor{currentstroke}{rgb}{0.000000,0.000000,0.000000}%
\pgfsetstrokecolor{currentstroke}%
\pgfsetdash{}{0pt}%
\pgfsys@defobject{currentmarker}{\pgfqpoint{-0.048611in}{0.000000in}}{\pgfqpoint{-0.000000in}{0.000000in}}{%
\pgfpathmoveto{\pgfqpoint{-0.000000in}{0.000000in}}%
\pgfpathlineto{\pgfqpoint{-0.048611in}{0.000000in}}%
\pgfusepath{stroke,fill}%
}%
\begin{pgfscope}%
\pgfsys@transformshift{0.750000in}{2.833636in}%
\pgfsys@useobject{currentmarker}{}%
\end{pgfscope}%
\end{pgfscope}%
\begin{pgfscope}%
\definecolor{textcolor}{rgb}{0.000000,0.000000,0.000000}%
\pgfsetstrokecolor{textcolor}%
\pgfsetfillcolor{textcolor}%
\pgftext[x=0.475308in, y=2.785411in, left, base]{\color{textcolor}\rmfamily\fontsize{10.000000}{12.000000}\selectfont \(\displaystyle {0.8}\)}%
\end{pgfscope}%
\begin{pgfscope}%
\pgfsetbuttcap%
\pgfsetroundjoin%
\definecolor{currentfill}{rgb}{0.000000,0.000000,0.000000}%
\pgfsetfillcolor{currentfill}%
\pgfsetlinewidth{0.803000pt}%
\definecolor{currentstroke}{rgb}{0.000000,0.000000,0.000000}%
\pgfsetstrokecolor{currentstroke}%
\pgfsetdash{}{0pt}%
\pgfsys@defobject{currentmarker}{\pgfqpoint{-0.048611in}{0.000000in}}{\pgfqpoint{-0.000000in}{0.000000in}}{%
\pgfpathmoveto{\pgfqpoint{-0.000000in}{0.000000in}}%
\pgfpathlineto{\pgfqpoint{-0.048611in}{0.000000in}}%
\pgfusepath{stroke,fill}%
}%
\begin{pgfscope}%
\pgfsys@transformshift{0.750000in}{3.382727in}%
\pgfsys@useobject{currentmarker}{}%
\end{pgfscope}%
\end{pgfscope}%
\begin{pgfscope}%
\definecolor{textcolor}{rgb}{0.000000,0.000000,0.000000}%
\pgfsetstrokecolor{textcolor}%
\pgfsetfillcolor{textcolor}%
\pgftext[x=0.475308in, y=3.334502in, left, base]{\color{textcolor}\rmfamily\fontsize{10.000000}{12.000000}\selectfont \(\displaystyle {1.0}\)}%
\end{pgfscope}%
\begin{pgfscope}%
\definecolor{textcolor}{rgb}{0.000000,0.000000,0.000000}%
\pgfsetstrokecolor{textcolor}%
\pgfsetfillcolor{textcolor}%
\pgftext[x=0.419753in,y=2.010000in,,bottom,rotate=90.000000]{\color{textcolor}\rmfamily\fontsize{10.000000}{12.000000}\selectfont sus}%
\end{pgfscope}%
\begin{pgfscope}%
\pgfpathrectangle{\pgfqpoint{0.750000in}{0.500000in}}{\pgfqpoint{4.650000in}{3.020000in}}%
\pgfusepath{clip}%
\pgfsetrectcap%
\pgfsetroundjoin%
\pgfsetlinewidth{1.505625pt}%
\definecolor{currentstroke}{rgb}{0.121569,0.466667,0.705882}%
\pgfsetstrokecolor{currentstroke}%
\pgfsetdash{}{0pt}%
\pgfpathmoveto{\pgfqpoint{0.961364in}{0.637273in}}%
\pgfpathlineto{\pgfqpoint{0.961619in}{0.637273in}}%
\pgfpathlineto{\pgfqpoint{0.962941in}{3.382727in}}%
\pgfpathlineto{\pgfqpoint{0.963158in}{0.637273in}}%
\pgfpathlineto{\pgfqpoint{0.963328in}{0.637273in}}%
\pgfpathlineto{\pgfqpoint{0.964372in}{3.382727in}}%
\pgfpathlineto{\pgfqpoint{0.964868in}{0.637273in}}%
\pgfpathlineto{\pgfqpoint{0.964929in}{0.637273in}}%
\pgfpathlineto{\pgfqpoint{0.966473in}{3.382727in}}%
\pgfpathlineto{\pgfqpoint{0.966752in}{3.382727in}}%
\pgfpathlineto{\pgfqpoint{0.968296in}{0.637273in}}%
\pgfpathlineto{\pgfqpoint{0.968683in}{0.637273in}}%
\pgfpathlineto{\pgfqpoint{0.968693in}{3.382727in}}%
\pgfpathlineto{\pgfqpoint{0.970223in}{0.637273in}}%
\pgfpathlineto{\pgfqpoint{0.970327in}{0.637273in}}%
\pgfpathlineto{\pgfqpoint{0.970520in}{3.382727in}}%
\pgfpathlineto{\pgfqpoint{0.971866in}{0.637273in}}%
\pgfpathlineto{\pgfqpoint{0.971918in}{0.637273in}}%
\pgfpathlineto{\pgfqpoint{0.972716in}{3.382727in}}%
\pgfpathlineto{\pgfqpoint{0.973457in}{0.637273in}}%
\pgfpathlineto{\pgfqpoint{0.986151in}{0.637273in}}%
\pgfpathlineto{\pgfqpoint{0.986217in}{3.382727in}}%
\pgfpathlineto{\pgfqpoint{0.987691in}{0.637273in}}%
\pgfpathlineto{\pgfqpoint{0.998580in}{0.637273in}}%
\pgfpathlineto{\pgfqpoint{0.998627in}{3.382727in}}%
\pgfpathlineto{\pgfqpoint{1.000120in}{0.637273in}}%
\pgfpathlineto{\pgfqpoint{1.004308in}{0.637273in}}%
\pgfpathlineto{\pgfqpoint{1.004337in}{3.382727in}}%
\pgfpathlineto{\pgfqpoint{1.005848in}{0.637273in}}%
\pgfpathlineto{\pgfqpoint{1.015968in}{0.637273in}}%
\pgfpathlineto{\pgfqpoint{1.016015in}{3.382727in}}%
\pgfpathlineto{\pgfqpoint{1.017507in}{0.637273in}}%
\pgfpathlineto{\pgfqpoint{1.029034in}{0.637273in}}%
\pgfpathlineto{\pgfqpoint{1.029072in}{3.382727in}}%
\pgfpathlineto{\pgfqpoint{1.030574in}{0.637273in}}%
\pgfpathlineto{\pgfqpoint{1.039981in}{0.637273in}}%
\pgfpathlineto{\pgfqpoint{1.040047in}{3.382727in}}%
\pgfpathlineto{\pgfqpoint{1.041520in}{0.637273in}}%
\pgfpathlineto{\pgfqpoint{1.050804in}{0.637273in}}%
\pgfpathlineto{\pgfqpoint{1.050832in}{3.382727in}}%
\pgfpathlineto{\pgfqpoint{1.052344in}{0.637273in}}%
\pgfpathlineto{\pgfqpoint{1.074793in}{0.637273in}}%
\pgfpathlineto{\pgfqpoint{1.074841in}{3.382727in}}%
\pgfpathlineto{\pgfqpoint{1.076333in}{0.637273in}}%
\pgfpathlineto{\pgfqpoint{1.085825in}{0.637273in}}%
\pgfpathlineto{\pgfqpoint{1.085853in}{3.382727in}}%
\pgfpathlineto{\pgfqpoint{1.087364in}{0.637273in}}%
\pgfpathlineto{\pgfqpoint{1.096941in}{0.637273in}}%
\pgfpathlineto{\pgfqpoint{1.096951in}{3.382727in}}%
\pgfpathlineto{\pgfqpoint{1.098481in}{0.637273in}}%
\pgfpathlineto{\pgfqpoint{1.107812in}{0.637273in}}%
\pgfpathlineto{\pgfqpoint{1.109153in}{3.382727in}}%
\pgfpathlineto{\pgfqpoint{1.109351in}{0.637273in}}%
\pgfpathlineto{\pgfqpoint{1.109455in}{0.637273in}}%
\pgfpathlineto{\pgfqpoint{1.110999in}{3.382727in}}%
\pgfpathlineto{\pgfqpoint{1.111160in}{0.637273in}}%
\pgfpathlineto{\pgfqpoint{1.112539in}{3.382727in}}%
\pgfpathlineto{\pgfqpoint{1.113946in}{3.382727in}}%
\pgfpathlineto{\pgfqpoint{1.115490in}{0.637273in}}%
\pgfpathlineto{\pgfqpoint{1.115566in}{0.637273in}}%
\pgfpathlineto{\pgfqpoint{1.117105in}{3.382727in}}%
\pgfpathlineto{\pgfqpoint{1.118650in}{0.637273in}}%
\pgfpathlineto{\pgfqpoint{1.120194in}{3.382727in}}%
\pgfpathlineto{\pgfqpoint{1.121157in}{3.382727in}}%
\pgfpathlineto{\pgfqpoint{1.122701in}{0.637273in}}%
\pgfpathlineto{\pgfqpoint{1.126904in}{0.637273in}}%
\pgfpathlineto{\pgfqpoint{1.126951in}{3.382727in}}%
\pgfpathlineto{\pgfqpoint{1.128444in}{0.637273in}}%
\pgfpathlineto{\pgfqpoint{1.138946in}{0.637273in}}%
\pgfpathlineto{\pgfqpoint{1.138993in}{3.382727in}}%
\pgfpathlineto{\pgfqpoint{1.140485in}{0.637273in}}%
\pgfpathlineto{\pgfqpoint{1.149883in}{0.637273in}}%
\pgfpathlineto{\pgfqpoint{1.149930in}{3.382727in}}%
\pgfpathlineto{\pgfqpoint{1.151422in}{0.637273in}}%
\pgfpathlineto{\pgfqpoint{1.162029in}{0.637273in}}%
\pgfpathlineto{\pgfqpoint{1.162057in}{3.382727in}}%
\pgfpathlineto{\pgfqpoint{1.163568in}{0.637273in}}%
\pgfpathlineto{\pgfqpoint{1.172951in}{0.637273in}}%
\pgfpathlineto{\pgfqpoint{1.172980in}{3.382727in}}%
\pgfpathlineto{\pgfqpoint{1.174491in}{0.637273in}}%
\pgfpathlineto{\pgfqpoint{1.183784in}{0.637273in}}%
\pgfpathlineto{\pgfqpoint{1.183813in}{3.382727in}}%
\pgfpathlineto{\pgfqpoint{1.185324in}{0.637273in}}%
\pgfpathlineto{\pgfqpoint{1.196237in}{0.637273in}}%
\pgfpathlineto{\pgfqpoint{1.196265in}{3.382727in}}%
\pgfpathlineto{\pgfqpoint{1.197777in}{0.637273in}}%
\pgfpathlineto{\pgfqpoint{1.207160in}{0.637273in}}%
\pgfpathlineto{\pgfqpoint{1.207188in}{3.382727in}}%
\pgfpathlineto{\pgfqpoint{1.208699in}{0.637273in}}%
\pgfpathlineto{\pgfqpoint{1.218285in}{0.637273in}}%
\pgfpathlineto{\pgfqpoint{1.218314in}{3.382727in}}%
\pgfpathlineto{\pgfqpoint{1.219825in}{0.637273in}}%
\pgfpathlineto{\pgfqpoint{1.229203in}{0.637273in}}%
\pgfpathlineto{\pgfqpoint{1.229232in}{3.382727in}}%
\pgfpathlineto{\pgfqpoint{1.230743in}{0.637273in}}%
\pgfpathlineto{\pgfqpoint{1.240060in}{0.637273in}}%
\pgfpathlineto{\pgfqpoint{1.240088in}{3.382727in}}%
\pgfpathlineto{\pgfqpoint{1.241600in}{0.637273in}}%
\pgfpathlineto{\pgfqpoint{1.251427in}{0.637273in}}%
\pgfpathlineto{\pgfqpoint{1.251455in}{3.382727in}}%
\pgfpathlineto{\pgfqpoint{1.252966in}{0.637273in}}%
\pgfpathlineto{\pgfqpoint{1.262279in}{0.637273in}}%
\pgfpathlineto{\pgfqpoint{1.262302in}{3.382727in}}%
\pgfpathlineto{\pgfqpoint{1.263818in}{0.637273in}}%
\pgfpathlineto{\pgfqpoint{1.273329in}{0.637273in}}%
\pgfpathlineto{\pgfqpoint{1.273357in}{3.382727in}}%
\pgfpathlineto{\pgfqpoint{1.274868in}{0.637273in}}%
\pgfpathlineto{\pgfqpoint{1.284171in}{0.637273in}}%
\pgfpathlineto{\pgfqpoint{1.284199in}{3.382727in}}%
\pgfpathlineto{\pgfqpoint{1.285711in}{0.637273in}}%
\pgfpathlineto{\pgfqpoint{1.295080in}{0.637273in}}%
\pgfpathlineto{\pgfqpoint{1.295108in}{3.382727in}}%
\pgfpathlineto{\pgfqpoint{1.296619in}{0.637273in}}%
\pgfpathlineto{\pgfqpoint{1.307122in}{0.637273in}}%
\pgfpathlineto{\pgfqpoint{1.307150in}{3.382727in}}%
\pgfpathlineto{\pgfqpoint{1.308661in}{0.637273in}}%
\pgfpathlineto{\pgfqpoint{1.318044in}{0.637273in}}%
\pgfpathlineto{\pgfqpoint{1.318073in}{3.382727in}}%
\pgfpathlineto{\pgfqpoint{1.319584in}{0.637273in}}%
\pgfpathlineto{\pgfqpoint{1.330190in}{0.637273in}}%
\pgfpathlineto{\pgfqpoint{1.330218in}{3.382727in}}%
\pgfpathlineto{\pgfqpoint{1.331729in}{0.637273in}}%
\pgfpathlineto{\pgfqpoint{1.341037in}{0.637273in}}%
\pgfpathlineto{\pgfqpoint{1.341065in}{3.382727in}}%
\pgfpathlineto{\pgfqpoint{1.342577in}{0.637273in}}%
\pgfpathlineto{\pgfqpoint{1.351946in}{0.637273in}}%
\pgfpathlineto{\pgfqpoint{1.351974in}{3.382727in}}%
\pgfpathlineto{\pgfqpoint{1.353485in}{0.637273in}}%
\pgfpathlineto{\pgfqpoint{1.363076in}{0.637273in}}%
\pgfpathlineto{\pgfqpoint{1.363105in}{3.382727in}}%
\pgfpathlineto{\pgfqpoint{1.364616in}{0.637273in}}%
\pgfpathlineto{\pgfqpoint{1.373952in}{0.637273in}}%
\pgfpathlineto{\pgfqpoint{1.373961in}{3.382727in}}%
\pgfpathlineto{\pgfqpoint{1.375491in}{0.637273in}}%
\pgfpathlineto{\pgfqpoint{1.381545in}{0.637273in}}%
\pgfpathlineto{\pgfqpoint{1.381573in}{3.382727in}}%
\pgfpathlineto{\pgfqpoint{1.383085in}{0.637273in}}%
\pgfpathlineto{\pgfqpoint{1.392463in}{0.637273in}}%
\pgfpathlineto{\pgfqpoint{1.392491in}{3.382727in}}%
\pgfpathlineto{\pgfqpoint{1.394003in}{0.637273in}}%
\pgfpathlineto{\pgfqpoint{1.403334in}{0.637273in}}%
\pgfpathlineto{\pgfqpoint{1.403362in}{3.382727in}}%
\pgfpathlineto{\pgfqpoint{1.404873in}{0.637273in}}%
\pgfpathlineto{\pgfqpoint{1.415395in}{0.637273in}}%
\pgfpathlineto{\pgfqpoint{1.415423in}{3.382727in}}%
\pgfpathlineto{\pgfqpoint{1.416934in}{0.637273in}}%
\pgfpathlineto{\pgfqpoint{1.426341in}{0.637273in}}%
\pgfpathlineto{\pgfqpoint{1.426369in}{3.382727in}}%
\pgfpathlineto{\pgfqpoint{1.427880in}{0.637273in}}%
\pgfpathlineto{\pgfqpoint{1.437443in}{0.637273in}}%
\pgfpathlineto{\pgfqpoint{1.437471in}{3.382727in}}%
\pgfpathlineto{\pgfqpoint{1.438983in}{0.637273in}}%
\pgfpathlineto{\pgfqpoint{1.448333in}{0.637273in}}%
\pgfpathlineto{\pgfqpoint{1.448356in}{3.382727in}}%
\pgfpathlineto{\pgfqpoint{1.449872in}{0.637273in}}%
\pgfpathlineto{\pgfqpoint{1.459213in}{0.637273in}}%
\pgfpathlineto{\pgfqpoint{1.459241in}{3.382727in}}%
\pgfpathlineto{\pgfqpoint{1.460752in}{0.637273in}}%
\pgfpathlineto{\pgfqpoint{1.470329in}{0.637273in}}%
\pgfpathlineto{\pgfqpoint{1.470358in}{3.382727in}}%
\pgfpathlineto{\pgfqpoint{1.471869in}{0.637273in}}%
\pgfpathlineto{\pgfqpoint{1.481124in}{0.637273in}}%
\pgfpathlineto{\pgfqpoint{1.481153in}{3.382727in}}%
\pgfpathlineto{\pgfqpoint{1.482664in}{0.637273in}}%
\pgfpathlineto{\pgfqpoint{1.487618in}{0.637273in}}%
\pgfpathlineto{\pgfqpoint{1.489053in}{3.382727in}}%
\pgfpathlineto{\pgfqpoint{1.489157in}{0.637273in}}%
\pgfpathlineto{\pgfqpoint{1.494734in}{0.637273in}}%
\pgfpathlineto{\pgfqpoint{1.494762in}{3.382727in}}%
\pgfpathlineto{\pgfqpoint{1.496274in}{0.637273in}}%
\pgfpathlineto{\pgfqpoint{1.505529in}{0.637273in}}%
\pgfpathlineto{\pgfqpoint{1.505558in}{3.382727in}}%
\pgfpathlineto{\pgfqpoint{1.507069in}{0.637273in}}%
\pgfpathlineto{\pgfqpoint{1.516438in}{0.637273in}}%
\pgfpathlineto{\pgfqpoint{1.516466in}{3.382727in}}%
\pgfpathlineto{\pgfqpoint{1.517977in}{0.637273in}}%
\pgfpathlineto{\pgfqpoint{1.528513in}{0.637273in}}%
\pgfpathlineto{\pgfqpoint{1.528541in}{3.382727in}}%
\pgfpathlineto{\pgfqpoint{1.530052in}{0.637273in}}%
\pgfpathlineto{\pgfqpoint{1.539388in}{0.637273in}}%
\pgfpathlineto{\pgfqpoint{1.540829in}{3.382727in}}%
\pgfpathlineto{\pgfqpoint{1.540928in}{0.637273in}}%
\pgfpathlineto{\pgfqpoint{1.551879in}{0.637273in}}%
\pgfpathlineto{\pgfqpoint{1.551907in}{3.382727in}}%
\pgfpathlineto{\pgfqpoint{1.553418in}{0.637273in}}%
\pgfpathlineto{\pgfqpoint{1.562797in}{0.637273in}}%
\pgfpathlineto{\pgfqpoint{1.562825in}{3.382727in}}%
\pgfpathlineto{\pgfqpoint{1.564336in}{0.637273in}}%
\pgfpathlineto{\pgfqpoint{1.573677in}{0.637273in}}%
\pgfpathlineto{\pgfqpoint{1.573705in}{3.382727in}}%
\pgfpathlineto{\pgfqpoint{1.575216in}{0.637273in}}%
\pgfpathlineto{\pgfqpoint{1.584765in}{0.637273in}}%
\pgfpathlineto{\pgfqpoint{1.584793in}{3.382727in}}%
\pgfpathlineto{\pgfqpoint{1.586304in}{0.637273in}}%
\pgfpathlineto{\pgfqpoint{1.595636in}{0.637273in}}%
\pgfpathlineto{\pgfqpoint{1.595664in}{3.382727in}}%
\pgfpathlineto{\pgfqpoint{1.597175in}{0.637273in}}%
\pgfpathlineto{\pgfqpoint{1.606766in}{0.637273in}}%
\pgfpathlineto{\pgfqpoint{1.606794in}{3.382727in}}%
\pgfpathlineto{\pgfqpoint{1.608306in}{0.637273in}}%
\pgfpathlineto{\pgfqpoint{1.617717in}{0.637273in}}%
\pgfpathlineto{\pgfqpoint{1.617745in}{3.382727in}}%
\pgfpathlineto{\pgfqpoint{1.619257in}{0.637273in}}%
\pgfpathlineto{\pgfqpoint{1.628597in}{0.637273in}}%
\pgfpathlineto{\pgfqpoint{1.628607in}{3.382727in}}%
\pgfpathlineto{\pgfqpoint{1.630137in}{0.637273in}}%
\pgfpathlineto{\pgfqpoint{1.639728in}{0.637273in}}%
\pgfpathlineto{\pgfqpoint{1.639737in}{3.382727in}}%
\pgfpathlineto{\pgfqpoint{1.641267in}{0.637273in}}%
\pgfpathlineto{\pgfqpoint{1.645253in}{0.637273in}}%
\pgfpathlineto{\pgfqpoint{1.646674in}{3.382727in}}%
\pgfpathlineto{\pgfqpoint{1.646792in}{0.637273in}}%
\pgfpathlineto{\pgfqpoint{1.646939in}{0.637273in}}%
\pgfpathlineto{\pgfqpoint{1.647099in}{3.382727in}}%
\pgfpathlineto{\pgfqpoint{1.648478in}{0.637273in}}%
\pgfpathlineto{\pgfqpoint{1.659188in}{0.637273in}}%
\pgfpathlineto{\pgfqpoint{1.659217in}{3.382727in}}%
\pgfpathlineto{\pgfqpoint{1.660728in}{0.637273in}}%
\pgfpathlineto{\pgfqpoint{1.670078in}{0.637273in}}%
\pgfpathlineto{\pgfqpoint{1.670106in}{3.382727in}}%
\pgfpathlineto{\pgfqpoint{1.671618in}{0.637273in}}%
\pgfpathlineto{\pgfqpoint{1.681336in}{0.637273in}}%
\pgfpathlineto{\pgfqpoint{1.681364in}{3.382727in}}%
\pgfpathlineto{\pgfqpoint{1.682876in}{0.637273in}}%
\pgfpathlineto{\pgfqpoint{1.686653in}{0.637273in}}%
\pgfpathlineto{\pgfqpoint{1.688146in}{3.382727in}}%
\pgfpathlineto{\pgfqpoint{1.688193in}{0.637273in}}%
\pgfpathlineto{\pgfqpoint{1.688476in}{0.637273in}}%
\pgfpathlineto{\pgfqpoint{1.689855in}{3.382727in}}%
\pgfpathlineto{\pgfqpoint{1.690016in}{0.637273in}}%
\pgfpathlineto{\pgfqpoint{1.690020in}{0.637273in}}%
\pgfpathlineto{\pgfqpoint{1.691565in}{3.382727in}}%
\pgfpathlineto{\pgfqpoint{1.691569in}{3.382727in}}%
\pgfpathlineto{\pgfqpoint{1.691990in}{0.637273in}}%
\pgfpathlineto{\pgfqpoint{1.693109in}{3.382727in}}%
\pgfpathlineto{\pgfqpoint{1.693369in}{3.382727in}}%
\pgfpathlineto{\pgfqpoint{1.694913in}{0.637273in}}%
\pgfpathlineto{\pgfqpoint{1.698988in}{0.637273in}}%
\pgfpathlineto{\pgfqpoint{1.700367in}{3.382727in}}%
\pgfpathlineto{\pgfqpoint{1.700528in}{0.637273in}}%
\pgfpathlineto{\pgfqpoint{1.700594in}{0.637273in}}%
\pgfpathlineto{\pgfqpoint{1.702110in}{3.382727in}}%
\pgfpathlineto{\pgfqpoint{1.702133in}{0.637273in}}%
\pgfpathlineto{\pgfqpoint{1.702180in}{0.637273in}}%
\pgfpathlineto{\pgfqpoint{1.702520in}{3.382727in}}%
\pgfpathlineto{\pgfqpoint{1.703720in}{0.637273in}}%
\pgfpathlineto{\pgfqpoint{1.713377in}{0.637273in}}%
\pgfpathlineto{\pgfqpoint{1.713419in}{3.382727in}}%
\pgfpathlineto{\pgfqpoint{1.714916in}{0.637273in}}%
\pgfpathlineto{\pgfqpoint{1.726184in}{0.637273in}}%
\pgfpathlineto{\pgfqpoint{1.726231in}{3.382727in}}%
\pgfpathlineto{\pgfqpoint{1.727723in}{0.637273in}}%
\pgfpathlineto{\pgfqpoint{1.737007in}{0.637273in}}%
\pgfpathlineto{\pgfqpoint{1.737055in}{3.382727in}}%
\pgfpathlineto{\pgfqpoint{1.738547in}{0.637273in}}%
\pgfpathlineto{\pgfqpoint{1.747855in}{0.637273in}}%
\pgfpathlineto{\pgfqpoint{1.747902in}{3.382727in}}%
\pgfpathlineto{\pgfqpoint{1.749394in}{0.637273in}}%
\pgfpathlineto{\pgfqpoint{1.759934in}{0.637273in}}%
\pgfpathlineto{\pgfqpoint{1.759981in}{3.382727in}}%
\pgfpathlineto{\pgfqpoint{1.761474in}{0.637273in}}%
\pgfpathlineto{\pgfqpoint{1.770838in}{0.637273in}}%
\pgfpathlineto{\pgfqpoint{1.770885in}{3.382727in}}%
\pgfpathlineto{\pgfqpoint{1.772377in}{0.637273in}}%
\pgfpathlineto{\pgfqpoint{1.781888in}{0.637273in}}%
\pgfpathlineto{\pgfqpoint{1.781954in}{3.382727in}}%
\pgfpathlineto{\pgfqpoint{1.783428in}{0.637273in}}%
\pgfpathlineto{\pgfqpoint{1.792806in}{0.637273in}}%
\pgfpathlineto{\pgfqpoint{1.792853in}{3.382727in}}%
\pgfpathlineto{\pgfqpoint{1.794346in}{0.637273in}}%
\pgfpathlineto{\pgfqpoint{1.803686in}{0.637273in}}%
\pgfpathlineto{\pgfqpoint{1.803752in}{3.382727in}}%
\pgfpathlineto{\pgfqpoint{1.805226in}{0.637273in}}%
\pgfpathlineto{\pgfqpoint{1.814774in}{0.637273in}}%
\pgfpathlineto{\pgfqpoint{1.814840in}{3.382727in}}%
\pgfpathlineto{\pgfqpoint{1.816314in}{0.637273in}}%
\pgfpathlineto{\pgfqpoint{1.825673in}{0.637273in}}%
\pgfpathlineto{\pgfqpoint{1.825740in}{3.382727in}}%
\pgfpathlineto{\pgfqpoint{1.827213in}{0.637273in}}%
\pgfpathlineto{\pgfqpoint{1.837819in}{0.637273in}}%
\pgfpathlineto{\pgfqpoint{1.837866in}{3.382727in}}%
\pgfpathlineto{\pgfqpoint{1.839359in}{0.637273in}}%
\pgfpathlineto{\pgfqpoint{1.848737in}{0.637273in}}%
\pgfpathlineto{\pgfqpoint{1.848784in}{3.382727in}}%
\pgfpathlineto{\pgfqpoint{1.850277in}{0.637273in}}%
\pgfpathlineto{\pgfqpoint{1.859547in}{0.637273in}}%
\pgfpathlineto{\pgfqpoint{1.859594in}{3.382727in}}%
\pgfpathlineto{\pgfqpoint{1.861086in}{0.637273in}}%
\pgfpathlineto{\pgfqpoint{1.870705in}{0.637273in}}%
\pgfpathlineto{\pgfqpoint{1.870753in}{3.382727in}}%
\pgfpathlineto{\pgfqpoint{1.872245in}{0.637273in}}%
\pgfpathlineto{\pgfqpoint{1.881581in}{0.637273in}}%
\pgfpathlineto{\pgfqpoint{1.881628in}{3.382727in}}%
\pgfpathlineto{\pgfqpoint{1.883120in}{0.637273in}}%
\pgfpathlineto{\pgfqpoint{1.892730in}{0.637273in}}%
\pgfpathlineto{\pgfqpoint{1.892740in}{3.382727in}}%
\pgfpathlineto{\pgfqpoint{1.894270in}{0.637273in}}%
\pgfpathlineto{\pgfqpoint{1.903629in}{0.637273in}}%
\pgfpathlineto{\pgfqpoint{1.903639in}{3.382727in}}%
\pgfpathlineto{\pgfqpoint{1.905169in}{0.637273in}}%
\pgfpathlineto{\pgfqpoint{1.909178in}{0.637273in}}%
\pgfpathlineto{\pgfqpoint{1.909225in}{3.382727in}}%
\pgfpathlineto{\pgfqpoint{1.910717in}{0.637273in}}%
\pgfpathlineto{\pgfqpoint{1.920261in}{0.637273in}}%
\pgfpathlineto{\pgfqpoint{1.920308in}{3.382727in}}%
\pgfpathlineto{\pgfqpoint{1.921801in}{0.637273in}}%
\pgfpathlineto{\pgfqpoint{1.925673in}{0.637273in}}%
\pgfpathlineto{\pgfqpoint{1.927203in}{3.382727in}}%
\pgfpathlineto{\pgfqpoint{1.927212in}{0.637273in}}%
\pgfpathlineto{\pgfqpoint{1.927467in}{0.637273in}}%
\pgfpathlineto{\pgfqpoint{1.928946in}{3.382727in}}%
\pgfpathlineto{\pgfqpoint{1.929007in}{0.637273in}}%
\pgfpathlineto{\pgfqpoint{1.929064in}{0.637273in}}%
\pgfpathlineto{\pgfqpoint{1.930532in}{3.382727in}}%
\pgfpathlineto{\pgfqpoint{1.930603in}{0.637273in}}%
\pgfpathlineto{\pgfqpoint{1.930679in}{0.637273in}}%
\pgfpathlineto{\pgfqpoint{1.932223in}{3.382727in}}%
\pgfpathlineto{\pgfqpoint{1.932563in}{3.382727in}}%
\pgfpathlineto{\pgfqpoint{1.934102in}{0.637273in}}%
\pgfpathlineto{\pgfqpoint{1.935647in}{3.382727in}}%
\pgfpathlineto{\pgfqpoint{1.937129in}{0.637273in}}%
\pgfpathlineto{\pgfqpoint{1.937186in}{3.382727in}}%
\pgfpathlineto{\pgfqpoint{1.937191in}{3.382727in}}%
\pgfpathlineto{\pgfqpoint{1.938735in}{0.637273in}}%
\pgfpathlineto{\pgfqpoint{1.940274in}{3.382727in}}%
\pgfpathlineto{\pgfqpoint{1.941804in}{0.637273in}}%
\pgfpathlineto{\pgfqpoint{1.941814in}{3.382727in}}%
\pgfpathlineto{\pgfqpoint{1.941819in}{3.382727in}}%
\pgfpathlineto{\pgfqpoint{1.943297in}{0.637273in}}%
\pgfpathlineto{\pgfqpoint{1.943358in}{3.382727in}}%
\pgfpathlineto{\pgfqpoint{1.944638in}{3.382727in}}%
\pgfpathlineto{\pgfqpoint{1.946182in}{0.637273in}}%
\pgfpathlineto{\pgfqpoint{1.946602in}{3.382727in}}%
\pgfpathlineto{\pgfqpoint{1.947721in}{0.637273in}}%
\pgfpathlineto{\pgfqpoint{1.950607in}{0.637273in}}%
\pgfpathlineto{\pgfqpoint{1.951948in}{3.382727in}}%
\pgfpathlineto{\pgfqpoint{1.952146in}{0.637273in}}%
\pgfpathlineto{\pgfqpoint{1.952217in}{0.637273in}}%
\pgfpathlineto{\pgfqpoint{1.953761in}{3.382727in}}%
\pgfpathlineto{\pgfqpoint{1.954007in}{3.382727in}}%
\pgfpathlineto{\pgfqpoint{1.955348in}{0.637273in}}%
\pgfpathlineto{\pgfqpoint{1.955546in}{3.382727in}}%
\pgfpathlineto{\pgfqpoint{1.956689in}{3.382727in}}%
\pgfpathlineto{\pgfqpoint{1.958233in}{0.637273in}}%
\pgfpathlineto{\pgfqpoint{1.959763in}{3.382727in}}%
\pgfpathlineto{\pgfqpoint{1.959773in}{0.637273in}}%
\pgfpathlineto{\pgfqpoint{1.960713in}{0.637273in}}%
\pgfpathlineto{\pgfqpoint{1.962257in}{3.382727in}}%
\pgfpathlineto{\pgfqpoint{1.962290in}{3.382727in}}%
\pgfpathlineto{\pgfqpoint{1.963834in}{0.637273in}}%
\pgfpathlineto{\pgfqpoint{1.964859in}{0.637273in}}%
\pgfpathlineto{\pgfqpoint{1.966403in}{3.382727in}}%
\pgfpathlineto{\pgfqpoint{1.966955in}{3.382727in}}%
\pgfpathlineto{\pgfqpoint{1.967196in}{0.637273in}}%
\pgfpathlineto{\pgfqpoint{1.968495in}{3.382727in}}%
\pgfpathlineto{\pgfqpoint{1.981486in}{3.382727in}}%
\pgfpathlineto{\pgfqpoint{1.981580in}{0.637273in}}%
\pgfpathlineto{\pgfqpoint{1.983025in}{3.382727in}}%
\pgfpathlineto{\pgfqpoint{1.985410in}{3.382727in}}%
\pgfpathlineto{\pgfqpoint{1.986954in}{0.637273in}}%
\pgfpathlineto{\pgfqpoint{1.986969in}{0.637273in}}%
\pgfpathlineto{\pgfqpoint{1.988513in}{3.382727in}}%
\pgfpathlineto{\pgfqpoint{1.988527in}{3.382727in}}%
\pgfpathlineto{\pgfqpoint{1.989854in}{0.637273in}}%
\pgfpathlineto{\pgfqpoint{1.990066in}{3.382727in}}%
\pgfpathlineto{\pgfqpoint{2.049959in}{3.382727in}}%
\pgfpathlineto{\pgfqpoint{2.051267in}{0.637273in}}%
\pgfpathlineto{\pgfqpoint{2.051499in}{3.382727in}}%
\pgfpathlineto{\pgfqpoint{2.053227in}{3.382727in}}%
\pgfpathlineto{\pgfqpoint{2.054771in}{0.637273in}}%
\pgfpathlineto{\pgfqpoint{2.055315in}{0.637273in}}%
\pgfpathlineto{\pgfqpoint{2.056859in}{3.382727in}}%
\pgfpathlineto{\pgfqpoint{2.074435in}{3.382727in}}%
\pgfpathlineto{\pgfqpoint{2.075979in}{0.637273in}}%
\pgfpathlineto{\pgfqpoint{2.080338in}{0.637273in}}%
\pgfpathlineto{\pgfqpoint{2.081849in}{3.382727in}}%
\pgfpathlineto{\pgfqpoint{2.081877in}{0.637273in}}%
\pgfpathlineto{\pgfqpoint{2.081948in}{0.637273in}}%
\pgfpathlineto{\pgfqpoint{2.083190in}{3.382727in}}%
\pgfpathlineto{\pgfqpoint{2.083488in}{0.637273in}}%
\pgfpathlineto{\pgfqpoint{2.084092in}{0.637273in}}%
\pgfpathlineto{\pgfqpoint{2.085636in}{3.382727in}}%
\pgfpathlineto{\pgfqpoint{2.086288in}{3.382727in}}%
\pgfpathlineto{\pgfqpoint{2.087832in}{0.637273in}}%
\pgfpathlineto{\pgfqpoint{2.095478in}{0.637273in}}%
\pgfpathlineto{\pgfqpoint{2.095525in}{3.382727in}}%
\pgfpathlineto{\pgfqpoint{2.097017in}{0.637273in}}%
\pgfpathlineto{\pgfqpoint{2.106330in}{0.637273in}}%
\pgfpathlineto{\pgfqpoint{2.107869in}{3.382727in}}%
\pgfpathlineto{\pgfqpoint{2.109413in}{0.637273in}}%
\pgfpathlineto{\pgfqpoint{2.118938in}{0.637273in}}%
\pgfpathlineto{\pgfqpoint{2.119004in}{3.382727in}}%
\pgfpathlineto{\pgfqpoint{2.120478in}{0.637273in}}%
\pgfpathlineto{\pgfqpoint{2.131004in}{0.637273in}}%
\pgfpathlineto{\pgfqpoint{2.131070in}{3.382727in}}%
\pgfpathlineto{\pgfqpoint{2.132543in}{0.637273in}}%
\pgfpathlineto{\pgfqpoint{2.141922in}{0.637273in}}%
\pgfpathlineto{\pgfqpoint{2.141988in}{3.382727in}}%
\pgfpathlineto{\pgfqpoint{2.143461in}{0.637273in}}%
\pgfpathlineto{\pgfqpoint{2.154115in}{0.637273in}}%
\pgfpathlineto{\pgfqpoint{2.154162in}{3.382727in}}%
\pgfpathlineto{\pgfqpoint{2.155654in}{0.637273in}}%
\pgfpathlineto{\pgfqpoint{2.165009in}{0.637273in}}%
\pgfpathlineto{\pgfqpoint{2.165037in}{3.382727in}}%
\pgfpathlineto{\pgfqpoint{2.166548in}{0.637273in}}%
\pgfpathlineto{\pgfqpoint{2.175889in}{0.637273in}}%
\pgfpathlineto{\pgfqpoint{2.175918in}{3.382727in}}%
\pgfpathlineto{\pgfqpoint{2.177429in}{0.637273in}}%
\pgfpathlineto{\pgfqpoint{2.187020in}{0.637273in}}%
\pgfpathlineto{\pgfqpoint{2.187048in}{3.382727in}}%
\pgfpathlineto{\pgfqpoint{2.188559in}{0.637273in}}%
\pgfpathlineto{\pgfqpoint{2.197895in}{0.637273in}}%
\pgfpathlineto{\pgfqpoint{2.197923in}{3.382727in}}%
\pgfpathlineto{\pgfqpoint{2.199435in}{0.637273in}}%
\pgfpathlineto{\pgfqpoint{2.210768in}{0.637273in}}%
\pgfpathlineto{\pgfqpoint{2.210796in}{3.382727in}}%
\pgfpathlineto{\pgfqpoint{2.212308in}{0.637273in}}%
\pgfpathlineto{\pgfqpoint{2.221714in}{0.637273in}}%
\pgfpathlineto{\pgfqpoint{2.221743in}{3.382727in}}%
\pgfpathlineto{\pgfqpoint{2.223254in}{0.637273in}}%
\pgfpathlineto{\pgfqpoint{2.232566in}{0.637273in}}%
\pgfpathlineto{\pgfqpoint{2.232595in}{3.382727in}}%
\pgfpathlineto{\pgfqpoint{2.234106in}{0.637273in}}%
\pgfpathlineto{\pgfqpoint{2.243697in}{0.637273in}}%
\pgfpathlineto{\pgfqpoint{2.243725in}{3.382727in}}%
\pgfpathlineto{\pgfqpoint{2.245236in}{0.637273in}}%
\pgfpathlineto{\pgfqpoint{2.249118in}{0.637273in}}%
\pgfpathlineto{\pgfqpoint{2.250596in}{3.382727in}}%
\pgfpathlineto{\pgfqpoint{2.250657in}{0.637273in}}%
\pgfpathlineto{\pgfqpoint{2.250700in}{0.637273in}}%
\pgfpathlineto{\pgfqpoint{2.250851in}{3.382727in}}%
\pgfpathlineto{\pgfqpoint{2.252239in}{0.637273in}}%
\pgfpathlineto{\pgfqpoint{2.256357in}{0.637273in}}%
\pgfpathlineto{\pgfqpoint{2.256386in}{3.382727in}}%
\pgfpathlineto{\pgfqpoint{2.257897in}{0.637273in}}%
\pgfpathlineto{\pgfqpoint{2.267445in}{0.637273in}}%
\pgfpathlineto{\pgfqpoint{2.267474in}{3.382727in}}%
\pgfpathlineto{\pgfqpoint{2.268985in}{0.637273in}}%
\pgfpathlineto{\pgfqpoint{2.278316in}{0.637273in}}%
\pgfpathlineto{\pgfqpoint{2.278344in}{3.382727in}}%
\pgfpathlineto{\pgfqpoint{2.279856in}{0.637273in}}%
\pgfpathlineto{\pgfqpoint{2.289210in}{0.637273in}}%
\pgfpathlineto{\pgfqpoint{2.289239in}{3.382727in}}%
\pgfpathlineto{\pgfqpoint{2.290750in}{0.637273in}}%
\pgfpathlineto{\pgfqpoint{2.300341in}{0.637273in}}%
\pgfpathlineto{\pgfqpoint{2.300350in}{3.382727in}}%
\pgfpathlineto{\pgfqpoint{2.301880in}{0.637273in}}%
\pgfpathlineto{\pgfqpoint{2.311169in}{0.637273in}}%
\pgfpathlineto{\pgfqpoint{2.311216in}{3.382727in}}%
\pgfpathlineto{\pgfqpoint{2.312709in}{0.637273in}}%
\pgfpathlineto{\pgfqpoint{2.323249in}{0.637273in}}%
\pgfpathlineto{\pgfqpoint{2.323258in}{3.382727in}}%
\pgfpathlineto{\pgfqpoint{2.324788in}{0.637273in}}%
\pgfpathlineto{\pgfqpoint{2.328797in}{0.637273in}}%
\pgfpathlineto{\pgfqpoint{2.328826in}{3.382727in}}%
\pgfpathlineto{\pgfqpoint{2.330337in}{0.637273in}}%
\pgfpathlineto{\pgfqpoint{2.339588in}{0.637273in}}%
\pgfpathlineto{\pgfqpoint{2.339612in}{3.382727in}}%
\pgfpathlineto{\pgfqpoint{2.341127in}{0.637273in}}%
\pgfpathlineto{\pgfqpoint{2.351521in}{0.637273in}}%
\pgfpathlineto{\pgfqpoint{2.351550in}{3.382727in}}%
\pgfpathlineto{\pgfqpoint{2.353061in}{0.637273in}}%
\pgfpathlineto{\pgfqpoint{2.362392in}{0.637273in}}%
\pgfpathlineto{\pgfqpoint{2.362420in}{3.382727in}}%
\pgfpathlineto{\pgfqpoint{2.363931in}{0.637273in}}%
\pgfpathlineto{\pgfqpoint{2.373537in}{0.637273in}}%
\pgfpathlineto{\pgfqpoint{2.373565in}{3.382727in}}%
\pgfpathlineto{\pgfqpoint{2.375076in}{0.637273in}}%
\pgfpathlineto{\pgfqpoint{2.378958in}{0.637273in}}%
\pgfpathlineto{\pgfqpoint{2.380370in}{3.382727in}}%
\pgfpathlineto{\pgfqpoint{2.380497in}{0.637273in}}%
\pgfpathlineto{\pgfqpoint{2.385919in}{0.637273in}}%
\pgfpathlineto{\pgfqpoint{2.385947in}{3.382727in}}%
\pgfpathlineto{\pgfqpoint{2.387458in}{0.637273in}}%
\pgfpathlineto{\pgfqpoint{2.396704in}{0.637273in}}%
\pgfpathlineto{\pgfqpoint{2.396733in}{3.382727in}}%
\pgfpathlineto{\pgfqpoint{2.398244in}{0.637273in}}%
\pgfpathlineto{\pgfqpoint{2.407835in}{0.637273in}}%
\pgfpathlineto{\pgfqpoint{2.409209in}{3.382727in}}%
\pgfpathlineto{\pgfqpoint{2.409374in}{0.637273in}}%
\pgfpathlineto{\pgfqpoint{2.409473in}{0.637273in}}%
\pgfpathlineto{\pgfqpoint{2.409615in}{3.382727in}}%
\pgfpathlineto{\pgfqpoint{2.411013in}{0.637273in}}%
\pgfpathlineto{\pgfqpoint{2.420505in}{0.637273in}}%
\pgfpathlineto{\pgfqpoint{2.422049in}{3.382727in}}%
\pgfpathlineto{\pgfqpoint{2.422686in}{3.382727in}}%
\pgfpathlineto{\pgfqpoint{2.423749in}{0.637273in}}%
\pgfpathlineto{\pgfqpoint{2.424226in}{3.382727in}}%
\pgfpathlineto{\pgfqpoint{2.424372in}{3.382727in}}%
\pgfpathlineto{\pgfqpoint{2.425860in}{0.637273in}}%
\pgfpathlineto{\pgfqpoint{2.425912in}{3.382727in}}%
\pgfpathlineto{\pgfqpoint{2.425931in}{3.382727in}}%
\pgfpathlineto{\pgfqpoint{2.426337in}{0.637273in}}%
\pgfpathlineto{\pgfqpoint{2.427470in}{3.382727in}}%
\pgfpathlineto{\pgfqpoint{2.427782in}{3.382727in}}%
\pgfpathlineto{\pgfqpoint{2.429321in}{0.637273in}}%
\pgfpathlineto{\pgfqpoint{2.430865in}{3.382727in}}%
\pgfpathlineto{\pgfqpoint{2.432254in}{3.382727in}}%
\pgfpathlineto{\pgfqpoint{2.433798in}{0.637273in}}%
\pgfpathlineto{\pgfqpoint{2.435130in}{3.382727in}}%
\pgfpathlineto{\pgfqpoint{2.435337in}{0.637273in}}%
\pgfpathlineto{\pgfqpoint{2.440537in}{0.637273in}}%
\pgfpathlineto{\pgfqpoint{2.442062in}{3.382727in}}%
\pgfpathlineto{\pgfqpoint{2.442076in}{0.637273in}}%
\pgfpathlineto{\pgfqpoint{2.442298in}{0.637273in}}%
\pgfpathlineto{\pgfqpoint{2.443842in}{3.382727in}}%
\pgfpathlineto{\pgfqpoint{2.444862in}{0.637273in}}%
\pgfpathlineto{\pgfqpoint{2.445382in}{3.382727in}}%
\pgfpathlineto{\pgfqpoint{2.446303in}{3.382727in}}%
\pgfpathlineto{\pgfqpoint{2.447837in}{0.637273in}}%
\pgfpathlineto{\pgfqpoint{2.447842in}{3.382727in}}%
\pgfpathlineto{\pgfqpoint{2.447852in}{3.382727in}}%
\pgfpathlineto{\pgfqpoint{2.449056in}{0.637273in}}%
\pgfpathlineto{\pgfqpoint{2.449391in}{3.382727in}}%
\pgfpathlineto{\pgfqpoint{2.450496in}{3.382727in}}%
\pgfpathlineto{\pgfqpoint{2.452026in}{0.637273in}}%
\pgfpathlineto{\pgfqpoint{2.452036in}{3.382727in}}%
\pgfpathlineto{\pgfqpoint{2.452040in}{3.382727in}}%
\pgfpathlineto{\pgfqpoint{2.453414in}{0.637273in}}%
\pgfpathlineto{\pgfqpoint{2.453580in}{3.382727in}}%
\pgfpathlineto{\pgfqpoint{2.454855in}{3.382727in}}%
\pgfpathlineto{\pgfqpoint{2.456390in}{0.637273in}}%
\pgfpathlineto{\pgfqpoint{2.456394in}{3.382727in}}%
\pgfpathlineto{\pgfqpoint{2.456404in}{3.382727in}}%
\pgfpathlineto{\pgfqpoint{2.457948in}{0.637273in}}%
\pgfpathlineto{\pgfqpoint{2.459473in}{3.382727in}}%
\pgfpathlineto{\pgfqpoint{2.459487in}{0.637273in}}%
\pgfpathlineto{\pgfqpoint{2.459714in}{0.637273in}}%
\pgfpathlineto{\pgfqpoint{2.461258in}{3.382727in}}%
\pgfpathlineto{\pgfqpoint{2.461268in}{3.382727in}}%
\pgfpathlineto{\pgfqpoint{2.462307in}{0.637273in}}%
\pgfpathlineto{\pgfqpoint{2.462807in}{3.382727in}}%
\pgfpathlineto{\pgfqpoint{2.463747in}{3.382727in}}%
\pgfpathlineto{\pgfqpoint{2.465282in}{0.637273in}}%
\pgfpathlineto{\pgfqpoint{2.465286in}{3.382727in}}%
\pgfpathlineto{\pgfqpoint{2.465296in}{3.382727in}}%
\pgfpathlineto{\pgfqpoint{2.466840in}{0.637273in}}%
\pgfpathlineto{\pgfqpoint{2.471997in}{0.637273in}}%
\pgfpathlineto{\pgfqpoint{2.473541in}{3.382727in}}%
\pgfpathlineto{\pgfqpoint{2.473787in}{3.382727in}}%
\pgfpathlineto{\pgfqpoint{2.475317in}{0.637273in}}%
\pgfpathlineto{\pgfqpoint{2.475326in}{3.382727in}}%
\pgfpathlineto{\pgfqpoint{2.475331in}{3.382727in}}%
\pgfpathlineto{\pgfqpoint{2.476672in}{0.637273in}}%
\pgfpathlineto{\pgfqpoint{2.476870in}{3.382727in}}%
\pgfpathlineto{\pgfqpoint{2.478112in}{3.382727in}}%
\pgfpathlineto{\pgfqpoint{2.479647in}{0.637273in}}%
\pgfpathlineto{\pgfqpoint{2.479652in}{3.382727in}}%
\pgfpathlineto{\pgfqpoint{2.479661in}{3.382727in}}%
\pgfpathlineto{\pgfqpoint{2.480865in}{0.637273in}}%
\pgfpathlineto{\pgfqpoint{2.481201in}{3.382727in}}%
\pgfpathlineto{\pgfqpoint{2.482306in}{3.382727in}}%
\pgfpathlineto{\pgfqpoint{2.483840in}{0.637273in}}%
\pgfpathlineto{\pgfqpoint{2.483845in}{3.382727in}}%
\pgfpathlineto{\pgfqpoint{2.483854in}{3.382727in}}%
\pgfpathlineto{\pgfqpoint{2.485361in}{0.637273in}}%
\pgfpathlineto{\pgfqpoint{2.485394in}{3.382727in}}%
\pgfpathlineto{\pgfqpoint{2.486801in}{3.382727in}}%
\pgfpathlineto{\pgfqpoint{2.488317in}{0.637273in}}%
\pgfpathlineto{\pgfqpoint{2.488341in}{3.382727in}}%
\pgfpathlineto{\pgfqpoint{2.488388in}{3.382727in}}%
\pgfpathlineto{\pgfqpoint{2.489885in}{0.637273in}}%
\pgfpathlineto{\pgfqpoint{2.489927in}{3.382727in}}%
\pgfpathlineto{\pgfqpoint{2.491325in}{3.382727in}}%
\pgfpathlineto{\pgfqpoint{2.492860in}{0.637273in}}%
\pgfpathlineto{\pgfqpoint{2.492865in}{3.382727in}}%
\pgfpathlineto{\pgfqpoint{2.492874in}{3.382727in}}%
\pgfpathlineto{\pgfqpoint{2.494244in}{0.637273in}}%
\pgfpathlineto{\pgfqpoint{2.494414in}{3.382727in}}%
\pgfpathlineto{\pgfqpoint{2.495684in}{3.382727in}}%
\pgfpathlineto{\pgfqpoint{2.497219in}{0.637273in}}%
\pgfpathlineto{\pgfqpoint{2.497223in}{3.382727in}}%
\pgfpathlineto{\pgfqpoint{2.497233in}{3.382727in}}%
\pgfpathlineto{\pgfqpoint{2.498777in}{0.637273in}}%
\pgfpathlineto{\pgfqpoint{2.503546in}{0.637273in}}%
\pgfpathlineto{\pgfqpoint{2.505091in}{3.382727in}}%
\pgfpathlineto{\pgfqpoint{2.505473in}{3.382727in}}%
\pgfpathlineto{\pgfqpoint{2.507008in}{0.637273in}}%
\pgfpathlineto{\pgfqpoint{2.507013in}{3.382727in}}%
\pgfpathlineto{\pgfqpoint{2.507022in}{3.382727in}}%
\pgfpathlineto{\pgfqpoint{2.508363in}{0.637273in}}%
\pgfpathlineto{\pgfqpoint{2.508562in}{3.382727in}}%
\pgfpathlineto{\pgfqpoint{2.509804in}{3.382727in}}%
\pgfpathlineto{\pgfqpoint{2.511324in}{0.637273in}}%
\pgfpathlineto{\pgfqpoint{2.511343in}{3.382727in}}%
\pgfpathlineto{\pgfqpoint{2.511348in}{3.382727in}}%
\pgfpathlineto{\pgfqpoint{2.512694in}{0.637273in}}%
\pgfpathlineto{\pgfqpoint{2.512887in}{3.382727in}}%
\pgfpathlineto{\pgfqpoint{2.514134in}{3.382727in}}%
\pgfpathlineto{\pgfqpoint{2.515669in}{0.637273in}}%
\pgfpathlineto{\pgfqpoint{2.515673in}{3.382727in}}%
\pgfpathlineto{\pgfqpoint{2.515683in}{3.382727in}}%
\pgfpathlineto{\pgfqpoint{2.517118in}{0.637273in}}%
\pgfpathlineto{\pgfqpoint{2.517222in}{3.382727in}}%
\pgfpathlineto{\pgfqpoint{2.518559in}{3.382727in}}%
\pgfpathlineto{\pgfqpoint{2.520093in}{0.637273in}}%
\pgfpathlineto{\pgfqpoint{2.520098in}{3.382727in}}%
\pgfpathlineto{\pgfqpoint{2.520108in}{3.382727in}}%
\pgfpathlineto{\pgfqpoint{2.521312in}{0.637273in}}%
\pgfpathlineto{\pgfqpoint{2.521647in}{3.382727in}}%
\pgfpathlineto{\pgfqpoint{2.522752in}{3.382727in}}%
\pgfpathlineto{\pgfqpoint{2.524287in}{0.637273in}}%
\pgfpathlineto{\pgfqpoint{2.524292in}{3.382727in}}%
\pgfpathlineto{\pgfqpoint{2.524301in}{3.382727in}}%
\pgfpathlineto{\pgfqpoint{2.525534in}{0.637273in}}%
\pgfpathlineto{\pgfqpoint{2.525840in}{3.382727in}}%
\pgfpathlineto{\pgfqpoint{2.526974in}{3.382727in}}%
\pgfpathlineto{\pgfqpoint{2.528504in}{0.637273in}}%
\pgfpathlineto{\pgfqpoint{2.528513in}{3.382727in}}%
\pgfpathlineto{\pgfqpoint{2.528523in}{3.382727in}}%
\pgfpathlineto{\pgfqpoint{2.530067in}{0.637273in}}%
\pgfpathlineto{\pgfqpoint{2.534001in}{0.637273in}}%
\pgfpathlineto{\pgfqpoint{2.535545in}{3.382727in}}%
\pgfpathlineto{\pgfqpoint{2.536763in}{3.382727in}}%
\pgfpathlineto{\pgfqpoint{2.538293in}{0.637273in}}%
\pgfpathlineto{\pgfqpoint{2.538303in}{3.382727in}}%
\pgfpathlineto{\pgfqpoint{2.538307in}{3.382727in}}%
\pgfpathlineto{\pgfqpoint{2.539649in}{0.637273in}}%
\pgfpathlineto{\pgfqpoint{2.539847in}{3.382727in}}%
\pgfpathlineto{\pgfqpoint{2.541089in}{3.382727in}}%
\pgfpathlineto{\pgfqpoint{2.542624in}{0.637273in}}%
\pgfpathlineto{\pgfqpoint{2.542628in}{3.382727in}}%
\pgfpathlineto{\pgfqpoint{2.542638in}{3.382727in}}%
\pgfpathlineto{\pgfqpoint{2.543875in}{0.637273in}}%
\pgfpathlineto{\pgfqpoint{2.544177in}{3.382727in}}%
\pgfpathlineto{\pgfqpoint{2.545315in}{3.382727in}}%
\pgfpathlineto{\pgfqpoint{2.546850in}{0.637273in}}%
\pgfpathlineto{\pgfqpoint{2.546855in}{3.382727in}}%
\pgfpathlineto{\pgfqpoint{2.546864in}{3.382727in}}%
\pgfpathlineto{\pgfqpoint{2.548149in}{0.637273in}}%
\pgfpathlineto{\pgfqpoint{2.548404in}{3.382727in}}%
\pgfpathlineto{\pgfqpoint{2.549589in}{3.382727in}}%
\pgfpathlineto{\pgfqpoint{2.551124in}{0.637273in}}%
\pgfpathlineto{\pgfqpoint{2.551128in}{3.382727in}}%
\pgfpathlineto{\pgfqpoint{2.551138in}{3.382727in}}%
\pgfpathlineto{\pgfqpoint{2.552337in}{0.637273in}}%
\pgfpathlineto{\pgfqpoint{2.552677in}{3.382727in}}%
\pgfpathlineto{\pgfqpoint{2.553778in}{3.382727in}}%
\pgfpathlineto{\pgfqpoint{2.555312in}{0.637273in}}%
\pgfpathlineto{\pgfqpoint{2.555317in}{3.382727in}}%
\pgfpathlineto{\pgfqpoint{2.555327in}{3.382727in}}%
\pgfpathlineto{\pgfqpoint{2.556554in}{0.637273in}}%
\pgfpathlineto{\pgfqpoint{2.556866in}{3.382727in}}%
\pgfpathlineto{\pgfqpoint{2.557995in}{3.382727in}}%
\pgfpathlineto{\pgfqpoint{2.559529in}{0.637273in}}%
\pgfpathlineto{\pgfqpoint{2.559534in}{3.382727in}}%
\pgfpathlineto{\pgfqpoint{2.559544in}{3.382727in}}%
\pgfpathlineto{\pgfqpoint{2.561088in}{0.637273in}}%
\pgfpathlineto{\pgfqpoint{2.565187in}{0.637273in}}%
\pgfpathlineto{\pgfqpoint{2.566731in}{3.382727in}}%
\pgfpathlineto{\pgfqpoint{2.567921in}{3.382727in}}%
\pgfpathlineto{\pgfqpoint{2.569456in}{0.637273in}}%
\pgfpathlineto{\pgfqpoint{2.569460in}{3.382727in}}%
\pgfpathlineto{\pgfqpoint{2.569470in}{3.382727in}}%
\pgfpathlineto{\pgfqpoint{2.570811in}{0.637273in}}%
\pgfpathlineto{\pgfqpoint{2.571009in}{3.382727in}}%
\pgfpathlineto{\pgfqpoint{2.572251in}{3.382727in}}%
\pgfpathlineto{\pgfqpoint{2.573786in}{0.637273in}}%
\pgfpathlineto{\pgfqpoint{2.573791in}{3.382727in}}%
\pgfpathlineto{\pgfqpoint{2.573800in}{3.382727in}}%
\pgfpathlineto{\pgfqpoint{2.575156in}{0.637273in}}%
\pgfpathlineto{\pgfqpoint{2.575340in}{3.382727in}}%
\pgfpathlineto{\pgfqpoint{2.576596in}{3.382727in}}%
\pgfpathlineto{\pgfqpoint{2.578126in}{0.637273in}}%
\pgfpathlineto{\pgfqpoint{2.578135in}{3.382727in}}%
\pgfpathlineto{\pgfqpoint{2.578140in}{3.382727in}}%
\pgfpathlineto{\pgfqpoint{2.579439in}{0.637273in}}%
\pgfpathlineto{\pgfqpoint{2.579680in}{3.382727in}}%
\pgfpathlineto{\pgfqpoint{2.580879in}{3.382727in}}%
\pgfpathlineto{\pgfqpoint{2.582414in}{0.637273in}}%
\pgfpathlineto{\pgfqpoint{2.582418in}{3.382727in}}%
\pgfpathlineto{\pgfqpoint{2.582428in}{3.382727in}}%
\pgfpathlineto{\pgfqpoint{2.583797in}{0.637273in}}%
\pgfpathlineto{\pgfqpoint{2.583967in}{3.382727in}}%
\pgfpathlineto{\pgfqpoint{2.585238in}{3.382727in}}%
\pgfpathlineto{\pgfqpoint{2.586768in}{0.637273in}}%
\pgfpathlineto{\pgfqpoint{2.586777in}{3.382727in}}%
\pgfpathlineto{\pgfqpoint{2.586782in}{3.382727in}}%
\pgfpathlineto{\pgfqpoint{2.587986in}{0.637273in}}%
\pgfpathlineto{\pgfqpoint{2.588321in}{3.382727in}}%
\pgfpathlineto{\pgfqpoint{2.589426in}{3.382727in}}%
\pgfpathlineto{\pgfqpoint{2.590961in}{0.637273in}}%
\pgfpathlineto{\pgfqpoint{2.590966in}{3.382727in}}%
\pgfpathlineto{\pgfqpoint{2.590975in}{3.382727in}}%
\pgfpathlineto{\pgfqpoint{2.592519in}{0.637273in}}%
\pgfpathlineto{\pgfqpoint{2.596453in}{0.637273in}}%
\pgfpathlineto{\pgfqpoint{2.597997in}{3.382727in}}%
\pgfpathlineto{\pgfqpoint{2.599079in}{3.382727in}}%
\pgfpathlineto{\pgfqpoint{2.600609in}{0.637273in}}%
\pgfpathlineto{\pgfqpoint{2.600618in}{3.382727in}}%
\pgfpathlineto{\pgfqpoint{2.600623in}{3.382727in}}%
\pgfpathlineto{\pgfqpoint{2.601822in}{0.637273in}}%
\pgfpathlineto{\pgfqpoint{2.602162in}{3.382727in}}%
\pgfpathlineto{\pgfqpoint{2.603263in}{3.382727in}}%
\pgfpathlineto{\pgfqpoint{2.604797in}{0.637273in}}%
\pgfpathlineto{\pgfqpoint{2.604802in}{3.382727in}}%
\pgfpathlineto{\pgfqpoint{2.604812in}{3.382727in}}%
\pgfpathlineto{\pgfqpoint{2.606011in}{0.637273in}}%
\pgfpathlineto{\pgfqpoint{2.606351in}{3.382727in}}%
\pgfpathlineto{\pgfqpoint{2.607451in}{3.382727in}}%
\pgfpathlineto{\pgfqpoint{2.608986in}{0.637273in}}%
\pgfpathlineto{\pgfqpoint{2.608991in}{3.382727in}}%
\pgfpathlineto{\pgfqpoint{2.608996in}{3.382727in}}%
\pgfpathlineto{\pgfqpoint{2.610483in}{0.637273in}}%
\pgfpathlineto{\pgfqpoint{2.610535in}{3.382727in}}%
\pgfpathlineto{\pgfqpoint{2.610563in}{3.382727in}}%
\pgfpathlineto{\pgfqpoint{2.611763in}{0.637273in}}%
\pgfpathlineto{\pgfqpoint{2.612103in}{3.382727in}}%
\pgfpathlineto{\pgfqpoint{2.623163in}{3.382727in}}%
\pgfpathlineto{\pgfqpoint{2.623314in}{0.637273in}}%
\pgfpathlineto{\pgfqpoint{2.624702in}{3.382727in}}%
\pgfpathlineto{\pgfqpoint{2.627998in}{3.382727in}}%
\pgfpathlineto{\pgfqpoint{2.628083in}{0.637273in}}%
\pgfpathlineto{\pgfqpoint{2.629538in}{3.382727in}}%
\pgfpathlineto{\pgfqpoint{2.637268in}{3.382727in}}%
\pgfpathlineto{\pgfqpoint{2.637273in}{0.637273in}}%
\pgfpathlineto{\pgfqpoint{2.638808in}{3.382727in}}%
\pgfpathlineto{\pgfqpoint{2.653706in}{3.382727in}}%
\pgfpathlineto{\pgfqpoint{2.655251in}{0.637273in}}%
\pgfpathlineto{\pgfqpoint{2.655468in}{0.637273in}}%
\pgfpathlineto{\pgfqpoint{2.657012in}{3.382727in}}%
\pgfpathlineto{\pgfqpoint{2.657229in}{3.382727in}}%
\pgfpathlineto{\pgfqpoint{2.657385in}{0.637273in}}%
\pgfpathlineto{\pgfqpoint{2.658769in}{3.382727in}}%
\pgfpathlineto{\pgfqpoint{2.659600in}{3.382727in}}%
\pgfpathlineto{\pgfqpoint{2.660837in}{0.637273in}}%
\pgfpathlineto{\pgfqpoint{2.661139in}{3.382727in}}%
\pgfpathlineto{\pgfqpoint{2.661423in}{3.382727in}}%
\pgfpathlineto{\pgfqpoint{2.662910in}{0.637273in}}%
\pgfpathlineto{\pgfqpoint{2.662962in}{3.382727in}}%
\pgfpathlineto{\pgfqpoint{2.662981in}{3.382727in}}%
\pgfpathlineto{\pgfqpoint{2.663397in}{0.637273in}}%
\pgfpathlineto{\pgfqpoint{2.664521in}{3.382727in}}%
\pgfpathlineto{\pgfqpoint{2.664837in}{3.382727in}}%
\pgfpathlineto{\pgfqpoint{2.666376in}{0.637273in}}%
\pgfpathlineto{\pgfqpoint{2.667921in}{3.382727in}}%
\pgfpathlineto{\pgfqpoint{2.667925in}{3.382727in}}%
\pgfpathlineto{\pgfqpoint{2.669469in}{0.637273in}}%
\pgfpathlineto{\pgfqpoint{2.671014in}{3.382727in}}%
\pgfpathlineto{\pgfqpoint{2.671018in}{3.382727in}}%
\pgfpathlineto{\pgfqpoint{2.672563in}{0.637273in}}%
\pgfpathlineto{\pgfqpoint{2.676496in}{0.637273in}}%
\pgfpathlineto{\pgfqpoint{2.678040in}{3.382727in}}%
\pgfpathlineto{\pgfqpoint{2.679264in}{3.382727in}}%
\pgfpathlineto{\pgfqpoint{2.680794in}{0.637273in}}%
\pgfpathlineto{\pgfqpoint{2.680803in}{3.382727in}}%
\pgfpathlineto{\pgfqpoint{2.680808in}{3.382727in}}%
\pgfpathlineto{\pgfqpoint{2.682012in}{0.637273in}}%
\pgfpathlineto{\pgfqpoint{2.682347in}{3.382727in}}%
\pgfpathlineto{\pgfqpoint{2.683452in}{3.382727in}}%
\pgfpathlineto{\pgfqpoint{2.684987in}{0.637273in}}%
\pgfpathlineto{\pgfqpoint{2.684992in}{3.382727in}}%
\pgfpathlineto{\pgfqpoint{2.685001in}{3.382727in}}%
\pgfpathlineto{\pgfqpoint{2.686205in}{0.637273in}}%
\pgfpathlineto{\pgfqpoint{2.686541in}{3.382727in}}%
\pgfpathlineto{\pgfqpoint{2.687646in}{3.382727in}}%
\pgfpathlineto{\pgfqpoint{2.689176in}{0.637273in}}%
\pgfpathlineto{\pgfqpoint{2.689185in}{3.382727in}}%
\pgfpathlineto{\pgfqpoint{2.689190in}{3.382727in}}%
\pgfpathlineto{\pgfqpoint{2.690413in}{0.637273in}}%
\pgfpathlineto{\pgfqpoint{2.690729in}{3.382727in}}%
\pgfpathlineto{\pgfqpoint{2.691853in}{3.382727in}}%
\pgfpathlineto{\pgfqpoint{2.693341in}{0.637273in}}%
\pgfpathlineto{\pgfqpoint{2.693393in}{3.382727in}}%
\pgfpathlineto{\pgfqpoint{2.693412in}{3.382727in}}%
\pgfpathlineto{\pgfqpoint{2.694663in}{0.637273in}}%
\pgfpathlineto{\pgfqpoint{2.694951in}{3.382727in}}%
\pgfpathlineto{\pgfqpoint{2.696103in}{3.382727in}}%
\pgfpathlineto{\pgfqpoint{2.697633in}{0.637273in}}%
\pgfpathlineto{\pgfqpoint{2.697643in}{3.382727in}}%
\pgfpathlineto{\pgfqpoint{2.697662in}{3.382727in}}%
\pgfpathlineto{\pgfqpoint{2.698852in}{0.637273in}}%
\pgfpathlineto{\pgfqpoint{2.699201in}{3.382727in}}%
\pgfpathlineto{\pgfqpoint{2.700292in}{3.382727in}}%
\pgfpathlineto{\pgfqpoint{2.701836in}{0.637273in}}%
\pgfpathlineto{\pgfqpoint{2.702724in}{3.382727in}}%
\pgfpathlineto{\pgfqpoint{2.703376in}{0.637273in}}%
\pgfpathlineto{\pgfqpoint{2.708202in}{0.637273in}}%
\pgfpathlineto{\pgfqpoint{2.709746in}{3.382727in}}%
\pgfpathlineto{\pgfqpoint{2.710100in}{3.382727in}}%
\pgfpathlineto{\pgfqpoint{2.711630in}{0.637273in}}%
\pgfpathlineto{\pgfqpoint{2.711640in}{3.382727in}}%
\pgfpathlineto{\pgfqpoint{2.711644in}{3.382727in}}%
\pgfpathlineto{\pgfqpoint{2.713014in}{0.637273in}}%
\pgfpathlineto{\pgfqpoint{2.713184in}{3.382727in}}%
\pgfpathlineto{\pgfqpoint{2.714459in}{3.382727in}}%
\pgfpathlineto{\pgfqpoint{2.715989in}{0.637273in}}%
\pgfpathlineto{\pgfqpoint{2.715998in}{3.382727in}}%
\pgfpathlineto{\pgfqpoint{2.716003in}{3.382727in}}%
\pgfpathlineto{\pgfqpoint{2.717207in}{0.637273in}}%
\pgfpathlineto{\pgfqpoint{2.717543in}{3.382727in}}%
\pgfpathlineto{\pgfqpoint{2.718648in}{3.382727in}}%
\pgfpathlineto{\pgfqpoint{2.720178in}{0.637273in}}%
\pgfpathlineto{\pgfqpoint{2.720187in}{3.382727in}}%
\pgfpathlineto{\pgfqpoint{2.720192in}{3.382727in}}%
\pgfpathlineto{\pgfqpoint{2.721642in}{0.637273in}}%
\pgfpathlineto{\pgfqpoint{2.721731in}{3.382727in}}%
\pgfpathlineto{\pgfqpoint{2.723082in}{3.382727in}}%
\pgfpathlineto{\pgfqpoint{2.724565in}{0.637273in}}%
\pgfpathlineto{\pgfqpoint{2.724621in}{3.382727in}}%
\pgfpathlineto{\pgfqpoint{2.724635in}{3.382727in}}%
\pgfpathlineto{\pgfqpoint{2.726024in}{0.637273in}}%
\pgfpathlineto{\pgfqpoint{2.726175in}{3.382727in}}%
\pgfpathlineto{\pgfqpoint{2.727464in}{3.382727in}}%
\pgfpathlineto{\pgfqpoint{2.728999in}{0.637273in}}%
\pgfpathlineto{\pgfqpoint{2.729004in}{3.382727in}}%
\pgfpathlineto{\pgfqpoint{2.729013in}{3.382727in}}%
\pgfpathlineto{\pgfqpoint{2.730354in}{0.637273in}}%
\pgfpathlineto{\pgfqpoint{2.730553in}{3.382727in}}%
\pgfpathlineto{\pgfqpoint{2.731794in}{3.382727in}}%
\pgfpathlineto{\pgfqpoint{2.733325in}{0.637273in}}%
\pgfpathlineto{\pgfqpoint{2.733334in}{3.382727in}}%
\pgfpathlineto{\pgfqpoint{2.733339in}{3.382727in}}%
\pgfpathlineto{\pgfqpoint{2.734883in}{0.637273in}}%
\pgfpathlineto{\pgfqpoint{2.739704in}{0.637273in}}%
\pgfpathlineto{\pgfqpoint{2.741249in}{3.382727in}}%
\pgfpathlineto{\pgfqpoint{2.741461in}{3.382727in}}%
\pgfpathlineto{\pgfqpoint{2.742991in}{0.637273in}}%
\pgfpathlineto{\pgfqpoint{2.743001in}{3.382727in}}%
\pgfpathlineto{\pgfqpoint{2.743010in}{3.382727in}}%
\pgfpathlineto{\pgfqpoint{2.744242in}{0.637273in}}%
\pgfpathlineto{\pgfqpoint{2.744549in}{3.382727in}}%
\pgfpathlineto{\pgfqpoint{2.745678in}{3.382727in}}%
\pgfpathlineto{\pgfqpoint{2.747213in}{0.637273in}}%
\pgfpathlineto{\pgfqpoint{2.747218in}{3.382727in}}%
\pgfpathlineto{\pgfqpoint{2.747227in}{3.382727in}}%
\pgfpathlineto{\pgfqpoint{2.748469in}{0.637273in}}%
\pgfpathlineto{\pgfqpoint{2.748766in}{3.382727in}}%
\pgfpathlineto{\pgfqpoint{2.749909in}{3.382727in}}%
\pgfpathlineto{\pgfqpoint{2.751444in}{0.637273in}}%
\pgfpathlineto{\pgfqpoint{2.751449in}{3.382727in}}%
\pgfpathlineto{\pgfqpoint{2.751458in}{3.382727in}}%
\pgfpathlineto{\pgfqpoint{2.752681in}{0.637273in}}%
\pgfpathlineto{\pgfqpoint{2.752998in}{3.382727in}}%
\pgfpathlineto{\pgfqpoint{2.754122in}{3.382727in}}%
\pgfpathlineto{\pgfqpoint{2.755652in}{0.637273in}}%
\pgfpathlineto{\pgfqpoint{2.755661in}{3.382727in}}%
\pgfpathlineto{\pgfqpoint{2.755666in}{3.382727in}}%
\pgfpathlineto{\pgfqpoint{2.756927in}{0.637273in}}%
\pgfpathlineto{\pgfqpoint{2.757205in}{3.382727in}}%
\pgfpathlineto{\pgfqpoint{2.758367in}{3.382727in}}%
\pgfpathlineto{\pgfqpoint{2.759902in}{0.637273in}}%
\pgfpathlineto{\pgfqpoint{2.759906in}{3.382727in}}%
\pgfpathlineto{\pgfqpoint{2.759911in}{3.382727in}}%
\pgfpathlineto{\pgfqpoint{2.761120in}{0.637273in}}%
\pgfpathlineto{\pgfqpoint{2.761451in}{3.382727in}}%
\pgfpathlineto{\pgfqpoint{2.762560in}{3.382727in}}%
\pgfpathlineto{\pgfqpoint{2.764090in}{0.637273in}}%
\pgfpathlineto{\pgfqpoint{2.764100in}{3.382727in}}%
\pgfpathlineto{\pgfqpoint{2.764105in}{3.382727in}}%
\pgfpathlineto{\pgfqpoint{2.765649in}{0.637273in}}%
\pgfpathlineto{\pgfqpoint{2.770692in}{0.637273in}}%
\pgfpathlineto{\pgfqpoint{2.772217in}{3.382727in}}%
\pgfpathlineto{\pgfqpoint{2.772232in}{0.637273in}}%
\pgfpathlineto{\pgfqpoint{2.772458in}{0.637273in}}%
\pgfpathlineto{\pgfqpoint{2.774002in}{3.382727in}}%
\pgfpathlineto{\pgfqpoint{2.774914in}{0.637273in}}%
\pgfpathlineto{\pgfqpoint{2.775542in}{3.382727in}}%
\pgfpathlineto{\pgfqpoint{2.776354in}{3.382727in}}%
\pgfpathlineto{\pgfqpoint{2.777884in}{0.637273in}}%
\pgfpathlineto{\pgfqpoint{2.777894in}{3.382727in}}%
\pgfpathlineto{\pgfqpoint{2.777898in}{3.382727in}}%
\pgfpathlineto{\pgfqpoint{2.779254in}{0.637273in}}%
\pgfpathlineto{\pgfqpoint{2.779438in}{3.382727in}}%
\pgfpathlineto{\pgfqpoint{2.780694in}{3.382727in}}%
\pgfpathlineto{\pgfqpoint{2.782229in}{0.637273in}}%
\pgfpathlineto{\pgfqpoint{2.782233in}{3.382727in}}%
\pgfpathlineto{\pgfqpoint{2.782243in}{3.382727in}}%
\pgfpathlineto{\pgfqpoint{2.783551in}{0.637273in}}%
\pgfpathlineto{\pgfqpoint{2.783782in}{3.382727in}}%
\pgfpathlineto{\pgfqpoint{2.784991in}{3.382727in}}%
\pgfpathlineto{\pgfqpoint{2.786521in}{0.637273in}}%
\pgfpathlineto{\pgfqpoint{2.786531in}{3.382727in}}%
\pgfpathlineto{\pgfqpoint{2.786535in}{3.382727in}}%
\pgfpathlineto{\pgfqpoint{2.787740in}{0.637273in}}%
\pgfpathlineto{\pgfqpoint{2.788075in}{3.382727in}}%
\pgfpathlineto{\pgfqpoint{2.788226in}{3.382727in}}%
\pgfpathlineto{\pgfqpoint{2.789553in}{0.637273in}}%
\pgfpathlineto{\pgfqpoint{2.789766in}{3.382727in}}%
\pgfpathlineto{\pgfqpoint{2.790176in}{3.382727in}}%
\pgfpathlineto{\pgfqpoint{2.791640in}{0.637273in}}%
\pgfpathlineto{\pgfqpoint{2.791716in}{3.382727in}}%
\pgfpathlineto{\pgfqpoint{2.791725in}{3.382727in}}%
\pgfpathlineto{\pgfqpoint{2.792146in}{0.637273in}}%
\pgfpathlineto{\pgfqpoint{2.793265in}{3.382727in}}%
\pgfpathlineto{\pgfqpoint{2.793586in}{3.382727in}}%
\pgfpathlineto{\pgfqpoint{2.795116in}{0.637273in}}%
\pgfpathlineto{\pgfqpoint{2.795125in}{3.382727in}}%
\pgfpathlineto{\pgfqpoint{2.795130in}{3.382727in}}%
\pgfpathlineto{\pgfqpoint{2.796334in}{0.637273in}}%
\pgfpathlineto{\pgfqpoint{2.796670in}{3.382727in}}%
\pgfpathlineto{\pgfqpoint{2.797675in}{3.382727in}}%
\pgfpathlineto{\pgfqpoint{2.799220in}{0.637273in}}%
\pgfpathlineto{\pgfqpoint{2.803153in}{0.637273in}}%
\pgfpathlineto{\pgfqpoint{2.804697in}{3.382727in}}%
\pgfpathlineto{\pgfqpoint{2.804707in}{3.382727in}}%
\pgfpathlineto{\pgfqpoint{2.806143in}{0.637273in}}%
\pgfpathlineto{\pgfqpoint{2.806246in}{3.382727in}}%
\pgfpathlineto{\pgfqpoint{2.807583in}{3.382727in}}%
\pgfpathlineto{\pgfqpoint{2.809070in}{0.637273in}}%
\pgfpathlineto{\pgfqpoint{2.809122in}{3.382727in}}%
\pgfpathlineto{\pgfqpoint{2.809141in}{3.382727in}}%
\pgfpathlineto{\pgfqpoint{2.810501in}{0.637273in}}%
\pgfpathlineto{\pgfqpoint{2.810681in}{3.382727in}}%
\pgfpathlineto{\pgfqpoint{2.811941in}{3.382727in}}%
\pgfpathlineto{\pgfqpoint{2.813472in}{0.637273in}}%
\pgfpathlineto{\pgfqpoint{2.813481in}{3.382727in}}%
\pgfpathlineto{\pgfqpoint{2.813505in}{3.382727in}}%
\pgfpathlineto{\pgfqpoint{2.814827in}{0.637273in}}%
\pgfpathlineto{\pgfqpoint{2.815044in}{3.382727in}}%
\pgfpathlineto{\pgfqpoint{2.816267in}{3.382727in}}%
\pgfpathlineto{\pgfqpoint{2.817802in}{0.637273in}}%
\pgfpathlineto{\pgfqpoint{2.817807in}{3.382727in}}%
\pgfpathlineto{\pgfqpoint{2.817816in}{3.382727in}}%
\pgfpathlineto{\pgfqpoint{2.819360in}{0.637273in}}%
\pgfpathlineto{\pgfqpoint{2.820229in}{0.637273in}}%
\pgfpathlineto{\pgfqpoint{2.821764in}{3.382727in}}%
\pgfpathlineto{\pgfqpoint{2.821769in}{0.637273in}}%
\pgfpathlineto{\pgfqpoint{2.822000in}{0.637273in}}%
\pgfpathlineto{\pgfqpoint{2.823535in}{3.382727in}}%
\pgfpathlineto{\pgfqpoint{2.823539in}{0.637273in}}%
\pgfpathlineto{\pgfqpoint{2.823634in}{0.637273in}}%
\pgfpathlineto{\pgfqpoint{2.825178in}{3.382727in}}%
\pgfpathlineto{\pgfqpoint{2.826453in}{3.382727in}}%
\pgfpathlineto{\pgfqpoint{2.827988in}{0.637273in}}%
\pgfpathlineto{\pgfqpoint{2.827993in}{3.382727in}}%
\pgfpathlineto{\pgfqpoint{2.828002in}{3.382727in}}%
\pgfpathlineto{\pgfqpoint{2.829206in}{0.637273in}}%
\pgfpathlineto{\pgfqpoint{2.829542in}{3.382727in}}%
\pgfpathlineto{\pgfqpoint{2.830647in}{3.382727in}}%
\pgfpathlineto{\pgfqpoint{2.832181in}{0.637273in}}%
\pgfpathlineto{\pgfqpoint{2.832186in}{3.382727in}}%
\pgfpathlineto{\pgfqpoint{2.832195in}{3.382727in}}%
\pgfpathlineto{\pgfqpoint{2.833740in}{0.637273in}}%
\pgfpathlineto{\pgfqpoint{2.837626in}{0.637273in}}%
\pgfpathlineto{\pgfqpoint{2.839170in}{3.382727in}}%
\pgfpathlineto{\pgfqpoint{2.840223in}{3.382727in}}%
\pgfpathlineto{\pgfqpoint{2.841753in}{0.637273in}}%
\pgfpathlineto{\pgfqpoint{2.841763in}{3.382727in}}%
\pgfpathlineto{\pgfqpoint{2.841768in}{3.382727in}}%
\pgfpathlineto{\pgfqpoint{2.843000in}{0.637273in}}%
\pgfpathlineto{\pgfqpoint{2.843307in}{3.382727in}}%
\pgfpathlineto{\pgfqpoint{2.844440in}{3.382727in}}%
\pgfpathlineto{\pgfqpoint{2.845975in}{0.637273in}}%
\pgfpathlineto{\pgfqpoint{2.845980in}{3.382727in}}%
\pgfpathlineto{\pgfqpoint{2.845989in}{3.382727in}}%
\pgfpathlineto{\pgfqpoint{2.847378in}{0.637273in}}%
\pgfpathlineto{\pgfqpoint{2.847529in}{3.382727in}}%
\pgfpathlineto{\pgfqpoint{2.848818in}{3.382727in}}%
\pgfpathlineto{\pgfqpoint{2.850353in}{0.637273in}}%
\pgfpathlineto{\pgfqpoint{2.850357in}{3.382727in}}%
\pgfpathlineto{\pgfqpoint{2.850367in}{3.382727in}}%
\pgfpathlineto{\pgfqpoint{2.851599in}{0.637273in}}%
\pgfpathlineto{\pgfqpoint{2.851906in}{3.382727in}}%
\pgfpathlineto{\pgfqpoint{2.853040in}{3.382727in}}%
\pgfpathlineto{\pgfqpoint{2.854574in}{0.637273in}}%
\pgfpathlineto{\pgfqpoint{2.854579in}{3.382727in}}%
\pgfpathlineto{\pgfqpoint{2.854589in}{3.382727in}}%
\pgfpathlineto{\pgfqpoint{2.856095in}{0.637273in}}%
\pgfpathlineto{\pgfqpoint{2.856128in}{3.382727in}}%
\pgfpathlineto{\pgfqpoint{2.857535in}{3.382727in}}%
\pgfpathlineto{\pgfqpoint{2.859065in}{0.637273in}}%
\pgfpathlineto{\pgfqpoint{2.859075in}{3.382727in}}%
\pgfpathlineto{\pgfqpoint{2.859080in}{3.382727in}}%
\pgfpathlineto{\pgfqpoint{2.860421in}{0.637273in}}%
\pgfpathlineto{\pgfqpoint{2.860619in}{3.382727in}}%
\pgfpathlineto{\pgfqpoint{2.861861in}{3.382727in}}%
\pgfpathlineto{\pgfqpoint{2.863396in}{0.637273in}}%
\pgfpathlineto{\pgfqpoint{2.863400in}{3.382727in}}%
\pgfpathlineto{\pgfqpoint{2.863410in}{3.382727in}}%
\pgfpathlineto{\pgfqpoint{2.864954in}{0.637273in}}%
\pgfpathlineto{\pgfqpoint{2.868888in}{0.637273in}}%
\pgfpathlineto{\pgfqpoint{2.870432in}{3.382727in}}%
\pgfpathlineto{\pgfqpoint{2.871622in}{3.382727in}}%
\pgfpathlineto{\pgfqpoint{2.873152in}{0.637273in}}%
\pgfpathlineto{\pgfqpoint{2.873161in}{3.382727in}}%
\pgfpathlineto{\pgfqpoint{2.873166in}{3.382727in}}%
\pgfpathlineto{\pgfqpoint{2.874399in}{0.637273in}}%
\pgfpathlineto{\pgfqpoint{2.874706in}{3.382727in}}%
\pgfpathlineto{\pgfqpoint{2.875844in}{3.382727in}}%
\pgfpathlineto{\pgfqpoint{2.877374in}{0.637273in}}%
\pgfpathlineto{\pgfqpoint{2.877383in}{3.382727in}}%
\pgfpathlineto{\pgfqpoint{2.877388in}{3.382727in}}%
\pgfpathlineto{\pgfqpoint{2.878805in}{0.637273in}}%
\pgfpathlineto{\pgfqpoint{2.878927in}{3.382727in}}%
\pgfpathlineto{\pgfqpoint{2.880245in}{3.382727in}}%
\pgfpathlineto{\pgfqpoint{2.881780in}{0.637273in}}%
\pgfpathlineto{\pgfqpoint{2.881784in}{3.382727in}}%
\pgfpathlineto{\pgfqpoint{2.881794in}{3.382727in}}%
\pgfpathlineto{\pgfqpoint{2.883163in}{0.637273in}}%
\pgfpathlineto{\pgfqpoint{2.883333in}{3.382727in}}%
\pgfpathlineto{\pgfqpoint{2.884604in}{3.382727in}}%
\pgfpathlineto{\pgfqpoint{2.886134in}{0.637273in}}%
\pgfpathlineto{\pgfqpoint{2.886143in}{3.382727in}}%
\pgfpathlineto{\pgfqpoint{2.886148in}{3.382727in}}%
\pgfpathlineto{\pgfqpoint{2.887654in}{0.637273in}}%
\pgfpathlineto{\pgfqpoint{2.887687in}{3.382727in}}%
\pgfpathlineto{\pgfqpoint{2.889095in}{3.382727in}}%
\pgfpathlineto{\pgfqpoint{2.890629in}{0.637273in}}%
\pgfpathlineto{\pgfqpoint{2.890634in}{3.382727in}}%
\pgfpathlineto{\pgfqpoint{2.890639in}{3.382727in}}%
\pgfpathlineto{\pgfqpoint{2.892183in}{0.637273in}}%
\pgfpathlineto{\pgfqpoint{2.897458in}{0.637273in}}%
\pgfpathlineto{\pgfqpoint{2.898983in}{3.382727in}}%
\pgfpathlineto{\pgfqpoint{2.898997in}{0.637273in}}%
\pgfpathlineto{\pgfqpoint{2.899224in}{0.637273in}}%
\pgfpathlineto{\pgfqpoint{2.900697in}{3.382727in}}%
\pgfpathlineto{\pgfqpoint{2.900763in}{0.637273in}}%
\pgfpathlineto{\pgfqpoint{2.900811in}{0.637273in}}%
\pgfpathlineto{\pgfqpoint{2.902355in}{3.382727in}}%
\pgfpathlineto{\pgfqpoint{2.903228in}{3.382727in}}%
\pgfpathlineto{\pgfqpoint{2.904711in}{0.637273in}}%
\pgfpathlineto{\pgfqpoint{2.904768in}{3.382727in}}%
\pgfpathlineto{\pgfqpoint{2.904782in}{3.382727in}}%
\pgfpathlineto{\pgfqpoint{2.906322in}{0.637273in}}%
\pgfpathlineto{\pgfqpoint{2.907866in}{3.382727in}}%
\pgfpathlineto{\pgfqpoint{2.907894in}{3.382727in}}%
\pgfpathlineto{\pgfqpoint{2.909438in}{0.637273in}}%
\pgfpathlineto{\pgfqpoint{2.910982in}{3.382727in}}%
\pgfpathlineto{\pgfqpoint{2.912305in}{3.382727in}}%
\pgfpathlineto{\pgfqpoint{2.913839in}{0.637273in}}%
\pgfpathlineto{\pgfqpoint{2.913844in}{3.382727in}}%
\pgfpathlineto{\pgfqpoint{2.913854in}{3.382727in}}%
\pgfpathlineto{\pgfqpoint{2.915360in}{0.637273in}}%
\pgfpathlineto{\pgfqpoint{2.915393in}{3.382727in}}%
\pgfpathlineto{\pgfqpoint{2.916800in}{3.382727in}}%
\pgfpathlineto{\pgfqpoint{2.918335in}{0.637273in}}%
\pgfpathlineto{\pgfqpoint{2.918340in}{3.382727in}}%
\pgfpathlineto{\pgfqpoint{2.918349in}{3.382727in}}%
\pgfpathlineto{\pgfqpoint{2.919893in}{0.637273in}}%
\pgfpathlineto{\pgfqpoint{2.925164in}{0.637273in}}%
\pgfpathlineto{\pgfqpoint{2.926689in}{3.382727in}}%
\pgfpathlineto{\pgfqpoint{2.926703in}{0.637273in}}%
\pgfpathlineto{\pgfqpoint{2.926930in}{0.637273in}}%
\pgfpathlineto{\pgfqpoint{2.928403in}{3.382727in}}%
\pgfpathlineto{\pgfqpoint{2.928469in}{0.637273in}}%
\pgfpathlineto{\pgfqpoint{2.928483in}{0.637273in}}%
\pgfpathlineto{\pgfqpoint{2.930028in}{3.382727in}}%
\pgfpathlineto{\pgfqpoint{2.931085in}{3.382727in}}%
\pgfpathlineto{\pgfqpoint{2.932573in}{0.637273in}}%
\pgfpathlineto{\pgfqpoint{2.932625in}{3.382727in}}%
\pgfpathlineto{\pgfqpoint{2.932644in}{3.382727in}}%
\pgfpathlineto{\pgfqpoint{2.934188in}{0.637273in}}%
\pgfpathlineto{\pgfqpoint{2.934216in}{0.637273in}}%
\pgfpathlineto{\pgfqpoint{2.935742in}{3.382727in}}%
\pgfpathlineto{\pgfqpoint{2.935756in}{0.637273in}}%
\pgfpathlineto{\pgfqpoint{2.935982in}{0.637273in}}%
\pgfpathlineto{\pgfqpoint{2.937456in}{3.382727in}}%
\pgfpathlineto{\pgfqpoint{2.937522in}{0.637273in}}%
\pgfpathlineto{\pgfqpoint{2.937545in}{0.637273in}}%
\pgfpathlineto{\pgfqpoint{2.939090in}{3.382727in}}%
\pgfpathlineto{\pgfqpoint{2.940152in}{3.382727in}}%
\pgfpathlineto{\pgfqpoint{2.941687in}{0.637273in}}%
\pgfpathlineto{\pgfqpoint{2.941692in}{3.382727in}}%
\pgfpathlineto{\pgfqpoint{2.941706in}{3.382727in}}%
\pgfpathlineto{\pgfqpoint{2.943207in}{0.637273in}}%
\pgfpathlineto{\pgfqpoint{2.943245in}{3.382727in}}%
\pgfpathlineto{\pgfqpoint{2.944648in}{3.382727in}}%
\pgfpathlineto{\pgfqpoint{2.946183in}{0.637273in}}%
\pgfpathlineto{\pgfqpoint{2.946187in}{3.382727in}}%
\pgfpathlineto{\pgfqpoint{2.946197in}{3.382727in}}%
\pgfpathlineto{\pgfqpoint{2.947741in}{0.637273in}}%
\pgfpathlineto{\pgfqpoint{2.953011in}{0.637273in}}%
\pgfpathlineto{\pgfqpoint{2.954536in}{3.382727in}}%
\pgfpathlineto{\pgfqpoint{2.954550in}{0.637273in}}%
\pgfpathlineto{\pgfqpoint{2.954777in}{0.637273in}}%
\pgfpathlineto{\pgfqpoint{2.956307in}{3.382727in}}%
\pgfpathlineto{\pgfqpoint{2.956317in}{0.637273in}}%
\pgfpathlineto{\pgfqpoint{2.956321in}{0.637273in}}%
\pgfpathlineto{\pgfqpoint{2.957866in}{3.382727in}}%
\pgfpathlineto{\pgfqpoint{2.958782in}{3.382727in}}%
\pgfpathlineto{\pgfqpoint{2.960312in}{0.637273in}}%
\pgfpathlineto{\pgfqpoint{2.960321in}{3.382727in}}%
\pgfpathlineto{\pgfqpoint{2.960326in}{3.382727in}}%
\pgfpathlineto{\pgfqpoint{2.961870in}{0.637273in}}%
\pgfpathlineto{\pgfqpoint{2.961884in}{0.637273in}}%
\pgfpathlineto{\pgfqpoint{2.963410in}{3.382727in}}%
\pgfpathlineto{\pgfqpoint{2.963424in}{0.637273in}}%
\pgfpathlineto{\pgfqpoint{2.963646in}{0.637273in}}%
\pgfpathlineto{\pgfqpoint{2.965124in}{3.382727in}}%
\pgfpathlineto{\pgfqpoint{2.965185in}{0.637273in}}%
\pgfpathlineto{\pgfqpoint{2.965213in}{0.637273in}}%
\pgfpathlineto{\pgfqpoint{2.966758in}{3.382727in}}%
\pgfpathlineto{\pgfqpoint{2.967820in}{3.382727in}}%
\pgfpathlineto{\pgfqpoint{2.969350in}{0.637273in}}%
\pgfpathlineto{\pgfqpoint{2.969360in}{3.382727in}}%
\pgfpathlineto{\pgfqpoint{2.969364in}{3.382727in}}%
\pgfpathlineto{\pgfqpoint{2.970909in}{0.637273in}}%
\pgfpathlineto{\pgfqpoint{2.970956in}{0.637273in}}%
\pgfpathlineto{\pgfqpoint{2.972481in}{3.382727in}}%
\pgfpathlineto{\pgfqpoint{2.972495in}{0.637273in}}%
\pgfpathlineto{\pgfqpoint{2.972750in}{0.637273in}}%
\pgfpathlineto{\pgfqpoint{2.974224in}{3.382727in}}%
\pgfpathlineto{\pgfqpoint{2.974290in}{0.637273in}}%
\pgfpathlineto{\pgfqpoint{2.974304in}{0.637273in}}%
\pgfpathlineto{\pgfqpoint{2.975848in}{3.382727in}}%
\pgfpathlineto{\pgfqpoint{2.976892in}{3.382727in}}%
\pgfpathlineto{\pgfqpoint{2.978426in}{0.637273in}}%
\pgfpathlineto{\pgfqpoint{2.978431in}{3.382727in}}%
\pgfpathlineto{\pgfqpoint{2.978441in}{3.382727in}}%
\pgfpathlineto{\pgfqpoint{2.979985in}{0.637273in}}%
\pgfpathlineto{\pgfqpoint{2.983918in}{0.637273in}}%
\pgfpathlineto{\pgfqpoint{2.985463in}{3.382727in}}%
\pgfpathlineto{\pgfqpoint{2.986653in}{3.382727in}}%
\pgfpathlineto{\pgfqpoint{2.988183in}{0.637273in}}%
\pgfpathlineto{\pgfqpoint{2.988192in}{3.382727in}}%
\pgfpathlineto{\pgfqpoint{2.988197in}{3.382727in}}%
\pgfpathlineto{\pgfqpoint{2.989618in}{0.637273in}}%
\pgfpathlineto{\pgfqpoint{2.989736in}{3.382727in}}%
\pgfpathlineto{\pgfqpoint{2.991059in}{3.382727in}}%
\pgfpathlineto{\pgfqpoint{2.992546in}{0.637273in}}%
\pgfpathlineto{\pgfqpoint{2.992598in}{3.382727in}}%
\pgfpathlineto{\pgfqpoint{2.992617in}{3.382727in}}%
\pgfpathlineto{\pgfqpoint{2.993977in}{0.637273in}}%
\pgfpathlineto{\pgfqpoint{2.994156in}{3.382727in}}%
\pgfpathlineto{\pgfqpoint{2.995417in}{3.382727in}}%
\pgfpathlineto{\pgfqpoint{2.996947in}{0.637273in}}%
\pgfpathlineto{\pgfqpoint{2.996957in}{3.382727in}}%
\pgfpathlineto{\pgfqpoint{2.996962in}{3.382727in}}%
\pgfpathlineto{\pgfqpoint{2.998336in}{0.637273in}}%
\pgfpathlineto{\pgfqpoint{2.998501in}{3.382727in}}%
\pgfpathlineto{\pgfqpoint{2.999776in}{3.382727in}}%
\pgfpathlineto{\pgfqpoint{3.001306in}{0.637273in}}%
\pgfpathlineto{\pgfqpoint{3.001315in}{3.382727in}}%
\pgfpathlineto{\pgfqpoint{3.001320in}{3.382727in}}%
\pgfpathlineto{\pgfqpoint{3.002751in}{0.637273in}}%
\pgfpathlineto{\pgfqpoint{3.002860in}{3.382727in}}%
\pgfpathlineto{\pgfqpoint{3.004191in}{3.382727in}}%
\pgfpathlineto{\pgfqpoint{3.005679in}{0.637273in}}%
\pgfpathlineto{\pgfqpoint{3.005731in}{3.382727in}}%
\pgfpathlineto{\pgfqpoint{3.005750in}{3.382727in}}%
\pgfpathlineto{\pgfqpoint{3.007247in}{0.637273in}}%
\pgfpathlineto{\pgfqpoint{3.007289in}{3.382727in}}%
\pgfpathlineto{\pgfqpoint{3.007393in}{3.382727in}}%
\pgfpathlineto{\pgfqpoint{3.008928in}{0.637273in}}%
\pgfpathlineto{\pgfqpoint{3.008933in}{3.382727in}}%
\pgfpathlineto{\pgfqpoint{3.008942in}{3.382727in}}%
\pgfpathlineto{\pgfqpoint{3.010458in}{0.637273in}}%
\pgfpathlineto{\pgfqpoint{3.010481in}{3.382727in}}%
\pgfpathlineto{\pgfqpoint{3.010784in}{3.382727in}}%
\pgfpathlineto{\pgfqpoint{3.012328in}{0.637273in}}%
\pgfpathlineto{\pgfqpoint{3.013971in}{0.637273in}}%
\pgfpathlineto{\pgfqpoint{3.015515in}{3.382727in}}%
\pgfpathlineto{\pgfqpoint{3.015520in}{3.382727in}}%
\pgfpathlineto{\pgfqpoint{3.016956in}{0.637273in}}%
\pgfpathlineto{\pgfqpoint{3.017060in}{3.382727in}}%
\pgfpathlineto{\pgfqpoint{3.018396in}{3.382727in}}%
\pgfpathlineto{\pgfqpoint{3.019931in}{0.637273in}}%
\pgfpathlineto{\pgfqpoint{3.019936in}{3.382727in}}%
\pgfpathlineto{\pgfqpoint{3.019945in}{3.382727in}}%
\pgfpathlineto{\pgfqpoint{3.021489in}{0.637273in}}%
\pgfpathlineto{\pgfqpoint{3.032251in}{0.637273in}}%
\pgfpathlineto{\pgfqpoint{3.032299in}{3.382727in}}%
\pgfpathlineto{\pgfqpoint{3.033791in}{0.637273in}}%
\pgfpathlineto{\pgfqpoint{3.045153in}{0.637273in}}%
\pgfpathlineto{\pgfqpoint{3.045200in}{3.382727in}}%
\pgfpathlineto{\pgfqpoint{3.046692in}{0.637273in}}%
\pgfpathlineto{\pgfqpoint{3.056066in}{0.637273in}}%
\pgfpathlineto{\pgfqpoint{3.056094in}{3.382727in}}%
\pgfpathlineto{\pgfqpoint{3.057605in}{0.637273in}}%
\pgfpathlineto{\pgfqpoint{3.066885in}{0.637273in}}%
\pgfpathlineto{\pgfqpoint{3.066913in}{3.382727in}}%
\pgfpathlineto{\pgfqpoint{3.068424in}{0.637273in}}%
\pgfpathlineto{\pgfqpoint{3.078015in}{0.637273in}}%
\pgfpathlineto{\pgfqpoint{3.078025in}{3.382727in}}%
\pgfpathlineto{\pgfqpoint{3.079555in}{0.637273in}}%
\pgfpathlineto{\pgfqpoint{3.088933in}{0.637273in}}%
\pgfpathlineto{\pgfqpoint{3.088962in}{3.382727in}}%
\pgfpathlineto{\pgfqpoint{3.090473in}{0.637273in}}%
\pgfpathlineto{\pgfqpoint{3.094732in}{0.637273in}}%
\pgfpathlineto{\pgfqpoint{3.094760in}{3.382727in}}%
\pgfpathlineto{\pgfqpoint{3.096272in}{0.637273in}}%
\pgfpathlineto{\pgfqpoint{3.105556in}{0.637273in}}%
\pgfpathlineto{\pgfqpoint{3.105584in}{3.382727in}}%
\pgfpathlineto{\pgfqpoint{3.107095in}{0.637273in}}%
\pgfpathlineto{\pgfqpoint{3.116370in}{0.637273in}}%
\pgfpathlineto{\pgfqpoint{3.116398in}{3.382727in}}%
\pgfpathlineto{\pgfqpoint{3.117909in}{0.637273in}}%
\pgfpathlineto{\pgfqpoint{3.127377in}{0.637273in}}%
\pgfpathlineto{\pgfqpoint{3.127406in}{3.382727in}}%
\pgfpathlineto{\pgfqpoint{3.128917in}{0.637273in}}%
\pgfpathlineto{\pgfqpoint{3.138201in}{0.637273in}}%
\pgfpathlineto{\pgfqpoint{3.138229in}{3.382727in}}%
\pgfpathlineto{\pgfqpoint{3.139740in}{0.637273in}}%
\pgfpathlineto{\pgfqpoint{3.150281in}{0.637273in}}%
\pgfpathlineto{\pgfqpoint{3.150309in}{3.382727in}}%
\pgfpathlineto{\pgfqpoint{3.151820in}{0.637273in}}%
\pgfpathlineto{\pgfqpoint{3.253109in}{0.637273in}}%
\pgfpathlineto{\pgfqpoint{3.254653in}{3.382727in}}%
\pgfpathlineto{\pgfqpoint{3.255328in}{3.382727in}}%
\pgfpathlineto{\pgfqpoint{3.256873in}{0.637273in}}%
\pgfpathlineto{\pgfqpoint{3.263630in}{0.637273in}}%
\pgfpathlineto{\pgfqpoint{3.265174in}{3.382727in}}%
\pgfpathlineto{\pgfqpoint{3.265269in}{3.382727in}}%
\pgfpathlineto{\pgfqpoint{3.266813in}{0.637273in}}%
\pgfpathlineto{\pgfqpoint{3.500940in}{0.637273in}}%
\pgfpathlineto{\pgfqpoint{3.502234in}{3.382727in}}%
\pgfpathlineto{\pgfqpoint{3.502480in}{0.637273in}}%
\pgfpathlineto{\pgfqpoint{3.771717in}{0.637273in}}%
\pgfpathlineto{\pgfqpoint{3.772992in}{3.382727in}}%
\pgfpathlineto{\pgfqpoint{3.773257in}{0.637273in}}%
\pgfpathlineto{\pgfqpoint{3.812547in}{0.637273in}}%
\pgfpathlineto{\pgfqpoint{3.813822in}{3.382727in}}%
\pgfpathlineto{\pgfqpoint{3.814086in}{0.637273in}}%
\pgfpathlineto{\pgfqpoint{4.219114in}{0.637273in}}%
\pgfpathlineto{\pgfqpoint{4.220658in}{3.382727in}}%
\pgfpathlineto{\pgfqpoint{4.220662in}{3.382727in}}%
\pgfpathlineto{\pgfqpoint{4.222207in}{0.637273in}}%
\pgfpathlineto{\pgfqpoint{4.361741in}{0.637273in}}%
\pgfpathlineto{\pgfqpoint{4.363262in}{3.382727in}}%
\pgfpathlineto{\pgfqpoint{4.363281in}{0.637273in}}%
\pgfpathlineto{\pgfqpoint{4.363300in}{0.637273in}}%
\pgfpathlineto{\pgfqpoint{4.363371in}{3.382727in}}%
\pgfpathlineto{\pgfqpoint{4.364839in}{0.637273in}}%
\pgfpathlineto{\pgfqpoint{4.459002in}{0.637273in}}%
\pgfpathlineto{\pgfqpoint{4.460508in}{3.382727in}}%
\pgfpathlineto{\pgfqpoint{4.460542in}{0.637273in}}%
\pgfpathlineto{\pgfqpoint{4.493371in}{0.637273in}}%
\pgfpathlineto{\pgfqpoint{4.494778in}{3.382727in}}%
\pgfpathlineto{\pgfqpoint{4.494911in}{0.637273in}}%
\pgfpathlineto{\pgfqpoint{4.522276in}{0.637273in}}%
\pgfpathlineto{\pgfqpoint{4.523650in}{3.382727in}}%
\pgfpathlineto{\pgfqpoint{4.523816in}{0.637273in}}%
\pgfpathlineto{\pgfqpoint{4.523858in}{0.637273in}}%
\pgfpathlineto{\pgfqpoint{4.523953in}{3.382727in}}%
\pgfpathlineto{\pgfqpoint{4.525398in}{0.637273in}}%
\pgfpathlineto{\pgfqpoint{4.562468in}{0.637273in}}%
\pgfpathlineto{\pgfqpoint{4.564012in}{3.382727in}}%
\pgfpathlineto{\pgfqpoint{4.565358in}{3.382727in}}%
\pgfpathlineto{\pgfqpoint{4.566902in}{0.637273in}}%
\pgfpathlineto{\pgfqpoint{4.570916in}{0.637273in}}%
\pgfpathlineto{\pgfqpoint{4.572319in}{3.382727in}}%
\pgfpathlineto{\pgfqpoint{4.572456in}{0.637273in}}%
\pgfpathlineto{\pgfqpoint{4.572550in}{0.637273in}}%
\pgfpathlineto{\pgfqpoint{4.574033in}{3.382727in}}%
\pgfpathlineto{\pgfqpoint{4.574089in}{0.637273in}}%
\pgfpathlineto{\pgfqpoint{4.574264in}{0.637273in}}%
\pgfpathlineto{\pgfqpoint{4.574292in}{3.382727in}}%
\pgfpathlineto{\pgfqpoint{4.575804in}{0.637273in}}%
\pgfpathlineto{\pgfqpoint{4.580082in}{0.637273in}}%
\pgfpathlineto{\pgfqpoint{4.581485in}{3.382727in}}%
\pgfpathlineto{\pgfqpoint{4.581622in}{0.637273in}}%
\pgfpathlineto{\pgfqpoint{4.581716in}{0.637273in}}%
\pgfpathlineto{\pgfqpoint{4.583260in}{3.382727in}}%
\pgfpathlineto{\pgfqpoint{4.584545in}{3.382727in}}%
\pgfpathlineto{\pgfqpoint{4.584752in}{0.637273in}}%
\pgfpathlineto{\pgfqpoint{4.586084in}{3.382727in}}%
\pgfpathlineto{\pgfqpoint{4.586127in}{3.382727in}}%
\pgfpathlineto{\pgfqpoint{4.587671in}{0.637273in}}%
\pgfpathlineto{\pgfqpoint{4.591826in}{0.637273in}}%
\pgfpathlineto{\pgfqpoint{4.593342in}{3.382727in}}%
\pgfpathlineto{\pgfqpoint{4.593366in}{0.637273in}}%
\pgfpathlineto{\pgfqpoint{4.593380in}{0.637273in}}%
\pgfpathlineto{\pgfqpoint{4.594882in}{3.382727in}}%
\pgfpathlineto{\pgfqpoint{4.594920in}{0.637273in}}%
\pgfpathlineto{\pgfqpoint{4.595113in}{0.637273in}}%
\pgfpathlineto{\pgfqpoint{4.595141in}{3.382727in}}%
\pgfpathlineto{\pgfqpoint{4.596653in}{0.637273in}}%
\pgfpathlineto{\pgfqpoint{4.600662in}{0.637273in}}%
\pgfpathlineto{\pgfqpoint{4.602187in}{3.382727in}}%
\pgfpathlineto{\pgfqpoint{4.602201in}{0.637273in}}%
\pgfpathlineto{\pgfqpoint{4.602291in}{0.637273in}}%
\pgfpathlineto{\pgfqpoint{4.603774in}{3.382727in}}%
\pgfpathlineto{\pgfqpoint{4.603831in}{0.637273in}}%
\pgfpathlineto{\pgfqpoint{4.604005in}{0.637273in}}%
\pgfpathlineto{\pgfqpoint{4.605526in}{3.382727in}}%
\pgfpathlineto{\pgfqpoint{4.605545in}{0.637273in}}%
\pgfpathlineto{\pgfqpoint{4.605668in}{0.637273in}}%
\pgfpathlineto{\pgfqpoint{4.605757in}{3.382727in}}%
\pgfpathlineto{\pgfqpoint{4.607207in}{0.637273in}}%
\pgfpathlineto{\pgfqpoint{4.611221in}{0.637273in}}%
\pgfpathlineto{\pgfqpoint{4.612600in}{3.382727in}}%
\pgfpathlineto{\pgfqpoint{4.612760in}{0.637273in}}%
\pgfpathlineto{\pgfqpoint{4.612770in}{0.637273in}}%
\pgfpathlineto{\pgfqpoint{4.614314in}{3.382727in}}%
\pgfpathlineto{\pgfqpoint{4.614361in}{3.382727in}}%
\pgfpathlineto{\pgfqpoint{4.614682in}{0.637273in}}%
\pgfpathlineto{\pgfqpoint{4.615901in}{3.382727in}}%
\pgfpathlineto{\pgfqpoint{4.615948in}{3.382727in}}%
\pgfpathlineto{\pgfqpoint{4.616227in}{0.637273in}}%
\pgfpathlineto{\pgfqpoint{4.617487in}{3.382727in}}%
\pgfpathlineto{\pgfqpoint{4.617605in}{3.382727in}}%
\pgfpathlineto{\pgfqpoint{4.619150in}{0.637273in}}%
\pgfpathlineto{\pgfqpoint{4.623371in}{0.637273in}}%
\pgfpathlineto{\pgfqpoint{4.624906in}{3.382727in}}%
\pgfpathlineto{\pgfqpoint{4.624911in}{0.637273in}}%
\pgfpathlineto{\pgfqpoint{4.624949in}{0.637273in}}%
\pgfpathlineto{\pgfqpoint{4.626441in}{3.382727in}}%
\pgfpathlineto{\pgfqpoint{4.626488in}{0.637273in}}%
\pgfpathlineto{\pgfqpoint{4.626601in}{0.637273in}}%
\pgfpathlineto{\pgfqpoint{4.628146in}{3.382727in}}%
\pgfpathlineto{\pgfqpoint{4.628240in}{3.382727in}}%
\pgfpathlineto{\pgfqpoint{4.629784in}{0.637273in}}%
\pgfpathlineto{\pgfqpoint{4.629912in}{0.637273in}}%
\pgfpathlineto{\pgfqpoint{4.631447in}{3.382727in}}%
\pgfpathlineto{\pgfqpoint{4.631451in}{0.637273in}}%
\pgfpathlineto{\pgfqpoint{4.631456in}{0.637273in}}%
\pgfpathlineto{\pgfqpoint{4.632977in}{3.382727in}}%
\pgfpathlineto{\pgfqpoint{4.632995in}{0.637273in}}%
\pgfpathlineto{\pgfqpoint{4.633033in}{0.637273in}}%
\pgfpathlineto{\pgfqpoint{4.634559in}{3.382727in}}%
\pgfpathlineto{\pgfqpoint{4.634573in}{0.637273in}}%
\pgfpathlineto{\pgfqpoint{4.634705in}{0.637273in}}%
\pgfpathlineto{\pgfqpoint{4.636235in}{3.382727in}}%
\pgfpathlineto{\pgfqpoint{4.636244in}{0.637273in}}%
\pgfpathlineto{\pgfqpoint{4.636296in}{0.637273in}}%
\pgfpathlineto{\pgfqpoint{4.637822in}{3.382727in}}%
\pgfpathlineto{\pgfqpoint{4.637836in}{0.637273in}}%
\pgfpathlineto{\pgfqpoint{4.638067in}{0.637273in}}%
\pgfpathlineto{\pgfqpoint{4.639611in}{3.382727in}}%
\pgfpathlineto{\pgfqpoint{4.640886in}{3.382727in}}%
\pgfpathlineto{\pgfqpoint{4.642431in}{0.637273in}}%
\pgfpathlineto{\pgfqpoint{4.642435in}{0.637273in}}%
\pgfpathlineto{\pgfqpoint{4.643980in}{3.382727in}}%
\pgfpathlineto{\pgfqpoint{4.645340in}{3.382727in}}%
\pgfpathlineto{\pgfqpoint{4.646884in}{0.637273in}}%
\pgfpathlineto{\pgfqpoint{4.646903in}{0.637273in}}%
\pgfpathlineto{\pgfqpoint{4.648447in}{3.382727in}}%
\pgfpathlineto{\pgfqpoint{4.649363in}{3.382727in}}%
\pgfpathlineto{\pgfqpoint{4.650827in}{0.637273in}}%
\pgfpathlineto{\pgfqpoint{4.650902in}{3.382727in}}%
\pgfpathlineto{\pgfqpoint{4.652248in}{3.382727in}}%
\pgfpathlineto{\pgfqpoint{4.652739in}{0.637273in}}%
\pgfpathlineto{\pgfqpoint{4.653788in}{3.382727in}}%
\pgfpathlineto{\pgfqpoint{4.654085in}{3.382727in}}%
\pgfpathlineto{\pgfqpoint{4.655630in}{0.637273in}}%
\pgfpathlineto{\pgfqpoint{4.655936in}{0.637273in}}%
\pgfpathlineto{\pgfqpoint{4.657481in}{3.382727in}}%
\pgfpathlineto{\pgfqpoint{4.657547in}{3.382727in}}%
\pgfpathlineto{\pgfqpoint{4.659091in}{0.637273in}}%
\pgfpathlineto{\pgfqpoint{4.661211in}{0.637273in}}%
\pgfpathlineto{\pgfqpoint{4.662755in}{3.382727in}}%
\pgfpathlineto{\pgfqpoint{4.663247in}{3.382727in}}%
\pgfpathlineto{\pgfqpoint{4.664791in}{0.637273in}}%
\pgfpathlineto{\pgfqpoint{4.666335in}{3.382727in}}%
\pgfpathlineto{\pgfqpoint{4.666642in}{3.382727in}}%
\pgfpathlineto{\pgfqpoint{4.668186in}{0.637273in}}%
\pgfpathlineto{\pgfqpoint{4.668191in}{0.637273in}}%
\pgfpathlineto{\pgfqpoint{4.669735in}{3.382727in}}%
\pgfpathlineto{\pgfqpoint{4.670292in}{3.382727in}}%
\pgfpathlineto{\pgfqpoint{4.671836in}{0.637273in}}%
\pgfpathlineto{\pgfqpoint{4.675765in}{0.637273in}}%
\pgfpathlineto{\pgfqpoint{4.676101in}{3.382727in}}%
\pgfpathlineto{\pgfqpoint{4.677305in}{0.637273in}}%
\pgfpathlineto{\pgfqpoint{4.680162in}{0.637273in}}%
\pgfpathlineto{\pgfqpoint{4.681706in}{3.382727in}}%
\pgfpathlineto{\pgfqpoint{4.690461in}{3.382727in}}%
\pgfpathlineto{\pgfqpoint{4.690900in}{0.637273in}}%
\pgfpathlineto{\pgfqpoint{4.692001in}{3.382727in}}%
\pgfpathlineto{\pgfqpoint{4.694716in}{3.382727in}}%
\pgfpathlineto{\pgfqpoint{4.694735in}{0.637273in}}%
\pgfpathlineto{\pgfqpoint{4.696256in}{3.382727in}}%
\pgfpathlineto{\pgfqpoint{4.697752in}{3.382727in}}%
\pgfpathlineto{\pgfqpoint{4.698428in}{0.637273in}}%
\pgfpathlineto{\pgfqpoint{4.699292in}{3.382727in}}%
\pgfpathlineto{\pgfqpoint{4.699774in}{3.382727in}}%
\pgfpathlineto{\pgfqpoint{4.701313in}{0.637273in}}%
\pgfpathlineto{\pgfqpoint{4.702848in}{3.382727in}}%
\pgfpathlineto{\pgfqpoint{4.702853in}{0.637273in}}%
\pgfpathlineto{\pgfqpoint{4.702867in}{0.637273in}}%
\pgfpathlineto{\pgfqpoint{4.704411in}{3.382727in}}%
\pgfpathlineto{\pgfqpoint{4.763190in}{3.382727in}}%
\pgfpathlineto{\pgfqpoint{4.763232in}{0.637273in}}%
\pgfpathlineto{\pgfqpoint{4.764729in}{3.382727in}}%
\pgfpathlineto{\pgfqpoint{4.768450in}{3.382727in}}%
\pgfpathlineto{\pgfqpoint{4.769801in}{0.637273in}}%
\pgfpathlineto{\pgfqpoint{4.769990in}{3.382727in}}%
\pgfpathlineto{\pgfqpoint{4.770060in}{3.382727in}}%
\pgfpathlineto{\pgfqpoint{4.771605in}{0.637273in}}%
\pgfpathlineto{\pgfqpoint{4.771973in}{0.637273in}}%
\pgfpathlineto{\pgfqpoint{4.773347in}{3.382727in}}%
\pgfpathlineto{\pgfqpoint{4.773512in}{0.637273in}}%
\pgfpathlineto{\pgfqpoint{4.773621in}{0.637273in}}%
\pgfpathlineto{\pgfqpoint{4.773654in}{3.382727in}}%
\pgfpathlineto{\pgfqpoint{4.775161in}{0.637273in}}%
\pgfpathlineto{\pgfqpoint{4.779269in}{0.637273in}}%
\pgfpathlineto{\pgfqpoint{4.780671in}{3.382727in}}%
\pgfpathlineto{\pgfqpoint{4.780808in}{0.637273in}}%
\pgfpathlineto{\pgfqpoint{4.780912in}{0.637273in}}%
\pgfpathlineto{\pgfqpoint{4.782452in}{3.382727in}}%
\pgfpathlineto{\pgfqpoint{4.782631in}{0.637273in}}%
\pgfpathlineto{\pgfqpoint{4.783991in}{3.382727in}}%
\pgfpathlineto{\pgfqpoint{4.784005in}{3.382727in}}%
\pgfpathlineto{\pgfqpoint{4.785550in}{0.637273in}}%
\pgfpathlineto{\pgfqpoint{4.932725in}{0.637273in}}%
\pgfpathlineto{\pgfqpoint{4.934000in}{3.382727in}}%
\pgfpathlineto{\pgfqpoint{4.934265in}{0.637273in}}%
\pgfpathlineto{\pgfqpoint{4.939742in}{0.637273in}}%
\pgfpathlineto{\pgfqpoint{4.941017in}{3.382727in}}%
\pgfpathlineto{\pgfqpoint{4.941282in}{0.637273in}}%
\pgfpathlineto{\pgfqpoint{5.074933in}{0.637273in}}%
\pgfpathlineto{\pgfqpoint{5.074942in}{3.382727in}}%
\pgfpathlineto{\pgfqpoint{5.076472in}{0.637273in}}%
\pgfpathlineto{\pgfqpoint{5.078267in}{0.637273in}}%
\pgfpathlineto{\pgfqpoint{5.078271in}{3.382727in}}%
\pgfpathlineto{\pgfqpoint{5.079806in}{0.637273in}}%
\pgfpathlineto{\pgfqpoint{5.087532in}{0.637273in}}%
\pgfpathlineto{\pgfqpoint{5.087928in}{3.382727in}}%
\pgfpathlineto{\pgfqpoint{5.089071in}{0.637273in}}%
\pgfpathlineto{\pgfqpoint{5.089780in}{0.637273in}}%
\pgfpathlineto{\pgfqpoint{5.089865in}{3.382727in}}%
\pgfpathlineto{\pgfqpoint{5.091319in}{0.637273in}}%
\pgfpathlineto{\pgfqpoint{5.101000in}{0.637273in}}%
\pgfpathlineto{\pgfqpoint{5.101151in}{3.382727in}}%
\pgfpathlineto{\pgfqpoint{5.102539in}{0.637273in}}%
\pgfpathlineto{\pgfqpoint{5.105495in}{0.637273in}}%
\pgfpathlineto{\pgfqpoint{5.106818in}{3.382727in}}%
\pgfpathlineto{\pgfqpoint{5.107035in}{0.637273in}}%
\pgfpathlineto{\pgfqpoint{5.113911in}{0.637273in}}%
\pgfpathlineto{\pgfqpoint{5.113953in}{3.382727in}}%
\pgfpathlineto{\pgfqpoint{5.115450in}{0.637273in}}%
\pgfpathlineto{\pgfqpoint{5.119488in}{0.637273in}}%
\pgfpathlineto{\pgfqpoint{5.121032in}{3.382727in}}%
\pgfpathlineto{\pgfqpoint{5.122491in}{3.382727in}}%
\pgfpathlineto{\pgfqpoint{5.123247in}{0.637273in}}%
\pgfpathlineto{\pgfqpoint{5.124030in}{3.382727in}}%
\pgfpathlineto{\pgfqpoint{5.124238in}{3.382727in}}%
\pgfpathlineto{\pgfqpoint{5.125735in}{0.637273in}}%
\pgfpathlineto{\pgfqpoint{5.125778in}{3.382727in}}%
\pgfpathlineto{\pgfqpoint{5.125806in}{3.382727in}}%
\pgfpathlineto{\pgfqpoint{5.127298in}{0.637273in}}%
\pgfpathlineto{\pgfqpoint{5.127346in}{3.382727in}}%
\pgfpathlineto{\pgfqpoint{5.127369in}{3.382727in}}%
\pgfpathlineto{\pgfqpoint{5.128913in}{0.637273in}}%
\pgfpathlineto{\pgfqpoint{5.134287in}{0.637273in}}%
\pgfpathlineto{\pgfqpoint{5.134297in}{3.382727in}}%
\pgfpathlineto{\pgfqpoint{5.135827in}{0.637273in}}%
\pgfpathlineto{\pgfqpoint{5.148941in}{0.637273in}}%
\pgfpathlineto{\pgfqpoint{5.149196in}{3.382727in}}%
\pgfpathlineto{\pgfqpoint{5.150480in}{0.637273in}}%
\pgfpathlineto{\pgfqpoint{5.151363in}{0.637273in}}%
\pgfpathlineto{\pgfqpoint{5.151510in}{3.382727in}}%
\pgfpathlineto{\pgfqpoint{5.152903in}{0.637273in}}%
\pgfpathlineto{\pgfqpoint{5.162574in}{0.637273in}}%
\pgfpathlineto{\pgfqpoint{5.163183in}{3.382727in}}%
\pgfpathlineto{\pgfqpoint{5.164113in}{0.637273in}}%
\pgfpathlineto{\pgfqpoint{5.170021in}{0.637273in}}%
\pgfpathlineto{\pgfqpoint{5.170026in}{3.382727in}}%
\pgfpathlineto{\pgfqpoint{5.171560in}{0.637273in}}%
\pgfpathlineto{\pgfqpoint{5.177317in}{0.637273in}}%
\pgfpathlineto{\pgfqpoint{5.177355in}{3.382727in}}%
\pgfpathlineto{\pgfqpoint{5.178856in}{0.637273in}}%
\pgfpathlineto{\pgfqpoint{5.188636in}{0.637273in}}%
\pgfpathlineto{\pgfqpoint{5.188636in}{0.637273in}}%
\pgfusepath{stroke}%
\end{pgfscope}%
\begin{pgfscope}%
\pgfpathrectangle{\pgfqpoint{0.750000in}{0.500000in}}{\pgfqpoint{4.650000in}{3.020000in}}%
\pgfusepath{clip}%
\pgfsetrectcap%
\pgfsetroundjoin%
\pgfsetlinewidth{1.505625pt}%
\definecolor{currentstroke}{rgb}{1.000000,0.000000,0.000000}%
\pgfsetstrokecolor{currentstroke}%
\pgfsetdash{}{0pt}%
\pgfpathmoveto{\pgfqpoint{1.013309in}{0.500000in}}%
\pgfpathlineto{\pgfqpoint{1.013309in}{3.520000in}}%
\pgfusepath{stroke}%
\end{pgfscope}%
\begin{pgfscope}%
\pgfpathrectangle{\pgfqpoint{0.750000in}{0.500000in}}{\pgfqpoint{4.650000in}{3.020000in}}%
\pgfusepath{clip}%
\pgfsetrectcap%
\pgfsetroundjoin%
\pgfsetlinewidth{1.505625pt}%
\definecolor{currentstroke}{rgb}{1.000000,0.000000,0.000000}%
\pgfsetstrokecolor{currentstroke}%
\pgfsetdash{}{0pt}%
\pgfpathmoveto{\pgfqpoint{1.967640in}{0.500000in}}%
\pgfpathlineto{\pgfqpoint{1.967640in}{3.520000in}}%
\pgfusepath{stroke}%
\end{pgfscope}%
\begin{pgfscope}%
\pgfpathrectangle{\pgfqpoint{0.750000in}{0.500000in}}{\pgfqpoint{4.650000in}{3.020000in}}%
\pgfusepath{clip}%
\pgfsetrectcap%
\pgfsetroundjoin%
\pgfsetlinewidth{1.505625pt}%
\definecolor{currentstroke}{rgb}{1.000000,0.000000,0.000000}%
\pgfsetstrokecolor{currentstroke}%
\pgfsetdash{}{0pt}%
\pgfpathmoveto{\pgfqpoint{2.081514in}{0.500000in}}%
\pgfpathlineto{\pgfqpoint{2.081514in}{3.520000in}}%
\pgfusepath{stroke}%
\end{pgfscope}%
\begin{pgfscope}%
\pgfpathrectangle{\pgfqpoint{0.750000in}{0.500000in}}{\pgfqpoint{4.650000in}{3.020000in}}%
\pgfusepath{clip}%
\pgfsetrectcap%
\pgfsetroundjoin%
\pgfsetlinewidth{1.505625pt}%
\definecolor{currentstroke}{rgb}{1.000000,0.000000,0.000000}%
\pgfsetstrokecolor{currentstroke}%
\pgfsetdash{}{0pt}%
\pgfpathmoveto{\pgfqpoint{3.253354in}{0.500000in}}%
\pgfpathlineto{\pgfqpoint{3.253354in}{3.520000in}}%
\pgfusepath{stroke}%
\end{pgfscope}%
\begin{pgfscope}%
\pgfpathrectangle{\pgfqpoint{0.750000in}{0.500000in}}{\pgfqpoint{4.650000in}{3.020000in}}%
\pgfusepath{clip}%
\pgfsetrectcap%
\pgfsetroundjoin%
\pgfsetlinewidth{1.505625pt}%
\definecolor{currentstroke}{rgb}{1.000000,0.000000,0.000000}%
\pgfsetstrokecolor{currentstroke}%
\pgfsetdash{}{0pt}%
\pgfpathmoveto{\pgfqpoint{4.680738in}{0.500000in}}%
\pgfpathlineto{\pgfqpoint{4.680738in}{3.520000in}}%
\pgfusepath{stroke}%
\end{pgfscope}%
\begin{pgfscope}%
\pgfpathrectangle{\pgfqpoint{0.750000in}{0.500000in}}{\pgfqpoint{4.650000in}{3.020000in}}%
\pgfusepath{clip}%
\pgfsetrectcap%
\pgfsetroundjoin%
\pgfsetlinewidth{1.505625pt}%
\definecolor{currentstroke}{rgb}{1.000000,0.000000,0.000000}%
\pgfsetstrokecolor{currentstroke}%
\pgfsetdash{}{0pt}%
\pgfpathmoveto{\pgfqpoint{4.817326in}{0.500000in}}%
\pgfpathlineto{\pgfqpoint{4.817326in}{3.520000in}}%
\pgfusepath{stroke}%
\end{pgfscope}%
\begin{pgfscope}%
\pgfsetrectcap%
\pgfsetmiterjoin%
\pgfsetlinewidth{0.803000pt}%
\definecolor{currentstroke}{rgb}{0.000000,0.000000,0.000000}%
\pgfsetstrokecolor{currentstroke}%
\pgfsetdash{}{0pt}%
\pgfpathmoveto{\pgfqpoint{0.750000in}{0.500000in}}%
\pgfpathlineto{\pgfqpoint{0.750000in}{3.520000in}}%
\pgfusepath{stroke}%
\end{pgfscope}%
\begin{pgfscope}%
\pgfsetrectcap%
\pgfsetmiterjoin%
\pgfsetlinewidth{0.803000pt}%
\definecolor{currentstroke}{rgb}{0.000000,0.000000,0.000000}%
\pgfsetstrokecolor{currentstroke}%
\pgfsetdash{}{0pt}%
\pgfpathmoveto{\pgfqpoint{5.400000in}{0.500000in}}%
\pgfpathlineto{\pgfqpoint{5.400000in}{3.520000in}}%
\pgfusepath{stroke}%
\end{pgfscope}%
\begin{pgfscope}%
\pgfsetrectcap%
\pgfsetmiterjoin%
\pgfsetlinewidth{0.803000pt}%
\definecolor{currentstroke}{rgb}{0.000000,0.000000,0.000000}%
\pgfsetstrokecolor{currentstroke}%
\pgfsetdash{}{0pt}%
\pgfpathmoveto{\pgfqpoint{0.750000in}{0.500000in}}%
\pgfpathlineto{\pgfqpoint{5.400000in}{0.500000in}}%
\pgfusepath{stroke}%
\end{pgfscope}%
\begin{pgfscope}%
\pgfsetrectcap%
\pgfsetmiterjoin%
\pgfsetlinewidth{0.803000pt}%
\definecolor{currentstroke}{rgb}{0.000000,0.000000,0.000000}%
\pgfsetstrokecolor{currentstroke}%
\pgfsetdash{}{0pt}%
\pgfpathmoveto{\pgfqpoint{0.750000in}{3.520000in}}%
\pgfpathlineto{\pgfqpoint{5.400000in}{3.520000in}}%
\pgfusepath{stroke}%
\end{pgfscope}%
\end{pgfpicture}%
\makeatother%
\endgroup%

    \caption{BETH Suspicious Outliers [Red Line = Anomaly]}
    \label{fig:beth_sus_outliers}
\end{figure}

Figure \ref{fig:beth_evil_outliers} shows the algorithms detection results compared to the values considered evil by the creators of the BETH dataset. The algorithm was able to detect the starts of the outliers within the debounce threshold and was able to detect the end of the faults with some delay. There are 2 false positives from the detector, which correspond to high UserID values in the dataset. These two locations are marked as suspicious so in this case the false positive would still be worth investigating even if it is deemed not dangerous.
 
\begin{figure}[H]
    %%\centering
    %% Creator: Matplotlib, PGF backend
%%
%% To include the figure in your LaTeX document, write
%%   \input{<filename>.pgf}
%%
%% Make sure the required packages are loaded in your preamble
%%   \usepackage{pgf}
%%
%% Also ensure that all the required font packages are loaded; for instance,
%% the lmodern package is sometimes necessary when using math font.
%%   \usepackage{lmodern}
%%
%% Figures using additional raster images can only be included by \input if
%% they are in the same directory as the main LaTeX file. For loading figures
%% from other directories you can use the `import` package
%%   \usepackage{import}
%%
%% and then include the figures with
%%   \import{<path to file>}{<filename>.pgf}
%%
%% Matplotlib used the following preamble
%%
\begingroup%
\makeatletter%
\begin{pgfpicture}%
\pgfpathrectangle{\pgfpointorigin}{\pgfqpoint{6.000000in}{4.000000in}}%
\pgfusepath{use as bounding box, clip}%
\begin{pgfscope}%
\pgfsetbuttcap%
\pgfsetmiterjoin%
\pgfsetlinewidth{0.000000pt}%
\definecolor{currentstroke}{rgb}{1.000000,1.000000,1.000000}%
\pgfsetstrokecolor{currentstroke}%
\pgfsetstrokeopacity{0.000000}%
\pgfsetdash{}{0pt}%
\pgfpathmoveto{\pgfqpoint{0.000000in}{0.000000in}}%
\pgfpathlineto{\pgfqpoint{6.000000in}{0.000000in}}%
\pgfpathlineto{\pgfqpoint{6.000000in}{4.000000in}}%
\pgfpathlineto{\pgfqpoint{0.000000in}{4.000000in}}%
\pgfpathlineto{\pgfqpoint{0.000000in}{0.000000in}}%
\pgfpathclose%
\pgfusepath{}%
\end{pgfscope}%
\begin{pgfscope}%
\pgfsetbuttcap%
\pgfsetmiterjoin%
\definecolor{currentfill}{rgb}{1.000000,1.000000,1.000000}%
\pgfsetfillcolor{currentfill}%
\pgfsetlinewidth{0.000000pt}%
\definecolor{currentstroke}{rgb}{0.000000,0.000000,0.000000}%
\pgfsetstrokecolor{currentstroke}%
\pgfsetstrokeopacity{0.000000}%
\pgfsetdash{}{0pt}%
\pgfpathmoveto{\pgfqpoint{0.750000in}{0.500000in}}%
\pgfpathlineto{\pgfqpoint{5.400000in}{0.500000in}}%
\pgfpathlineto{\pgfqpoint{5.400000in}{3.520000in}}%
\pgfpathlineto{\pgfqpoint{0.750000in}{3.520000in}}%
\pgfpathlineto{\pgfqpoint{0.750000in}{0.500000in}}%
\pgfpathclose%
\pgfusepath{fill}%
\end{pgfscope}%
\begin{pgfscope}%
\pgfsetbuttcap%
\pgfsetroundjoin%
\definecolor{currentfill}{rgb}{0.000000,0.000000,0.000000}%
\pgfsetfillcolor{currentfill}%
\pgfsetlinewidth{0.803000pt}%
\definecolor{currentstroke}{rgb}{0.000000,0.000000,0.000000}%
\pgfsetstrokecolor{currentstroke}%
\pgfsetdash{}{0pt}%
\pgfsys@defobject{currentmarker}{\pgfqpoint{0.000000in}{-0.048611in}}{\pgfqpoint{0.000000in}{0.000000in}}{%
\pgfpathmoveto{\pgfqpoint{0.000000in}{0.000000in}}%
\pgfpathlineto{\pgfqpoint{0.000000in}{-0.048611in}}%
\pgfusepath{stroke,fill}%
}%
\begin{pgfscope}%
\pgfsys@transformshift{0.961364in}{0.500000in}%
\pgfsys@useobject{currentmarker}{}%
\end{pgfscope}%
\end{pgfscope}%
\begin{pgfscope}%
\definecolor{textcolor}{rgb}{0.000000,0.000000,0.000000}%
\pgfsetstrokecolor{textcolor}%
\pgfsetfillcolor{textcolor}%
\pgftext[x=0.961364in,y=0.402778in,,top]{\color{textcolor}\rmfamily\fontsize{10.000000}{12.000000}\selectfont \(\displaystyle {0}\)}%
\end{pgfscope}%
\begin{pgfscope}%
\pgfsetbuttcap%
\pgfsetroundjoin%
\definecolor{currentfill}{rgb}{0.000000,0.000000,0.000000}%
\pgfsetfillcolor{currentfill}%
\pgfsetlinewidth{0.803000pt}%
\definecolor{currentstroke}{rgb}{0.000000,0.000000,0.000000}%
\pgfsetstrokecolor{currentstroke}%
\pgfsetdash{}{0pt}%
\pgfsys@defobject{currentmarker}{\pgfqpoint{0.000000in}{-0.048611in}}{\pgfqpoint{0.000000in}{0.000000in}}{%
\pgfpathmoveto{\pgfqpoint{0.000000in}{0.000000in}}%
\pgfpathlineto{\pgfqpoint{0.000000in}{-0.048611in}}%
\pgfusepath{stroke,fill}%
}%
\begin{pgfscope}%
\pgfsys@transformshift{1.905825in}{0.500000in}%
\pgfsys@useobject{currentmarker}{}%
\end{pgfscope}%
\end{pgfscope}%
\begin{pgfscope}%
\definecolor{textcolor}{rgb}{0.000000,0.000000,0.000000}%
\pgfsetstrokecolor{textcolor}%
\pgfsetfillcolor{textcolor}%
\pgftext[x=1.905825in,y=0.402778in,,top]{\color{textcolor}\rmfamily\fontsize{10.000000}{12.000000}\selectfont \(\displaystyle {200000}\)}%
\end{pgfscope}%
\begin{pgfscope}%
\pgfsetbuttcap%
\pgfsetroundjoin%
\definecolor{currentfill}{rgb}{0.000000,0.000000,0.000000}%
\pgfsetfillcolor{currentfill}%
\pgfsetlinewidth{0.803000pt}%
\definecolor{currentstroke}{rgb}{0.000000,0.000000,0.000000}%
\pgfsetstrokecolor{currentstroke}%
\pgfsetdash{}{0pt}%
\pgfsys@defobject{currentmarker}{\pgfqpoint{0.000000in}{-0.048611in}}{\pgfqpoint{0.000000in}{0.000000in}}{%
\pgfpathmoveto{\pgfqpoint{0.000000in}{0.000000in}}%
\pgfpathlineto{\pgfqpoint{0.000000in}{-0.048611in}}%
\pgfusepath{stroke,fill}%
}%
\begin{pgfscope}%
\pgfsys@transformshift{2.850287in}{0.500000in}%
\pgfsys@useobject{currentmarker}{}%
\end{pgfscope}%
\end{pgfscope}%
\begin{pgfscope}%
\definecolor{textcolor}{rgb}{0.000000,0.000000,0.000000}%
\pgfsetstrokecolor{textcolor}%
\pgfsetfillcolor{textcolor}%
\pgftext[x=2.850287in,y=0.402778in,,top]{\color{textcolor}\rmfamily\fontsize{10.000000}{12.000000}\selectfont \(\displaystyle {400000}\)}%
\end{pgfscope}%
\begin{pgfscope}%
\pgfsetbuttcap%
\pgfsetroundjoin%
\definecolor{currentfill}{rgb}{0.000000,0.000000,0.000000}%
\pgfsetfillcolor{currentfill}%
\pgfsetlinewidth{0.803000pt}%
\definecolor{currentstroke}{rgb}{0.000000,0.000000,0.000000}%
\pgfsetstrokecolor{currentstroke}%
\pgfsetdash{}{0pt}%
\pgfsys@defobject{currentmarker}{\pgfqpoint{0.000000in}{-0.048611in}}{\pgfqpoint{0.000000in}{0.000000in}}{%
\pgfpathmoveto{\pgfqpoint{0.000000in}{0.000000in}}%
\pgfpathlineto{\pgfqpoint{0.000000in}{-0.048611in}}%
\pgfusepath{stroke,fill}%
}%
\begin{pgfscope}%
\pgfsys@transformshift{3.794748in}{0.500000in}%
\pgfsys@useobject{currentmarker}{}%
\end{pgfscope}%
\end{pgfscope}%
\begin{pgfscope}%
\definecolor{textcolor}{rgb}{0.000000,0.000000,0.000000}%
\pgfsetstrokecolor{textcolor}%
\pgfsetfillcolor{textcolor}%
\pgftext[x=3.794748in,y=0.402778in,,top]{\color{textcolor}\rmfamily\fontsize{10.000000}{12.000000}\selectfont \(\displaystyle {600000}\)}%
\end{pgfscope}%
\begin{pgfscope}%
\pgfsetbuttcap%
\pgfsetroundjoin%
\definecolor{currentfill}{rgb}{0.000000,0.000000,0.000000}%
\pgfsetfillcolor{currentfill}%
\pgfsetlinewidth{0.803000pt}%
\definecolor{currentstroke}{rgb}{0.000000,0.000000,0.000000}%
\pgfsetstrokecolor{currentstroke}%
\pgfsetdash{}{0pt}%
\pgfsys@defobject{currentmarker}{\pgfqpoint{0.000000in}{-0.048611in}}{\pgfqpoint{0.000000in}{0.000000in}}{%
\pgfpathmoveto{\pgfqpoint{0.000000in}{0.000000in}}%
\pgfpathlineto{\pgfqpoint{0.000000in}{-0.048611in}}%
\pgfusepath{stroke,fill}%
}%
\begin{pgfscope}%
\pgfsys@transformshift{4.739210in}{0.500000in}%
\pgfsys@useobject{currentmarker}{}%
\end{pgfscope}%
\end{pgfscope}%
\begin{pgfscope}%
\definecolor{textcolor}{rgb}{0.000000,0.000000,0.000000}%
\pgfsetstrokecolor{textcolor}%
\pgfsetfillcolor{textcolor}%
\pgftext[x=4.739210in,y=0.402778in,,top]{\color{textcolor}\rmfamily\fontsize{10.000000}{12.000000}\selectfont \(\displaystyle {800000}\)}%
\end{pgfscope}%
\begin{pgfscope}%
\definecolor{textcolor}{rgb}{0.000000,0.000000,0.000000}%
\pgfsetstrokecolor{textcolor}%
\pgfsetfillcolor{textcolor}%
\pgftext[x=3.075000in,y=0.223766in,,top]{\color{textcolor}\rmfamily\fontsize{10.000000}{12.000000}\selectfont time}%
\end{pgfscope}%
\begin{pgfscope}%
\pgfsetbuttcap%
\pgfsetroundjoin%
\definecolor{currentfill}{rgb}{0.000000,0.000000,0.000000}%
\pgfsetfillcolor{currentfill}%
\pgfsetlinewidth{0.803000pt}%
\definecolor{currentstroke}{rgb}{0.000000,0.000000,0.000000}%
\pgfsetstrokecolor{currentstroke}%
\pgfsetdash{}{0pt}%
\pgfsys@defobject{currentmarker}{\pgfqpoint{-0.048611in}{0.000000in}}{\pgfqpoint{-0.000000in}{0.000000in}}{%
\pgfpathmoveto{\pgfqpoint{-0.000000in}{0.000000in}}%
\pgfpathlineto{\pgfqpoint{-0.048611in}{0.000000in}}%
\pgfusepath{stroke,fill}%
}%
\begin{pgfscope}%
\pgfsys@transformshift{0.750000in}{0.637273in}%
\pgfsys@useobject{currentmarker}{}%
\end{pgfscope}%
\end{pgfscope}%
\begin{pgfscope}%
\definecolor{textcolor}{rgb}{0.000000,0.000000,0.000000}%
\pgfsetstrokecolor{textcolor}%
\pgfsetfillcolor{textcolor}%
\pgftext[x=0.475308in, y=0.589047in, left, base]{\color{textcolor}\rmfamily\fontsize{10.000000}{12.000000}\selectfont \(\displaystyle {0.0}\)}%
\end{pgfscope}%
\begin{pgfscope}%
\pgfsetbuttcap%
\pgfsetroundjoin%
\definecolor{currentfill}{rgb}{0.000000,0.000000,0.000000}%
\pgfsetfillcolor{currentfill}%
\pgfsetlinewidth{0.803000pt}%
\definecolor{currentstroke}{rgb}{0.000000,0.000000,0.000000}%
\pgfsetstrokecolor{currentstroke}%
\pgfsetdash{}{0pt}%
\pgfsys@defobject{currentmarker}{\pgfqpoint{-0.048611in}{0.000000in}}{\pgfqpoint{-0.000000in}{0.000000in}}{%
\pgfpathmoveto{\pgfqpoint{-0.000000in}{0.000000in}}%
\pgfpathlineto{\pgfqpoint{-0.048611in}{0.000000in}}%
\pgfusepath{stroke,fill}%
}%
\begin{pgfscope}%
\pgfsys@transformshift{0.750000in}{1.186364in}%
\pgfsys@useobject{currentmarker}{}%
\end{pgfscope}%
\end{pgfscope}%
\begin{pgfscope}%
\definecolor{textcolor}{rgb}{0.000000,0.000000,0.000000}%
\pgfsetstrokecolor{textcolor}%
\pgfsetfillcolor{textcolor}%
\pgftext[x=0.475308in, y=1.138138in, left, base]{\color{textcolor}\rmfamily\fontsize{10.000000}{12.000000}\selectfont \(\displaystyle {0.2}\)}%
\end{pgfscope}%
\begin{pgfscope}%
\pgfsetbuttcap%
\pgfsetroundjoin%
\definecolor{currentfill}{rgb}{0.000000,0.000000,0.000000}%
\pgfsetfillcolor{currentfill}%
\pgfsetlinewidth{0.803000pt}%
\definecolor{currentstroke}{rgb}{0.000000,0.000000,0.000000}%
\pgfsetstrokecolor{currentstroke}%
\pgfsetdash{}{0pt}%
\pgfsys@defobject{currentmarker}{\pgfqpoint{-0.048611in}{0.000000in}}{\pgfqpoint{-0.000000in}{0.000000in}}{%
\pgfpathmoveto{\pgfqpoint{-0.000000in}{0.000000in}}%
\pgfpathlineto{\pgfqpoint{-0.048611in}{0.000000in}}%
\pgfusepath{stroke,fill}%
}%
\begin{pgfscope}%
\pgfsys@transformshift{0.750000in}{1.735455in}%
\pgfsys@useobject{currentmarker}{}%
\end{pgfscope}%
\end{pgfscope}%
\begin{pgfscope}%
\definecolor{textcolor}{rgb}{0.000000,0.000000,0.000000}%
\pgfsetstrokecolor{textcolor}%
\pgfsetfillcolor{textcolor}%
\pgftext[x=0.475308in, y=1.687229in, left, base]{\color{textcolor}\rmfamily\fontsize{10.000000}{12.000000}\selectfont \(\displaystyle {0.4}\)}%
\end{pgfscope}%
\begin{pgfscope}%
\pgfsetbuttcap%
\pgfsetroundjoin%
\definecolor{currentfill}{rgb}{0.000000,0.000000,0.000000}%
\pgfsetfillcolor{currentfill}%
\pgfsetlinewidth{0.803000pt}%
\definecolor{currentstroke}{rgb}{0.000000,0.000000,0.000000}%
\pgfsetstrokecolor{currentstroke}%
\pgfsetdash{}{0pt}%
\pgfsys@defobject{currentmarker}{\pgfqpoint{-0.048611in}{0.000000in}}{\pgfqpoint{-0.000000in}{0.000000in}}{%
\pgfpathmoveto{\pgfqpoint{-0.000000in}{0.000000in}}%
\pgfpathlineto{\pgfqpoint{-0.048611in}{0.000000in}}%
\pgfusepath{stroke,fill}%
}%
\begin{pgfscope}%
\pgfsys@transformshift{0.750000in}{2.284545in}%
\pgfsys@useobject{currentmarker}{}%
\end{pgfscope}%
\end{pgfscope}%
\begin{pgfscope}%
\definecolor{textcolor}{rgb}{0.000000,0.000000,0.000000}%
\pgfsetstrokecolor{textcolor}%
\pgfsetfillcolor{textcolor}%
\pgftext[x=0.475308in, y=2.236320in, left, base]{\color{textcolor}\rmfamily\fontsize{10.000000}{12.000000}\selectfont \(\displaystyle {0.6}\)}%
\end{pgfscope}%
\begin{pgfscope}%
\pgfsetbuttcap%
\pgfsetroundjoin%
\definecolor{currentfill}{rgb}{0.000000,0.000000,0.000000}%
\pgfsetfillcolor{currentfill}%
\pgfsetlinewidth{0.803000pt}%
\definecolor{currentstroke}{rgb}{0.000000,0.000000,0.000000}%
\pgfsetstrokecolor{currentstroke}%
\pgfsetdash{}{0pt}%
\pgfsys@defobject{currentmarker}{\pgfqpoint{-0.048611in}{0.000000in}}{\pgfqpoint{-0.000000in}{0.000000in}}{%
\pgfpathmoveto{\pgfqpoint{-0.000000in}{0.000000in}}%
\pgfpathlineto{\pgfqpoint{-0.048611in}{0.000000in}}%
\pgfusepath{stroke,fill}%
}%
\begin{pgfscope}%
\pgfsys@transformshift{0.750000in}{2.833636in}%
\pgfsys@useobject{currentmarker}{}%
\end{pgfscope}%
\end{pgfscope}%
\begin{pgfscope}%
\definecolor{textcolor}{rgb}{0.000000,0.000000,0.000000}%
\pgfsetstrokecolor{textcolor}%
\pgfsetfillcolor{textcolor}%
\pgftext[x=0.475308in, y=2.785411in, left, base]{\color{textcolor}\rmfamily\fontsize{10.000000}{12.000000}\selectfont \(\displaystyle {0.8}\)}%
\end{pgfscope}%
\begin{pgfscope}%
\pgfsetbuttcap%
\pgfsetroundjoin%
\definecolor{currentfill}{rgb}{0.000000,0.000000,0.000000}%
\pgfsetfillcolor{currentfill}%
\pgfsetlinewidth{0.803000pt}%
\definecolor{currentstroke}{rgb}{0.000000,0.000000,0.000000}%
\pgfsetstrokecolor{currentstroke}%
\pgfsetdash{}{0pt}%
\pgfsys@defobject{currentmarker}{\pgfqpoint{-0.048611in}{0.000000in}}{\pgfqpoint{-0.000000in}{0.000000in}}{%
\pgfpathmoveto{\pgfqpoint{-0.000000in}{0.000000in}}%
\pgfpathlineto{\pgfqpoint{-0.048611in}{0.000000in}}%
\pgfusepath{stroke,fill}%
}%
\begin{pgfscope}%
\pgfsys@transformshift{0.750000in}{3.382727in}%
\pgfsys@useobject{currentmarker}{}%
\end{pgfscope}%
\end{pgfscope}%
\begin{pgfscope}%
\definecolor{textcolor}{rgb}{0.000000,0.000000,0.000000}%
\pgfsetstrokecolor{textcolor}%
\pgfsetfillcolor{textcolor}%
\pgftext[x=0.475308in, y=3.334502in, left, base]{\color{textcolor}\rmfamily\fontsize{10.000000}{12.000000}\selectfont \(\displaystyle {1.0}\)}%
\end{pgfscope}%
\begin{pgfscope}%
\definecolor{textcolor}{rgb}{0.000000,0.000000,0.000000}%
\pgfsetstrokecolor{textcolor}%
\pgfsetfillcolor{textcolor}%
\pgftext[x=0.419753in,y=2.010000in,,bottom,rotate=90.000000]{\color{textcolor}\rmfamily\fontsize{10.000000}{12.000000}\selectfont evil}%
\end{pgfscope}%
\begin{pgfscope}%
\pgfpathrectangle{\pgfqpoint{0.750000in}{0.500000in}}{\pgfqpoint{4.650000in}{3.020000in}}%
\pgfusepath{clip}%
\pgfsetrectcap%
\pgfsetroundjoin%
\pgfsetlinewidth{1.505625pt}%
\definecolor{currentstroke}{rgb}{0.121569,0.466667,0.705882}%
\pgfsetstrokecolor{currentstroke}%
\pgfsetdash{}{0pt}%
\pgfpathmoveto{\pgfqpoint{0.961364in}{0.637273in}}%
\pgfpathlineto{\pgfqpoint{1.967352in}{0.637273in}}%
\pgfpathlineto{\pgfqpoint{1.968896in}{3.382727in}}%
\pgfpathlineto{\pgfqpoint{1.981387in}{3.382727in}}%
\pgfpathlineto{\pgfqpoint{1.981590in}{0.637273in}}%
\pgfpathlineto{\pgfqpoint{1.982926in}{3.382727in}}%
\pgfpathlineto{\pgfqpoint{1.985368in}{3.382727in}}%
\pgfpathlineto{\pgfqpoint{1.986237in}{0.637273in}}%
\pgfpathlineto{\pgfqpoint{1.986907in}{3.382727in}}%
\pgfpathlineto{\pgfqpoint{1.986950in}{3.382727in}}%
\pgfpathlineto{\pgfqpoint{1.988494in}{0.637273in}}%
\pgfpathlineto{\pgfqpoint{2.081419in}{0.637273in}}%
\pgfpathlineto{\pgfqpoint{2.082949in}{3.382727in}}%
\pgfpathlineto{\pgfqpoint{2.082959in}{0.637273in}}%
\pgfpathlineto{\pgfqpoint{2.083006in}{0.637273in}}%
\pgfpathlineto{\pgfqpoint{2.083190in}{3.382727in}}%
\pgfpathlineto{\pgfqpoint{2.084546in}{0.637273in}}%
\pgfpathlineto{\pgfqpoint{4.671020in}{0.637273in}}%
\pgfpathlineto{\pgfqpoint{4.671024in}{3.382727in}}%
\pgfpathlineto{\pgfqpoint{4.672559in}{0.637273in}}%
\pgfpathlineto{\pgfqpoint{4.680162in}{0.637273in}}%
\pgfpathlineto{\pgfqpoint{4.681706in}{3.382727in}}%
\pgfpathlineto{\pgfqpoint{4.690423in}{3.382727in}}%
\pgfpathlineto{\pgfqpoint{4.690900in}{0.637273in}}%
\pgfpathlineto{\pgfqpoint{4.691963in}{3.382727in}}%
\pgfpathlineto{\pgfqpoint{4.694617in}{3.382727in}}%
\pgfpathlineto{\pgfqpoint{4.694749in}{0.637273in}}%
\pgfpathlineto{\pgfqpoint{4.696156in}{3.382727in}}%
\pgfpathlineto{\pgfqpoint{4.697710in}{3.382727in}}%
\pgfpathlineto{\pgfqpoint{4.698428in}{0.637273in}}%
\pgfpathlineto{\pgfqpoint{4.699249in}{3.382727in}}%
\pgfpathlineto{\pgfqpoint{4.699774in}{3.382727in}}%
\pgfpathlineto{\pgfqpoint{4.701313in}{0.637273in}}%
\pgfpathlineto{\pgfqpoint{4.701346in}{3.382727in}}%
\pgfpathlineto{\pgfqpoint{4.702853in}{0.637273in}}%
\pgfpathlineto{\pgfqpoint{4.763308in}{0.637273in}}%
\pgfpathlineto{\pgfqpoint{4.764852in}{3.382727in}}%
\pgfpathlineto{\pgfqpoint{4.768450in}{3.382727in}}%
\pgfpathlineto{\pgfqpoint{4.769801in}{0.637273in}}%
\pgfpathlineto{\pgfqpoint{4.769990in}{3.382727in}}%
\pgfpathlineto{\pgfqpoint{4.770060in}{3.382727in}}%
\pgfpathlineto{\pgfqpoint{4.771605in}{0.637273in}}%
\pgfpathlineto{\pgfqpoint{5.188636in}{0.637273in}}%
\pgfpathlineto{\pgfqpoint{5.188636in}{0.637273in}}%
\pgfusepath{stroke}%
\end{pgfscope}%
\begin{pgfscope}%
\pgfpathrectangle{\pgfqpoint{0.750000in}{0.500000in}}{\pgfqpoint{4.650000in}{3.020000in}}%
\pgfusepath{clip}%
\pgfsetrectcap%
\pgfsetroundjoin%
\pgfsetlinewidth{1.505625pt}%
\definecolor{currentstroke}{rgb}{1.000000,0.000000,0.000000}%
\pgfsetstrokecolor{currentstroke}%
\pgfsetdash{}{0pt}%
\pgfpathmoveto{\pgfqpoint{1.013309in}{0.500000in}}%
\pgfpathlineto{\pgfqpoint{1.013309in}{3.520000in}}%
\pgfusepath{stroke}%
\end{pgfscope}%
\begin{pgfscope}%
\pgfpathrectangle{\pgfqpoint{0.750000in}{0.500000in}}{\pgfqpoint{4.650000in}{3.020000in}}%
\pgfusepath{clip}%
\pgfsetrectcap%
\pgfsetroundjoin%
\pgfsetlinewidth{1.505625pt}%
\definecolor{currentstroke}{rgb}{1.000000,0.000000,0.000000}%
\pgfsetstrokecolor{currentstroke}%
\pgfsetdash{}{0pt}%
\pgfpathmoveto{\pgfqpoint{1.967640in}{0.500000in}}%
\pgfpathlineto{\pgfqpoint{1.967640in}{3.520000in}}%
\pgfusepath{stroke}%
\end{pgfscope}%
\begin{pgfscope}%
\pgfpathrectangle{\pgfqpoint{0.750000in}{0.500000in}}{\pgfqpoint{4.650000in}{3.020000in}}%
\pgfusepath{clip}%
\pgfsetrectcap%
\pgfsetroundjoin%
\pgfsetlinewidth{1.505625pt}%
\definecolor{currentstroke}{rgb}{1.000000,0.000000,0.000000}%
\pgfsetstrokecolor{currentstroke}%
\pgfsetdash{}{0pt}%
\pgfpathmoveto{\pgfqpoint{2.081514in}{0.500000in}}%
\pgfpathlineto{\pgfqpoint{2.081514in}{3.520000in}}%
\pgfusepath{stroke}%
\end{pgfscope}%
\begin{pgfscope}%
\pgfpathrectangle{\pgfqpoint{0.750000in}{0.500000in}}{\pgfqpoint{4.650000in}{3.020000in}}%
\pgfusepath{clip}%
\pgfsetrectcap%
\pgfsetroundjoin%
\pgfsetlinewidth{1.505625pt}%
\definecolor{currentstroke}{rgb}{1.000000,0.000000,0.000000}%
\pgfsetstrokecolor{currentstroke}%
\pgfsetdash{}{0pt}%
\pgfpathmoveto{\pgfqpoint{3.253354in}{0.500000in}}%
\pgfpathlineto{\pgfqpoint{3.253354in}{3.520000in}}%
\pgfusepath{stroke}%
\end{pgfscope}%
\begin{pgfscope}%
\pgfpathrectangle{\pgfqpoint{0.750000in}{0.500000in}}{\pgfqpoint{4.650000in}{3.020000in}}%
\pgfusepath{clip}%
\pgfsetrectcap%
\pgfsetroundjoin%
\pgfsetlinewidth{1.505625pt}%
\definecolor{currentstroke}{rgb}{1.000000,0.000000,0.000000}%
\pgfsetstrokecolor{currentstroke}%
\pgfsetdash{}{0pt}%
\pgfpathmoveto{\pgfqpoint{4.680738in}{0.500000in}}%
\pgfpathlineto{\pgfqpoint{4.680738in}{3.520000in}}%
\pgfusepath{stroke}%
\end{pgfscope}%
\begin{pgfscope}%
\pgfpathrectangle{\pgfqpoint{0.750000in}{0.500000in}}{\pgfqpoint{4.650000in}{3.020000in}}%
\pgfusepath{clip}%
\pgfsetrectcap%
\pgfsetroundjoin%
\pgfsetlinewidth{1.505625pt}%
\definecolor{currentstroke}{rgb}{1.000000,0.000000,0.000000}%
\pgfsetstrokecolor{currentstroke}%
\pgfsetdash{}{0pt}%
\pgfpathmoveto{\pgfqpoint{4.817326in}{0.500000in}}%
\pgfpathlineto{\pgfqpoint{4.817326in}{3.520000in}}%
\pgfusepath{stroke}%
\end{pgfscope}%
\begin{pgfscope}%
\pgfsetrectcap%
\pgfsetmiterjoin%
\pgfsetlinewidth{0.803000pt}%
\definecolor{currentstroke}{rgb}{0.000000,0.000000,0.000000}%
\pgfsetstrokecolor{currentstroke}%
\pgfsetdash{}{0pt}%
\pgfpathmoveto{\pgfqpoint{0.750000in}{0.500000in}}%
\pgfpathlineto{\pgfqpoint{0.750000in}{3.520000in}}%
\pgfusepath{stroke}%
\end{pgfscope}%
\begin{pgfscope}%
\pgfsetrectcap%
\pgfsetmiterjoin%
\pgfsetlinewidth{0.803000pt}%
\definecolor{currentstroke}{rgb}{0.000000,0.000000,0.000000}%
\pgfsetstrokecolor{currentstroke}%
\pgfsetdash{}{0pt}%
\pgfpathmoveto{\pgfqpoint{5.400000in}{0.500000in}}%
\pgfpathlineto{\pgfqpoint{5.400000in}{3.520000in}}%
\pgfusepath{stroke}%
\end{pgfscope}%
\begin{pgfscope}%
\pgfsetrectcap%
\pgfsetmiterjoin%
\pgfsetlinewidth{0.803000pt}%
\definecolor{currentstroke}{rgb}{0.000000,0.000000,0.000000}%
\pgfsetstrokecolor{currentstroke}%
\pgfsetdash{}{0pt}%
\pgfpathmoveto{\pgfqpoint{0.750000in}{0.500000in}}%
\pgfpathlineto{\pgfqpoint{5.400000in}{0.500000in}}%
\pgfusepath{stroke}%
\end{pgfscope}%
\begin{pgfscope}%
\pgfsetrectcap%
\pgfsetmiterjoin%
\pgfsetlinewidth{0.803000pt}%
\definecolor{currentstroke}{rgb}{0.000000,0.000000,0.000000}%
\pgfsetstrokecolor{currentstroke}%
\pgfsetdash{}{0pt}%
\pgfpathmoveto{\pgfqpoint{0.750000in}{3.520000in}}%
\pgfpathlineto{\pgfqpoint{5.400000in}{3.520000in}}%
\pgfusepath{stroke}%
\end{pgfscope}%
\end{pgfpicture}%
\makeatother%
\endgroup%

    \caption{BETH Dangerous Outliers [Red Line = Anomaly]}
    \label{fig:beth_evil_outliers}
\end{figure}

\subsection{Summary}