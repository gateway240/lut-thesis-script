\section{Methods}
\label{ref_methods}

To understand and evaluate effective algorithms and techniques for outlier detection, a literature review will be conducted. This review will include both scientific literature, library api documentation, and online resources. The goal of this review will be to understand the cutting edge techniques and the logic behind the decision making of the algorithms. Additionally, the researchers will examine a variety of interdisciplinary methods (ex. statistics, Deep Learning, etc.) to compare ideal use cases in given scenarios to build an anomaly detection pipeline and toolkit.  

The researchers will also survey the existing literature to determine the most popular and common datasets used to benchmark these anomaly detection techniques. Since anomalies are by nature rare, most datasets contain a very small number of them, and it is critical that the anomalies they do contain are a good representative sample. The researchers have conducted preliminary investigations in section \ref{ref_results} that show the most common datasets currently used in literature.  Additionally, through a discussion on improving existing machine learnign techniques in python, the researchers have contacted experts in the field, a Ph.D candidate in machine learning at Imperial College London and an industry cyber-security expert to evaluate the shortcomings in the "standard" datasets for modern applications. 

Data collection will be performed as follows:
\begin{enumerate}
    \item Setup data processing pipelines for a variety of existing anomaly detection techniques.
    \item Create a simulation data set with pre-marked, simple anomalies (ex. sensor failure) 
    \item Graph the data to understand the different type of faults/anomalies to be detected
    \item Setup an algorithm pipeline to run the data through multiple streaming and non-streaming techniques and collect results
    \item Select a real dataset and identify or use pre-identified markers for anomalies
    \item Run the datasets through the algorithm pipeline and collect results
    \item Add additional components (drift detection, etc) to the pipeline and record data results 
\end{enumerate}

Data  analysis will be performed as follows:
\begin{enumerate}
    \item Compare results between different algorithms and techniques (ex. streaming vs. batch, statistical vs. streaming ML) methods for the dataset 
    \item Determine the benefits and shortcomings of each method using standard performance metrics and identify for what type of anomalies certain methods excel
    \item Implement the best methods for the given problem in a real-world control systems problem
\end{enumerate}

\subsection{Measuring Algorithms and Methods}

There will always be phenomena the algorithm cannot detect or behavior that is not anomalous that the algorithm detects accidentally. The goal is to optimize these behaviors for the given task. Therefore, many existing algorithms and implementations are being compared. This comparison will identify which algorithms perform the best for the task and will help identify which techniques are best for which types of data. This will provide insight on the existing shortcomings in the field and will show where the algorithms can be improved.

There are industry standard ways of comparing machine learning techniques. Some of these include ROC curves which allow you to compare false positive and true positive rates. There are also conventional statistical techniques which can be utilized. The researchers will explore further to determine which metrics are most appropriate for the task.

\subsection{Reliability, validity, and sensitivity aspects}
Anomalies will be identified (or pre-identified) in the dataset. Then the experimental results will be examined to see if the algorithm detected the anomalies correctly. When analyzing this data, the researchers will try to determine why certain false or true positives and negatives are occurring and propose techniques to mitigate them.

In the preliminary research considerations the researchers are analyzing and determining appropriate datasets for the experiments. One of the considerations is are the current industry standard datasets sufficient for benchmarking performance for modern algorithms. The researchers suspect that they are not, and thus will utilize a systematic technique to select appropriate and diverse datasets to adequately obtain valid algorithmic results that can be generalized to real world phenomena. 


\subsection{Resources}

The researchers have access to a variety of computational resources from universities and research institutions across Europe. In Finland the group has access to the CSC supercomputer. CSC provided some supercomputer resources and hosted cloud services available for the project. Additionally, the researchers have access to limited computational resources via their personal computers. Additionally, Each university in the consortium has access to additional compute resources that can be utilized by their respective members.

\subsection{Dataset Selection}
\label{ref_datasets}

In this research, three separate data sets will be used to test and evaluate the developed detection technique. The data sets span many fields of engineering and technology and provide a good survey of the generalizeability and applicability of the proposed detection technique.  

\subsubsection{Hydraulic Simulation Dataset}
\label{ref_hydraulic_dataset}

In this study, the Matlab software toolkit Simscape Multibody will be used to design a model of the hydraulic system outlined in figure \ref{fig:boom_structure}. This model will be used to study the behavior and response of the system to optimize behavior and system parameters. For this study, a mass m of 240 KG and a supply pressure of 185 bar is used.

\begin{figure}[H]
    %\centering
    \includegraphics[width=0.8\textwidth]{1_hydraulic_sim/BoomStructure.PNG}
    \caption{Hydraulic Boom Lift Structure}
    \label{fig:boom_structure}
\end{figure}


The control signal shown in figure \ref{fig:hydraulic_cs} is used as input to the simulation of the hydraulic cylinder. This control signal is proportionally negative to the original control signal.

\begin{figure}[H]
    %\centering
    %% Creator: Matplotlib, PGF backend
%%
%% To include the figure in your LaTeX document, write
%%   \input{<filename>.pgf}
%%
%% Make sure the required packages are loaded in your preamble
%%   \usepackage{pgf}
%%
%% Also ensure that all the required font packages are loaded; for instance,
%% the lmodern package is sometimes necessary when using math font.
%%   \usepackage{lmodern}
%%
%% Figures using additional raster images can only be included by \input if
%% they are in the same directory as the main LaTeX file. For loading figures
%% from other directories you can use the `import` package
%%   \usepackage{import}
%%
%% and then include the figures with
%%   \import{<path to file>}{<filename>.pgf}
%%
%% Matplotlib used the following preamble
%%
\begingroup%
\makeatletter%
\begin{pgfpicture}%
\pgfpathrectangle{\pgfpointorigin}{\pgfqpoint{6.000000in}{4.000000in}}%
\pgfusepath{use as bounding box, clip}%
\begin{pgfscope}%
\pgfsetbuttcap%
\pgfsetmiterjoin%
\pgfsetlinewidth{0.000000pt}%
\definecolor{currentstroke}{rgb}{1.000000,1.000000,1.000000}%
\pgfsetstrokecolor{currentstroke}%
\pgfsetstrokeopacity{0.000000}%
\pgfsetdash{}{0pt}%
\pgfpathmoveto{\pgfqpoint{0.000000in}{0.000000in}}%
\pgfpathlineto{\pgfqpoint{6.000000in}{0.000000in}}%
\pgfpathlineto{\pgfqpoint{6.000000in}{4.000000in}}%
\pgfpathlineto{\pgfqpoint{0.000000in}{4.000000in}}%
\pgfpathlineto{\pgfqpoint{0.000000in}{0.000000in}}%
\pgfpathclose%
\pgfusepath{}%
\end{pgfscope}%
\begin{pgfscope}%
\pgfsetbuttcap%
\pgfsetmiterjoin%
\definecolor{currentfill}{rgb}{1.000000,1.000000,1.000000}%
\pgfsetfillcolor{currentfill}%
\pgfsetlinewidth{0.000000pt}%
\definecolor{currentstroke}{rgb}{0.000000,0.000000,0.000000}%
\pgfsetstrokecolor{currentstroke}%
\pgfsetstrokeopacity{0.000000}%
\pgfsetdash{}{0pt}%
\pgfpathmoveto{\pgfqpoint{0.750000in}{0.500000in}}%
\pgfpathlineto{\pgfqpoint{5.400000in}{0.500000in}}%
\pgfpathlineto{\pgfqpoint{5.400000in}{3.520000in}}%
\pgfpathlineto{\pgfqpoint{0.750000in}{3.520000in}}%
\pgfpathlineto{\pgfqpoint{0.750000in}{0.500000in}}%
\pgfpathclose%
\pgfusepath{fill}%
\end{pgfscope}%
\begin{pgfscope}%
\pgfsetbuttcap%
\pgfsetroundjoin%
\definecolor{currentfill}{rgb}{0.000000,0.000000,0.000000}%
\pgfsetfillcolor{currentfill}%
\pgfsetlinewidth{0.803000pt}%
\definecolor{currentstroke}{rgb}{0.000000,0.000000,0.000000}%
\pgfsetstrokecolor{currentstroke}%
\pgfsetdash{}{0pt}%
\pgfsys@defobject{currentmarker}{\pgfqpoint{0.000000in}{-0.048611in}}{\pgfqpoint{0.000000in}{0.000000in}}{%
\pgfpathmoveto{\pgfqpoint{0.000000in}{0.000000in}}%
\pgfpathlineto{\pgfqpoint{0.000000in}{-0.048611in}}%
\pgfusepath{stroke,fill}%
}%
\begin{pgfscope}%
\pgfsys@transformshift{0.961364in}{0.500000in}%
\pgfsys@useobject{currentmarker}{}%
\end{pgfscope}%
\end{pgfscope}%
\begin{pgfscope}%
\definecolor{textcolor}{rgb}{0.000000,0.000000,0.000000}%
\pgfsetstrokecolor{textcolor}%
\pgfsetfillcolor{textcolor}%
\pgftext[x=0.961364in,y=0.402778in,,top]{\color{textcolor}\rmfamily\fontsize{10.000000}{12.000000}\selectfont \(\displaystyle {0.0}\)}%
\end{pgfscope}%
\begin{pgfscope}%
\pgfsetbuttcap%
\pgfsetroundjoin%
\definecolor{currentfill}{rgb}{0.000000,0.000000,0.000000}%
\pgfsetfillcolor{currentfill}%
\pgfsetlinewidth{0.803000pt}%
\definecolor{currentstroke}{rgb}{0.000000,0.000000,0.000000}%
\pgfsetstrokecolor{currentstroke}%
\pgfsetdash{}{0pt}%
\pgfsys@defobject{currentmarker}{\pgfqpoint{0.000000in}{-0.048611in}}{\pgfqpoint{0.000000in}{0.000000in}}{%
\pgfpathmoveto{\pgfqpoint{0.000000in}{0.000000in}}%
\pgfpathlineto{\pgfqpoint{0.000000in}{-0.048611in}}%
\pgfusepath{stroke,fill}%
}%
\begin{pgfscope}%
\pgfsys@transformshift{1.525000in}{0.500000in}%
\pgfsys@useobject{currentmarker}{}%
\end{pgfscope}%
\end{pgfscope}%
\begin{pgfscope}%
\definecolor{textcolor}{rgb}{0.000000,0.000000,0.000000}%
\pgfsetstrokecolor{textcolor}%
\pgfsetfillcolor{textcolor}%
\pgftext[x=1.525000in,y=0.402778in,,top]{\color{textcolor}\rmfamily\fontsize{10.000000}{12.000000}\selectfont \(\displaystyle {0.2}\)}%
\end{pgfscope}%
\begin{pgfscope}%
\pgfsetbuttcap%
\pgfsetroundjoin%
\definecolor{currentfill}{rgb}{0.000000,0.000000,0.000000}%
\pgfsetfillcolor{currentfill}%
\pgfsetlinewidth{0.803000pt}%
\definecolor{currentstroke}{rgb}{0.000000,0.000000,0.000000}%
\pgfsetstrokecolor{currentstroke}%
\pgfsetdash{}{0pt}%
\pgfsys@defobject{currentmarker}{\pgfqpoint{0.000000in}{-0.048611in}}{\pgfqpoint{0.000000in}{0.000000in}}{%
\pgfpathmoveto{\pgfqpoint{0.000000in}{0.000000in}}%
\pgfpathlineto{\pgfqpoint{0.000000in}{-0.048611in}}%
\pgfusepath{stroke,fill}%
}%
\begin{pgfscope}%
\pgfsys@transformshift{2.088636in}{0.500000in}%
\pgfsys@useobject{currentmarker}{}%
\end{pgfscope}%
\end{pgfscope}%
\begin{pgfscope}%
\definecolor{textcolor}{rgb}{0.000000,0.000000,0.000000}%
\pgfsetstrokecolor{textcolor}%
\pgfsetfillcolor{textcolor}%
\pgftext[x=2.088636in,y=0.402778in,,top]{\color{textcolor}\rmfamily\fontsize{10.000000}{12.000000}\selectfont \(\displaystyle {0.4}\)}%
\end{pgfscope}%
\begin{pgfscope}%
\pgfsetbuttcap%
\pgfsetroundjoin%
\definecolor{currentfill}{rgb}{0.000000,0.000000,0.000000}%
\pgfsetfillcolor{currentfill}%
\pgfsetlinewidth{0.803000pt}%
\definecolor{currentstroke}{rgb}{0.000000,0.000000,0.000000}%
\pgfsetstrokecolor{currentstroke}%
\pgfsetdash{}{0pt}%
\pgfsys@defobject{currentmarker}{\pgfqpoint{0.000000in}{-0.048611in}}{\pgfqpoint{0.000000in}{0.000000in}}{%
\pgfpathmoveto{\pgfqpoint{0.000000in}{0.000000in}}%
\pgfpathlineto{\pgfqpoint{0.000000in}{-0.048611in}}%
\pgfusepath{stroke,fill}%
}%
\begin{pgfscope}%
\pgfsys@transformshift{2.652273in}{0.500000in}%
\pgfsys@useobject{currentmarker}{}%
\end{pgfscope}%
\end{pgfscope}%
\begin{pgfscope}%
\definecolor{textcolor}{rgb}{0.000000,0.000000,0.000000}%
\pgfsetstrokecolor{textcolor}%
\pgfsetfillcolor{textcolor}%
\pgftext[x=2.652273in,y=0.402778in,,top]{\color{textcolor}\rmfamily\fontsize{10.000000}{12.000000}\selectfont \(\displaystyle {0.6}\)}%
\end{pgfscope}%
\begin{pgfscope}%
\pgfsetbuttcap%
\pgfsetroundjoin%
\definecolor{currentfill}{rgb}{0.000000,0.000000,0.000000}%
\pgfsetfillcolor{currentfill}%
\pgfsetlinewidth{0.803000pt}%
\definecolor{currentstroke}{rgb}{0.000000,0.000000,0.000000}%
\pgfsetstrokecolor{currentstroke}%
\pgfsetdash{}{0pt}%
\pgfsys@defobject{currentmarker}{\pgfqpoint{0.000000in}{-0.048611in}}{\pgfqpoint{0.000000in}{0.000000in}}{%
\pgfpathmoveto{\pgfqpoint{0.000000in}{0.000000in}}%
\pgfpathlineto{\pgfqpoint{0.000000in}{-0.048611in}}%
\pgfusepath{stroke,fill}%
}%
\begin{pgfscope}%
\pgfsys@transformshift{3.215909in}{0.500000in}%
\pgfsys@useobject{currentmarker}{}%
\end{pgfscope}%
\end{pgfscope}%
\begin{pgfscope}%
\definecolor{textcolor}{rgb}{0.000000,0.000000,0.000000}%
\pgfsetstrokecolor{textcolor}%
\pgfsetfillcolor{textcolor}%
\pgftext[x=3.215909in,y=0.402778in,,top]{\color{textcolor}\rmfamily\fontsize{10.000000}{12.000000}\selectfont \(\displaystyle {0.8}\)}%
\end{pgfscope}%
\begin{pgfscope}%
\pgfsetbuttcap%
\pgfsetroundjoin%
\definecolor{currentfill}{rgb}{0.000000,0.000000,0.000000}%
\pgfsetfillcolor{currentfill}%
\pgfsetlinewidth{0.803000pt}%
\definecolor{currentstroke}{rgb}{0.000000,0.000000,0.000000}%
\pgfsetstrokecolor{currentstroke}%
\pgfsetdash{}{0pt}%
\pgfsys@defobject{currentmarker}{\pgfqpoint{0.000000in}{-0.048611in}}{\pgfqpoint{0.000000in}{0.000000in}}{%
\pgfpathmoveto{\pgfqpoint{0.000000in}{0.000000in}}%
\pgfpathlineto{\pgfqpoint{0.000000in}{-0.048611in}}%
\pgfusepath{stroke,fill}%
}%
\begin{pgfscope}%
\pgfsys@transformshift{3.779545in}{0.500000in}%
\pgfsys@useobject{currentmarker}{}%
\end{pgfscope}%
\end{pgfscope}%
\begin{pgfscope}%
\definecolor{textcolor}{rgb}{0.000000,0.000000,0.000000}%
\pgfsetstrokecolor{textcolor}%
\pgfsetfillcolor{textcolor}%
\pgftext[x=3.779545in,y=0.402778in,,top]{\color{textcolor}\rmfamily\fontsize{10.000000}{12.000000}\selectfont \(\displaystyle {1.0}\)}%
\end{pgfscope}%
\begin{pgfscope}%
\pgfsetbuttcap%
\pgfsetroundjoin%
\definecolor{currentfill}{rgb}{0.000000,0.000000,0.000000}%
\pgfsetfillcolor{currentfill}%
\pgfsetlinewidth{0.803000pt}%
\definecolor{currentstroke}{rgb}{0.000000,0.000000,0.000000}%
\pgfsetstrokecolor{currentstroke}%
\pgfsetdash{}{0pt}%
\pgfsys@defobject{currentmarker}{\pgfqpoint{0.000000in}{-0.048611in}}{\pgfqpoint{0.000000in}{0.000000in}}{%
\pgfpathmoveto{\pgfqpoint{0.000000in}{0.000000in}}%
\pgfpathlineto{\pgfqpoint{0.000000in}{-0.048611in}}%
\pgfusepath{stroke,fill}%
}%
\begin{pgfscope}%
\pgfsys@transformshift{4.343182in}{0.500000in}%
\pgfsys@useobject{currentmarker}{}%
\end{pgfscope}%
\end{pgfscope}%
\begin{pgfscope}%
\definecolor{textcolor}{rgb}{0.000000,0.000000,0.000000}%
\pgfsetstrokecolor{textcolor}%
\pgfsetfillcolor{textcolor}%
\pgftext[x=4.343182in,y=0.402778in,,top]{\color{textcolor}\rmfamily\fontsize{10.000000}{12.000000}\selectfont \(\displaystyle {1.2}\)}%
\end{pgfscope}%
\begin{pgfscope}%
\pgfsetbuttcap%
\pgfsetroundjoin%
\definecolor{currentfill}{rgb}{0.000000,0.000000,0.000000}%
\pgfsetfillcolor{currentfill}%
\pgfsetlinewidth{0.803000pt}%
\definecolor{currentstroke}{rgb}{0.000000,0.000000,0.000000}%
\pgfsetstrokecolor{currentstroke}%
\pgfsetdash{}{0pt}%
\pgfsys@defobject{currentmarker}{\pgfqpoint{0.000000in}{-0.048611in}}{\pgfqpoint{0.000000in}{0.000000in}}{%
\pgfpathmoveto{\pgfqpoint{0.000000in}{0.000000in}}%
\pgfpathlineto{\pgfqpoint{0.000000in}{-0.048611in}}%
\pgfusepath{stroke,fill}%
}%
\begin{pgfscope}%
\pgfsys@transformshift{4.906818in}{0.500000in}%
\pgfsys@useobject{currentmarker}{}%
\end{pgfscope}%
\end{pgfscope}%
\begin{pgfscope}%
\definecolor{textcolor}{rgb}{0.000000,0.000000,0.000000}%
\pgfsetstrokecolor{textcolor}%
\pgfsetfillcolor{textcolor}%
\pgftext[x=4.906818in,y=0.402778in,,top]{\color{textcolor}\rmfamily\fontsize{10.000000}{12.000000}\selectfont \(\displaystyle {1.4}\)}%
\end{pgfscope}%
\begin{pgfscope}%
\definecolor{textcolor}{rgb}{0.000000,0.000000,0.000000}%
\pgfsetstrokecolor{textcolor}%
\pgfsetfillcolor{textcolor}%
\pgftext[x=3.075000in,y=0.223766in,,top]{\color{textcolor}\rmfamily\fontsize{10.000000}{12.000000}\selectfont Time (s)}%
\end{pgfscope}%
\begin{pgfscope}%
\pgfsetbuttcap%
\pgfsetroundjoin%
\definecolor{currentfill}{rgb}{0.000000,0.000000,0.000000}%
\pgfsetfillcolor{currentfill}%
\pgfsetlinewidth{0.803000pt}%
\definecolor{currentstroke}{rgb}{0.000000,0.000000,0.000000}%
\pgfsetstrokecolor{currentstroke}%
\pgfsetdash{}{0pt}%
\pgfsys@defobject{currentmarker}{\pgfqpoint{-0.048611in}{0.000000in}}{\pgfqpoint{-0.000000in}{0.000000in}}{%
\pgfpathmoveto{\pgfqpoint{-0.000000in}{0.000000in}}%
\pgfpathlineto{\pgfqpoint{-0.048611in}{0.000000in}}%
\pgfusepath{stroke,fill}%
}%
\begin{pgfscope}%
\pgfsys@transformshift{0.750000in}{0.637273in}%
\pgfsys@useobject{currentmarker}{}%
\end{pgfscope}%
\end{pgfscope}%
\begin{pgfscope}%
\definecolor{textcolor}{rgb}{0.000000,0.000000,0.000000}%
\pgfsetstrokecolor{textcolor}%
\pgfsetfillcolor{textcolor}%
\pgftext[x=0.297838in, y=0.589047in, left, base]{\color{textcolor}\rmfamily\fontsize{10.000000}{12.000000}\selectfont \(\displaystyle {\ensuremath{-}10.0}\)}%
\end{pgfscope}%
\begin{pgfscope}%
\pgfsetbuttcap%
\pgfsetroundjoin%
\definecolor{currentfill}{rgb}{0.000000,0.000000,0.000000}%
\pgfsetfillcolor{currentfill}%
\pgfsetlinewidth{0.803000pt}%
\definecolor{currentstroke}{rgb}{0.000000,0.000000,0.000000}%
\pgfsetstrokecolor{currentstroke}%
\pgfsetdash{}{0pt}%
\pgfsys@defobject{currentmarker}{\pgfqpoint{-0.048611in}{0.000000in}}{\pgfqpoint{-0.000000in}{0.000000in}}{%
\pgfpathmoveto{\pgfqpoint{-0.000000in}{0.000000in}}%
\pgfpathlineto{\pgfqpoint{-0.048611in}{0.000000in}}%
\pgfusepath{stroke,fill}%
}%
\begin{pgfscope}%
\pgfsys@transformshift{0.750000in}{0.980455in}%
\pgfsys@useobject{currentmarker}{}%
\end{pgfscope}%
\end{pgfscope}%
\begin{pgfscope}%
\definecolor{textcolor}{rgb}{0.000000,0.000000,0.000000}%
\pgfsetstrokecolor{textcolor}%
\pgfsetfillcolor{textcolor}%
\pgftext[x=0.367283in, y=0.932229in, left, base]{\color{textcolor}\rmfamily\fontsize{10.000000}{12.000000}\selectfont \(\displaystyle {\ensuremath{-}7.5}\)}%
\end{pgfscope}%
\begin{pgfscope}%
\pgfsetbuttcap%
\pgfsetroundjoin%
\definecolor{currentfill}{rgb}{0.000000,0.000000,0.000000}%
\pgfsetfillcolor{currentfill}%
\pgfsetlinewidth{0.803000pt}%
\definecolor{currentstroke}{rgb}{0.000000,0.000000,0.000000}%
\pgfsetstrokecolor{currentstroke}%
\pgfsetdash{}{0pt}%
\pgfsys@defobject{currentmarker}{\pgfqpoint{-0.048611in}{0.000000in}}{\pgfqpoint{-0.000000in}{0.000000in}}{%
\pgfpathmoveto{\pgfqpoint{-0.000000in}{0.000000in}}%
\pgfpathlineto{\pgfqpoint{-0.048611in}{0.000000in}}%
\pgfusepath{stroke,fill}%
}%
\begin{pgfscope}%
\pgfsys@transformshift{0.750000in}{1.323636in}%
\pgfsys@useobject{currentmarker}{}%
\end{pgfscope}%
\end{pgfscope}%
\begin{pgfscope}%
\definecolor{textcolor}{rgb}{0.000000,0.000000,0.000000}%
\pgfsetstrokecolor{textcolor}%
\pgfsetfillcolor{textcolor}%
\pgftext[x=0.367283in, y=1.275411in, left, base]{\color{textcolor}\rmfamily\fontsize{10.000000}{12.000000}\selectfont \(\displaystyle {\ensuremath{-}5.0}\)}%
\end{pgfscope}%
\begin{pgfscope}%
\pgfsetbuttcap%
\pgfsetroundjoin%
\definecolor{currentfill}{rgb}{0.000000,0.000000,0.000000}%
\pgfsetfillcolor{currentfill}%
\pgfsetlinewidth{0.803000pt}%
\definecolor{currentstroke}{rgb}{0.000000,0.000000,0.000000}%
\pgfsetstrokecolor{currentstroke}%
\pgfsetdash{}{0pt}%
\pgfsys@defobject{currentmarker}{\pgfqpoint{-0.048611in}{0.000000in}}{\pgfqpoint{-0.000000in}{0.000000in}}{%
\pgfpathmoveto{\pgfqpoint{-0.000000in}{0.000000in}}%
\pgfpathlineto{\pgfqpoint{-0.048611in}{0.000000in}}%
\pgfusepath{stroke,fill}%
}%
\begin{pgfscope}%
\pgfsys@transformshift{0.750000in}{1.666818in}%
\pgfsys@useobject{currentmarker}{}%
\end{pgfscope}%
\end{pgfscope}%
\begin{pgfscope}%
\definecolor{textcolor}{rgb}{0.000000,0.000000,0.000000}%
\pgfsetstrokecolor{textcolor}%
\pgfsetfillcolor{textcolor}%
\pgftext[x=0.367283in, y=1.618593in, left, base]{\color{textcolor}\rmfamily\fontsize{10.000000}{12.000000}\selectfont \(\displaystyle {\ensuremath{-}2.5}\)}%
\end{pgfscope}%
\begin{pgfscope}%
\pgfsetbuttcap%
\pgfsetroundjoin%
\definecolor{currentfill}{rgb}{0.000000,0.000000,0.000000}%
\pgfsetfillcolor{currentfill}%
\pgfsetlinewidth{0.803000pt}%
\definecolor{currentstroke}{rgb}{0.000000,0.000000,0.000000}%
\pgfsetstrokecolor{currentstroke}%
\pgfsetdash{}{0pt}%
\pgfsys@defobject{currentmarker}{\pgfqpoint{-0.048611in}{0.000000in}}{\pgfqpoint{-0.000000in}{0.000000in}}{%
\pgfpathmoveto{\pgfqpoint{-0.000000in}{0.000000in}}%
\pgfpathlineto{\pgfqpoint{-0.048611in}{0.000000in}}%
\pgfusepath{stroke,fill}%
}%
\begin{pgfscope}%
\pgfsys@transformshift{0.750000in}{2.010000in}%
\pgfsys@useobject{currentmarker}{}%
\end{pgfscope}%
\end{pgfscope}%
\begin{pgfscope}%
\definecolor{textcolor}{rgb}{0.000000,0.000000,0.000000}%
\pgfsetstrokecolor{textcolor}%
\pgfsetfillcolor{textcolor}%
\pgftext[x=0.475308in, y=1.961775in, left, base]{\color{textcolor}\rmfamily\fontsize{10.000000}{12.000000}\selectfont \(\displaystyle {0.0}\)}%
\end{pgfscope}%
\begin{pgfscope}%
\pgfsetbuttcap%
\pgfsetroundjoin%
\definecolor{currentfill}{rgb}{0.000000,0.000000,0.000000}%
\pgfsetfillcolor{currentfill}%
\pgfsetlinewidth{0.803000pt}%
\definecolor{currentstroke}{rgb}{0.000000,0.000000,0.000000}%
\pgfsetstrokecolor{currentstroke}%
\pgfsetdash{}{0pt}%
\pgfsys@defobject{currentmarker}{\pgfqpoint{-0.048611in}{0.000000in}}{\pgfqpoint{-0.000000in}{0.000000in}}{%
\pgfpathmoveto{\pgfqpoint{-0.000000in}{0.000000in}}%
\pgfpathlineto{\pgfqpoint{-0.048611in}{0.000000in}}%
\pgfusepath{stroke,fill}%
}%
\begin{pgfscope}%
\pgfsys@transformshift{0.750000in}{2.353182in}%
\pgfsys@useobject{currentmarker}{}%
\end{pgfscope}%
\end{pgfscope}%
\begin{pgfscope}%
\definecolor{textcolor}{rgb}{0.000000,0.000000,0.000000}%
\pgfsetstrokecolor{textcolor}%
\pgfsetfillcolor{textcolor}%
\pgftext[x=0.475308in, y=2.304957in, left, base]{\color{textcolor}\rmfamily\fontsize{10.000000}{12.000000}\selectfont \(\displaystyle {2.5}\)}%
\end{pgfscope}%
\begin{pgfscope}%
\pgfsetbuttcap%
\pgfsetroundjoin%
\definecolor{currentfill}{rgb}{0.000000,0.000000,0.000000}%
\pgfsetfillcolor{currentfill}%
\pgfsetlinewidth{0.803000pt}%
\definecolor{currentstroke}{rgb}{0.000000,0.000000,0.000000}%
\pgfsetstrokecolor{currentstroke}%
\pgfsetdash{}{0pt}%
\pgfsys@defobject{currentmarker}{\pgfqpoint{-0.048611in}{0.000000in}}{\pgfqpoint{-0.000000in}{0.000000in}}{%
\pgfpathmoveto{\pgfqpoint{-0.000000in}{0.000000in}}%
\pgfpathlineto{\pgfqpoint{-0.048611in}{0.000000in}}%
\pgfusepath{stroke,fill}%
}%
\begin{pgfscope}%
\pgfsys@transformshift{0.750000in}{2.696364in}%
\pgfsys@useobject{currentmarker}{}%
\end{pgfscope}%
\end{pgfscope}%
\begin{pgfscope}%
\definecolor{textcolor}{rgb}{0.000000,0.000000,0.000000}%
\pgfsetstrokecolor{textcolor}%
\pgfsetfillcolor{textcolor}%
\pgftext[x=0.475308in, y=2.648138in, left, base]{\color{textcolor}\rmfamily\fontsize{10.000000}{12.000000}\selectfont \(\displaystyle {5.0}\)}%
\end{pgfscope}%
\begin{pgfscope}%
\pgfsetbuttcap%
\pgfsetroundjoin%
\definecolor{currentfill}{rgb}{0.000000,0.000000,0.000000}%
\pgfsetfillcolor{currentfill}%
\pgfsetlinewidth{0.803000pt}%
\definecolor{currentstroke}{rgb}{0.000000,0.000000,0.000000}%
\pgfsetstrokecolor{currentstroke}%
\pgfsetdash{}{0pt}%
\pgfsys@defobject{currentmarker}{\pgfqpoint{-0.048611in}{0.000000in}}{\pgfqpoint{-0.000000in}{0.000000in}}{%
\pgfpathmoveto{\pgfqpoint{-0.000000in}{0.000000in}}%
\pgfpathlineto{\pgfqpoint{-0.048611in}{0.000000in}}%
\pgfusepath{stroke,fill}%
}%
\begin{pgfscope}%
\pgfsys@transformshift{0.750000in}{3.039545in}%
\pgfsys@useobject{currentmarker}{}%
\end{pgfscope}%
\end{pgfscope}%
\begin{pgfscope}%
\definecolor{textcolor}{rgb}{0.000000,0.000000,0.000000}%
\pgfsetstrokecolor{textcolor}%
\pgfsetfillcolor{textcolor}%
\pgftext[x=0.475308in, y=2.991320in, left, base]{\color{textcolor}\rmfamily\fontsize{10.000000}{12.000000}\selectfont \(\displaystyle {7.5}\)}%
\end{pgfscope}%
\begin{pgfscope}%
\pgfsetbuttcap%
\pgfsetroundjoin%
\definecolor{currentfill}{rgb}{0.000000,0.000000,0.000000}%
\pgfsetfillcolor{currentfill}%
\pgfsetlinewidth{0.803000pt}%
\definecolor{currentstroke}{rgb}{0.000000,0.000000,0.000000}%
\pgfsetstrokecolor{currentstroke}%
\pgfsetdash{}{0pt}%
\pgfsys@defobject{currentmarker}{\pgfqpoint{-0.048611in}{0.000000in}}{\pgfqpoint{-0.000000in}{0.000000in}}{%
\pgfpathmoveto{\pgfqpoint{-0.000000in}{0.000000in}}%
\pgfpathlineto{\pgfqpoint{-0.048611in}{0.000000in}}%
\pgfusepath{stroke,fill}%
}%
\begin{pgfscope}%
\pgfsys@transformshift{0.750000in}{3.382727in}%
\pgfsys@useobject{currentmarker}{}%
\end{pgfscope}%
\end{pgfscope}%
\begin{pgfscope}%
\definecolor{textcolor}{rgb}{0.000000,0.000000,0.000000}%
\pgfsetstrokecolor{textcolor}%
\pgfsetfillcolor{textcolor}%
\pgftext[x=0.405863in, y=3.334502in, left, base]{\color{textcolor}\rmfamily\fontsize{10.000000}{12.000000}\selectfont \(\displaystyle {10.0}\)}%
\end{pgfscope}%
\begin{pgfscope}%
\definecolor{textcolor}{rgb}{0.000000,0.000000,0.000000}%
\pgfsetstrokecolor{textcolor}%
\pgfsetfillcolor{textcolor}%
\pgftext[x=0.242283in,y=2.010000in,,bottom,rotate=90.000000]{\color{textcolor}\rmfamily\fontsize{10.000000}{12.000000}\selectfont Voltage (\(\displaystyle V\))}%
\end{pgfscope}%
\begin{pgfscope}%
\pgfpathrectangle{\pgfqpoint{0.750000in}{0.500000in}}{\pgfqpoint{4.650000in}{3.020000in}}%
\pgfusepath{clip}%
\pgfsetrectcap%
\pgfsetroundjoin%
\pgfsetlinewidth{1.505625pt}%
\definecolor{currentstroke}{rgb}{0.121569,0.466667,0.705882}%
\pgfsetstrokecolor{currentstroke}%
\pgfsetdash{}{0pt}%
\pgfpathmoveto{\pgfqpoint{0.961364in}{2.010000in}}%
\pgfpathlineto{\pgfqpoint{1.525156in}{2.011521in}}%
\pgfpathlineto{\pgfqpoint{1.665909in}{3.382727in}}%
\pgfpathlineto{\pgfqpoint{1.665909in}{3.382727in}}%
\pgfpathlineto{\pgfqpoint{2.652273in}{3.382727in}}%
\pgfpathlineto{\pgfqpoint{2.652273in}{3.382727in}}%
\pgfpathlineto{\pgfqpoint{2.934091in}{0.637273in}}%
\pgfpathlineto{\pgfqpoint{2.934091in}{0.637273in}}%
\pgfpathlineto{\pgfqpoint{3.920455in}{0.637273in}}%
\pgfpathlineto{\pgfqpoint{3.920455in}{0.637273in}}%
\pgfpathlineto{\pgfqpoint{4.062434in}{2.010001in}}%
\pgfpathlineto{\pgfqpoint{4.062434in}{2.010001in}}%
\pgfpathlineto{\pgfqpoint{5.188636in}{2.011445in}}%
\pgfpathlineto{\pgfqpoint{5.188636in}{2.011445in}}%
\pgfusepath{stroke}%
\end{pgfscope}%
\begin{pgfscope}%
\pgfsetrectcap%
\pgfsetmiterjoin%
\pgfsetlinewidth{0.803000pt}%
\definecolor{currentstroke}{rgb}{0.000000,0.000000,0.000000}%
\pgfsetstrokecolor{currentstroke}%
\pgfsetdash{}{0pt}%
\pgfpathmoveto{\pgfqpoint{0.750000in}{0.500000in}}%
\pgfpathlineto{\pgfqpoint{0.750000in}{3.520000in}}%
\pgfusepath{stroke}%
\end{pgfscope}%
\begin{pgfscope}%
\pgfsetrectcap%
\pgfsetmiterjoin%
\pgfsetlinewidth{0.803000pt}%
\definecolor{currentstroke}{rgb}{0.000000,0.000000,0.000000}%
\pgfsetstrokecolor{currentstroke}%
\pgfsetdash{}{0pt}%
\pgfpathmoveto{\pgfqpoint{5.400000in}{0.500000in}}%
\pgfpathlineto{\pgfqpoint{5.400000in}{3.520000in}}%
\pgfusepath{stroke}%
\end{pgfscope}%
\begin{pgfscope}%
\pgfsetrectcap%
\pgfsetmiterjoin%
\pgfsetlinewidth{0.803000pt}%
\definecolor{currentstroke}{rgb}{0.000000,0.000000,0.000000}%
\pgfsetstrokecolor{currentstroke}%
\pgfsetdash{}{0pt}%
\pgfpathmoveto{\pgfqpoint{0.750000in}{0.500000in}}%
\pgfpathlineto{\pgfqpoint{5.400000in}{0.500000in}}%
\pgfusepath{stroke}%
\end{pgfscope}%
\begin{pgfscope}%
\pgfsetrectcap%
\pgfsetmiterjoin%
\pgfsetlinewidth{0.803000pt}%
\definecolor{currentstroke}{rgb}{0.000000,0.000000,0.000000}%
\pgfsetstrokecolor{currentstroke}%
\pgfsetdash{}{0pt}%
\pgfpathmoveto{\pgfqpoint{0.750000in}{3.520000in}}%
\pgfpathlineto{\pgfqpoint{5.400000in}{3.520000in}}%
\pgfusepath{stroke}%
\end{pgfscope}%
\end{pgfpicture}%
\makeatother%
\endgroup%

    \caption{Hydraulic System Control Signal}
    \label{fig:hydraulic_cs}
\end{figure}

Figure \ref{fig:hydraulic_pos} shows the pressure for the system over time. Introducing a proportionally negative signal does not cause the end effector to return to its initial position. As the control signal is varied inversely, the position of the end effector only returns halfway to its initial position. This indicates the system exhibits a non-linear relationship between the control signal and the end effector position.    The system experiences mild oscillation after coming to rest when the control signal is back to zero at the end of the simulation. To achieve predictable system response and reduce oscillation, a more complex control methodology would be required.

\begin{figure}[H]
    %\centering
    %% Creator: Matplotlib, PGF backend
%%
%% To include the figure in your LaTeX document, write
%%   \input{<filename>.pgf}
%%
%% Make sure the required packages are loaded in your preamble
%%   \usepackage{pgf}
%%
%% Also ensure that all the required font packages are loaded; for instance,
%% the lmodern package is sometimes necessary when using math font.
%%   \usepackage{lmodern}
%%
%% Figures using additional raster images can only be included by \input if
%% they are in the same directory as the main LaTeX file. For loading figures
%% from other directories you can use the `import` package
%%   \usepackage{import}
%%
%% and then include the figures with
%%   \import{<path to file>}{<filename>.pgf}
%%
%% Matplotlib used the following preamble
%%
\begingroup%
\makeatletter%
\begin{pgfpicture}%
\pgfpathrectangle{\pgfpointorigin}{\pgfqpoint{6.000000in}{4.000000in}}%
\pgfusepath{use as bounding box, clip}%
\begin{pgfscope}%
\pgfsetbuttcap%
\pgfsetmiterjoin%
\pgfsetlinewidth{0.000000pt}%
\definecolor{currentstroke}{rgb}{1.000000,1.000000,1.000000}%
\pgfsetstrokecolor{currentstroke}%
\pgfsetstrokeopacity{0.000000}%
\pgfsetdash{}{0pt}%
\pgfpathmoveto{\pgfqpoint{0.000000in}{0.000000in}}%
\pgfpathlineto{\pgfqpoint{6.000000in}{0.000000in}}%
\pgfpathlineto{\pgfqpoint{6.000000in}{4.000000in}}%
\pgfpathlineto{\pgfqpoint{0.000000in}{4.000000in}}%
\pgfpathlineto{\pgfqpoint{0.000000in}{0.000000in}}%
\pgfpathclose%
\pgfusepath{}%
\end{pgfscope}%
\begin{pgfscope}%
\pgfsetbuttcap%
\pgfsetmiterjoin%
\definecolor{currentfill}{rgb}{1.000000,1.000000,1.000000}%
\pgfsetfillcolor{currentfill}%
\pgfsetlinewidth{0.000000pt}%
\definecolor{currentstroke}{rgb}{0.000000,0.000000,0.000000}%
\pgfsetstrokecolor{currentstroke}%
\pgfsetstrokeopacity{0.000000}%
\pgfsetdash{}{0pt}%
\pgfpathmoveto{\pgfqpoint{0.750000in}{0.500000in}}%
\pgfpathlineto{\pgfqpoint{5.400000in}{0.500000in}}%
\pgfpathlineto{\pgfqpoint{5.400000in}{3.520000in}}%
\pgfpathlineto{\pgfqpoint{0.750000in}{3.520000in}}%
\pgfpathlineto{\pgfqpoint{0.750000in}{0.500000in}}%
\pgfpathclose%
\pgfusepath{fill}%
\end{pgfscope}%
\begin{pgfscope}%
\pgfsetbuttcap%
\pgfsetroundjoin%
\definecolor{currentfill}{rgb}{0.000000,0.000000,0.000000}%
\pgfsetfillcolor{currentfill}%
\pgfsetlinewidth{0.803000pt}%
\definecolor{currentstroke}{rgb}{0.000000,0.000000,0.000000}%
\pgfsetstrokecolor{currentstroke}%
\pgfsetdash{}{0pt}%
\pgfsys@defobject{currentmarker}{\pgfqpoint{0.000000in}{-0.048611in}}{\pgfqpoint{0.000000in}{0.000000in}}{%
\pgfpathmoveto{\pgfqpoint{0.000000in}{0.000000in}}%
\pgfpathlineto{\pgfqpoint{0.000000in}{-0.048611in}}%
\pgfusepath{stroke,fill}%
}%
\begin{pgfscope}%
\pgfsys@transformshift{1.195592in}{0.500000in}%
\pgfsys@useobject{currentmarker}{}%
\end{pgfscope}%
\end{pgfscope}%
\begin{pgfscope}%
\definecolor{textcolor}{rgb}{0.000000,0.000000,0.000000}%
\pgfsetstrokecolor{textcolor}%
\pgfsetfillcolor{textcolor}%
\pgftext[x=1.195592in,y=0.402778in,,top]{\color{textcolor}\rmfamily\fontsize{10.000000}{12.000000}\selectfont \(\displaystyle {500}\)}%
\end{pgfscope}%
\begin{pgfscope}%
\pgfsetbuttcap%
\pgfsetroundjoin%
\definecolor{currentfill}{rgb}{0.000000,0.000000,0.000000}%
\pgfsetfillcolor{currentfill}%
\pgfsetlinewidth{0.803000pt}%
\definecolor{currentstroke}{rgb}{0.000000,0.000000,0.000000}%
\pgfsetstrokecolor{currentstroke}%
\pgfsetdash{}{0pt}%
\pgfsys@defobject{currentmarker}{\pgfqpoint{0.000000in}{-0.048611in}}{\pgfqpoint{0.000000in}{0.000000in}}{%
\pgfpathmoveto{\pgfqpoint{0.000000in}{0.000000in}}%
\pgfpathlineto{\pgfqpoint{0.000000in}{-0.048611in}}%
\pgfusepath{stroke,fill}%
}%
\begin{pgfscope}%
\pgfsys@transformshift{1.753280in}{0.500000in}%
\pgfsys@useobject{currentmarker}{}%
\end{pgfscope}%
\end{pgfscope}%
\begin{pgfscope}%
\definecolor{textcolor}{rgb}{0.000000,0.000000,0.000000}%
\pgfsetstrokecolor{textcolor}%
\pgfsetfillcolor{textcolor}%
\pgftext[x=1.753280in,y=0.402778in,,top]{\color{textcolor}\rmfamily\fontsize{10.000000}{12.000000}\selectfont \(\displaystyle {550}\)}%
\end{pgfscope}%
\begin{pgfscope}%
\pgfsetbuttcap%
\pgfsetroundjoin%
\definecolor{currentfill}{rgb}{0.000000,0.000000,0.000000}%
\pgfsetfillcolor{currentfill}%
\pgfsetlinewidth{0.803000pt}%
\definecolor{currentstroke}{rgb}{0.000000,0.000000,0.000000}%
\pgfsetstrokecolor{currentstroke}%
\pgfsetdash{}{0pt}%
\pgfsys@defobject{currentmarker}{\pgfqpoint{0.000000in}{-0.048611in}}{\pgfqpoint{0.000000in}{0.000000in}}{%
\pgfpathmoveto{\pgfqpoint{0.000000in}{0.000000in}}%
\pgfpathlineto{\pgfqpoint{0.000000in}{-0.048611in}}%
\pgfusepath{stroke,fill}%
}%
\begin{pgfscope}%
\pgfsys@transformshift{2.310968in}{0.500000in}%
\pgfsys@useobject{currentmarker}{}%
\end{pgfscope}%
\end{pgfscope}%
\begin{pgfscope}%
\definecolor{textcolor}{rgb}{0.000000,0.000000,0.000000}%
\pgfsetstrokecolor{textcolor}%
\pgfsetfillcolor{textcolor}%
\pgftext[x=2.310968in,y=0.402778in,,top]{\color{textcolor}\rmfamily\fontsize{10.000000}{12.000000}\selectfont \(\displaystyle {600}\)}%
\end{pgfscope}%
\begin{pgfscope}%
\pgfsetbuttcap%
\pgfsetroundjoin%
\definecolor{currentfill}{rgb}{0.000000,0.000000,0.000000}%
\pgfsetfillcolor{currentfill}%
\pgfsetlinewidth{0.803000pt}%
\definecolor{currentstroke}{rgb}{0.000000,0.000000,0.000000}%
\pgfsetstrokecolor{currentstroke}%
\pgfsetdash{}{0pt}%
\pgfsys@defobject{currentmarker}{\pgfqpoint{0.000000in}{-0.048611in}}{\pgfqpoint{0.000000in}{0.000000in}}{%
\pgfpathmoveto{\pgfqpoint{0.000000in}{0.000000in}}%
\pgfpathlineto{\pgfqpoint{0.000000in}{-0.048611in}}%
\pgfusepath{stroke,fill}%
}%
\begin{pgfscope}%
\pgfsys@transformshift{2.868656in}{0.500000in}%
\pgfsys@useobject{currentmarker}{}%
\end{pgfscope}%
\end{pgfscope}%
\begin{pgfscope}%
\definecolor{textcolor}{rgb}{0.000000,0.000000,0.000000}%
\pgfsetstrokecolor{textcolor}%
\pgfsetfillcolor{textcolor}%
\pgftext[x=2.868656in,y=0.402778in,,top]{\color{textcolor}\rmfamily\fontsize{10.000000}{12.000000}\selectfont \(\displaystyle {650}\)}%
\end{pgfscope}%
\begin{pgfscope}%
\pgfsetbuttcap%
\pgfsetroundjoin%
\definecolor{currentfill}{rgb}{0.000000,0.000000,0.000000}%
\pgfsetfillcolor{currentfill}%
\pgfsetlinewidth{0.803000pt}%
\definecolor{currentstroke}{rgb}{0.000000,0.000000,0.000000}%
\pgfsetstrokecolor{currentstroke}%
\pgfsetdash{}{0pt}%
\pgfsys@defobject{currentmarker}{\pgfqpoint{0.000000in}{-0.048611in}}{\pgfqpoint{0.000000in}{0.000000in}}{%
\pgfpathmoveto{\pgfqpoint{0.000000in}{0.000000in}}%
\pgfpathlineto{\pgfqpoint{0.000000in}{-0.048611in}}%
\pgfusepath{stroke,fill}%
}%
\begin{pgfscope}%
\pgfsys@transformshift{3.426343in}{0.500000in}%
\pgfsys@useobject{currentmarker}{}%
\end{pgfscope}%
\end{pgfscope}%
\begin{pgfscope}%
\definecolor{textcolor}{rgb}{0.000000,0.000000,0.000000}%
\pgfsetstrokecolor{textcolor}%
\pgfsetfillcolor{textcolor}%
\pgftext[x=3.426343in,y=0.402778in,,top]{\color{textcolor}\rmfamily\fontsize{10.000000}{12.000000}\selectfont \(\displaystyle {700}\)}%
\end{pgfscope}%
\begin{pgfscope}%
\pgfsetbuttcap%
\pgfsetroundjoin%
\definecolor{currentfill}{rgb}{0.000000,0.000000,0.000000}%
\pgfsetfillcolor{currentfill}%
\pgfsetlinewidth{0.803000pt}%
\definecolor{currentstroke}{rgb}{0.000000,0.000000,0.000000}%
\pgfsetstrokecolor{currentstroke}%
\pgfsetdash{}{0pt}%
\pgfsys@defobject{currentmarker}{\pgfqpoint{0.000000in}{-0.048611in}}{\pgfqpoint{0.000000in}{0.000000in}}{%
\pgfpathmoveto{\pgfqpoint{0.000000in}{0.000000in}}%
\pgfpathlineto{\pgfqpoint{0.000000in}{-0.048611in}}%
\pgfusepath{stroke,fill}%
}%
\begin{pgfscope}%
\pgfsys@transformshift{3.984031in}{0.500000in}%
\pgfsys@useobject{currentmarker}{}%
\end{pgfscope}%
\end{pgfscope}%
\begin{pgfscope}%
\definecolor{textcolor}{rgb}{0.000000,0.000000,0.000000}%
\pgfsetstrokecolor{textcolor}%
\pgfsetfillcolor{textcolor}%
\pgftext[x=3.984031in,y=0.402778in,,top]{\color{textcolor}\rmfamily\fontsize{10.000000}{12.000000}\selectfont \(\displaystyle {750}\)}%
\end{pgfscope}%
\begin{pgfscope}%
\pgfsetbuttcap%
\pgfsetroundjoin%
\definecolor{currentfill}{rgb}{0.000000,0.000000,0.000000}%
\pgfsetfillcolor{currentfill}%
\pgfsetlinewidth{0.803000pt}%
\definecolor{currentstroke}{rgb}{0.000000,0.000000,0.000000}%
\pgfsetstrokecolor{currentstroke}%
\pgfsetdash{}{0pt}%
\pgfsys@defobject{currentmarker}{\pgfqpoint{0.000000in}{-0.048611in}}{\pgfqpoint{0.000000in}{0.000000in}}{%
\pgfpathmoveto{\pgfqpoint{0.000000in}{0.000000in}}%
\pgfpathlineto{\pgfqpoint{0.000000in}{-0.048611in}}%
\pgfusepath{stroke,fill}%
}%
\begin{pgfscope}%
\pgfsys@transformshift{4.541719in}{0.500000in}%
\pgfsys@useobject{currentmarker}{}%
\end{pgfscope}%
\end{pgfscope}%
\begin{pgfscope}%
\definecolor{textcolor}{rgb}{0.000000,0.000000,0.000000}%
\pgfsetstrokecolor{textcolor}%
\pgfsetfillcolor{textcolor}%
\pgftext[x=4.541719in,y=0.402778in,,top]{\color{textcolor}\rmfamily\fontsize{10.000000}{12.000000}\selectfont \(\displaystyle {800}\)}%
\end{pgfscope}%
\begin{pgfscope}%
\pgfsetbuttcap%
\pgfsetroundjoin%
\definecolor{currentfill}{rgb}{0.000000,0.000000,0.000000}%
\pgfsetfillcolor{currentfill}%
\pgfsetlinewidth{0.803000pt}%
\definecolor{currentstroke}{rgb}{0.000000,0.000000,0.000000}%
\pgfsetstrokecolor{currentstroke}%
\pgfsetdash{}{0pt}%
\pgfsys@defobject{currentmarker}{\pgfqpoint{0.000000in}{-0.048611in}}{\pgfqpoint{0.000000in}{0.000000in}}{%
\pgfpathmoveto{\pgfqpoint{0.000000in}{0.000000in}}%
\pgfpathlineto{\pgfqpoint{0.000000in}{-0.048611in}}%
\pgfusepath{stroke,fill}%
}%
\begin{pgfscope}%
\pgfsys@transformshift{5.099406in}{0.500000in}%
\pgfsys@useobject{currentmarker}{}%
\end{pgfscope}%
\end{pgfscope}%
\begin{pgfscope}%
\definecolor{textcolor}{rgb}{0.000000,0.000000,0.000000}%
\pgfsetstrokecolor{textcolor}%
\pgfsetfillcolor{textcolor}%
\pgftext[x=5.099406in,y=0.402778in,,top]{\color{textcolor}\rmfamily\fontsize{10.000000}{12.000000}\selectfont \(\displaystyle {850}\)}%
\end{pgfscope}%
\begin{pgfscope}%
\definecolor{textcolor}{rgb}{0.000000,0.000000,0.000000}%
\pgfsetstrokecolor{textcolor}%
\pgfsetfillcolor{textcolor}%
\pgftext[x=3.075000in,y=0.223766in,,top]{\color{textcolor}\rmfamily\fontsize{10.000000}{12.000000}\selectfont time [s]}%
\end{pgfscope}%
\begin{pgfscope}%
\pgfsetbuttcap%
\pgfsetroundjoin%
\definecolor{currentfill}{rgb}{0.000000,0.000000,0.000000}%
\pgfsetfillcolor{currentfill}%
\pgfsetlinewidth{0.803000pt}%
\definecolor{currentstroke}{rgb}{0.000000,0.000000,0.000000}%
\pgfsetstrokecolor{currentstroke}%
\pgfsetdash{}{0pt}%
\pgfsys@defobject{currentmarker}{\pgfqpoint{-0.048611in}{0.000000in}}{\pgfqpoint{-0.000000in}{0.000000in}}{%
\pgfpathmoveto{\pgfqpoint{-0.000000in}{0.000000in}}%
\pgfpathlineto{\pgfqpoint{-0.048611in}{0.000000in}}%
\pgfusepath{stroke,fill}%
}%
\begin{pgfscope}%
\pgfsys@transformshift{0.750000in}{0.757969in}%
\pgfsys@useobject{currentmarker}{}%
\end{pgfscope}%
\end{pgfscope}%
\begin{pgfscope}%
\definecolor{textcolor}{rgb}{0.000000,0.000000,0.000000}%
\pgfsetstrokecolor{textcolor}%
\pgfsetfillcolor{textcolor}%
\pgftext[x=0.336419in, y=0.709744in, left, base]{\color{textcolor}\rmfamily\fontsize{10.000000}{12.000000}\selectfont \(\displaystyle {0.008}\)}%
\end{pgfscope}%
\begin{pgfscope}%
\pgfsetbuttcap%
\pgfsetroundjoin%
\definecolor{currentfill}{rgb}{0.000000,0.000000,0.000000}%
\pgfsetfillcolor{currentfill}%
\pgfsetlinewidth{0.803000pt}%
\definecolor{currentstroke}{rgb}{0.000000,0.000000,0.000000}%
\pgfsetstrokecolor{currentstroke}%
\pgfsetdash{}{0pt}%
\pgfsys@defobject{currentmarker}{\pgfqpoint{-0.048611in}{0.000000in}}{\pgfqpoint{-0.000000in}{0.000000in}}{%
\pgfpathmoveto{\pgfqpoint{-0.000000in}{0.000000in}}%
\pgfpathlineto{\pgfqpoint{-0.048611in}{0.000000in}}%
\pgfusepath{stroke,fill}%
}%
\begin{pgfscope}%
\pgfsys@transformshift{0.750000in}{1.131544in}%
\pgfsys@useobject{currentmarker}{}%
\end{pgfscope}%
\end{pgfscope}%
\begin{pgfscope}%
\definecolor{textcolor}{rgb}{0.000000,0.000000,0.000000}%
\pgfsetstrokecolor{textcolor}%
\pgfsetfillcolor{textcolor}%
\pgftext[x=0.336419in, y=1.083319in, left, base]{\color{textcolor}\rmfamily\fontsize{10.000000}{12.000000}\selectfont \(\displaystyle {0.010}\)}%
\end{pgfscope}%
\begin{pgfscope}%
\pgfsetbuttcap%
\pgfsetroundjoin%
\definecolor{currentfill}{rgb}{0.000000,0.000000,0.000000}%
\pgfsetfillcolor{currentfill}%
\pgfsetlinewidth{0.803000pt}%
\definecolor{currentstroke}{rgb}{0.000000,0.000000,0.000000}%
\pgfsetstrokecolor{currentstroke}%
\pgfsetdash{}{0pt}%
\pgfsys@defobject{currentmarker}{\pgfqpoint{-0.048611in}{0.000000in}}{\pgfqpoint{-0.000000in}{0.000000in}}{%
\pgfpathmoveto{\pgfqpoint{-0.000000in}{0.000000in}}%
\pgfpathlineto{\pgfqpoint{-0.048611in}{0.000000in}}%
\pgfusepath{stroke,fill}%
}%
\begin{pgfscope}%
\pgfsys@transformshift{0.750000in}{1.505118in}%
\pgfsys@useobject{currentmarker}{}%
\end{pgfscope}%
\end{pgfscope}%
\begin{pgfscope}%
\definecolor{textcolor}{rgb}{0.000000,0.000000,0.000000}%
\pgfsetstrokecolor{textcolor}%
\pgfsetfillcolor{textcolor}%
\pgftext[x=0.336419in, y=1.456893in, left, base]{\color{textcolor}\rmfamily\fontsize{10.000000}{12.000000}\selectfont \(\displaystyle {0.012}\)}%
\end{pgfscope}%
\begin{pgfscope}%
\pgfsetbuttcap%
\pgfsetroundjoin%
\definecolor{currentfill}{rgb}{0.000000,0.000000,0.000000}%
\pgfsetfillcolor{currentfill}%
\pgfsetlinewidth{0.803000pt}%
\definecolor{currentstroke}{rgb}{0.000000,0.000000,0.000000}%
\pgfsetstrokecolor{currentstroke}%
\pgfsetdash{}{0pt}%
\pgfsys@defobject{currentmarker}{\pgfqpoint{-0.048611in}{0.000000in}}{\pgfqpoint{-0.000000in}{0.000000in}}{%
\pgfpathmoveto{\pgfqpoint{-0.000000in}{0.000000in}}%
\pgfpathlineto{\pgfqpoint{-0.048611in}{0.000000in}}%
\pgfusepath{stroke,fill}%
}%
\begin{pgfscope}%
\pgfsys@transformshift{0.750000in}{1.878693in}%
\pgfsys@useobject{currentmarker}{}%
\end{pgfscope}%
\end{pgfscope}%
\begin{pgfscope}%
\definecolor{textcolor}{rgb}{0.000000,0.000000,0.000000}%
\pgfsetstrokecolor{textcolor}%
\pgfsetfillcolor{textcolor}%
\pgftext[x=0.336419in, y=1.830468in, left, base]{\color{textcolor}\rmfamily\fontsize{10.000000}{12.000000}\selectfont \(\displaystyle {0.014}\)}%
\end{pgfscope}%
\begin{pgfscope}%
\pgfsetbuttcap%
\pgfsetroundjoin%
\definecolor{currentfill}{rgb}{0.000000,0.000000,0.000000}%
\pgfsetfillcolor{currentfill}%
\pgfsetlinewidth{0.803000pt}%
\definecolor{currentstroke}{rgb}{0.000000,0.000000,0.000000}%
\pgfsetstrokecolor{currentstroke}%
\pgfsetdash{}{0pt}%
\pgfsys@defobject{currentmarker}{\pgfqpoint{-0.048611in}{0.000000in}}{\pgfqpoint{-0.000000in}{0.000000in}}{%
\pgfpathmoveto{\pgfqpoint{-0.000000in}{0.000000in}}%
\pgfpathlineto{\pgfqpoint{-0.048611in}{0.000000in}}%
\pgfusepath{stroke,fill}%
}%
\begin{pgfscope}%
\pgfsys@transformshift{0.750000in}{2.252267in}%
\pgfsys@useobject{currentmarker}{}%
\end{pgfscope}%
\end{pgfscope}%
\begin{pgfscope}%
\definecolor{textcolor}{rgb}{0.000000,0.000000,0.000000}%
\pgfsetstrokecolor{textcolor}%
\pgfsetfillcolor{textcolor}%
\pgftext[x=0.336419in, y=2.204042in, left, base]{\color{textcolor}\rmfamily\fontsize{10.000000}{12.000000}\selectfont \(\displaystyle {0.016}\)}%
\end{pgfscope}%
\begin{pgfscope}%
\pgfsetbuttcap%
\pgfsetroundjoin%
\definecolor{currentfill}{rgb}{0.000000,0.000000,0.000000}%
\pgfsetfillcolor{currentfill}%
\pgfsetlinewidth{0.803000pt}%
\definecolor{currentstroke}{rgb}{0.000000,0.000000,0.000000}%
\pgfsetstrokecolor{currentstroke}%
\pgfsetdash{}{0pt}%
\pgfsys@defobject{currentmarker}{\pgfqpoint{-0.048611in}{0.000000in}}{\pgfqpoint{-0.000000in}{0.000000in}}{%
\pgfpathmoveto{\pgfqpoint{-0.000000in}{0.000000in}}%
\pgfpathlineto{\pgfqpoint{-0.048611in}{0.000000in}}%
\pgfusepath{stroke,fill}%
}%
\begin{pgfscope}%
\pgfsys@transformshift{0.750000in}{2.625842in}%
\pgfsys@useobject{currentmarker}{}%
\end{pgfscope}%
\end{pgfscope}%
\begin{pgfscope}%
\definecolor{textcolor}{rgb}{0.000000,0.000000,0.000000}%
\pgfsetstrokecolor{textcolor}%
\pgfsetfillcolor{textcolor}%
\pgftext[x=0.336419in, y=2.577617in, left, base]{\color{textcolor}\rmfamily\fontsize{10.000000}{12.000000}\selectfont \(\displaystyle {0.018}\)}%
\end{pgfscope}%
\begin{pgfscope}%
\pgfsetbuttcap%
\pgfsetroundjoin%
\definecolor{currentfill}{rgb}{0.000000,0.000000,0.000000}%
\pgfsetfillcolor{currentfill}%
\pgfsetlinewidth{0.803000pt}%
\definecolor{currentstroke}{rgb}{0.000000,0.000000,0.000000}%
\pgfsetstrokecolor{currentstroke}%
\pgfsetdash{}{0pt}%
\pgfsys@defobject{currentmarker}{\pgfqpoint{-0.048611in}{0.000000in}}{\pgfqpoint{-0.000000in}{0.000000in}}{%
\pgfpathmoveto{\pgfqpoint{-0.000000in}{0.000000in}}%
\pgfpathlineto{\pgfqpoint{-0.048611in}{0.000000in}}%
\pgfusepath{stroke,fill}%
}%
\begin{pgfscope}%
\pgfsys@transformshift{0.750000in}{2.999417in}%
\pgfsys@useobject{currentmarker}{}%
\end{pgfscope}%
\end{pgfscope}%
\begin{pgfscope}%
\definecolor{textcolor}{rgb}{0.000000,0.000000,0.000000}%
\pgfsetstrokecolor{textcolor}%
\pgfsetfillcolor{textcolor}%
\pgftext[x=0.336419in, y=2.951191in, left, base]{\color{textcolor}\rmfamily\fontsize{10.000000}{12.000000}\selectfont \(\displaystyle {0.020}\)}%
\end{pgfscope}%
\begin{pgfscope}%
\pgfsetbuttcap%
\pgfsetroundjoin%
\definecolor{currentfill}{rgb}{0.000000,0.000000,0.000000}%
\pgfsetfillcolor{currentfill}%
\pgfsetlinewidth{0.803000pt}%
\definecolor{currentstroke}{rgb}{0.000000,0.000000,0.000000}%
\pgfsetstrokecolor{currentstroke}%
\pgfsetdash{}{0pt}%
\pgfsys@defobject{currentmarker}{\pgfqpoint{-0.048611in}{0.000000in}}{\pgfqpoint{-0.000000in}{0.000000in}}{%
\pgfpathmoveto{\pgfqpoint{-0.000000in}{0.000000in}}%
\pgfpathlineto{\pgfqpoint{-0.048611in}{0.000000in}}%
\pgfusepath{stroke,fill}%
}%
\begin{pgfscope}%
\pgfsys@transformshift{0.750000in}{3.372991in}%
\pgfsys@useobject{currentmarker}{}%
\end{pgfscope}%
\end{pgfscope}%
\begin{pgfscope}%
\definecolor{textcolor}{rgb}{0.000000,0.000000,0.000000}%
\pgfsetstrokecolor{textcolor}%
\pgfsetfillcolor{textcolor}%
\pgftext[x=0.336419in, y=3.324766in, left, base]{\color{textcolor}\rmfamily\fontsize{10.000000}{12.000000}\selectfont \(\displaystyle {0.022}\)}%
\end{pgfscope}%
\begin{pgfscope}%
\definecolor{textcolor}{rgb}{0.000000,0.000000,0.000000}%
\pgfsetstrokecolor{textcolor}%
\pgfsetfillcolor{textcolor}%
\pgftext[x=0.280863in,y=2.010000in,,bottom,rotate=90.000000]{\color{textcolor}\rmfamily\fontsize{10.000000}{12.000000}\selectfont position [m]}%
\end{pgfscope}%
\begin{pgfscope}%
\pgfpathrectangle{\pgfqpoint{0.750000in}{0.500000in}}{\pgfqpoint{4.650000in}{3.020000in}}%
\pgfusepath{clip}%
\pgfsetrectcap%
\pgfsetroundjoin%
\pgfsetlinewidth{1.505625pt}%
\definecolor{currentstroke}{rgb}{0.121569,0.466667,0.705882}%
\pgfsetstrokecolor{currentstroke}%
\pgfsetdash{}{0pt}%
\pgfpathmoveto{\pgfqpoint{0.961364in}{0.637273in}}%
\pgfpathlineto{\pgfqpoint{1.619435in}{0.638349in}}%
\pgfpathlineto{\pgfqpoint{1.652896in}{0.640622in}}%
\pgfpathlineto{\pgfqpoint{1.675204in}{0.644950in}}%
\pgfpathlineto{\pgfqpoint{1.697511in}{0.653890in}}%
\pgfpathlineto{\pgfqpoint{1.719819in}{0.668285in}}%
\pgfpathlineto{\pgfqpoint{1.742126in}{0.689293in}}%
\pgfpathlineto{\pgfqpoint{1.764434in}{0.723900in}}%
\pgfpathlineto{\pgfqpoint{1.786741in}{0.723900in}}%
\pgfpathlineto{\pgfqpoint{1.797895in}{0.727551in}}%
\pgfpathlineto{\pgfqpoint{1.831356in}{0.751035in}}%
\pgfpathlineto{\pgfqpoint{1.842510in}{0.763268in}}%
\pgfpathlineto{\pgfqpoint{1.853664in}{0.779016in}}%
\pgfpathlineto{\pgfqpoint{1.864818in}{0.801684in}}%
\pgfpathlineto{\pgfqpoint{1.887125in}{0.866639in}}%
\pgfpathlineto{\pgfqpoint{1.931740in}{1.012283in}}%
\pgfpathlineto{\pgfqpoint{1.987509in}{1.192769in}}%
\pgfpathlineto{\pgfqpoint{2.009817in}{1.253487in}}%
\pgfpathlineto{\pgfqpoint{2.032124in}{1.301065in}}%
\pgfpathlineto{\pgfqpoint{2.054432in}{1.339843in}}%
\pgfpathlineto{\pgfqpoint{2.087893in}{1.395468in}}%
\pgfpathlineto{\pgfqpoint{2.099047in}{1.416505in}}%
\pgfpathlineto{\pgfqpoint{2.110200in}{1.426433in}}%
\pgfpathlineto{\pgfqpoint{2.132508in}{1.426433in}}%
\pgfpathlineto{\pgfqpoint{2.143662in}{1.432731in}}%
\pgfpathlineto{\pgfqpoint{2.165969in}{1.459439in}}%
\pgfpathlineto{\pgfqpoint{2.177123in}{1.481498in}}%
\pgfpathlineto{\pgfqpoint{2.199430in}{1.541593in}}%
\pgfpathlineto{\pgfqpoint{2.221738in}{1.610289in}}%
\pgfpathlineto{\pgfqpoint{2.288660in}{1.830748in}}%
\pgfpathlineto{\pgfqpoint{2.310968in}{1.910496in}}%
\pgfpathlineto{\pgfqpoint{2.333275in}{1.978472in}}%
\pgfpathlineto{\pgfqpoint{2.355583in}{2.036900in}}%
\pgfpathlineto{\pgfqpoint{2.400198in}{2.147875in}}%
\pgfpathlineto{\pgfqpoint{2.422505in}{2.211822in}}%
\pgfpathlineto{\pgfqpoint{2.444813in}{2.284948in}}%
\pgfpathlineto{\pgfqpoint{2.478274in}{2.407671in}}%
\pgfpathlineto{\pgfqpoint{2.522889in}{2.572204in}}%
\pgfpathlineto{\pgfqpoint{2.556350in}{2.680266in}}%
\pgfpathlineto{\pgfqpoint{2.589812in}{2.782080in}}%
\pgfpathlineto{\pgfqpoint{2.600965in}{2.797344in}}%
\pgfpathlineto{\pgfqpoint{2.623273in}{2.797344in}}%
\pgfpathlineto{\pgfqpoint{2.634427in}{2.807211in}}%
\pgfpathlineto{\pgfqpoint{2.656734in}{2.847094in}}%
\pgfpathlineto{\pgfqpoint{2.690195in}{2.958122in}}%
\pgfpathlineto{\pgfqpoint{2.723657in}{3.083638in}}%
\pgfpathlineto{\pgfqpoint{2.734811in}{3.126228in}}%
\pgfpathlineto{\pgfqpoint{2.757118in}{3.126228in}}%
\pgfpathlineto{\pgfqpoint{2.768272in}{3.128166in}}%
\pgfpathlineto{\pgfqpoint{2.812887in}{3.145040in}}%
\pgfpathlineto{\pgfqpoint{2.824041in}{3.152309in}}%
\pgfpathlineto{\pgfqpoint{2.835194in}{3.163623in}}%
\pgfpathlineto{\pgfqpoint{2.846348in}{3.183210in}}%
\pgfpathlineto{\pgfqpoint{2.879809in}{3.281319in}}%
\pgfpathlineto{\pgfqpoint{2.902117in}{3.334052in}}%
\pgfpathlineto{\pgfqpoint{2.913271in}{3.354240in}}%
\pgfpathlineto{\pgfqpoint{2.924424in}{3.369392in}}%
\pgfpathlineto{\pgfqpoint{2.935578in}{3.378990in}}%
\pgfpathlineto{\pgfqpoint{2.946732in}{3.382695in}}%
\pgfpathlineto{\pgfqpoint{3.058269in}{3.381650in}}%
\pgfpathlineto{\pgfqpoint{3.080577in}{3.378679in}}%
\pgfpathlineto{\pgfqpoint{3.091731in}{3.376170in}}%
\pgfpathlineto{\pgfqpoint{3.114038in}{3.367036in}}%
\pgfpathlineto{\pgfqpoint{3.125192in}{3.358402in}}%
\pgfpathlineto{\pgfqpoint{3.147499in}{3.327662in}}%
\pgfpathlineto{\pgfqpoint{3.180961in}{3.270519in}}%
\pgfpathlineto{\pgfqpoint{3.214422in}{3.214490in}}%
\pgfpathlineto{\pgfqpoint{3.225576in}{3.198090in}}%
\pgfpathlineto{\pgfqpoint{3.236729in}{3.189583in}}%
\pgfpathlineto{\pgfqpoint{3.259037in}{3.189583in}}%
\pgfpathlineto{\pgfqpoint{3.270191in}{3.185847in}}%
\pgfpathlineto{\pgfqpoint{3.281344in}{3.178575in}}%
\pgfpathlineto{\pgfqpoint{3.303652in}{3.168649in}}%
\pgfpathlineto{\pgfqpoint{3.325959in}{3.153361in}}%
\pgfpathlineto{\pgfqpoint{3.337113in}{3.142855in}}%
\pgfpathlineto{\pgfqpoint{3.359421in}{3.117976in}}%
\pgfpathlineto{\pgfqpoint{3.392882in}{3.082260in}}%
\pgfpathlineto{\pgfqpoint{3.404036in}{3.068169in}}%
\pgfpathlineto{\pgfqpoint{3.415189in}{3.058267in}}%
\pgfpathlineto{\pgfqpoint{3.437497in}{3.058267in}}%
\pgfpathlineto{\pgfqpoint{3.448651in}{3.053273in}}%
\pgfpathlineto{\pgfqpoint{3.470958in}{3.031920in}}%
\pgfpathlineto{\pgfqpoint{3.493266in}{2.989395in}}%
\pgfpathlineto{\pgfqpoint{3.549035in}{2.871069in}}%
\pgfpathlineto{\pgfqpoint{3.560188in}{2.857044in}}%
\pgfpathlineto{\pgfqpoint{3.582496in}{2.857044in}}%
\pgfpathlineto{\pgfqpoint{3.593650in}{2.849018in}}%
\pgfpathlineto{\pgfqpoint{3.615957in}{2.818357in}}%
\pgfpathlineto{\pgfqpoint{3.638265in}{2.773604in}}%
\pgfpathlineto{\pgfqpoint{3.694033in}{2.669519in}}%
\pgfpathlineto{\pgfqpoint{3.727495in}{2.596369in}}%
\pgfpathlineto{\pgfqpoint{3.783263in}{2.439324in}}%
\pgfpathlineto{\pgfqpoint{3.816725in}{2.349379in}}%
\pgfpathlineto{\pgfqpoint{3.850186in}{2.245824in}}%
\pgfpathlineto{\pgfqpoint{3.872493in}{2.172324in}}%
\pgfpathlineto{\pgfqpoint{3.883647in}{2.162396in}}%
\pgfpathlineto{\pgfqpoint{3.905955in}{2.162396in}}%
\pgfpathlineto{\pgfqpoint{3.917108in}{2.146198in}}%
\pgfpathlineto{\pgfqpoint{3.939416in}{2.082235in}}%
\pgfpathlineto{\pgfqpoint{3.961723in}{1.985170in}}%
\pgfpathlineto{\pgfqpoint{3.984031in}{1.985170in}}%
\pgfpathlineto{\pgfqpoint{3.995185in}{1.983147in}}%
\pgfpathlineto{\pgfqpoint{4.039800in}{1.967277in}}%
\pgfpathlineto{\pgfqpoint{4.050953in}{1.961168in}}%
\pgfpathlineto{\pgfqpoint{4.062107in}{1.952419in}}%
\pgfpathlineto{\pgfqpoint{4.073261in}{1.939920in}}%
\pgfpathlineto{\pgfqpoint{4.117876in}{1.871693in}}%
\pgfpathlineto{\pgfqpoint{4.129030in}{1.860759in}}%
\pgfpathlineto{\pgfqpoint{4.140183in}{1.855157in}}%
\pgfpathlineto{\pgfqpoint{4.251721in}{1.856023in}}%
\pgfpathlineto{\pgfqpoint{4.262875in}{1.860231in}}%
\pgfpathlineto{\pgfqpoint{4.285182in}{1.874438in}}%
\pgfpathlineto{\pgfqpoint{4.340951in}{1.918832in}}%
\pgfpathlineto{\pgfqpoint{4.352105in}{1.924566in}}%
\pgfpathlineto{\pgfqpoint{4.363259in}{1.928170in}}%
\pgfpathlineto{\pgfqpoint{4.374412in}{1.929417in}}%
\pgfpathlineto{\pgfqpoint{4.385566in}{1.928233in}}%
\pgfpathlineto{\pgfqpoint{4.396720in}{1.924704in}}%
\pgfpathlineto{\pgfqpoint{4.419027in}{1.911712in}}%
\pgfpathlineto{\pgfqpoint{4.474796in}{1.868189in}}%
\pgfpathlineto{\pgfqpoint{4.485950in}{1.862204in}}%
\pgfpathlineto{\pgfqpoint{4.497104in}{1.858230in}}%
\pgfpathlineto{\pgfqpoint{4.508257in}{1.856515in}}%
\pgfpathlineto{\pgfqpoint{4.519411in}{1.857158in}}%
\pgfpathlineto{\pgfqpoint{4.530565in}{1.860103in}}%
\pgfpathlineto{\pgfqpoint{4.541719in}{1.865142in}}%
\pgfpathlineto{\pgfqpoint{4.564026in}{1.880032in}}%
\pgfpathlineto{\pgfqpoint{4.608641in}{1.914303in}}%
\pgfpathlineto{\pgfqpoint{4.619795in}{1.920542in}}%
\pgfpathlineto{\pgfqpoint{4.630949in}{1.924923in}}%
\pgfpathlineto{\pgfqpoint{4.642102in}{1.927170in}}%
\pgfpathlineto{\pgfqpoint{4.653256in}{1.927142in}}%
\pgfpathlineto{\pgfqpoint{4.664410in}{1.924852in}}%
\pgfpathlineto{\pgfqpoint{4.675564in}{1.920457in}}%
\pgfpathlineto{\pgfqpoint{4.697871in}{1.906661in}}%
\pgfpathlineto{\pgfqpoint{4.742486in}{1.873113in}}%
\pgfpathlineto{\pgfqpoint{4.764794in}{1.862049in}}%
\pgfpathlineto{\pgfqpoint{4.775947in}{1.859419in}}%
\pgfpathlineto{\pgfqpoint{4.787101in}{1.858981in}}%
\pgfpathlineto{\pgfqpoint{4.798255in}{1.860753in}}%
\pgfpathlineto{\pgfqpoint{4.809409in}{1.864605in}}%
\pgfpathlineto{\pgfqpoint{4.831716in}{1.877382in}}%
\pgfpathlineto{\pgfqpoint{4.876331in}{1.915455in}}%
\pgfpathlineto{\pgfqpoint{4.887485in}{1.921662in}}%
\pgfpathlineto{\pgfqpoint{4.898639in}{1.925148in}}%
\pgfpathlineto{\pgfqpoint{4.909793in}{1.925602in}}%
\pgfpathlineto{\pgfqpoint{4.920946in}{1.922997in}}%
\pgfpathlineto{\pgfqpoint{4.932100in}{1.917592in}}%
\pgfpathlineto{\pgfqpoint{4.954408in}{1.900690in}}%
\pgfpathlineto{\pgfqpoint{4.987869in}{1.872771in}}%
\pgfpathlineto{\pgfqpoint{4.999023in}{1.866310in}}%
\pgfpathlineto{\pgfqpoint{5.010176in}{1.862410in}}%
\pgfpathlineto{\pgfqpoint{5.021330in}{1.861421in}}%
\pgfpathlineto{\pgfqpoint{5.032484in}{1.863417in}}%
\pgfpathlineto{\pgfqpoint{5.043638in}{1.868192in}}%
\pgfpathlineto{\pgfqpoint{5.065945in}{1.884016in}}%
\pgfpathlineto{\pgfqpoint{5.099406in}{1.911580in}}%
\pgfpathlineto{\pgfqpoint{5.110560in}{1.918370in}}%
\pgfpathlineto{\pgfqpoint{5.121714in}{1.922799in}}%
\pgfpathlineto{\pgfqpoint{5.132868in}{1.924465in}}%
\pgfpathlineto{\pgfqpoint{5.144021in}{1.923230in}}%
\pgfpathlineto{\pgfqpoint{5.155175in}{1.919224in}}%
\pgfpathlineto{\pgfqpoint{5.188636in}{1.900306in}}%
\pgfpathlineto{\pgfqpoint{5.188636in}{1.900306in}}%
\pgfusepath{stroke}%
\end{pgfscope}%
\begin{pgfscope}%
\pgfsetrectcap%
\pgfsetmiterjoin%
\pgfsetlinewidth{0.803000pt}%
\definecolor{currentstroke}{rgb}{0.000000,0.000000,0.000000}%
\pgfsetstrokecolor{currentstroke}%
\pgfsetdash{}{0pt}%
\pgfpathmoveto{\pgfqpoint{0.750000in}{0.500000in}}%
\pgfpathlineto{\pgfqpoint{0.750000in}{3.520000in}}%
\pgfusepath{stroke}%
\end{pgfscope}%
\begin{pgfscope}%
\pgfsetrectcap%
\pgfsetmiterjoin%
\pgfsetlinewidth{0.803000pt}%
\definecolor{currentstroke}{rgb}{0.000000,0.000000,0.000000}%
\pgfsetstrokecolor{currentstroke}%
\pgfsetdash{}{0pt}%
\pgfpathmoveto{\pgfqpoint{5.400000in}{0.500000in}}%
\pgfpathlineto{\pgfqpoint{5.400000in}{3.520000in}}%
\pgfusepath{stroke}%
\end{pgfscope}%
\begin{pgfscope}%
\pgfsetrectcap%
\pgfsetmiterjoin%
\pgfsetlinewidth{0.803000pt}%
\definecolor{currentstroke}{rgb}{0.000000,0.000000,0.000000}%
\pgfsetstrokecolor{currentstroke}%
\pgfsetdash{}{0pt}%
\pgfpathmoveto{\pgfqpoint{0.750000in}{0.500000in}}%
\pgfpathlineto{\pgfqpoint{5.400000in}{0.500000in}}%
\pgfusepath{stroke}%
\end{pgfscope}%
\begin{pgfscope}%
\pgfsetrectcap%
\pgfsetmiterjoin%
\pgfsetlinewidth{0.803000pt}%
\definecolor{currentstroke}{rgb}{0.000000,0.000000,0.000000}%
\pgfsetstrokecolor{currentstroke}%
\pgfsetdash{}{0pt}%
\pgfpathmoveto{\pgfqpoint{0.750000in}{3.520000in}}%
\pgfpathlineto{\pgfqpoint{5.400000in}{3.520000in}}%
\pgfusepath{stroke}%
\end{pgfscope}%
\end{pgfpicture}%
\makeatother%
\endgroup%

    \caption{Hydraulic Crane End Effector Position}
    \label{fig:hydraulic_pos}
\end{figure}

\subsubsection{Power Electronic Converter Dataset}
\label{ref_pec_dataset}

Transitioning from the conventional power systems to power electronics-dominated grids (PEDG) has increased demand for grid-forming converters (GFM) to facilitate operational reliability. Although significant research has been made on GFMs (as shown in Figure \ref{fig:sys}) to expedite stability under different grid conditions, its operation during faults or large signal disturbances still remain a challenge. \enquote{Compared to synchronous generators (SGs) that can support up to seven times over their rated current, power converters can only cope with a small percent of overcurrent (typically 20\%). Therefore, GFMs have to be protected against extreme faults (ex. short circuit, line-tripping/reclosing, etc) while being able to stay synchronized to the power system.}\parencite{trainsient-stability-9523750} Since the network infrastructure of power systems keeps expanding, it is important to identify these faults accurately under varying grid parameter uncertainties.

\begin{figure}[H]
	\includegraphics[width=0.65\textwidth]{Images/GFMSchema.pdf}
	\caption{Main circuit and control system structure of a grid-forming converter.}
	\label{fig:sys}
\end{figure}


In this dataset there are four faults considered below. This study will examine the frequency ($f_c$) of the system throughout various fault conditions. Figure \ref{fig:pec_all} shows the entire $f_c$ signal. Each fault has significantly different characteristics and magnitude. The faults are magnified and explained individually below.

\begin{figure}[H]
    %\centering
    %% Creator: Matplotlib, PGF backend
%%
%% To include the figure in your LaTeX document, write
%%   \input{<filename>.pgf}
%%
%% Make sure the required packages are loaded in your preamble
%%   \usepackage{pgf}
%%
%% Also ensure that all the required font packages are loaded; for instance,
%% the lmodern package is sometimes necessary when using math font.
%%   \usepackage{lmodern}
%%
%% Figures using additional raster images can only be included by \input if
%% they are in the same directory as the main LaTeX file. For loading figures
%% from other directories you can use the `import` package
%%   \usepackage{import}
%%
%% and then include the figures with
%%   \import{<path to file>}{<filename>.pgf}
%%
%% Matplotlib used the following preamble
%%
\begingroup%
\makeatletter%
\begin{pgfpicture}%
\pgfpathrectangle{\pgfpointorigin}{\pgfqpoint{6.000000in}{4.000000in}}%
\pgfusepath{use as bounding box, clip}%
\begin{pgfscope}%
\pgfsetbuttcap%
\pgfsetmiterjoin%
\pgfsetlinewidth{0.000000pt}%
\definecolor{currentstroke}{rgb}{1.000000,1.000000,1.000000}%
\pgfsetstrokecolor{currentstroke}%
\pgfsetstrokeopacity{0.000000}%
\pgfsetdash{}{0pt}%
\pgfpathmoveto{\pgfqpoint{0.000000in}{0.000000in}}%
\pgfpathlineto{\pgfqpoint{6.000000in}{0.000000in}}%
\pgfpathlineto{\pgfqpoint{6.000000in}{4.000000in}}%
\pgfpathlineto{\pgfqpoint{0.000000in}{4.000000in}}%
\pgfpathlineto{\pgfqpoint{0.000000in}{0.000000in}}%
\pgfpathclose%
\pgfusepath{}%
\end{pgfscope}%
\begin{pgfscope}%
\pgfsetbuttcap%
\pgfsetmiterjoin%
\definecolor{currentfill}{rgb}{1.000000,1.000000,1.000000}%
\pgfsetfillcolor{currentfill}%
\pgfsetlinewidth{0.000000pt}%
\definecolor{currentstroke}{rgb}{0.000000,0.000000,0.000000}%
\pgfsetstrokecolor{currentstroke}%
\pgfsetstrokeopacity{0.000000}%
\pgfsetdash{}{0pt}%
\pgfpathmoveto{\pgfqpoint{0.750000in}{0.500000in}}%
\pgfpathlineto{\pgfqpoint{5.400000in}{0.500000in}}%
\pgfpathlineto{\pgfqpoint{5.400000in}{3.520000in}}%
\pgfpathlineto{\pgfqpoint{0.750000in}{3.520000in}}%
\pgfpathlineto{\pgfqpoint{0.750000in}{0.500000in}}%
\pgfpathclose%
\pgfusepath{fill}%
\end{pgfscope}%
\begin{pgfscope}%
\pgfsetbuttcap%
\pgfsetroundjoin%
\definecolor{currentfill}{rgb}{0.000000,0.000000,0.000000}%
\pgfsetfillcolor{currentfill}%
\pgfsetlinewidth{0.803000pt}%
\definecolor{currentstroke}{rgb}{0.000000,0.000000,0.000000}%
\pgfsetstrokecolor{currentstroke}%
\pgfsetdash{}{0pt}%
\pgfsys@defobject{currentmarker}{\pgfqpoint{0.000000in}{-0.048611in}}{\pgfqpoint{0.000000in}{0.000000in}}{%
\pgfpathmoveto{\pgfqpoint{0.000000in}{0.000000in}}%
\pgfpathlineto{\pgfqpoint{0.000000in}{-0.048611in}}%
\pgfusepath{stroke,fill}%
}%
\begin{pgfscope}%
\pgfsys@transformshift{0.961364in}{0.500000in}%
\pgfsys@useobject{currentmarker}{}%
\end{pgfscope}%
\end{pgfscope}%
\begin{pgfscope}%
\definecolor{textcolor}{rgb}{0.000000,0.000000,0.000000}%
\pgfsetstrokecolor{textcolor}%
\pgfsetfillcolor{textcolor}%
\pgftext[x=0.961364in,y=0.402778in,,top]{\color{textcolor}\rmfamily\fontsize{10.000000}{12.000000}\selectfont \(\displaystyle {0}\)}%
\end{pgfscope}%
\begin{pgfscope}%
\pgfsetbuttcap%
\pgfsetroundjoin%
\definecolor{currentfill}{rgb}{0.000000,0.000000,0.000000}%
\pgfsetfillcolor{currentfill}%
\pgfsetlinewidth{0.803000pt}%
\definecolor{currentstroke}{rgb}{0.000000,0.000000,0.000000}%
\pgfsetstrokecolor{currentstroke}%
\pgfsetdash{}{0pt}%
\pgfsys@defobject{currentmarker}{\pgfqpoint{0.000000in}{-0.048611in}}{\pgfqpoint{0.000000in}{0.000000in}}{%
\pgfpathmoveto{\pgfqpoint{0.000000in}{0.000000in}}%
\pgfpathlineto{\pgfqpoint{0.000000in}{-0.048611in}}%
\pgfusepath{stroke,fill}%
}%
\begin{pgfscope}%
\pgfsys@transformshift{1.621766in}{0.500000in}%
\pgfsys@useobject{currentmarker}{}%
\end{pgfscope}%
\end{pgfscope}%
\begin{pgfscope}%
\definecolor{textcolor}{rgb}{0.000000,0.000000,0.000000}%
\pgfsetstrokecolor{textcolor}%
\pgfsetfillcolor{textcolor}%
\pgftext[x=1.621766in,y=0.402778in,,top]{\color{textcolor}\rmfamily\fontsize{10.000000}{12.000000}\selectfont \(\displaystyle {50000}\)}%
\end{pgfscope}%
\begin{pgfscope}%
\pgfsetbuttcap%
\pgfsetroundjoin%
\definecolor{currentfill}{rgb}{0.000000,0.000000,0.000000}%
\pgfsetfillcolor{currentfill}%
\pgfsetlinewidth{0.803000pt}%
\definecolor{currentstroke}{rgb}{0.000000,0.000000,0.000000}%
\pgfsetstrokecolor{currentstroke}%
\pgfsetdash{}{0pt}%
\pgfsys@defobject{currentmarker}{\pgfqpoint{0.000000in}{-0.048611in}}{\pgfqpoint{0.000000in}{0.000000in}}{%
\pgfpathmoveto{\pgfqpoint{0.000000in}{0.000000in}}%
\pgfpathlineto{\pgfqpoint{0.000000in}{-0.048611in}}%
\pgfusepath{stroke,fill}%
}%
\begin{pgfscope}%
\pgfsys@transformshift{2.282168in}{0.500000in}%
\pgfsys@useobject{currentmarker}{}%
\end{pgfscope}%
\end{pgfscope}%
\begin{pgfscope}%
\definecolor{textcolor}{rgb}{0.000000,0.000000,0.000000}%
\pgfsetstrokecolor{textcolor}%
\pgfsetfillcolor{textcolor}%
\pgftext[x=2.282168in,y=0.402778in,,top]{\color{textcolor}\rmfamily\fontsize{10.000000}{12.000000}\selectfont \(\displaystyle {100000}\)}%
\end{pgfscope}%
\begin{pgfscope}%
\pgfsetbuttcap%
\pgfsetroundjoin%
\definecolor{currentfill}{rgb}{0.000000,0.000000,0.000000}%
\pgfsetfillcolor{currentfill}%
\pgfsetlinewidth{0.803000pt}%
\definecolor{currentstroke}{rgb}{0.000000,0.000000,0.000000}%
\pgfsetstrokecolor{currentstroke}%
\pgfsetdash{}{0pt}%
\pgfsys@defobject{currentmarker}{\pgfqpoint{0.000000in}{-0.048611in}}{\pgfqpoint{0.000000in}{0.000000in}}{%
\pgfpathmoveto{\pgfqpoint{0.000000in}{0.000000in}}%
\pgfpathlineto{\pgfqpoint{0.000000in}{-0.048611in}}%
\pgfusepath{stroke,fill}%
}%
\begin{pgfscope}%
\pgfsys@transformshift{2.942570in}{0.500000in}%
\pgfsys@useobject{currentmarker}{}%
\end{pgfscope}%
\end{pgfscope}%
\begin{pgfscope}%
\definecolor{textcolor}{rgb}{0.000000,0.000000,0.000000}%
\pgfsetstrokecolor{textcolor}%
\pgfsetfillcolor{textcolor}%
\pgftext[x=2.942570in,y=0.402778in,,top]{\color{textcolor}\rmfamily\fontsize{10.000000}{12.000000}\selectfont \(\displaystyle {150000}\)}%
\end{pgfscope}%
\begin{pgfscope}%
\pgfsetbuttcap%
\pgfsetroundjoin%
\definecolor{currentfill}{rgb}{0.000000,0.000000,0.000000}%
\pgfsetfillcolor{currentfill}%
\pgfsetlinewidth{0.803000pt}%
\definecolor{currentstroke}{rgb}{0.000000,0.000000,0.000000}%
\pgfsetstrokecolor{currentstroke}%
\pgfsetdash{}{0pt}%
\pgfsys@defobject{currentmarker}{\pgfqpoint{0.000000in}{-0.048611in}}{\pgfqpoint{0.000000in}{0.000000in}}{%
\pgfpathmoveto{\pgfqpoint{0.000000in}{0.000000in}}%
\pgfpathlineto{\pgfqpoint{0.000000in}{-0.048611in}}%
\pgfusepath{stroke,fill}%
}%
\begin{pgfscope}%
\pgfsys@transformshift{3.602972in}{0.500000in}%
\pgfsys@useobject{currentmarker}{}%
\end{pgfscope}%
\end{pgfscope}%
\begin{pgfscope}%
\definecolor{textcolor}{rgb}{0.000000,0.000000,0.000000}%
\pgfsetstrokecolor{textcolor}%
\pgfsetfillcolor{textcolor}%
\pgftext[x=3.602972in,y=0.402778in,,top]{\color{textcolor}\rmfamily\fontsize{10.000000}{12.000000}\selectfont \(\displaystyle {200000}\)}%
\end{pgfscope}%
\begin{pgfscope}%
\pgfsetbuttcap%
\pgfsetroundjoin%
\definecolor{currentfill}{rgb}{0.000000,0.000000,0.000000}%
\pgfsetfillcolor{currentfill}%
\pgfsetlinewidth{0.803000pt}%
\definecolor{currentstroke}{rgb}{0.000000,0.000000,0.000000}%
\pgfsetstrokecolor{currentstroke}%
\pgfsetdash{}{0pt}%
\pgfsys@defobject{currentmarker}{\pgfqpoint{0.000000in}{-0.048611in}}{\pgfqpoint{0.000000in}{0.000000in}}{%
\pgfpathmoveto{\pgfqpoint{0.000000in}{0.000000in}}%
\pgfpathlineto{\pgfqpoint{0.000000in}{-0.048611in}}%
\pgfusepath{stroke,fill}%
}%
\begin{pgfscope}%
\pgfsys@transformshift{4.263374in}{0.500000in}%
\pgfsys@useobject{currentmarker}{}%
\end{pgfscope}%
\end{pgfscope}%
\begin{pgfscope}%
\definecolor{textcolor}{rgb}{0.000000,0.000000,0.000000}%
\pgfsetstrokecolor{textcolor}%
\pgfsetfillcolor{textcolor}%
\pgftext[x=4.263374in,y=0.402778in,,top]{\color{textcolor}\rmfamily\fontsize{10.000000}{12.000000}\selectfont \(\displaystyle {250000}\)}%
\end{pgfscope}%
\begin{pgfscope}%
\pgfsetbuttcap%
\pgfsetroundjoin%
\definecolor{currentfill}{rgb}{0.000000,0.000000,0.000000}%
\pgfsetfillcolor{currentfill}%
\pgfsetlinewidth{0.803000pt}%
\definecolor{currentstroke}{rgb}{0.000000,0.000000,0.000000}%
\pgfsetstrokecolor{currentstroke}%
\pgfsetdash{}{0pt}%
\pgfsys@defobject{currentmarker}{\pgfqpoint{0.000000in}{-0.048611in}}{\pgfqpoint{0.000000in}{0.000000in}}{%
\pgfpathmoveto{\pgfqpoint{0.000000in}{0.000000in}}%
\pgfpathlineto{\pgfqpoint{0.000000in}{-0.048611in}}%
\pgfusepath{stroke,fill}%
}%
\begin{pgfscope}%
\pgfsys@transformshift{4.923776in}{0.500000in}%
\pgfsys@useobject{currentmarker}{}%
\end{pgfscope}%
\end{pgfscope}%
\begin{pgfscope}%
\definecolor{textcolor}{rgb}{0.000000,0.000000,0.000000}%
\pgfsetstrokecolor{textcolor}%
\pgfsetfillcolor{textcolor}%
\pgftext[x=4.923776in,y=0.402778in,,top]{\color{textcolor}\rmfamily\fontsize{10.000000}{12.000000}\selectfont \(\displaystyle {300000}\)}%
\end{pgfscope}%
\begin{pgfscope}%
\definecolor{textcolor}{rgb}{0.000000,0.000000,0.000000}%
\pgfsetstrokecolor{textcolor}%
\pgfsetfillcolor{textcolor}%
\pgftext[x=3.075000in,y=0.223766in,,top]{\color{textcolor}\rmfamily\fontsize{10.000000}{12.000000}\selectfont Time (s)}%
\end{pgfscope}%
\begin{pgfscope}%
\pgfsetbuttcap%
\pgfsetroundjoin%
\definecolor{currentfill}{rgb}{0.000000,0.000000,0.000000}%
\pgfsetfillcolor{currentfill}%
\pgfsetlinewidth{0.803000pt}%
\definecolor{currentstroke}{rgb}{0.000000,0.000000,0.000000}%
\pgfsetstrokecolor{currentstroke}%
\pgfsetdash{}{0pt}%
\pgfsys@defobject{currentmarker}{\pgfqpoint{-0.048611in}{0.000000in}}{\pgfqpoint{-0.000000in}{0.000000in}}{%
\pgfpathmoveto{\pgfqpoint{-0.000000in}{0.000000in}}%
\pgfpathlineto{\pgfqpoint{-0.048611in}{0.000000in}}%
\pgfusepath{stroke,fill}%
}%
\begin{pgfscope}%
\pgfsys@transformshift{0.750000in}{0.708438in}%
\pgfsys@useobject{currentmarker}{}%
\end{pgfscope}%
\end{pgfscope}%
\begin{pgfscope}%
\definecolor{textcolor}{rgb}{0.000000,0.000000,0.000000}%
\pgfsetstrokecolor{textcolor}%
\pgfsetfillcolor{textcolor}%
\pgftext[x=0.475308in, y=0.660213in, left, base]{\color{textcolor}\rmfamily\fontsize{10.000000}{12.000000}\selectfont \(\displaystyle {\ensuremath{-}7}\)}%
\end{pgfscope}%
\begin{pgfscope}%
\pgfsetbuttcap%
\pgfsetroundjoin%
\definecolor{currentfill}{rgb}{0.000000,0.000000,0.000000}%
\pgfsetfillcolor{currentfill}%
\pgfsetlinewidth{0.803000pt}%
\definecolor{currentstroke}{rgb}{0.000000,0.000000,0.000000}%
\pgfsetstrokecolor{currentstroke}%
\pgfsetdash{}{0pt}%
\pgfsys@defobject{currentmarker}{\pgfqpoint{-0.048611in}{0.000000in}}{\pgfqpoint{-0.000000in}{0.000000in}}{%
\pgfpathmoveto{\pgfqpoint{-0.000000in}{0.000000in}}%
\pgfpathlineto{\pgfqpoint{-0.048611in}{0.000000in}}%
\pgfusepath{stroke,fill}%
}%
\begin{pgfscope}%
\pgfsys@transformshift{0.750000in}{1.085955in}%
\pgfsys@useobject{currentmarker}{}%
\end{pgfscope}%
\end{pgfscope}%
\begin{pgfscope}%
\definecolor{textcolor}{rgb}{0.000000,0.000000,0.000000}%
\pgfsetstrokecolor{textcolor}%
\pgfsetfillcolor{textcolor}%
\pgftext[x=0.475308in, y=1.037730in, left, base]{\color{textcolor}\rmfamily\fontsize{10.000000}{12.000000}\selectfont \(\displaystyle {\ensuremath{-}6}\)}%
\end{pgfscope}%
\begin{pgfscope}%
\pgfsetbuttcap%
\pgfsetroundjoin%
\definecolor{currentfill}{rgb}{0.000000,0.000000,0.000000}%
\pgfsetfillcolor{currentfill}%
\pgfsetlinewidth{0.803000pt}%
\definecolor{currentstroke}{rgb}{0.000000,0.000000,0.000000}%
\pgfsetstrokecolor{currentstroke}%
\pgfsetdash{}{0pt}%
\pgfsys@defobject{currentmarker}{\pgfqpoint{-0.048611in}{0.000000in}}{\pgfqpoint{-0.000000in}{0.000000in}}{%
\pgfpathmoveto{\pgfqpoint{-0.000000in}{0.000000in}}%
\pgfpathlineto{\pgfqpoint{-0.048611in}{0.000000in}}%
\pgfusepath{stroke,fill}%
}%
\begin{pgfscope}%
\pgfsys@transformshift{0.750000in}{1.463473in}%
\pgfsys@useobject{currentmarker}{}%
\end{pgfscope}%
\end{pgfscope}%
\begin{pgfscope}%
\definecolor{textcolor}{rgb}{0.000000,0.000000,0.000000}%
\pgfsetstrokecolor{textcolor}%
\pgfsetfillcolor{textcolor}%
\pgftext[x=0.475308in, y=1.415247in, left, base]{\color{textcolor}\rmfamily\fontsize{10.000000}{12.000000}\selectfont \(\displaystyle {\ensuremath{-}5}\)}%
\end{pgfscope}%
\begin{pgfscope}%
\pgfsetbuttcap%
\pgfsetroundjoin%
\definecolor{currentfill}{rgb}{0.000000,0.000000,0.000000}%
\pgfsetfillcolor{currentfill}%
\pgfsetlinewidth{0.803000pt}%
\definecolor{currentstroke}{rgb}{0.000000,0.000000,0.000000}%
\pgfsetstrokecolor{currentstroke}%
\pgfsetdash{}{0pt}%
\pgfsys@defobject{currentmarker}{\pgfqpoint{-0.048611in}{0.000000in}}{\pgfqpoint{-0.000000in}{0.000000in}}{%
\pgfpathmoveto{\pgfqpoint{-0.000000in}{0.000000in}}%
\pgfpathlineto{\pgfqpoint{-0.048611in}{0.000000in}}%
\pgfusepath{stroke,fill}%
}%
\begin{pgfscope}%
\pgfsys@transformshift{0.750000in}{1.840990in}%
\pgfsys@useobject{currentmarker}{}%
\end{pgfscope}%
\end{pgfscope}%
\begin{pgfscope}%
\definecolor{textcolor}{rgb}{0.000000,0.000000,0.000000}%
\pgfsetstrokecolor{textcolor}%
\pgfsetfillcolor{textcolor}%
\pgftext[x=0.475308in, y=1.792765in, left, base]{\color{textcolor}\rmfamily\fontsize{10.000000}{12.000000}\selectfont \(\displaystyle {\ensuremath{-}4}\)}%
\end{pgfscope}%
\begin{pgfscope}%
\pgfsetbuttcap%
\pgfsetroundjoin%
\definecolor{currentfill}{rgb}{0.000000,0.000000,0.000000}%
\pgfsetfillcolor{currentfill}%
\pgfsetlinewidth{0.803000pt}%
\definecolor{currentstroke}{rgb}{0.000000,0.000000,0.000000}%
\pgfsetstrokecolor{currentstroke}%
\pgfsetdash{}{0pt}%
\pgfsys@defobject{currentmarker}{\pgfqpoint{-0.048611in}{0.000000in}}{\pgfqpoint{-0.000000in}{0.000000in}}{%
\pgfpathmoveto{\pgfqpoint{-0.000000in}{0.000000in}}%
\pgfpathlineto{\pgfqpoint{-0.048611in}{0.000000in}}%
\pgfusepath{stroke,fill}%
}%
\begin{pgfscope}%
\pgfsys@transformshift{0.750000in}{2.218508in}%
\pgfsys@useobject{currentmarker}{}%
\end{pgfscope}%
\end{pgfscope}%
\begin{pgfscope}%
\definecolor{textcolor}{rgb}{0.000000,0.000000,0.000000}%
\pgfsetstrokecolor{textcolor}%
\pgfsetfillcolor{textcolor}%
\pgftext[x=0.475308in, y=2.170282in, left, base]{\color{textcolor}\rmfamily\fontsize{10.000000}{12.000000}\selectfont \(\displaystyle {\ensuremath{-}3}\)}%
\end{pgfscope}%
\begin{pgfscope}%
\pgfsetbuttcap%
\pgfsetroundjoin%
\definecolor{currentfill}{rgb}{0.000000,0.000000,0.000000}%
\pgfsetfillcolor{currentfill}%
\pgfsetlinewidth{0.803000pt}%
\definecolor{currentstroke}{rgb}{0.000000,0.000000,0.000000}%
\pgfsetstrokecolor{currentstroke}%
\pgfsetdash{}{0pt}%
\pgfsys@defobject{currentmarker}{\pgfqpoint{-0.048611in}{0.000000in}}{\pgfqpoint{-0.000000in}{0.000000in}}{%
\pgfpathmoveto{\pgfqpoint{-0.000000in}{0.000000in}}%
\pgfpathlineto{\pgfqpoint{-0.048611in}{0.000000in}}%
\pgfusepath{stroke,fill}%
}%
\begin{pgfscope}%
\pgfsys@transformshift{0.750000in}{2.596025in}%
\pgfsys@useobject{currentmarker}{}%
\end{pgfscope}%
\end{pgfscope}%
\begin{pgfscope}%
\definecolor{textcolor}{rgb}{0.000000,0.000000,0.000000}%
\pgfsetstrokecolor{textcolor}%
\pgfsetfillcolor{textcolor}%
\pgftext[x=0.475308in, y=2.547800in, left, base]{\color{textcolor}\rmfamily\fontsize{10.000000}{12.000000}\selectfont \(\displaystyle {\ensuremath{-}2}\)}%
\end{pgfscope}%
\begin{pgfscope}%
\pgfsetbuttcap%
\pgfsetroundjoin%
\definecolor{currentfill}{rgb}{0.000000,0.000000,0.000000}%
\pgfsetfillcolor{currentfill}%
\pgfsetlinewidth{0.803000pt}%
\definecolor{currentstroke}{rgb}{0.000000,0.000000,0.000000}%
\pgfsetstrokecolor{currentstroke}%
\pgfsetdash{}{0pt}%
\pgfsys@defobject{currentmarker}{\pgfqpoint{-0.048611in}{0.000000in}}{\pgfqpoint{-0.000000in}{0.000000in}}{%
\pgfpathmoveto{\pgfqpoint{-0.000000in}{0.000000in}}%
\pgfpathlineto{\pgfqpoint{-0.048611in}{0.000000in}}%
\pgfusepath{stroke,fill}%
}%
\begin{pgfscope}%
\pgfsys@transformshift{0.750000in}{2.973543in}%
\pgfsys@useobject{currentmarker}{}%
\end{pgfscope}%
\end{pgfscope}%
\begin{pgfscope}%
\definecolor{textcolor}{rgb}{0.000000,0.000000,0.000000}%
\pgfsetstrokecolor{textcolor}%
\pgfsetfillcolor{textcolor}%
\pgftext[x=0.475308in, y=2.925317in, left, base]{\color{textcolor}\rmfamily\fontsize{10.000000}{12.000000}\selectfont \(\displaystyle {\ensuremath{-}1}\)}%
\end{pgfscope}%
\begin{pgfscope}%
\pgfsetbuttcap%
\pgfsetroundjoin%
\definecolor{currentfill}{rgb}{0.000000,0.000000,0.000000}%
\pgfsetfillcolor{currentfill}%
\pgfsetlinewidth{0.803000pt}%
\definecolor{currentstroke}{rgb}{0.000000,0.000000,0.000000}%
\pgfsetstrokecolor{currentstroke}%
\pgfsetdash{}{0pt}%
\pgfsys@defobject{currentmarker}{\pgfqpoint{-0.048611in}{0.000000in}}{\pgfqpoint{-0.000000in}{0.000000in}}{%
\pgfpathmoveto{\pgfqpoint{-0.000000in}{0.000000in}}%
\pgfpathlineto{\pgfqpoint{-0.048611in}{0.000000in}}%
\pgfusepath{stroke,fill}%
}%
\begin{pgfscope}%
\pgfsys@transformshift{0.750000in}{3.351060in}%
\pgfsys@useobject{currentmarker}{}%
\end{pgfscope}%
\end{pgfscope}%
\begin{pgfscope}%
\definecolor{textcolor}{rgb}{0.000000,0.000000,0.000000}%
\pgfsetstrokecolor{textcolor}%
\pgfsetfillcolor{textcolor}%
\pgftext[x=0.583333in, y=3.302835in, left, base]{\color{textcolor}\rmfamily\fontsize{10.000000}{12.000000}\selectfont \(\displaystyle {0}\)}%
\end{pgfscope}%
\begin{pgfscope}%
\definecolor{textcolor}{rgb}{0.000000,0.000000,0.000000}%
\pgfsetstrokecolor{textcolor}%
\pgfsetfillcolor{textcolor}%
\pgftext[x=0.419753in,y=2.010000in,,bottom,rotate=90.000000]{\color{textcolor}\rmfamily\fontsize{10.000000}{12.000000}\selectfont Frequency (Hz)}%
\end{pgfscope}%
\begin{pgfscope}%
\definecolor{textcolor}{rgb}{0.000000,0.000000,0.000000}%
\pgfsetstrokecolor{textcolor}%
\pgfsetfillcolor{textcolor}%
\pgftext[x=0.750000in,y=3.561667in,left,base]{\color{textcolor}\rmfamily\fontsize{10.000000}{12.000000}\selectfont \(\displaystyle \times{10^{3}}{}\)}%
\end{pgfscope}%
\begin{pgfscope}%
\pgfpathrectangle{\pgfqpoint{0.750000in}{0.500000in}}{\pgfqpoint{4.650000in}{3.020000in}}%
\pgfusepath{clip}%
\pgfsetrectcap%
\pgfsetroundjoin%
\pgfsetlinewidth{1.505625pt}%
\definecolor{currentstroke}{rgb}{0.121569,0.466667,0.705882}%
\pgfsetstrokecolor{currentstroke}%
\pgfsetdash{}{0pt}%
\pgfpathmoveto{\pgfqpoint{0.961364in}{3.369936in}}%
\pgfpathlineto{\pgfqpoint{1.494176in}{3.369936in}}%
\pgfpathlineto{\pgfqpoint{1.494202in}{1.391339in}}%
\pgfpathlineto{\pgfqpoint{1.495708in}{3.350639in}}%
\pgfpathlineto{\pgfqpoint{1.499420in}{3.350348in}}%
\pgfpathlineto{\pgfqpoint{1.499921in}{3.347264in}}%
\pgfpathlineto{\pgfqpoint{1.500080in}{3.292320in}}%
\pgfpathlineto{\pgfqpoint{1.500304in}{0.659620in}}%
\pgfpathlineto{\pgfqpoint{1.500199in}{3.333591in}}%
\pgfpathlineto{\pgfqpoint{1.501995in}{1.043640in}}%
\pgfpathlineto{\pgfqpoint{1.503091in}{3.369482in}}%
\pgfpathlineto{\pgfqpoint{1.502022in}{0.708803in}}%
\pgfpathlineto{\pgfqpoint{1.503554in}{3.365091in}}%
\pgfpathlineto{\pgfqpoint{1.505218in}{3.361161in}}%
\pgfpathlineto{\pgfqpoint{1.505614in}{3.361358in}}%
\pgfpathlineto{\pgfqpoint{1.506816in}{3.364470in}}%
\pgfpathlineto{\pgfqpoint{1.508639in}{3.376281in}}%
\pgfpathlineto{\pgfqpoint{1.509629in}{3.370712in}}%
\pgfpathlineto{\pgfqpoint{1.513499in}{3.352974in}}%
\pgfpathlineto{\pgfqpoint{1.514252in}{3.354347in}}%
\pgfpathlineto{\pgfqpoint{1.516418in}{3.367468in}}%
\pgfpathlineto{\pgfqpoint{1.517633in}{3.382727in}}%
\pgfpathlineto{\pgfqpoint{1.518466in}{3.373422in}}%
\pgfpathlineto{\pgfqpoint{1.520103in}{3.363214in}}%
\pgfpathlineto{\pgfqpoint{1.520645in}{3.363941in}}%
\pgfpathlineto{\pgfqpoint{1.521913in}{3.371582in}}%
\pgfpathlineto{\pgfqpoint{1.522917in}{3.378926in}}%
\pgfpathlineto{\pgfqpoint{1.523683in}{3.373902in}}%
\pgfpathlineto{\pgfqpoint{1.525413in}{3.365533in}}%
\pgfpathlineto{\pgfqpoint{1.525994in}{3.366377in}}%
\pgfpathlineto{\pgfqpoint{1.528081in}{3.375631in}}%
\pgfpathlineto{\pgfqpoint{1.529190in}{3.370717in}}%
\pgfpathlineto{\pgfqpoint{1.530617in}{3.366736in}}%
\pgfpathlineto{\pgfqpoint{1.531291in}{3.367647in}}%
\pgfpathlineto{\pgfqpoint{1.533298in}{3.373559in}}%
\pgfpathlineto{\pgfqpoint{1.534276in}{3.370953in}}%
\pgfpathlineto{\pgfqpoint{1.535900in}{3.367629in}}%
\pgfpathlineto{\pgfqpoint{1.536547in}{3.368351in}}%
\pgfpathlineto{\pgfqpoint{1.538608in}{3.372272in}}%
\pgfpathlineto{\pgfqpoint{1.539506in}{3.370667in}}%
\pgfpathlineto{\pgfqpoint{1.541289in}{3.368326in}}%
\pgfpathlineto{\pgfqpoint{1.541870in}{3.368875in}}%
\pgfpathlineto{\pgfqpoint{1.544050in}{3.371405in}}%
\pgfpathlineto{\pgfqpoint{1.544816in}{3.370402in}}%
\pgfpathlineto{\pgfqpoint{1.546784in}{3.368906in}}%
\pgfpathlineto{\pgfqpoint{1.547259in}{3.369326in}}%
\pgfpathlineto{\pgfqpoint{1.549676in}{3.370677in}}%
\pgfpathlineto{\pgfqpoint{1.550112in}{3.370226in}}%
\pgfpathlineto{\pgfqpoint{1.552688in}{3.369642in}}%
\pgfpathlineto{\pgfqpoint{1.555514in}{3.370037in}}%
\pgfpathlineto{\pgfqpoint{1.558222in}{3.369892in}}%
\pgfpathlineto{\pgfqpoint{1.561075in}{3.369859in}}%
\pgfpathlineto{\pgfqpoint{1.564033in}{3.370120in}}%
\pgfpathlineto{\pgfqpoint{1.567481in}{3.369668in}}%
\pgfpathlineto{\pgfqpoint{1.839011in}{3.369936in}}%
\pgfpathlineto{\pgfqpoint{3.605600in}{3.368401in}}%
\pgfpathlineto{\pgfqpoint{3.606062in}{3.368339in}}%
\pgfpathlineto{\pgfqpoint{3.606617in}{3.369537in}}%
\pgfpathlineto{\pgfqpoint{3.607634in}{3.371722in}}%
\pgfpathlineto{\pgfqpoint{3.608479in}{3.370297in}}%
\pgfpathlineto{\pgfqpoint{3.611293in}{3.368533in}}%
\pgfpathlineto{\pgfqpoint{3.612283in}{3.370616in}}%
\pgfpathlineto{\pgfqpoint{3.612891in}{3.371347in}}%
\pgfpathlineto{\pgfqpoint{3.613710in}{3.370068in}}%
\pgfpathlineto{\pgfqpoint{3.616391in}{3.368750in}}%
\pgfpathlineto{\pgfqpoint{3.616932in}{3.369062in}}%
\pgfpathlineto{\pgfqpoint{3.617342in}{3.369981in}}%
\pgfpathlineto{\pgfqpoint{3.618148in}{3.371110in}}%
\pgfpathlineto{\pgfqpoint{3.618953in}{3.369906in}}%
\pgfpathlineto{\pgfqpoint{3.619587in}{3.369635in}}%
\pgfpathlineto{\pgfqpoint{3.620459in}{3.370417in}}%
\pgfpathlineto{\pgfqpoint{3.621635in}{3.368937in}}%
\pgfpathlineto{\pgfqpoint{3.622123in}{3.369010in}}%
\pgfpathlineto{\pgfqpoint{3.622665in}{3.370064in}}%
\pgfpathlineto{\pgfqpoint{3.623418in}{3.370954in}}%
\pgfpathlineto{\pgfqpoint{3.624210in}{3.369788in}}%
\pgfpathlineto{\pgfqpoint{3.624818in}{3.369530in}}%
\pgfpathlineto{\pgfqpoint{3.625623in}{3.370452in}}%
\pgfpathlineto{\pgfqpoint{3.626244in}{3.370131in}}%
\pgfpathlineto{\pgfqpoint{3.626706in}{3.369326in}}%
\pgfpathlineto{\pgfqpoint{3.627301in}{3.368987in}}%
\pgfpathlineto{\pgfqpoint{3.627988in}{3.370142in}}%
\pgfpathlineto{\pgfqpoint{3.628688in}{3.370854in}}%
\pgfpathlineto{\pgfqpoint{3.629480in}{3.369697in}}%
\pgfpathlineto{\pgfqpoint{3.630074in}{3.369455in}}%
\pgfpathlineto{\pgfqpoint{3.630841in}{3.370438in}}%
\pgfpathlineto{\pgfqpoint{3.631356in}{3.370446in}}%
\pgfpathlineto{\pgfqpoint{3.631990in}{3.369418in}}%
\pgfpathlineto{\pgfqpoint{3.632597in}{3.369054in}}%
\pgfpathlineto{\pgfqpoint{3.633310in}{3.370215in}}%
\pgfpathlineto{\pgfqpoint{3.633958in}{3.370790in}}%
\pgfpathlineto{\pgfqpoint{3.634737in}{3.369657in}}%
\pgfpathlineto{\pgfqpoint{3.635331in}{3.369393in}}%
\pgfpathlineto{\pgfqpoint{3.636084in}{3.370414in}}%
\pgfpathlineto{\pgfqpoint{3.636612in}{3.370528in}}%
\pgfpathlineto{\pgfqpoint{3.637286in}{3.369462in}}%
\pgfpathlineto{\pgfqpoint{3.637894in}{3.369105in}}%
\pgfpathlineto{\pgfqpoint{3.638607in}{3.370236in}}%
\pgfpathlineto{\pgfqpoint{3.639241in}{3.370744in}}%
\pgfpathlineto{\pgfqpoint{3.640007in}{3.369621in}}%
\pgfpathlineto{\pgfqpoint{3.640601in}{3.369350in}}%
\pgfpathlineto{\pgfqpoint{3.641341in}{3.370390in}}%
\pgfpathlineto{\pgfqpoint{3.641882in}{3.370577in}}%
\pgfpathlineto{\pgfqpoint{3.642582in}{3.369487in}}%
\pgfpathlineto{\pgfqpoint{3.643190in}{3.369144in}}%
\pgfpathlineto{\pgfqpoint{3.643903in}{3.370258in}}%
\pgfpathlineto{\pgfqpoint{3.644524in}{3.370713in}}%
\pgfpathlineto{\pgfqpoint{3.645290in}{3.369583in}}%
\pgfpathlineto{\pgfqpoint{3.645884in}{3.369325in}}%
\pgfpathlineto{\pgfqpoint{3.646611in}{3.370376in}}%
\pgfpathlineto{\pgfqpoint{3.647152in}{3.370611in}}%
\pgfpathlineto{\pgfqpoint{3.647879in}{3.369498in}}%
\pgfpathlineto{\pgfqpoint{3.648486in}{3.369175in}}%
\pgfpathlineto{\pgfqpoint{3.649200in}{3.370281in}}%
\pgfpathlineto{\pgfqpoint{3.649807in}{3.370692in}}%
\pgfpathlineto{\pgfqpoint{3.650560in}{3.369575in}}%
\pgfpathlineto{\pgfqpoint{3.651155in}{3.369298in}}%
\pgfpathlineto{\pgfqpoint{3.651881in}{3.370358in}}%
\pgfpathlineto{\pgfqpoint{3.652436in}{3.370628in}}%
\pgfpathlineto{\pgfqpoint{3.653162in}{3.369520in}}%
\pgfpathlineto{\pgfqpoint{3.653770in}{3.369191in}}%
\pgfpathlineto{\pgfqpoint{3.654483in}{3.370282in}}%
\pgfpathlineto{\pgfqpoint{3.655090in}{3.370678in}}%
\pgfpathlineto{\pgfqpoint{3.655843in}{3.369556in}}%
\pgfpathlineto{\pgfqpoint{3.656438in}{3.369285in}}%
\pgfpathlineto{\pgfqpoint{3.657164in}{3.370358in}}%
\pgfpathlineto{\pgfqpoint{3.657719in}{3.370639in}}%
\pgfpathlineto{\pgfqpoint{3.658445in}{3.369536in}}%
\pgfpathlineto{\pgfqpoint{3.659053in}{3.369202in}}%
\pgfpathlineto{\pgfqpoint{3.659779in}{3.370305in}}%
\pgfpathlineto{\pgfqpoint{3.660374in}{3.370669in}}%
\pgfpathlineto{\pgfqpoint{3.661113in}{3.369563in}}%
\pgfpathlineto{\pgfqpoint{3.661708in}{3.369266in}}%
\pgfpathlineto{\pgfqpoint{3.662434in}{3.370338in}}%
\pgfpathlineto{\pgfqpoint{3.663002in}{3.370646in}}%
\pgfpathlineto{\pgfqpoint{3.663742in}{3.369527in}}%
\pgfpathlineto{\pgfqpoint{3.664336in}{3.369211in}}%
\pgfpathlineto{\pgfqpoint{3.665063in}{3.370306in}}%
\pgfpathlineto{\pgfqpoint{3.665657in}{3.370663in}}%
\pgfpathlineto{\pgfqpoint{3.666397in}{3.369554in}}%
\pgfpathlineto{\pgfqpoint{3.666991in}{3.369259in}}%
\pgfpathlineto{\pgfqpoint{3.667717in}{3.370337in}}%
\pgfpathlineto{\pgfqpoint{3.668285in}{3.370651in}}%
\pgfpathlineto{\pgfqpoint{3.669025in}{3.369534in}}%
\pgfpathlineto{\pgfqpoint{3.669619in}{3.369217in}}%
\pgfpathlineto{\pgfqpoint{3.670346in}{3.370307in}}%
\pgfpathlineto{\pgfqpoint{3.670927in}{3.370667in}}%
\pgfpathlineto{\pgfqpoint{3.671680in}{3.369547in}}%
\pgfpathlineto{\pgfqpoint{3.672274in}{3.369254in}}%
\pgfpathlineto{\pgfqpoint{3.673001in}{3.370336in}}%
\pgfpathlineto{\pgfqpoint{3.673569in}{3.370654in}}%
\pgfpathlineto{\pgfqpoint{3.674308in}{3.369540in}}%
\pgfpathlineto{\pgfqpoint{3.674916in}{3.369230in}}%
\pgfpathlineto{\pgfqpoint{3.675629in}{3.370308in}}%
\pgfpathlineto{\pgfqpoint{3.676210in}{3.370665in}}%
\pgfpathlineto{\pgfqpoint{3.676963in}{3.369543in}}%
\pgfpathlineto{\pgfqpoint{3.677557in}{3.369250in}}%
\pgfpathlineto{\pgfqpoint{3.678284in}{3.370335in}}%
\pgfpathlineto{\pgfqpoint{3.678852in}{3.370656in}}%
\pgfpathlineto{\pgfqpoint{3.679591in}{3.369544in}}%
\pgfpathlineto{\pgfqpoint{3.680199in}{3.369233in}}%
\pgfpathlineto{\pgfqpoint{3.680912in}{3.370309in}}%
\pgfpathlineto{\pgfqpoint{3.681493in}{3.370663in}}%
\pgfpathlineto{\pgfqpoint{3.682246in}{3.369539in}}%
\pgfpathlineto{\pgfqpoint{3.682841in}{3.369247in}}%
\pgfpathlineto{\pgfqpoint{3.683567in}{3.370335in}}%
\pgfpathlineto{\pgfqpoint{3.684135in}{3.370657in}}%
\pgfpathlineto{\pgfqpoint{3.684875in}{3.369546in}}%
\pgfpathlineto{\pgfqpoint{3.685482in}{3.369235in}}%
\pgfpathlineto{\pgfqpoint{3.686195in}{3.370309in}}%
\pgfpathlineto{\pgfqpoint{3.686777in}{3.370662in}}%
\pgfpathlineto{\pgfqpoint{3.687529in}{3.369537in}}%
\pgfpathlineto{\pgfqpoint{3.688124in}{3.369245in}}%
\pgfpathlineto{\pgfqpoint{3.688850in}{3.370334in}}%
\pgfpathlineto{\pgfqpoint{3.689418in}{3.370658in}}%
\pgfpathlineto{\pgfqpoint{3.690158in}{3.369548in}}%
\pgfpathlineto{\pgfqpoint{3.690765in}{3.369237in}}%
\pgfpathlineto{\pgfqpoint{3.691479in}{3.370310in}}%
\pgfpathlineto{\pgfqpoint{3.692060in}{3.370661in}}%
\pgfpathlineto{\pgfqpoint{3.692813in}{3.369536in}}%
\pgfpathlineto{\pgfqpoint{3.693407in}{3.369244in}}%
\pgfpathlineto{\pgfqpoint{3.694133in}{3.370334in}}%
\pgfpathlineto{\pgfqpoint{3.694701in}{3.370659in}}%
\pgfpathlineto{\pgfqpoint{3.695441in}{3.369549in}}%
\pgfpathlineto{\pgfqpoint{3.696049in}{3.369238in}}%
\pgfpathlineto{\pgfqpoint{3.696762in}{3.370310in}}%
\pgfpathlineto{\pgfqpoint{3.697343in}{3.370661in}}%
\pgfpathlineto{\pgfqpoint{3.698096in}{3.369535in}}%
\pgfpathlineto{\pgfqpoint{3.698690in}{3.369243in}}%
\pgfpathlineto{\pgfqpoint{3.699417in}{3.370333in}}%
\pgfpathlineto{\pgfqpoint{3.699985in}{3.370659in}}%
\pgfpathlineto{\pgfqpoint{3.700724in}{3.369550in}}%
\pgfpathlineto{\pgfqpoint{3.701332in}{3.369239in}}%
\pgfpathlineto{\pgfqpoint{3.702045in}{3.370310in}}%
\pgfpathlineto{\pgfqpoint{3.702626in}{3.370660in}}%
\pgfpathlineto{\pgfqpoint{3.703379in}{3.369534in}}%
\pgfpathlineto{\pgfqpoint{3.703973in}{3.369242in}}%
\pgfpathlineto{\pgfqpoint{3.704700in}{3.370333in}}%
\pgfpathlineto{\pgfqpoint{3.705268in}{3.370660in}}%
\pgfpathlineto{\pgfqpoint{3.706007in}{3.369551in}}%
\pgfpathlineto{\pgfqpoint{3.706615in}{3.369239in}}%
\pgfpathlineto{\pgfqpoint{3.707328in}{3.370311in}}%
\pgfpathlineto{\pgfqpoint{3.707909in}{3.370660in}}%
\pgfpathlineto{\pgfqpoint{3.708662in}{3.369533in}}%
\pgfpathlineto{\pgfqpoint{3.709257in}{3.369242in}}%
\pgfpathlineto{\pgfqpoint{3.709983in}{3.370333in}}%
\pgfpathlineto{\pgfqpoint{3.710551in}{3.370660in}}%
\pgfpathlineto{\pgfqpoint{3.711291in}{3.369551in}}%
\pgfpathlineto{\pgfqpoint{3.711898in}{3.369240in}}%
\pgfpathlineto{\pgfqpoint{3.712612in}{3.370311in}}%
\pgfpathlineto{\pgfqpoint{3.713193in}{3.370660in}}%
\pgfpathlineto{\pgfqpoint{3.713932in}{3.369553in}}%
\pgfpathlineto{\pgfqpoint{3.714540in}{3.369242in}}%
\pgfpathlineto{\pgfqpoint{3.715266in}{3.370333in}}%
\pgfpathlineto{\pgfqpoint{3.715834in}{3.370660in}}%
\pgfpathlineto{\pgfqpoint{3.716574in}{3.369552in}}%
\pgfpathlineto{\pgfqpoint{3.717181in}{3.369240in}}%
\pgfpathlineto{\pgfqpoint{3.717895in}{3.370311in}}%
\pgfpathlineto{\pgfqpoint{3.718476in}{3.370660in}}%
\pgfpathlineto{\pgfqpoint{3.719216in}{3.369553in}}%
\pgfpathlineto{\pgfqpoint{3.719823in}{3.369241in}}%
\pgfpathlineto{\pgfqpoint{3.720550in}{3.370333in}}%
\pgfpathlineto{\pgfqpoint{3.721131in}{3.370651in}}%
\pgfpathlineto{\pgfqpoint{3.721857in}{3.369552in}}%
\pgfpathlineto{\pgfqpoint{3.722465in}{3.369240in}}%
\pgfpathlineto{\pgfqpoint{3.723178in}{3.370311in}}%
\pgfpathlineto{\pgfqpoint{3.723759in}{3.370660in}}%
\pgfpathlineto{\pgfqpoint{3.724499in}{3.369553in}}%
\pgfpathlineto{\pgfqpoint{3.725106in}{3.369241in}}%
\pgfpathlineto{\pgfqpoint{3.725833in}{3.370332in}}%
\pgfpathlineto{\pgfqpoint{3.726414in}{3.370651in}}%
\pgfpathlineto{\pgfqpoint{3.727140in}{3.369552in}}%
\pgfpathlineto{\pgfqpoint{3.727748in}{3.369240in}}%
\pgfpathlineto{\pgfqpoint{3.728461in}{3.370311in}}%
\pgfpathlineto{\pgfqpoint{3.729042in}{3.370660in}}%
\pgfpathlineto{\pgfqpoint{3.729782in}{3.369553in}}%
\pgfpathlineto{\pgfqpoint{3.730390in}{3.369241in}}%
\pgfpathlineto{\pgfqpoint{3.731116in}{3.370332in}}%
\pgfpathlineto{\pgfqpoint{3.731697in}{3.370651in}}%
\pgfpathlineto{\pgfqpoint{3.732424in}{3.369552in}}%
\pgfpathlineto{\pgfqpoint{3.733031in}{3.369241in}}%
\pgfpathlineto{\pgfqpoint{3.733758in}{3.370332in}}%
\pgfpathlineto{\pgfqpoint{3.734339in}{3.370651in}}%
\pgfpathlineto{\pgfqpoint{3.735065in}{3.369552in}}%
\pgfpathlineto{\pgfqpoint{3.735673in}{3.369241in}}%
\pgfpathlineto{\pgfqpoint{3.736399in}{3.370332in}}%
\pgfpathlineto{\pgfqpoint{3.736980in}{3.370651in}}%
\pgfpathlineto{\pgfqpoint{3.737707in}{3.369552in}}%
\pgfpathlineto{\pgfqpoint{3.738314in}{3.369241in}}%
\pgfpathlineto{\pgfqpoint{3.739041in}{3.370332in}}%
\pgfpathlineto{\pgfqpoint{3.739622in}{3.370651in}}%
\pgfpathlineto{\pgfqpoint{3.740348in}{3.369552in}}%
\pgfpathlineto{\pgfqpoint{3.740956in}{3.369241in}}%
\pgfpathlineto{\pgfqpoint{3.741682in}{3.370332in}}%
\pgfpathlineto{\pgfqpoint{3.742264in}{3.370651in}}%
\pgfpathlineto{\pgfqpoint{3.742990in}{3.369552in}}%
\pgfpathlineto{\pgfqpoint{3.743598in}{3.369241in}}%
\pgfpathlineto{\pgfqpoint{3.744324in}{3.370332in}}%
\pgfpathlineto{\pgfqpoint{3.744905in}{3.370651in}}%
\pgfpathlineto{\pgfqpoint{3.745632in}{3.369552in}}%
\pgfpathlineto{\pgfqpoint{3.746239in}{3.369241in}}%
\pgfpathlineto{\pgfqpoint{3.746966in}{3.370332in}}%
\pgfpathlineto{\pgfqpoint{3.747547in}{3.370651in}}%
\pgfpathlineto{\pgfqpoint{3.748273in}{3.369552in}}%
\pgfpathlineto{\pgfqpoint{3.748881in}{3.369241in}}%
\pgfpathlineto{\pgfqpoint{3.749607in}{3.370332in}}%
\pgfpathlineto{\pgfqpoint{3.750188in}{3.370651in}}%
\pgfpathlineto{\pgfqpoint{3.750915in}{3.369552in}}%
\pgfpathlineto{\pgfqpoint{3.751522in}{3.369241in}}%
\pgfpathlineto{\pgfqpoint{3.752249in}{3.370332in}}%
\pgfpathlineto{\pgfqpoint{3.752830in}{3.370651in}}%
\pgfpathlineto{\pgfqpoint{3.753556in}{3.369552in}}%
\pgfpathlineto{\pgfqpoint{3.754164in}{3.369241in}}%
\pgfpathlineto{\pgfqpoint{3.754890in}{3.370332in}}%
\pgfpathlineto{\pgfqpoint{3.755472in}{3.370651in}}%
\pgfpathlineto{\pgfqpoint{3.756198in}{3.369552in}}%
\pgfpathlineto{\pgfqpoint{3.756806in}{3.369241in}}%
\pgfpathlineto{\pgfqpoint{3.757532in}{3.370332in}}%
\pgfpathlineto{\pgfqpoint{3.758113in}{3.370651in}}%
\pgfpathlineto{\pgfqpoint{3.758840in}{3.369552in}}%
\pgfpathlineto{\pgfqpoint{3.759447in}{3.369241in}}%
\pgfpathlineto{\pgfqpoint{3.760174in}{3.370332in}}%
\pgfpathlineto{\pgfqpoint{3.760755in}{3.370651in}}%
\pgfpathlineto{\pgfqpoint{3.761481in}{3.369552in}}%
\pgfpathlineto{\pgfqpoint{3.762089in}{3.369241in}}%
\pgfpathlineto{\pgfqpoint{3.762815in}{3.370332in}}%
\pgfpathlineto{\pgfqpoint{3.763396in}{3.370651in}}%
\pgfpathlineto{\pgfqpoint{3.764123in}{3.369552in}}%
\pgfpathlineto{\pgfqpoint{3.764730in}{3.369241in}}%
\pgfpathlineto{\pgfqpoint{3.765457in}{3.370332in}}%
\pgfpathlineto{\pgfqpoint{3.766038in}{3.370651in}}%
\pgfpathlineto{\pgfqpoint{3.766764in}{3.369552in}}%
\pgfpathlineto{\pgfqpoint{3.767372in}{3.369241in}}%
\pgfpathlineto{\pgfqpoint{3.768098in}{3.370332in}}%
\pgfpathlineto{\pgfqpoint{3.768680in}{3.370651in}}%
\pgfpathlineto{\pgfqpoint{3.769406in}{3.369552in}}%
\pgfpathlineto{\pgfqpoint{3.770014in}{3.369241in}}%
\pgfpathlineto{\pgfqpoint{3.770740in}{3.370332in}}%
\pgfpathlineto{\pgfqpoint{3.771321in}{3.370651in}}%
\pgfpathlineto{\pgfqpoint{3.772048in}{3.369552in}}%
\pgfpathlineto{\pgfqpoint{3.772655in}{3.369241in}}%
\pgfpathlineto{\pgfqpoint{3.773382in}{3.370332in}}%
\pgfpathlineto{\pgfqpoint{3.773963in}{3.370651in}}%
\pgfpathlineto{\pgfqpoint{3.774689in}{3.369552in}}%
\pgfpathlineto{\pgfqpoint{3.775297in}{3.369241in}}%
\pgfpathlineto{\pgfqpoint{3.776023in}{3.370332in}}%
\pgfpathlineto{\pgfqpoint{3.776604in}{3.370651in}}%
\pgfpathlineto{\pgfqpoint{3.777331in}{3.369552in}}%
\pgfpathlineto{\pgfqpoint{3.777938in}{3.369241in}}%
\pgfpathlineto{\pgfqpoint{3.778665in}{3.370332in}}%
\pgfpathlineto{\pgfqpoint{3.779246in}{3.370651in}}%
\pgfpathlineto{\pgfqpoint{3.779973in}{3.369552in}}%
\pgfpathlineto{\pgfqpoint{3.780580in}{3.369241in}}%
\pgfpathlineto{\pgfqpoint{3.781307in}{3.370332in}}%
\pgfpathlineto{\pgfqpoint{3.781888in}{3.370651in}}%
\pgfpathlineto{\pgfqpoint{3.782614in}{3.369552in}}%
\pgfpathlineto{\pgfqpoint{3.783222in}{3.369241in}}%
\pgfpathlineto{\pgfqpoint{3.783948in}{3.370332in}}%
\pgfpathlineto{\pgfqpoint{3.784529in}{3.370651in}}%
\pgfpathlineto{\pgfqpoint{3.785256in}{3.369552in}}%
\pgfpathlineto{\pgfqpoint{3.785863in}{3.369241in}}%
\pgfpathlineto{\pgfqpoint{3.786590in}{3.370332in}}%
\pgfpathlineto{\pgfqpoint{3.787171in}{3.370651in}}%
\pgfpathlineto{\pgfqpoint{3.787897in}{3.369552in}}%
\pgfpathlineto{\pgfqpoint{3.788505in}{3.369241in}}%
\pgfpathlineto{\pgfqpoint{3.789231in}{3.370332in}}%
\pgfpathlineto{\pgfqpoint{3.789813in}{3.370651in}}%
\pgfpathlineto{\pgfqpoint{3.790539in}{3.369552in}}%
\pgfpathlineto{\pgfqpoint{3.791147in}{3.369241in}}%
\pgfpathlineto{\pgfqpoint{3.791873in}{3.370332in}}%
\pgfpathlineto{\pgfqpoint{3.792454in}{3.370651in}}%
\pgfpathlineto{\pgfqpoint{3.793181in}{3.369552in}}%
\pgfpathlineto{\pgfqpoint{3.793788in}{3.369241in}}%
\pgfpathlineto{\pgfqpoint{3.794515in}{3.370332in}}%
\pgfpathlineto{\pgfqpoint{3.795096in}{3.370651in}}%
\pgfpathlineto{\pgfqpoint{3.795822in}{3.369552in}}%
\pgfpathlineto{\pgfqpoint{3.796430in}{3.369241in}}%
\pgfpathlineto{\pgfqpoint{3.797156in}{3.370332in}}%
\pgfpathlineto{\pgfqpoint{3.797737in}{3.370651in}}%
\pgfpathlineto{\pgfqpoint{3.798464in}{3.369552in}}%
\pgfpathlineto{\pgfqpoint{3.799071in}{3.369241in}}%
\pgfpathlineto{\pgfqpoint{3.799798in}{3.370332in}}%
\pgfpathlineto{\pgfqpoint{3.800379in}{3.370651in}}%
\pgfpathlineto{\pgfqpoint{3.801105in}{3.369552in}}%
\pgfpathlineto{\pgfqpoint{3.801713in}{3.369241in}}%
\pgfpathlineto{\pgfqpoint{3.802439in}{3.370332in}}%
\pgfpathlineto{\pgfqpoint{3.803021in}{3.370651in}}%
\pgfpathlineto{\pgfqpoint{3.803747in}{3.369552in}}%
\pgfpathlineto{\pgfqpoint{3.804355in}{3.369241in}}%
\pgfpathlineto{\pgfqpoint{3.805081in}{3.370332in}}%
\pgfpathlineto{\pgfqpoint{3.805662in}{3.370651in}}%
\pgfpathlineto{\pgfqpoint{3.806389in}{3.369552in}}%
\pgfpathlineto{\pgfqpoint{3.806996in}{3.369241in}}%
\pgfpathlineto{\pgfqpoint{3.807723in}{3.370332in}}%
\pgfpathlineto{\pgfqpoint{3.808304in}{3.370651in}}%
\pgfpathlineto{\pgfqpoint{3.809030in}{3.369552in}}%
\pgfpathlineto{\pgfqpoint{3.809638in}{3.369241in}}%
\pgfpathlineto{\pgfqpoint{3.810364in}{3.370332in}}%
\pgfpathlineto{\pgfqpoint{3.810945in}{3.370651in}}%
\pgfpathlineto{\pgfqpoint{3.811672in}{3.369552in}}%
\pgfpathlineto{\pgfqpoint{3.812279in}{3.369241in}}%
\pgfpathlineto{\pgfqpoint{3.813006in}{3.370332in}}%
\pgfpathlineto{\pgfqpoint{3.813587in}{3.370651in}}%
\pgfpathlineto{\pgfqpoint{3.814313in}{3.369552in}}%
\pgfpathlineto{\pgfqpoint{3.814921in}{3.369241in}}%
\pgfpathlineto{\pgfqpoint{3.815647in}{3.370332in}}%
\pgfpathlineto{\pgfqpoint{3.816229in}{3.370651in}}%
\pgfpathlineto{\pgfqpoint{3.816955in}{3.369552in}}%
\pgfpathlineto{\pgfqpoint{3.817563in}{3.369241in}}%
\pgfpathlineto{\pgfqpoint{3.818289in}{3.370332in}}%
\pgfpathlineto{\pgfqpoint{3.818870in}{3.370651in}}%
\pgfpathlineto{\pgfqpoint{3.819597in}{3.369552in}}%
\pgfpathlineto{\pgfqpoint{3.820204in}{3.369241in}}%
\pgfpathlineto{\pgfqpoint{3.820931in}{3.370332in}}%
\pgfpathlineto{\pgfqpoint{3.821512in}{3.370651in}}%
\pgfpathlineto{\pgfqpoint{3.822238in}{3.369552in}}%
\pgfpathlineto{\pgfqpoint{3.822846in}{3.369241in}}%
\pgfpathlineto{\pgfqpoint{3.823572in}{3.370332in}}%
\pgfpathlineto{\pgfqpoint{3.824153in}{3.370651in}}%
\pgfpathlineto{\pgfqpoint{3.824880in}{3.369552in}}%
\pgfpathlineto{\pgfqpoint{3.825487in}{3.369241in}}%
\pgfpathlineto{\pgfqpoint{3.826214in}{3.370332in}}%
\pgfpathlineto{\pgfqpoint{3.826795in}{3.370651in}}%
\pgfpathlineto{\pgfqpoint{3.827521in}{3.369552in}}%
\pgfpathlineto{\pgfqpoint{3.828129in}{3.369241in}}%
\pgfpathlineto{\pgfqpoint{3.828855in}{3.370332in}}%
\pgfpathlineto{\pgfqpoint{3.829437in}{3.370651in}}%
\pgfpathlineto{\pgfqpoint{3.830163in}{3.369552in}}%
\pgfpathlineto{\pgfqpoint{3.830771in}{3.369241in}}%
\pgfpathlineto{\pgfqpoint{3.831497in}{3.370332in}}%
\pgfpathlineto{\pgfqpoint{3.832078in}{3.370651in}}%
\pgfpathlineto{\pgfqpoint{3.832805in}{3.369552in}}%
\pgfpathlineto{\pgfqpoint{3.833412in}{3.369241in}}%
\pgfpathlineto{\pgfqpoint{3.834139in}{3.370332in}}%
\pgfpathlineto{\pgfqpoint{3.834720in}{3.370651in}}%
\pgfpathlineto{\pgfqpoint{3.835446in}{3.369552in}}%
\pgfpathlineto{\pgfqpoint{3.836054in}{3.369241in}}%
\pgfpathlineto{\pgfqpoint{3.836780in}{3.370332in}}%
\pgfpathlineto{\pgfqpoint{3.837361in}{3.370651in}}%
\pgfpathlineto{\pgfqpoint{3.838088in}{3.369552in}}%
\pgfpathlineto{\pgfqpoint{3.838695in}{3.369241in}}%
\pgfpathlineto{\pgfqpoint{3.839422in}{3.370332in}}%
\pgfpathlineto{\pgfqpoint{3.840003in}{3.370651in}}%
\pgfpathlineto{\pgfqpoint{3.840729in}{3.369552in}}%
\pgfpathlineto{\pgfqpoint{3.841337in}{3.369241in}}%
\pgfpathlineto{\pgfqpoint{3.842064in}{3.370332in}}%
\pgfpathlineto{\pgfqpoint{3.842645in}{3.370651in}}%
\pgfpathlineto{\pgfqpoint{3.843371in}{3.369552in}}%
\pgfpathlineto{\pgfqpoint{3.843979in}{3.369241in}}%
\pgfpathlineto{\pgfqpoint{3.844705in}{3.370332in}}%
\pgfpathlineto{\pgfqpoint{3.845286in}{3.370651in}}%
\pgfpathlineto{\pgfqpoint{3.846013in}{3.369552in}}%
\pgfpathlineto{\pgfqpoint{3.846620in}{3.369241in}}%
\pgfpathlineto{\pgfqpoint{3.847347in}{3.370332in}}%
\pgfpathlineto{\pgfqpoint{3.847928in}{3.370651in}}%
\pgfpathlineto{\pgfqpoint{3.848654in}{3.369552in}}%
\pgfpathlineto{\pgfqpoint{3.849262in}{3.369241in}}%
\pgfpathlineto{\pgfqpoint{3.849988in}{3.370332in}}%
\pgfpathlineto{\pgfqpoint{3.850569in}{3.370651in}}%
\pgfpathlineto{\pgfqpoint{3.851296in}{3.369552in}}%
\pgfpathlineto{\pgfqpoint{3.851903in}{3.369241in}}%
\pgfpathlineto{\pgfqpoint{3.852630in}{3.370332in}}%
\pgfpathlineto{\pgfqpoint{3.853211in}{3.370651in}}%
\pgfpathlineto{\pgfqpoint{3.853938in}{3.369552in}}%
\pgfpathlineto{\pgfqpoint{3.854545in}{3.369241in}}%
\pgfpathlineto{\pgfqpoint{3.855272in}{3.370332in}}%
\pgfpathlineto{\pgfqpoint{3.855853in}{3.370651in}}%
\pgfpathlineto{\pgfqpoint{3.856579in}{3.369552in}}%
\pgfpathlineto{\pgfqpoint{3.857187in}{3.369241in}}%
\pgfpathlineto{\pgfqpoint{3.857913in}{3.370332in}}%
\pgfpathlineto{\pgfqpoint{3.858494in}{3.370651in}}%
\pgfpathlineto{\pgfqpoint{3.859221in}{3.369552in}}%
\pgfpathlineto{\pgfqpoint{3.859828in}{3.369241in}}%
\pgfpathlineto{\pgfqpoint{3.860555in}{3.370332in}}%
\pgfpathlineto{\pgfqpoint{3.861136in}{3.370651in}}%
\pgfpathlineto{\pgfqpoint{3.861862in}{3.369552in}}%
\pgfpathlineto{\pgfqpoint{3.862470in}{3.369241in}}%
\pgfpathlineto{\pgfqpoint{3.863196in}{3.370332in}}%
\pgfpathlineto{\pgfqpoint{3.863778in}{3.370651in}}%
\pgfpathlineto{\pgfqpoint{3.864504in}{3.369552in}}%
\pgfpathlineto{\pgfqpoint{3.865112in}{3.369241in}}%
\pgfpathlineto{\pgfqpoint{3.865838in}{3.370332in}}%
\pgfpathlineto{\pgfqpoint{3.866419in}{3.370651in}}%
\pgfpathlineto{\pgfqpoint{3.867146in}{3.369552in}}%
\pgfpathlineto{\pgfqpoint{3.867753in}{3.369241in}}%
\pgfpathlineto{\pgfqpoint{3.868480in}{3.370332in}}%
\pgfpathlineto{\pgfqpoint{3.869061in}{3.370651in}}%
\pgfpathlineto{\pgfqpoint{3.869787in}{3.369552in}}%
\pgfpathlineto{\pgfqpoint{3.870395in}{3.369241in}}%
\pgfpathlineto{\pgfqpoint{3.871121in}{3.370332in}}%
\pgfpathlineto{\pgfqpoint{3.871702in}{3.370651in}}%
\pgfpathlineto{\pgfqpoint{3.872429in}{3.369552in}}%
\pgfpathlineto{\pgfqpoint{3.873036in}{3.369241in}}%
\pgfpathlineto{\pgfqpoint{3.873763in}{3.370332in}}%
\pgfpathlineto{\pgfqpoint{3.874344in}{3.370651in}}%
\pgfpathlineto{\pgfqpoint{3.875070in}{3.369552in}}%
\pgfpathlineto{\pgfqpoint{3.875678in}{3.369241in}}%
\pgfpathlineto{\pgfqpoint{3.876404in}{3.370332in}}%
\pgfpathlineto{\pgfqpoint{3.876986in}{3.370651in}}%
\pgfpathlineto{\pgfqpoint{3.877712in}{3.369552in}}%
\pgfpathlineto{\pgfqpoint{3.878320in}{3.369241in}}%
\pgfpathlineto{\pgfqpoint{3.879046in}{3.370332in}}%
\pgfpathlineto{\pgfqpoint{3.879627in}{3.370651in}}%
\pgfpathlineto{\pgfqpoint{3.880354in}{3.369552in}}%
\pgfpathlineto{\pgfqpoint{3.880961in}{3.369241in}}%
\pgfpathlineto{\pgfqpoint{3.881688in}{3.370332in}}%
\pgfpathlineto{\pgfqpoint{3.882269in}{3.370651in}}%
\pgfpathlineto{\pgfqpoint{3.882995in}{3.369552in}}%
\pgfpathlineto{\pgfqpoint{3.883603in}{3.369241in}}%
\pgfpathlineto{\pgfqpoint{3.884329in}{3.370332in}}%
\pgfpathlineto{\pgfqpoint{3.884910in}{3.370651in}}%
\pgfpathlineto{\pgfqpoint{3.885637in}{3.369552in}}%
\pgfpathlineto{\pgfqpoint{3.886244in}{3.369241in}}%
\pgfpathlineto{\pgfqpoint{3.886971in}{3.370332in}}%
\pgfpathlineto{\pgfqpoint{3.887552in}{3.370651in}}%
\pgfpathlineto{\pgfqpoint{3.888278in}{3.369552in}}%
\pgfpathlineto{\pgfqpoint{3.888886in}{3.369241in}}%
\pgfpathlineto{\pgfqpoint{3.889612in}{3.370332in}}%
\pgfpathlineto{\pgfqpoint{3.890194in}{3.370651in}}%
\pgfpathlineto{\pgfqpoint{3.890920in}{3.369552in}}%
\pgfpathlineto{\pgfqpoint{3.891528in}{3.369241in}}%
\pgfpathlineto{\pgfqpoint{3.892254in}{3.370332in}}%
\pgfpathlineto{\pgfqpoint{3.892835in}{3.370651in}}%
\pgfpathlineto{\pgfqpoint{3.893562in}{3.369552in}}%
\pgfpathlineto{\pgfqpoint{3.894169in}{3.369241in}}%
\pgfpathlineto{\pgfqpoint{3.894896in}{3.370332in}}%
\pgfpathlineto{\pgfqpoint{3.895477in}{3.370651in}}%
\pgfpathlineto{\pgfqpoint{3.896203in}{3.369552in}}%
\pgfpathlineto{\pgfqpoint{3.896811in}{3.369241in}}%
\pgfpathlineto{\pgfqpoint{3.897537in}{3.370332in}}%
\pgfpathlineto{\pgfqpoint{3.898118in}{3.370651in}}%
\pgfpathlineto{\pgfqpoint{3.898845in}{3.369552in}}%
\pgfpathlineto{\pgfqpoint{3.899452in}{3.369241in}}%
\pgfpathlineto{\pgfqpoint{3.900179in}{3.370332in}}%
\pgfpathlineto{\pgfqpoint{3.900760in}{3.370651in}}%
\pgfpathlineto{\pgfqpoint{3.901486in}{3.369552in}}%
\pgfpathlineto{\pgfqpoint{3.902094in}{3.369241in}}%
\pgfpathlineto{\pgfqpoint{3.902820in}{3.370332in}}%
\pgfpathlineto{\pgfqpoint{3.903402in}{3.370651in}}%
\pgfpathlineto{\pgfqpoint{3.904128in}{3.369552in}}%
\pgfpathlineto{\pgfqpoint{3.904736in}{3.369241in}}%
\pgfpathlineto{\pgfqpoint{3.905462in}{3.370332in}}%
\pgfpathlineto{\pgfqpoint{3.906043in}{3.370651in}}%
\pgfpathlineto{\pgfqpoint{3.906770in}{3.369552in}}%
\pgfpathlineto{\pgfqpoint{3.907377in}{3.369241in}}%
\pgfpathlineto{\pgfqpoint{3.908104in}{3.370332in}}%
\pgfpathlineto{\pgfqpoint{3.908685in}{3.370651in}}%
\pgfpathlineto{\pgfqpoint{3.909411in}{3.369552in}}%
\pgfpathlineto{\pgfqpoint{3.910019in}{3.369241in}}%
\pgfpathlineto{\pgfqpoint{3.910745in}{3.370332in}}%
\pgfpathlineto{\pgfqpoint{3.911326in}{3.370651in}}%
\pgfpathlineto{\pgfqpoint{3.912053in}{3.369552in}}%
\pgfpathlineto{\pgfqpoint{3.912660in}{3.369241in}}%
\pgfpathlineto{\pgfqpoint{3.913387in}{3.370332in}}%
\pgfpathlineto{\pgfqpoint{3.913968in}{3.370651in}}%
\pgfpathlineto{\pgfqpoint{3.914695in}{3.369552in}}%
\pgfpathlineto{\pgfqpoint{3.915302in}{3.369241in}}%
\pgfpathlineto{\pgfqpoint{3.916029in}{3.370332in}}%
\pgfpathlineto{\pgfqpoint{3.916610in}{3.370651in}}%
\pgfpathlineto{\pgfqpoint{3.917336in}{3.369552in}}%
\pgfpathlineto{\pgfqpoint{3.917944in}{3.369241in}}%
\pgfpathlineto{\pgfqpoint{3.918670in}{3.370332in}}%
\pgfpathlineto{\pgfqpoint{3.919251in}{3.370651in}}%
\pgfpathlineto{\pgfqpoint{3.919978in}{3.369552in}}%
\pgfpathlineto{\pgfqpoint{3.920585in}{3.369241in}}%
\pgfpathlineto{\pgfqpoint{3.921312in}{3.370332in}}%
\pgfpathlineto{\pgfqpoint{3.921893in}{3.370651in}}%
\pgfpathlineto{\pgfqpoint{3.922619in}{3.369552in}}%
\pgfpathlineto{\pgfqpoint{3.923227in}{3.369241in}}%
\pgfpathlineto{\pgfqpoint{3.923953in}{3.370332in}}%
\pgfpathlineto{\pgfqpoint{3.924535in}{3.370651in}}%
\pgfpathlineto{\pgfqpoint{3.925261in}{3.369552in}}%
\pgfpathlineto{\pgfqpoint{3.925869in}{3.369241in}}%
\pgfpathlineto{\pgfqpoint{3.926595in}{3.370332in}}%
\pgfpathlineto{\pgfqpoint{3.927176in}{3.370651in}}%
\pgfpathlineto{\pgfqpoint{3.927903in}{3.369552in}}%
\pgfpathlineto{\pgfqpoint{3.928510in}{3.369241in}}%
\pgfpathlineto{\pgfqpoint{3.929237in}{3.370332in}}%
\pgfpathlineto{\pgfqpoint{3.929818in}{3.370651in}}%
\pgfpathlineto{\pgfqpoint{3.930544in}{3.369552in}}%
\pgfpathlineto{\pgfqpoint{3.931152in}{3.369241in}}%
\pgfpathlineto{\pgfqpoint{3.931878in}{3.370332in}}%
\pgfpathlineto{\pgfqpoint{3.932459in}{3.370651in}}%
\pgfpathlineto{\pgfqpoint{3.933186in}{3.369552in}}%
\pgfpathlineto{\pgfqpoint{3.933793in}{3.369241in}}%
\pgfpathlineto{\pgfqpoint{3.934520in}{3.370332in}}%
\pgfpathlineto{\pgfqpoint{3.935101in}{3.370651in}}%
\pgfpathlineto{\pgfqpoint{3.935827in}{3.369552in}}%
\pgfpathlineto{\pgfqpoint{3.936435in}{3.369241in}}%
\pgfpathlineto{\pgfqpoint{3.937161in}{3.370332in}}%
\pgfpathlineto{\pgfqpoint{3.937743in}{3.370651in}}%
\pgfpathlineto{\pgfqpoint{3.938469in}{3.369552in}}%
\pgfpathlineto{\pgfqpoint{3.939077in}{3.369241in}}%
\pgfpathlineto{\pgfqpoint{3.939803in}{3.370332in}}%
\pgfpathlineto{\pgfqpoint{3.940384in}{3.370651in}}%
\pgfpathlineto{\pgfqpoint{3.941111in}{3.369552in}}%
\pgfpathlineto{\pgfqpoint{3.941718in}{3.369241in}}%
\pgfpathlineto{\pgfqpoint{3.942445in}{3.370332in}}%
\pgfpathlineto{\pgfqpoint{3.943026in}{3.370651in}}%
\pgfpathlineto{\pgfqpoint{3.943752in}{3.369552in}}%
\pgfpathlineto{\pgfqpoint{3.944360in}{3.369241in}}%
\pgfpathlineto{\pgfqpoint{3.945086in}{3.370332in}}%
\pgfpathlineto{\pgfqpoint{3.945667in}{3.370651in}}%
\pgfpathlineto{\pgfqpoint{3.946394in}{3.369552in}}%
\pgfpathlineto{\pgfqpoint{3.947001in}{3.369241in}}%
\pgfpathlineto{\pgfqpoint{3.947728in}{3.370332in}}%
\pgfpathlineto{\pgfqpoint{3.948309in}{3.370651in}}%
\pgfpathlineto{\pgfqpoint{3.949035in}{3.369552in}}%
\pgfpathlineto{\pgfqpoint{3.949643in}{3.369241in}}%
\pgfpathlineto{\pgfqpoint{3.950369in}{3.370332in}}%
\pgfpathlineto{\pgfqpoint{3.950951in}{3.370651in}}%
\pgfpathlineto{\pgfqpoint{3.951677in}{3.369552in}}%
\pgfpathlineto{\pgfqpoint{3.952285in}{3.369241in}}%
\pgfpathlineto{\pgfqpoint{3.953011in}{3.370332in}}%
\pgfpathlineto{\pgfqpoint{3.953592in}{3.370651in}}%
\pgfpathlineto{\pgfqpoint{3.954319in}{3.369552in}}%
\pgfpathlineto{\pgfqpoint{3.954926in}{3.369241in}}%
\pgfpathlineto{\pgfqpoint{3.955653in}{3.370332in}}%
\pgfpathlineto{\pgfqpoint{3.956234in}{3.370651in}}%
\pgfpathlineto{\pgfqpoint{3.956960in}{3.369552in}}%
\pgfpathlineto{\pgfqpoint{3.957568in}{3.369241in}}%
\pgfpathlineto{\pgfqpoint{3.958294in}{3.370332in}}%
\pgfpathlineto{\pgfqpoint{3.958875in}{3.370651in}}%
\pgfpathlineto{\pgfqpoint{3.959602in}{3.369552in}}%
\pgfpathlineto{\pgfqpoint{3.960209in}{3.369241in}}%
\pgfpathlineto{\pgfqpoint{3.960936in}{3.370332in}}%
\pgfpathlineto{\pgfqpoint{3.961517in}{3.370651in}}%
\pgfpathlineto{\pgfqpoint{3.962243in}{3.369552in}}%
\pgfpathlineto{\pgfqpoint{3.962851in}{3.369241in}}%
\pgfpathlineto{\pgfqpoint{3.963577in}{3.370332in}}%
\pgfpathlineto{\pgfqpoint{3.964159in}{3.370651in}}%
\pgfpathlineto{\pgfqpoint{3.964885in}{3.369552in}}%
\pgfpathlineto{\pgfqpoint{3.965493in}{3.369241in}}%
\pgfpathlineto{\pgfqpoint{3.966219in}{3.370332in}}%
\pgfpathlineto{\pgfqpoint{3.966800in}{3.370651in}}%
\pgfpathlineto{\pgfqpoint{3.967527in}{3.369552in}}%
\pgfpathlineto{\pgfqpoint{3.968134in}{3.369241in}}%
\pgfpathlineto{\pgfqpoint{3.968861in}{3.370332in}}%
\pgfpathlineto{\pgfqpoint{3.969442in}{3.370651in}}%
\pgfpathlineto{\pgfqpoint{3.970168in}{3.369552in}}%
\pgfpathlineto{\pgfqpoint{3.970776in}{3.369241in}}%
\pgfpathlineto{\pgfqpoint{3.971502in}{3.370332in}}%
\pgfpathlineto{\pgfqpoint{3.972083in}{3.370651in}}%
\pgfpathlineto{\pgfqpoint{3.972810in}{3.369552in}}%
\pgfpathlineto{\pgfqpoint{3.973417in}{3.369241in}}%
\pgfpathlineto{\pgfqpoint{3.974144in}{3.370332in}}%
\pgfpathlineto{\pgfqpoint{3.974725in}{3.370651in}}%
\pgfpathlineto{\pgfqpoint{3.975452in}{3.369552in}}%
\pgfpathlineto{\pgfqpoint{3.976059in}{3.369241in}}%
\pgfpathlineto{\pgfqpoint{3.976786in}{3.370332in}}%
\pgfpathlineto{\pgfqpoint{3.977367in}{3.370651in}}%
\pgfpathlineto{\pgfqpoint{3.978093in}{3.369552in}}%
\pgfpathlineto{\pgfqpoint{3.978701in}{3.369241in}}%
\pgfpathlineto{\pgfqpoint{3.979427in}{3.370332in}}%
\pgfpathlineto{\pgfqpoint{3.980008in}{3.370651in}}%
\pgfpathlineto{\pgfqpoint{3.980735in}{3.369552in}}%
\pgfpathlineto{\pgfqpoint{3.981342in}{3.369241in}}%
\pgfpathlineto{\pgfqpoint{3.982069in}{3.370332in}}%
\pgfpathlineto{\pgfqpoint{3.982650in}{3.370651in}}%
\pgfpathlineto{\pgfqpoint{3.983376in}{3.369552in}}%
\pgfpathlineto{\pgfqpoint{3.983984in}{3.369241in}}%
\pgfpathlineto{\pgfqpoint{3.984710in}{3.370332in}}%
\pgfpathlineto{\pgfqpoint{3.985291in}{3.370651in}}%
\pgfpathlineto{\pgfqpoint{3.986018in}{3.369552in}}%
\pgfpathlineto{\pgfqpoint{3.986626in}{3.369241in}}%
\pgfpathlineto{\pgfqpoint{3.987352in}{3.370332in}}%
\pgfpathlineto{\pgfqpoint{3.987933in}{3.370651in}}%
\pgfpathlineto{\pgfqpoint{3.988660in}{3.369552in}}%
\pgfpathlineto{\pgfqpoint{3.989267in}{3.369241in}}%
\pgfpathlineto{\pgfqpoint{3.989994in}{3.370332in}}%
\pgfpathlineto{\pgfqpoint{3.990575in}{3.370651in}}%
\pgfpathlineto{\pgfqpoint{3.991301in}{3.369552in}}%
\pgfpathlineto{\pgfqpoint{3.991909in}{3.369241in}}%
\pgfpathlineto{\pgfqpoint{3.992635in}{3.370332in}}%
\pgfpathlineto{\pgfqpoint{3.993216in}{3.370651in}}%
\pgfpathlineto{\pgfqpoint{3.993943in}{3.369552in}}%
\pgfpathlineto{\pgfqpoint{3.994550in}{3.369241in}}%
\pgfpathlineto{\pgfqpoint{3.995277in}{3.370332in}}%
\pgfpathlineto{\pgfqpoint{3.995858in}{3.370651in}}%
\pgfpathlineto{\pgfqpoint{3.996584in}{3.369552in}}%
\pgfpathlineto{\pgfqpoint{3.997192in}{3.369241in}}%
\pgfpathlineto{\pgfqpoint{3.997918in}{3.370332in}}%
\pgfpathlineto{\pgfqpoint{3.998500in}{3.370651in}}%
\pgfpathlineto{\pgfqpoint{3.999226in}{3.369552in}}%
\pgfpathlineto{\pgfqpoint{3.999834in}{3.369241in}}%
\pgfpathlineto{\pgfqpoint{4.000560in}{3.370332in}}%
\pgfpathlineto{\pgfqpoint{4.001141in}{3.370651in}}%
\pgfpathlineto{\pgfqpoint{4.001868in}{3.369552in}}%
\pgfpathlineto{\pgfqpoint{4.002475in}{3.369241in}}%
\pgfpathlineto{\pgfqpoint{4.003202in}{3.370332in}}%
\pgfpathlineto{\pgfqpoint{4.003783in}{3.370651in}}%
\pgfpathlineto{\pgfqpoint{4.004509in}{3.369552in}}%
\pgfpathlineto{\pgfqpoint{4.005117in}{3.369241in}}%
\pgfpathlineto{\pgfqpoint{4.005843in}{3.370332in}}%
\pgfpathlineto{\pgfqpoint{4.006424in}{3.370651in}}%
\pgfpathlineto{\pgfqpoint{4.007151in}{3.369552in}}%
\pgfpathlineto{\pgfqpoint{4.007758in}{3.369241in}}%
\pgfpathlineto{\pgfqpoint{4.008485in}{3.370332in}}%
\pgfpathlineto{\pgfqpoint{4.009066in}{3.370651in}}%
\pgfpathlineto{\pgfqpoint{4.009792in}{3.369552in}}%
\pgfpathlineto{\pgfqpoint{4.010400in}{3.369241in}}%
\pgfpathlineto{\pgfqpoint{4.011126in}{3.370332in}}%
\pgfpathlineto{\pgfqpoint{4.011708in}{3.370651in}}%
\pgfpathlineto{\pgfqpoint{4.012434in}{3.369552in}}%
\pgfpathlineto{\pgfqpoint{4.013042in}{3.369241in}}%
\pgfpathlineto{\pgfqpoint{4.013768in}{3.370332in}}%
\pgfpathlineto{\pgfqpoint{4.014349in}{3.370651in}}%
\pgfpathlineto{\pgfqpoint{4.015076in}{3.369552in}}%
\pgfpathlineto{\pgfqpoint{4.015683in}{3.369241in}}%
\pgfpathlineto{\pgfqpoint{4.016410in}{3.370332in}}%
\pgfpathlineto{\pgfqpoint{4.016991in}{3.370651in}}%
\pgfpathlineto{\pgfqpoint{4.017717in}{3.369552in}}%
\pgfpathlineto{\pgfqpoint{4.018325in}{3.369241in}}%
\pgfpathlineto{\pgfqpoint{4.019051in}{3.370332in}}%
\pgfpathlineto{\pgfqpoint{4.019632in}{3.370651in}}%
\pgfpathlineto{\pgfqpoint{4.020359in}{3.369552in}}%
\pgfpathlineto{\pgfqpoint{4.020966in}{3.369241in}}%
\pgfpathlineto{\pgfqpoint{4.021693in}{3.370332in}}%
\pgfpathlineto{\pgfqpoint{4.022274in}{3.370651in}}%
\pgfpathlineto{\pgfqpoint{4.023000in}{3.369552in}}%
\pgfpathlineto{\pgfqpoint{4.023608in}{3.369241in}}%
\pgfpathlineto{\pgfqpoint{4.024334in}{3.370332in}}%
\pgfpathlineto{\pgfqpoint{4.024916in}{3.370651in}}%
\pgfpathlineto{\pgfqpoint{4.025642in}{3.369552in}}%
\pgfpathlineto{\pgfqpoint{4.026250in}{3.369241in}}%
\pgfpathlineto{\pgfqpoint{4.026976in}{3.370332in}}%
\pgfpathlineto{\pgfqpoint{4.027557in}{3.370651in}}%
\pgfpathlineto{\pgfqpoint{4.028284in}{3.369552in}}%
\pgfpathlineto{\pgfqpoint{4.028891in}{3.369241in}}%
\pgfpathlineto{\pgfqpoint{4.029618in}{3.370332in}}%
\pgfpathlineto{\pgfqpoint{4.030199in}{3.370651in}}%
\pgfpathlineto{\pgfqpoint{4.030925in}{3.369552in}}%
\pgfpathlineto{\pgfqpoint{4.031533in}{3.369241in}}%
\pgfpathlineto{\pgfqpoint{4.032259in}{3.370332in}}%
\pgfpathlineto{\pgfqpoint{4.032840in}{3.370651in}}%
\pgfpathlineto{\pgfqpoint{4.033567in}{3.369552in}}%
\pgfpathlineto{\pgfqpoint{4.034174in}{3.369241in}}%
\pgfpathlineto{\pgfqpoint{4.034901in}{3.370332in}}%
\pgfpathlineto{\pgfqpoint{4.035482in}{3.370651in}}%
\pgfpathlineto{\pgfqpoint{4.036208in}{3.369552in}}%
\pgfpathlineto{\pgfqpoint{4.036816in}{3.369241in}}%
\pgfpathlineto{\pgfqpoint{4.037542in}{3.370332in}}%
\pgfpathlineto{\pgfqpoint{4.038124in}{3.370651in}}%
\pgfpathlineto{\pgfqpoint{4.038850in}{3.369552in}}%
\pgfpathlineto{\pgfqpoint{4.039458in}{3.369241in}}%
\pgfpathlineto{\pgfqpoint{4.040184in}{3.370332in}}%
\pgfpathlineto{\pgfqpoint{4.040765in}{3.370651in}}%
\pgfpathlineto{\pgfqpoint{4.041492in}{3.369552in}}%
\pgfpathlineto{\pgfqpoint{4.042099in}{3.369241in}}%
\pgfpathlineto{\pgfqpoint{4.042826in}{3.370332in}}%
\pgfpathlineto{\pgfqpoint{4.043407in}{3.370651in}}%
\pgfpathlineto{\pgfqpoint{4.044133in}{3.369552in}}%
\pgfpathlineto{\pgfqpoint{4.044741in}{3.369241in}}%
\pgfpathlineto{\pgfqpoint{4.045467in}{3.370332in}}%
\pgfpathlineto{\pgfqpoint{4.046048in}{3.370651in}}%
\pgfpathlineto{\pgfqpoint{4.046775in}{3.369552in}}%
\pgfpathlineto{\pgfqpoint{4.047382in}{3.369241in}}%
\pgfpathlineto{\pgfqpoint{4.048109in}{3.370332in}}%
\pgfpathlineto{\pgfqpoint{4.048690in}{3.370651in}}%
\pgfpathlineto{\pgfqpoint{4.049417in}{3.369552in}}%
\pgfpathlineto{\pgfqpoint{4.050024in}{3.369241in}}%
\pgfpathlineto{\pgfqpoint{4.050751in}{3.370332in}}%
\pgfpathlineto{\pgfqpoint{4.051332in}{3.370651in}}%
\pgfpathlineto{\pgfqpoint{4.052058in}{3.369552in}}%
\pgfpathlineto{\pgfqpoint{4.052666in}{3.369241in}}%
\pgfpathlineto{\pgfqpoint{4.053392in}{3.370332in}}%
\pgfpathlineto{\pgfqpoint{4.053973in}{3.370651in}}%
\pgfpathlineto{\pgfqpoint{4.054700in}{3.369552in}}%
\pgfpathlineto{\pgfqpoint{4.055307in}{3.369241in}}%
\pgfpathlineto{\pgfqpoint{4.056034in}{3.370332in}}%
\pgfpathlineto{\pgfqpoint{4.056615in}{3.370651in}}%
\pgfpathlineto{\pgfqpoint{4.057341in}{3.369552in}}%
\pgfpathlineto{\pgfqpoint{4.057949in}{3.369241in}}%
\pgfpathlineto{\pgfqpoint{4.058675in}{3.370332in}}%
\pgfpathlineto{\pgfqpoint{4.059257in}{3.370651in}}%
\pgfpathlineto{\pgfqpoint{4.059983in}{3.369552in}}%
\pgfpathlineto{\pgfqpoint{4.060591in}{3.369241in}}%
\pgfpathlineto{\pgfqpoint{4.061317in}{3.370332in}}%
\pgfpathlineto{\pgfqpoint{4.061898in}{3.370651in}}%
\pgfpathlineto{\pgfqpoint{4.062625in}{3.369552in}}%
\pgfpathlineto{\pgfqpoint{4.063232in}{3.369241in}}%
\pgfpathlineto{\pgfqpoint{4.063959in}{3.370332in}}%
\pgfpathlineto{\pgfqpoint{4.064540in}{3.370651in}}%
\pgfpathlineto{\pgfqpoint{4.065266in}{3.369552in}}%
\pgfpathlineto{\pgfqpoint{4.065874in}{3.369241in}}%
\pgfpathlineto{\pgfqpoint{4.066600in}{3.370332in}}%
\pgfpathlineto{\pgfqpoint{4.067181in}{3.370651in}}%
\pgfpathlineto{\pgfqpoint{4.067908in}{3.369552in}}%
\pgfpathlineto{\pgfqpoint{4.068515in}{3.369241in}}%
\pgfpathlineto{\pgfqpoint{4.069242in}{3.370332in}}%
\pgfpathlineto{\pgfqpoint{4.069823in}{3.370651in}}%
\pgfpathlineto{\pgfqpoint{4.070549in}{3.369552in}}%
\pgfpathlineto{\pgfqpoint{4.071157in}{3.369241in}}%
\pgfpathlineto{\pgfqpoint{4.071883in}{3.370332in}}%
\pgfpathlineto{\pgfqpoint{4.072465in}{3.370651in}}%
\pgfpathlineto{\pgfqpoint{4.073191in}{3.369552in}}%
\pgfpathlineto{\pgfqpoint{4.073799in}{3.369241in}}%
\pgfpathlineto{\pgfqpoint{4.074525in}{3.370332in}}%
\pgfpathlineto{\pgfqpoint{4.075106in}{3.370651in}}%
\pgfpathlineto{\pgfqpoint{4.075833in}{3.369552in}}%
\pgfpathlineto{\pgfqpoint{4.076440in}{3.369241in}}%
\pgfpathlineto{\pgfqpoint{4.077167in}{3.370332in}}%
\pgfpathlineto{\pgfqpoint{4.077748in}{3.370651in}}%
\pgfpathlineto{\pgfqpoint{4.078474in}{3.369552in}}%
\pgfpathlineto{\pgfqpoint{4.079082in}{3.369241in}}%
\pgfpathlineto{\pgfqpoint{4.079808in}{3.370332in}}%
\pgfpathlineto{\pgfqpoint{4.080389in}{3.370651in}}%
\pgfpathlineto{\pgfqpoint{4.081116in}{3.369552in}}%
\pgfpathlineto{\pgfqpoint{4.081723in}{3.369241in}}%
\pgfpathlineto{\pgfqpoint{4.082450in}{3.370332in}}%
\pgfpathlineto{\pgfqpoint{4.083031in}{3.370651in}}%
\pgfpathlineto{\pgfqpoint{4.083757in}{3.369552in}}%
\pgfpathlineto{\pgfqpoint{4.084365in}{3.369241in}}%
\pgfpathlineto{\pgfqpoint{4.085091in}{3.370332in}}%
\pgfpathlineto{\pgfqpoint{4.085673in}{3.370651in}}%
\pgfpathlineto{\pgfqpoint{4.086399in}{3.369552in}}%
\pgfpathlineto{\pgfqpoint{4.087007in}{3.369241in}}%
\pgfpathlineto{\pgfqpoint{4.087733in}{3.370332in}}%
\pgfpathlineto{\pgfqpoint{4.088314in}{3.370651in}}%
\pgfpathlineto{\pgfqpoint{4.089041in}{3.369552in}}%
\pgfpathlineto{\pgfqpoint{4.089648in}{3.369241in}}%
\pgfpathlineto{\pgfqpoint{4.090375in}{3.370332in}}%
\pgfpathlineto{\pgfqpoint{4.090956in}{3.370651in}}%
\pgfpathlineto{\pgfqpoint{4.091682in}{3.369552in}}%
\pgfpathlineto{\pgfqpoint{4.092290in}{3.369241in}}%
\pgfpathlineto{\pgfqpoint{4.093016in}{3.370332in}}%
\pgfpathlineto{\pgfqpoint{4.093597in}{3.370651in}}%
\pgfpathlineto{\pgfqpoint{4.094324in}{3.369552in}}%
\pgfpathlineto{\pgfqpoint{4.094931in}{3.369241in}}%
\pgfpathlineto{\pgfqpoint{4.095658in}{3.370332in}}%
\pgfpathlineto{\pgfqpoint{4.096239in}{3.370651in}}%
\pgfpathlineto{\pgfqpoint{4.096965in}{3.369552in}}%
\pgfpathlineto{\pgfqpoint{4.097573in}{3.369241in}}%
\pgfpathlineto{\pgfqpoint{4.098299in}{3.370332in}}%
\pgfpathlineto{\pgfqpoint{4.098881in}{3.370651in}}%
\pgfpathlineto{\pgfqpoint{4.099607in}{3.369552in}}%
\pgfpathlineto{\pgfqpoint{4.100215in}{3.369241in}}%
\pgfpathlineto{\pgfqpoint{4.100941in}{3.370332in}}%
\pgfpathlineto{\pgfqpoint{4.101522in}{3.370651in}}%
\pgfpathlineto{\pgfqpoint{4.102249in}{3.369552in}}%
\pgfpathlineto{\pgfqpoint{4.102856in}{3.369241in}}%
\pgfpathlineto{\pgfqpoint{4.103583in}{3.370332in}}%
\pgfpathlineto{\pgfqpoint{4.104164in}{3.370651in}}%
\pgfpathlineto{\pgfqpoint{4.104890in}{3.369552in}}%
\pgfpathlineto{\pgfqpoint{4.105498in}{3.369241in}}%
\pgfpathlineto{\pgfqpoint{4.106224in}{3.370332in}}%
\pgfpathlineto{\pgfqpoint{4.106805in}{3.370651in}}%
\pgfpathlineto{\pgfqpoint{4.107532in}{3.369552in}}%
\pgfpathlineto{\pgfqpoint{4.108139in}{3.369241in}}%
\pgfpathlineto{\pgfqpoint{4.108866in}{3.370332in}}%
\pgfpathlineto{\pgfqpoint{4.109447in}{3.370651in}}%
\pgfpathlineto{\pgfqpoint{4.110174in}{3.369552in}}%
\pgfpathlineto{\pgfqpoint{4.110781in}{3.369241in}}%
\pgfpathlineto{\pgfqpoint{4.111508in}{3.370332in}}%
\pgfpathlineto{\pgfqpoint{4.112089in}{3.370651in}}%
\pgfpathlineto{\pgfqpoint{4.112815in}{3.369552in}}%
\pgfpathlineto{\pgfqpoint{4.113423in}{3.369241in}}%
\pgfpathlineto{\pgfqpoint{4.114149in}{3.370332in}}%
\pgfpathlineto{\pgfqpoint{4.114730in}{3.370651in}}%
\pgfpathlineto{\pgfqpoint{4.115457in}{3.369552in}}%
\pgfpathlineto{\pgfqpoint{4.116064in}{3.369241in}}%
\pgfpathlineto{\pgfqpoint{4.116791in}{3.370332in}}%
\pgfpathlineto{\pgfqpoint{4.117372in}{3.370651in}}%
\pgfpathlineto{\pgfqpoint{4.118098in}{3.369552in}}%
\pgfpathlineto{\pgfqpoint{4.118706in}{3.369241in}}%
\pgfpathlineto{\pgfqpoint{4.119432in}{3.370332in}}%
\pgfpathlineto{\pgfqpoint{4.120013in}{3.370651in}}%
\pgfpathlineto{\pgfqpoint{4.120740in}{3.369552in}}%
\pgfpathlineto{\pgfqpoint{4.121348in}{3.369241in}}%
\pgfpathlineto{\pgfqpoint{4.122074in}{3.370332in}}%
\pgfpathlineto{\pgfqpoint{4.122655in}{3.370651in}}%
\pgfpathlineto{\pgfqpoint{4.123382in}{3.369552in}}%
\pgfpathlineto{\pgfqpoint{4.123989in}{3.369241in}}%
\pgfpathlineto{\pgfqpoint{4.124716in}{3.370332in}}%
\pgfpathlineto{\pgfqpoint{4.125297in}{3.370651in}}%
\pgfpathlineto{\pgfqpoint{4.126023in}{3.369552in}}%
\pgfpathlineto{\pgfqpoint{4.126631in}{3.369241in}}%
\pgfpathlineto{\pgfqpoint{4.127357in}{3.370332in}}%
\pgfpathlineto{\pgfqpoint{4.127938in}{3.370651in}}%
\pgfpathlineto{\pgfqpoint{4.128665in}{3.369552in}}%
\pgfpathlineto{\pgfqpoint{4.129272in}{3.369241in}}%
\pgfpathlineto{\pgfqpoint{4.129999in}{3.370332in}}%
\pgfpathlineto{\pgfqpoint{4.130580in}{3.370651in}}%
\pgfpathlineto{\pgfqpoint{4.131306in}{3.369552in}}%
\pgfpathlineto{\pgfqpoint{4.131914in}{3.369241in}}%
\pgfpathlineto{\pgfqpoint{4.132891in}{3.369936in}}%
\pgfpathlineto{\pgfqpoint{4.664819in}{3.369936in}}%
\pgfpathlineto{\pgfqpoint{4.666272in}{0.638755in}}%
\pgfpathlineto{\pgfqpoint{4.666364in}{0.653396in}}%
\pgfpathlineto{\pgfqpoint{4.667341in}{0.637273in}}%
\pgfpathlineto{\pgfqpoint{4.667579in}{0.654479in}}%
\pgfpathlineto{\pgfqpoint{4.667896in}{0.652932in}}%
\pgfpathlineto{\pgfqpoint{4.667936in}{0.652932in}}%
\pgfpathlineto{\pgfqpoint{4.668345in}{0.637314in}}%
\pgfpathlineto{\pgfqpoint{4.668055in}{0.654478in}}%
\pgfpathlineto{\pgfqpoint{4.669481in}{0.647535in}}%
\pgfpathlineto{\pgfqpoint{4.669508in}{0.647535in}}%
\pgfpathlineto{\pgfqpoint{4.669877in}{0.638131in}}%
\pgfpathlineto{\pgfqpoint{4.669679in}{0.653982in}}%
\pgfpathlineto{\pgfqpoint{4.671040in}{0.651916in}}%
\pgfpathlineto{\pgfqpoint{4.671066in}{0.651916in}}%
\pgfpathlineto{\pgfqpoint{4.671502in}{0.638921in}}%
\pgfpathlineto{\pgfqpoint{4.671185in}{0.653080in}}%
\pgfpathlineto{\pgfqpoint{4.672611in}{0.650360in}}%
\pgfpathlineto{\pgfqpoint{4.673008in}{0.640118in}}%
\pgfpathlineto{\pgfqpoint{4.672704in}{0.651770in}}%
\pgfpathlineto{\pgfqpoint{4.674157in}{0.646515in}}%
\pgfpathlineto{\pgfqpoint{4.674183in}{0.646515in}}%
\pgfpathlineto{\pgfqpoint{4.674553in}{0.641183in}}%
\pgfpathlineto{\pgfqpoint{4.674368in}{0.650633in}}%
\pgfpathlineto{\pgfqpoint{4.675715in}{0.649037in}}%
\pgfpathlineto{\pgfqpoint{4.676178in}{0.641732in}}%
\pgfpathlineto{\pgfqpoint{4.675808in}{0.650046in}}%
\pgfpathlineto{\pgfqpoint{4.677274in}{0.645549in}}%
\pgfpathlineto{\pgfqpoint{4.677300in}{0.645549in}}%
\pgfpathlineto{\pgfqpoint{4.678700in}{0.641411in}}%
\pgfpathlineto{\pgfqpoint{4.678463in}{0.650380in}}%
\pgfpathlineto{\pgfqpoint{4.678832in}{0.641508in}}%
\pgfpathlineto{\pgfqpoint{4.680034in}{0.651043in}}%
\pgfpathlineto{\pgfqpoint{4.680219in}{0.640787in}}%
\pgfpathlineto{\pgfqpoint{4.680378in}{0.650362in}}%
\pgfpathlineto{\pgfqpoint{4.680404in}{0.650362in}}%
\pgfpathlineto{\pgfqpoint{4.681870in}{0.640096in}}%
\pgfpathlineto{\pgfqpoint{4.681540in}{0.651710in}}%
\pgfpathlineto{\pgfqpoint{4.681950in}{0.645889in}}%
\pgfpathlineto{\pgfqpoint{4.681963in}{0.645889in}}%
\pgfpathlineto{\pgfqpoint{4.683442in}{0.639618in}}%
\pgfpathlineto{\pgfqpoint{4.683099in}{0.652117in}}%
\pgfpathlineto{\pgfqpoint{4.683495in}{0.646768in}}%
\pgfpathlineto{\pgfqpoint{4.683535in}{0.646768in}}%
\pgfpathlineto{\pgfqpoint{4.684987in}{0.639513in}}%
\pgfpathlineto{\pgfqpoint{4.684129in}{0.652297in}}%
\pgfpathlineto{\pgfqpoint{4.685067in}{0.647714in}}%
\pgfpathlineto{\pgfqpoint{4.685080in}{0.647714in}}%
\pgfpathlineto{\pgfqpoint{4.686057in}{0.639525in}}%
\pgfpathlineto{\pgfqpoint{4.685238in}{0.652326in}}%
\pgfpathlineto{\pgfqpoint{4.686612in}{0.639813in}}%
\pgfpathlineto{\pgfqpoint{4.686770in}{0.652176in}}%
\pgfpathlineto{\pgfqpoint{4.687087in}{0.639665in}}%
\pgfpathlineto{\pgfqpoint{4.688157in}{0.649044in}}%
\pgfpathlineto{\pgfqpoint{4.688184in}{0.639919in}}%
\pgfpathlineto{\pgfqpoint{4.688316in}{0.651798in}}%
\pgfpathlineto{\pgfqpoint{4.689703in}{0.640188in}}%
\pgfpathlineto{\pgfqpoint{4.690456in}{0.651480in}}%
\pgfpathlineto{\pgfqpoint{4.691248in}{0.649391in}}%
\pgfpathlineto{\pgfqpoint{4.691274in}{0.640463in}}%
\pgfpathlineto{\pgfqpoint{4.692014in}{0.651248in}}%
\pgfpathlineto{\pgfqpoint{4.692780in}{0.646665in}}%
\pgfpathlineto{\pgfqpoint{4.693573in}{0.651151in}}%
\pgfpathlineto{\pgfqpoint{4.692846in}{0.640633in}}%
\pgfpathlineto{\pgfqpoint{4.694325in}{0.649529in}}%
\pgfpathlineto{\pgfqpoint{4.694365in}{0.649529in}}%
\pgfpathlineto{\pgfqpoint{4.695144in}{0.651190in}}%
\pgfpathlineto{\pgfqpoint{4.695910in}{0.640572in}}%
\pgfpathlineto{\pgfqpoint{4.695924in}{0.640572in}}%
\pgfpathlineto{\pgfqpoint{4.697284in}{0.651314in}}%
\pgfpathlineto{\pgfqpoint{4.696994in}{0.640431in}}%
\pgfpathlineto{\pgfqpoint{4.697469in}{0.648726in}}%
\pgfpathlineto{\pgfqpoint{4.697548in}{0.647768in}}%
\pgfpathlineto{\pgfqpoint{4.698565in}{0.640265in}}%
\pgfpathlineto{\pgfqpoint{4.698816in}{0.651544in}}%
\pgfpathlineto{\pgfqpoint{4.699080in}{0.648415in}}%
\pgfpathlineto{\pgfqpoint{4.699146in}{0.648239in}}%
\pgfpathlineto{\pgfqpoint{4.699648in}{0.640176in}}%
\pgfpathlineto{\pgfqpoint{4.700335in}{0.651702in}}%
\pgfpathlineto{\pgfqpoint{4.700679in}{0.650395in}}%
\pgfpathlineto{\pgfqpoint{4.700705in}{0.650395in}}%
\pgfpathlineto{\pgfqpoint{4.701894in}{0.651744in}}%
\pgfpathlineto{\pgfqpoint{4.702237in}{0.640077in}}%
\pgfpathlineto{\pgfqpoint{4.702950in}{0.651747in}}%
\pgfpathlineto{\pgfqpoint{4.703769in}{0.640082in}}%
\pgfpathlineto{\pgfqpoint{4.703782in}{0.640082in}}%
\pgfpathlineto{\pgfqpoint{4.704562in}{0.651693in}}%
\pgfpathlineto{\pgfqpoint{4.705328in}{0.649183in}}%
\pgfpathlineto{\pgfqpoint{4.705354in}{0.649183in}}%
\pgfpathlineto{\pgfqpoint{4.706411in}{0.640215in}}%
\pgfpathlineto{\pgfqpoint{4.705566in}{0.651630in}}%
\pgfpathlineto{\pgfqpoint{4.706900in}{0.640324in}}%
\pgfpathlineto{\pgfqpoint{4.707190in}{0.651558in}}%
\pgfpathlineto{\pgfqpoint{4.707930in}{0.640300in}}%
\pgfpathlineto{\pgfqpoint{4.708445in}{0.648380in}}%
\pgfpathlineto{\pgfqpoint{4.708524in}{0.648177in}}%
\pgfpathlineto{\pgfqpoint{4.709026in}{0.640353in}}%
\pgfpathlineto{\pgfqpoint{4.708683in}{0.651472in}}%
\pgfpathlineto{\pgfqpoint{4.710056in}{0.648482in}}%
\pgfpathlineto{\pgfqpoint{4.710083in}{0.648482in}}%
\pgfpathlineto{\pgfqpoint{4.710611in}{0.640364in}}%
\pgfpathlineto{\pgfqpoint{4.711351in}{0.651450in}}%
\pgfpathlineto{\pgfqpoint{4.711615in}{0.647704in}}%
\pgfpathlineto{\pgfqpoint{4.712870in}{0.651458in}}%
\pgfpathlineto{\pgfqpoint{4.712117in}{0.640361in}}%
\pgfpathlineto{\pgfqpoint{4.713068in}{0.640858in}}%
\pgfpathlineto{\pgfqpoint{4.713081in}{0.640858in}}%
\pgfpathlineto{\pgfqpoint{4.713966in}{0.651494in}}%
\pgfpathlineto{\pgfqpoint{4.713187in}{0.640351in}}%
\pgfpathlineto{\pgfqpoint{4.714626in}{0.646986in}}%
\pgfpathlineto{\pgfqpoint{4.714639in}{0.646986in}}%
\pgfpathlineto{\pgfqpoint{4.715511in}{0.651545in}}%
\pgfpathlineto{\pgfqpoint{4.714705in}{0.640290in}}%
\pgfpathlineto{\pgfqpoint{4.716145in}{0.641181in}}%
\pgfpathlineto{\pgfqpoint{4.716172in}{0.641181in}}%
\pgfpathlineto{\pgfqpoint{4.717083in}{0.651556in}}%
\pgfpathlineto{\pgfqpoint{4.716304in}{0.640259in}}%
\pgfpathlineto{\pgfqpoint{4.717136in}{0.642736in}}%
\pgfpathlineto{\pgfqpoint{4.717162in}{3.368415in}}%
\pgfpathlineto{\pgfqpoint{4.718681in}{3.364135in}}%
\pgfpathlineto{\pgfqpoint{4.719658in}{3.366343in}}%
\pgfpathlineto{\pgfqpoint{4.721547in}{3.375070in}}%
\pgfpathlineto{\pgfqpoint{4.722485in}{3.371362in}}%
\pgfpathlineto{\pgfqpoint{4.724110in}{3.366495in}}%
\pgfpathlineto{\pgfqpoint{4.724757in}{3.367348in}}%
\pgfpathlineto{\pgfqpoint{4.726830in}{3.373490in}}%
\pgfpathlineto{\pgfqpoint{4.727847in}{3.370738in}}%
\pgfpathlineto{\pgfqpoint{4.729432in}{3.367580in}}%
\pgfpathlineto{\pgfqpoint{4.730080in}{3.368307in}}%
\pgfpathlineto{\pgfqpoint{4.732140in}{3.372278in}}%
\pgfpathlineto{\pgfqpoint{4.733038in}{3.370677in}}%
\pgfpathlineto{\pgfqpoint{4.734795in}{3.368308in}}%
\pgfpathlineto{\pgfqpoint{4.735402in}{3.368893in}}%
\pgfpathlineto{\pgfqpoint{4.737555in}{3.371418in}}%
\pgfpathlineto{\pgfqpoint{4.738308in}{3.370437in}}%
\pgfpathlineto{\pgfqpoint{4.740289in}{3.368895in}}%
\pgfpathlineto{\pgfqpoint{4.740765in}{3.369313in}}%
\pgfpathlineto{\pgfqpoint{4.743182in}{3.370688in}}%
\pgfpathlineto{\pgfqpoint{4.743631in}{3.370221in}}%
\pgfpathlineto{\pgfqpoint{4.746193in}{3.369633in}}%
\pgfpathlineto{\pgfqpoint{4.749033in}{3.370033in}}%
\pgfpathlineto{\pgfqpoint{4.751728in}{3.369887in}}%
\pgfpathlineto{\pgfqpoint{4.754581in}{3.369862in}}%
\pgfpathlineto{\pgfqpoint{4.757526in}{3.370114in}}%
\pgfpathlineto{\pgfqpoint{4.760907in}{3.369667in}}%
\pgfpathlineto{\pgfqpoint{4.809896in}{3.369936in}}%
\pgfpathlineto{\pgfqpoint{4.966464in}{3.369936in}}%
\pgfpathlineto{\pgfqpoint{5.188636in}{3.369936in}}%
\pgfpathlineto{\pgfqpoint{5.188636in}{3.369936in}}%
\pgfusepath{stroke}%
\end{pgfscope}%
\begin{pgfscope}%
\pgfsetrectcap%
\pgfsetmiterjoin%
\pgfsetlinewidth{0.803000pt}%
\definecolor{currentstroke}{rgb}{0.000000,0.000000,0.000000}%
\pgfsetstrokecolor{currentstroke}%
\pgfsetdash{}{0pt}%
\pgfpathmoveto{\pgfqpoint{0.750000in}{0.500000in}}%
\pgfpathlineto{\pgfqpoint{0.750000in}{3.520000in}}%
\pgfusepath{stroke}%
\end{pgfscope}%
\begin{pgfscope}%
\pgfsetrectcap%
\pgfsetmiterjoin%
\pgfsetlinewidth{0.803000pt}%
\definecolor{currentstroke}{rgb}{0.000000,0.000000,0.000000}%
\pgfsetstrokecolor{currentstroke}%
\pgfsetdash{}{0pt}%
\pgfpathmoveto{\pgfqpoint{5.400000in}{0.500000in}}%
\pgfpathlineto{\pgfqpoint{5.400000in}{3.520000in}}%
\pgfusepath{stroke}%
\end{pgfscope}%
\begin{pgfscope}%
\pgfsetrectcap%
\pgfsetmiterjoin%
\pgfsetlinewidth{0.803000pt}%
\definecolor{currentstroke}{rgb}{0.000000,0.000000,0.000000}%
\pgfsetstrokecolor{currentstroke}%
\pgfsetdash{}{0pt}%
\pgfpathmoveto{\pgfqpoint{0.750000in}{0.500000in}}%
\pgfpathlineto{\pgfqpoint{5.400000in}{0.500000in}}%
\pgfusepath{stroke}%
\end{pgfscope}%
\begin{pgfscope}%
\pgfsetrectcap%
\pgfsetmiterjoin%
\pgfsetlinewidth{0.803000pt}%
\definecolor{currentstroke}{rgb}{0.000000,0.000000,0.000000}%
\pgfsetstrokecolor{currentstroke}%
\pgfsetdash{}{0pt}%
\pgfpathmoveto{\pgfqpoint{0.750000in}{3.520000in}}%
\pgfpathlineto{\pgfqpoint{5.400000in}{3.520000in}}%
\pgfusepath{stroke}%
\end{pgfscope}%
\end{pgfpicture}%
\makeatother%
\endgroup%

    \caption{PEC Data Set $f_c$ Signal}
    \label{fig:pec_all}
\end{figure}

Figure \ref{fig:pec_ll_fault} illustrates the frequency behavior during a line-to-line (LL) Fault. A LL Fault is also referred to as an unsymmetrical fault and occurs when there is a short circuit between two conductors. In three phase power, this can occur between two phases of the system. This fault causes a significant decrease in frequency that is orders of magnitude greater than the standard frequency of the system. 

\begin{figure}[H]
    %\centering
    %% Creator: Matplotlib, PGF backend
%%
%% To include the figure in your LaTeX document, write
%%   \input{<filename>.pgf}
%%
%% Make sure the required packages are loaded in your preamble
%%   \usepackage{pgf}
%%
%% Also ensure that all the required font packages are loaded; for instance,
%% the lmodern package is sometimes necessary when using math font.
%%   \usepackage{lmodern}
%%
%% Figures using additional raster images can only be included by \input if
%% they are in the same directory as the main LaTeX file. For loading figures
%% from other directories you can use the `import` package
%%   \usepackage{import}
%%
%% and then include the figures with
%%   \import{<path to file>}{<filename>.pgf}
%%
%% Matplotlib used the following preamble
%%
\begingroup%
\makeatletter%
\begin{pgfpicture}%
\pgfpathrectangle{\pgfpointorigin}{\pgfqpoint{6.000000in}{4.000000in}}%
\pgfusepath{use as bounding box, clip}%
\begin{pgfscope}%
\pgfsetbuttcap%
\pgfsetmiterjoin%
\pgfsetlinewidth{0.000000pt}%
\definecolor{currentstroke}{rgb}{1.000000,1.000000,1.000000}%
\pgfsetstrokecolor{currentstroke}%
\pgfsetstrokeopacity{0.000000}%
\pgfsetdash{}{0pt}%
\pgfpathmoveto{\pgfqpoint{0.000000in}{0.000000in}}%
\pgfpathlineto{\pgfqpoint{6.000000in}{0.000000in}}%
\pgfpathlineto{\pgfqpoint{6.000000in}{4.000000in}}%
\pgfpathlineto{\pgfqpoint{0.000000in}{4.000000in}}%
\pgfpathlineto{\pgfqpoint{0.000000in}{0.000000in}}%
\pgfpathclose%
\pgfusepath{}%
\end{pgfscope}%
\begin{pgfscope}%
\pgfsetbuttcap%
\pgfsetmiterjoin%
\definecolor{currentfill}{rgb}{1.000000,1.000000,1.000000}%
\pgfsetfillcolor{currentfill}%
\pgfsetlinewidth{0.000000pt}%
\definecolor{currentstroke}{rgb}{0.000000,0.000000,0.000000}%
\pgfsetstrokecolor{currentstroke}%
\pgfsetstrokeopacity{0.000000}%
\pgfsetdash{}{0pt}%
\pgfpathmoveto{\pgfqpoint{0.750000in}{0.500000in}}%
\pgfpathlineto{\pgfqpoint{5.400000in}{0.500000in}}%
\pgfpathlineto{\pgfqpoint{5.400000in}{3.520000in}}%
\pgfpathlineto{\pgfqpoint{0.750000in}{3.520000in}}%
\pgfpathlineto{\pgfqpoint{0.750000in}{0.500000in}}%
\pgfpathclose%
\pgfusepath{fill}%
\end{pgfscope}%
\begin{pgfscope}%
\pgfsetbuttcap%
\pgfsetroundjoin%
\definecolor{currentfill}{rgb}{0.000000,0.000000,0.000000}%
\pgfsetfillcolor{currentfill}%
\pgfsetlinewidth{0.803000pt}%
\definecolor{currentstroke}{rgb}{0.000000,0.000000,0.000000}%
\pgfsetstrokecolor{currentstroke}%
\pgfsetdash{}{0pt}%
\pgfsys@defobject{currentmarker}{\pgfqpoint{0.000000in}{-0.048611in}}{\pgfqpoint{0.000000in}{0.000000in}}{%
\pgfpathmoveto{\pgfqpoint{0.000000in}{0.000000in}}%
\pgfpathlineto{\pgfqpoint{0.000000in}{-0.048611in}}%
\pgfusepath{stroke,fill}%
}%
\begin{pgfscope}%
\pgfsys@transformshift{0.961364in}{0.500000in}%
\pgfsys@useobject{currentmarker}{}%
\end{pgfscope}%
\end{pgfscope}%
\begin{pgfscope}%
\definecolor{textcolor}{rgb}{0.000000,0.000000,0.000000}%
\pgfsetstrokecolor{textcolor}%
\pgfsetfillcolor{textcolor}%
\pgftext[x=0.961364in,y=0.402778in,,top]{\color{textcolor}\rmfamily\fontsize{10.000000}{12.000000}\selectfont \(\displaystyle {40200}\)}%
\end{pgfscope}%
\begin{pgfscope}%
\pgfsetbuttcap%
\pgfsetroundjoin%
\definecolor{currentfill}{rgb}{0.000000,0.000000,0.000000}%
\pgfsetfillcolor{currentfill}%
\pgfsetlinewidth{0.803000pt}%
\definecolor{currentstroke}{rgb}{0.000000,0.000000,0.000000}%
\pgfsetstrokecolor{currentstroke}%
\pgfsetdash{}{0pt}%
\pgfsys@defobject{currentmarker}{\pgfqpoint{0.000000in}{-0.048611in}}{\pgfqpoint{0.000000in}{0.000000in}}{%
\pgfpathmoveto{\pgfqpoint{0.000000in}{0.000000in}}%
\pgfpathlineto{\pgfqpoint{0.000000in}{-0.048611in}}%
\pgfusepath{stroke,fill}%
}%
\begin{pgfscope}%
\pgfsys@transformshift{1.807664in}{0.500000in}%
\pgfsys@useobject{currentmarker}{}%
\end{pgfscope}%
\end{pgfscope}%
\begin{pgfscope}%
\definecolor{textcolor}{rgb}{0.000000,0.000000,0.000000}%
\pgfsetstrokecolor{textcolor}%
\pgfsetfillcolor{textcolor}%
\pgftext[x=1.807664in,y=0.402778in,,top]{\color{textcolor}\rmfamily\fontsize{10.000000}{12.000000}\selectfont \(\displaystyle {40400}\)}%
\end{pgfscope}%
\begin{pgfscope}%
\pgfsetbuttcap%
\pgfsetroundjoin%
\definecolor{currentfill}{rgb}{0.000000,0.000000,0.000000}%
\pgfsetfillcolor{currentfill}%
\pgfsetlinewidth{0.803000pt}%
\definecolor{currentstroke}{rgb}{0.000000,0.000000,0.000000}%
\pgfsetstrokecolor{currentstroke}%
\pgfsetdash{}{0pt}%
\pgfsys@defobject{currentmarker}{\pgfqpoint{0.000000in}{-0.048611in}}{\pgfqpoint{0.000000in}{0.000000in}}{%
\pgfpathmoveto{\pgfqpoint{0.000000in}{0.000000in}}%
\pgfpathlineto{\pgfqpoint{0.000000in}{-0.048611in}}%
\pgfusepath{stroke,fill}%
}%
\begin{pgfscope}%
\pgfsys@transformshift{2.653965in}{0.500000in}%
\pgfsys@useobject{currentmarker}{}%
\end{pgfscope}%
\end{pgfscope}%
\begin{pgfscope}%
\definecolor{textcolor}{rgb}{0.000000,0.000000,0.000000}%
\pgfsetstrokecolor{textcolor}%
\pgfsetfillcolor{textcolor}%
\pgftext[x=2.653965in,y=0.402778in,,top]{\color{textcolor}\rmfamily\fontsize{10.000000}{12.000000}\selectfont \(\displaystyle {40600}\)}%
\end{pgfscope}%
\begin{pgfscope}%
\pgfsetbuttcap%
\pgfsetroundjoin%
\definecolor{currentfill}{rgb}{0.000000,0.000000,0.000000}%
\pgfsetfillcolor{currentfill}%
\pgfsetlinewidth{0.803000pt}%
\definecolor{currentstroke}{rgb}{0.000000,0.000000,0.000000}%
\pgfsetstrokecolor{currentstroke}%
\pgfsetdash{}{0pt}%
\pgfsys@defobject{currentmarker}{\pgfqpoint{0.000000in}{-0.048611in}}{\pgfqpoint{0.000000in}{0.000000in}}{%
\pgfpathmoveto{\pgfqpoint{0.000000in}{0.000000in}}%
\pgfpathlineto{\pgfqpoint{0.000000in}{-0.048611in}}%
\pgfusepath{stroke,fill}%
}%
\begin{pgfscope}%
\pgfsys@transformshift{3.500266in}{0.500000in}%
\pgfsys@useobject{currentmarker}{}%
\end{pgfscope}%
\end{pgfscope}%
\begin{pgfscope}%
\definecolor{textcolor}{rgb}{0.000000,0.000000,0.000000}%
\pgfsetstrokecolor{textcolor}%
\pgfsetfillcolor{textcolor}%
\pgftext[x=3.500266in,y=0.402778in,,top]{\color{textcolor}\rmfamily\fontsize{10.000000}{12.000000}\selectfont \(\displaystyle {40800}\)}%
\end{pgfscope}%
\begin{pgfscope}%
\pgfsetbuttcap%
\pgfsetroundjoin%
\definecolor{currentfill}{rgb}{0.000000,0.000000,0.000000}%
\pgfsetfillcolor{currentfill}%
\pgfsetlinewidth{0.803000pt}%
\definecolor{currentstroke}{rgb}{0.000000,0.000000,0.000000}%
\pgfsetstrokecolor{currentstroke}%
\pgfsetdash{}{0pt}%
\pgfsys@defobject{currentmarker}{\pgfqpoint{0.000000in}{-0.048611in}}{\pgfqpoint{0.000000in}{0.000000in}}{%
\pgfpathmoveto{\pgfqpoint{0.000000in}{0.000000in}}%
\pgfpathlineto{\pgfqpoint{0.000000in}{-0.048611in}}%
\pgfusepath{stroke,fill}%
}%
\begin{pgfscope}%
\pgfsys@transformshift{4.346567in}{0.500000in}%
\pgfsys@useobject{currentmarker}{}%
\end{pgfscope}%
\end{pgfscope}%
\begin{pgfscope}%
\definecolor{textcolor}{rgb}{0.000000,0.000000,0.000000}%
\pgfsetstrokecolor{textcolor}%
\pgfsetfillcolor{textcolor}%
\pgftext[x=4.346567in,y=0.402778in,,top]{\color{textcolor}\rmfamily\fontsize{10.000000}{12.000000}\selectfont \(\displaystyle {41000}\)}%
\end{pgfscope}%
\begin{pgfscope}%
\pgfsetbuttcap%
\pgfsetroundjoin%
\definecolor{currentfill}{rgb}{0.000000,0.000000,0.000000}%
\pgfsetfillcolor{currentfill}%
\pgfsetlinewidth{0.803000pt}%
\definecolor{currentstroke}{rgb}{0.000000,0.000000,0.000000}%
\pgfsetstrokecolor{currentstroke}%
\pgfsetdash{}{0pt}%
\pgfsys@defobject{currentmarker}{\pgfqpoint{0.000000in}{-0.048611in}}{\pgfqpoint{0.000000in}{0.000000in}}{%
\pgfpathmoveto{\pgfqpoint{0.000000in}{0.000000in}}%
\pgfpathlineto{\pgfqpoint{0.000000in}{-0.048611in}}%
\pgfusepath{stroke,fill}%
}%
\begin{pgfscope}%
\pgfsys@transformshift{5.192868in}{0.500000in}%
\pgfsys@useobject{currentmarker}{}%
\end{pgfscope}%
\end{pgfscope}%
\begin{pgfscope}%
\definecolor{textcolor}{rgb}{0.000000,0.000000,0.000000}%
\pgfsetstrokecolor{textcolor}%
\pgfsetfillcolor{textcolor}%
\pgftext[x=5.192868in,y=0.402778in,,top]{\color{textcolor}\rmfamily\fontsize{10.000000}{12.000000}\selectfont \(\displaystyle {41200}\)}%
\end{pgfscope}%
\begin{pgfscope}%
\definecolor{textcolor}{rgb}{0.000000,0.000000,0.000000}%
\pgfsetstrokecolor{textcolor}%
\pgfsetfillcolor{textcolor}%
\pgftext[x=3.075000in,y=0.223766in,,top]{\color{textcolor}\rmfamily\fontsize{10.000000}{12.000000}\selectfont Time (s)}%
\end{pgfscope}%
\begin{pgfscope}%
\pgfsetbuttcap%
\pgfsetroundjoin%
\definecolor{currentfill}{rgb}{0.000000,0.000000,0.000000}%
\pgfsetfillcolor{currentfill}%
\pgfsetlinewidth{0.803000pt}%
\definecolor{currentstroke}{rgb}{0.000000,0.000000,0.000000}%
\pgfsetstrokecolor{currentstroke}%
\pgfsetdash{}{0pt}%
\pgfsys@defobject{currentmarker}{\pgfqpoint{-0.048611in}{0.000000in}}{\pgfqpoint{-0.000000in}{0.000000in}}{%
\pgfpathmoveto{\pgfqpoint{-0.000000in}{0.000000in}}%
\pgfpathlineto{\pgfqpoint{-0.048611in}{0.000000in}}%
\pgfusepath{stroke,fill}%
}%
\begin{pgfscope}%
\pgfsys@transformshift{0.750000in}{0.686723in}%
\pgfsys@useobject{currentmarker}{}%
\end{pgfscope}%
\end{pgfscope}%
\begin{pgfscope}%
\definecolor{textcolor}{rgb}{0.000000,0.000000,0.000000}%
\pgfsetstrokecolor{textcolor}%
\pgfsetfillcolor{textcolor}%
\pgftext[x=0.266974in, y=0.638498in, left, base]{\color{textcolor}\rmfamily\fontsize{10.000000}{12.000000}\selectfont \(\displaystyle {\ensuremath{-}7000}\)}%
\end{pgfscope}%
\begin{pgfscope}%
\pgfsetbuttcap%
\pgfsetroundjoin%
\definecolor{currentfill}{rgb}{0.000000,0.000000,0.000000}%
\pgfsetfillcolor{currentfill}%
\pgfsetlinewidth{0.803000pt}%
\definecolor{currentstroke}{rgb}{0.000000,0.000000,0.000000}%
\pgfsetstrokecolor{currentstroke}%
\pgfsetdash{}{0pt}%
\pgfsys@defobject{currentmarker}{\pgfqpoint{-0.048611in}{0.000000in}}{\pgfqpoint{-0.000000in}{0.000000in}}{%
\pgfpathmoveto{\pgfqpoint{-0.000000in}{0.000000in}}%
\pgfpathlineto{\pgfqpoint{-0.048611in}{0.000000in}}%
\pgfusepath{stroke,fill}%
}%
\begin{pgfscope}%
\pgfsys@transformshift{0.750000in}{1.069135in}%
\pgfsys@useobject{currentmarker}{}%
\end{pgfscope}%
\end{pgfscope}%
\begin{pgfscope}%
\definecolor{textcolor}{rgb}{0.000000,0.000000,0.000000}%
\pgfsetstrokecolor{textcolor}%
\pgfsetfillcolor{textcolor}%
\pgftext[x=0.266974in, y=1.020910in, left, base]{\color{textcolor}\rmfamily\fontsize{10.000000}{12.000000}\selectfont \(\displaystyle {\ensuremath{-}6000}\)}%
\end{pgfscope}%
\begin{pgfscope}%
\pgfsetbuttcap%
\pgfsetroundjoin%
\definecolor{currentfill}{rgb}{0.000000,0.000000,0.000000}%
\pgfsetfillcolor{currentfill}%
\pgfsetlinewidth{0.803000pt}%
\definecolor{currentstroke}{rgb}{0.000000,0.000000,0.000000}%
\pgfsetstrokecolor{currentstroke}%
\pgfsetdash{}{0pt}%
\pgfsys@defobject{currentmarker}{\pgfqpoint{-0.048611in}{0.000000in}}{\pgfqpoint{-0.000000in}{0.000000in}}{%
\pgfpathmoveto{\pgfqpoint{-0.000000in}{0.000000in}}%
\pgfpathlineto{\pgfqpoint{-0.048611in}{0.000000in}}%
\pgfusepath{stroke,fill}%
}%
\begin{pgfscope}%
\pgfsys@transformshift{0.750000in}{1.451547in}%
\pgfsys@useobject{currentmarker}{}%
\end{pgfscope}%
\end{pgfscope}%
\begin{pgfscope}%
\definecolor{textcolor}{rgb}{0.000000,0.000000,0.000000}%
\pgfsetstrokecolor{textcolor}%
\pgfsetfillcolor{textcolor}%
\pgftext[x=0.266974in, y=1.403322in, left, base]{\color{textcolor}\rmfamily\fontsize{10.000000}{12.000000}\selectfont \(\displaystyle {\ensuremath{-}5000}\)}%
\end{pgfscope}%
\begin{pgfscope}%
\pgfsetbuttcap%
\pgfsetroundjoin%
\definecolor{currentfill}{rgb}{0.000000,0.000000,0.000000}%
\pgfsetfillcolor{currentfill}%
\pgfsetlinewidth{0.803000pt}%
\definecolor{currentstroke}{rgb}{0.000000,0.000000,0.000000}%
\pgfsetstrokecolor{currentstroke}%
\pgfsetdash{}{0pt}%
\pgfsys@defobject{currentmarker}{\pgfqpoint{-0.048611in}{0.000000in}}{\pgfqpoint{-0.000000in}{0.000000in}}{%
\pgfpathmoveto{\pgfqpoint{-0.000000in}{0.000000in}}%
\pgfpathlineto{\pgfqpoint{-0.048611in}{0.000000in}}%
\pgfusepath{stroke,fill}%
}%
\begin{pgfscope}%
\pgfsys@transformshift{0.750000in}{1.833959in}%
\pgfsys@useobject{currentmarker}{}%
\end{pgfscope}%
\end{pgfscope}%
\begin{pgfscope}%
\definecolor{textcolor}{rgb}{0.000000,0.000000,0.000000}%
\pgfsetstrokecolor{textcolor}%
\pgfsetfillcolor{textcolor}%
\pgftext[x=0.266974in, y=1.785734in, left, base]{\color{textcolor}\rmfamily\fontsize{10.000000}{12.000000}\selectfont \(\displaystyle {\ensuremath{-}4000}\)}%
\end{pgfscope}%
\begin{pgfscope}%
\pgfsetbuttcap%
\pgfsetroundjoin%
\definecolor{currentfill}{rgb}{0.000000,0.000000,0.000000}%
\pgfsetfillcolor{currentfill}%
\pgfsetlinewidth{0.803000pt}%
\definecolor{currentstroke}{rgb}{0.000000,0.000000,0.000000}%
\pgfsetstrokecolor{currentstroke}%
\pgfsetdash{}{0pt}%
\pgfsys@defobject{currentmarker}{\pgfqpoint{-0.048611in}{0.000000in}}{\pgfqpoint{-0.000000in}{0.000000in}}{%
\pgfpathmoveto{\pgfqpoint{-0.000000in}{0.000000in}}%
\pgfpathlineto{\pgfqpoint{-0.048611in}{0.000000in}}%
\pgfusepath{stroke,fill}%
}%
\begin{pgfscope}%
\pgfsys@transformshift{0.750000in}{2.216371in}%
\pgfsys@useobject{currentmarker}{}%
\end{pgfscope}%
\end{pgfscope}%
\begin{pgfscope}%
\definecolor{textcolor}{rgb}{0.000000,0.000000,0.000000}%
\pgfsetstrokecolor{textcolor}%
\pgfsetfillcolor{textcolor}%
\pgftext[x=0.266974in, y=2.168146in, left, base]{\color{textcolor}\rmfamily\fontsize{10.000000}{12.000000}\selectfont \(\displaystyle {\ensuremath{-}3000}\)}%
\end{pgfscope}%
\begin{pgfscope}%
\pgfsetbuttcap%
\pgfsetroundjoin%
\definecolor{currentfill}{rgb}{0.000000,0.000000,0.000000}%
\pgfsetfillcolor{currentfill}%
\pgfsetlinewidth{0.803000pt}%
\definecolor{currentstroke}{rgb}{0.000000,0.000000,0.000000}%
\pgfsetstrokecolor{currentstroke}%
\pgfsetdash{}{0pt}%
\pgfsys@defobject{currentmarker}{\pgfqpoint{-0.048611in}{0.000000in}}{\pgfqpoint{-0.000000in}{0.000000in}}{%
\pgfpathmoveto{\pgfqpoint{-0.000000in}{0.000000in}}%
\pgfpathlineto{\pgfqpoint{-0.048611in}{0.000000in}}%
\pgfusepath{stroke,fill}%
}%
\begin{pgfscope}%
\pgfsys@transformshift{0.750000in}{2.598783in}%
\pgfsys@useobject{currentmarker}{}%
\end{pgfscope}%
\end{pgfscope}%
\begin{pgfscope}%
\definecolor{textcolor}{rgb}{0.000000,0.000000,0.000000}%
\pgfsetstrokecolor{textcolor}%
\pgfsetfillcolor{textcolor}%
\pgftext[x=0.266974in, y=2.550558in, left, base]{\color{textcolor}\rmfamily\fontsize{10.000000}{12.000000}\selectfont \(\displaystyle {\ensuremath{-}2000}\)}%
\end{pgfscope}%
\begin{pgfscope}%
\pgfsetbuttcap%
\pgfsetroundjoin%
\definecolor{currentfill}{rgb}{0.000000,0.000000,0.000000}%
\pgfsetfillcolor{currentfill}%
\pgfsetlinewidth{0.803000pt}%
\definecolor{currentstroke}{rgb}{0.000000,0.000000,0.000000}%
\pgfsetstrokecolor{currentstroke}%
\pgfsetdash{}{0pt}%
\pgfsys@defobject{currentmarker}{\pgfqpoint{-0.048611in}{0.000000in}}{\pgfqpoint{-0.000000in}{0.000000in}}{%
\pgfpathmoveto{\pgfqpoint{-0.000000in}{0.000000in}}%
\pgfpathlineto{\pgfqpoint{-0.048611in}{0.000000in}}%
\pgfusepath{stroke,fill}%
}%
\begin{pgfscope}%
\pgfsys@transformshift{0.750000in}{2.981195in}%
\pgfsys@useobject{currentmarker}{}%
\end{pgfscope}%
\end{pgfscope}%
\begin{pgfscope}%
\definecolor{textcolor}{rgb}{0.000000,0.000000,0.000000}%
\pgfsetstrokecolor{textcolor}%
\pgfsetfillcolor{textcolor}%
\pgftext[x=0.266974in, y=2.932969in, left, base]{\color{textcolor}\rmfamily\fontsize{10.000000}{12.000000}\selectfont \(\displaystyle {\ensuremath{-}1000}\)}%
\end{pgfscope}%
\begin{pgfscope}%
\pgfsetbuttcap%
\pgfsetroundjoin%
\definecolor{currentfill}{rgb}{0.000000,0.000000,0.000000}%
\pgfsetfillcolor{currentfill}%
\pgfsetlinewidth{0.803000pt}%
\definecolor{currentstroke}{rgb}{0.000000,0.000000,0.000000}%
\pgfsetstrokecolor{currentstroke}%
\pgfsetdash{}{0pt}%
\pgfsys@defobject{currentmarker}{\pgfqpoint{-0.048611in}{0.000000in}}{\pgfqpoint{-0.000000in}{0.000000in}}{%
\pgfpathmoveto{\pgfqpoint{-0.000000in}{0.000000in}}%
\pgfpathlineto{\pgfqpoint{-0.048611in}{0.000000in}}%
\pgfusepath{stroke,fill}%
}%
\begin{pgfscope}%
\pgfsys@transformshift{0.750000in}{3.363607in}%
\pgfsys@useobject{currentmarker}{}%
\end{pgfscope}%
\end{pgfscope}%
\begin{pgfscope}%
\definecolor{textcolor}{rgb}{0.000000,0.000000,0.000000}%
\pgfsetstrokecolor{textcolor}%
\pgfsetfillcolor{textcolor}%
\pgftext[x=0.583333in, y=3.315381in, left, base]{\color{textcolor}\rmfamily\fontsize{10.000000}{12.000000}\selectfont \(\displaystyle {0}\)}%
\end{pgfscope}%
\begin{pgfscope}%
\definecolor{textcolor}{rgb}{0.000000,0.000000,0.000000}%
\pgfsetstrokecolor{textcolor}%
\pgfsetfillcolor{textcolor}%
\pgftext[x=0.211419in,y=2.010000in,,bottom,rotate=90.000000]{\color{textcolor}\rmfamily\fontsize{10.000000}{12.000000}\selectfont Frequency (Hz)}%
\end{pgfscope}%
\begin{pgfscope}%
\pgfpathrectangle{\pgfqpoint{0.750000in}{0.500000in}}{\pgfqpoint{4.650000in}{3.020000in}}%
\pgfusepath{clip}%
\pgfsetrectcap%
\pgfsetroundjoin%
\pgfsetlinewidth{1.505625pt}%
\definecolor{currentstroke}{rgb}{0.121569,0.466667,0.705882}%
\pgfsetstrokecolor{currentstroke}%
\pgfsetdash{}{0pt}%
\pgfpathmoveto{\pgfqpoint{0.961364in}{3.382727in}}%
\pgfpathlineto{\pgfqpoint{1.553774in}{3.382727in}}%
\pgfpathlineto{\pgfqpoint{1.558006in}{2.736538in}}%
\pgfpathlineto{\pgfqpoint{1.562237in}{1.378478in}}%
\pgfpathlineto{\pgfqpoint{1.566469in}{2.770381in}}%
\pgfpathlineto{\pgfqpoint{1.570700in}{3.382506in}}%
\pgfpathlineto{\pgfqpoint{1.579163in}{3.379132in}}%
\pgfpathlineto{\pgfqpoint{1.596089in}{3.368131in}}%
\pgfpathlineto{\pgfqpoint{1.608784in}{3.364581in}}%
\pgfpathlineto{\pgfqpoint{1.629941in}{3.363290in}}%
\pgfpathlineto{\pgfqpoint{1.752655in}{3.363086in}}%
\pgfpathlineto{\pgfqpoint{3.373321in}{3.361142in}}%
\pgfpathlineto{\pgfqpoint{3.402942in}{3.358737in}}%
\pgfpathlineto{\pgfqpoint{3.419868in}{3.354386in}}%
\pgfpathlineto{\pgfqpoint{3.428331in}{3.349173in}}%
\pgfpathlineto{\pgfqpoint{3.432562in}{3.344632in}}%
\pgfpathlineto{\pgfqpoint{3.436794in}{3.337534in}}%
\pgfpathlineto{\pgfqpoint{3.441025in}{3.325681in}}%
\pgfpathlineto{\pgfqpoint{3.445257in}{3.304105in}}%
\pgfpathlineto{\pgfqpoint{3.449488in}{3.260038in}}%
\pgfpathlineto{\pgfqpoint{3.453720in}{3.154692in}}%
\pgfpathlineto{\pgfqpoint{3.457951in}{2.843666in}}%
\pgfpathlineto{\pgfqpoint{3.462183in}{1.740723in}}%
\pgfpathlineto{\pgfqpoint{3.466414in}{1.610636in}}%
\pgfpathlineto{\pgfqpoint{3.470646in}{2.756479in}}%
\pgfpathlineto{\pgfqpoint{3.474877in}{1.432004in}}%
\pgfpathlineto{\pgfqpoint{3.479109in}{2.611013in}}%
\pgfpathlineto{\pgfqpoint{3.483340in}{3.345911in}}%
\pgfpathlineto{\pgfqpoint{3.487572in}{3.339736in}}%
\pgfpathlineto{\pgfqpoint{3.491803in}{3.329631in}}%
\pgfpathlineto{\pgfqpoint{3.496035in}{3.311728in}}%
\pgfpathlineto{\pgfqpoint{3.500266in}{3.276493in}}%
\pgfpathlineto{\pgfqpoint{3.504498in}{3.196462in}}%
\pgfpathlineto{\pgfqpoint{3.508729in}{2.975450in}}%
\pgfpathlineto{\pgfqpoint{3.512961in}{2.214345in}}%
\pgfpathlineto{\pgfqpoint{3.517192in}{0.637273in}}%
\pgfpathlineto{\pgfqpoint{3.521424in}{2.139394in}}%
\pgfpathlineto{\pgfqpoint{3.525655in}{0.708641in}}%
\pgfpathlineto{\pgfqpoint{3.529887in}{1.975309in}}%
\pgfpathlineto{\pgfqpoint{3.534118in}{0.968333in}}%
\pgfpathlineto{\pgfqpoint{3.538350in}{2.106271in}}%
\pgfpathlineto{\pgfqpoint{3.542581in}{0.749075in}}%
\pgfpathlineto{\pgfqpoint{3.546813in}{1.959230in}}%
\pgfpathlineto{\pgfqpoint{3.551044in}{1.004382in}}%
\pgfpathlineto{\pgfqpoint{3.555276in}{2.135140in}}%
\pgfpathlineto{\pgfqpoint{3.559507in}{0.712922in}}%
\pgfpathlineto{\pgfqpoint{3.563739in}{1.971773in}}%
\pgfpathlineto{\pgfqpoint{3.567970in}{0.976678in}}%
\pgfpathlineto{\pgfqpoint{3.572202in}{2.112971in}}%
\pgfpathlineto{\pgfqpoint{3.576433in}{0.739658in}}%
\pgfpathlineto{\pgfqpoint{3.580665in}{1.960454in}}%
\pgfpathlineto{\pgfqpoint{3.584896in}{1.002274in}}%
\pgfpathlineto{\pgfqpoint{3.589128in}{2.133614in}}%
\pgfpathlineto{\pgfqpoint{3.593359in}{0.714079in}}%
\pgfpathlineto{\pgfqpoint{3.597591in}{1.970210in}}%
\pgfpathlineto{\pgfqpoint{3.601822in}{0.980829in}}%
\pgfpathlineto{\pgfqpoint{3.606054in}{2.116449in}}%
\pgfpathlineto{\pgfqpoint{3.610285in}{0.734546in}}%
\pgfpathlineto{\pgfqpoint{3.614517in}{1.961084in}}%
\pgfpathlineto{\pgfqpoint{3.618748in}{1.001608in}}%
\pgfpathlineto{\pgfqpoint{3.622980in}{2.133256in}}%
\pgfpathlineto{\pgfqpoint{3.627211in}{0.713824in}}%
\pgfpathlineto{\pgfqpoint{3.631443in}{1.969417in}}%
\pgfpathlineto{\pgfqpoint{3.635674in}{0.983379in}}%
\pgfpathlineto{\pgfqpoint{3.639906in}{2.118650in}}%
\pgfpathlineto{\pgfqpoint{3.644137in}{0.731050in}}%
\pgfpathlineto{\pgfqpoint{3.648369in}{1.961371in}}%
\pgfpathlineto{\pgfqpoint{3.652600in}{1.001799in}}%
\pgfpathlineto{\pgfqpoint{3.656832in}{2.133562in}}%
\pgfpathlineto{\pgfqpoint{3.661063in}{0.712740in}}%
\pgfpathlineto{\pgfqpoint{3.665295in}{1.969066in}}%
\pgfpathlineto{\pgfqpoint{3.669526in}{0.985027in}}%
\pgfpathlineto{\pgfqpoint{3.673758in}{2.120098in}}%
\pgfpathlineto{\pgfqpoint{3.677989in}{0.728467in}}%
\pgfpathlineto{\pgfqpoint{3.682221in}{1.961425in}}%
\pgfpathlineto{\pgfqpoint{3.686452in}{1.002585in}}%
\pgfpathlineto{\pgfqpoint{3.690684in}{2.134302in}}%
\pgfpathlineto{\pgfqpoint{3.694915in}{0.711107in}}%
\pgfpathlineto{\pgfqpoint{3.699147in}{1.969038in}}%
\pgfpathlineto{\pgfqpoint{3.703378in}{0.986023in}}%
\pgfpathlineto{\pgfqpoint{3.707610in}{2.120969in}}%
\pgfpathlineto{\pgfqpoint{3.711841in}{0.726566in}}%
\pgfpathlineto{\pgfqpoint{3.716073in}{1.961292in}}%
\pgfpathlineto{\pgfqpoint{3.720304in}{1.003849in}}%
\pgfpathlineto{\pgfqpoint{3.724536in}{2.135366in}}%
\pgfpathlineto{\pgfqpoint{3.728767in}{0.709070in}}%
\pgfpathlineto{\pgfqpoint{3.732999in}{1.969295in}}%
\pgfpathlineto{\pgfqpoint{3.737230in}{0.986433in}}%
\pgfpathlineto{\pgfqpoint{3.741462in}{2.121304in}}%
\pgfpathlineto{\pgfqpoint{3.745693in}{0.725297in}}%
\pgfpathlineto{\pgfqpoint{3.749925in}{1.960983in}}%
\pgfpathlineto{\pgfqpoint{3.754156in}{1.005553in}}%
\pgfpathlineto{\pgfqpoint{3.758388in}{2.136712in}}%
\pgfpathlineto{\pgfqpoint{3.762619in}{0.706699in}}%
\pgfpathlineto{\pgfqpoint{3.766851in}{1.969846in}}%
\pgfpathlineto{\pgfqpoint{3.771082in}{0.986225in}}%
\pgfpathlineto{\pgfqpoint{3.775314in}{2.121065in}}%
\pgfpathlineto{\pgfqpoint{3.779545in}{0.724715in}}%
\pgfpathlineto{\pgfqpoint{3.783777in}{1.960486in}}%
\pgfpathlineto{\pgfqpoint{3.788008in}{1.007704in}}%
\pgfpathlineto{\pgfqpoint{3.792240in}{2.138339in}}%
\pgfpathlineto{\pgfqpoint{3.796471in}{0.704017in}}%
\pgfpathlineto{\pgfqpoint{3.800703in}{1.970728in}}%
\pgfpathlineto{\pgfqpoint{3.804934in}{0.985301in}}%
\pgfpathlineto{\pgfqpoint{3.809166in}{2.120171in}}%
\pgfpathlineto{\pgfqpoint{3.813397in}{0.724936in}}%
\pgfpathlineto{\pgfqpoint{3.817629in}{1.959785in}}%
\pgfpathlineto{\pgfqpoint{3.821860in}{1.010329in}}%
\pgfpathlineto{\pgfqpoint{3.826092in}{2.140261in}}%
\pgfpathlineto{\pgfqpoint{3.830324in}{0.701032in}}%
\pgfpathlineto{\pgfqpoint{3.834555in}{1.971993in}}%
\pgfpathlineto{\pgfqpoint{3.838787in}{0.983536in}}%
\pgfpathlineto{\pgfqpoint{3.843018in}{2.118520in}}%
\pgfpathlineto{\pgfqpoint{3.847250in}{0.726111in}}%
\pgfpathlineto{\pgfqpoint{3.851481in}{1.958876in}}%
\pgfpathlineto{\pgfqpoint{3.855713in}{1.013425in}}%
\pgfpathlineto{\pgfqpoint{3.859944in}{2.142475in}}%
\pgfpathlineto{\pgfqpoint{3.864176in}{0.697778in}}%
\pgfpathlineto{\pgfqpoint{3.868407in}{1.973683in}}%
\pgfpathlineto{\pgfqpoint{3.872639in}{0.980831in}}%
\pgfpathlineto{\pgfqpoint{3.876870in}{2.116043in}}%
\pgfpathlineto{\pgfqpoint{3.881102in}{0.728367in}}%
\pgfpathlineto{\pgfqpoint{3.885333in}{1.957798in}}%
\pgfpathlineto{\pgfqpoint{3.889565in}{1.016886in}}%
\pgfpathlineto{\pgfqpoint{3.893796in}{2.144893in}}%
\pgfpathlineto{\pgfqpoint{3.898028in}{0.694375in}}%
\pgfpathlineto{\pgfqpoint{3.902259in}{1.975768in}}%
\pgfpathlineto{\pgfqpoint{3.906491in}{0.977246in}}%
\pgfpathlineto{\pgfqpoint{3.910722in}{2.112797in}}%
\pgfpathlineto{\pgfqpoint{3.914954in}{0.731683in}}%
\pgfpathlineto{\pgfqpoint{3.919185in}{1.956674in}}%
\pgfpathlineto{\pgfqpoint{3.923417in}{1.020420in}}%
\pgfpathlineto{\pgfqpoint{3.927648in}{2.147269in}}%
\pgfpathlineto{\pgfqpoint{3.931880in}{0.691102in}}%
\pgfpathlineto{\pgfqpoint{3.936111in}{1.978063in}}%
\pgfpathlineto{\pgfqpoint{3.940343in}{0.973182in}}%
\pgfpathlineto{\pgfqpoint{3.944574in}{2.109106in}}%
\pgfpathlineto{\pgfqpoint{3.948806in}{0.735698in}}%
\pgfpathlineto{\pgfqpoint{3.953037in}{1.955717in}}%
\pgfpathlineto{\pgfqpoint{3.957269in}{1.023523in}}%
\pgfpathlineto{\pgfqpoint{3.961500in}{2.149181in}}%
\pgfpathlineto{\pgfqpoint{3.965732in}{0.688403in}}%
\pgfpathlineto{\pgfqpoint{3.969963in}{1.980172in}}%
\pgfpathlineto{\pgfqpoint{3.974195in}{0.969483in}}%
\pgfpathlineto{\pgfqpoint{3.978426in}{2.105617in}}%
\pgfpathlineto{\pgfqpoint{3.982658in}{0.739590in}}%
\pgfpathlineto{\pgfqpoint{3.986889in}{1.955153in}}%
\pgfpathlineto{\pgfqpoint{3.991121in}{1.025656in}}%
\pgfpathlineto{\pgfqpoint{3.995352in}{2.150171in}}%
\pgfpathlineto{\pgfqpoint{3.999584in}{0.686733in}}%
\pgfpathlineto{\pgfqpoint{4.003815in}{1.981614in}}%
\pgfpathlineto{\pgfqpoint{4.008047in}{0.967167in}}%
\pgfpathlineto{\pgfqpoint{4.012278in}{2.103079in}}%
\pgfpathlineto{\pgfqpoint{4.016510in}{0.742340in}}%
\pgfpathlineto{\pgfqpoint{4.020741in}{1.955095in}}%
\pgfpathlineto{\pgfqpoint{4.024973in}{1.026547in}}%
\pgfpathlineto{\pgfqpoint{4.029204in}{2.150005in}}%
\pgfpathlineto{\pgfqpoint{4.033436in}{0.686316in}}%
\pgfpathlineto{\pgfqpoint{4.037667in}{1.982115in}}%
\pgfpathlineto{\pgfqpoint{4.041899in}{0.966790in}}%
\pgfpathlineto{\pgfqpoint{4.046130in}{2.101889in}}%
\pgfpathlineto{\pgfqpoint{4.050362in}{0.743371in}}%
\pgfpathlineto{\pgfqpoint{4.054593in}{1.955502in}}%
\pgfpathlineto{\pgfqpoint{4.058825in}{1.026271in}}%
\pgfpathlineto{\pgfqpoint{4.063056in}{2.148753in}}%
\pgfpathlineto{\pgfqpoint{4.067288in}{0.687093in}}%
\pgfpathlineto{\pgfqpoint{4.071519in}{1.981739in}}%
\pgfpathlineto{\pgfqpoint{4.075751in}{0.968201in}}%
\pgfpathlineto{\pgfqpoint{4.079982in}{2.101925in}}%
\pgfpathlineto{\pgfqpoint{4.084214in}{0.742825in}}%
\pgfpathlineto{\pgfqpoint{4.088445in}{1.956286in}}%
\pgfpathlineto{\pgfqpoint{4.092677in}{1.025026in}}%
\pgfpathlineto{\pgfqpoint{4.096908in}{2.146591in}}%
\pgfpathlineto{\pgfqpoint{4.101140in}{0.688914in}}%
\pgfpathlineto{\pgfqpoint{4.105371in}{1.980701in}}%
\pgfpathlineto{\pgfqpoint{4.109603in}{0.970927in}}%
\pgfpathlineto{\pgfqpoint{4.113834in}{2.102851in}}%
\pgfpathlineto{\pgfqpoint{4.118066in}{0.741169in}}%
\pgfpathlineto{\pgfqpoint{4.122297in}{1.957386in}}%
\pgfpathlineto{\pgfqpoint{4.126529in}{1.022947in}}%
\pgfpathlineto{\pgfqpoint{4.130760in}{2.143650in}}%
\pgfpathlineto{\pgfqpoint{4.134992in}{0.691688in}}%
\pgfpathlineto{\pgfqpoint{4.139223in}{1.979199in}}%
\pgfpathlineto{\pgfqpoint{4.143455in}{0.974550in}}%
\pgfpathlineto{\pgfqpoint{4.147686in}{2.104378in}}%
\pgfpathlineto{\pgfqpoint{4.151918in}{0.738818in}}%
\pgfpathlineto{\pgfqpoint{4.156149in}{1.958774in}}%
\pgfpathlineto{\pgfqpoint{4.160381in}{1.020096in}}%
\pgfpathlineto{\pgfqpoint{4.164612in}{2.139999in}}%
\pgfpathlineto{\pgfqpoint{4.168844in}{0.695395in}}%
\pgfpathlineto{\pgfqpoint{4.173075in}{1.977390in}}%
\pgfpathlineto{\pgfqpoint{4.177307in}{0.978731in}}%
\pgfpathlineto{\pgfqpoint{4.181538in}{2.106275in}}%
\pgfpathlineto{\pgfqpoint{4.185770in}{0.736100in}}%
\pgfpathlineto{\pgfqpoint{4.190001in}{1.960430in}}%
\pgfpathlineto{\pgfqpoint{4.194233in}{1.016522in}}%
\pgfpathlineto{\pgfqpoint{4.198464in}{2.135702in}}%
\pgfpathlineto{\pgfqpoint{4.202696in}{0.700023in}}%
\pgfpathlineto{\pgfqpoint{4.206927in}{1.975436in}}%
\pgfpathlineto{\pgfqpoint{4.211159in}{0.983123in}}%
\pgfpathlineto{\pgfqpoint{4.215390in}{2.108300in}}%
\pgfpathlineto{\pgfqpoint{4.219622in}{0.733353in}}%
\pgfpathlineto{\pgfqpoint{4.223853in}{1.962307in}}%
\pgfpathlineto{\pgfqpoint{4.228085in}{1.012336in}}%
\pgfpathlineto{\pgfqpoint{4.232316in}{2.130873in}}%
\pgfpathlineto{\pgfqpoint{4.236548in}{0.705501in}}%
\pgfpathlineto{\pgfqpoint{4.240779in}{1.973536in}}%
\pgfpathlineto{\pgfqpoint{4.245011in}{0.987291in}}%
\pgfpathlineto{\pgfqpoint{4.249242in}{2.110135in}}%
\pgfpathlineto{\pgfqpoint{4.253474in}{0.730995in}}%
\pgfpathlineto{\pgfqpoint{4.257705in}{1.964302in}}%
\pgfpathlineto{\pgfqpoint{4.261937in}{1.007774in}}%
\pgfpathlineto{\pgfqpoint{4.266168in}{2.125731in}}%
\pgfpathlineto{\pgfqpoint{4.270400in}{0.711628in}}%
\pgfpathlineto{\pgfqpoint{4.274631in}{1.971923in}}%
\pgfpathlineto{\pgfqpoint{4.278863in}{0.990717in}}%
\pgfpathlineto{\pgfqpoint{4.283094in}{2.111383in}}%
\pgfpathlineto{\pgfqpoint{4.287326in}{0.729530in}}%
\pgfpathlineto{\pgfqpoint{4.291557in}{1.966248in}}%
\pgfpathlineto{\pgfqpoint{4.295789in}{1.003218in}}%
\pgfpathlineto{\pgfqpoint{4.300020in}{2.120604in}}%
\pgfpathlineto{\pgfqpoint{4.304252in}{0.718032in}}%
\pgfpathlineto{\pgfqpoint{4.308483in}{1.970821in}}%
\pgfpathlineto{\pgfqpoint{4.312715in}{0.992901in}}%
\pgfpathlineto{\pgfqpoint{4.316946in}{2.111655in}}%
\pgfpathlineto{\pgfqpoint{4.321178in}{0.729442in}}%
\pgfpathlineto{\pgfqpoint{4.325410in}{1.967958in}}%
\pgfpathlineto{\pgfqpoint{4.329641in}{0.999087in}}%
\pgfpathlineto{\pgfqpoint{4.333873in}{2.115843in}}%
\pgfpathlineto{\pgfqpoint{4.338104in}{0.724270in}}%
\pgfpathlineto{\pgfqpoint{4.342336in}{1.970353in}}%
\pgfpathlineto{\pgfqpoint{4.346567in}{0.993566in}}%
\pgfpathlineto{\pgfqpoint{4.350799in}{2.110734in}}%
\pgfpathlineto{\pgfqpoint{4.355030in}{0.731001in}}%
\pgfpathlineto{\pgfqpoint{4.359262in}{1.969327in}}%
\pgfpathlineto{\pgfqpoint{4.363493in}{0.995609in}}%
\pgfpathlineto{\pgfqpoint{4.367725in}{2.111634in}}%
\pgfpathlineto{\pgfqpoint{4.371956in}{0.730057in}}%
\pgfpathlineto{\pgfqpoint{4.376188in}{1.970480in}}%
\pgfpathlineto{\pgfqpoint{4.380419in}{0.992796in}}%
\pgfpathlineto{\pgfqpoint{4.384651in}{2.108696in}}%
\pgfpathlineto{\pgfqpoint{4.388882in}{0.734127in}}%
\pgfpathlineto{\pgfqpoint{4.393114in}{1.970413in}}%
\pgfpathlineto{\pgfqpoint{4.397345in}{0.992658in}}%
\pgfpathlineto{\pgfqpoint{4.401577in}{2.107883in}}%
\pgfpathlineto{\pgfqpoint{4.405808in}{0.735464in}}%
\pgfpathlineto{\pgfqpoint{4.410040in}{3.382267in}}%
\pgfpathlineto{\pgfqpoint{4.528522in}{3.378556in}}%
\pgfpathlineto{\pgfqpoint{4.702013in}{3.375458in}}%
\pgfpathlineto{\pgfqpoint{4.981293in}{3.373906in}}%
\pgfpathlineto{\pgfqpoint{5.188636in}{3.373964in}}%
\pgfpathlineto{\pgfqpoint{5.188636in}{3.373964in}}%
\pgfusepath{stroke}%
\end{pgfscope}%
\begin{pgfscope}%
\pgfsetrectcap%
\pgfsetmiterjoin%
\pgfsetlinewidth{0.803000pt}%
\definecolor{currentstroke}{rgb}{0.000000,0.000000,0.000000}%
\pgfsetstrokecolor{currentstroke}%
\pgfsetdash{}{0pt}%
\pgfpathmoveto{\pgfqpoint{0.750000in}{0.500000in}}%
\pgfpathlineto{\pgfqpoint{0.750000in}{3.520000in}}%
\pgfusepath{stroke}%
\end{pgfscope}%
\begin{pgfscope}%
\pgfsetrectcap%
\pgfsetmiterjoin%
\pgfsetlinewidth{0.803000pt}%
\definecolor{currentstroke}{rgb}{0.000000,0.000000,0.000000}%
\pgfsetstrokecolor{currentstroke}%
\pgfsetdash{}{0pt}%
\pgfpathmoveto{\pgfqpoint{5.400000in}{0.500000in}}%
\pgfpathlineto{\pgfqpoint{5.400000in}{3.520000in}}%
\pgfusepath{stroke}%
\end{pgfscope}%
\begin{pgfscope}%
\pgfsetrectcap%
\pgfsetmiterjoin%
\pgfsetlinewidth{0.803000pt}%
\definecolor{currentstroke}{rgb}{0.000000,0.000000,0.000000}%
\pgfsetstrokecolor{currentstroke}%
\pgfsetdash{}{0pt}%
\pgfpathmoveto{\pgfqpoint{0.750000in}{0.500000in}}%
\pgfpathlineto{\pgfqpoint{5.400000in}{0.500000in}}%
\pgfusepath{stroke}%
\end{pgfscope}%
\begin{pgfscope}%
\pgfsetrectcap%
\pgfsetmiterjoin%
\pgfsetlinewidth{0.803000pt}%
\definecolor{currentstroke}{rgb}{0.000000,0.000000,0.000000}%
\pgfsetstrokecolor{currentstroke}%
\pgfsetdash{}{0pt}%
\pgfpathmoveto{\pgfqpoint{0.750000in}{3.520000in}}%
\pgfpathlineto{\pgfqpoint{5.400000in}{3.520000in}}%
\pgfusepath{stroke}%
\end{pgfscope}%
\end{pgfpicture}%
\makeatother%
\endgroup%

    \caption{PEC Data Set LL Fault}
    \label{fig:pec_ll_fault}
\end{figure}

Figure \ref{fig:pec_three_phase_sensor} shows the frequency behavior during a three phase sensor fault. In this fault condition, there is nothing wrong with the system itself but the sensor is faulty. This causes the detected frequency to rise slightly. This slight rise is significantly less than the other examined faults and is very close to the reference frequency of 50 Hz. 

\begin{figure}[H]
    %\centering
    %% Creator: Matplotlib, PGF backend
%%
%% To include the figure in your LaTeX document, write
%%   \input{<filename>.pgf}
%%
%% Make sure the required packages are loaded in your preamble
%%   \usepackage{pgf}
%%
%% Also ensure that all the required font packages are loaded; for instance,
%% the lmodern package is sometimes necessary when using math font.
%%   \usepackage{lmodern}
%%
%% Figures using additional raster images can only be included by \input if
%% they are in the same directory as the main LaTeX file. For loading figures
%% from other directories you can use the `import` package
%%   \usepackage{import}
%%
%% and then include the figures with
%%   \import{<path to file>}{<filename>.pgf}
%%
%% Matplotlib used the following preamble
%%
\begingroup%
\makeatletter%
\begin{pgfpicture}%
\pgfpathrectangle{\pgfpointorigin}{\pgfqpoint{6.000000in}{4.000000in}}%
\pgfusepath{use as bounding box, clip}%
\begin{pgfscope}%
\pgfsetbuttcap%
\pgfsetmiterjoin%
\pgfsetlinewidth{0.000000pt}%
\definecolor{currentstroke}{rgb}{1.000000,1.000000,1.000000}%
\pgfsetstrokecolor{currentstroke}%
\pgfsetstrokeopacity{0.000000}%
\pgfsetdash{}{0pt}%
\pgfpathmoveto{\pgfqpoint{0.000000in}{0.000000in}}%
\pgfpathlineto{\pgfqpoint{6.000000in}{0.000000in}}%
\pgfpathlineto{\pgfqpoint{6.000000in}{4.000000in}}%
\pgfpathlineto{\pgfqpoint{0.000000in}{4.000000in}}%
\pgfpathlineto{\pgfqpoint{0.000000in}{0.000000in}}%
\pgfpathclose%
\pgfusepath{}%
\end{pgfscope}%
\begin{pgfscope}%
\pgfsetbuttcap%
\pgfsetmiterjoin%
\definecolor{currentfill}{rgb}{1.000000,1.000000,1.000000}%
\pgfsetfillcolor{currentfill}%
\pgfsetlinewidth{0.000000pt}%
\definecolor{currentstroke}{rgb}{0.000000,0.000000,0.000000}%
\pgfsetstrokecolor{currentstroke}%
\pgfsetstrokeopacity{0.000000}%
\pgfsetdash{}{0pt}%
\pgfpathmoveto{\pgfqpoint{0.750000in}{0.500000in}}%
\pgfpathlineto{\pgfqpoint{5.400000in}{0.500000in}}%
\pgfpathlineto{\pgfqpoint{5.400000in}{3.520000in}}%
\pgfpathlineto{\pgfqpoint{0.750000in}{3.520000in}}%
\pgfpathlineto{\pgfqpoint{0.750000in}{0.500000in}}%
\pgfpathclose%
\pgfusepath{fill}%
\end{pgfscope}%
\begin{pgfscope}%
\pgfsetbuttcap%
\pgfsetroundjoin%
\definecolor{currentfill}{rgb}{0.000000,0.000000,0.000000}%
\pgfsetfillcolor{currentfill}%
\pgfsetlinewidth{0.803000pt}%
\definecolor{currentstroke}{rgb}{0.000000,0.000000,0.000000}%
\pgfsetstrokecolor{currentstroke}%
\pgfsetdash{}{0pt}%
\pgfsys@defobject{currentmarker}{\pgfqpoint{0.000000in}{-0.048611in}}{\pgfqpoint{0.000000in}{0.000000in}}{%
\pgfpathmoveto{\pgfqpoint{0.000000in}{0.000000in}}%
\pgfpathlineto{\pgfqpoint{0.000000in}{-0.048611in}}%
\pgfusepath{stroke,fill}%
}%
\begin{pgfscope}%
\pgfsys@transformshift{1.384099in}{0.500000in}%
\pgfsys@useobject{currentmarker}{}%
\end{pgfscope}%
\end{pgfscope}%
\begin{pgfscope}%
\definecolor{textcolor}{rgb}{0.000000,0.000000,0.000000}%
\pgfsetstrokecolor{textcolor}%
\pgfsetfillcolor{textcolor}%
\pgftext[x=1.384099in,y=0.402778in,,top]{\color{textcolor}\rmfamily\fontsize{10.000000}{12.000000}\selectfont \(\displaystyle {120000}\)}%
\end{pgfscope}%
\begin{pgfscope}%
\pgfsetbuttcap%
\pgfsetroundjoin%
\definecolor{currentfill}{rgb}{0.000000,0.000000,0.000000}%
\pgfsetfillcolor{currentfill}%
\pgfsetlinewidth{0.803000pt}%
\definecolor{currentstroke}{rgb}{0.000000,0.000000,0.000000}%
\pgfsetstrokecolor{currentstroke}%
\pgfsetdash{}{0pt}%
\pgfsys@defobject{currentmarker}{\pgfqpoint{0.000000in}{-0.048611in}}{\pgfqpoint{0.000000in}{0.000000in}}{%
\pgfpathmoveto{\pgfqpoint{0.000000in}{0.000000in}}%
\pgfpathlineto{\pgfqpoint{0.000000in}{-0.048611in}}%
\pgfusepath{stroke,fill}%
}%
\begin{pgfscope}%
\pgfsys@transformshift{2.229571in}{0.500000in}%
\pgfsys@useobject{currentmarker}{}%
\end{pgfscope}%
\end{pgfscope}%
\begin{pgfscope}%
\definecolor{textcolor}{rgb}{0.000000,0.000000,0.000000}%
\pgfsetstrokecolor{textcolor}%
\pgfsetfillcolor{textcolor}%
\pgftext[x=2.229571in,y=0.402778in,,top]{\color{textcolor}\rmfamily\fontsize{10.000000}{12.000000}\selectfont \(\displaystyle {130000}\)}%
\end{pgfscope}%
\begin{pgfscope}%
\pgfsetbuttcap%
\pgfsetroundjoin%
\definecolor{currentfill}{rgb}{0.000000,0.000000,0.000000}%
\pgfsetfillcolor{currentfill}%
\pgfsetlinewidth{0.803000pt}%
\definecolor{currentstroke}{rgb}{0.000000,0.000000,0.000000}%
\pgfsetstrokecolor{currentstroke}%
\pgfsetdash{}{0pt}%
\pgfsys@defobject{currentmarker}{\pgfqpoint{0.000000in}{-0.048611in}}{\pgfqpoint{0.000000in}{0.000000in}}{%
\pgfpathmoveto{\pgfqpoint{0.000000in}{0.000000in}}%
\pgfpathlineto{\pgfqpoint{0.000000in}{-0.048611in}}%
\pgfusepath{stroke,fill}%
}%
\begin{pgfscope}%
\pgfsys@transformshift{3.075042in}{0.500000in}%
\pgfsys@useobject{currentmarker}{}%
\end{pgfscope}%
\end{pgfscope}%
\begin{pgfscope}%
\definecolor{textcolor}{rgb}{0.000000,0.000000,0.000000}%
\pgfsetstrokecolor{textcolor}%
\pgfsetfillcolor{textcolor}%
\pgftext[x=3.075042in,y=0.402778in,,top]{\color{textcolor}\rmfamily\fontsize{10.000000}{12.000000}\selectfont \(\displaystyle {140000}\)}%
\end{pgfscope}%
\begin{pgfscope}%
\pgfsetbuttcap%
\pgfsetroundjoin%
\definecolor{currentfill}{rgb}{0.000000,0.000000,0.000000}%
\pgfsetfillcolor{currentfill}%
\pgfsetlinewidth{0.803000pt}%
\definecolor{currentstroke}{rgb}{0.000000,0.000000,0.000000}%
\pgfsetstrokecolor{currentstroke}%
\pgfsetdash{}{0pt}%
\pgfsys@defobject{currentmarker}{\pgfqpoint{0.000000in}{-0.048611in}}{\pgfqpoint{0.000000in}{0.000000in}}{%
\pgfpathmoveto{\pgfqpoint{0.000000in}{0.000000in}}%
\pgfpathlineto{\pgfqpoint{0.000000in}{-0.048611in}}%
\pgfusepath{stroke,fill}%
}%
\begin{pgfscope}%
\pgfsys@transformshift{3.920514in}{0.500000in}%
\pgfsys@useobject{currentmarker}{}%
\end{pgfscope}%
\end{pgfscope}%
\begin{pgfscope}%
\definecolor{textcolor}{rgb}{0.000000,0.000000,0.000000}%
\pgfsetstrokecolor{textcolor}%
\pgfsetfillcolor{textcolor}%
\pgftext[x=3.920514in,y=0.402778in,,top]{\color{textcolor}\rmfamily\fontsize{10.000000}{12.000000}\selectfont \(\displaystyle {150000}\)}%
\end{pgfscope}%
\begin{pgfscope}%
\pgfsetbuttcap%
\pgfsetroundjoin%
\definecolor{currentfill}{rgb}{0.000000,0.000000,0.000000}%
\pgfsetfillcolor{currentfill}%
\pgfsetlinewidth{0.803000pt}%
\definecolor{currentstroke}{rgb}{0.000000,0.000000,0.000000}%
\pgfsetstrokecolor{currentstroke}%
\pgfsetdash{}{0pt}%
\pgfsys@defobject{currentmarker}{\pgfqpoint{0.000000in}{-0.048611in}}{\pgfqpoint{0.000000in}{0.000000in}}{%
\pgfpathmoveto{\pgfqpoint{0.000000in}{0.000000in}}%
\pgfpathlineto{\pgfqpoint{0.000000in}{-0.048611in}}%
\pgfusepath{stroke,fill}%
}%
\begin{pgfscope}%
\pgfsys@transformshift{4.765985in}{0.500000in}%
\pgfsys@useobject{currentmarker}{}%
\end{pgfscope}%
\end{pgfscope}%
\begin{pgfscope}%
\definecolor{textcolor}{rgb}{0.000000,0.000000,0.000000}%
\pgfsetstrokecolor{textcolor}%
\pgfsetfillcolor{textcolor}%
\pgftext[x=4.765985in,y=0.402778in,,top]{\color{textcolor}\rmfamily\fontsize{10.000000}{12.000000}\selectfont \(\displaystyle {160000}\)}%
\end{pgfscope}%
\begin{pgfscope}%
\definecolor{textcolor}{rgb}{0.000000,0.000000,0.000000}%
\pgfsetstrokecolor{textcolor}%
\pgfsetfillcolor{textcolor}%
\pgftext[x=3.075000in,y=0.223766in,,top]{\color{textcolor}\rmfamily\fontsize{10.000000}{12.000000}\selectfont Time (s)}%
\end{pgfscope}%
\begin{pgfscope}%
\pgfsetbuttcap%
\pgfsetroundjoin%
\definecolor{currentfill}{rgb}{0.000000,0.000000,0.000000}%
\pgfsetfillcolor{currentfill}%
\pgfsetlinewidth{0.803000pt}%
\definecolor{currentstroke}{rgb}{0.000000,0.000000,0.000000}%
\pgfsetstrokecolor{currentstroke}%
\pgfsetdash{}{0pt}%
\pgfsys@defobject{currentmarker}{\pgfqpoint{-0.048611in}{0.000000in}}{\pgfqpoint{-0.000000in}{0.000000in}}{%
\pgfpathmoveto{\pgfqpoint{-0.000000in}{0.000000in}}%
\pgfpathlineto{\pgfqpoint{-0.048611in}{0.000000in}}%
\pgfusepath{stroke,fill}%
}%
\begin{pgfscope}%
\pgfsys@transformshift{0.750000in}{0.637276in}%
\pgfsys@useobject{currentmarker}{}%
\end{pgfscope}%
\end{pgfscope}%
\begin{pgfscope}%
\definecolor{textcolor}{rgb}{0.000000,0.000000,0.000000}%
\pgfsetstrokecolor{textcolor}%
\pgfsetfillcolor{textcolor}%
\pgftext[x=0.336419in, y=0.589050in, left, base]{\color{textcolor}\rmfamily\fontsize{10.000000}{12.000000}\selectfont \(\displaystyle {50.00}\)}%
\end{pgfscope}%
\begin{pgfscope}%
\pgfsetbuttcap%
\pgfsetroundjoin%
\definecolor{currentfill}{rgb}{0.000000,0.000000,0.000000}%
\pgfsetfillcolor{currentfill}%
\pgfsetlinewidth{0.803000pt}%
\definecolor{currentstroke}{rgb}{0.000000,0.000000,0.000000}%
\pgfsetstrokecolor{currentstroke}%
\pgfsetdash{}{0pt}%
\pgfsys@defobject{currentmarker}{\pgfqpoint{-0.048611in}{0.000000in}}{\pgfqpoint{-0.000000in}{0.000000in}}{%
\pgfpathmoveto{\pgfqpoint{-0.000000in}{0.000000in}}%
\pgfpathlineto{\pgfqpoint{-0.048611in}{0.000000in}}%
\pgfusepath{stroke,fill}%
}%
\begin{pgfscope}%
\pgfsys@transformshift{0.750000in}{1.078880in}%
\pgfsys@useobject{currentmarker}{}%
\end{pgfscope}%
\end{pgfscope}%
\begin{pgfscope}%
\definecolor{textcolor}{rgb}{0.000000,0.000000,0.000000}%
\pgfsetstrokecolor{textcolor}%
\pgfsetfillcolor{textcolor}%
\pgftext[x=0.336419in, y=1.030655in, left, base]{\color{textcolor}\rmfamily\fontsize{10.000000}{12.000000}\selectfont \(\displaystyle {50.02}\)}%
\end{pgfscope}%
\begin{pgfscope}%
\pgfsetbuttcap%
\pgfsetroundjoin%
\definecolor{currentfill}{rgb}{0.000000,0.000000,0.000000}%
\pgfsetfillcolor{currentfill}%
\pgfsetlinewidth{0.803000pt}%
\definecolor{currentstroke}{rgb}{0.000000,0.000000,0.000000}%
\pgfsetstrokecolor{currentstroke}%
\pgfsetdash{}{0pt}%
\pgfsys@defobject{currentmarker}{\pgfqpoint{-0.048611in}{0.000000in}}{\pgfqpoint{-0.000000in}{0.000000in}}{%
\pgfpathmoveto{\pgfqpoint{-0.000000in}{0.000000in}}%
\pgfpathlineto{\pgfqpoint{-0.048611in}{0.000000in}}%
\pgfusepath{stroke,fill}%
}%
\begin{pgfscope}%
\pgfsys@transformshift{0.750000in}{1.520485in}%
\pgfsys@useobject{currentmarker}{}%
\end{pgfscope}%
\end{pgfscope}%
\begin{pgfscope}%
\definecolor{textcolor}{rgb}{0.000000,0.000000,0.000000}%
\pgfsetstrokecolor{textcolor}%
\pgfsetfillcolor{textcolor}%
\pgftext[x=0.336419in, y=1.472260in, left, base]{\color{textcolor}\rmfamily\fontsize{10.000000}{12.000000}\selectfont \(\displaystyle {50.04}\)}%
\end{pgfscope}%
\begin{pgfscope}%
\pgfsetbuttcap%
\pgfsetroundjoin%
\definecolor{currentfill}{rgb}{0.000000,0.000000,0.000000}%
\pgfsetfillcolor{currentfill}%
\pgfsetlinewidth{0.803000pt}%
\definecolor{currentstroke}{rgb}{0.000000,0.000000,0.000000}%
\pgfsetstrokecolor{currentstroke}%
\pgfsetdash{}{0pt}%
\pgfsys@defobject{currentmarker}{\pgfqpoint{-0.048611in}{0.000000in}}{\pgfqpoint{-0.000000in}{0.000000in}}{%
\pgfpathmoveto{\pgfqpoint{-0.000000in}{0.000000in}}%
\pgfpathlineto{\pgfqpoint{-0.048611in}{0.000000in}}%
\pgfusepath{stroke,fill}%
}%
\begin{pgfscope}%
\pgfsys@transformshift{0.750000in}{1.962090in}%
\pgfsys@useobject{currentmarker}{}%
\end{pgfscope}%
\end{pgfscope}%
\begin{pgfscope}%
\definecolor{textcolor}{rgb}{0.000000,0.000000,0.000000}%
\pgfsetstrokecolor{textcolor}%
\pgfsetfillcolor{textcolor}%
\pgftext[x=0.336419in, y=1.913864in, left, base]{\color{textcolor}\rmfamily\fontsize{10.000000}{12.000000}\selectfont \(\displaystyle {50.06}\)}%
\end{pgfscope}%
\begin{pgfscope}%
\pgfsetbuttcap%
\pgfsetroundjoin%
\definecolor{currentfill}{rgb}{0.000000,0.000000,0.000000}%
\pgfsetfillcolor{currentfill}%
\pgfsetlinewidth{0.803000pt}%
\definecolor{currentstroke}{rgb}{0.000000,0.000000,0.000000}%
\pgfsetstrokecolor{currentstroke}%
\pgfsetdash{}{0pt}%
\pgfsys@defobject{currentmarker}{\pgfqpoint{-0.048611in}{0.000000in}}{\pgfqpoint{-0.000000in}{0.000000in}}{%
\pgfpathmoveto{\pgfqpoint{-0.000000in}{0.000000in}}%
\pgfpathlineto{\pgfqpoint{-0.048611in}{0.000000in}}%
\pgfusepath{stroke,fill}%
}%
\begin{pgfscope}%
\pgfsys@transformshift{0.750000in}{2.403694in}%
\pgfsys@useobject{currentmarker}{}%
\end{pgfscope}%
\end{pgfscope}%
\begin{pgfscope}%
\definecolor{textcolor}{rgb}{0.000000,0.000000,0.000000}%
\pgfsetstrokecolor{textcolor}%
\pgfsetfillcolor{textcolor}%
\pgftext[x=0.336419in, y=2.355469in, left, base]{\color{textcolor}\rmfamily\fontsize{10.000000}{12.000000}\selectfont \(\displaystyle {50.08}\)}%
\end{pgfscope}%
\begin{pgfscope}%
\pgfsetbuttcap%
\pgfsetroundjoin%
\definecolor{currentfill}{rgb}{0.000000,0.000000,0.000000}%
\pgfsetfillcolor{currentfill}%
\pgfsetlinewidth{0.803000pt}%
\definecolor{currentstroke}{rgb}{0.000000,0.000000,0.000000}%
\pgfsetstrokecolor{currentstroke}%
\pgfsetdash{}{0pt}%
\pgfsys@defobject{currentmarker}{\pgfqpoint{-0.048611in}{0.000000in}}{\pgfqpoint{-0.000000in}{0.000000in}}{%
\pgfpathmoveto{\pgfqpoint{-0.000000in}{0.000000in}}%
\pgfpathlineto{\pgfqpoint{-0.048611in}{0.000000in}}%
\pgfusepath{stroke,fill}%
}%
\begin{pgfscope}%
\pgfsys@transformshift{0.750000in}{2.845299in}%
\pgfsys@useobject{currentmarker}{}%
\end{pgfscope}%
\end{pgfscope}%
\begin{pgfscope}%
\definecolor{textcolor}{rgb}{0.000000,0.000000,0.000000}%
\pgfsetstrokecolor{textcolor}%
\pgfsetfillcolor{textcolor}%
\pgftext[x=0.336419in, y=2.797074in, left, base]{\color{textcolor}\rmfamily\fontsize{10.000000}{12.000000}\selectfont \(\displaystyle {50.10}\)}%
\end{pgfscope}%
\begin{pgfscope}%
\pgfsetbuttcap%
\pgfsetroundjoin%
\definecolor{currentfill}{rgb}{0.000000,0.000000,0.000000}%
\pgfsetfillcolor{currentfill}%
\pgfsetlinewidth{0.803000pt}%
\definecolor{currentstroke}{rgb}{0.000000,0.000000,0.000000}%
\pgfsetstrokecolor{currentstroke}%
\pgfsetdash{}{0pt}%
\pgfsys@defobject{currentmarker}{\pgfqpoint{-0.048611in}{0.000000in}}{\pgfqpoint{-0.000000in}{0.000000in}}{%
\pgfpathmoveto{\pgfqpoint{-0.000000in}{0.000000in}}%
\pgfpathlineto{\pgfqpoint{-0.048611in}{0.000000in}}%
\pgfusepath{stroke,fill}%
}%
\begin{pgfscope}%
\pgfsys@transformshift{0.750000in}{3.286904in}%
\pgfsys@useobject{currentmarker}{}%
\end{pgfscope}%
\end{pgfscope}%
\begin{pgfscope}%
\definecolor{textcolor}{rgb}{0.000000,0.000000,0.000000}%
\pgfsetstrokecolor{textcolor}%
\pgfsetfillcolor{textcolor}%
\pgftext[x=0.336419in, y=3.238678in, left, base]{\color{textcolor}\rmfamily\fontsize{10.000000}{12.000000}\selectfont \(\displaystyle {50.12}\)}%
\end{pgfscope}%
\begin{pgfscope}%
\definecolor{textcolor}{rgb}{0.000000,0.000000,0.000000}%
\pgfsetstrokecolor{textcolor}%
\pgfsetfillcolor{textcolor}%
\pgftext[x=0.280863in,y=2.010000in,,bottom,rotate=90.000000]{\color{textcolor}\rmfamily\fontsize{10.000000}{12.000000}\selectfont Frequency (Hz)}%
\end{pgfscope}%
\begin{pgfscope}%
\pgfpathrectangle{\pgfqpoint{0.750000in}{0.500000in}}{\pgfqpoint{4.650000in}{3.020000in}}%
\pgfusepath{clip}%
\pgfsetrectcap%
\pgfsetroundjoin%
\pgfsetlinewidth{1.505625pt}%
\definecolor{currentstroke}{rgb}{0.121569,0.466667,0.705882}%
\pgfsetstrokecolor{currentstroke}%
\pgfsetdash{}{0pt}%
\pgfpathmoveto{\pgfqpoint{0.961364in}{0.637276in}}%
\pgfpathlineto{\pgfqpoint{1.387228in}{0.637276in}}%
\pgfpathlineto{\pgfqpoint{1.388834in}{3.382727in}}%
\pgfpathlineto{\pgfqpoint{4.770382in}{3.382727in}}%
\pgfpathlineto{\pgfqpoint{4.771988in}{0.637279in}}%
\pgfpathlineto{\pgfqpoint{5.188636in}{0.637276in}}%
\pgfpathlineto{\pgfqpoint{5.188636in}{0.637276in}}%
\pgfusepath{stroke}%
\end{pgfscope}%
\begin{pgfscope}%
\pgfsetrectcap%
\pgfsetmiterjoin%
\pgfsetlinewidth{0.803000pt}%
\definecolor{currentstroke}{rgb}{0.000000,0.000000,0.000000}%
\pgfsetstrokecolor{currentstroke}%
\pgfsetdash{}{0pt}%
\pgfpathmoveto{\pgfqpoint{0.750000in}{0.500000in}}%
\pgfpathlineto{\pgfqpoint{0.750000in}{3.520000in}}%
\pgfusepath{stroke}%
\end{pgfscope}%
\begin{pgfscope}%
\pgfsetrectcap%
\pgfsetmiterjoin%
\pgfsetlinewidth{0.803000pt}%
\definecolor{currentstroke}{rgb}{0.000000,0.000000,0.000000}%
\pgfsetstrokecolor{currentstroke}%
\pgfsetdash{}{0pt}%
\pgfpathmoveto{\pgfqpoint{5.400000in}{0.500000in}}%
\pgfpathlineto{\pgfqpoint{5.400000in}{3.520000in}}%
\pgfusepath{stroke}%
\end{pgfscope}%
\begin{pgfscope}%
\pgfsetrectcap%
\pgfsetmiterjoin%
\pgfsetlinewidth{0.803000pt}%
\definecolor{currentstroke}{rgb}{0.000000,0.000000,0.000000}%
\pgfsetstrokecolor{currentstroke}%
\pgfsetdash{}{0pt}%
\pgfpathmoveto{\pgfqpoint{0.750000in}{0.500000in}}%
\pgfpathlineto{\pgfqpoint{5.400000in}{0.500000in}}%
\pgfusepath{stroke}%
\end{pgfscope}%
\begin{pgfscope}%
\pgfsetrectcap%
\pgfsetmiterjoin%
\pgfsetlinewidth{0.803000pt}%
\definecolor{currentstroke}{rgb}{0.000000,0.000000,0.000000}%
\pgfsetstrokecolor{currentstroke}%
\pgfsetdash{}{0pt}%
\pgfpathmoveto{\pgfqpoint{0.750000in}{3.520000in}}%
\pgfpathlineto{\pgfqpoint{5.400000in}{3.520000in}}%
\pgfusepath{stroke}%
\end{pgfscope}%
\end{pgfpicture}%
\makeatother%
\endgroup%

    \caption{PEC Data Set Three Phase Sensor Fault}
    \label{fig:pec_three_phase_sensor}
\end{figure}

Figure \ref{fig:pec_three_phase_grid_fault} shows the frequency behavior of the system during a single phase voltage sag. When this occurs, the frequency oscillates continuously until the fault is over. This is a significant fault in the system and detection is critical to take remedial action. This is another fault where the magnitude is not very large in comparison to the LL Fault of the Three Phase Grid Fault.

\begin{figure}[H]
    %\centering
    %% Creator: Matplotlib, PGF backend
%%
%% To include the figure in your LaTeX document, write
%%   \input{<filename>.pgf}
%%
%% Make sure the required packages are loaded in your preamble
%%   \usepackage{pgf}
%%
%% Also ensure that all the required font packages are loaded; for instance,
%% the lmodern package is sometimes necessary when using math font.
%%   \usepackage{lmodern}
%%
%% Figures using additional raster images can only be included by \input if
%% they are in the same directory as the main LaTeX file. For loading figures
%% from other directories you can use the `import` package
%%   \usepackage{import}
%%
%% and then include the figures with
%%   \import{<path to file>}{<filename>.pgf}
%%
%% Matplotlib used the following preamble
%%
\begingroup%
\makeatletter%
\begin{pgfpicture}%
\pgfpathrectangle{\pgfpointorigin}{\pgfqpoint{5.000000in}{4.000000in}}%
\pgfusepath{use as bounding box, clip}%
\begin{pgfscope}%
\pgfsetbuttcap%
\pgfsetmiterjoin%
\pgfsetlinewidth{0.000000pt}%
\definecolor{currentstroke}{rgb}{1.000000,1.000000,1.000000}%
\pgfsetstrokecolor{currentstroke}%
\pgfsetstrokeopacity{0.000000}%
\pgfsetdash{}{0pt}%
\pgfpathmoveto{\pgfqpoint{0.000000in}{0.000000in}}%
\pgfpathlineto{\pgfqpoint{5.000000in}{0.000000in}}%
\pgfpathlineto{\pgfqpoint{5.000000in}{4.000000in}}%
\pgfpathlineto{\pgfqpoint{0.000000in}{4.000000in}}%
\pgfpathlineto{\pgfqpoint{0.000000in}{0.000000in}}%
\pgfpathclose%
\pgfusepath{}%
\end{pgfscope}%
\begin{pgfscope}%
\pgfsetbuttcap%
\pgfsetmiterjoin%
\definecolor{currentfill}{rgb}{1.000000,1.000000,1.000000}%
\pgfsetfillcolor{currentfill}%
\pgfsetlinewidth{0.000000pt}%
\definecolor{currentstroke}{rgb}{0.000000,0.000000,0.000000}%
\pgfsetstrokecolor{currentstroke}%
\pgfsetstrokeopacity{0.000000}%
\pgfsetdash{}{0pt}%
\pgfpathmoveto{\pgfqpoint{0.625000in}{0.500000in}}%
\pgfpathlineto{\pgfqpoint{4.500000in}{0.500000in}}%
\pgfpathlineto{\pgfqpoint{4.500000in}{3.520000in}}%
\pgfpathlineto{\pgfqpoint{0.625000in}{3.520000in}}%
\pgfpathlineto{\pgfqpoint{0.625000in}{0.500000in}}%
\pgfpathclose%
\pgfusepath{fill}%
\end{pgfscope}%
\begin{pgfscope}%
\pgfsetbuttcap%
\pgfsetroundjoin%
\definecolor{currentfill}{rgb}{0.000000,0.000000,0.000000}%
\pgfsetfillcolor{currentfill}%
\pgfsetlinewidth{0.803000pt}%
\definecolor{currentstroke}{rgb}{0.000000,0.000000,0.000000}%
\pgfsetstrokecolor{currentstroke}%
\pgfsetdash{}{0pt}%
\pgfsys@defobject{currentmarker}{\pgfqpoint{0.000000in}{-0.048611in}}{\pgfqpoint{0.000000in}{0.000000in}}{%
\pgfpathmoveto{\pgfqpoint{0.000000in}{0.000000in}}%
\pgfpathlineto{\pgfqpoint{0.000000in}{-0.048611in}}%
\pgfusepath{stroke,fill}%
}%
\begin{pgfscope}%
\pgfsys@transformshift{1.088712in}{0.500000in}%
\pgfsys@useobject{currentmarker}{}%
\end{pgfscope}%
\end{pgfscope}%
\begin{pgfscope}%
\definecolor{textcolor}{rgb}{0.000000,0.000000,0.000000}%
\pgfsetstrokecolor{textcolor}%
\pgfsetfillcolor{textcolor}%
\pgftext[x=1.088712in,y=0.402778in,,top]{\color{textcolor}\rmfamily\fontsize{10.000000}{12.000000}\selectfont \(\displaystyle {200000}\)}%
\end{pgfscope}%
\begin{pgfscope}%
\pgfsetbuttcap%
\pgfsetroundjoin%
\definecolor{currentfill}{rgb}{0.000000,0.000000,0.000000}%
\pgfsetfillcolor{currentfill}%
\pgfsetlinewidth{0.803000pt}%
\definecolor{currentstroke}{rgb}{0.000000,0.000000,0.000000}%
\pgfsetstrokecolor{currentstroke}%
\pgfsetdash{}{0pt}%
\pgfsys@defobject{currentmarker}{\pgfqpoint{0.000000in}{-0.048611in}}{\pgfqpoint{0.000000in}{0.000000in}}{%
\pgfpathmoveto{\pgfqpoint{0.000000in}{0.000000in}}%
\pgfpathlineto{\pgfqpoint{0.000000in}{-0.048611in}}%
\pgfusepath{stroke,fill}%
}%
\begin{pgfscope}%
\pgfsys@transformshift{1.807650in}{0.500000in}%
\pgfsys@useobject{currentmarker}{}%
\end{pgfscope}%
\end{pgfscope}%
\begin{pgfscope}%
\definecolor{textcolor}{rgb}{0.000000,0.000000,0.000000}%
\pgfsetstrokecolor{textcolor}%
\pgfsetfillcolor{textcolor}%
\pgftext[x=1.807650in,y=0.402778in,,top]{\color{textcolor}\rmfamily\fontsize{10.000000}{12.000000}\selectfont \(\displaystyle {210000}\)}%
\end{pgfscope}%
\begin{pgfscope}%
\pgfsetbuttcap%
\pgfsetroundjoin%
\definecolor{currentfill}{rgb}{0.000000,0.000000,0.000000}%
\pgfsetfillcolor{currentfill}%
\pgfsetlinewidth{0.803000pt}%
\definecolor{currentstroke}{rgb}{0.000000,0.000000,0.000000}%
\pgfsetstrokecolor{currentstroke}%
\pgfsetdash{}{0pt}%
\pgfsys@defobject{currentmarker}{\pgfqpoint{0.000000in}{-0.048611in}}{\pgfqpoint{0.000000in}{0.000000in}}{%
\pgfpathmoveto{\pgfqpoint{0.000000in}{0.000000in}}%
\pgfpathlineto{\pgfqpoint{0.000000in}{-0.048611in}}%
\pgfusepath{stroke,fill}%
}%
\begin{pgfscope}%
\pgfsys@transformshift{2.526589in}{0.500000in}%
\pgfsys@useobject{currentmarker}{}%
\end{pgfscope}%
\end{pgfscope}%
\begin{pgfscope}%
\definecolor{textcolor}{rgb}{0.000000,0.000000,0.000000}%
\pgfsetstrokecolor{textcolor}%
\pgfsetfillcolor{textcolor}%
\pgftext[x=2.526589in,y=0.402778in,,top]{\color{textcolor}\rmfamily\fontsize{10.000000}{12.000000}\selectfont \(\displaystyle {220000}\)}%
\end{pgfscope}%
\begin{pgfscope}%
\pgfsetbuttcap%
\pgfsetroundjoin%
\definecolor{currentfill}{rgb}{0.000000,0.000000,0.000000}%
\pgfsetfillcolor{currentfill}%
\pgfsetlinewidth{0.803000pt}%
\definecolor{currentstroke}{rgb}{0.000000,0.000000,0.000000}%
\pgfsetstrokecolor{currentstroke}%
\pgfsetdash{}{0pt}%
\pgfsys@defobject{currentmarker}{\pgfqpoint{0.000000in}{-0.048611in}}{\pgfqpoint{0.000000in}{0.000000in}}{%
\pgfpathmoveto{\pgfqpoint{0.000000in}{0.000000in}}%
\pgfpathlineto{\pgfqpoint{0.000000in}{-0.048611in}}%
\pgfusepath{stroke,fill}%
}%
\begin{pgfscope}%
\pgfsys@transformshift{3.245528in}{0.500000in}%
\pgfsys@useobject{currentmarker}{}%
\end{pgfscope}%
\end{pgfscope}%
\begin{pgfscope}%
\definecolor{textcolor}{rgb}{0.000000,0.000000,0.000000}%
\pgfsetstrokecolor{textcolor}%
\pgfsetfillcolor{textcolor}%
\pgftext[x=3.245528in,y=0.402778in,,top]{\color{textcolor}\rmfamily\fontsize{10.000000}{12.000000}\selectfont \(\displaystyle {230000}\)}%
\end{pgfscope}%
\begin{pgfscope}%
\pgfsetbuttcap%
\pgfsetroundjoin%
\definecolor{currentfill}{rgb}{0.000000,0.000000,0.000000}%
\pgfsetfillcolor{currentfill}%
\pgfsetlinewidth{0.803000pt}%
\definecolor{currentstroke}{rgb}{0.000000,0.000000,0.000000}%
\pgfsetstrokecolor{currentstroke}%
\pgfsetdash{}{0pt}%
\pgfsys@defobject{currentmarker}{\pgfqpoint{0.000000in}{-0.048611in}}{\pgfqpoint{0.000000in}{0.000000in}}{%
\pgfpathmoveto{\pgfqpoint{0.000000in}{0.000000in}}%
\pgfpathlineto{\pgfqpoint{0.000000in}{-0.048611in}}%
\pgfusepath{stroke,fill}%
}%
\begin{pgfscope}%
\pgfsys@transformshift{3.964466in}{0.500000in}%
\pgfsys@useobject{currentmarker}{}%
\end{pgfscope}%
\end{pgfscope}%
\begin{pgfscope}%
\definecolor{textcolor}{rgb}{0.000000,0.000000,0.000000}%
\pgfsetstrokecolor{textcolor}%
\pgfsetfillcolor{textcolor}%
\pgftext[x=3.964466in,y=0.402778in,,top]{\color{textcolor}\rmfamily\fontsize{10.000000}{12.000000}\selectfont \(\displaystyle {240000}\)}%
\end{pgfscope}%
\begin{pgfscope}%
\definecolor{textcolor}{rgb}{0.000000,0.000000,0.000000}%
\pgfsetstrokecolor{textcolor}%
\pgfsetfillcolor{textcolor}%
\pgftext[x=2.562500in,y=0.223766in,,top]{\color{textcolor}\rmfamily\fontsize{10.000000}{12.000000}\selectfont Time (s)}%
\end{pgfscope}%
\begin{pgfscope}%
\pgfsetbuttcap%
\pgfsetroundjoin%
\definecolor{currentfill}{rgb}{0.000000,0.000000,0.000000}%
\pgfsetfillcolor{currentfill}%
\pgfsetlinewidth{0.803000pt}%
\definecolor{currentstroke}{rgb}{0.000000,0.000000,0.000000}%
\pgfsetstrokecolor{currentstroke}%
\pgfsetdash{}{0pt}%
\pgfsys@defobject{currentmarker}{\pgfqpoint{-0.048611in}{0.000000in}}{\pgfqpoint{-0.000000in}{0.000000in}}{%
\pgfpathmoveto{\pgfqpoint{-0.000000in}{0.000000in}}%
\pgfpathlineto{\pgfqpoint{-0.048611in}{0.000000in}}%
\pgfusepath{stroke,fill}%
}%
\begin{pgfscope}%
\pgfsys@transformshift{0.625000in}{0.776091in}%
\pgfsys@useobject{currentmarker}{}%
\end{pgfscope}%
\end{pgfscope}%
\begin{pgfscope}%
\definecolor{textcolor}{rgb}{0.000000,0.000000,0.000000}%
\pgfsetstrokecolor{textcolor}%
\pgfsetfillcolor{textcolor}%
\pgftext[x=0.388888in, y=0.727866in, left, base]{\color{textcolor}\rmfamily\fontsize{10.000000}{12.000000}\selectfont \(\displaystyle {46}\)}%
\end{pgfscope}%
\begin{pgfscope}%
\pgfsetbuttcap%
\pgfsetroundjoin%
\definecolor{currentfill}{rgb}{0.000000,0.000000,0.000000}%
\pgfsetfillcolor{currentfill}%
\pgfsetlinewidth{0.803000pt}%
\definecolor{currentstroke}{rgb}{0.000000,0.000000,0.000000}%
\pgfsetstrokecolor{currentstroke}%
\pgfsetdash{}{0pt}%
\pgfsys@defobject{currentmarker}{\pgfqpoint{-0.048611in}{0.000000in}}{\pgfqpoint{-0.000000in}{0.000000in}}{%
\pgfpathmoveto{\pgfqpoint{-0.000000in}{0.000000in}}%
\pgfpathlineto{\pgfqpoint{-0.048611in}{0.000000in}}%
\pgfusepath{stroke,fill}%
}%
\begin{pgfscope}%
\pgfsys@transformshift{0.625000in}{1.371375in}%
\pgfsys@useobject{currentmarker}{}%
\end{pgfscope}%
\end{pgfscope}%
\begin{pgfscope}%
\definecolor{textcolor}{rgb}{0.000000,0.000000,0.000000}%
\pgfsetstrokecolor{textcolor}%
\pgfsetfillcolor{textcolor}%
\pgftext[x=0.388888in, y=1.323149in, left, base]{\color{textcolor}\rmfamily\fontsize{10.000000}{12.000000}\selectfont \(\displaystyle {48}\)}%
\end{pgfscope}%
\begin{pgfscope}%
\pgfsetbuttcap%
\pgfsetroundjoin%
\definecolor{currentfill}{rgb}{0.000000,0.000000,0.000000}%
\pgfsetfillcolor{currentfill}%
\pgfsetlinewidth{0.803000pt}%
\definecolor{currentstroke}{rgb}{0.000000,0.000000,0.000000}%
\pgfsetstrokecolor{currentstroke}%
\pgfsetdash{}{0pt}%
\pgfsys@defobject{currentmarker}{\pgfqpoint{-0.048611in}{0.000000in}}{\pgfqpoint{-0.000000in}{0.000000in}}{%
\pgfpathmoveto{\pgfqpoint{-0.000000in}{0.000000in}}%
\pgfpathlineto{\pgfqpoint{-0.048611in}{0.000000in}}%
\pgfusepath{stroke,fill}%
}%
\begin{pgfscope}%
\pgfsys@transformshift{0.625000in}{1.966658in}%
\pgfsys@useobject{currentmarker}{}%
\end{pgfscope}%
\end{pgfscope}%
\begin{pgfscope}%
\definecolor{textcolor}{rgb}{0.000000,0.000000,0.000000}%
\pgfsetstrokecolor{textcolor}%
\pgfsetfillcolor{textcolor}%
\pgftext[x=0.388888in, y=1.918433in, left, base]{\color{textcolor}\rmfamily\fontsize{10.000000}{12.000000}\selectfont \(\displaystyle {50}\)}%
\end{pgfscope}%
\begin{pgfscope}%
\pgfsetbuttcap%
\pgfsetroundjoin%
\definecolor{currentfill}{rgb}{0.000000,0.000000,0.000000}%
\pgfsetfillcolor{currentfill}%
\pgfsetlinewidth{0.803000pt}%
\definecolor{currentstroke}{rgb}{0.000000,0.000000,0.000000}%
\pgfsetstrokecolor{currentstroke}%
\pgfsetdash{}{0pt}%
\pgfsys@defobject{currentmarker}{\pgfqpoint{-0.048611in}{0.000000in}}{\pgfqpoint{-0.000000in}{0.000000in}}{%
\pgfpathmoveto{\pgfqpoint{-0.000000in}{0.000000in}}%
\pgfpathlineto{\pgfqpoint{-0.048611in}{0.000000in}}%
\pgfusepath{stroke,fill}%
}%
\begin{pgfscope}%
\pgfsys@transformshift{0.625000in}{2.561942in}%
\pgfsys@useobject{currentmarker}{}%
\end{pgfscope}%
\end{pgfscope}%
\begin{pgfscope}%
\definecolor{textcolor}{rgb}{0.000000,0.000000,0.000000}%
\pgfsetstrokecolor{textcolor}%
\pgfsetfillcolor{textcolor}%
\pgftext[x=0.388888in, y=2.513716in, left, base]{\color{textcolor}\rmfamily\fontsize{10.000000}{12.000000}\selectfont \(\displaystyle {52}\)}%
\end{pgfscope}%
\begin{pgfscope}%
\pgfsetbuttcap%
\pgfsetroundjoin%
\definecolor{currentfill}{rgb}{0.000000,0.000000,0.000000}%
\pgfsetfillcolor{currentfill}%
\pgfsetlinewidth{0.803000pt}%
\definecolor{currentstroke}{rgb}{0.000000,0.000000,0.000000}%
\pgfsetstrokecolor{currentstroke}%
\pgfsetdash{}{0pt}%
\pgfsys@defobject{currentmarker}{\pgfqpoint{-0.048611in}{0.000000in}}{\pgfqpoint{-0.000000in}{0.000000in}}{%
\pgfpathmoveto{\pgfqpoint{-0.000000in}{0.000000in}}%
\pgfpathlineto{\pgfqpoint{-0.048611in}{0.000000in}}%
\pgfusepath{stroke,fill}%
}%
\begin{pgfscope}%
\pgfsys@transformshift{0.625000in}{3.157225in}%
\pgfsys@useobject{currentmarker}{}%
\end{pgfscope}%
\end{pgfscope}%
\begin{pgfscope}%
\definecolor{textcolor}{rgb}{0.000000,0.000000,0.000000}%
\pgfsetstrokecolor{textcolor}%
\pgfsetfillcolor{textcolor}%
\pgftext[x=0.388888in, y=3.109000in, left, base]{\color{textcolor}\rmfamily\fontsize{10.000000}{12.000000}\selectfont \(\displaystyle {54}\)}%
\end{pgfscope}%
\begin{pgfscope}%
\definecolor{textcolor}{rgb}{0.000000,0.000000,0.000000}%
\pgfsetstrokecolor{textcolor}%
\pgfsetfillcolor{textcolor}%
\pgftext[x=0.333333in,y=2.010000in,,bottom,rotate=90.000000]{\color{textcolor}\rmfamily\fontsize{10.000000}{12.000000}\selectfont Frequency (Hz)}%
\end{pgfscope}%
\begin{pgfscope}%
\pgfpathrectangle{\pgfqpoint{0.625000in}{0.500000in}}{\pgfqpoint{3.875000in}{3.020000in}}%
\pgfusepath{clip}%
\pgfsetrectcap%
\pgfsetroundjoin%
\pgfsetlinewidth{1.505625pt}%
\definecolor{currentstroke}{rgb}{0.121569,0.466667,0.705882}%
\pgfsetstrokecolor{currentstroke}%
\pgfsetdash{}{0pt}%
\pgfpathmoveto{\pgfqpoint{0.801136in}{1.966658in}}%
\pgfpathlineto{\pgfqpoint{1.095614in}{1.965395in}}%
\pgfpathlineto{\pgfqpoint{1.096476in}{1.948288in}}%
\pgfpathlineto{\pgfqpoint{1.097914in}{1.835375in}}%
\pgfpathlineto{\pgfqpoint{1.100502in}{1.318483in}}%
\pgfpathlineto{\pgfqpoint{1.104457in}{0.637273in}}%
\pgfpathlineto{\pgfqpoint{1.104744in}{0.641646in}}%
\pgfpathlineto{\pgfqpoint{1.105679in}{0.728756in}}%
\pgfpathlineto{\pgfqpoint{1.107692in}{1.287813in}}%
\pgfpathlineto{\pgfqpoint{1.113803in}{3.382727in}}%
\pgfpathlineto{\pgfqpoint{1.114450in}{3.347492in}}%
\pgfpathlineto{\pgfqpoint{1.116103in}{3.011044in}}%
\pgfpathlineto{\pgfqpoint{1.121567in}{1.887712in}}%
\pgfpathlineto{\pgfqpoint{1.121783in}{1.889015in}}%
\pgfpathlineto{\pgfqpoint{1.122718in}{1.927530in}}%
\pgfpathlineto{\pgfqpoint{1.125665in}{2.104232in}}%
\pgfpathlineto{\pgfqpoint{1.126528in}{2.076558in}}%
\pgfpathlineto{\pgfqpoint{1.128182in}{1.854195in}}%
\pgfpathlineto{\pgfqpoint{1.133502in}{0.846846in}}%
\pgfpathlineto{\pgfqpoint{1.134436in}{0.896858in}}%
\pgfpathlineto{\pgfqpoint{1.136162in}{1.265887in}}%
\pgfpathlineto{\pgfqpoint{1.142273in}{3.090983in}}%
\pgfpathlineto{\pgfqpoint{1.143279in}{3.023797in}}%
\pgfpathlineto{\pgfqpoint{1.145508in}{2.509828in}}%
\pgfpathlineto{\pgfqpoint{1.149606in}{1.774774in}}%
\pgfpathlineto{\pgfqpoint{1.149678in}{1.774549in}}%
\pgfpathlineto{\pgfqpoint{1.149965in}{1.778245in}}%
\pgfpathlineto{\pgfqpoint{1.151044in}{1.849564in}}%
\pgfpathlineto{\pgfqpoint{1.154926in}{2.242609in}}%
\pgfpathlineto{\pgfqpoint{1.155933in}{2.197919in}}%
\pgfpathlineto{\pgfqpoint{1.157802in}{1.878667in}}%
\pgfpathlineto{\pgfqpoint{1.162547in}{0.997242in}}%
\pgfpathlineto{\pgfqpoint{1.162978in}{1.009572in}}%
\pgfpathlineto{\pgfqpoint{1.164200in}{1.165074in}}%
\pgfpathlineto{\pgfqpoint{1.167004in}{2.028491in}}%
\pgfpathlineto{\pgfqpoint{1.170886in}{2.906153in}}%
\pgfpathlineto{\pgfqpoint{1.170958in}{2.905631in}}%
\pgfpathlineto{\pgfqpoint{1.171605in}{2.869851in}}%
\pgfpathlineto{\pgfqpoint{1.173403in}{2.533005in}}%
\pgfpathlineto{\pgfqpoint{1.178076in}{1.674480in}}%
\pgfpathlineto{\pgfqpoint{1.178292in}{1.677042in}}%
\pgfpathlineto{\pgfqpoint{1.179226in}{1.736714in}}%
\pgfpathlineto{\pgfqpoint{1.182461in}{2.236919in}}%
\pgfpathlineto{\pgfqpoint{1.184043in}{2.346869in}}%
\pgfpathlineto{\pgfqpoint{1.184618in}{2.328042in}}%
\pgfpathlineto{\pgfqpoint{1.186056in}{2.137831in}}%
\pgfpathlineto{\pgfqpoint{1.191448in}{1.104205in}}%
\pgfpathlineto{\pgfqpoint{1.192455in}{1.163295in}}%
\pgfpathlineto{\pgfqpoint{1.194396in}{1.580944in}}%
\pgfpathlineto{\pgfqpoint{1.199500in}{2.787529in}}%
\pgfpathlineto{\pgfqpoint{1.199932in}{2.774069in}}%
\pgfpathlineto{\pgfqpoint{1.201226in}{2.602686in}}%
\pgfpathlineto{\pgfqpoint{1.206618in}{1.594978in}}%
\pgfpathlineto{\pgfqpoint{1.207552in}{1.640008in}}%
\pgfpathlineto{\pgfqpoint{1.209709in}{1.988341in}}%
\pgfpathlineto{\pgfqpoint{1.212944in}{2.420589in}}%
\pgfpathlineto{\pgfqpoint{1.213304in}{2.414247in}}%
\pgfpathlineto{\pgfqpoint{1.214382in}{2.312615in}}%
\pgfpathlineto{\pgfqpoint{1.217258in}{1.643186in}}%
\pgfpathlineto{\pgfqpoint{1.220349in}{1.180532in}}%
\pgfpathlineto{\pgfqpoint{1.220565in}{1.183709in}}%
\pgfpathlineto{\pgfqpoint{1.221500in}{1.259098in}}%
\pgfpathlineto{\pgfqpoint{1.223728in}{1.774557in}}%
\pgfpathlineto{\pgfqpoint{1.228114in}{2.711098in}}%
\pgfpathlineto{\pgfqpoint{1.228330in}{2.708199in}}%
\pgfpathlineto{\pgfqpoint{1.229264in}{2.630342in}}%
\pgfpathlineto{\pgfqpoint{1.231781in}{2.069096in}}%
\pgfpathlineto{\pgfqpoint{1.235231in}{1.534638in}}%
\pgfpathlineto{\pgfqpoint{1.235375in}{1.535979in}}%
\pgfpathlineto{\pgfqpoint{1.236166in}{1.580691in}}%
\pgfpathlineto{\pgfqpoint{1.238251in}{1.933866in}}%
\pgfpathlineto{\pgfqpoint{1.241846in}{2.471245in}}%
\pgfpathlineto{\pgfqpoint{1.242205in}{2.464130in}}%
\pgfpathlineto{\pgfqpoint{1.243355in}{2.346848in}}%
\pgfpathlineto{\pgfqpoint{1.246663in}{1.558309in}}%
\pgfpathlineto{\pgfqpoint{1.249179in}{1.235146in}}%
\pgfpathlineto{\pgfqpoint{1.249538in}{1.242818in}}%
\pgfpathlineto{\pgfqpoint{1.250689in}{1.363873in}}%
\pgfpathlineto{\pgfqpoint{1.253564in}{2.104471in}}%
\pgfpathlineto{\pgfqpoint{1.256800in}{2.661666in}}%
\pgfpathlineto{\pgfqpoint{1.257015in}{2.658365in}}%
\pgfpathlineto{\pgfqpoint{1.257950in}{2.580022in}}%
\pgfpathlineto{\pgfqpoint{1.260466in}{2.021756in}}%
\pgfpathlineto{\pgfqpoint{1.263917in}{1.489977in}}%
\pgfpathlineto{\pgfqpoint{1.264061in}{1.491581in}}%
\pgfpathlineto{\pgfqpoint{1.264852in}{1.538501in}}%
\pgfpathlineto{\pgfqpoint{1.266937in}{1.906746in}}%
\pgfpathlineto{\pgfqpoint{1.270675in}{2.505472in}}%
\pgfpathlineto{\pgfqpoint{1.271035in}{2.498965in}}%
\pgfpathlineto{\pgfqpoint{1.272113in}{2.393303in}}%
\pgfpathlineto{\pgfqpoint{1.275133in}{1.679379in}}%
\pgfpathlineto{\pgfqpoint{1.278008in}{1.274382in}}%
\pgfpathlineto{\pgfqpoint{1.278224in}{1.277531in}}%
\pgfpathlineto{\pgfqpoint{1.279159in}{1.350915in}}%
\pgfpathlineto{\pgfqpoint{1.281387in}{1.840084in}}%
\pgfpathlineto{\pgfqpoint{1.285485in}{2.629710in}}%
\pgfpathlineto{\pgfqpoint{1.285701in}{2.626591in}}%
\pgfpathlineto{\pgfqpoint{1.286636in}{2.549703in}}%
\pgfpathlineto{\pgfqpoint{1.289152in}{1.994681in}}%
\pgfpathlineto{\pgfqpoint{1.292603in}{1.457243in}}%
\pgfpathlineto{\pgfqpoint{1.292747in}{1.458625in}}%
\pgfpathlineto{\pgfqpoint{1.293537in}{1.504924in}}%
\pgfpathlineto{\pgfqpoint{1.295550in}{1.862930in}}%
\pgfpathlineto{\pgfqpoint{1.299505in}{2.528406in}}%
\pgfpathlineto{\pgfqpoint{1.299864in}{2.521467in}}%
\pgfpathlineto{\pgfqpoint{1.301014in}{2.402293in}}%
\pgfpathlineto{\pgfqpoint{1.304537in}{1.564475in}}%
\pgfpathlineto{\pgfqpoint{1.306766in}{1.302547in}}%
\pgfpathlineto{\pgfqpoint{1.307197in}{1.312107in}}%
\pgfpathlineto{\pgfqpoint{1.308419in}{1.447916in}}%
\pgfpathlineto{\pgfqpoint{1.312014in}{2.342923in}}%
\pgfpathlineto{\pgfqpoint{1.314171in}{2.609031in}}%
\pgfpathlineto{\pgfqpoint{1.314674in}{2.594498in}}%
\pgfpathlineto{\pgfqpoint{1.316040in}{2.413950in}}%
\pgfpathlineto{\pgfqpoint{1.321288in}{1.433530in}}%
\pgfpathlineto{\pgfqpoint{1.322223in}{1.478170in}}%
\pgfpathlineto{\pgfqpoint{1.324164in}{1.819739in}}%
\pgfpathlineto{\pgfqpoint{1.328334in}{2.543609in}}%
\pgfpathlineto{\pgfqpoint{1.328694in}{2.535489in}}%
\pgfpathlineto{\pgfqpoint{1.329844in}{2.412286in}}%
\pgfpathlineto{\pgfqpoint{1.335595in}{1.322807in}}%
\pgfpathlineto{\pgfqpoint{1.337033in}{1.439387in}}%
\pgfpathlineto{\pgfqpoint{1.339981in}{2.147218in}}%
\pgfpathlineto{\pgfqpoint{1.342928in}{2.595644in}}%
\pgfpathlineto{\pgfqpoint{1.343216in}{2.589844in}}%
\pgfpathlineto{\pgfqpoint{1.344294in}{2.483652in}}%
\pgfpathlineto{\pgfqpoint{1.347530in}{1.729944in}}%
\pgfpathlineto{\pgfqpoint{1.350046in}{1.416406in}}%
\pgfpathlineto{\pgfqpoint{1.350405in}{1.423616in}}%
\pgfpathlineto{\pgfqpoint{1.351556in}{1.536458in}}%
\pgfpathlineto{\pgfqpoint{1.355007in}{2.314854in}}%
\pgfpathlineto{\pgfqpoint{1.357092in}{2.553613in}}%
\pgfpathlineto{\pgfqpoint{1.357595in}{2.539849in}}%
\pgfpathlineto{\pgfqpoint{1.358889in}{2.376377in}}%
\pgfpathlineto{\pgfqpoint{1.364353in}{1.337291in}}%
\pgfpathlineto{\pgfqpoint{1.365431in}{1.402106in}}%
\pgfpathlineto{\pgfqpoint{1.367588in}{1.842455in}}%
\pgfpathlineto{\pgfqpoint{1.371614in}{2.587027in}}%
\pgfpathlineto{\pgfqpoint{1.371902in}{2.582822in}}%
\pgfpathlineto{\pgfqpoint{1.372908in}{2.492901in}}%
\pgfpathlineto{\pgfqpoint{1.375640in}{1.871080in}}%
\pgfpathlineto{\pgfqpoint{1.378804in}{1.404131in}}%
\pgfpathlineto{\pgfqpoint{1.379019in}{1.407150in}}%
\pgfpathlineto{\pgfqpoint{1.379954in}{1.477186in}}%
\pgfpathlineto{\pgfqpoint{1.382326in}{1.967771in}}%
\pgfpathlineto{\pgfqpoint{1.385921in}{2.560134in}}%
\pgfpathlineto{\pgfqpoint{1.386137in}{2.556620in}}%
\pgfpathlineto{\pgfqpoint{1.387071in}{2.479566in}}%
\pgfpathlineto{\pgfqpoint{1.389516in}{1.939476in}}%
\pgfpathlineto{\pgfqpoint{1.393110in}{1.347690in}}%
\pgfpathlineto{\pgfqpoint{1.393254in}{1.348810in}}%
\pgfpathlineto{\pgfqpoint{1.393973in}{1.388361in}}%
\pgfpathlineto{\pgfqpoint{1.395842in}{1.719668in}}%
\pgfpathlineto{\pgfqpoint{1.400372in}{2.581506in}}%
\pgfpathlineto{\pgfqpoint{1.400803in}{2.570554in}}%
\pgfpathlineto{\pgfqpoint{1.402025in}{2.426508in}}%
\pgfpathlineto{\pgfqpoint{1.407561in}{1.395345in}}%
\pgfpathlineto{\pgfqpoint{1.408783in}{1.478964in}}%
\pgfpathlineto{\pgfqpoint{1.411300in}{2.019928in}}%
\pgfpathlineto{\pgfqpoint{1.414679in}{2.564437in}}%
\pgfpathlineto{\pgfqpoint{1.414894in}{2.561333in}}%
\pgfpathlineto{\pgfqpoint{1.415829in}{2.485868in}}%
\pgfpathlineto{\pgfqpoint{1.418273in}{1.948055in}}%
\pgfpathlineto{\pgfqpoint{1.421868in}{1.355138in}}%
\pgfpathlineto{\pgfqpoint{1.422012in}{1.356134in}}%
\pgfpathlineto{\pgfqpoint{1.422731in}{1.394990in}}%
\pgfpathlineto{\pgfqpoint{1.424528in}{1.706696in}}%
\pgfpathlineto{\pgfqpoint{1.429129in}{2.577940in}}%
\pgfpathlineto{\pgfqpoint{1.429632in}{2.562235in}}%
\pgfpathlineto{\pgfqpoint{1.430998in}{2.380020in}}%
\pgfpathlineto{\pgfqpoint{1.436247in}{1.389058in}}%
\pgfpathlineto{\pgfqpoint{1.437181in}{1.432739in}}%
\pgfpathlineto{\pgfqpoint{1.439051in}{1.763241in}}%
\pgfpathlineto{\pgfqpoint{1.443436in}{2.567200in}}%
\pgfpathlineto{\pgfqpoint{1.443867in}{2.556085in}}%
\pgfpathlineto{\pgfqpoint{1.445090in}{2.411833in}}%
\pgfpathlineto{\pgfqpoint{1.450625in}{1.360461in}}%
\pgfpathlineto{\pgfqpoint{1.451920in}{1.448774in}}%
\pgfpathlineto{\pgfqpoint{1.454436in}{2.004251in}}%
\pgfpathlineto{\pgfqpoint{1.457887in}{2.575648in}}%
\pgfpathlineto{\pgfqpoint{1.458102in}{2.572344in}}%
\pgfpathlineto{\pgfqpoint{1.459037in}{2.495873in}}%
\pgfpathlineto{\pgfqpoint{1.461481in}{1.959453in}}%
\pgfpathlineto{\pgfqpoint{1.465004in}{1.384546in}}%
\pgfpathlineto{\pgfqpoint{1.465220in}{1.386371in}}%
\pgfpathlineto{\pgfqpoint{1.466083in}{1.443651in}}%
\pgfpathlineto{\pgfqpoint{1.468239in}{1.867412in}}%
\pgfpathlineto{\pgfqpoint{1.472194in}{2.568963in}}%
\pgfpathlineto{\pgfqpoint{1.472481in}{2.564275in}}%
\pgfpathlineto{\pgfqpoint{1.473488in}{2.473136in}}%
\pgfpathlineto{\pgfqpoint{1.476220in}{1.845718in}}%
\pgfpathlineto{\pgfqpoint{1.479383in}{1.364257in}}%
\pgfpathlineto{\pgfqpoint{1.479599in}{1.366403in}}%
\pgfpathlineto{\pgfqpoint{1.480461in}{1.425287in}}%
\pgfpathlineto{\pgfqpoint{1.482618in}{1.855212in}}%
\pgfpathlineto{\pgfqpoint{1.486644in}{2.574184in}}%
\pgfpathlineto{\pgfqpoint{1.486860in}{2.570703in}}%
\pgfpathlineto{\pgfqpoint{1.487795in}{2.493533in}}%
\pgfpathlineto{\pgfqpoint{1.490311in}{1.937277in}}%
\pgfpathlineto{\pgfqpoint{1.493762in}{1.381335in}}%
\pgfpathlineto{\pgfqpoint{1.493977in}{1.383224in}}%
\pgfpathlineto{\pgfqpoint{1.494840in}{1.440813in}}%
\pgfpathlineto{\pgfqpoint{1.496997in}{1.865674in}}%
\pgfpathlineto{\pgfqpoint{1.500951in}{2.570081in}}%
\pgfpathlineto{\pgfqpoint{1.501239in}{2.565593in}}%
\pgfpathlineto{\pgfqpoint{1.502245in}{2.475086in}}%
\pgfpathlineto{\pgfqpoint{1.504977in}{1.848624in}}%
\pgfpathlineto{\pgfqpoint{1.508141in}{1.366959in}}%
\pgfpathlineto{\pgfqpoint{1.508356in}{1.369054in}}%
\pgfpathlineto{\pgfqpoint{1.509219in}{1.427689in}}%
\pgfpathlineto{\pgfqpoint{1.511376in}{1.856712in}}%
\pgfpathlineto{\pgfqpoint{1.515402in}{2.573254in}}%
\pgfpathlineto{\pgfqpoint{1.515618in}{2.569646in}}%
\pgfpathlineto{\pgfqpoint{1.516552in}{2.491975in}}%
\pgfpathlineto{\pgfqpoint{1.519068in}{1.934894in}}%
\pgfpathlineto{\pgfqpoint{1.522519in}{1.379053in}}%
\pgfpathlineto{\pgfqpoint{1.522735in}{1.380984in}}%
\pgfpathlineto{\pgfqpoint{1.523598in}{1.438775in}}%
\pgfpathlineto{\pgfqpoint{1.525755in}{1.864380in}}%
\pgfpathlineto{\pgfqpoint{1.529709in}{2.570783in}}%
\pgfpathlineto{\pgfqpoint{1.529996in}{2.566439in}}%
\pgfpathlineto{\pgfqpoint{1.531003in}{2.476386in}}%
\pgfpathlineto{\pgfqpoint{1.533735in}{1.850638in}}%
\pgfpathlineto{\pgfqpoint{1.536898in}{1.368880in}}%
\pgfpathlineto{\pgfqpoint{1.537114in}{1.370941in}}%
\pgfpathlineto{\pgfqpoint{1.537977in}{1.429413in}}%
\pgfpathlineto{\pgfqpoint{1.540133in}{1.857826in}}%
\pgfpathlineto{\pgfqpoint{1.544088in}{2.572671in}}%
\pgfpathlineto{\pgfqpoint{1.544375in}{2.568971in}}%
\pgfpathlineto{\pgfqpoint{1.545310in}{2.490939in}}%
\pgfpathlineto{\pgfqpoint{1.547826in}{1.933249in}}%
\pgfpathlineto{\pgfqpoint{1.551277in}{1.377433in}}%
\pgfpathlineto{\pgfqpoint{1.551493in}{1.379391in}}%
\pgfpathlineto{\pgfqpoint{1.552355in}{1.437315in}}%
\pgfpathlineto{\pgfqpoint{1.554512in}{1.863423in}}%
\pgfpathlineto{\pgfqpoint{1.558466in}{2.571220in}}%
\pgfpathlineto{\pgfqpoint{1.558754in}{2.566978in}}%
\pgfpathlineto{\pgfqpoint{1.559760in}{2.477250in}}%
\pgfpathlineto{\pgfqpoint{1.562492in}{1.852033in}}%
\pgfpathlineto{\pgfqpoint{1.565656in}{1.370243in}}%
\pgfpathlineto{\pgfqpoint{1.565871in}{1.372282in}}%
\pgfpathlineto{\pgfqpoint{1.566734in}{1.430647in}}%
\pgfpathlineto{\pgfqpoint{1.568891in}{1.858647in}}%
\pgfpathlineto{\pgfqpoint{1.572845in}{2.572328in}}%
\pgfpathlineto{\pgfqpoint{1.573133in}{2.568542in}}%
\pgfpathlineto{\pgfqpoint{1.574067in}{2.490252in}}%
\pgfpathlineto{\pgfqpoint{1.576584in}{1.932113in}}%
\pgfpathlineto{\pgfqpoint{1.580034in}{1.376284in}}%
\pgfpathlineto{\pgfqpoint{1.580250in}{1.378259in}}%
\pgfpathlineto{\pgfqpoint{1.581113in}{1.436270in}}%
\pgfpathlineto{\pgfqpoint{1.583270in}{1.862718in}}%
\pgfpathlineto{\pgfqpoint{1.587224in}{2.571489in}}%
\pgfpathlineto{\pgfqpoint{1.587511in}{2.567319in}}%
\pgfpathlineto{\pgfqpoint{1.588518in}{2.477825in}}%
\pgfpathlineto{\pgfqpoint{1.591250in}{1.852998in}}%
\pgfpathlineto{\pgfqpoint{1.594413in}{1.371209in}}%
\pgfpathlineto{\pgfqpoint{1.594629in}{1.373234in}}%
\pgfpathlineto{\pgfqpoint{1.595492in}{1.431529in}}%
\pgfpathlineto{\pgfqpoint{1.597648in}{1.859251in}}%
\pgfpathlineto{\pgfqpoint{1.601603in}{2.572119in}}%
\pgfpathlineto{\pgfqpoint{1.601890in}{2.568272in}}%
\pgfpathlineto{\pgfqpoint{1.602897in}{2.479840in}}%
\pgfpathlineto{\pgfqpoint{1.605557in}{1.875397in}}%
\pgfpathlineto{\pgfqpoint{1.608792in}{1.375471in}}%
\pgfpathlineto{\pgfqpoint{1.609008in}{1.377457in}}%
\pgfpathlineto{\pgfqpoint{1.609870in}{1.435525in}}%
\pgfpathlineto{\pgfqpoint{1.612027in}{1.862202in}}%
\pgfpathlineto{\pgfqpoint{1.615981in}{2.571652in}}%
\pgfpathlineto{\pgfqpoint{1.616269in}{2.567533in}}%
\pgfpathlineto{\pgfqpoint{1.617275in}{2.478206in}}%
\pgfpathlineto{\pgfqpoint{1.620007in}{1.853666in}}%
\pgfpathlineto{\pgfqpoint{1.623171in}{1.371892in}}%
\pgfpathlineto{\pgfqpoint{1.623386in}{1.373909in}}%
\pgfpathlineto{\pgfqpoint{1.624249in}{1.432159in}}%
\pgfpathlineto{\pgfqpoint{1.626406in}{1.859692in}}%
\pgfpathlineto{\pgfqpoint{1.630360in}{2.571993in}}%
\pgfpathlineto{\pgfqpoint{1.630648in}{2.568103in}}%
\pgfpathlineto{\pgfqpoint{1.631654in}{2.479530in}}%
\pgfpathlineto{\pgfqpoint{1.634386in}{1.856391in}}%
\pgfpathlineto{\pgfqpoint{1.637550in}{1.374896in}}%
\pgfpathlineto{\pgfqpoint{1.637765in}{1.376889in}}%
\pgfpathlineto{\pgfqpoint{1.638628in}{1.434993in}}%
\pgfpathlineto{\pgfqpoint{1.640785in}{1.861825in}}%
\pgfpathlineto{\pgfqpoint{1.644739in}{2.571748in}}%
\pgfpathlineto{\pgfqpoint{1.645026in}{2.567666in}}%
\pgfpathlineto{\pgfqpoint{1.646033in}{2.478458in}}%
\pgfpathlineto{\pgfqpoint{1.648765in}{1.854127in}}%
\pgfpathlineto{\pgfqpoint{1.651928in}{1.372375in}}%
\pgfpathlineto{\pgfqpoint{1.652144in}{1.374387in}}%
\pgfpathlineto{\pgfqpoint{1.653007in}{1.432607in}}%
\pgfpathlineto{\pgfqpoint{1.655164in}{1.860013in}}%
\pgfpathlineto{\pgfqpoint{1.659118in}{2.571920in}}%
\pgfpathlineto{\pgfqpoint{1.659405in}{2.567999in}}%
\pgfpathlineto{\pgfqpoint{1.660412in}{2.479326in}}%
\pgfpathlineto{\pgfqpoint{1.663144in}{1.856007in}}%
\pgfpathlineto{\pgfqpoint{1.666307in}{1.374490in}}%
\pgfpathlineto{\pgfqpoint{1.666523in}{1.376487in}}%
\pgfpathlineto{\pgfqpoint{1.667385in}{1.434615in}}%
\pgfpathlineto{\pgfqpoint{1.669542in}{1.861551in}}%
\pgfpathlineto{\pgfqpoint{1.673496in}{2.571804in}}%
\pgfpathlineto{\pgfqpoint{1.673784in}{2.567747in}}%
\pgfpathlineto{\pgfqpoint{1.674791in}{2.478624in}}%
\pgfpathlineto{\pgfqpoint{1.677523in}{1.854446in}}%
\pgfpathlineto{\pgfqpoint{1.680686in}{1.372716in}}%
\pgfpathlineto{\pgfqpoint{1.680902in}{1.374725in}}%
\pgfpathlineto{\pgfqpoint{1.681764in}{1.432925in}}%
\pgfpathlineto{\pgfqpoint{1.683921in}{1.860247in}}%
\pgfpathlineto{\pgfqpoint{1.687875in}{2.571878in}}%
\pgfpathlineto{\pgfqpoint{1.688163in}{2.567935in}}%
\pgfpathlineto{\pgfqpoint{1.689169in}{2.479191in}}%
\pgfpathlineto{\pgfqpoint{1.691901in}{1.855742in}}%
\pgfpathlineto{\pgfqpoint{1.695065in}{1.374204in}}%
\pgfpathlineto{\pgfqpoint{1.695280in}{1.376204in}}%
\pgfpathlineto{\pgfqpoint{1.696143in}{1.434347in}}%
\pgfpathlineto{\pgfqpoint{1.698300in}{1.861352in}}%
\pgfpathlineto{\pgfqpoint{1.702254in}{2.571835in}}%
\pgfpathlineto{\pgfqpoint{1.702542in}{2.567796in}}%
\pgfpathlineto{\pgfqpoint{1.703548in}{2.478733in}}%
\pgfpathlineto{\pgfqpoint{1.706280in}{1.854666in}}%
\pgfpathlineto{\pgfqpoint{1.709443in}{1.372957in}}%
\pgfpathlineto{\pgfqpoint{1.709659in}{1.374963in}}%
\pgfpathlineto{\pgfqpoint{1.710522in}{1.433152in}}%
\pgfpathlineto{\pgfqpoint{1.712679in}{1.860416in}}%
\pgfpathlineto{\pgfqpoint{1.716633in}{2.571856in}}%
\pgfpathlineto{\pgfqpoint{1.716920in}{2.567897in}}%
\pgfpathlineto{\pgfqpoint{1.717927in}{2.479102in}}%
\pgfpathlineto{\pgfqpoint{1.720659in}{1.855559in}}%
\pgfpathlineto{\pgfqpoint{1.723822in}{1.374002in}}%
\pgfpathlineto{\pgfqpoint{1.724038in}{1.376003in}}%
\pgfpathlineto{\pgfqpoint{1.724901in}{1.434156in}}%
\pgfpathlineto{\pgfqpoint{1.727057in}{1.861209in}}%
\pgfpathlineto{\pgfqpoint{1.731012in}{2.571851in}}%
\pgfpathlineto{\pgfqpoint{1.731299in}{2.567825in}}%
\pgfpathlineto{\pgfqpoint{1.732306in}{2.478805in}}%
\pgfpathlineto{\pgfqpoint{1.735038in}{1.854818in}}%
\pgfpathlineto{\pgfqpoint{1.738201in}{1.373126in}}%
\pgfpathlineto{\pgfqpoint{1.738417in}{1.375131in}}%
\pgfpathlineto{\pgfqpoint{1.739279in}{1.433312in}}%
\pgfpathlineto{\pgfqpoint{1.741436in}{1.860538in}}%
\pgfpathlineto{\pgfqpoint{1.745390in}{2.571844in}}%
\pgfpathlineto{\pgfqpoint{1.745678in}{2.567875in}}%
\pgfpathlineto{\pgfqpoint{1.746684in}{2.479044in}}%
\pgfpathlineto{\pgfqpoint{1.749416in}{1.855433in}}%
\pgfpathlineto{\pgfqpoint{1.752580in}{1.373860in}}%
\pgfpathlineto{\pgfqpoint{1.752795in}{1.375862in}}%
\pgfpathlineto{\pgfqpoint{1.753658in}{1.434021in}}%
\pgfpathlineto{\pgfqpoint{1.755815in}{1.861105in}}%
\pgfpathlineto{\pgfqpoint{1.759769in}{2.571858in}}%
\pgfpathlineto{\pgfqpoint{1.760057in}{2.567842in}}%
\pgfpathlineto{\pgfqpoint{1.761063in}{2.478852in}}%
\pgfpathlineto{\pgfqpoint{1.763795in}{1.854923in}}%
\pgfpathlineto{\pgfqpoint{1.766958in}{1.373245in}}%
\pgfpathlineto{\pgfqpoint{1.767174in}{1.375249in}}%
\pgfpathlineto{\pgfqpoint{1.768037in}{1.433426in}}%
\pgfpathlineto{\pgfqpoint{1.770194in}{1.860625in}}%
\pgfpathlineto{\pgfqpoint{1.774148in}{2.571839in}}%
\pgfpathlineto{\pgfqpoint{1.774435in}{2.567863in}}%
\pgfpathlineto{\pgfqpoint{1.775442in}{2.479006in}}%
\pgfpathlineto{\pgfqpoint{1.778174in}{1.855346in}}%
\pgfpathlineto{\pgfqpoint{1.781337in}{1.373760in}}%
\pgfpathlineto{\pgfqpoint{1.781553in}{1.375763in}}%
\pgfpathlineto{\pgfqpoint{1.782416in}{1.433926in}}%
\pgfpathlineto{\pgfqpoint{1.784572in}{1.861031in}}%
\pgfpathlineto{\pgfqpoint{1.788527in}{2.571861in}}%
\pgfpathlineto{\pgfqpoint{1.788814in}{2.567851in}}%
\pgfpathlineto{\pgfqpoint{1.789821in}{2.478883in}}%
\pgfpathlineto{\pgfqpoint{1.792553in}{1.854995in}}%
\pgfpathlineto{\pgfqpoint{1.795716in}{1.373329in}}%
\pgfpathlineto{\pgfqpoint{1.795932in}{1.375333in}}%
\pgfpathlineto{\pgfqpoint{1.796794in}{1.433506in}}%
\pgfpathlineto{\pgfqpoint{1.798951in}{1.860688in}}%
\pgfpathlineto{\pgfqpoint{1.802905in}{2.571838in}}%
\pgfpathlineto{\pgfqpoint{1.803193in}{2.567856in}}%
\pgfpathlineto{\pgfqpoint{1.804200in}{2.478982in}}%
\pgfpathlineto{\pgfqpoint{1.806931in}{1.855286in}}%
\pgfpathlineto{\pgfqpoint{1.810095in}{1.373690in}}%
\pgfpathlineto{\pgfqpoint{1.810310in}{1.375693in}}%
\pgfpathlineto{\pgfqpoint{1.811173in}{1.433858in}}%
\pgfpathlineto{\pgfqpoint{1.813330in}{1.860977in}}%
\pgfpathlineto{\pgfqpoint{1.817284in}{2.571861in}}%
\pgfpathlineto{\pgfqpoint{1.817572in}{2.567856in}}%
\pgfpathlineto{\pgfqpoint{1.818578in}{2.478902in}}%
\pgfpathlineto{\pgfqpoint{1.821310in}{1.855045in}}%
\pgfpathlineto{\pgfqpoint{1.824474in}{1.373388in}}%
\pgfpathlineto{\pgfqpoint{1.824689in}{1.375391in}}%
\pgfpathlineto{\pgfqpoint{1.825552in}{1.433563in}}%
\pgfpathlineto{\pgfqpoint{1.827709in}{1.860733in}}%
\pgfpathlineto{\pgfqpoint{1.831663in}{2.571839in}}%
\pgfpathlineto{\pgfqpoint{1.831951in}{2.567853in}}%
\pgfpathlineto{\pgfqpoint{1.832957in}{2.478966in}}%
\pgfpathlineto{\pgfqpoint{1.835689in}{1.855245in}}%
\pgfpathlineto{\pgfqpoint{1.838852in}{1.373640in}}%
\pgfpathlineto{\pgfqpoint{1.839068in}{1.375644in}}%
\pgfpathlineto{\pgfqpoint{1.839931in}{1.433810in}}%
\pgfpathlineto{\pgfqpoint{1.842088in}{1.860939in}}%
\pgfpathlineto{\pgfqpoint{1.846042in}{2.571860in}}%
\pgfpathlineto{\pgfqpoint{1.846329in}{2.567858in}}%
\pgfpathlineto{\pgfqpoint{1.847336in}{2.478915in}}%
\pgfpathlineto{\pgfqpoint{1.850068in}{1.855079in}}%
\pgfpathlineto{\pgfqpoint{1.853231in}{1.373429in}}%
\pgfpathlineto{\pgfqpoint{1.853447in}{1.375433in}}%
\pgfpathlineto{\pgfqpoint{1.854310in}{1.433603in}}%
\pgfpathlineto{\pgfqpoint{1.856466in}{1.860766in}}%
\pgfpathlineto{\pgfqpoint{1.860421in}{2.571840in}}%
\pgfpathlineto{\pgfqpoint{1.860708in}{2.567851in}}%
\pgfpathlineto{\pgfqpoint{1.861715in}{2.478955in}}%
\pgfpathlineto{\pgfqpoint{1.864447in}{1.855217in}}%
\pgfpathlineto{\pgfqpoint{1.867610in}{1.373606in}}%
\pgfpathlineto{\pgfqpoint{1.867826in}{1.375609in}}%
\pgfpathlineto{\pgfqpoint{1.868688in}{1.433777in}}%
\pgfpathlineto{\pgfqpoint{1.870845in}{1.860912in}}%
\pgfpathlineto{\pgfqpoint{1.874799in}{2.571859in}}%
\pgfpathlineto{\pgfqpoint{1.875087in}{2.567859in}}%
\pgfpathlineto{\pgfqpoint{1.876093in}{2.478924in}}%
\pgfpathlineto{\pgfqpoint{1.878825in}{1.855103in}}%
\pgfpathlineto{\pgfqpoint{1.881989in}{1.373458in}}%
\pgfpathlineto{\pgfqpoint{1.882204in}{1.375461in}}%
\pgfpathlineto{\pgfqpoint{1.883067in}{1.433631in}}%
\pgfpathlineto{\pgfqpoint{1.885224in}{1.860789in}}%
\pgfpathlineto{\pgfqpoint{1.889178in}{2.571842in}}%
\pgfpathlineto{\pgfqpoint{1.889466in}{2.567851in}}%
\pgfpathlineto{\pgfqpoint{1.890472in}{2.478949in}}%
\pgfpathlineto{\pgfqpoint{1.893204in}{1.855197in}}%
\pgfpathlineto{\pgfqpoint{1.896367in}{1.373581in}}%
\pgfpathlineto{\pgfqpoint{1.896583in}{1.375585in}}%
\pgfpathlineto{\pgfqpoint{1.897446in}{1.433753in}}%
\pgfpathlineto{\pgfqpoint{1.899603in}{1.860893in}}%
\pgfpathlineto{\pgfqpoint{1.903557in}{2.571857in}}%
\pgfpathlineto{\pgfqpoint{1.903844in}{2.567859in}}%
\pgfpathlineto{\pgfqpoint{1.904851in}{2.478929in}}%
\pgfpathlineto{\pgfqpoint{1.907583in}{1.855119in}}%
\pgfpathlineto{\pgfqpoint{1.910746in}{1.373478in}}%
\pgfpathlineto{\pgfqpoint{1.910962in}{1.375482in}}%
\pgfpathlineto{\pgfqpoint{1.911825in}{1.433651in}}%
\pgfpathlineto{\pgfqpoint{1.913981in}{1.860805in}}%
\pgfpathlineto{\pgfqpoint{1.917936in}{2.571843in}}%
\pgfpathlineto{\pgfqpoint{1.918223in}{2.567851in}}%
\pgfpathlineto{\pgfqpoint{1.919230in}{2.478944in}}%
\pgfpathlineto{\pgfqpoint{1.921962in}{1.855184in}}%
\pgfpathlineto{\pgfqpoint{1.925125in}{1.373564in}}%
\pgfpathlineto{\pgfqpoint{1.925341in}{1.375568in}}%
\pgfpathlineto{\pgfqpoint{1.926203in}{1.433737in}}%
\pgfpathlineto{\pgfqpoint{1.928360in}{1.860879in}}%
\pgfpathlineto{\pgfqpoint{1.932314in}{2.571856in}}%
\pgfpathlineto{\pgfqpoint{1.932602in}{2.567858in}}%
\pgfpathlineto{\pgfqpoint{1.933608in}{2.478932in}}%
\pgfpathlineto{\pgfqpoint{1.936340in}{1.855130in}}%
\pgfpathlineto{\pgfqpoint{1.939504in}{1.373492in}}%
\pgfpathlineto{\pgfqpoint{1.939719in}{1.375496in}}%
\pgfpathlineto{\pgfqpoint{1.940582in}{1.433665in}}%
\pgfpathlineto{\pgfqpoint{1.942739in}{1.860817in}}%
\pgfpathlineto{\pgfqpoint{1.946693in}{2.571845in}}%
\pgfpathlineto{\pgfqpoint{1.946981in}{2.567852in}}%
\pgfpathlineto{\pgfqpoint{1.947987in}{2.478942in}}%
\pgfpathlineto{\pgfqpoint{1.950719in}{1.855174in}}%
\pgfpathlineto{\pgfqpoint{1.953883in}{1.373553in}}%
\pgfpathlineto{\pgfqpoint{1.954098in}{1.375556in}}%
\pgfpathlineto{\pgfqpoint{1.954961in}{1.433725in}}%
\pgfpathlineto{\pgfqpoint{1.957118in}{1.860869in}}%
\pgfpathlineto{\pgfqpoint{1.961072in}{2.571855in}}%
\pgfpathlineto{\pgfqpoint{1.961359in}{2.567858in}}%
\pgfpathlineto{\pgfqpoint{1.962366in}{2.478934in}}%
\pgfpathlineto{\pgfqpoint{1.965098in}{1.855138in}}%
\pgfpathlineto{\pgfqpoint{1.968261in}{1.373502in}}%
\pgfpathlineto{\pgfqpoint{1.968477in}{1.375506in}}%
\pgfpathlineto{\pgfqpoint{1.969340in}{1.433674in}}%
\pgfpathlineto{\pgfqpoint{1.971497in}{1.860825in}}%
\pgfpathlineto{\pgfqpoint{1.975451in}{2.571846in}}%
\pgfpathlineto{\pgfqpoint{1.975738in}{2.567852in}}%
\pgfpathlineto{\pgfqpoint{1.976745in}{2.478940in}}%
\pgfpathlineto{\pgfqpoint{1.979477in}{1.855168in}}%
\pgfpathlineto{\pgfqpoint{1.982640in}{1.373544in}}%
\pgfpathlineto{\pgfqpoint{1.982856in}{1.375548in}}%
\pgfpathlineto{\pgfqpoint{1.983718in}{1.433717in}}%
\pgfpathlineto{\pgfqpoint{1.985875in}{1.860862in}}%
\pgfpathlineto{\pgfqpoint{1.989829in}{2.571853in}}%
\pgfpathlineto{\pgfqpoint{1.990117in}{2.567857in}}%
\pgfpathlineto{\pgfqpoint{1.991124in}{2.478936in}}%
\pgfpathlineto{\pgfqpoint{1.993856in}{1.855143in}}%
\pgfpathlineto{\pgfqpoint{1.997019in}{1.373509in}}%
\pgfpathlineto{\pgfqpoint{1.997235in}{1.375513in}}%
\pgfpathlineto{\pgfqpoint{1.998097in}{1.433681in}}%
\pgfpathlineto{\pgfqpoint{2.000254in}{1.860831in}}%
\pgfpathlineto{\pgfqpoint{2.004208in}{2.571847in}}%
\pgfpathlineto{\pgfqpoint{2.004496in}{2.567853in}}%
\pgfpathlineto{\pgfqpoint{2.005502in}{2.478939in}}%
\pgfpathlineto{\pgfqpoint{2.008234in}{1.855164in}}%
\pgfpathlineto{\pgfqpoint{2.011398in}{1.373539in}}%
\pgfpathlineto{\pgfqpoint{2.011613in}{1.375542in}}%
\pgfpathlineto{\pgfqpoint{2.012476in}{1.433711in}}%
\pgfpathlineto{\pgfqpoint{2.014633in}{1.860857in}}%
\pgfpathlineto{\pgfqpoint{2.018587in}{2.571853in}}%
\pgfpathlineto{\pgfqpoint{2.018875in}{2.567857in}}%
\pgfpathlineto{\pgfqpoint{2.019881in}{2.478936in}}%
\pgfpathlineto{\pgfqpoint{2.022613in}{1.855146in}}%
\pgfpathlineto{\pgfqpoint{2.025776in}{1.373514in}}%
\pgfpathlineto{\pgfqpoint{2.025992in}{1.375518in}}%
\pgfpathlineto{\pgfqpoint{2.026855in}{1.433686in}}%
\pgfpathlineto{\pgfqpoint{2.029012in}{1.860835in}}%
\pgfpathlineto{\pgfqpoint{2.032966in}{2.571847in}}%
\pgfpathlineto{\pgfqpoint{2.033253in}{2.567853in}}%
\pgfpathlineto{\pgfqpoint{2.034260in}{2.478938in}}%
\pgfpathlineto{\pgfqpoint{2.036992in}{1.855161in}}%
\pgfpathlineto{\pgfqpoint{2.040155in}{1.373534in}}%
\pgfpathlineto{\pgfqpoint{2.040371in}{1.375538in}}%
\pgfpathlineto{\pgfqpoint{2.041234in}{1.433707in}}%
\pgfpathlineto{\pgfqpoint{2.043390in}{1.860853in}}%
\pgfpathlineto{\pgfqpoint{2.047345in}{2.571852in}}%
\pgfpathlineto{\pgfqpoint{2.047632in}{2.567857in}}%
\pgfpathlineto{\pgfqpoint{2.048639in}{2.478937in}}%
\pgfpathlineto{\pgfqpoint{2.051371in}{1.855149in}}%
\pgfpathlineto{\pgfqpoint{2.054534in}{1.373517in}}%
\pgfpathlineto{\pgfqpoint{2.054750in}{1.375521in}}%
\pgfpathlineto{\pgfqpoint{2.055612in}{1.433690in}}%
\pgfpathlineto{\pgfqpoint{2.057769in}{1.860838in}}%
\pgfpathlineto{\pgfqpoint{2.061723in}{2.571848in}}%
\pgfpathlineto{\pgfqpoint{2.062011in}{2.567854in}}%
\pgfpathlineto{\pgfqpoint{2.063017in}{2.478938in}}%
\pgfpathlineto{\pgfqpoint{2.065749in}{1.855159in}}%
\pgfpathlineto{\pgfqpoint{2.068913in}{1.373532in}}%
\pgfpathlineto{\pgfqpoint{2.069128in}{1.375535in}}%
\pgfpathlineto{\pgfqpoint{2.069991in}{1.433704in}}%
\pgfpathlineto{\pgfqpoint{2.072148in}{1.860851in}}%
\pgfpathlineto{\pgfqpoint{2.076102in}{2.571851in}}%
\pgfpathlineto{\pgfqpoint{2.076390in}{2.567856in}}%
\pgfpathlineto{\pgfqpoint{2.077396in}{2.478937in}}%
\pgfpathlineto{\pgfqpoint{2.080128in}{1.855151in}}%
\pgfpathlineto{\pgfqpoint{2.083291in}{1.373520in}}%
\pgfpathlineto{\pgfqpoint{2.083507in}{1.375523in}}%
\pgfpathlineto{\pgfqpoint{2.084370in}{1.433692in}}%
\pgfpathlineto{\pgfqpoint{2.086527in}{1.860840in}}%
\pgfpathlineto{\pgfqpoint{2.090481in}{2.571848in}}%
\pgfpathlineto{\pgfqpoint{2.090768in}{2.567854in}}%
\pgfpathlineto{\pgfqpoint{2.091775in}{2.478938in}}%
\pgfpathlineto{\pgfqpoint{2.094507in}{1.855157in}}%
\pgfpathlineto{\pgfqpoint{2.097670in}{1.373530in}}%
\pgfpathlineto{\pgfqpoint{2.097886in}{1.375533in}}%
\pgfpathlineto{\pgfqpoint{2.098749in}{1.433702in}}%
\pgfpathlineto{\pgfqpoint{2.100905in}{1.860849in}}%
\pgfpathlineto{\pgfqpoint{2.104860in}{2.571851in}}%
\pgfpathlineto{\pgfqpoint{2.105147in}{2.567856in}}%
\pgfpathlineto{\pgfqpoint{2.106154in}{2.478937in}}%
\pgfpathlineto{\pgfqpoint{2.108886in}{1.855152in}}%
\pgfpathlineto{\pgfqpoint{2.112049in}{1.373521in}}%
\pgfpathlineto{\pgfqpoint{2.112265in}{1.375525in}}%
\pgfpathlineto{\pgfqpoint{2.113127in}{1.433694in}}%
\pgfpathlineto{\pgfqpoint{2.115284in}{1.860842in}}%
\pgfpathlineto{\pgfqpoint{2.119238in}{2.571849in}}%
\pgfpathlineto{\pgfqpoint{2.119526in}{2.567854in}}%
\pgfpathlineto{\pgfqpoint{2.120532in}{2.478938in}}%
\pgfpathlineto{\pgfqpoint{2.123264in}{1.855156in}}%
\pgfpathlineto{\pgfqpoint{2.126428in}{1.373528in}}%
\pgfpathlineto{\pgfqpoint{2.126643in}{1.375532in}}%
\pgfpathlineto{\pgfqpoint{2.127506in}{1.433701in}}%
\pgfpathlineto{\pgfqpoint{2.129663in}{1.860848in}}%
\pgfpathlineto{\pgfqpoint{2.133617in}{2.571851in}}%
\pgfpathlineto{\pgfqpoint{2.133905in}{2.567856in}}%
\pgfpathlineto{\pgfqpoint{2.134911in}{2.478938in}}%
\pgfpathlineto{\pgfqpoint{2.137643in}{1.855153in}}%
\pgfpathlineto{\pgfqpoint{2.140807in}{1.373523in}}%
\pgfpathlineto{\pgfqpoint{2.141022in}{1.375526in}}%
\pgfpathlineto{\pgfqpoint{2.141885in}{1.433695in}}%
\pgfpathlineto{\pgfqpoint{2.144042in}{1.860843in}}%
\pgfpathlineto{\pgfqpoint{2.147996in}{2.571849in}}%
\pgfpathlineto{\pgfqpoint{2.148284in}{2.567854in}}%
\pgfpathlineto{\pgfqpoint{2.149290in}{2.478938in}}%
\pgfpathlineto{\pgfqpoint{2.152022in}{1.855156in}}%
\pgfpathlineto{\pgfqpoint{2.155185in}{1.373527in}}%
\pgfpathlineto{\pgfqpoint{2.155401in}{1.375531in}}%
\pgfpathlineto{\pgfqpoint{2.156264in}{1.433700in}}%
\pgfpathlineto{\pgfqpoint{2.158421in}{1.860847in}}%
\pgfpathlineto{\pgfqpoint{2.162375in}{2.571850in}}%
\pgfpathlineto{\pgfqpoint{2.162662in}{2.567856in}}%
\pgfpathlineto{\pgfqpoint{2.163669in}{2.478938in}}%
\pgfpathlineto{\pgfqpoint{2.166401in}{1.855153in}}%
\pgfpathlineto{\pgfqpoint{2.169564in}{1.373523in}}%
\pgfpathlineto{\pgfqpoint{2.169780in}{1.375527in}}%
\pgfpathlineto{\pgfqpoint{2.170643in}{1.433696in}}%
\pgfpathlineto{\pgfqpoint{2.172799in}{1.860843in}}%
\pgfpathlineto{\pgfqpoint{2.176753in}{2.571849in}}%
\pgfpathlineto{\pgfqpoint{2.177041in}{2.567855in}}%
\pgfpathlineto{\pgfqpoint{2.178048in}{2.478938in}}%
\pgfpathlineto{\pgfqpoint{2.180780in}{1.855155in}}%
\pgfpathlineto{\pgfqpoint{2.183943in}{1.373527in}}%
\pgfpathlineto{\pgfqpoint{2.184159in}{1.375530in}}%
\pgfpathlineto{\pgfqpoint{2.185021in}{1.433699in}}%
\pgfpathlineto{\pgfqpoint{2.187178in}{1.860846in}}%
\pgfpathlineto{\pgfqpoint{2.191132in}{2.571850in}}%
\pgfpathlineto{\pgfqpoint{2.191420in}{2.567855in}}%
\pgfpathlineto{\pgfqpoint{2.192426in}{2.478938in}}%
\pgfpathlineto{\pgfqpoint{2.195158in}{1.855153in}}%
\pgfpathlineto{\pgfqpoint{2.198322in}{1.373524in}}%
\pgfpathlineto{\pgfqpoint{2.198537in}{1.375528in}}%
\pgfpathlineto{\pgfqpoint{2.199400in}{1.433696in}}%
\pgfpathlineto{\pgfqpoint{2.201557in}{1.860844in}}%
\pgfpathlineto{\pgfqpoint{2.205511in}{2.571849in}}%
\pgfpathlineto{\pgfqpoint{2.205799in}{2.567855in}}%
\pgfpathlineto{\pgfqpoint{2.206805in}{2.478938in}}%
\pgfpathlineto{\pgfqpoint{2.209537in}{1.855155in}}%
\pgfpathlineto{\pgfqpoint{2.212700in}{1.373526in}}%
\pgfpathlineto{\pgfqpoint{2.212916in}{1.375530in}}%
\pgfpathlineto{\pgfqpoint{2.213779in}{1.433698in}}%
\pgfpathlineto{\pgfqpoint{2.215936in}{1.860846in}}%
\pgfpathlineto{\pgfqpoint{2.219890in}{2.571850in}}%
\pgfpathlineto{\pgfqpoint{2.220177in}{2.567855in}}%
\pgfpathlineto{\pgfqpoint{2.221184in}{2.478938in}}%
\pgfpathlineto{\pgfqpoint{2.223916in}{1.855154in}}%
\pgfpathlineto{\pgfqpoint{2.227079in}{1.373524in}}%
\pgfpathlineto{\pgfqpoint{2.227295in}{1.375528in}}%
\pgfpathlineto{\pgfqpoint{2.228158in}{1.433696in}}%
\pgfpathlineto{\pgfqpoint{2.230314in}{1.860844in}}%
\pgfpathlineto{\pgfqpoint{2.234269in}{2.571850in}}%
\pgfpathlineto{\pgfqpoint{2.234556in}{2.567855in}}%
\pgfpathlineto{\pgfqpoint{2.235563in}{2.478938in}}%
\pgfpathlineto{\pgfqpoint{2.238295in}{1.855155in}}%
\pgfpathlineto{\pgfqpoint{2.241458in}{1.373526in}}%
\pgfpathlineto{\pgfqpoint{2.241674in}{1.375530in}}%
\pgfpathlineto{\pgfqpoint{2.242536in}{1.433698in}}%
\pgfpathlineto{\pgfqpoint{2.244693in}{1.860846in}}%
\pgfpathlineto{\pgfqpoint{2.248647in}{2.571850in}}%
\pgfpathlineto{\pgfqpoint{2.248935in}{2.567855in}}%
\pgfpathlineto{\pgfqpoint{2.249941in}{2.478938in}}%
\pgfpathlineto{\pgfqpoint{2.252673in}{1.855154in}}%
\pgfpathlineto{\pgfqpoint{2.255837in}{1.373525in}}%
\pgfpathlineto{\pgfqpoint{2.256052in}{1.375528in}}%
\pgfpathlineto{\pgfqpoint{2.256915in}{1.433697in}}%
\pgfpathlineto{\pgfqpoint{2.259072in}{1.860844in}}%
\pgfpathlineto{\pgfqpoint{2.263026in}{2.571850in}}%
\pgfpathlineto{\pgfqpoint{2.263314in}{2.567855in}}%
\pgfpathlineto{\pgfqpoint{2.264320in}{2.478938in}}%
\pgfpathlineto{\pgfqpoint{2.267052in}{1.855155in}}%
\pgfpathlineto{\pgfqpoint{2.270216in}{1.373526in}}%
\pgfpathlineto{\pgfqpoint{2.270431in}{1.375529in}}%
\pgfpathlineto{\pgfqpoint{2.271294in}{1.433698in}}%
\pgfpathlineto{\pgfqpoint{2.273451in}{1.860846in}}%
\pgfpathlineto{\pgfqpoint{2.277405in}{2.571850in}}%
\pgfpathlineto{\pgfqpoint{2.277692in}{2.567855in}}%
\pgfpathlineto{\pgfqpoint{2.278699in}{2.478938in}}%
\pgfpathlineto{\pgfqpoint{2.281431in}{1.855154in}}%
\pgfpathlineto{\pgfqpoint{2.284594in}{1.373525in}}%
\pgfpathlineto{\pgfqpoint{2.284810in}{1.375528in}}%
\pgfpathlineto{\pgfqpoint{2.285673in}{1.433697in}}%
\pgfpathlineto{\pgfqpoint{2.287830in}{1.860845in}}%
\pgfpathlineto{\pgfqpoint{2.291784in}{2.571850in}}%
\pgfpathlineto{\pgfqpoint{2.292071in}{2.567855in}}%
\pgfpathlineto{\pgfqpoint{2.293078in}{2.478938in}}%
\pgfpathlineto{\pgfqpoint{2.295810in}{1.855155in}}%
\pgfpathlineto{\pgfqpoint{2.298973in}{1.373526in}}%
\pgfpathlineto{\pgfqpoint{2.299189in}{1.375529in}}%
\pgfpathlineto{\pgfqpoint{2.300051in}{1.433698in}}%
\pgfpathlineto{\pgfqpoint{2.302208in}{1.860845in}}%
\pgfpathlineto{\pgfqpoint{2.306162in}{2.571850in}}%
\pgfpathlineto{\pgfqpoint{2.306450in}{2.567855in}}%
\pgfpathlineto{\pgfqpoint{2.307457in}{2.478938in}}%
\pgfpathlineto{\pgfqpoint{2.310188in}{1.855154in}}%
\pgfpathlineto{\pgfqpoint{2.313352in}{1.373525in}}%
\pgfpathlineto{\pgfqpoint{2.313568in}{1.375529in}}%
\pgfpathlineto{\pgfqpoint{2.314430in}{1.433697in}}%
\pgfpathlineto{\pgfqpoint{2.316587in}{1.860845in}}%
\pgfpathlineto{\pgfqpoint{2.320541in}{2.571850in}}%
\pgfpathlineto{\pgfqpoint{2.320829in}{2.567855in}}%
\pgfpathlineto{\pgfqpoint{2.321835in}{2.478938in}}%
\pgfpathlineto{\pgfqpoint{2.324567in}{1.855154in}}%
\pgfpathlineto{\pgfqpoint{2.327731in}{1.373525in}}%
\pgfpathlineto{\pgfqpoint{2.327946in}{1.375529in}}%
\pgfpathlineto{\pgfqpoint{2.328809in}{1.433698in}}%
\pgfpathlineto{\pgfqpoint{2.330966in}{1.860845in}}%
\pgfpathlineto{\pgfqpoint{2.334920in}{2.571850in}}%
\pgfpathlineto{\pgfqpoint{2.335208in}{2.567855in}}%
\pgfpathlineto{\pgfqpoint{2.336214in}{2.478938in}}%
\pgfpathlineto{\pgfqpoint{2.338946in}{1.855154in}}%
\pgfpathlineto{\pgfqpoint{2.342109in}{1.373525in}}%
\pgfpathlineto{\pgfqpoint{2.342325in}{1.375529in}}%
\pgfpathlineto{\pgfqpoint{2.343188in}{1.433697in}}%
\pgfpathlineto{\pgfqpoint{2.345345in}{1.860845in}}%
\pgfpathlineto{\pgfqpoint{2.349299in}{2.571850in}}%
\pgfpathlineto{\pgfqpoint{2.349586in}{2.567855in}}%
\pgfpathlineto{\pgfqpoint{2.350593in}{2.478938in}}%
\pgfpathlineto{\pgfqpoint{2.353325in}{1.855154in}}%
\pgfpathlineto{\pgfqpoint{2.356488in}{1.373525in}}%
\pgfpathlineto{\pgfqpoint{2.356704in}{1.375529in}}%
\pgfpathlineto{\pgfqpoint{2.357567in}{1.433698in}}%
\pgfpathlineto{\pgfqpoint{2.359723in}{1.860845in}}%
\pgfpathlineto{\pgfqpoint{2.363678in}{2.571850in}}%
\pgfpathlineto{\pgfqpoint{2.363965in}{2.567855in}}%
\pgfpathlineto{\pgfqpoint{2.364972in}{2.478938in}}%
\pgfpathlineto{\pgfqpoint{2.367704in}{1.855154in}}%
\pgfpathlineto{\pgfqpoint{2.370867in}{1.373525in}}%
\pgfpathlineto{\pgfqpoint{2.371083in}{1.375529in}}%
\pgfpathlineto{\pgfqpoint{2.371945in}{1.433697in}}%
\pgfpathlineto{\pgfqpoint{2.374102in}{1.860845in}}%
\pgfpathlineto{\pgfqpoint{2.378056in}{2.571850in}}%
\pgfpathlineto{\pgfqpoint{2.378344in}{2.567855in}}%
\pgfpathlineto{\pgfqpoint{2.379350in}{2.478938in}}%
\pgfpathlineto{\pgfqpoint{2.382082in}{1.855154in}}%
\pgfpathlineto{\pgfqpoint{2.385246in}{1.373525in}}%
\pgfpathlineto{\pgfqpoint{2.385461in}{1.375529in}}%
\pgfpathlineto{\pgfqpoint{2.386324in}{1.433698in}}%
\pgfpathlineto{\pgfqpoint{2.388481in}{1.860845in}}%
\pgfpathlineto{\pgfqpoint{2.392435in}{2.571850in}}%
\pgfpathlineto{\pgfqpoint{2.392723in}{2.567855in}}%
\pgfpathlineto{\pgfqpoint{2.393729in}{2.478938in}}%
\pgfpathlineto{\pgfqpoint{2.396461in}{1.855154in}}%
\pgfpathlineto{\pgfqpoint{2.399624in}{1.373525in}}%
\pgfpathlineto{\pgfqpoint{2.399840in}{1.375529in}}%
\pgfpathlineto{\pgfqpoint{2.400703in}{1.433697in}}%
\pgfpathlineto{\pgfqpoint{2.402860in}{1.860845in}}%
\pgfpathlineto{\pgfqpoint{2.406814in}{2.571850in}}%
\pgfpathlineto{\pgfqpoint{2.407101in}{2.567855in}}%
\pgfpathlineto{\pgfqpoint{2.408108in}{2.478938in}}%
\pgfpathlineto{\pgfqpoint{2.410840in}{1.855154in}}%
\pgfpathlineto{\pgfqpoint{2.414003in}{1.373525in}}%
\pgfpathlineto{\pgfqpoint{2.414219in}{1.375529in}}%
\pgfpathlineto{\pgfqpoint{2.415082in}{1.433697in}}%
\pgfpathlineto{\pgfqpoint{2.417238in}{1.860845in}}%
\pgfpathlineto{\pgfqpoint{2.421193in}{2.571850in}}%
\pgfpathlineto{\pgfqpoint{2.421480in}{2.567855in}}%
\pgfpathlineto{\pgfqpoint{2.422487in}{2.478938in}}%
\pgfpathlineto{\pgfqpoint{2.425219in}{1.855154in}}%
\pgfpathlineto{\pgfqpoint{2.428382in}{1.373525in}}%
\pgfpathlineto{\pgfqpoint{2.428598in}{1.375529in}}%
\pgfpathlineto{\pgfqpoint{2.429460in}{1.433697in}}%
\pgfpathlineto{\pgfqpoint{2.431617in}{1.860845in}}%
\pgfpathlineto{\pgfqpoint{2.435571in}{2.571850in}}%
\pgfpathlineto{\pgfqpoint{2.435859in}{2.567855in}}%
\pgfpathlineto{\pgfqpoint{2.436865in}{2.478938in}}%
\pgfpathlineto{\pgfqpoint{2.439597in}{1.855154in}}%
\pgfpathlineto{\pgfqpoint{2.442761in}{1.373525in}}%
\pgfpathlineto{\pgfqpoint{2.442976in}{1.375529in}}%
\pgfpathlineto{\pgfqpoint{2.443839in}{1.433697in}}%
\pgfpathlineto{\pgfqpoint{2.445996in}{1.860845in}}%
\pgfpathlineto{\pgfqpoint{2.449950in}{2.571850in}}%
\pgfpathlineto{\pgfqpoint{2.450238in}{2.567855in}}%
\pgfpathlineto{\pgfqpoint{2.451244in}{2.478938in}}%
\pgfpathlineto{\pgfqpoint{2.453976in}{1.855154in}}%
\pgfpathlineto{\pgfqpoint{2.457140in}{1.373525in}}%
\pgfpathlineto{\pgfqpoint{2.457355in}{1.375529in}}%
\pgfpathlineto{\pgfqpoint{2.458218in}{1.433697in}}%
\pgfpathlineto{\pgfqpoint{2.460375in}{1.860845in}}%
\pgfpathlineto{\pgfqpoint{2.464329in}{2.571850in}}%
\pgfpathlineto{\pgfqpoint{2.464617in}{2.567855in}}%
\pgfpathlineto{\pgfqpoint{2.465623in}{2.478938in}}%
\pgfpathlineto{\pgfqpoint{2.468355in}{1.855154in}}%
\pgfpathlineto{\pgfqpoint{2.471518in}{1.373525in}}%
\pgfpathlineto{\pgfqpoint{2.471734in}{1.375529in}}%
\pgfpathlineto{\pgfqpoint{2.472597in}{1.433697in}}%
\pgfpathlineto{\pgfqpoint{2.474754in}{1.860845in}}%
\pgfpathlineto{\pgfqpoint{2.478708in}{2.571850in}}%
\pgfpathlineto{\pgfqpoint{2.478995in}{2.567855in}}%
\pgfpathlineto{\pgfqpoint{2.480002in}{2.478938in}}%
\pgfpathlineto{\pgfqpoint{2.482734in}{1.855154in}}%
\pgfpathlineto{\pgfqpoint{2.485897in}{1.373525in}}%
\pgfpathlineto{\pgfqpoint{2.486113in}{1.375529in}}%
\pgfpathlineto{\pgfqpoint{2.486975in}{1.433697in}}%
\pgfpathlineto{\pgfqpoint{2.489132in}{1.860845in}}%
\pgfpathlineto{\pgfqpoint{2.493086in}{2.571850in}}%
\pgfpathlineto{\pgfqpoint{2.493374in}{2.567855in}}%
\pgfpathlineto{\pgfqpoint{2.494381in}{2.478938in}}%
\pgfpathlineto{\pgfqpoint{2.497113in}{1.855154in}}%
\pgfpathlineto{\pgfqpoint{2.500276in}{1.373525in}}%
\pgfpathlineto{\pgfqpoint{2.500492in}{1.375529in}}%
\pgfpathlineto{\pgfqpoint{2.501354in}{1.433697in}}%
\pgfpathlineto{\pgfqpoint{2.503511in}{1.860845in}}%
\pgfpathlineto{\pgfqpoint{2.507465in}{2.571850in}}%
\pgfpathlineto{\pgfqpoint{2.507753in}{2.567855in}}%
\pgfpathlineto{\pgfqpoint{2.508759in}{2.478938in}}%
\pgfpathlineto{\pgfqpoint{2.511491in}{1.855154in}}%
\pgfpathlineto{\pgfqpoint{2.514655in}{1.373525in}}%
\pgfpathlineto{\pgfqpoint{2.514870in}{1.375529in}}%
\pgfpathlineto{\pgfqpoint{2.515733in}{1.433697in}}%
\pgfpathlineto{\pgfqpoint{2.517890in}{1.860845in}}%
\pgfpathlineto{\pgfqpoint{2.521844in}{2.571850in}}%
\pgfpathlineto{\pgfqpoint{2.522132in}{2.567855in}}%
\pgfpathlineto{\pgfqpoint{2.523138in}{2.478938in}}%
\pgfpathlineto{\pgfqpoint{2.525870in}{1.855154in}}%
\pgfpathlineto{\pgfqpoint{2.529033in}{1.373525in}}%
\pgfpathlineto{\pgfqpoint{2.529249in}{1.375529in}}%
\pgfpathlineto{\pgfqpoint{2.530112in}{1.433697in}}%
\pgfpathlineto{\pgfqpoint{2.532269in}{1.860845in}}%
\pgfpathlineto{\pgfqpoint{2.536223in}{2.571850in}}%
\pgfpathlineto{\pgfqpoint{2.536510in}{2.567855in}}%
\pgfpathlineto{\pgfqpoint{2.537517in}{2.478938in}}%
\pgfpathlineto{\pgfqpoint{2.540249in}{1.855154in}}%
\pgfpathlineto{\pgfqpoint{2.543412in}{1.373525in}}%
\pgfpathlineto{\pgfqpoint{2.543628in}{1.375529in}}%
\pgfpathlineto{\pgfqpoint{2.544491in}{1.433697in}}%
\pgfpathlineto{\pgfqpoint{2.546647in}{1.860845in}}%
\pgfpathlineto{\pgfqpoint{2.550602in}{2.571850in}}%
\pgfpathlineto{\pgfqpoint{2.550889in}{2.567855in}}%
\pgfpathlineto{\pgfqpoint{2.551896in}{2.478938in}}%
\pgfpathlineto{\pgfqpoint{2.554628in}{1.855154in}}%
\pgfpathlineto{\pgfqpoint{2.557791in}{1.373525in}}%
\pgfpathlineto{\pgfqpoint{2.558007in}{1.375529in}}%
\pgfpathlineto{\pgfqpoint{2.558869in}{1.433697in}}%
\pgfpathlineto{\pgfqpoint{2.561026in}{1.860845in}}%
\pgfpathlineto{\pgfqpoint{2.564980in}{2.571850in}}%
\pgfpathlineto{\pgfqpoint{2.565268in}{2.567855in}}%
\pgfpathlineto{\pgfqpoint{2.566274in}{2.478938in}}%
\pgfpathlineto{\pgfqpoint{2.569006in}{1.855154in}}%
\pgfpathlineto{\pgfqpoint{2.572170in}{1.373525in}}%
\pgfpathlineto{\pgfqpoint{2.572385in}{1.375529in}}%
\pgfpathlineto{\pgfqpoint{2.573248in}{1.433697in}}%
\pgfpathlineto{\pgfqpoint{2.575405in}{1.860845in}}%
\pgfpathlineto{\pgfqpoint{2.579359in}{2.571850in}}%
\pgfpathlineto{\pgfqpoint{2.579647in}{2.567855in}}%
\pgfpathlineto{\pgfqpoint{2.580653in}{2.478938in}}%
\pgfpathlineto{\pgfqpoint{2.583385in}{1.855154in}}%
\pgfpathlineto{\pgfqpoint{2.586548in}{1.373525in}}%
\pgfpathlineto{\pgfqpoint{2.586764in}{1.375529in}}%
\pgfpathlineto{\pgfqpoint{2.587627in}{1.433697in}}%
\pgfpathlineto{\pgfqpoint{2.589784in}{1.860845in}}%
\pgfpathlineto{\pgfqpoint{2.593738in}{2.571850in}}%
\pgfpathlineto{\pgfqpoint{2.594025in}{2.567855in}}%
\pgfpathlineto{\pgfqpoint{2.595032in}{2.478938in}}%
\pgfpathlineto{\pgfqpoint{2.597764in}{1.855154in}}%
\pgfpathlineto{\pgfqpoint{2.600927in}{1.373525in}}%
\pgfpathlineto{\pgfqpoint{2.601143in}{1.375529in}}%
\pgfpathlineto{\pgfqpoint{2.602006in}{1.433697in}}%
\pgfpathlineto{\pgfqpoint{2.604162in}{1.860845in}}%
\pgfpathlineto{\pgfqpoint{2.608117in}{2.571850in}}%
\pgfpathlineto{\pgfqpoint{2.608404in}{2.567855in}}%
\pgfpathlineto{\pgfqpoint{2.609411in}{2.478938in}}%
\pgfpathlineto{\pgfqpoint{2.612143in}{1.855154in}}%
\pgfpathlineto{\pgfqpoint{2.615306in}{1.373525in}}%
\pgfpathlineto{\pgfqpoint{2.615522in}{1.375529in}}%
\pgfpathlineto{\pgfqpoint{2.616384in}{1.433697in}}%
\pgfpathlineto{\pgfqpoint{2.618541in}{1.860845in}}%
\pgfpathlineto{\pgfqpoint{2.622495in}{2.571850in}}%
\pgfpathlineto{\pgfqpoint{2.622783in}{2.567855in}}%
\pgfpathlineto{\pgfqpoint{2.623790in}{2.478938in}}%
\pgfpathlineto{\pgfqpoint{2.626521in}{1.855154in}}%
\pgfpathlineto{\pgfqpoint{2.629685in}{1.373525in}}%
\pgfpathlineto{\pgfqpoint{2.629900in}{1.375529in}}%
\pgfpathlineto{\pgfqpoint{2.630763in}{1.433697in}}%
\pgfpathlineto{\pgfqpoint{2.632920in}{1.860845in}}%
\pgfpathlineto{\pgfqpoint{2.636874in}{2.571850in}}%
\pgfpathlineto{\pgfqpoint{2.637162in}{2.567855in}}%
\pgfpathlineto{\pgfqpoint{2.638168in}{2.478938in}}%
\pgfpathlineto{\pgfqpoint{2.640900in}{1.855154in}}%
\pgfpathlineto{\pgfqpoint{2.644064in}{1.373525in}}%
\pgfpathlineto{\pgfqpoint{2.644279in}{1.375529in}}%
\pgfpathlineto{\pgfqpoint{2.645142in}{1.433697in}}%
\pgfpathlineto{\pgfqpoint{2.647299in}{1.860845in}}%
\pgfpathlineto{\pgfqpoint{2.651253in}{2.571850in}}%
\pgfpathlineto{\pgfqpoint{2.651541in}{2.567855in}}%
\pgfpathlineto{\pgfqpoint{2.652547in}{2.478938in}}%
\pgfpathlineto{\pgfqpoint{2.655279in}{1.855154in}}%
\pgfpathlineto{\pgfqpoint{2.658442in}{1.373525in}}%
\pgfpathlineto{\pgfqpoint{2.658658in}{1.375529in}}%
\pgfpathlineto{\pgfqpoint{2.659521in}{1.433697in}}%
\pgfpathlineto{\pgfqpoint{2.661678in}{1.860845in}}%
\pgfpathlineto{\pgfqpoint{2.665632in}{2.571850in}}%
\pgfpathlineto{\pgfqpoint{2.665919in}{2.567855in}}%
\pgfpathlineto{\pgfqpoint{2.666926in}{2.478938in}}%
\pgfpathlineto{\pgfqpoint{2.669658in}{1.855154in}}%
\pgfpathlineto{\pgfqpoint{2.672821in}{1.373525in}}%
\pgfpathlineto{\pgfqpoint{2.673037in}{1.375529in}}%
\pgfpathlineto{\pgfqpoint{2.673900in}{1.433697in}}%
\pgfpathlineto{\pgfqpoint{2.676056in}{1.860845in}}%
\pgfpathlineto{\pgfqpoint{2.680011in}{2.571850in}}%
\pgfpathlineto{\pgfqpoint{2.680298in}{2.567855in}}%
\pgfpathlineto{\pgfqpoint{2.681305in}{2.478938in}}%
\pgfpathlineto{\pgfqpoint{2.684037in}{1.855154in}}%
\pgfpathlineto{\pgfqpoint{2.687200in}{1.373525in}}%
\pgfpathlineto{\pgfqpoint{2.687416in}{1.375529in}}%
\pgfpathlineto{\pgfqpoint{2.688278in}{1.433697in}}%
\pgfpathlineto{\pgfqpoint{2.690435in}{1.860845in}}%
\pgfpathlineto{\pgfqpoint{2.694389in}{2.571850in}}%
\pgfpathlineto{\pgfqpoint{2.694677in}{2.567855in}}%
\pgfpathlineto{\pgfqpoint{2.695683in}{2.478938in}}%
\pgfpathlineto{\pgfqpoint{2.698415in}{1.855154in}}%
\pgfpathlineto{\pgfqpoint{2.701579in}{1.373525in}}%
\pgfpathlineto{\pgfqpoint{2.701794in}{1.375529in}}%
\pgfpathlineto{\pgfqpoint{2.702657in}{1.433697in}}%
\pgfpathlineto{\pgfqpoint{2.704814in}{1.860845in}}%
\pgfpathlineto{\pgfqpoint{2.708768in}{2.571850in}}%
\pgfpathlineto{\pgfqpoint{2.709056in}{2.567855in}}%
\pgfpathlineto{\pgfqpoint{2.710062in}{2.478938in}}%
\pgfpathlineto{\pgfqpoint{2.712794in}{1.855154in}}%
\pgfpathlineto{\pgfqpoint{2.715957in}{1.373525in}}%
\pgfpathlineto{\pgfqpoint{2.716173in}{1.375529in}}%
\pgfpathlineto{\pgfqpoint{2.717036in}{1.433697in}}%
\pgfpathlineto{\pgfqpoint{2.719193in}{1.860845in}}%
\pgfpathlineto{\pgfqpoint{2.723147in}{2.571850in}}%
\pgfpathlineto{\pgfqpoint{2.723434in}{2.567855in}}%
\pgfpathlineto{\pgfqpoint{2.724441in}{2.478938in}}%
\pgfpathlineto{\pgfqpoint{2.727173in}{1.855154in}}%
\pgfpathlineto{\pgfqpoint{2.730336in}{1.373525in}}%
\pgfpathlineto{\pgfqpoint{2.730552in}{1.375529in}}%
\pgfpathlineto{\pgfqpoint{2.731415in}{1.433697in}}%
\pgfpathlineto{\pgfqpoint{2.733571in}{1.860845in}}%
\pgfpathlineto{\pgfqpoint{2.737526in}{2.571850in}}%
\pgfpathlineto{\pgfqpoint{2.737813in}{2.567855in}}%
\pgfpathlineto{\pgfqpoint{2.738820in}{2.478938in}}%
\pgfpathlineto{\pgfqpoint{2.741552in}{1.855154in}}%
\pgfpathlineto{\pgfqpoint{2.744715in}{1.373525in}}%
\pgfpathlineto{\pgfqpoint{2.744931in}{1.375529in}}%
\pgfpathlineto{\pgfqpoint{2.745793in}{1.433697in}}%
\pgfpathlineto{\pgfqpoint{2.747950in}{1.860845in}}%
\pgfpathlineto{\pgfqpoint{2.751904in}{2.571850in}}%
\pgfpathlineto{\pgfqpoint{2.752192in}{2.567855in}}%
\pgfpathlineto{\pgfqpoint{2.753198in}{2.478938in}}%
\pgfpathlineto{\pgfqpoint{2.755930in}{1.855154in}}%
\pgfpathlineto{\pgfqpoint{2.759094in}{1.373525in}}%
\pgfpathlineto{\pgfqpoint{2.759309in}{1.375529in}}%
\pgfpathlineto{\pgfqpoint{2.760172in}{1.433697in}}%
\pgfpathlineto{\pgfqpoint{2.762329in}{1.860845in}}%
\pgfpathlineto{\pgfqpoint{2.766283in}{2.571850in}}%
\pgfpathlineto{\pgfqpoint{2.766571in}{2.567855in}}%
\pgfpathlineto{\pgfqpoint{2.767577in}{2.478938in}}%
\pgfpathlineto{\pgfqpoint{2.770309in}{1.855154in}}%
\pgfpathlineto{\pgfqpoint{2.773473in}{1.373525in}}%
\pgfpathlineto{\pgfqpoint{2.773688in}{1.375529in}}%
\pgfpathlineto{\pgfqpoint{2.774551in}{1.433697in}}%
\pgfpathlineto{\pgfqpoint{2.776708in}{1.860845in}}%
\pgfpathlineto{\pgfqpoint{2.780662in}{2.571850in}}%
\pgfpathlineto{\pgfqpoint{2.780949in}{2.567855in}}%
\pgfpathlineto{\pgfqpoint{2.781956in}{2.478938in}}%
\pgfpathlineto{\pgfqpoint{2.784688in}{1.855154in}}%
\pgfpathlineto{\pgfqpoint{2.787851in}{1.373525in}}%
\pgfpathlineto{\pgfqpoint{2.788067in}{1.375529in}}%
\pgfpathlineto{\pgfqpoint{2.788930in}{1.433697in}}%
\pgfpathlineto{\pgfqpoint{2.791087in}{1.860845in}}%
\pgfpathlineto{\pgfqpoint{2.795041in}{2.571850in}}%
\pgfpathlineto{\pgfqpoint{2.795328in}{2.567855in}}%
\pgfpathlineto{\pgfqpoint{2.796335in}{2.478938in}}%
\pgfpathlineto{\pgfqpoint{2.799067in}{1.855154in}}%
\pgfpathlineto{\pgfqpoint{2.802230in}{1.373525in}}%
\pgfpathlineto{\pgfqpoint{2.802446in}{1.375529in}}%
\pgfpathlineto{\pgfqpoint{2.803308in}{1.433697in}}%
\pgfpathlineto{\pgfqpoint{2.805465in}{1.860845in}}%
\pgfpathlineto{\pgfqpoint{2.809419in}{2.571850in}}%
\pgfpathlineto{\pgfqpoint{2.809707in}{2.567855in}}%
\pgfpathlineto{\pgfqpoint{2.810714in}{2.478938in}}%
\pgfpathlineto{\pgfqpoint{2.813446in}{1.855154in}}%
\pgfpathlineto{\pgfqpoint{2.816609in}{1.373525in}}%
\pgfpathlineto{\pgfqpoint{2.816825in}{1.375529in}}%
\pgfpathlineto{\pgfqpoint{2.817687in}{1.433697in}}%
\pgfpathlineto{\pgfqpoint{2.819844in}{1.860845in}}%
\pgfpathlineto{\pgfqpoint{2.823798in}{2.571850in}}%
\pgfpathlineto{\pgfqpoint{2.824086in}{2.567855in}}%
\pgfpathlineto{\pgfqpoint{2.825092in}{2.478938in}}%
\pgfpathlineto{\pgfqpoint{2.827824in}{1.855154in}}%
\pgfpathlineto{\pgfqpoint{2.830988in}{1.373525in}}%
\pgfpathlineto{\pgfqpoint{2.831203in}{1.375529in}}%
\pgfpathlineto{\pgfqpoint{2.832066in}{1.433697in}}%
\pgfpathlineto{\pgfqpoint{2.834223in}{1.860845in}}%
\pgfpathlineto{\pgfqpoint{2.838177in}{2.571850in}}%
\pgfpathlineto{\pgfqpoint{2.838465in}{2.567855in}}%
\pgfpathlineto{\pgfqpoint{2.839471in}{2.478938in}}%
\pgfpathlineto{\pgfqpoint{2.842203in}{1.855154in}}%
\pgfpathlineto{\pgfqpoint{2.845366in}{1.373525in}}%
\pgfpathlineto{\pgfqpoint{2.845582in}{1.375529in}}%
\pgfpathlineto{\pgfqpoint{2.846445in}{1.433697in}}%
\pgfpathlineto{\pgfqpoint{2.848602in}{1.860845in}}%
\pgfpathlineto{\pgfqpoint{2.852556in}{2.571850in}}%
\pgfpathlineto{\pgfqpoint{2.852843in}{2.567855in}}%
\pgfpathlineto{\pgfqpoint{2.853850in}{2.478938in}}%
\pgfpathlineto{\pgfqpoint{2.856582in}{1.855154in}}%
\pgfpathlineto{\pgfqpoint{2.859745in}{1.373525in}}%
\pgfpathlineto{\pgfqpoint{2.859961in}{1.375529in}}%
\pgfpathlineto{\pgfqpoint{2.860824in}{1.433697in}}%
\pgfpathlineto{\pgfqpoint{2.862980in}{1.860845in}}%
\pgfpathlineto{\pgfqpoint{2.866935in}{2.571850in}}%
\pgfpathlineto{\pgfqpoint{2.867222in}{2.567855in}}%
\pgfpathlineto{\pgfqpoint{2.868229in}{2.478938in}}%
\pgfpathlineto{\pgfqpoint{2.870961in}{1.855154in}}%
\pgfpathlineto{\pgfqpoint{2.874124in}{1.373525in}}%
\pgfpathlineto{\pgfqpoint{2.874340in}{1.375529in}}%
\pgfpathlineto{\pgfqpoint{2.875202in}{1.433697in}}%
\pgfpathlineto{\pgfqpoint{2.877359in}{1.860845in}}%
\pgfpathlineto{\pgfqpoint{2.881313in}{2.571850in}}%
\pgfpathlineto{\pgfqpoint{2.881601in}{2.567855in}}%
\pgfpathlineto{\pgfqpoint{2.882607in}{2.478938in}}%
\pgfpathlineto{\pgfqpoint{2.885339in}{1.855154in}}%
\pgfpathlineto{\pgfqpoint{2.888503in}{1.373525in}}%
\pgfpathlineto{\pgfqpoint{2.888718in}{1.375529in}}%
\pgfpathlineto{\pgfqpoint{2.889581in}{1.433697in}}%
\pgfpathlineto{\pgfqpoint{2.891738in}{1.860845in}}%
\pgfpathlineto{\pgfqpoint{2.895692in}{2.571850in}}%
\pgfpathlineto{\pgfqpoint{2.895980in}{2.567855in}}%
\pgfpathlineto{\pgfqpoint{2.896986in}{2.478938in}}%
\pgfpathlineto{\pgfqpoint{2.899718in}{1.855154in}}%
\pgfpathlineto{\pgfqpoint{2.902881in}{1.373525in}}%
\pgfpathlineto{\pgfqpoint{2.903097in}{1.375529in}}%
\pgfpathlineto{\pgfqpoint{2.903960in}{1.433697in}}%
\pgfpathlineto{\pgfqpoint{2.906117in}{1.860845in}}%
\pgfpathlineto{\pgfqpoint{2.910071in}{2.571850in}}%
\pgfpathlineto{\pgfqpoint{2.910358in}{2.567855in}}%
\pgfpathlineto{\pgfqpoint{2.911365in}{2.478938in}}%
\pgfpathlineto{\pgfqpoint{2.914097in}{1.855154in}}%
\pgfpathlineto{\pgfqpoint{2.917260in}{1.373525in}}%
\pgfpathlineto{\pgfqpoint{2.917476in}{1.375529in}}%
\pgfpathlineto{\pgfqpoint{2.918339in}{1.433697in}}%
\pgfpathlineto{\pgfqpoint{2.920495in}{1.860845in}}%
\pgfpathlineto{\pgfqpoint{2.924450in}{2.571850in}}%
\pgfpathlineto{\pgfqpoint{2.924737in}{2.567855in}}%
\pgfpathlineto{\pgfqpoint{2.925744in}{2.478938in}}%
\pgfpathlineto{\pgfqpoint{2.928476in}{1.855154in}}%
\pgfpathlineto{\pgfqpoint{2.931639in}{1.373525in}}%
\pgfpathlineto{\pgfqpoint{2.931855in}{1.375529in}}%
\pgfpathlineto{\pgfqpoint{2.932717in}{1.433697in}}%
\pgfpathlineto{\pgfqpoint{2.934874in}{1.860845in}}%
\pgfpathlineto{\pgfqpoint{2.938828in}{2.571850in}}%
\pgfpathlineto{\pgfqpoint{2.939116in}{2.567855in}}%
\pgfpathlineto{\pgfqpoint{2.940123in}{2.478938in}}%
\pgfpathlineto{\pgfqpoint{2.942854in}{1.855154in}}%
\pgfpathlineto{\pgfqpoint{2.946018in}{1.373525in}}%
\pgfpathlineto{\pgfqpoint{2.946233in}{1.375529in}}%
\pgfpathlineto{\pgfqpoint{2.947096in}{1.433697in}}%
\pgfpathlineto{\pgfqpoint{2.949253in}{1.860845in}}%
\pgfpathlineto{\pgfqpoint{2.953207in}{2.571850in}}%
\pgfpathlineto{\pgfqpoint{2.953495in}{2.567855in}}%
\pgfpathlineto{\pgfqpoint{2.954501in}{2.478938in}}%
\pgfpathlineto{\pgfqpoint{2.957233in}{1.855154in}}%
\pgfpathlineto{\pgfqpoint{2.960397in}{1.373525in}}%
\pgfpathlineto{\pgfqpoint{2.960612in}{1.375529in}}%
\pgfpathlineto{\pgfqpoint{2.961475in}{1.433697in}}%
\pgfpathlineto{\pgfqpoint{2.963632in}{1.860845in}}%
\pgfpathlineto{\pgfqpoint{2.967586in}{2.571850in}}%
\pgfpathlineto{\pgfqpoint{2.967874in}{2.567855in}}%
\pgfpathlineto{\pgfqpoint{2.968880in}{2.478938in}}%
\pgfpathlineto{\pgfqpoint{2.971612in}{1.855154in}}%
\pgfpathlineto{\pgfqpoint{2.974775in}{1.373525in}}%
\pgfpathlineto{\pgfqpoint{2.974991in}{1.375529in}}%
\pgfpathlineto{\pgfqpoint{2.975854in}{1.433697in}}%
\pgfpathlineto{\pgfqpoint{2.978011in}{1.860845in}}%
\pgfpathlineto{\pgfqpoint{2.981965in}{2.571850in}}%
\pgfpathlineto{\pgfqpoint{2.982252in}{2.567855in}}%
\pgfpathlineto{\pgfqpoint{2.983259in}{2.478938in}}%
\pgfpathlineto{\pgfqpoint{2.985991in}{1.855154in}}%
\pgfpathlineto{\pgfqpoint{2.989154in}{1.373525in}}%
\pgfpathlineto{\pgfqpoint{2.989370in}{1.375529in}}%
\pgfpathlineto{\pgfqpoint{2.990233in}{1.433697in}}%
\pgfpathlineto{\pgfqpoint{2.992389in}{1.860845in}}%
\pgfpathlineto{\pgfqpoint{2.996344in}{2.571850in}}%
\pgfpathlineto{\pgfqpoint{2.996631in}{2.567855in}}%
\pgfpathlineto{\pgfqpoint{2.997638in}{2.478938in}}%
\pgfpathlineto{\pgfqpoint{3.000370in}{1.855154in}}%
\pgfpathlineto{\pgfqpoint{3.003533in}{1.373525in}}%
\pgfpathlineto{\pgfqpoint{3.003749in}{1.375529in}}%
\pgfpathlineto{\pgfqpoint{3.004611in}{1.433697in}}%
\pgfpathlineto{\pgfqpoint{3.006768in}{1.860845in}}%
\pgfpathlineto{\pgfqpoint{3.010722in}{2.571850in}}%
\pgfpathlineto{\pgfqpoint{3.011010in}{2.567855in}}%
\pgfpathlineto{\pgfqpoint{3.012016in}{2.478938in}}%
\pgfpathlineto{\pgfqpoint{3.014748in}{1.855154in}}%
\pgfpathlineto{\pgfqpoint{3.017912in}{1.373525in}}%
\pgfpathlineto{\pgfqpoint{3.018127in}{1.375529in}}%
\pgfpathlineto{\pgfqpoint{3.018990in}{1.433697in}}%
\pgfpathlineto{\pgfqpoint{3.021147in}{1.860845in}}%
\pgfpathlineto{\pgfqpoint{3.025101in}{2.571850in}}%
\pgfpathlineto{\pgfqpoint{3.025389in}{2.567855in}}%
\pgfpathlineto{\pgfqpoint{3.026395in}{2.478938in}}%
\pgfpathlineto{\pgfqpoint{3.029127in}{1.855154in}}%
\pgfpathlineto{\pgfqpoint{3.032290in}{1.373525in}}%
\pgfpathlineto{\pgfqpoint{3.032506in}{1.375529in}}%
\pgfpathlineto{\pgfqpoint{3.033369in}{1.433697in}}%
\pgfpathlineto{\pgfqpoint{3.035526in}{1.860845in}}%
\pgfpathlineto{\pgfqpoint{3.039480in}{2.571850in}}%
\pgfpathlineto{\pgfqpoint{3.039767in}{2.567855in}}%
\pgfpathlineto{\pgfqpoint{3.040774in}{2.478938in}}%
\pgfpathlineto{\pgfqpoint{3.043506in}{1.855154in}}%
\pgfpathlineto{\pgfqpoint{3.046669in}{1.373525in}}%
\pgfpathlineto{\pgfqpoint{3.046885in}{1.375529in}}%
\pgfpathlineto{\pgfqpoint{3.047748in}{1.433697in}}%
\pgfpathlineto{\pgfqpoint{3.049904in}{1.860845in}}%
\pgfpathlineto{\pgfqpoint{3.053859in}{2.571850in}}%
\pgfpathlineto{\pgfqpoint{3.054146in}{2.567855in}}%
\pgfpathlineto{\pgfqpoint{3.055153in}{2.478938in}}%
\pgfpathlineto{\pgfqpoint{3.057885in}{1.855154in}}%
\pgfpathlineto{\pgfqpoint{3.061048in}{1.373525in}}%
\pgfpathlineto{\pgfqpoint{3.061264in}{1.375529in}}%
\pgfpathlineto{\pgfqpoint{3.062126in}{1.433697in}}%
\pgfpathlineto{\pgfqpoint{3.064283in}{1.860845in}}%
\pgfpathlineto{\pgfqpoint{3.068237in}{2.571850in}}%
\pgfpathlineto{\pgfqpoint{3.068525in}{2.567855in}}%
\pgfpathlineto{\pgfqpoint{3.069531in}{2.478938in}}%
\pgfpathlineto{\pgfqpoint{3.072263in}{1.855154in}}%
\pgfpathlineto{\pgfqpoint{3.075427in}{1.373525in}}%
\pgfpathlineto{\pgfqpoint{3.075642in}{1.375529in}}%
\pgfpathlineto{\pgfqpoint{3.076505in}{1.433697in}}%
\pgfpathlineto{\pgfqpoint{3.078662in}{1.860845in}}%
\pgfpathlineto{\pgfqpoint{3.082616in}{2.571850in}}%
\pgfpathlineto{\pgfqpoint{3.082904in}{2.567855in}}%
\pgfpathlineto{\pgfqpoint{3.083910in}{2.478938in}}%
\pgfpathlineto{\pgfqpoint{3.086642in}{1.855154in}}%
\pgfpathlineto{\pgfqpoint{3.089806in}{1.373525in}}%
\pgfpathlineto{\pgfqpoint{3.090021in}{1.375529in}}%
\pgfpathlineto{\pgfqpoint{3.090884in}{1.433697in}}%
\pgfpathlineto{\pgfqpoint{3.093041in}{1.860845in}}%
\pgfpathlineto{\pgfqpoint{3.096995in}{2.571850in}}%
\pgfpathlineto{\pgfqpoint{3.097282in}{2.567855in}}%
\pgfpathlineto{\pgfqpoint{3.098289in}{2.478938in}}%
\pgfpathlineto{\pgfqpoint{3.101021in}{1.855154in}}%
\pgfpathlineto{\pgfqpoint{3.104184in}{1.373525in}}%
\pgfpathlineto{\pgfqpoint{3.104400in}{1.375529in}}%
\pgfpathlineto{\pgfqpoint{3.105263in}{1.433697in}}%
\pgfpathlineto{\pgfqpoint{3.107420in}{1.860845in}}%
\pgfpathlineto{\pgfqpoint{3.111374in}{2.571850in}}%
\pgfpathlineto{\pgfqpoint{3.111661in}{2.567855in}}%
\pgfpathlineto{\pgfqpoint{3.112668in}{2.478938in}}%
\pgfpathlineto{\pgfqpoint{3.115400in}{1.855154in}}%
\pgfpathlineto{\pgfqpoint{3.118563in}{1.373525in}}%
\pgfpathlineto{\pgfqpoint{3.118779in}{1.375529in}}%
\pgfpathlineto{\pgfqpoint{3.119641in}{1.433697in}}%
\pgfpathlineto{\pgfqpoint{3.121798in}{1.860845in}}%
\pgfpathlineto{\pgfqpoint{3.125752in}{2.571850in}}%
\pgfpathlineto{\pgfqpoint{3.126040in}{2.567855in}}%
\pgfpathlineto{\pgfqpoint{3.127047in}{2.478938in}}%
\pgfpathlineto{\pgfqpoint{3.129779in}{1.855154in}}%
\pgfpathlineto{\pgfqpoint{3.132942in}{1.373525in}}%
\pgfpathlineto{\pgfqpoint{3.133158in}{1.375529in}}%
\pgfpathlineto{\pgfqpoint{3.134020in}{1.433697in}}%
\pgfpathlineto{\pgfqpoint{3.136177in}{1.860845in}}%
\pgfpathlineto{\pgfqpoint{3.140131in}{2.571850in}}%
\pgfpathlineto{\pgfqpoint{3.140419in}{2.567855in}}%
\pgfpathlineto{\pgfqpoint{3.141425in}{2.478938in}}%
\pgfpathlineto{\pgfqpoint{3.144157in}{1.855154in}}%
\pgfpathlineto{\pgfqpoint{3.147321in}{1.373525in}}%
\pgfpathlineto{\pgfqpoint{3.147536in}{1.375529in}}%
\pgfpathlineto{\pgfqpoint{3.148399in}{1.433697in}}%
\pgfpathlineto{\pgfqpoint{3.150556in}{1.860845in}}%
\pgfpathlineto{\pgfqpoint{3.154510in}{2.571850in}}%
\pgfpathlineto{\pgfqpoint{3.154798in}{2.567855in}}%
\pgfpathlineto{\pgfqpoint{3.155804in}{2.478938in}}%
\pgfpathlineto{\pgfqpoint{3.158536in}{1.855154in}}%
\pgfpathlineto{\pgfqpoint{3.161699in}{1.373525in}}%
\pgfpathlineto{\pgfqpoint{3.161915in}{1.375529in}}%
\pgfpathlineto{\pgfqpoint{3.162778in}{1.433697in}}%
\pgfpathlineto{\pgfqpoint{3.164935in}{1.860845in}}%
\pgfpathlineto{\pgfqpoint{3.168889in}{2.571850in}}%
\pgfpathlineto{\pgfqpoint{3.169176in}{2.567855in}}%
\pgfpathlineto{\pgfqpoint{3.170183in}{2.478938in}}%
\pgfpathlineto{\pgfqpoint{3.172915in}{1.855154in}}%
\pgfpathlineto{\pgfqpoint{3.176078in}{1.373525in}}%
\pgfpathlineto{\pgfqpoint{3.176294in}{1.375529in}}%
\pgfpathlineto{\pgfqpoint{3.177157in}{1.433697in}}%
\pgfpathlineto{\pgfqpoint{3.179313in}{1.860845in}}%
\pgfpathlineto{\pgfqpoint{3.183268in}{2.571850in}}%
\pgfpathlineto{\pgfqpoint{3.183555in}{2.567855in}}%
\pgfpathlineto{\pgfqpoint{3.184562in}{2.478938in}}%
\pgfpathlineto{\pgfqpoint{3.187294in}{1.855154in}}%
\pgfpathlineto{\pgfqpoint{3.190457in}{1.373525in}}%
\pgfpathlineto{\pgfqpoint{3.190673in}{1.375529in}}%
\pgfpathlineto{\pgfqpoint{3.191535in}{1.433697in}}%
\pgfpathlineto{\pgfqpoint{3.193692in}{1.860845in}}%
\pgfpathlineto{\pgfqpoint{3.197646in}{2.571850in}}%
\pgfpathlineto{\pgfqpoint{3.197934in}{2.567855in}}%
\pgfpathlineto{\pgfqpoint{3.198940in}{2.478938in}}%
\pgfpathlineto{\pgfqpoint{3.201672in}{1.855154in}}%
\pgfpathlineto{\pgfqpoint{3.204836in}{1.373525in}}%
\pgfpathlineto{\pgfqpoint{3.205051in}{1.375529in}}%
\pgfpathlineto{\pgfqpoint{3.205914in}{1.433697in}}%
\pgfpathlineto{\pgfqpoint{3.208071in}{1.860845in}}%
\pgfpathlineto{\pgfqpoint{3.212025in}{2.571850in}}%
\pgfpathlineto{\pgfqpoint{3.212313in}{2.567855in}}%
\pgfpathlineto{\pgfqpoint{3.213319in}{2.478938in}}%
\pgfpathlineto{\pgfqpoint{3.216051in}{1.855154in}}%
\pgfpathlineto{\pgfqpoint{3.219214in}{1.373525in}}%
\pgfpathlineto{\pgfqpoint{3.219430in}{1.375529in}}%
\pgfpathlineto{\pgfqpoint{3.220293in}{1.433697in}}%
\pgfpathlineto{\pgfqpoint{3.222450in}{1.860845in}}%
\pgfpathlineto{\pgfqpoint{3.226404in}{2.571850in}}%
\pgfpathlineto{\pgfqpoint{3.226691in}{2.567855in}}%
\pgfpathlineto{\pgfqpoint{3.227698in}{2.478938in}}%
\pgfpathlineto{\pgfqpoint{3.230430in}{1.855154in}}%
\pgfpathlineto{\pgfqpoint{3.233593in}{1.373525in}}%
\pgfpathlineto{\pgfqpoint{3.233809in}{1.375529in}}%
\pgfpathlineto{\pgfqpoint{3.234672in}{1.433697in}}%
\pgfpathlineto{\pgfqpoint{3.236828in}{1.860845in}}%
\pgfpathlineto{\pgfqpoint{3.240783in}{2.571850in}}%
\pgfpathlineto{\pgfqpoint{3.241070in}{2.567855in}}%
\pgfpathlineto{\pgfqpoint{3.242077in}{2.478938in}}%
\pgfpathlineto{\pgfqpoint{3.244809in}{1.855154in}}%
\pgfpathlineto{\pgfqpoint{3.247972in}{1.373525in}}%
\pgfpathlineto{\pgfqpoint{3.248188in}{1.375529in}}%
\pgfpathlineto{\pgfqpoint{3.249050in}{1.433697in}}%
\pgfpathlineto{\pgfqpoint{3.251207in}{1.860845in}}%
\pgfpathlineto{\pgfqpoint{3.255161in}{2.571850in}}%
\pgfpathlineto{\pgfqpoint{3.255449in}{2.567855in}}%
\pgfpathlineto{\pgfqpoint{3.256455in}{2.478938in}}%
\pgfpathlineto{\pgfqpoint{3.259187in}{1.855154in}}%
\pgfpathlineto{\pgfqpoint{3.262351in}{1.373525in}}%
\pgfpathlineto{\pgfqpoint{3.262566in}{1.375529in}}%
\pgfpathlineto{\pgfqpoint{3.263429in}{1.433697in}}%
\pgfpathlineto{\pgfqpoint{3.265586in}{1.860845in}}%
\pgfpathlineto{\pgfqpoint{3.269540in}{2.571850in}}%
\pgfpathlineto{\pgfqpoint{3.269828in}{2.567855in}}%
\pgfpathlineto{\pgfqpoint{3.270834in}{2.478938in}}%
\pgfpathlineto{\pgfqpoint{3.273566in}{1.855154in}}%
\pgfpathlineto{\pgfqpoint{3.276730in}{1.373525in}}%
\pgfpathlineto{\pgfqpoint{3.276945in}{1.375529in}}%
\pgfpathlineto{\pgfqpoint{3.277808in}{1.433697in}}%
\pgfpathlineto{\pgfqpoint{3.279965in}{1.860845in}}%
\pgfpathlineto{\pgfqpoint{3.283919in}{2.571850in}}%
\pgfpathlineto{\pgfqpoint{3.284207in}{2.567855in}}%
\pgfpathlineto{\pgfqpoint{3.285213in}{2.478938in}}%
\pgfpathlineto{\pgfqpoint{3.287945in}{1.855154in}}%
\pgfpathlineto{\pgfqpoint{3.291108in}{1.373525in}}%
\pgfpathlineto{\pgfqpoint{3.291324in}{1.375529in}}%
\pgfpathlineto{\pgfqpoint{3.292187in}{1.433697in}}%
\pgfpathlineto{\pgfqpoint{3.294344in}{1.860845in}}%
\pgfpathlineto{\pgfqpoint{3.298298in}{2.571850in}}%
\pgfpathlineto{\pgfqpoint{3.298585in}{2.567855in}}%
\pgfpathlineto{\pgfqpoint{3.299592in}{2.478938in}}%
\pgfpathlineto{\pgfqpoint{3.302324in}{1.855154in}}%
\pgfpathlineto{\pgfqpoint{3.305487in}{1.373525in}}%
\pgfpathlineto{\pgfqpoint{3.305703in}{1.375529in}}%
\pgfpathlineto{\pgfqpoint{3.306566in}{1.433697in}}%
\pgfpathlineto{\pgfqpoint{3.308722in}{1.860845in}}%
\pgfpathlineto{\pgfqpoint{3.312676in}{2.571850in}}%
\pgfpathlineto{\pgfqpoint{3.312964in}{2.567855in}}%
\pgfpathlineto{\pgfqpoint{3.313971in}{2.478938in}}%
\pgfpathlineto{\pgfqpoint{3.316703in}{1.855154in}}%
\pgfpathlineto{\pgfqpoint{3.319866in}{1.373525in}}%
\pgfpathlineto{\pgfqpoint{3.320082in}{1.375529in}}%
\pgfpathlineto{\pgfqpoint{3.320944in}{1.433697in}}%
\pgfpathlineto{\pgfqpoint{3.323101in}{1.860845in}}%
\pgfpathlineto{\pgfqpoint{3.327055in}{2.571850in}}%
\pgfpathlineto{\pgfqpoint{3.327343in}{2.567855in}}%
\pgfpathlineto{\pgfqpoint{3.328349in}{2.478938in}}%
\pgfpathlineto{\pgfqpoint{3.331081in}{1.855154in}}%
\pgfpathlineto{\pgfqpoint{3.334245in}{1.373525in}}%
\pgfpathlineto{\pgfqpoint{3.334460in}{1.375529in}}%
\pgfpathlineto{\pgfqpoint{3.335323in}{1.433697in}}%
\pgfpathlineto{\pgfqpoint{3.337480in}{1.860845in}}%
\pgfpathlineto{\pgfqpoint{3.341434in}{2.571850in}}%
\pgfpathlineto{\pgfqpoint{3.341722in}{2.567855in}}%
\pgfpathlineto{\pgfqpoint{3.342728in}{2.478938in}}%
\pgfpathlineto{\pgfqpoint{3.345460in}{1.855154in}}%
\pgfpathlineto{\pgfqpoint{3.348623in}{1.373525in}}%
\pgfpathlineto{\pgfqpoint{3.348839in}{1.375529in}}%
\pgfpathlineto{\pgfqpoint{3.349702in}{1.433697in}}%
\pgfpathlineto{\pgfqpoint{3.351859in}{1.860845in}}%
\pgfpathlineto{\pgfqpoint{3.355813in}{2.571850in}}%
\pgfpathlineto{\pgfqpoint{3.356100in}{2.567855in}}%
\pgfpathlineto{\pgfqpoint{3.357107in}{2.478938in}}%
\pgfpathlineto{\pgfqpoint{3.359839in}{1.855154in}}%
\pgfpathlineto{\pgfqpoint{3.363002in}{1.373525in}}%
\pgfpathlineto{\pgfqpoint{3.363218in}{1.375529in}}%
\pgfpathlineto{\pgfqpoint{3.364081in}{1.433697in}}%
\pgfpathlineto{\pgfqpoint{3.366237in}{1.860845in}}%
\pgfpathlineto{\pgfqpoint{3.370192in}{2.571850in}}%
\pgfpathlineto{\pgfqpoint{3.370479in}{2.567855in}}%
\pgfpathlineto{\pgfqpoint{3.371486in}{2.478938in}}%
\pgfpathlineto{\pgfqpoint{3.374218in}{1.855154in}}%
\pgfpathlineto{\pgfqpoint{3.377381in}{1.373525in}}%
\pgfpathlineto{\pgfqpoint{3.377597in}{1.375529in}}%
\pgfpathlineto{\pgfqpoint{3.378459in}{1.433697in}}%
\pgfpathlineto{\pgfqpoint{3.380616in}{1.860845in}}%
\pgfpathlineto{\pgfqpoint{3.384570in}{2.571850in}}%
\pgfpathlineto{\pgfqpoint{3.384858in}{2.567855in}}%
\pgfpathlineto{\pgfqpoint{3.385864in}{2.478938in}}%
\pgfpathlineto{\pgfqpoint{3.388596in}{1.855154in}}%
\pgfpathlineto{\pgfqpoint{3.391760in}{1.373525in}}%
\pgfpathlineto{\pgfqpoint{3.391975in}{1.375529in}}%
\pgfpathlineto{\pgfqpoint{3.392838in}{1.433697in}}%
\pgfpathlineto{\pgfqpoint{3.394995in}{1.860845in}}%
\pgfpathlineto{\pgfqpoint{3.398949in}{2.571850in}}%
\pgfpathlineto{\pgfqpoint{3.399237in}{2.567855in}}%
\pgfpathlineto{\pgfqpoint{3.400243in}{2.478938in}}%
\pgfpathlineto{\pgfqpoint{3.402975in}{1.855154in}}%
\pgfpathlineto{\pgfqpoint{3.406139in}{1.373525in}}%
\pgfpathlineto{\pgfqpoint{3.406354in}{1.375529in}}%
\pgfpathlineto{\pgfqpoint{3.407217in}{1.433697in}}%
\pgfpathlineto{\pgfqpoint{3.409374in}{1.860845in}}%
\pgfpathlineto{\pgfqpoint{3.413328in}{2.571850in}}%
\pgfpathlineto{\pgfqpoint{3.413615in}{2.567855in}}%
\pgfpathlineto{\pgfqpoint{3.414622in}{2.478938in}}%
\pgfpathlineto{\pgfqpoint{3.417354in}{1.855154in}}%
\pgfpathlineto{\pgfqpoint{3.420517in}{1.373525in}}%
\pgfpathlineto{\pgfqpoint{3.420733in}{1.375529in}}%
\pgfpathlineto{\pgfqpoint{3.421596in}{1.433697in}}%
\pgfpathlineto{\pgfqpoint{3.423753in}{1.860845in}}%
\pgfpathlineto{\pgfqpoint{3.427707in}{2.571850in}}%
\pgfpathlineto{\pgfqpoint{3.427994in}{2.567855in}}%
\pgfpathlineto{\pgfqpoint{3.429001in}{2.478938in}}%
\pgfpathlineto{\pgfqpoint{3.431733in}{1.855154in}}%
\pgfpathlineto{\pgfqpoint{3.434896in}{1.373525in}}%
\pgfpathlineto{\pgfqpoint{3.435112in}{1.375529in}}%
\pgfpathlineto{\pgfqpoint{3.435974in}{1.433697in}}%
\pgfpathlineto{\pgfqpoint{3.438131in}{1.860845in}}%
\pgfpathlineto{\pgfqpoint{3.442085in}{2.571850in}}%
\pgfpathlineto{\pgfqpoint{3.442373in}{2.567855in}}%
\pgfpathlineto{\pgfqpoint{3.443380in}{2.478938in}}%
\pgfpathlineto{\pgfqpoint{3.446111in}{1.855154in}}%
\pgfpathlineto{\pgfqpoint{3.449275in}{1.373525in}}%
\pgfpathlineto{\pgfqpoint{3.449491in}{1.375529in}}%
\pgfpathlineto{\pgfqpoint{3.450353in}{1.433697in}}%
\pgfpathlineto{\pgfqpoint{3.452510in}{1.860845in}}%
\pgfpathlineto{\pgfqpoint{3.456464in}{2.571850in}}%
\pgfpathlineto{\pgfqpoint{3.456752in}{2.567855in}}%
\pgfpathlineto{\pgfqpoint{3.457758in}{2.478938in}}%
\pgfpathlineto{\pgfqpoint{3.460490in}{1.855154in}}%
\pgfpathlineto{\pgfqpoint{3.463654in}{1.373525in}}%
\pgfpathlineto{\pgfqpoint{3.463869in}{1.375529in}}%
\pgfpathlineto{\pgfqpoint{3.464732in}{1.433697in}}%
\pgfpathlineto{\pgfqpoint{3.466889in}{1.860845in}}%
\pgfpathlineto{\pgfqpoint{3.470843in}{2.571850in}}%
\pgfpathlineto{\pgfqpoint{3.471131in}{2.567855in}}%
\pgfpathlineto{\pgfqpoint{3.472137in}{2.478938in}}%
\pgfpathlineto{\pgfqpoint{3.474869in}{1.855154in}}%
\pgfpathlineto{\pgfqpoint{3.478032in}{1.373525in}}%
\pgfpathlineto{\pgfqpoint{3.478248in}{1.375529in}}%
\pgfpathlineto{\pgfqpoint{3.479111in}{1.433697in}}%
\pgfpathlineto{\pgfqpoint{3.481268in}{1.860845in}}%
\pgfpathlineto{\pgfqpoint{3.485222in}{2.571850in}}%
\pgfpathlineto{\pgfqpoint{3.485509in}{2.567855in}}%
\pgfpathlineto{\pgfqpoint{3.486516in}{2.478938in}}%
\pgfpathlineto{\pgfqpoint{3.489248in}{1.855154in}}%
\pgfpathlineto{\pgfqpoint{3.492411in}{1.373525in}}%
\pgfpathlineto{\pgfqpoint{3.492627in}{1.375529in}}%
\pgfpathlineto{\pgfqpoint{3.493490in}{1.433697in}}%
\pgfpathlineto{\pgfqpoint{3.495646in}{1.860845in}}%
\pgfpathlineto{\pgfqpoint{3.499601in}{2.571850in}}%
\pgfpathlineto{\pgfqpoint{3.499888in}{2.567855in}}%
\pgfpathlineto{\pgfqpoint{3.500895in}{2.478938in}}%
\pgfpathlineto{\pgfqpoint{3.503627in}{1.855154in}}%
\pgfpathlineto{\pgfqpoint{3.506790in}{1.373525in}}%
\pgfpathlineto{\pgfqpoint{3.507006in}{1.375529in}}%
\pgfpathlineto{\pgfqpoint{3.507868in}{1.433697in}}%
\pgfpathlineto{\pgfqpoint{3.510025in}{1.860845in}}%
\pgfpathlineto{\pgfqpoint{3.513979in}{2.571850in}}%
\pgfpathlineto{\pgfqpoint{3.514267in}{2.567855in}}%
\pgfpathlineto{\pgfqpoint{3.515273in}{2.478938in}}%
\pgfpathlineto{\pgfqpoint{3.518005in}{1.855154in}}%
\pgfpathlineto{\pgfqpoint{3.521169in}{1.373525in}}%
\pgfpathlineto{\pgfqpoint{3.521384in}{1.375529in}}%
\pgfpathlineto{\pgfqpoint{3.522247in}{1.433697in}}%
\pgfpathlineto{\pgfqpoint{3.524404in}{1.860845in}}%
\pgfpathlineto{\pgfqpoint{3.528358in}{2.571850in}}%
\pgfpathlineto{\pgfqpoint{3.528646in}{2.567855in}}%
\pgfpathlineto{\pgfqpoint{3.529652in}{2.478938in}}%
\pgfpathlineto{\pgfqpoint{3.532384in}{1.855154in}}%
\pgfpathlineto{\pgfqpoint{3.535547in}{1.373525in}}%
\pgfpathlineto{\pgfqpoint{3.535763in}{1.375529in}}%
\pgfpathlineto{\pgfqpoint{3.536626in}{1.433697in}}%
\pgfpathlineto{\pgfqpoint{3.538783in}{1.860845in}}%
\pgfpathlineto{\pgfqpoint{3.542737in}{2.571850in}}%
\pgfpathlineto{\pgfqpoint{3.543024in}{2.567855in}}%
\pgfpathlineto{\pgfqpoint{3.544031in}{2.478938in}}%
\pgfpathlineto{\pgfqpoint{3.546763in}{1.855154in}}%
\pgfpathlineto{\pgfqpoint{3.549926in}{1.373525in}}%
\pgfpathlineto{\pgfqpoint{3.550142in}{1.375529in}}%
\pgfpathlineto{\pgfqpoint{3.551005in}{1.433697in}}%
\pgfpathlineto{\pgfqpoint{3.553161in}{1.860845in}}%
\pgfpathlineto{\pgfqpoint{3.557116in}{2.571850in}}%
\pgfpathlineto{\pgfqpoint{3.557403in}{2.567855in}}%
\pgfpathlineto{\pgfqpoint{3.558410in}{2.478938in}}%
\pgfpathlineto{\pgfqpoint{3.561142in}{1.855154in}}%
\pgfpathlineto{\pgfqpoint{3.564305in}{1.373525in}}%
\pgfpathlineto{\pgfqpoint{3.564521in}{1.375529in}}%
\pgfpathlineto{\pgfqpoint{3.565383in}{1.433697in}}%
\pgfpathlineto{\pgfqpoint{3.567540in}{1.860845in}}%
\pgfpathlineto{\pgfqpoint{3.571494in}{2.571850in}}%
\pgfpathlineto{\pgfqpoint{3.571782in}{2.567855in}}%
\pgfpathlineto{\pgfqpoint{3.572788in}{2.478938in}}%
\pgfpathlineto{\pgfqpoint{3.575520in}{1.855154in}}%
\pgfpathlineto{\pgfqpoint{3.578684in}{1.373525in}}%
\pgfpathlineto{\pgfqpoint{3.578899in}{1.375529in}}%
\pgfpathlineto{\pgfqpoint{3.579762in}{1.433697in}}%
\pgfpathlineto{\pgfqpoint{3.581919in}{1.860845in}}%
\pgfpathlineto{\pgfqpoint{3.585873in}{2.571850in}}%
\pgfpathlineto{\pgfqpoint{3.586161in}{2.567855in}}%
\pgfpathlineto{\pgfqpoint{3.587167in}{2.478938in}}%
\pgfpathlineto{\pgfqpoint{3.589899in}{1.855154in}}%
\pgfpathlineto{\pgfqpoint{3.593063in}{1.373525in}}%
\pgfpathlineto{\pgfqpoint{3.593278in}{1.375529in}}%
\pgfpathlineto{\pgfqpoint{3.594141in}{1.433697in}}%
\pgfpathlineto{\pgfqpoint{3.596298in}{1.860845in}}%
\pgfpathlineto{\pgfqpoint{3.600252in}{2.571850in}}%
\pgfpathlineto{\pgfqpoint{3.600540in}{2.567855in}}%
\pgfpathlineto{\pgfqpoint{3.601546in}{2.478938in}}%
\pgfpathlineto{\pgfqpoint{3.604278in}{1.855154in}}%
\pgfpathlineto{\pgfqpoint{3.607441in}{1.373525in}}%
\pgfpathlineto{\pgfqpoint{3.607657in}{1.375529in}}%
\pgfpathlineto{\pgfqpoint{3.608520in}{1.433697in}}%
\pgfpathlineto{\pgfqpoint{3.610677in}{1.860845in}}%
\pgfpathlineto{\pgfqpoint{3.614631in}{2.571850in}}%
\pgfpathlineto{\pgfqpoint{3.614918in}{2.567855in}}%
\pgfpathlineto{\pgfqpoint{3.615925in}{2.478938in}}%
\pgfpathlineto{\pgfqpoint{3.618657in}{1.855154in}}%
\pgfpathlineto{\pgfqpoint{3.621820in}{1.373525in}}%
\pgfpathlineto{\pgfqpoint{3.622036in}{1.375529in}}%
\pgfpathlineto{\pgfqpoint{3.622898in}{1.433697in}}%
\pgfpathlineto{\pgfqpoint{3.625055in}{1.860845in}}%
\pgfpathlineto{\pgfqpoint{3.629009in}{2.571850in}}%
\pgfpathlineto{\pgfqpoint{3.629297in}{2.567855in}}%
\pgfpathlineto{\pgfqpoint{3.630304in}{2.478938in}}%
\pgfpathlineto{\pgfqpoint{3.633036in}{1.855154in}}%
\pgfpathlineto{\pgfqpoint{3.636199in}{1.373525in}}%
\pgfpathlineto{\pgfqpoint{3.636415in}{1.375529in}}%
\pgfpathlineto{\pgfqpoint{3.637277in}{1.433697in}}%
\pgfpathlineto{\pgfqpoint{3.639434in}{1.860845in}}%
\pgfpathlineto{\pgfqpoint{3.643388in}{2.571850in}}%
\pgfpathlineto{\pgfqpoint{3.643676in}{2.567855in}}%
\pgfpathlineto{\pgfqpoint{3.644682in}{2.478938in}}%
\pgfpathlineto{\pgfqpoint{3.647414in}{1.855154in}}%
\pgfpathlineto{\pgfqpoint{3.650578in}{1.373525in}}%
\pgfpathlineto{\pgfqpoint{3.650793in}{1.375529in}}%
\pgfpathlineto{\pgfqpoint{3.651656in}{1.433697in}}%
\pgfpathlineto{\pgfqpoint{3.653813in}{1.860845in}}%
\pgfpathlineto{\pgfqpoint{3.657767in}{2.571850in}}%
\pgfpathlineto{\pgfqpoint{3.658055in}{2.567855in}}%
\pgfpathlineto{\pgfqpoint{3.659061in}{2.478938in}}%
\pgfpathlineto{\pgfqpoint{3.661793in}{1.855154in}}%
\pgfpathlineto{\pgfqpoint{3.664956in}{1.373525in}}%
\pgfpathlineto{\pgfqpoint{3.665172in}{1.375529in}}%
\pgfpathlineto{\pgfqpoint{3.666035in}{1.433697in}}%
\pgfpathlineto{\pgfqpoint{3.668192in}{1.860845in}}%
\pgfpathlineto{\pgfqpoint{3.672146in}{2.571850in}}%
\pgfpathlineto{\pgfqpoint{3.672433in}{2.567855in}}%
\pgfpathlineto{\pgfqpoint{3.673440in}{2.478938in}}%
\pgfpathlineto{\pgfqpoint{3.676172in}{1.855154in}}%
\pgfpathlineto{\pgfqpoint{3.679335in}{1.373525in}}%
\pgfpathlineto{\pgfqpoint{3.679551in}{1.375529in}}%
\pgfpathlineto{\pgfqpoint{3.680414in}{1.433697in}}%
\pgfpathlineto{\pgfqpoint{3.682570in}{1.860845in}}%
\pgfpathlineto{\pgfqpoint{3.686525in}{2.571850in}}%
\pgfpathlineto{\pgfqpoint{3.686812in}{2.567855in}}%
\pgfpathlineto{\pgfqpoint{3.687819in}{2.478938in}}%
\pgfpathlineto{\pgfqpoint{3.690551in}{1.855154in}}%
\pgfpathlineto{\pgfqpoint{3.693714in}{1.373525in}}%
\pgfpathlineto{\pgfqpoint{3.693930in}{1.375529in}}%
\pgfpathlineto{\pgfqpoint{3.694792in}{1.433697in}}%
\pgfpathlineto{\pgfqpoint{3.696949in}{1.860845in}}%
\pgfpathlineto{\pgfqpoint{3.700903in}{2.571850in}}%
\pgfpathlineto{\pgfqpoint{3.701191in}{2.567855in}}%
\pgfpathlineto{\pgfqpoint{3.702197in}{2.478938in}}%
\pgfpathlineto{\pgfqpoint{3.704929in}{1.855154in}}%
\pgfpathlineto{\pgfqpoint{3.708093in}{1.373525in}}%
\pgfpathlineto{\pgfqpoint{3.708308in}{1.375529in}}%
\pgfpathlineto{\pgfqpoint{3.709171in}{1.433697in}}%
\pgfpathlineto{\pgfqpoint{3.711328in}{1.860845in}}%
\pgfpathlineto{\pgfqpoint{3.715282in}{2.571850in}}%
\pgfpathlineto{\pgfqpoint{3.715570in}{2.567855in}}%
\pgfpathlineto{\pgfqpoint{3.716576in}{2.478938in}}%
\pgfpathlineto{\pgfqpoint{3.719308in}{1.855154in}}%
\pgfpathlineto{\pgfqpoint{3.722471in}{1.373525in}}%
\pgfpathlineto{\pgfqpoint{3.722687in}{1.375529in}}%
\pgfpathlineto{\pgfqpoint{3.723550in}{1.433697in}}%
\pgfpathlineto{\pgfqpoint{3.725707in}{1.860845in}}%
\pgfpathlineto{\pgfqpoint{3.729661in}{2.571850in}}%
\pgfpathlineto{\pgfqpoint{3.729948in}{2.567855in}}%
\pgfpathlineto{\pgfqpoint{3.730955in}{2.478938in}}%
\pgfpathlineto{\pgfqpoint{3.733687in}{1.855154in}}%
\pgfpathlineto{\pgfqpoint{3.736850in}{1.373525in}}%
\pgfpathlineto{\pgfqpoint{3.737066in}{1.375529in}}%
\pgfpathlineto{\pgfqpoint{3.737929in}{1.433697in}}%
\pgfpathlineto{\pgfqpoint{3.740085in}{1.860845in}}%
\pgfpathlineto{\pgfqpoint{3.744040in}{2.571850in}}%
\pgfpathlineto{\pgfqpoint{3.744327in}{2.567855in}}%
\pgfpathlineto{\pgfqpoint{3.745334in}{2.478938in}}%
\pgfpathlineto{\pgfqpoint{3.748066in}{1.855154in}}%
\pgfpathlineto{\pgfqpoint{3.751229in}{1.373525in}}%
\pgfpathlineto{\pgfqpoint{3.751445in}{1.375529in}}%
\pgfpathlineto{\pgfqpoint{3.752307in}{1.433697in}}%
\pgfpathlineto{\pgfqpoint{3.754464in}{1.860845in}}%
\pgfpathlineto{\pgfqpoint{3.758418in}{2.571850in}}%
\pgfpathlineto{\pgfqpoint{3.758706in}{2.567855in}}%
\pgfpathlineto{\pgfqpoint{3.759713in}{2.478938in}}%
\pgfpathlineto{\pgfqpoint{3.762444in}{1.855154in}}%
\pgfpathlineto{\pgfqpoint{3.765608in}{1.373525in}}%
\pgfpathlineto{\pgfqpoint{3.765823in}{1.375529in}}%
\pgfpathlineto{\pgfqpoint{3.766686in}{1.433697in}}%
\pgfpathlineto{\pgfqpoint{3.768843in}{1.860845in}}%
\pgfpathlineto{\pgfqpoint{3.772797in}{2.571850in}}%
\pgfpathlineto{\pgfqpoint{3.773085in}{2.567855in}}%
\pgfpathlineto{\pgfqpoint{3.774091in}{2.478938in}}%
\pgfpathlineto{\pgfqpoint{3.776823in}{1.855154in}}%
\pgfpathlineto{\pgfqpoint{3.779987in}{1.373525in}}%
\pgfpathlineto{\pgfqpoint{3.780202in}{1.375529in}}%
\pgfpathlineto{\pgfqpoint{3.781065in}{1.433697in}}%
\pgfpathlineto{\pgfqpoint{3.783222in}{1.860845in}}%
\pgfpathlineto{\pgfqpoint{3.787176in}{2.571850in}}%
\pgfpathlineto{\pgfqpoint{3.787464in}{2.567855in}}%
\pgfpathlineto{\pgfqpoint{3.788470in}{2.478938in}}%
\pgfpathlineto{\pgfqpoint{3.791202in}{1.855154in}}%
\pgfpathlineto{\pgfqpoint{3.794365in}{1.373525in}}%
\pgfpathlineto{\pgfqpoint{3.794581in}{1.375529in}}%
\pgfpathlineto{\pgfqpoint{3.795444in}{1.433697in}}%
\pgfpathlineto{\pgfqpoint{3.797601in}{1.860845in}}%
\pgfpathlineto{\pgfqpoint{3.801555in}{2.571850in}}%
\pgfpathlineto{\pgfqpoint{3.801842in}{2.567855in}}%
\pgfpathlineto{\pgfqpoint{3.802849in}{2.478938in}}%
\pgfpathlineto{\pgfqpoint{3.805581in}{1.855154in}}%
\pgfpathlineto{\pgfqpoint{3.808744in}{1.373525in}}%
\pgfpathlineto{\pgfqpoint{3.808960in}{1.375529in}}%
\pgfpathlineto{\pgfqpoint{3.809823in}{1.433697in}}%
\pgfpathlineto{\pgfqpoint{3.811979in}{1.860845in}}%
\pgfpathlineto{\pgfqpoint{3.815934in}{2.571850in}}%
\pgfpathlineto{\pgfqpoint{3.816221in}{2.567855in}}%
\pgfpathlineto{\pgfqpoint{3.817228in}{2.478938in}}%
\pgfpathlineto{\pgfqpoint{3.819960in}{1.855154in}}%
\pgfpathlineto{\pgfqpoint{3.823123in}{1.373525in}}%
\pgfpathlineto{\pgfqpoint{3.823339in}{1.375529in}}%
\pgfpathlineto{\pgfqpoint{3.824201in}{1.433697in}}%
\pgfpathlineto{\pgfqpoint{3.826358in}{1.860845in}}%
\pgfpathlineto{\pgfqpoint{3.830312in}{2.571850in}}%
\pgfpathlineto{\pgfqpoint{3.830600in}{2.567855in}}%
\pgfpathlineto{\pgfqpoint{3.831606in}{2.478938in}}%
\pgfpathlineto{\pgfqpoint{3.834338in}{1.855154in}}%
\pgfpathlineto{\pgfqpoint{3.837502in}{1.373525in}}%
\pgfpathlineto{\pgfqpoint{3.837717in}{1.375529in}}%
\pgfpathlineto{\pgfqpoint{3.838580in}{1.433697in}}%
\pgfpathlineto{\pgfqpoint{3.840737in}{1.860845in}}%
\pgfpathlineto{\pgfqpoint{3.844691in}{2.571850in}}%
\pgfpathlineto{\pgfqpoint{3.844979in}{2.567855in}}%
\pgfpathlineto{\pgfqpoint{3.845985in}{2.478938in}}%
\pgfpathlineto{\pgfqpoint{3.848717in}{1.855154in}}%
\pgfpathlineto{\pgfqpoint{3.851880in}{1.373525in}}%
\pgfpathlineto{\pgfqpoint{3.852096in}{1.375529in}}%
\pgfpathlineto{\pgfqpoint{3.852959in}{1.433697in}}%
\pgfpathlineto{\pgfqpoint{3.855116in}{1.860845in}}%
\pgfpathlineto{\pgfqpoint{3.859070in}{2.571850in}}%
\pgfpathlineto{\pgfqpoint{3.859357in}{2.567855in}}%
\pgfpathlineto{\pgfqpoint{3.860364in}{2.478938in}}%
\pgfpathlineto{\pgfqpoint{3.863096in}{1.855154in}}%
\pgfpathlineto{\pgfqpoint{3.866259in}{1.373525in}}%
\pgfpathlineto{\pgfqpoint{3.866475in}{1.375529in}}%
\pgfpathlineto{\pgfqpoint{3.867338in}{1.433697in}}%
\pgfpathlineto{\pgfqpoint{3.869494in}{1.860845in}}%
\pgfpathlineto{\pgfqpoint{3.873449in}{2.571850in}}%
\pgfpathlineto{\pgfqpoint{3.873736in}{2.567855in}}%
\pgfpathlineto{\pgfqpoint{3.874743in}{2.478938in}}%
\pgfpathlineto{\pgfqpoint{3.877475in}{1.855154in}}%
\pgfpathlineto{\pgfqpoint{3.880638in}{1.373525in}}%
\pgfpathlineto{\pgfqpoint{3.880854in}{1.375529in}}%
\pgfpathlineto{\pgfqpoint{3.881716in}{1.433697in}}%
\pgfpathlineto{\pgfqpoint{3.883873in}{1.860845in}}%
\pgfpathlineto{\pgfqpoint{3.887827in}{2.571850in}}%
\pgfpathlineto{\pgfqpoint{3.888115in}{2.567855in}}%
\pgfpathlineto{\pgfqpoint{3.889121in}{2.478938in}}%
\pgfpathlineto{\pgfqpoint{3.891853in}{1.855154in}}%
\pgfpathlineto{\pgfqpoint{3.895017in}{1.373525in}}%
\pgfpathlineto{\pgfqpoint{3.895232in}{1.375529in}}%
\pgfpathlineto{\pgfqpoint{3.896095in}{1.433697in}}%
\pgfpathlineto{\pgfqpoint{3.898252in}{1.860845in}}%
\pgfpathlineto{\pgfqpoint{3.902206in}{2.571850in}}%
\pgfpathlineto{\pgfqpoint{3.902494in}{2.567855in}}%
\pgfpathlineto{\pgfqpoint{3.903500in}{2.478938in}}%
\pgfpathlineto{\pgfqpoint{3.906232in}{1.855154in}}%
\pgfpathlineto{\pgfqpoint{3.909396in}{1.373525in}}%
\pgfpathlineto{\pgfqpoint{3.909611in}{1.375529in}}%
\pgfpathlineto{\pgfqpoint{3.910474in}{1.433697in}}%
\pgfpathlineto{\pgfqpoint{3.912631in}{1.860845in}}%
\pgfpathlineto{\pgfqpoint{3.916585in}{2.571850in}}%
\pgfpathlineto{\pgfqpoint{3.916872in}{2.567855in}}%
\pgfpathlineto{\pgfqpoint{3.917879in}{2.478938in}}%
\pgfpathlineto{\pgfqpoint{3.920611in}{1.855154in}}%
\pgfpathlineto{\pgfqpoint{3.923774in}{1.373525in}}%
\pgfpathlineto{\pgfqpoint{3.923990in}{1.375529in}}%
\pgfpathlineto{\pgfqpoint{3.924853in}{1.433697in}}%
\pgfpathlineto{\pgfqpoint{3.927010in}{1.860845in}}%
\pgfpathlineto{\pgfqpoint{3.930964in}{2.571850in}}%
\pgfpathlineto{\pgfqpoint{3.931251in}{2.567855in}}%
\pgfpathlineto{\pgfqpoint{3.932258in}{2.478938in}}%
\pgfpathlineto{\pgfqpoint{3.934990in}{1.855154in}}%
\pgfpathlineto{\pgfqpoint{3.938153in}{1.373525in}}%
\pgfpathlineto{\pgfqpoint{3.938369in}{1.375529in}}%
\pgfpathlineto{\pgfqpoint{3.939231in}{1.433697in}}%
\pgfpathlineto{\pgfqpoint{3.941388in}{1.860845in}}%
\pgfpathlineto{\pgfqpoint{3.945342in}{2.571850in}}%
\pgfpathlineto{\pgfqpoint{3.945630in}{2.567855in}}%
\pgfpathlineto{\pgfqpoint{3.946637in}{2.478938in}}%
\pgfpathlineto{\pgfqpoint{3.949369in}{1.855154in}}%
\pgfpathlineto{\pgfqpoint{3.952532in}{1.373525in}}%
\pgfpathlineto{\pgfqpoint{3.952748in}{1.375529in}}%
\pgfpathlineto{\pgfqpoint{3.953610in}{1.433697in}}%
\pgfpathlineto{\pgfqpoint{3.955767in}{1.860845in}}%
\pgfpathlineto{\pgfqpoint{3.959721in}{2.571850in}}%
\pgfpathlineto{\pgfqpoint{3.960009in}{2.567855in}}%
\pgfpathlineto{\pgfqpoint{3.961015in}{2.478938in}}%
\pgfpathlineto{\pgfqpoint{3.963747in}{1.855154in}}%
\pgfpathlineto{\pgfqpoint{3.966911in}{1.373525in}}%
\pgfpathlineto{\pgfqpoint{3.967126in}{1.375529in}}%
\pgfpathlineto{\pgfqpoint{3.967989in}{1.433697in}}%
\pgfpathlineto{\pgfqpoint{3.968277in}{1.470027in}}%
\pgfpathlineto{\pgfqpoint{3.969858in}{1.966658in}}%
\pgfpathlineto{\pgfqpoint{4.323864in}{1.966658in}}%
\pgfpathlineto{\pgfqpoint{4.323864in}{1.966658in}}%
\pgfusepath{stroke}%
\end{pgfscope}%
\begin{pgfscope}%
\pgfsetrectcap%
\pgfsetmiterjoin%
\pgfsetlinewidth{0.803000pt}%
\definecolor{currentstroke}{rgb}{0.000000,0.000000,0.000000}%
\pgfsetstrokecolor{currentstroke}%
\pgfsetdash{}{0pt}%
\pgfpathmoveto{\pgfqpoint{0.625000in}{0.500000in}}%
\pgfpathlineto{\pgfqpoint{0.625000in}{3.520000in}}%
\pgfusepath{stroke}%
\end{pgfscope}%
\begin{pgfscope}%
\pgfsetrectcap%
\pgfsetmiterjoin%
\pgfsetlinewidth{0.803000pt}%
\definecolor{currentstroke}{rgb}{0.000000,0.000000,0.000000}%
\pgfsetstrokecolor{currentstroke}%
\pgfsetdash{}{0pt}%
\pgfpathmoveto{\pgfqpoint{4.500000in}{0.500000in}}%
\pgfpathlineto{\pgfqpoint{4.500000in}{3.520000in}}%
\pgfusepath{stroke}%
\end{pgfscope}%
\begin{pgfscope}%
\pgfsetrectcap%
\pgfsetmiterjoin%
\pgfsetlinewidth{0.803000pt}%
\definecolor{currentstroke}{rgb}{0.000000,0.000000,0.000000}%
\pgfsetstrokecolor{currentstroke}%
\pgfsetdash{}{0pt}%
\pgfpathmoveto{\pgfqpoint{0.625000in}{0.500000in}}%
\pgfpathlineto{\pgfqpoint{4.500000in}{0.500000in}}%
\pgfusepath{stroke}%
\end{pgfscope}%
\begin{pgfscope}%
\pgfsetrectcap%
\pgfsetmiterjoin%
\pgfsetlinewidth{0.803000pt}%
\definecolor{currentstroke}{rgb}{0.000000,0.000000,0.000000}%
\pgfsetstrokecolor{currentstroke}%
\pgfsetdash{}{0pt}%
\pgfpathmoveto{\pgfqpoint{0.625000in}{3.520000in}}%
\pgfpathlineto{\pgfqpoint{4.500000in}{3.520000in}}%
\pgfusepath{stroke}%
\end{pgfscope}%
\end{pgfpicture}%
\makeatother%
\endgroup%

    \caption{PEC Data Set Single Phase Sag}
    \label{fig:pec_single_phase_sag}
\end{figure}

Figure \ref{fig:pec_three_phase_grid_fault} shows the frequency behavior of the system during a Three Phase Grid Fault. This is another sever fault where there is a problem with the grid and corrective action needs to be taken promptly. In this fault there is a large magnitude drop in frequency for the duration of the fault. This is similar behavior to the Three Phase Sensor Fault but the frequency change of the difference is orders of magnitude larger and in the negative direction.

\begin{figure}[H]
    %\centering
    %% Creator: Matplotlib, PGF backend
%%
%% To include the figure in your LaTeX document, write
%%   \input{<filename>.pgf}
%%
%% Make sure the required packages are loaded in your preamble
%%   \usepackage{pgf}
%%
%% Also ensure that all the required font packages are loaded; for instance,
%% the lmodern package is sometimes necessary when using math font.
%%   \usepackage{lmodern}
%%
%% Figures using additional raster images can only be included by \input if
%% they are in the same directory as the main LaTeX file. For loading figures
%% from other directories you can use the `import` package
%%   \usepackage{import}
%%
%% and then include the figures with
%%   \import{<path to file>}{<filename>.pgf}
%%
%% Matplotlib used the following preamble
%%
\begingroup%
\makeatletter%
\begin{pgfpicture}%
\pgfpathrectangle{\pgfpointorigin}{\pgfqpoint{6.000000in}{4.000000in}}%
\pgfusepath{use as bounding box, clip}%
\begin{pgfscope}%
\pgfsetbuttcap%
\pgfsetmiterjoin%
\pgfsetlinewidth{0.000000pt}%
\definecolor{currentstroke}{rgb}{1.000000,1.000000,1.000000}%
\pgfsetstrokecolor{currentstroke}%
\pgfsetstrokeopacity{0.000000}%
\pgfsetdash{}{0pt}%
\pgfpathmoveto{\pgfqpoint{0.000000in}{0.000000in}}%
\pgfpathlineto{\pgfqpoint{6.000000in}{0.000000in}}%
\pgfpathlineto{\pgfqpoint{6.000000in}{4.000000in}}%
\pgfpathlineto{\pgfqpoint{0.000000in}{4.000000in}}%
\pgfpathlineto{\pgfqpoint{0.000000in}{0.000000in}}%
\pgfpathclose%
\pgfusepath{}%
\end{pgfscope}%
\begin{pgfscope}%
\pgfsetbuttcap%
\pgfsetmiterjoin%
\definecolor{currentfill}{rgb}{1.000000,1.000000,1.000000}%
\pgfsetfillcolor{currentfill}%
\pgfsetlinewidth{0.000000pt}%
\definecolor{currentstroke}{rgb}{0.000000,0.000000,0.000000}%
\pgfsetstrokecolor{currentstroke}%
\pgfsetstrokeopacity{0.000000}%
\pgfsetdash{}{0pt}%
\pgfpathmoveto{\pgfqpoint{0.750000in}{0.500000in}}%
\pgfpathlineto{\pgfqpoint{5.400000in}{0.500000in}}%
\pgfpathlineto{\pgfqpoint{5.400000in}{3.520000in}}%
\pgfpathlineto{\pgfqpoint{0.750000in}{3.520000in}}%
\pgfpathlineto{\pgfqpoint{0.750000in}{0.500000in}}%
\pgfpathclose%
\pgfusepath{fill}%
\end{pgfscope}%
\begin{pgfscope}%
\pgfsetbuttcap%
\pgfsetroundjoin%
\definecolor{currentfill}{rgb}{0.000000,0.000000,0.000000}%
\pgfsetfillcolor{currentfill}%
\pgfsetlinewidth{0.803000pt}%
\definecolor{currentstroke}{rgb}{0.000000,0.000000,0.000000}%
\pgfsetstrokecolor{currentstroke}%
\pgfsetdash{}{0pt}%
\pgfsys@defobject{currentmarker}{\pgfqpoint{0.000000in}{-0.048611in}}{\pgfqpoint{0.000000in}{0.000000in}}{%
\pgfpathmoveto{\pgfqpoint{0.000000in}{0.000000in}}%
\pgfpathlineto{\pgfqpoint{0.000000in}{-0.048611in}}%
\pgfusepath{stroke,fill}%
}%
\begin{pgfscope}%
\pgfsys@transformshift{0.961364in}{0.500000in}%
\pgfsys@useobject{currentmarker}{}%
\end{pgfscope}%
\end{pgfscope}%
\begin{pgfscope}%
\definecolor{textcolor}{rgb}{0.000000,0.000000,0.000000}%
\pgfsetstrokecolor{textcolor}%
\pgfsetfillcolor{textcolor}%
\pgftext[x=0.961364in,y=0.402778in,,top]{\color{textcolor}\rmfamily\fontsize{10.000000}{12.000000}\selectfont \(\displaystyle {280000}\)}%
\end{pgfscope}%
\begin{pgfscope}%
\pgfsetbuttcap%
\pgfsetroundjoin%
\definecolor{currentfill}{rgb}{0.000000,0.000000,0.000000}%
\pgfsetfillcolor{currentfill}%
\pgfsetlinewidth{0.803000pt}%
\definecolor{currentstroke}{rgb}{0.000000,0.000000,0.000000}%
\pgfsetstrokecolor{currentstroke}%
\pgfsetdash{}{0pt}%
\pgfsys@defobject{currentmarker}{\pgfqpoint{0.000000in}{-0.048611in}}{\pgfqpoint{0.000000in}{0.000000in}}{%
\pgfpathmoveto{\pgfqpoint{0.000000in}{0.000000in}}%
\pgfpathlineto{\pgfqpoint{0.000000in}{-0.048611in}}%
\pgfusepath{stroke,fill}%
}%
\begin{pgfscope}%
\pgfsys@transformshift{1.806987in}{0.500000in}%
\pgfsys@useobject{currentmarker}{}%
\end{pgfscope}%
\end{pgfscope}%
\begin{pgfscope}%
\definecolor{textcolor}{rgb}{0.000000,0.000000,0.000000}%
\pgfsetstrokecolor{textcolor}%
\pgfsetfillcolor{textcolor}%
\pgftext[x=1.806987in,y=0.402778in,,top]{\color{textcolor}\rmfamily\fontsize{10.000000}{12.000000}\selectfont \(\displaystyle {281000}\)}%
\end{pgfscope}%
\begin{pgfscope}%
\pgfsetbuttcap%
\pgfsetroundjoin%
\definecolor{currentfill}{rgb}{0.000000,0.000000,0.000000}%
\pgfsetfillcolor{currentfill}%
\pgfsetlinewidth{0.803000pt}%
\definecolor{currentstroke}{rgb}{0.000000,0.000000,0.000000}%
\pgfsetstrokecolor{currentstroke}%
\pgfsetdash{}{0pt}%
\pgfsys@defobject{currentmarker}{\pgfqpoint{0.000000in}{-0.048611in}}{\pgfqpoint{0.000000in}{0.000000in}}{%
\pgfpathmoveto{\pgfqpoint{0.000000in}{0.000000in}}%
\pgfpathlineto{\pgfqpoint{0.000000in}{-0.048611in}}%
\pgfusepath{stroke,fill}%
}%
\begin{pgfscope}%
\pgfsys@transformshift{2.652611in}{0.500000in}%
\pgfsys@useobject{currentmarker}{}%
\end{pgfscope}%
\end{pgfscope}%
\begin{pgfscope}%
\definecolor{textcolor}{rgb}{0.000000,0.000000,0.000000}%
\pgfsetstrokecolor{textcolor}%
\pgfsetfillcolor{textcolor}%
\pgftext[x=2.652611in,y=0.402778in,,top]{\color{textcolor}\rmfamily\fontsize{10.000000}{12.000000}\selectfont \(\displaystyle {282000}\)}%
\end{pgfscope}%
\begin{pgfscope}%
\pgfsetbuttcap%
\pgfsetroundjoin%
\definecolor{currentfill}{rgb}{0.000000,0.000000,0.000000}%
\pgfsetfillcolor{currentfill}%
\pgfsetlinewidth{0.803000pt}%
\definecolor{currentstroke}{rgb}{0.000000,0.000000,0.000000}%
\pgfsetstrokecolor{currentstroke}%
\pgfsetdash{}{0pt}%
\pgfsys@defobject{currentmarker}{\pgfqpoint{0.000000in}{-0.048611in}}{\pgfqpoint{0.000000in}{0.000000in}}{%
\pgfpathmoveto{\pgfqpoint{0.000000in}{0.000000in}}%
\pgfpathlineto{\pgfqpoint{0.000000in}{-0.048611in}}%
\pgfusepath{stroke,fill}%
}%
\begin{pgfscope}%
\pgfsys@transformshift{3.498235in}{0.500000in}%
\pgfsys@useobject{currentmarker}{}%
\end{pgfscope}%
\end{pgfscope}%
\begin{pgfscope}%
\definecolor{textcolor}{rgb}{0.000000,0.000000,0.000000}%
\pgfsetstrokecolor{textcolor}%
\pgfsetfillcolor{textcolor}%
\pgftext[x=3.498235in,y=0.402778in,,top]{\color{textcolor}\rmfamily\fontsize{10.000000}{12.000000}\selectfont \(\displaystyle {283000}\)}%
\end{pgfscope}%
\begin{pgfscope}%
\pgfsetbuttcap%
\pgfsetroundjoin%
\definecolor{currentfill}{rgb}{0.000000,0.000000,0.000000}%
\pgfsetfillcolor{currentfill}%
\pgfsetlinewidth{0.803000pt}%
\definecolor{currentstroke}{rgb}{0.000000,0.000000,0.000000}%
\pgfsetstrokecolor{currentstroke}%
\pgfsetdash{}{0pt}%
\pgfsys@defobject{currentmarker}{\pgfqpoint{0.000000in}{-0.048611in}}{\pgfqpoint{0.000000in}{0.000000in}}{%
\pgfpathmoveto{\pgfqpoint{0.000000in}{0.000000in}}%
\pgfpathlineto{\pgfqpoint{0.000000in}{-0.048611in}}%
\pgfusepath{stroke,fill}%
}%
\begin{pgfscope}%
\pgfsys@transformshift{4.343858in}{0.500000in}%
\pgfsys@useobject{currentmarker}{}%
\end{pgfscope}%
\end{pgfscope}%
\begin{pgfscope}%
\definecolor{textcolor}{rgb}{0.000000,0.000000,0.000000}%
\pgfsetstrokecolor{textcolor}%
\pgfsetfillcolor{textcolor}%
\pgftext[x=4.343858in,y=0.402778in,,top]{\color{textcolor}\rmfamily\fontsize{10.000000}{12.000000}\selectfont \(\displaystyle {284000}\)}%
\end{pgfscope}%
\begin{pgfscope}%
\pgfsetbuttcap%
\pgfsetroundjoin%
\definecolor{currentfill}{rgb}{0.000000,0.000000,0.000000}%
\pgfsetfillcolor{currentfill}%
\pgfsetlinewidth{0.803000pt}%
\definecolor{currentstroke}{rgb}{0.000000,0.000000,0.000000}%
\pgfsetstrokecolor{currentstroke}%
\pgfsetdash{}{0pt}%
\pgfsys@defobject{currentmarker}{\pgfqpoint{0.000000in}{-0.048611in}}{\pgfqpoint{0.000000in}{0.000000in}}{%
\pgfpathmoveto{\pgfqpoint{0.000000in}{0.000000in}}%
\pgfpathlineto{\pgfqpoint{0.000000in}{-0.048611in}}%
\pgfusepath{stroke,fill}%
}%
\begin{pgfscope}%
\pgfsys@transformshift{5.189482in}{0.500000in}%
\pgfsys@useobject{currentmarker}{}%
\end{pgfscope}%
\end{pgfscope}%
\begin{pgfscope}%
\definecolor{textcolor}{rgb}{0.000000,0.000000,0.000000}%
\pgfsetstrokecolor{textcolor}%
\pgfsetfillcolor{textcolor}%
\pgftext[x=5.189482in,y=0.402778in,,top]{\color{textcolor}\rmfamily\fontsize{10.000000}{12.000000}\selectfont \(\displaystyle {285000}\)}%
\end{pgfscope}%
\begin{pgfscope}%
\definecolor{textcolor}{rgb}{0.000000,0.000000,0.000000}%
\pgfsetstrokecolor{textcolor}%
\pgfsetfillcolor{textcolor}%
\pgftext[x=3.075000in,y=0.223766in,,top]{\color{textcolor}\rmfamily\fontsize{10.000000}{12.000000}\selectfont Time (s)}%
\end{pgfscope}%
\begin{pgfscope}%
\pgfsetbuttcap%
\pgfsetroundjoin%
\definecolor{currentfill}{rgb}{0.000000,0.000000,0.000000}%
\pgfsetfillcolor{currentfill}%
\pgfsetlinewidth{0.803000pt}%
\definecolor{currentstroke}{rgb}{0.000000,0.000000,0.000000}%
\pgfsetstrokecolor{currentstroke}%
\pgfsetdash{}{0pt}%
\pgfsys@defobject{currentmarker}{\pgfqpoint{-0.048611in}{0.000000in}}{\pgfqpoint{-0.000000in}{0.000000in}}{%
\pgfpathmoveto{\pgfqpoint{-0.000000in}{0.000000in}}%
\pgfpathlineto{\pgfqpoint{-0.048611in}{0.000000in}}%
\pgfusepath{stroke,fill}%
}%
\begin{pgfscope}%
\pgfsys@transformshift{0.750000in}{0.708637in}%
\pgfsys@useobject{currentmarker}{}%
\end{pgfscope}%
\end{pgfscope}%
\begin{pgfscope}%
\definecolor{textcolor}{rgb}{0.000000,0.000000,0.000000}%
\pgfsetstrokecolor{textcolor}%
\pgfsetfillcolor{textcolor}%
\pgftext[x=0.266974in, y=0.660412in, left, base]{\color{textcolor}\rmfamily\fontsize{10.000000}{12.000000}\selectfont \(\displaystyle {\ensuremath{-}7000}\)}%
\end{pgfscope}%
\begin{pgfscope}%
\pgfsetbuttcap%
\pgfsetroundjoin%
\definecolor{currentfill}{rgb}{0.000000,0.000000,0.000000}%
\pgfsetfillcolor{currentfill}%
\pgfsetlinewidth{0.803000pt}%
\definecolor{currentstroke}{rgb}{0.000000,0.000000,0.000000}%
\pgfsetstrokecolor{currentstroke}%
\pgfsetdash{}{0pt}%
\pgfsys@defobject{currentmarker}{\pgfqpoint{-0.048611in}{0.000000in}}{\pgfqpoint{-0.000000in}{0.000000in}}{%
\pgfpathmoveto{\pgfqpoint{-0.000000in}{0.000000in}}%
\pgfpathlineto{\pgfqpoint{-0.048611in}{0.000000in}}%
\pgfusepath{stroke,fill}%
}%
\begin{pgfscope}%
\pgfsys@transformshift{0.750000in}{1.087210in}%
\pgfsys@useobject{currentmarker}{}%
\end{pgfscope}%
\end{pgfscope}%
\begin{pgfscope}%
\definecolor{textcolor}{rgb}{0.000000,0.000000,0.000000}%
\pgfsetstrokecolor{textcolor}%
\pgfsetfillcolor{textcolor}%
\pgftext[x=0.266974in, y=1.038984in, left, base]{\color{textcolor}\rmfamily\fontsize{10.000000}{12.000000}\selectfont \(\displaystyle {\ensuremath{-}6000}\)}%
\end{pgfscope}%
\begin{pgfscope}%
\pgfsetbuttcap%
\pgfsetroundjoin%
\definecolor{currentfill}{rgb}{0.000000,0.000000,0.000000}%
\pgfsetfillcolor{currentfill}%
\pgfsetlinewidth{0.803000pt}%
\definecolor{currentstroke}{rgb}{0.000000,0.000000,0.000000}%
\pgfsetstrokecolor{currentstroke}%
\pgfsetdash{}{0pt}%
\pgfsys@defobject{currentmarker}{\pgfqpoint{-0.048611in}{0.000000in}}{\pgfqpoint{-0.000000in}{0.000000in}}{%
\pgfpathmoveto{\pgfqpoint{-0.000000in}{0.000000in}}%
\pgfpathlineto{\pgfqpoint{-0.048611in}{0.000000in}}%
\pgfusepath{stroke,fill}%
}%
\begin{pgfscope}%
\pgfsys@transformshift{0.750000in}{1.465782in}%
\pgfsys@useobject{currentmarker}{}%
\end{pgfscope}%
\end{pgfscope}%
\begin{pgfscope}%
\definecolor{textcolor}{rgb}{0.000000,0.000000,0.000000}%
\pgfsetstrokecolor{textcolor}%
\pgfsetfillcolor{textcolor}%
\pgftext[x=0.266974in, y=1.417557in, left, base]{\color{textcolor}\rmfamily\fontsize{10.000000}{12.000000}\selectfont \(\displaystyle {\ensuremath{-}5000}\)}%
\end{pgfscope}%
\begin{pgfscope}%
\pgfsetbuttcap%
\pgfsetroundjoin%
\definecolor{currentfill}{rgb}{0.000000,0.000000,0.000000}%
\pgfsetfillcolor{currentfill}%
\pgfsetlinewidth{0.803000pt}%
\definecolor{currentstroke}{rgb}{0.000000,0.000000,0.000000}%
\pgfsetstrokecolor{currentstroke}%
\pgfsetdash{}{0pt}%
\pgfsys@defobject{currentmarker}{\pgfqpoint{-0.048611in}{0.000000in}}{\pgfqpoint{-0.000000in}{0.000000in}}{%
\pgfpathmoveto{\pgfqpoint{-0.000000in}{0.000000in}}%
\pgfpathlineto{\pgfqpoint{-0.048611in}{0.000000in}}%
\pgfusepath{stroke,fill}%
}%
\begin{pgfscope}%
\pgfsys@transformshift{0.750000in}{1.844355in}%
\pgfsys@useobject{currentmarker}{}%
\end{pgfscope}%
\end{pgfscope}%
\begin{pgfscope}%
\definecolor{textcolor}{rgb}{0.000000,0.000000,0.000000}%
\pgfsetstrokecolor{textcolor}%
\pgfsetfillcolor{textcolor}%
\pgftext[x=0.266974in, y=1.796130in, left, base]{\color{textcolor}\rmfamily\fontsize{10.000000}{12.000000}\selectfont \(\displaystyle {\ensuremath{-}4000}\)}%
\end{pgfscope}%
\begin{pgfscope}%
\pgfsetbuttcap%
\pgfsetroundjoin%
\definecolor{currentfill}{rgb}{0.000000,0.000000,0.000000}%
\pgfsetfillcolor{currentfill}%
\pgfsetlinewidth{0.803000pt}%
\definecolor{currentstroke}{rgb}{0.000000,0.000000,0.000000}%
\pgfsetstrokecolor{currentstroke}%
\pgfsetdash{}{0pt}%
\pgfsys@defobject{currentmarker}{\pgfqpoint{-0.048611in}{0.000000in}}{\pgfqpoint{-0.000000in}{0.000000in}}{%
\pgfpathmoveto{\pgfqpoint{-0.000000in}{0.000000in}}%
\pgfpathlineto{\pgfqpoint{-0.048611in}{0.000000in}}%
\pgfusepath{stroke,fill}%
}%
\begin{pgfscope}%
\pgfsys@transformshift{0.750000in}{2.222928in}%
\pgfsys@useobject{currentmarker}{}%
\end{pgfscope}%
\end{pgfscope}%
\begin{pgfscope}%
\definecolor{textcolor}{rgb}{0.000000,0.000000,0.000000}%
\pgfsetstrokecolor{textcolor}%
\pgfsetfillcolor{textcolor}%
\pgftext[x=0.266974in, y=2.174703in, left, base]{\color{textcolor}\rmfamily\fontsize{10.000000}{12.000000}\selectfont \(\displaystyle {\ensuremath{-}3000}\)}%
\end{pgfscope}%
\begin{pgfscope}%
\pgfsetbuttcap%
\pgfsetroundjoin%
\definecolor{currentfill}{rgb}{0.000000,0.000000,0.000000}%
\pgfsetfillcolor{currentfill}%
\pgfsetlinewidth{0.803000pt}%
\definecolor{currentstroke}{rgb}{0.000000,0.000000,0.000000}%
\pgfsetstrokecolor{currentstroke}%
\pgfsetdash{}{0pt}%
\pgfsys@defobject{currentmarker}{\pgfqpoint{-0.048611in}{0.000000in}}{\pgfqpoint{-0.000000in}{0.000000in}}{%
\pgfpathmoveto{\pgfqpoint{-0.000000in}{0.000000in}}%
\pgfpathlineto{\pgfqpoint{-0.048611in}{0.000000in}}%
\pgfusepath{stroke,fill}%
}%
\begin{pgfscope}%
\pgfsys@transformshift{0.750000in}{2.601501in}%
\pgfsys@useobject{currentmarker}{}%
\end{pgfscope}%
\end{pgfscope}%
\begin{pgfscope}%
\definecolor{textcolor}{rgb}{0.000000,0.000000,0.000000}%
\pgfsetstrokecolor{textcolor}%
\pgfsetfillcolor{textcolor}%
\pgftext[x=0.266974in, y=2.553275in, left, base]{\color{textcolor}\rmfamily\fontsize{10.000000}{12.000000}\selectfont \(\displaystyle {\ensuremath{-}2000}\)}%
\end{pgfscope}%
\begin{pgfscope}%
\pgfsetbuttcap%
\pgfsetroundjoin%
\definecolor{currentfill}{rgb}{0.000000,0.000000,0.000000}%
\pgfsetfillcolor{currentfill}%
\pgfsetlinewidth{0.803000pt}%
\definecolor{currentstroke}{rgb}{0.000000,0.000000,0.000000}%
\pgfsetstrokecolor{currentstroke}%
\pgfsetdash{}{0pt}%
\pgfsys@defobject{currentmarker}{\pgfqpoint{-0.048611in}{0.000000in}}{\pgfqpoint{-0.000000in}{0.000000in}}{%
\pgfpathmoveto{\pgfqpoint{-0.000000in}{0.000000in}}%
\pgfpathlineto{\pgfqpoint{-0.048611in}{0.000000in}}%
\pgfusepath{stroke,fill}%
}%
\begin{pgfscope}%
\pgfsys@transformshift{0.750000in}{2.980074in}%
\pgfsys@useobject{currentmarker}{}%
\end{pgfscope}%
\end{pgfscope}%
\begin{pgfscope}%
\definecolor{textcolor}{rgb}{0.000000,0.000000,0.000000}%
\pgfsetstrokecolor{textcolor}%
\pgfsetfillcolor{textcolor}%
\pgftext[x=0.266974in, y=2.931848in, left, base]{\color{textcolor}\rmfamily\fontsize{10.000000}{12.000000}\selectfont \(\displaystyle {\ensuremath{-}1000}\)}%
\end{pgfscope}%
\begin{pgfscope}%
\pgfsetbuttcap%
\pgfsetroundjoin%
\definecolor{currentfill}{rgb}{0.000000,0.000000,0.000000}%
\pgfsetfillcolor{currentfill}%
\pgfsetlinewidth{0.803000pt}%
\definecolor{currentstroke}{rgb}{0.000000,0.000000,0.000000}%
\pgfsetstrokecolor{currentstroke}%
\pgfsetdash{}{0pt}%
\pgfsys@defobject{currentmarker}{\pgfqpoint{-0.048611in}{0.000000in}}{\pgfqpoint{-0.000000in}{0.000000in}}{%
\pgfpathmoveto{\pgfqpoint{-0.000000in}{0.000000in}}%
\pgfpathlineto{\pgfqpoint{-0.048611in}{0.000000in}}%
\pgfusepath{stroke,fill}%
}%
\begin{pgfscope}%
\pgfsys@transformshift{0.750000in}{3.358646in}%
\pgfsys@useobject{currentmarker}{}%
\end{pgfscope}%
\end{pgfscope}%
\begin{pgfscope}%
\definecolor{textcolor}{rgb}{0.000000,0.000000,0.000000}%
\pgfsetstrokecolor{textcolor}%
\pgfsetfillcolor{textcolor}%
\pgftext[x=0.583333in, y=3.310421in, left, base]{\color{textcolor}\rmfamily\fontsize{10.000000}{12.000000}\selectfont \(\displaystyle {0}\)}%
\end{pgfscope}%
\begin{pgfscope}%
\definecolor{textcolor}{rgb}{0.000000,0.000000,0.000000}%
\pgfsetstrokecolor{textcolor}%
\pgfsetfillcolor{textcolor}%
\pgftext[x=0.211419in,y=2.010000in,,bottom,rotate=90.000000]{\color{textcolor}\rmfamily\fontsize{10.000000}{12.000000}\selectfont Frequency (Hz)}%
\end{pgfscope}%
\begin{pgfscope}%
\pgfpathrectangle{\pgfqpoint{0.750000in}{0.500000in}}{\pgfqpoint{4.650000in}{3.020000in}}%
\pgfusepath{clip}%
\pgfsetrectcap%
\pgfsetroundjoin%
\pgfsetlinewidth{1.505625pt}%
\definecolor{currentstroke}{rgb}{0.121569,0.466667,0.705882}%
\pgfsetstrokecolor{currentstroke}%
\pgfsetdash{}{0pt}%
\pgfpathmoveto{\pgfqpoint{0.961364in}{3.377575in}}%
\pgfpathlineto{\pgfqpoint{1.294539in}{3.377575in}}%
\pgfpathlineto{\pgfqpoint{1.296231in}{0.645884in}}%
\pgfpathlineto{\pgfqpoint{1.313989in}{0.646536in}}%
\pgfpathlineto{\pgfqpoint{1.314834in}{0.644231in}}%
\pgfpathlineto{\pgfqpoint{1.315680in}{0.647527in}}%
\pgfpathlineto{\pgfqpoint{1.316526in}{0.645675in}}%
\pgfpathlineto{\pgfqpoint{1.317371in}{0.644534in}}%
\pgfpathlineto{\pgfqpoint{1.318217in}{0.648054in}}%
\pgfpathlineto{\pgfqpoint{1.319062in}{0.644641in}}%
\pgfpathlineto{\pgfqpoint{1.319908in}{0.645346in}}%
\pgfpathlineto{\pgfqpoint{1.320754in}{0.648087in}}%
\pgfpathlineto{\pgfqpoint{1.321599in}{0.643727in}}%
\pgfpathlineto{\pgfqpoint{1.322445in}{0.646522in}}%
\pgfpathlineto{\pgfqpoint{1.323291in}{0.647507in}}%
\pgfpathlineto{\pgfqpoint{1.324136in}{0.643240in}}%
\pgfpathlineto{\pgfqpoint{1.324982in}{0.647779in}}%
\pgfpathlineto{\pgfqpoint{1.325827in}{0.646369in}}%
\pgfpathlineto{\pgfqpoint{1.326673in}{0.643409in}}%
\pgfpathlineto{\pgfqpoint{1.327519in}{0.648759in}}%
\pgfpathlineto{\pgfqpoint{1.328364in}{0.644905in}}%
\pgfpathlineto{\pgfqpoint{1.329210in}{0.644299in}}%
\pgfpathlineto{\pgfqpoint{1.330056in}{0.649133in}}%
\pgfpathlineto{\pgfqpoint{1.330901in}{0.643479in}}%
\pgfpathlineto{\pgfqpoint{1.331747in}{0.645771in}}%
\pgfpathlineto{\pgfqpoint{1.332592in}{0.648707in}}%
\pgfpathlineto{\pgfqpoint{1.333438in}{0.642496in}}%
\pgfpathlineto{\pgfqpoint{1.334284in}{0.647496in}}%
\pgfpathlineto{\pgfqpoint{1.335129in}{0.647487in}}%
\pgfpathlineto{\pgfqpoint{1.335975in}{0.642289in}}%
\pgfpathlineto{\pgfqpoint{1.336821in}{0.649033in}}%
\pgfpathlineto{\pgfqpoint{1.337666in}{0.645704in}}%
\pgfpathlineto{\pgfqpoint{1.338512in}{0.643012in}}%
\pgfpathlineto{\pgfqpoint{1.339357in}{0.649943in}}%
\pgfpathlineto{\pgfqpoint{1.340203in}{0.643770in}}%
\pgfpathlineto{\pgfqpoint{1.341049in}{0.644580in}}%
\pgfpathlineto{\pgfqpoint{1.341894in}{0.649915in}}%
\pgfpathlineto{\pgfqpoint{1.342740in}{0.642183in}}%
\pgfpathlineto{\pgfqpoint{1.343586in}{0.646664in}}%
\pgfpathlineto{\pgfqpoint{1.344431in}{0.648860in}}%
\pgfpathlineto{\pgfqpoint{1.345277in}{0.641399in}}%
\pgfpathlineto{\pgfqpoint{1.346122in}{0.648764in}}%
\pgfpathlineto{\pgfqpoint{1.346968in}{0.646955in}}%
\pgfpathlineto{\pgfqpoint{1.347814in}{0.641703in}}%
\pgfpathlineto{\pgfqpoint{1.348659in}{0.650330in}}%
\pgfpathlineto{\pgfqpoint{1.349505in}{0.644618in}}%
\pgfpathlineto{\pgfqpoint{1.350351in}{0.643115in}}%
\pgfpathlineto{\pgfqpoint{1.351196in}{0.650908in}}%
\pgfpathlineto{\pgfqpoint{1.352042in}{0.642418in}}%
\pgfpathlineto{\pgfqpoint{1.352887in}{0.645362in}}%
\pgfpathlineto{\pgfqpoint{1.353733in}{0.650274in}}%
\pgfpathlineto{\pgfqpoint{1.354579in}{0.640931in}}%
\pgfpathlineto{\pgfqpoint{1.355424in}{0.647925in}}%
\pgfpathlineto{\pgfqpoint{1.356270in}{0.648498in}}%
\pgfpathlineto{\pgfqpoint{1.357116in}{0.640597in}}%
\pgfpathlineto{\pgfqpoint{1.357961in}{0.650167in}}%
\pgfpathlineto{\pgfqpoint{1.358807in}{0.645954in}}%
\pgfpathlineto{\pgfqpoint{1.359652in}{0.641586in}}%
\pgfpathlineto{\pgfqpoint{1.360498in}{0.651492in}}%
\pgfpathlineto{\pgfqpoint{1.361344in}{0.643234in}}%
\pgfpathlineto{\pgfqpoint{1.362189in}{0.643737in}}%
\pgfpathlineto{\pgfqpoint{1.363035in}{0.651507in}}%
\pgfpathlineto{\pgfqpoint{1.363881in}{0.641016in}}%
\pgfpathlineto{\pgfqpoint{1.364726in}{0.646572in}}%
\pgfpathlineto{\pgfqpoint{1.365572in}{0.650131in}}%
\pgfpathlineto{\pgfqpoint{1.366417in}{0.639894in}}%
\pgfpathlineto{\pgfqpoint{1.367263in}{0.649408in}}%
\pgfpathlineto{\pgfqpoint{1.368109in}{0.647635in}}%
\pgfpathlineto{\pgfqpoint{1.368954in}{0.640219in}}%
\pgfpathlineto{\pgfqpoint{1.369800in}{0.651526in}}%
\pgfpathlineto{\pgfqpoint{1.370645in}{0.644582in}}%
\pgfpathlineto{\pgfqpoint{1.371491in}{0.641992in}}%
\pgfpathlineto{\pgfqpoint{1.372337in}{0.652357in}}%
\pgfpathlineto{\pgfqpoint{1.373182in}{0.641707in}}%
\pgfpathlineto{\pgfqpoint{1.374028in}{0.644839in}}%
\pgfpathlineto{\pgfqpoint{1.374874in}{0.651632in}}%
\pgfpathlineto{\pgfqpoint{1.375719in}{0.639741in}}%
\pgfpathlineto{\pgfqpoint{1.376565in}{0.648090in}}%
\pgfpathlineto{\pgfqpoint{1.377410in}{0.649467in}}%
\pgfpathlineto{\pgfqpoint{1.378256in}{0.639223in}}%
\pgfpathlineto{\pgfqpoint{1.379102in}{0.650944in}}%
\pgfpathlineto{\pgfqpoint{1.379947in}{0.646336in}}%
\pgfpathlineto{\pgfqpoint{1.380793in}{0.640351in}}%
\pgfpathlineto{\pgfqpoint{1.381639in}{0.652668in}}%
\pgfpathlineto{\pgfqpoint{1.382484in}{0.642976in}}%
\pgfpathlineto{\pgfqpoint{1.383330in}{0.642915in}}%
\pgfpathlineto{\pgfqpoint{1.384175in}{0.652792in}}%
\pgfpathlineto{\pgfqpoint{1.385021in}{0.640213in}}%
\pgfpathlineto{\pgfqpoint{1.385867in}{0.646327in}}%
\pgfpathlineto{\pgfqpoint{1.386712in}{0.651231in}}%
\pgfpathlineto{\pgfqpoint{1.387558in}{0.638759in}}%
\pgfpathlineto{\pgfqpoint{1.388404in}{0.649759in}}%
\pgfpathlineto{\pgfqpoint{1.389249in}{0.648314in}}%
\pgfpathlineto{\pgfqpoint{1.390095in}{0.639029in}}%
\pgfpathlineto{\pgfqpoint{1.390940in}{0.652354in}}%
\pgfpathlineto{\pgfqpoint{1.391786in}{0.644717in}}%
\pgfpathlineto{\pgfqpoint{1.392632in}{0.641021in}}%
\pgfpathlineto{\pgfqpoint{1.393477in}{0.653441in}}%
\pgfpathlineto{\pgfqpoint{1.394323in}{0.641304in}}%
\pgfpathlineto{\pgfqpoint{1.395169in}{0.644292in}}%
\pgfpathlineto{\pgfqpoint{1.396014in}{0.652713in}}%
\pgfpathlineto{\pgfqpoint{1.396860in}{0.638925in}}%
\pgfpathlineto{\pgfqpoint{1.397705in}{0.648062in}}%
\pgfpathlineto{\pgfqpoint{1.398551in}{0.650304in}}%
\pgfpathlineto{\pgfqpoint{1.399397in}{0.638206in}}%
\pgfpathlineto{\pgfqpoint{1.400242in}{0.651401in}}%
\pgfpathlineto{\pgfqpoint{1.401088in}{0.646763in}}%
\pgfpathlineto{\pgfqpoint{1.401934in}{0.639377in}}%
\pgfpathlineto{\pgfqpoint{1.402779in}{0.653470in}}%
\pgfpathlineto{\pgfqpoint{1.403625in}{0.642930in}}%
\pgfpathlineto{\pgfqpoint{1.404470in}{0.642200in}}%
\pgfpathlineto{\pgfqpoint{1.405316in}{0.653731in}}%
\pgfpathlineto{\pgfqpoint{1.406162in}{0.639741in}}%
\pgfpathlineto{\pgfqpoint{1.407007in}{0.646010in}}%
\pgfpathlineto{\pgfqpoint{1.407853in}{0.652088in}}%
\pgfpathlineto{\pgfqpoint{1.408699in}{0.638000in}}%
\pgfpathlineto{\pgfqpoint{1.409544in}{0.649877in}}%
\pgfpathlineto{\pgfqpoint{1.410390in}{0.648909in}}%
\pgfpathlineto{\pgfqpoint{1.411235in}{0.638177in}}%
\pgfpathlineto{\pgfqpoint{1.412081in}{0.652840in}}%
\pgfpathlineto{\pgfqpoint{1.412927in}{0.644944in}}%
\pgfpathlineto{\pgfqpoint{1.413772in}{0.640272in}}%
\pgfpathlineto{\pgfqpoint{1.414618in}{0.654152in}}%
\pgfpathlineto{\pgfqpoint{1.415464in}{0.641149in}}%
\pgfpathlineto{\pgfqpoint{1.416309in}{0.643804in}}%
\pgfpathlineto{\pgfqpoint{1.417155in}{0.653472in}}%
\pgfpathlineto{\pgfqpoint{1.418000in}{0.638458in}}%
\pgfpathlineto{\pgfqpoint{1.418846in}{0.647915in}}%
\pgfpathlineto{\pgfqpoint{1.419692in}{0.650941in}}%
\pgfpathlineto{\pgfqpoint{1.420537in}{0.637560in}}%
\pgfpathlineto{\pgfqpoint{1.421383in}{0.651590in}}%
\pgfpathlineto{\pgfqpoint{1.422229in}{0.647156in}}%
\pgfpathlineto{\pgfqpoint{1.423074in}{0.638712in}}%
\pgfpathlineto{\pgfqpoint{1.423920in}{0.653915in}}%
\pgfpathlineto{\pgfqpoint{1.424765in}{0.643023in}}%
\pgfpathlineto{\pgfqpoint{1.425611in}{0.641662in}}%
\pgfpathlineto{\pgfqpoint{1.426457in}{0.654306in}}%
\pgfpathlineto{\pgfqpoint{1.427302in}{0.639550in}}%
\pgfpathlineto{\pgfqpoint{1.428148in}{0.645699in}}%
\pgfpathlineto{\pgfqpoint{1.428994in}{0.652656in}}%
\pgfpathlineto{\pgfqpoint{1.429839in}{0.637600in}}%
\pgfpathlineto{\pgfqpoint{1.430685in}{0.649830in}}%
\pgfpathlineto{\pgfqpoint{1.431530in}{0.649353in}}%
\pgfpathlineto{\pgfqpoint{1.432376in}{0.637679in}}%
\pgfpathlineto{\pgfqpoint{1.433222in}{0.653029in}}%
\pgfpathlineto{\pgfqpoint{1.434067in}{0.645191in}}%
\pgfpathlineto{\pgfqpoint{1.434913in}{0.639794in}}%
\pgfpathlineto{\pgfqpoint{1.435759in}{0.654503in}}%
\pgfpathlineto{\pgfqpoint{1.437450in}{0.641177in}}%
\pgfpathlineto{\pgfqpoint{1.438295in}{0.641177in}}%
\pgfpathlineto{\pgfqpoint{1.439987in}{0.643440in}}%
\pgfpathlineto{\pgfqpoint{1.442524in}{0.643440in}}%
\pgfpathlineto{\pgfqpoint{1.443369in}{0.653888in}}%
\pgfpathlineto{\pgfqpoint{1.445060in}{0.638297in}}%
\pgfpathlineto{\pgfqpoint{1.445906in}{0.638297in}}%
\pgfpathlineto{\pgfqpoint{1.447597in}{0.647720in}}%
\pgfpathlineto{\pgfqpoint{1.450980in}{0.647720in}}%
\pgfpathlineto{\pgfqpoint{1.452671in}{0.651331in}}%
\pgfpathlineto{\pgfqpoint{1.454362in}{0.637273in}}%
\pgfpathlineto{\pgfqpoint{1.456053in}{0.637273in}}%
\pgfpathlineto{\pgfqpoint{1.457745in}{0.651572in}}%
\pgfpathlineto{\pgfqpoint{1.461127in}{0.651572in}}%
\pgfpathlineto{\pgfqpoint{1.462818in}{0.638376in}}%
\pgfpathlineto{\pgfqpoint{1.463664in}{0.654043in}}%
\pgfpathlineto{\pgfqpoint{1.464510in}{0.643187in}}%
\pgfpathlineto{\pgfqpoint{1.466201in}{0.643187in}}%
\pgfpathlineto{\pgfqpoint{1.468738in}{0.641348in}}%
\pgfpathlineto{\pgfqpoint{1.470429in}{0.641348in}}%
\pgfpathlineto{\pgfqpoint{1.472120in}{0.654527in}}%
\pgfpathlineto{\pgfqpoint{1.473812in}{0.639581in}}%
\pgfpathlineto{\pgfqpoint{1.474657in}{0.639581in}}%
\pgfpathlineto{\pgfqpoint{1.476348in}{0.645458in}}%
\pgfpathlineto{\pgfqpoint{1.478885in}{0.645458in}}%
\pgfpathlineto{\pgfqpoint{1.480577in}{0.652912in}}%
\pgfpathlineto{\pgfqpoint{1.482268in}{0.637523in}}%
\pgfpathlineto{\pgfqpoint{1.483113in}{0.637523in}}%
\pgfpathlineto{\pgfqpoint{1.484805in}{0.649684in}}%
\pgfpathlineto{\pgfqpoint{1.488187in}{0.649597in}}%
\pgfpathlineto{\pgfqpoint{1.489878in}{0.637530in}}%
\pgfpathlineto{\pgfqpoint{1.490724in}{0.637530in}}%
\pgfpathlineto{\pgfqpoint{1.492415in}{0.652976in}}%
\pgfpathlineto{\pgfqpoint{1.494107in}{0.652976in}}%
\pgfpathlineto{\pgfqpoint{1.495798in}{0.645392in}}%
\pgfpathlineto{\pgfqpoint{1.497489in}{0.645392in}}%
\pgfpathlineto{\pgfqpoint{1.499180in}{0.639610in}}%
\pgfpathlineto{\pgfqpoint{1.500872in}{0.639610in}}%
\pgfpathlineto{\pgfqpoint{1.502563in}{0.654526in}}%
\pgfpathlineto{\pgfqpoint{1.504254in}{0.641324in}}%
\pgfpathlineto{\pgfqpoint{1.505100in}{0.641324in}}%
\pgfpathlineto{\pgfqpoint{1.506791in}{0.643250in}}%
\pgfpathlineto{\pgfqpoint{1.508482in}{0.643250in}}%
\pgfpathlineto{\pgfqpoint{1.510173in}{0.653966in}}%
\pgfpathlineto{\pgfqpoint{1.511865in}{0.638389in}}%
\pgfpathlineto{\pgfqpoint{1.512710in}{0.638389in}}%
\pgfpathlineto{\pgfqpoint{1.514402in}{0.647540in}}%
\pgfpathlineto{\pgfqpoint{1.516093in}{0.647540in}}%
\pgfpathlineto{\pgfqpoint{1.517784in}{0.651447in}}%
\pgfpathlineto{\pgfqpoint{1.519475in}{0.637314in}}%
\pgfpathlineto{\pgfqpoint{1.520321in}{0.637314in}}%
\pgfpathlineto{\pgfqpoint{1.522012in}{0.651410in}}%
\pgfpathlineto{\pgfqpoint{1.522858in}{0.651410in}}%
\pgfpathlineto{\pgfqpoint{1.524549in}{0.647596in}}%
\pgfpathlineto{\pgfqpoint{1.525395in}{0.647596in}}%
\pgfpathlineto{\pgfqpoint{1.527086in}{0.638368in}}%
\pgfpathlineto{\pgfqpoint{1.529623in}{0.638368in}}%
\pgfpathlineto{\pgfqpoint{1.531314in}{0.653904in}}%
\pgfpathlineto{\pgfqpoint{1.533851in}{0.653904in}}%
\pgfpathlineto{\pgfqpoint{1.535542in}{0.643358in}}%
\pgfpathlineto{\pgfqpoint{1.536388in}{0.643358in}}%
\pgfpathlineto{\pgfqpoint{1.538079in}{0.641287in}}%
\pgfpathlineto{\pgfqpoint{1.539770in}{0.641287in}}%
\pgfpathlineto{\pgfqpoint{1.541461in}{0.654421in}}%
\pgfpathlineto{\pgfqpoint{1.543153in}{0.639772in}}%
\pgfpathlineto{\pgfqpoint{1.543998in}{0.639772in}}%
\pgfpathlineto{\pgfqpoint{1.545690in}{0.645338in}}%
\pgfpathlineto{\pgfqpoint{1.546535in}{0.645338in}}%
\pgfpathlineto{\pgfqpoint{1.548226in}{0.652855in}}%
\pgfpathlineto{\pgfqpoint{1.549918in}{0.637719in}}%
\pgfpathlineto{\pgfqpoint{1.550763in}{0.637719in}}%
\pgfpathlineto{\pgfqpoint{1.552455in}{0.649501in}}%
\pgfpathlineto{\pgfqpoint{1.556683in}{0.649613in}}%
\pgfpathlineto{\pgfqpoint{1.558374in}{0.637704in}}%
\pgfpathlineto{\pgfqpoint{1.559220in}{0.637704in}}%
\pgfpathlineto{\pgfqpoint{1.560911in}{0.652741in}}%
\pgfpathlineto{\pgfqpoint{1.562602in}{0.652741in}}%
\pgfpathlineto{\pgfqpoint{1.564293in}{0.645499in}}%
\pgfpathlineto{\pgfqpoint{1.565139in}{0.645499in}}%
\pgfpathlineto{\pgfqpoint{1.566830in}{0.639725in}}%
\pgfpathlineto{\pgfqpoint{1.568521in}{0.639725in}}%
\pgfpathlineto{\pgfqpoint{1.570213in}{0.654267in}}%
\pgfpathlineto{\pgfqpoint{1.571058in}{0.654267in}}%
\pgfpathlineto{\pgfqpoint{1.572750in}{0.641528in}}%
\pgfpathlineto{\pgfqpoint{1.573595in}{0.641528in}}%
\pgfpathlineto{\pgfqpoint{1.576132in}{0.643265in}}%
\pgfpathlineto{\pgfqpoint{1.576978in}{0.643265in}}%
\pgfpathlineto{\pgfqpoint{1.578669in}{0.653728in}}%
\pgfpathlineto{\pgfqpoint{1.580360in}{0.638675in}}%
\pgfpathlineto{\pgfqpoint{1.581206in}{0.638675in}}%
\pgfpathlineto{\pgfqpoint{1.582897in}{0.647427in}}%
\pgfpathlineto{\pgfqpoint{1.583743in}{0.647427in}}%
\pgfpathlineto{\pgfqpoint{1.585434in}{0.651286in}}%
\pgfpathlineto{\pgfqpoint{1.587125in}{0.651286in}}%
\pgfpathlineto{\pgfqpoint{1.588816in}{0.637639in}}%
\pgfpathlineto{\pgfqpoint{1.589662in}{0.637639in}}%
\pgfpathlineto{\pgfqpoint{1.591353in}{0.651164in}}%
\pgfpathlineto{\pgfqpoint{1.592199in}{0.651164in}}%
\pgfpathlineto{\pgfqpoint{1.593890in}{0.647564in}}%
\pgfpathlineto{\pgfqpoint{1.594736in}{0.647564in}}%
\pgfpathlineto{\pgfqpoint{1.596427in}{0.638667in}}%
\pgfpathlineto{\pgfqpoint{1.598118in}{0.653554in}}%
\pgfpathlineto{\pgfqpoint{1.598964in}{0.653554in}}%
\pgfpathlineto{\pgfqpoint{1.600655in}{0.643488in}}%
\pgfpathlineto{\pgfqpoint{1.601501in}{0.643488in}}%
\pgfpathlineto{\pgfqpoint{1.603192in}{0.641485in}}%
\pgfpathlineto{\pgfqpoint{1.604883in}{0.641485in}}%
\pgfpathlineto{\pgfqpoint{1.606574in}{0.654029in}}%
\pgfpathlineto{\pgfqpoint{1.607420in}{0.654029in}}%
\pgfpathlineto{\pgfqpoint{1.609111in}{0.640065in}}%
\pgfpathlineto{\pgfqpoint{1.609957in}{0.640065in}}%
\pgfpathlineto{\pgfqpoint{1.611648in}{0.645370in}}%
\pgfpathlineto{\pgfqpoint{1.612494in}{0.645370in}}%
\pgfpathlineto{\pgfqpoint{1.614185in}{0.652505in}}%
\pgfpathlineto{\pgfqpoint{1.615876in}{0.652505in}}%
\pgfpathlineto{\pgfqpoint{1.617568in}{0.638134in}}%
\pgfpathlineto{\pgfqpoint{1.618413in}{0.638134in}}%
\pgfpathlineto{\pgfqpoint{1.620104in}{0.649332in}}%
\pgfpathlineto{\pgfqpoint{1.624333in}{0.649392in}}%
\pgfpathlineto{\pgfqpoint{1.626024in}{0.638161in}}%
\pgfpathlineto{\pgfqpoint{1.627715in}{0.652383in}}%
\pgfpathlineto{\pgfqpoint{1.628561in}{0.652383in}}%
\pgfpathlineto{\pgfqpoint{1.630252in}{0.645478in}}%
\pgfpathlineto{\pgfqpoint{1.631098in}{0.645478in}}%
\pgfpathlineto{\pgfqpoint{1.632789in}{0.640120in}}%
\pgfpathlineto{\pgfqpoint{1.634480in}{0.640120in}}%
\pgfpathlineto{\pgfqpoint{1.636171in}{0.653780in}}%
\pgfpathlineto{\pgfqpoint{1.637863in}{0.653780in}}%
\pgfpathlineto{\pgfqpoint{1.639554in}{0.641740in}}%
\pgfpathlineto{\pgfqpoint{1.640399in}{0.641740in}}%
\pgfpathlineto{\pgfqpoint{1.642936in}{0.643494in}}%
\pgfpathlineto{\pgfqpoint{1.644628in}{0.653213in}}%
\pgfpathlineto{\pgfqpoint{1.646319in}{0.653213in}}%
\pgfpathlineto{\pgfqpoint{1.648010in}{0.639098in}}%
\pgfpathlineto{\pgfqpoint{1.648856in}{0.639098in}}%
\pgfpathlineto{\pgfqpoint{1.650547in}{0.647414in}}%
\pgfpathlineto{\pgfqpoint{1.651393in}{0.647414in}}%
\pgfpathlineto{\pgfqpoint{1.653084in}{0.650861in}}%
\pgfpathlineto{\pgfqpoint{1.654775in}{0.650861in}}%
\pgfpathlineto{\pgfqpoint{1.656466in}{0.638196in}}%
\pgfpathlineto{\pgfqpoint{1.658158in}{0.650887in}}%
\pgfpathlineto{\pgfqpoint{1.659003in}{0.650887in}}%
\pgfpathlineto{\pgfqpoint{1.660694in}{0.647340in}}%
\pgfpathlineto{\pgfqpoint{1.661540in}{0.647340in}}%
\pgfpathlineto{\pgfqpoint{1.663231in}{0.639237in}}%
\pgfpathlineto{\pgfqpoint{1.664923in}{0.653053in}}%
\pgfpathlineto{\pgfqpoint{1.665768in}{0.653053in}}%
\pgfpathlineto{\pgfqpoint{1.667459in}{0.643540in}}%
\pgfpathlineto{\pgfqpoint{1.668305in}{0.643540in}}%
\pgfpathlineto{\pgfqpoint{1.670842in}{0.641931in}}%
\pgfpathlineto{\pgfqpoint{1.671688in}{0.641931in}}%
\pgfpathlineto{\pgfqpoint{1.673379in}{0.653404in}}%
\pgfpathlineto{\pgfqpoint{1.674224in}{0.653404in}}%
\pgfpathlineto{\pgfqpoint{1.675916in}{0.640406in}}%
\pgfpathlineto{\pgfqpoint{1.676761in}{0.640406in}}%
\pgfpathlineto{\pgfqpoint{1.678453in}{0.645569in}}%
\pgfpathlineto{\pgfqpoint{1.679298in}{0.645569in}}%
\pgfpathlineto{\pgfqpoint{1.680989in}{0.651895in}}%
\pgfpathlineto{\pgfqpoint{1.682681in}{0.651895in}}%
\pgfpathlineto{\pgfqpoint{1.684372in}{0.638709in}}%
\pgfpathlineto{\pgfqpoint{1.686063in}{0.649217in}}%
\pgfpathlineto{\pgfqpoint{1.690291in}{0.648944in}}%
\pgfpathlineto{\pgfqpoint{1.691982in}{0.638851in}}%
\pgfpathlineto{\pgfqpoint{1.693674in}{0.651957in}}%
\pgfpathlineto{\pgfqpoint{1.694519in}{0.651957in}}%
\pgfpathlineto{\pgfqpoint{1.696211in}{0.645313in}}%
\pgfpathlineto{\pgfqpoint{1.697902in}{0.645313in}}%
\pgfpathlineto{\pgfqpoint{1.699593in}{0.640764in}}%
\pgfpathlineto{\pgfqpoint{1.701284in}{0.640764in}}%
\pgfpathlineto{\pgfqpoint{1.702976in}{0.653124in}}%
\pgfpathlineto{\pgfqpoint{1.704667in}{0.653124in}}%
\pgfpathlineto{\pgfqpoint{1.706358in}{0.641918in}}%
\pgfpathlineto{\pgfqpoint{1.707204in}{0.641918in}}%
\pgfpathlineto{\pgfqpoint{1.708895in}{0.643931in}}%
\pgfpathlineto{\pgfqpoint{1.709741in}{0.643931in}}%
\pgfpathlineto{\pgfqpoint{1.711432in}{0.652469in}}%
\pgfpathlineto{\pgfqpoint{1.713123in}{0.652469in}}%
\pgfpathlineto{\pgfqpoint{1.714814in}{0.639602in}}%
\pgfpathlineto{\pgfqpoint{1.716506in}{0.647524in}}%
\pgfpathlineto{\pgfqpoint{1.717351in}{0.647524in}}%
\pgfpathlineto{\pgfqpoint{1.719042in}{0.650202in}}%
\pgfpathlineto{\pgfqpoint{1.720734in}{0.650202in}}%
\pgfpathlineto{\pgfqpoint{1.722425in}{0.638926in}}%
\pgfpathlineto{\pgfqpoint{1.724116in}{0.650621in}}%
\pgfpathlineto{\pgfqpoint{1.724962in}{0.650621in}}%
\pgfpathlineto{\pgfqpoint{1.726653in}{0.646930in}}%
\pgfpathlineto{\pgfqpoint{1.728344in}{0.646930in}}%
\pgfpathlineto{\pgfqpoint{1.730036in}{0.640030in}}%
\pgfpathlineto{\pgfqpoint{1.731727in}{0.652456in}}%
\pgfpathlineto{\pgfqpoint{1.732572in}{0.652456in}}%
\pgfpathlineto{\pgfqpoint{1.734264in}{0.643492in}}%
\pgfpathlineto{\pgfqpoint{1.737646in}{0.642595in}}%
\pgfpathlineto{\pgfqpoint{1.739337in}{0.652602in}}%
\pgfpathlineto{\pgfqpoint{1.741874in}{0.652602in}}%
\pgfpathlineto{\pgfqpoint{1.743566in}{0.640753in}}%
\pgfpathlineto{\pgfqpoint{1.744411in}{0.640753in}}%
\pgfpathlineto{\pgfqpoint{1.746102in}{0.645936in}}%
\pgfpathlineto{\pgfqpoint{1.746948in}{0.645936in}}%
\pgfpathlineto{\pgfqpoint{1.748639in}{0.651071in}}%
\pgfpathlineto{\pgfqpoint{1.750331in}{0.651071in}}%
\pgfpathlineto{\pgfqpoint{1.752022in}{0.639389in}}%
\pgfpathlineto{\pgfqpoint{1.753713in}{0.649181in}}%
\pgfpathlineto{\pgfqpoint{1.757941in}{0.648294in}}%
\pgfpathlineto{\pgfqpoint{1.759632in}{0.639716in}}%
\pgfpathlineto{\pgfqpoint{1.761324in}{0.651508in}}%
\pgfpathlineto{\pgfqpoint{1.762169in}{0.651508in}}%
\pgfpathlineto{\pgfqpoint{1.763860in}{0.645002in}}%
\pgfpathlineto{\pgfqpoint{1.765552in}{0.645002in}}%
\pgfpathlineto{\pgfqpoint{1.767243in}{0.641613in}}%
\pgfpathlineto{\pgfqpoint{1.768934in}{0.641613in}}%
\pgfpathlineto{\pgfqpoint{1.770625in}{0.652354in}}%
\pgfpathlineto{\pgfqpoint{1.772317in}{0.652354in}}%
\pgfpathlineto{\pgfqpoint{1.774008in}{0.642037in}}%
\pgfpathlineto{\pgfqpoint{1.775699in}{0.642037in}}%
\pgfpathlineto{\pgfqpoint{1.777390in}{0.644554in}}%
\pgfpathlineto{\pgfqpoint{1.778236in}{0.644554in}}%
\pgfpathlineto{\pgfqpoint{1.779927in}{0.651550in}}%
\pgfpathlineto{\pgfqpoint{1.781619in}{0.651550in}}%
\pgfpathlineto{\pgfqpoint{1.783310in}{0.640142in}}%
\pgfpathlineto{\pgfqpoint{1.785001in}{0.647758in}}%
\pgfpathlineto{\pgfqpoint{1.785847in}{0.647758in}}%
\pgfpathlineto{\pgfqpoint{1.788384in}{0.649351in}}%
\pgfpathlineto{\pgfqpoint{1.789229in}{0.649351in}}%
\pgfpathlineto{\pgfqpoint{1.790920in}{0.639774in}}%
\pgfpathlineto{\pgfqpoint{1.792612in}{0.650397in}}%
\pgfpathlineto{\pgfqpoint{1.793457in}{0.650397in}}%
\pgfpathlineto{\pgfqpoint{1.795149in}{0.646352in}}%
\pgfpathlineto{\pgfqpoint{1.796840in}{0.646352in}}%
\pgfpathlineto{\pgfqpoint{1.798531in}{0.640992in}}%
\pgfpathlineto{\pgfqpoint{1.800222in}{0.651811in}}%
\pgfpathlineto{\pgfqpoint{1.801068in}{0.651811in}}%
\pgfpathlineto{\pgfqpoint{1.802759in}{0.643337in}}%
\pgfpathlineto{\pgfqpoint{1.806142in}{0.643440in}}%
\pgfpathlineto{\pgfqpoint{1.807833in}{0.651679in}}%
\pgfpathlineto{\pgfqpoint{1.809524in}{0.651679in}}%
\pgfpathlineto{\pgfqpoint{1.811215in}{0.641074in}}%
\pgfpathlineto{\pgfqpoint{1.812907in}{0.646452in}}%
\pgfpathlineto{\pgfqpoint{1.813752in}{0.646452in}}%
\pgfpathlineto{\pgfqpoint{1.815444in}{0.650084in}}%
\pgfpathlineto{\pgfqpoint{1.817135in}{0.650084in}}%
\pgfpathlineto{\pgfqpoint{1.818826in}{0.640126in}}%
\pgfpathlineto{\pgfqpoint{1.820517in}{0.649233in}}%
\pgfpathlineto{\pgfqpoint{1.821363in}{0.649233in}}%
\pgfpathlineto{\pgfqpoint{1.823900in}{0.647478in}}%
\pgfpathlineto{\pgfqpoint{1.824745in}{0.647478in}}%
\pgfpathlineto{\pgfqpoint{1.826437in}{0.640703in}}%
\pgfpathlineto{\pgfqpoint{1.828128in}{0.651071in}}%
\pgfpathlineto{\pgfqpoint{1.828974in}{0.651071in}}%
\pgfpathlineto{\pgfqpoint{1.830665in}{0.644558in}}%
\pgfpathlineto{\pgfqpoint{1.832356in}{0.644558in}}%
\pgfpathlineto{\pgfqpoint{1.834047in}{0.642616in}}%
\pgfpathlineto{\pgfqpoint{1.835739in}{0.642616in}}%
\pgfpathlineto{\pgfqpoint{1.837430in}{0.651522in}}%
\pgfpathlineto{\pgfqpoint{1.839967in}{0.651522in}}%
\pgfpathlineto{\pgfqpoint{1.841658in}{0.642083in}}%
\pgfpathlineto{\pgfqpoint{1.842504in}{0.642083in}}%
\pgfpathlineto{\pgfqpoint{1.844195in}{0.645328in}}%
\pgfpathlineto{\pgfqpoint{1.845040in}{0.645328in}}%
\pgfpathlineto{\pgfqpoint{1.846732in}{0.650514in}}%
\pgfpathlineto{\pgfqpoint{1.848423in}{0.650514in}}%
\pgfpathlineto{\pgfqpoint{1.850114in}{0.640681in}}%
\pgfpathlineto{\pgfqpoint{1.851805in}{0.648110in}}%
\pgfpathlineto{\pgfqpoint{1.856033in}{0.648355in}}%
\pgfpathlineto{\pgfqpoint{1.857725in}{0.640689in}}%
\pgfpathlineto{\pgfqpoint{1.859416in}{0.650230in}}%
\pgfpathlineto{\pgfqpoint{1.860262in}{0.650230in}}%
\pgfpathlineto{\pgfqpoint{1.861953in}{0.645637in}}%
\pgfpathlineto{\pgfqpoint{1.863644in}{0.645637in}}%
\pgfpathlineto{\pgfqpoint{1.865335in}{0.642068in}}%
\pgfpathlineto{\pgfqpoint{1.867027in}{0.642068in}}%
\pgfpathlineto{\pgfqpoint{1.868718in}{0.651158in}}%
\pgfpathlineto{\pgfqpoint{1.870409in}{0.651158in}}%
\pgfpathlineto{\pgfqpoint{1.872100in}{0.643081in}}%
\pgfpathlineto{\pgfqpoint{1.875483in}{0.644418in}}%
\pgfpathlineto{\pgfqpoint{1.877174in}{0.650689in}}%
\pgfpathlineto{\pgfqpoint{1.878865in}{0.650689in}}%
\pgfpathlineto{\pgfqpoint{1.880557in}{0.641350in}}%
\pgfpathlineto{\pgfqpoint{1.882248in}{0.647091in}}%
\pgfpathlineto{\pgfqpoint{1.883093in}{0.647091in}}%
\pgfpathlineto{\pgfqpoint{1.884785in}{0.648990in}}%
\pgfpathlineto{\pgfqpoint{1.886476in}{0.648990in}}%
\pgfpathlineto{\pgfqpoint{1.888167in}{0.640878in}}%
\pgfpathlineto{\pgfqpoint{1.889858in}{0.649372in}}%
\pgfpathlineto{\pgfqpoint{1.890704in}{0.649372in}}%
\pgfpathlineto{\pgfqpoint{1.892395in}{0.646541in}}%
\pgfpathlineto{\pgfqpoint{1.894087in}{0.646541in}}%
\pgfpathlineto{\pgfqpoint{1.895778in}{0.641757in}}%
\pgfpathlineto{\pgfqpoint{1.897469in}{0.650670in}}%
\pgfpathlineto{\pgfqpoint{1.898315in}{0.650670in}}%
\pgfpathlineto{\pgfqpoint{1.900006in}{0.644006in}}%
\pgfpathlineto{\pgfqpoint{1.905080in}{0.643719in}}%
\pgfpathlineto{\pgfqpoint{1.906771in}{0.650671in}}%
\pgfpathlineto{\pgfqpoint{1.908462in}{0.650671in}}%
\pgfpathlineto{\pgfqpoint{1.910153in}{0.642056in}}%
\pgfpathlineto{\pgfqpoint{1.911845in}{0.646213in}}%
\pgfpathlineto{\pgfqpoint{1.912690in}{0.646213in}}%
\pgfpathlineto{\pgfqpoint{1.914382in}{0.649414in}}%
\pgfpathlineto{\pgfqpoint{1.916073in}{0.649414in}}%
\pgfpathlineto{\pgfqpoint{1.917764in}{0.641194in}}%
\pgfpathlineto{\pgfqpoint{1.919455in}{0.648557in}}%
\pgfpathlineto{\pgfqpoint{1.923683in}{0.647268in}}%
\pgfpathlineto{\pgfqpoint{1.925375in}{0.641625in}}%
\pgfpathlineto{\pgfqpoint{1.927066in}{0.650127in}}%
\pgfpathlineto{\pgfqpoint{1.927911in}{0.650127in}}%
\pgfpathlineto{\pgfqpoint{1.929603in}{0.644825in}}%
\pgfpathlineto{\pgfqpoint{1.931294in}{0.644825in}}%
\pgfpathlineto{\pgfqpoint{1.933831in}{0.643204in}}%
\pgfpathlineto{\pgfqpoint{1.934676in}{0.643204in}}%
\pgfpathlineto{\pgfqpoint{1.936368in}{0.650527in}}%
\pgfpathlineto{\pgfqpoint{1.939750in}{0.650527in}}%
\pgfpathlineto{\pgfqpoint{1.941441in}{0.642743in}}%
\pgfpathlineto{\pgfqpoint{1.943133in}{0.642743in}}%
\pgfpathlineto{\pgfqpoint{1.944824in}{0.645478in}}%
\pgfpathlineto{\pgfqpoint{1.945670in}{0.645478in}}%
\pgfpathlineto{\pgfqpoint{1.947361in}{0.649678in}}%
\pgfpathlineto{\pgfqpoint{1.949052in}{0.649678in}}%
\pgfpathlineto{\pgfqpoint{1.950743in}{0.641573in}}%
\pgfpathlineto{\pgfqpoint{1.952435in}{0.647817in}}%
\pgfpathlineto{\pgfqpoint{1.956663in}{0.647841in}}%
\pgfpathlineto{\pgfqpoint{1.958354in}{0.641613in}}%
\pgfpathlineto{\pgfqpoint{1.960045in}{0.649584in}}%
\pgfpathlineto{\pgfqpoint{1.960891in}{0.649584in}}%
\pgfpathlineto{\pgfqpoint{1.962582in}{0.645530in}}%
\pgfpathlineto{\pgfqpoint{1.964273in}{0.645530in}}%
\pgfpathlineto{\pgfqpoint{1.965965in}{0.642830in}}%
\pgfpathlineto{\pgfqpoint{1.967656in}{0.642830in}}%
\pgfpathlineto{\pgfqpoint{1.969347in}{0.650316in}}%
\pgfpathlineto{\pgfqpoint{1.971884in}{0.650316in}}%
\pgfpathlineto{\pgfqpoint{1.973575in}{0.643378in}}%
\pgfpathlineto{\pgfqpoint{1.976958in}{0.644870in}}%
\pgfpathlineto{\pgfqpoint{1.978649in}{0.649836in}}%
\pgfpathlineto{\pgfqpoint{1.980340in}{0.649836in}}%
\pgfpathlineto{\pgfqpoint{1.982031in}{0.641966in}}%
\pgfpathlineto{\pgfqpoint{1.983723in}{0.647160in}}%
\pgfpathlineto{\pgfqpoint{1.987951in}{0.648299in}}%
\pgfpathlineto{\pgfqpoint{1.989642in}{0.641666in}}%
\pgfpathlineto{\pgfqpoint{1.991333in}{0.649070in}}%
\pgfpathlineto{\pgfqpoint{1.992179in}{0.649070in}}%
\pgfpathlineto{\pgfqpoint{1.993870in}{0.646141in}}%
\pgfpathlineto{\pgfqpoint{1.995561in}{0.646141in}}%
\pgfpathlineto{\pgfqpoint{1.997253in}{0.642547in}}%
\pgfpathlineto{\pgfqpoint{1.998944in}{0.650082in}}%
\pgfpathlineto{\pgfqpoint{1.999790in}{0.650082in}}%
\pgfpathlineto{\pgfqpoint{2.001481in}{0.643958in}}%
\pgfpathlineto{\pgfqpoint{2.006554in}{0.644351in}}%
\pgfpathlineto{\pgfqpoint{2.008246in}{0.649935in}}%
\pgfpathlineto{\pgfqpoint{2.010783in}{0.649935in}}%
\pgfpathlineto{\pgfqpoint{2.012474in}{0.642346in}}%
\pgfpathlineto{\pgfqpoint{2.013319in}{0.642346in}}%
\pgfpathlineto{\pgfqpoint{2.015011in}{0.646571in}}%
\pgfpathlineto{\pgfqpoint{2.016702in}{0.646571in}}%
\pgfpathlineto{\pgfqpoint{2.018393in}{0.648685in}}%
\pgfpathlineto{\pgfqpoint{2.020084in}{0.648685in}}%
\pgfpathlineto{\pgfqpoint{2.021776in}{0.641744in}}%
\pgfpathlineto{\pgfqpoint{2.023467in}{0.648588in}}%
\pgfpathlineto{\pgfqpoint{2.024313in}{0.648588in}}%
\pgfpathlineto{\pgfqpoint{2.026004in}{0.646688in}}%
\pgfpathlineto{\pgfqpoint{2.027695in}{0.646688in}}%
\pgfpathlineto{\pgfqpoint{2.029386in}{0.642311in}}%
\pgfpathlineto{\pgfqpoint{2.031078in}{0.649848in}}%
\pgfpathlineto{\pgfqpoint{2.031923in}{0.649848in}}%
\pgfpathlineto{\pgfqpoint{2.033614in}{0.644495in}}%
\pgfpathlineto{\pgfqpoint{2.038688in}{0.643883in}}%
\pgfpathlineto{\pgfqpoint{2.040379in}{0.650011in}}%
\pgfpathlineto{\pgfqpoint{2.042071in}{0.650011in}}%
\pgfpathlineto{\pgfqpoint{2.043762in}{0.642712in}}%
\pgfpathlineto{\pgfqpoint{2.044608in}{0.642712in}}%
\pgfpathlineto{\pgfqpoint{2.046299in}{0.646019in}}%
\pgfpathlineto{\pgfqpoint{2.047144in}{0.646019in}}%
\pgfpathlineto{\pgfqpoint{2.048836in}{0.649041in}}%
\pgfpathlineto{\pgfqpoint{2.050527in}{0.649041in}}%
\pgfpathlineto{\pgfqpoint{2.052218in}{0.641829in}}%
\pgfpathlineto{\pgfqpoint{2.053909in}{0.648124in}}%
\pgfpathlineto{\pgfqpoint{2.058138in}{0.647209in}}%
\pgfpathlineto{\pgfqpoint{2.059829in}{0.642091in}}%
\pgfpathlineto{\pgfqpoint{2.061520in}{0.649615in}}%
\pgfpathlineto{\pgfqpoint{2.062366in}{0.649615in}}%
\pgfpathlineto{\pgfqpoint{2.064057in}{0.645022in}}%
\pgfpathlineto{\pgfqpoint{2.065748in}{0.645022in}}%
\pgfpathlineto{\pgfqpoint{2.067439in}{0.643427in}}%
\pgfpathlineto{\pgfqpoint{2.069131in}{0.650080in}}%
\pgfpathlineto{\pgfqpoint{2.069976in}{0.650080in}}%
\pgfpathlineto{\pgfqpoint{2.071668in}{0.643081in}}%
\pgfpathlineto{\pgfqpoint{2.072513in}{0.643081in}}%
\pgfpathlineto{\pgfqpoint{2.074204in}{0.645467in}}%
\pgfpathlineto{\pgfqpoint{2.075050in}{0.645467in}}%
\pgfpathlineto{\pgfqpoint{2.076741in}{0.649392in}}%
\pgfpathlineto{\pgfqpoint{2.078433in}{0.649392in}}%
\pgfpathlineto{\pgfqpoint{2.080124in}{0.641923in}}%
\pgfpathlineto{\pgfqpoint{2.081815in}{0.647648in}}%
\pgfpathlineto{\pgfqpoint{2.086043in}{0.647739in}}%
\pgfpathlineto{\pgfqpoint{2.087734in}{0.641875in}}%
\pgfpathlineto{\pgfqpoint{2.089426in}{0.649362in}}%
\pgfpathlineto{\pgfqpoint{2.090271in}{0.649362in}}%
\pgfpathlineto{\pgfqpoint{2.091962in}{0.645572in}}%
\pgfpathlineto{\pgfqpoint{2.093654in}{0.645572in}}%
\pgfpathlineto{\pgfqpoint{2.095345in}{0.642959in}}%
\pgfpathlineto{\pgfqpoint{2.097036in}{0.650136in}}%
\pgfpathlineto{\pgfqpoint{2.097882in}{0.650136in}}%
\pgfpathlineto{\pgfqpoint{2.099573in}{0.643482in}}%
\pgfpathlineto{\pgfqpoint{2.104647in}{0.644884in}}%
\pgfpathlineto{\pgfqpoint{2.106338in}{0.649749in}}%
\pgfpathlineto{\pgfqpoint{2.108875in}{0.649749in}}%
\pgfpathlineto{\pgfqpoint{2.110566in}{0.642048in}}%
\pgfpathlineto{\pgfqpoint{2.112257in}{0.642048in}}%
\pgfpathlineto{\pgfqpoint{2.113949in}{0.647123in}}%
\pgfpathlineto{\pgfqpoint{2.119022in}{0.648300in}}%
\pgfpathlineto{\pgfqpoint{2.120714in}{0.641672in}}%
\pgfpathlineto{\pgfqpoint{2.122405in}{0.649059in}}%
\pgfpathlineto{\pgfqpoint{2.123251in}{0.649059in}}%
\pgfpathlineto{\pgfqpoint{2.124942in}{0.646178in}}%
\pgfpathlineto{\pgfqpoint{2.126633in}{0.646178in}}%
\pgfpathlineto{\pgfqpoint{2.128324in}{0.642474in}}%
\pgfpathlineto{\pgfqpoint{2.130016in}{0.650155in}}%
\pgfpathlineto{\pgfqpoint{2.130861in}{0.650155in}}%
\pgfpathlineto{\pgfqpoint{2.132552in}{0.643954in}}%
\pgfpathlineto{\pgfqpoint{2.137626in}{0.644248in}}%
\pgfpathlineto{\pgfqpoint{2.139317in}{0.650100in}}%
\pgfpathlineto{\pgfqpoint{2.141854in}{0.650100in}}%
\pgfpathlineto{\pgfqpoint{2.143546in}{0.642237in}}%
\pgfpathlineto{\pgfqpoint{2.144391in}{0.642237in}}%
\pgfpathlineto{\pgfqpoint{2.146082in}{0.646519in}}%
\pgfpathlineto{\pgfqpoint{2.146928in}{0.646519in}}%
\pgfpathlineto{\pgfqpoint{2.148619in}{0.648898in}}%
\pgfpathlineto{\pgfqpoint{2.150311in}{0.648898in}}%
\pgfpathlineto{\pgfqpoint{2.152002in}{0.641507in}}%
\pgfpathlineto{\pgfqpoint{2.153693in}{0.648666in}}%
\pgfpathlineto{\pgfqpoint{2.154539in}{0.648666in}}%
\pgfpathlineto{\pgfqpoint{2.157076in}{0.646860in}}%
\pgfpathlineto{\pgfqpoint{2.157921in}{0.646860in}}%
\pgfpathlineto{\pgfqpoint{2.159612in}{0.641983in}}%
\pgfpathlineto{\pgfqpoint{2.161304in}{0.650100in}}%
\pgfpathlineto{\pgfqpoint{2.162149in}{0.650100in}}%
\pgfpathlineto{\pgfqpoint{2.163840in}{0.644528in}}%
\pgfpathlineto{\pgfqpoint{2.167223in}{0.643555in}}%
\pgfpathlineto{\pgfqpoint{2.168914in}{0.650417in}}%
\pgfpathlineto{\pgfqpoint{2.169760in}{0.650417in}}%
\pgfpathlineto{\pgfqpoint{2.171451in}{0.642533in}}%
\pgfpathlineto{\pgfqpoint{2.173142in}{0.642533in}}%
\pgfpathlineto{\pgfqpoint{2.174834in}{0.645813in}}%
\pgfpathlineto{\pgfqpoint{2.176525in}{0.645813in}}%
\pgfpathlineto{\pgfqpoint{2.178216in}{0.649516in}}%
\pgfpathlineto{\pgfqpoint{2.180753in}{0.649516in}}%
\pgfpathlineto{\pgfqpoint{2.182444in}{0.641423in}}%
\pgfpathlineto{\pgfqpoint{2.183290in}{0.641423in}}%
\pgfpathlineto{\pgfqpoint{2.184981in}{0.648148in}}%
\pgfpathlineto{\pgfqpoint{2.190055in}{0.647621in}}%
\pgfpathlineto{\pgfqpoint{2.191746in}{0.641520in}}%
\pgfpathlineto{\pgfqpoint{2.193437in}{0.649926in}}%
\pgfpathlineto{\pgfqpoint{2.194283in}{0.649926in}}%
\pgfpathlineto{\pgfqpoint{2.195974in}{0.645228in}}%
\pgfpathlineto{\pgfqpoint{2.197665in}{0.645228in}}%
\pgfpathlineto{\pgfqpoint{2.199357in}{0.642823in}}%
\pgfpathlineto{\pgfqpoint{2.201048in}{0.650654in}}%
\pgfpathlineto{\pgfqpoint{2.201894in}{0.650654in}}%
\pgfpathlineto{\pgfqpoint{2.203585in}{0.642972in}}%
\pgfpathlineto{\pgfqpoint{2.205276in}{0.642972in}}%
\pgfpathlineto{\pgfqpoint{2.206967in}{0.645001in}}%
\pgfpathlineto{\pgfqpoint{2.208659in}{0.645001in}}%
\pgfpathlineto{\pgfqpoint{2.210350in}{0.650120in}}%
\pgfpathlineto{\pgfqpoint{2.212041in}{0.650120in}}%
\pgfpathlineto{\pgfqpoint{2.213732in}{0.641465in}}%
\pgfpathlineto{\pgfqpoint{2.215424in}{0.641465in}}%
\pgfpathlineto{\pgfqpoint{2.217115in}{0.647480in}}%
\pgfpathlineto{\pgfqpoint{2.221343in}{0.648445in}}%
\pgfpathlineto{\pgfqpoint{2.223034in}{0.641131in}}%
\pgfpathlineto{\pgfqpoint{2.224725in}{0.649592in}}%
\pgfpathlineto{\pgfqpoint{2.225571in}{0.649592in}}%
\pgfpathlineto{\pgfqpoint{2.227262in}{0.646058in}}%
\pgfpathlineto{\pgfqpoint{2.228954in}{0.646058in}}%
\pgfpathlineto{\pgfqpoint{2.230645in}{0.642087in}}%
\pgfpathlineto{\pgfqpoint{2.232336in}{0.650763in}}%
\pgfpathlineto{\pgfqpoint{2.233182in}{0.650763in}}%
\pgfpathlineto{\pgfqpoint{2.234873in}{0.643583in}}%
\pgfpathlineto{\pgfqpoint{2.238255in}{0.644102in}}%
\pgfpathlineto{\pgfqpoint{2.239947in}{0.650661in}}%
\pgfpathlineto{\pgfqpoint{2.240792in}{0.650661in}}%
\pgfpathlineto{\pgfqpoint{2.242483in}{0.641680in}}%
\pgfpathlineto{\pgfqpoint{2.243329in}{0.641680in}}%
\pgfpathlineto{\pgfqpoint{2.245020in}{0.646654in}}%
\pgfpathlineto{\pgfqpoint{2.245866in}{0.646654in}}%
\pgfpathlineto{\pgfqpoint{2.247557in}{0.649294in}}%
\pgfpathlineto{\pgfqpoint{2.249248in}{0.649294in}}%
\pgfpathlineto{\pgfqpoint{2.250940in}{0.640872in}}%
\pgfpathlineto{\pgfqpoint{2.252631in}{0.649067in}}%
\pgfpathlineto{\pgfqpoint{2.253477in}{0.649067in}}%
\pgfpathlineto{\pgfqpoint{2.255168in}{0.647002in}}%
\pgfpathlineto{\pgfqpoint{2.256859in}{0.647002in}}%
\pgfpathlineto{\pgfqpoint{2.258550in}{0.641400in}}%
\pgfpathlineto{\pgfqpoint{2.260242in}{0.650691in}}%
\pgfpathlineto{\pgfqpoint{2.261087in}{0.650691in}}%
\pgfpathlineto{\pgfqpoint{2.262778in}{0.644375in}}%
\pgfpathlineto{\pgfqpoint{2.266161in}{0.643152in}}%
\pgfpathlineto{\pgfqpoint{2.267852in}{0.651081in}}%
\pgfpathlineto{\pgfqpoint{2.268698in}{0.651081in}}%
\pgfpathlineto{\pgfqpoint{2.270389in}{0.642103in}}%
\pgfpathlineto{\pgfqpoint{2.272080in}{0.642103in}}%
\pgfpathlineto{\pgfqpoint{2.273772in}{0.645685in}}%
\pgfpathlineto{\pgfqpoint{2.275463in}{0.645685in}}%
\pgfpathlineto{\pgfqpoint{2.277154in}{0.650114in}}%
\pgfpathlineto{\pgfqpoint{2.279691in}{0.650114in}}%
\pgfpathlineto{\pgfqpoint{2.281382in}{0.640797in}}%
\pgfpathlineto{\pgfqpoint{2.282228in}{0.640797in}}%
\pgfpathlineto{\pgfqpoint{2.283919in}{0.648335in}}%
\pgfpathlineto{\pgfqpoint{2.288147in}{0.648023in}}%
\pgfpathlineto{\pgfqpoint{2.289838in}{0.640825in}}%
\pgfpathlineto{\pgfqpoint{2.291530in}{0.650399in}}%
\pgfpathlineto{\pgfqpoint{2.292375in}{0.650399in}}%
\pgfpathlineto{\pgfqpoint{2.294067in}{0.645337in}}%
\pgfpathlineto{\pgfqpoint{2.295758in}{0.645337in}}%
\pgfpathlineto{\pgfqpoint{2.297449in}{0.642209in}}%
\pgfpathlineto{\pgfqpoint{2.299140in}{0.651321in}}%
\pgfpathlineto{\pgfqpoint{2.299986in}{0.651321in}}%
\pgfpathlineto{\pgfqpoint{2.301677in}{0.642753in}}%
\pgfpathlineto{\pgfqpoint{2.303368in}{0.642753in}}%
\pgfpathlineto{\pgfqpoint{2.305905in}{0.644608in}}%
\pgfpathlineto{\pgfqpoint{2.306751in}{0.644608in}}%
\pgfpathlineto{\pgfqpoint{2.308442in}{0.650841in}}%
\pgfpathlineto{\pgfqpoint{2.310133in}{0.650841in}}%
\pgfpathlineto{\pgfqpoint{2.311825in}{0.640952in}}%
\pgfpathlineto{\pgfqpoint{2.312670in}{0.640952in}}%
\pgfpathlineto{\pgfqpoint{2.314362in}{0.647403in}}%
\pgfpathlineto{\pgfqpoint{2.315207in}{0.647403in}}%
\pgfpathlineto{\pgfqpoint{2.317744in}{0.649065in}}%
\pgfpathlineto{\pgfqpoint{2.318590in}{0.649065in}}%
\pgfpathlineto{\pgfqpoint{2.320281in}{0.640425in}}%
\pgfpathlineto{\pgfqpoint{2.321972in}{0.649861in}}%
\pgfpathlineto{\pgfqpoint{2.322818in}{0.649861in}}%
\pgfpathlineto{\pgfqpoint{2.324509in}{0.646436in}}%
\pgfpathlineto{\pgfqpoint{2.326200in}{0.646436in}}%
\pgfpathlineto{\pgfqpoint{2.327891in}{0.641339in}}%
\pgfpathlineto{\pgfqpoint{2.329583in}{0.651328in}}%
\pgfpathlineto{\pgfqpoint{2.330428in}{0.651328in}}%
\pgfpathlineto{\pgfqpoint{2.332120in}{0.643626in}}%
\pgfpathlineto{\pgfqpoint{2.335502in}{0.643479in}}%
\pgfpathlineto{\pgfqpoint{2.337193in}{0.651408in}}%
\pgfpathlineto{\pgfqpoint{2.338039in}{0.651408in}}%
\pgfpathlineto{\pgfqpoint{2.339730in}{0.641366in}}%
\pgfpathlineto{\pgfqpoint{2.340576in}{0.641366in}}%
\pgfpathlineto{\pgfqpoint{2.342267in}{0.646302in}}%
\pgfpathlineto{\pgfqpoint{2.343113in}{0.646302in}}%
\pgfpathlineto{\pgfqpoint{2.344804in}{0.650060in}}%
\pgfpathlineto{\pgfqpoint{2.347341in}{0.650060in}}%
\pgfpathlineto{\pgfqpoint{2.349032in}{0.640258in}}%
\pgfpathlineto{\pgfqpoint{2.350723in}{0.649075in}}%
\pgfpathlineto{\pgfqpoint{2.354951in}{0.647616in}}%
\pgfpathlineto{\pgfqpoint{2.356643in}{0.640615in}}%
\pgfpathlineto{\pgfqpoint{2.358334in}{0.651067in}}%
\pgfpathlineto{\pgfqpoint{2.359180in}{0.651067in}}%
\pgfpathlineto{\pgfqpoint{2.360871in}{0.644695in}}%
\pgfpathlineto{\pgfqpoint{2.362562in}{0.644695in}}%
\pgfpathlineto{\pgfqpoint{2.364253in}{0.642370in}}%
\pgfpathlineto{\pgfqpoint{2.365945in}{0.651751in}}%
\pgfpathlineto{\pgfqpoint{2.366790in}{0.651751in}}%
\pgfpathlineto{\pgfqpoint{2.368481in}{0.642047in}}%
\pgfpathlineto{\pgfqpoint{2.370173in}{0.642047in}}%
\pgfpathlineto{\pgfqpoint{2.371864in}{0.645085in}}%
\pgfpathlineto{\pgfqpoint{2.373555in}{0.645085in}}%
\pgfpathlineto{\pgfqpoint{2.375246in}{0.650933in}}%
\pgfpathlineto{\pgfqpoint{2.376092in}{0.650933in}}%
\pgfpathlineto{\pgfqpoint{2.377783in}{0.640365in}}%
\pgfpathlineto{\pgfqpoint{2.378629in}{0.640365in}}%
\pgfpathlineto{\pgfqpoint{2.380320in}{0.648060in}}%
\pgfpathlineto{\pgfqpoint{2.384548in}{0.648810in}}%
\pgfpathlineto{\pgfqpoint{2.386240in}{0.640104in}}%
\pgfpathlineto{\pgfqpoint{2.387931in}{0.650521in}}%
\pgfpathlineto{\pgfqpoint{2.388776in}{0.650521in}}%
\pgfpathlineto{\pgfqpoint{2.390468in}{0.645913in}}%
\pgfpathlineto{\pgfqpoint{2.392159in}{0.645913in}}%
\pgfpathlineto{\pgfqpoint{2.393850in}{0.641357in}}%
\pgfpathlineto{\pgfqpoint{2.395541in}{0.651821in}}%
\pgfpathlineto{\pgfqpoint{2.396387in}{0.651821in}}%
\pgfpathlineto{\pgfqpoint{2.398078in}{0.642980in}}%
\pgfpathlineto{\pgfqpoint{2.403152in}{0.643819in}}%
\pgfpathlineto{\pgfqpoint{2.404843in}{0.651613in}}%
\pgfpathlineto{\pgfqpoint{2.407380in}{0.651613in}}%
\pgfpathlineto{\pgfqpoint{2.409071in}{0.640769in}}%
\pgfpathlineto{\pgfqpoint{2.409917in}{0.640769in}}%
\pgfpathlineto{\pgfqpoint{2.411608in}{0.646862in}}%
\pgfpathlineto{\pgfqpoint{2.412454in}{0.646862in}}%
\pgfpathlineto{\pgfqpoint{2.414145in}{0.649937in}}%
\pgfpathlineto{\pgfqpoint{2.415836in}{0.649937in}}%
\pgfpathlineto{\pgfqpoint{2.417528in}{0.639864in}}%
\pgfpathlineto{\pgfqpoint{2.419219in}{0.649698in}}%
\pgfpathlineto{\pgfqpoint{2.420064in}{0.649698in}}%
\pgfpathlineto{\pgfqpoint{2.421756in}{0.647212in}}%
\pgfpathlineto{\pgfqpoint{2.423447in}{0.647212in}}%
\pgfpathlineto{\pgfqpoint{2.425138in}{0.640520in}}%
\pgfpathlineto{\pgfqpoint{2.426829in}{0.651587in}}%
\pgfpathlineto{\pgfqpoint{2.427675in}{0.651587in}}%
\pgfpathlineto{\pgfqpoint{2.429366in}{0.644125in}}%
\pgfpathlineto{\pgfqpoint{2.432749in}{0.642587in}}%
\pgfpathlineto{\pgfqpoint{2.434440in}{0.652036in}}%
\pgfpathlineto{\pgfqpoint{2.435286in}{0.652036in}}%
\pgfpathlineto{\pgfqpoint{2.436977in}{0.641468in}}%
\pgfpathlineto{\pgfqpoint{2.438668in}{0.641468in}}%
\pgfpathlineto{\pgfqpoint{2.440359in}{0.645545in}}%
\pgfpathlineto{\pgfqpoint{2.442051in}{0.645545in}}%
\pgfpathlineto{\pgfqpoint{2.443742in}{0.650919in}}%
\pgfpathlineto{\pgfqpoint{2.446279in}{0.650919in}}%
\pgfpathlineto{\pgfqpoint{2.447970in}{0.639930in}}%
\pgfpathlineto{\pgfqpoint{2.448816in}{0.639930in}}%
\pgfpathlineto{\pgfqpoint{2.450507in}{0.648630in}}%
\pgfpathlineto{\pgfqpoint{2.454735in}{0.648512in}}%
\pgfpathlineto{\pgfqpoint{2.456426in}{0.639925in}}%
\pgfpathlineto{\pgfqpoint{2.458118in}{0.651044in}}%
\pgfpathlineto{\pgfqpoint{2.458963in}{0.651044in}}%
\pgfpathlineto{\pgfqpoint{2.460654in}{0.645422in}}%
\pgfpathlineto{\pgfqpoint{2.462346in}{0.645422in}}%
\pgfpathlineto{\pgfqpoint{2.464037in}{0.641472in}}%
\pgfpathlineto{\pgfqpoint{2.465728in}{0.652158in}}%
\pgfpathlineto{\pgfqpoint{2.466574in}{0.652158in}}%
\pgfpathlineto{\pgfqpoint{2.468265in}{0.642434in}}%
\pgfpathlineto{\pgfqpoint{2.469956in}{0.642434in}}%
\pgfpathlineto{\pgfqpoint{2.472493in}{0.644187in}}%
\pgfpathlineto{\pgfqpoint{2.473339in}{0.644187in}}%
\pgfpathlineto{\pgfqpoint{2.475030in}{0.651682in}}%
\pgfpathlineto{\pgfqpoint{2.476721in}{0.651682in}}%
\pgfpathlineto{\pgfqpoint{2.478413in}{0.640317in}}%
\pgfpathlineto{\pgfqpoint{2.479258in}{0.640317in}}%
\pgfpathlineto{\pgfqpoint{2.480949in}{0.647373in}}%
\pgfpathlineto{\pgfqpoint{2.481795in}{0.647373in}}%
\pgfpathlineto{\pgfqpoint{2.483486in}{0.649731in}}%
\pgfpathlineto{\pgfqpoint{2.485177in}{0.649731in}}%
\pgfpathlineto{\pgfqpoint{2.486869in}{0.639625in}}%
\pgfpathlineto{\pgfqpoint{2.488560in}{0.650208in}}%
\pgfpathlineto{\pgfqpoint{2.489406in}{0.650208in}}%
\pgfpathlineto{\pgfqpoint{2.491097in}{0.646795in}}%
\pgfpathlineto{\pgfqpoint{2.492788in}{0.646795in}}%
\pgfpathlineto{\pgfqpoint{2.494479in}{0.640553in}}%
\pgfpathlineto{\pgfqpoint{2.496171in}{0.651957in}}%
\pgfpathlineto{\pgfqpoint{2.497016in}{0.651957in}}%
\pgfpathlineto{\pgfqpoint{2.498707in}{0.643618in}}%
\pgfpathlineto{\pgfqpoint{2.502090in}{0.642876in}}%
\pgfpathlineto{\pgfqpoint{2.503781in}{0.652169in}}%
\pgfpathlineto{\pgfqpoint{2.504627in}{0.652169in}}%
\pgfpathlineto{\pgfqpoint{2.506318in}{0.641012in}}%
\pgfpathlineto{\pgfqpoint{2.507164in}{0.641012in}}%
\pgfpathlineto{\pgfqpoint{2.508855in}{0.646000in}}%
\pgfpathlineto{\pgfqpoint{2.509701in}{0.646000in}}%
\pgfpathlineto{\pgfqpoint{2.511392in}{0.650787in}}%
\pgfpathlineto{\pgfqpoint{2.513083in}{0.650787in}}%
\pgfpathlineto{\pgfqpoint{2.514774in}{0.639650in}}%
\pgfpathlineto{\pgfqpoint{2.516466in}{0.649121in}}%
\pgfpathlineto{\pgfqpoint{2.520694in}{0.648159in}}%
\pgfpathlineto{\pgfqpoint{2.522385in}{0.639894in}}%
\pgfpathlineto{\pgfqpoint{2.524076in}{0.651433in}}%
\pgfpathlineto{\pgfqpoint{2.524922in}{0.651433in}}%
\pgfpathlineto{\pgfqpoint{2.526613in}{0.644952in}}%
\pgfpathlineto{\pgfqpoint{2.528304in}{0.644952in}}%
\pgfpathlineto{\pgfqpoint{2.529996in}{0.641695in}}%
\pgfpathlineto{\pgfqpoint{2.531687in}{0.652339in}}%
\pgfpathlineto{\pgfqpoint{2.532532in}{0.652339in}}%
\pgfpathlineto{\pgfqpoint{2.534224in}{0.641979in}}%
\pgfpathlineto{\pgfqpoint{2.535915in}{0.641979in}}%
\pgfpathlineto{\pgfqpoint{2.537606in}{0.644594in}}%
\pgfpathlineto{\pgfqpoint{2.539297in}{0.644594in}}%
\pgfpathlineto{\pgfqpoint{2.540989in}{0.651610in}}%
\pgfpathlineto{\pgfqpoint{2.543526in}{0.651610in}}%
\pgfpathlineto{\pgfqpoint{2.545217in}{0.640005in}}%
\pgfpathlineto{\pgfqpoint{2.546908in}{0.640005in}}%
\pgfpathlineto{\pgfqpoint{2.548599in}{0.647847in}}%
\pgfpathlineto{\pgfqpoint{2.549445in}{0.647847in}}%
\pgfpathlineto{\pgfqpoint{2.551982in}{0.649431in}}%
\pgfpathlineto{\pgfqpoint{2.552827in}{0.649431in}}%
\pgfpathlineto{\pgfqpoint{2.554519in}{0.639542in}}%
\pgfpathlineto{\pgfqpoint{2.556210in}{0.650612in}}%
\pgfpathlineto{\pgfqpoint{2.557056in}{0.650612in}}%
\pgfpathlineto{\pgfqpoint{2.558747in}{0.646355in}}%
\pgfpathlineto{\pgfqpoint{2.560438in}{0.646355in}}%
\pgfpathlineto{\pgfqpoint{2.562129in}{0.640720in}}%
\pgfpathlineto{\pgfqpoint{2.563820in}{0.652179in}}%
\pgfpathlineto{\pgfqpoint{2.564666in}{0.652179in}}%
\pgfpathlineto{\pgfqpoint{2.566357in}{0.643164in}}%
\pgfpathlineto{\pgfqpoint{2.571431in}{0.643243in}}%
\pgfpathlineto{\pgfqpoint{2.573122in}{0.652148in}}%
\pgfpathlineto{\pgfqpoint{2.575659in}{0.652148in}}%
\pgfpathlineto{\pgfqpoint{2.577350in}{0.640671in}}%
\pgfpathlineto{\pgfqpoint{2.578196in}{0.640671in}}%
\pgfpathlineto{\pgfqpoint{2.579887in}{0.646461in}}%
\pgfpathlineto{\pgfqpoint{2.580733in}{0.646461in}}%
\pgfpathlineto{\pgfqpoint{2.582424in}{0.650532in}}%
\pgfpathlineto{\pgfqpoint{2.584115in}{0.650532in}}%
\pgfpathlineto{\pgfqpoint{2.585807in}{0.639520in}}%
\pgfpathlineto{\pgfqpoint{2.587498in}{0.649543in}}%
\pgfpathlineto{\pgfqpoint{2.588344in}{0.649543in}}%
\pgfpathlineto{\pgfqpoint{2.590880in}{0.647743in}}%
\pgfpathlineto{\pgfqpoint{2.591726in}{0.647743in}}%
\pgfpathlineto{\pgfqpoint{2.593417in}{0.640012in}}%
\pgfpathlineto{\pgfqpoint{2.595109in}{0.651695in}}%
\pgfpathlineto{\pgfqpoint{2.595954in}{0.651695in}}%
\pgfpathlineto{\pgfqpoint{2.597645in}{0.644492in}}%
\pgfpathlineto{\pgfqpoint{2.599337in}{0.644492in}}%
\pgfpathlineto{\pgfqpoint{2.601028in}{0.642028in}}%
\pgfpathlineto{\pgfqpoint{2.602719in}{0.652368in}}%
\pgfpathlineto{\pgfqpoint{2.603565in}{0.652368in}}%
\pgfpathlineto{\pgfqpoint{2.605256in}{0.641607in}}%
\pgfpathlineto{\pgfqpoint{2.606947in}{0.641607in}}%
\pgfpathlineto{\pgfqpoint{2.608639in}{0.645048in}}%
\pgfpathlineto{\pgfqpoint{2.610330in}{0.645048in}}%
\pgfpathlineto{\pgfqpoint{2.612021in}{0.651397in}}%
\pgfpathlineto{\pgfqpoint{2.614558in}{0.651397in}}%
\pgfpathlineto{\pgfqpoint{2.616249in}{0.639826in}}%
\pgfpathlineto{\pgfqpoint{2.617940in}{0.648291in}}%
\pgfpathlineto{\pgfqpoint{2.622169in}{0.649035in}}%
\pgfpathlineto{\pgfqpoint{2.623860in}{0.639610in}}%
\pgfpathlineto{\pgfqpoint{2.625551in}{0.650920in}}%
\pgfpathlineto{\pgfqpoint{2.626397in}{0.650920in}}%
\pgfpathlineto{\pgfqpoint{2.628088in}{0.645886in}}%
\pgfpathlineto{\pgfqpoint{2.629779in}{0.645886in}}%
\pgfpathlineto{\pgfqpoint{2.631470in}{0.641020in}}%
\pgfpathlineto{\pgfqpoint{2.633162in}{0.652262in}}%
\pgfpathlineto{\pgfqpoint{2.634007in}{0.652262in}}%
\pgfpathlineto{\pgfqpoint{2.635699in}{0.642753in}}%
\pgfpathlineto{\pgfqpoint{2.640772in}{0.643692in}}%
\pgfpathlineto{\pgfqpoint{2.642463in}{0.651980in}}%
\pgfpathlineto{\pgfqpoint{2.644155in}{0.651980in}}%
\pgfpathlineto{\pgfqpoint{2.645846in}{0.640436in}}%
\pgfpathlineto{\pgfqpoint{2.646692in}{0.640436in}}%
\pgfpathlineto{\pgfqpoint{2.648383in}{0.646934in}}%
\pgfpathlineto{\pgfqpoint{2.649228in}{0.646934in}}%
\pgfpathlineto{\pgfqpoint{2.650920in}{0.650156in}}%
\pgfpathlineto{\pgfqpoint{2.652611in}{0.650156in}}%
\pgfpathlineto{\pgfqpoint{2.654302in}{0.639531in}}%
\pgfpathlineto{\pgfqpoint{2.655993in}{0.649905in}}%
\pgfpathlineto{\pgfqpoint{2.656839in}{0.649905in}}%
\pgfpathlineto{\pgfqpoint{2.658530in}{0.647262in}}%
\pgfpathlineto{\pgfqpoint{2.660222in}{0.647262in}}%
\pgfpathlineto{\pgfqpoint{2.661913in}{0.640273in}}%
\pgfpathlineto{\pgfqpoint{2.663604in}{0.651842in}}%
\pgfpathlineto{\pgfqpoint{2.664450in}{0.651842in}}%
\pgfpathlineto{\pgfqpoint{2.666141in}{0.644037in}}%
\pgfpathlineto{\pgfqpoint{2.667832in}{0.644037in}}%
\pgfpathlineto{\pgfqpoint{2.669523in}{0.642470in}}%
\pgfpathlineto{\pgfqpoint{2.671215in}{0.652255in}}%
\pgfpathlineto{\pgfqpoint{2.672060in}{0.652255in}}%
\pgfpathlineto{\pgfqpoint{2.673752in}{0.641306in}}%
\pgfpathlineto{\pgfqpoint{2.674597in}{0.641306in}}%
\pgfpathlineto{\pgfqpoint{2.676288in}{0.645551in}}%
\pgfpathlineto{\pgfqpoint{2.677134in}{0.645551in}}%
\pgfpathlineto{\pgfqpoint{2.678825in}{0.651050in}}%
\pgfpathlineto{\pgfqpoint{2.680517in}{0.651050in}}%
\pgfpathlineto{\pgfqpoint{2.682208in}{0.639769in}}%
\pgfpathlineto{\pgfqpoint{2.683899in}{0.648715in}}%
\pgfpathlineto{\pgfqpoint{2.688127in}{0.648544in}}%
\pgfpathlineto{\pgfqpoint{2.689818in}{0.639820in}}%
\pgfpathlineto{\pgfqpoint{2.691510in}{0.651143in}}%
\pgfpathlineto{\pgfqpoint{2.692355in}{0.651143in}}%
\pgfpathlineto{\pgfqpoint{2.694047in}{0.645384in}}%
\pgfpathlineto{\pgfqpoint{2.695738in}{0.645384in}}%
\pgfpathlineto{\pgfqpoint{2.697429in}{0.641446in}}%
\pgfpathlineto{\pgfqpoint{2.699120in}{0.652218in}}%
\pgfpathlineto{\pgfqpoint{2.699966in}{0.652218in}}%
\pgfpathlineto{\pgfqpoint{2.701657in}{0.642378in}}%
\pgfpathlineto{\pgfqpoint{2.703348in}{0.642378in}}%
\pgfpathlineto{\pgfqpoint{2.705885in}{0.644221in}}%
\pgfpathlineto{\pgfqpoint{2.706731in}{0.644221in}}%
\pgfpathlineto{\pgfqpoint{2.708422in}{0.651675in}}%
\pgfpathlineto{\pgfqpoint{2.710959in}{0.651675in}}%
\pgfpathlineto{\pgfqpoint{2.712650in}{0.640296in}}%
\pgfpathlineto{\pgfqpoint{2.713496in}{0.640296in}}%
\pgfpathlineto{\pgfqpoint{2.715187in}{0.647422in}}%
\pgfpathlineto{\pgfqpoint{2.716033in}{0.647422in}}%
\pgfpathlineto{\pgfqpoint{2.717724in}{0.649667in}}%
\pgfpathlineto{\pgfqpoint{2.719415in}{0.649667in}}%
\pgfpathlineto{\pgfqpoint{2.721106in}{0.639672in}}%
\pgfpathlineto{\pgfqpoint{2.722798in}{0.650216in}}%
\pgfpathlineto{\pgfqpoint{2.723643in}{0.650216in}}%
\pgfpathlineto{\pgfqpoint{2.725335in}{0.646716in}}%
\pgfpathlineto{\pgfqpoint{2.727026in}{0.646716in}}%
\pgfpathlineto{\pgfqpoint{2.728717in}{0.640665in}}%
\pgfpathlineto{\pgfqpoint{2.730408in}{0.651886in}}%
\pgfpathlineto{\pgfqpoint{2.731254in}{0.651886in}}%
\pgfpathlineto{\pgfqpoint{2.732945in}{0.643582in}}%
\pgfpathlineto{\pgfqpoint{2.738019in}{0.643014in}}%
\pgfpathlineto{\pgfqpoint{2.739710in}{0.652013in}}%
\pgfpathlineto{\pgfqpoint{2.742247in}{0.652013in}}%
\pgfpathlineto{\pgfqpoint{2.743938in}{0.641069in}}%
\pgfpathlineto{\pgfqpoint{2.744784in}{0.641069in}}%
\pgfpathlineto{\pgfqpoint{2.746475in}{0.646101in}}%
\pgfpathlineto{\pgfqpoint{2.747321in}{0.646101in}}%
\pgfpathlineto{\pgfqpoint{2.749012in}{0.650579in}}%
\pgfpathlineto{\pgfqpoint{2.750703in}{0.650579in}}%
\pgfpathlineto{\pgfqpoint{2.752395in}{0.639820in}}%
\pgfpathlineto{\pgfqpoint{2.754086in}{0.649121in}}%
\pgfpathlineto{\pgfqpoint{2.758314in}{0.647966in}}%
\pgfpathlineto{\pgfqpoint{2.760005in}{0.640156in}}%
\pgfpathlineto{\pgfqpoint{2.761696in}{0.651291in}}%
\pgfpathlineto{\pgfqpoint{2.762542in}{0.651291in}}%
\pgfpathlineto{\pgfqpoint{2.764233in}{0.644850in}}%
\pgfpathlineto{\pgfqpoint{2.765925in}{0.644850in}}%
\pgfpathlineto{\pgfqpoint{2.767616in}{0.641985in}}%
\pgfpathlineto{\pgfqpoint{2.769307in}{0.652062in}}%
\pgfpathlineto{\pgfqpoint{2.770153in}{0.652062in}}%
\pgfpathlineto{\pgfqpoint{2.771844in}{0.642033in}}%
\pgfpathlineto{\pgfqpoint{2.773535in}{0.642033in}}%
\pgfpathlineto{\pgfqpoint{2.775226in}{0.644822in}}%
\pgfpathlineto{\pgfqpoint{2.776918in}{0.644822in}}%
\pgfpathlineto{\pgfqpoint{2.778609in}{0.651249in}}%
\pgfpathlineto{\pgfqpoint{2.781146in}{0.651249in}}%
\pgfpathlineto{\pgfqpoint{2.782837in}{0.640238in}}%
\pgfpathlineto{\pgfqpoint{2.784528in}{0.647924in}}%
\pgfpathlineto{\pgfqpoint{2.788756in}{0.649076in}}%
\pgfpathlineto{\pgfqpoint{2.790448in}{0.639927in}}%
\pgfpathlineto{\pgfqpoint{2.792139in}{0.650482in}}%
\pgfpathlineto{\pgfqpoint{2.792985in}{0.650482in}}%
\pgfpathlineto{\pgfqpoint{2.794676in}{0.646114in}}%
\pgfpathlineto{\pgfqpoint{2.796367in}{0.646114in}}%
\pgfpathlineto{\pgfqpoint{2.798058in}{0.641173in}}%
\pgfpathlineto{\pgfqpoint{2.799749in}{0.651839in}}%
\pgfpathlineto{\pgfqpoint{2.800595in}{0.651839in}}%
\pgfpathlineto{\pgfqpoint{2.802286in}{0.643128in}}%
\pgfpathlineto{\pgfqpoint{2.807360in}{0.643646in}}%
\pgfpathlineto{\pgfqpoint{2.809051in}{0.651658in}}%
\pgfpathlineto{\pgfqpoint{2.810743in}{0.651658in}}%
\pgfpathlineto{\pgfqpoint{2.812434in}{0.640887in}}%
\pgfpathlineto{\pgfqpoint{2.813279in}{0.640887in}}%
\pgfpathlineto{\pgfqpoint{2.814971in}{0.646691in}}%
\pgfpathlineto{\pgfqpoint{2.815816in}{0.646691in}}%
\pgfpathlineto{\pgfqpoint{2.817508in}{0.650003in}}%
\pgfpathlineto{\pgfqpoint{2.819199in}{0.650003in}}%
\pgfpathlineto{\pgfqpoint{2.820890in}{0.639967in}}%
\pgfpathlineto{\pgfqpoint{2.822581in}{0.649511in}}%
\pgfpathlineto{\pgfqpoint{2.823427in}{0.649511in}}%
\pgfpathlineto{\pgfqpoint{2.825118in}{0.647315in}}%
\pgfpathlineto{\pgfqpoint{2.826809in}{0.647315in}}%
\pgfpathlineto{\pgfqpoint{2.828501in}{0.640603in}}%
\pgfpathlineto{\pgfqpoint{2.830192in}{0.651374in}}%
\pgfpathlineto{\pgfqpoint{2.831038in}{0.651374in}}%
\pgfpathlineto{\pgfqpoint{2.832729in}{0.644292in}}%
\pgfpathlineto{\pgfqpoint{2.834420in}{0.644292in}}%
\pgfpathlineto{\pgfqpoint{2.836111in}{0.642620in}}%
\pgfpathlineto{\pgfqpoint{2.837803in}{0.651808in}}%
\pgfpathlineto{\pgfqpoint{2.838648in}{0.651808in}}%
\pgfpathlineto{\pgfqpoint{2.840339in}{0.641717in}}%
\pgfpathlineto{\pgfqpoint{2.841185in}{0.641717in}}%
\pgfpathlineto{\pgfqpoint{2.842876in}{0.645483in}}%
\pgfpathlineto{\pgfqpoint{2.843722in}{0.645483in}}%
\pgfpathlineto{\pgfqpoint{2.845413in}{0.650718in}}%
\pgfpathlineto{\pgfqpoint{2.847104in}{0.650718in}}%
\pgfpathlineto{\pgfqpoint{2.848796in}{0.640254in}}%
\pgfpathlineto{\pgfqpoint{2.850487in}{0.648436in}}%
\pgfpathlineto{\pgfqpoint{2.854715in}{0.648402in}}%
\pgfpathlineto{\pgfqpoint{2.856406in}{0.640279in}}%
\pgfpathlineto{\pgfqpoint{2.858098in}{0.650707in}}%
\pgfpathlineto{\pgfqpoint{2.858943in}{0.650707in}}%
\pgfpathlineto{\pgfqpoint{2.860634in}{0.645468in}}%
\pgfpathlineto{\pgfqpoint{2.862326in}{0.645468in}}%
\pgfpathlineto{\pgfqpoint{2.864017in}{0.641778in}}%
\pgfpathlineto{\pgfqpoint{2.865708in}{0.651713in}}%
\pgfpathlineto{\pgfqpoint{2.866554in}{0.651713in}}%
\pgfpathlineto{\pgfqpoint{2.868245in}{0.642678in}}%
\pgfpathlineto{\pgfqpoint{2.869936in}{0.642678in}}%
\pgfpathlineto{\pgfqpoint{2.872473in}{0.644350in}}%
\pgfpathlineto{\pgfqpoint{2.873319in}{0.644350in}}%
\pgfpathlineto{\pgfqpoint{2.875010in}{0.651208in}}%
\pgfpathlineto{\pgfqpoint{2.877547in}{0.651208in}}%
\pgfpathlineto{\pgfqpoint{2.879238in}{0.640755in}}%
\pgfpathlineto{\pgfqpoint{2.880084in}{0.640755in}}%
\pgfpathlineto{\pgfqpoint{2.881775in}{0.647311in}}%
\pgfpathlineto{\pgfqpoint{2.882621in}{0.647311in}}%
\pgfpathlineto{\pgfqpoint{2.884312in}{0.649340in}}%
\pgfpathlineto{\pgfqpoint{2.886003in}{0.649340in}}%
\pgfpathlineto{\pgfqpoint{2.887694in}{0.640196in}}%
\pgfpathlineto{\pgfqpoint{2.889386in}{0.649883in}}%
\pgfpathlineto{\pgfqpoint{2.890231in}{0.649883in}}%
\pgfpathlineto{\pgfqpoint{2.891922in}{0.646605in}}%
\pgfpathlineto{\pgfqpoint{2.893614in}{0.646605in}}%
\pgfpathlineto{\pgfqpoint{2.895305in}{0.641140in}}%
\pgfpathlineto{\pgfqpoint{2.896996in}{0.651397in}}%
\pgfpathlineto{\pgfqpoint{2.897842in}{0.651397in}}%
\pgfpathlineto{\pgfqpoint{2.899533in}{0.643718in}}%
\pgfpathlineto{\pgfqpoint{2.904607in}{0.643333in}}%
\pgfpathlineto{\pgfqpoint{2.906298in}{0.651472in}}%
\pgfpathlineto{\pgfqpoint{2.907989in}{0.651472in}}%
\pgfpathlineto{\pgfqpoint{2.909681in}{0.641430in}}%
\pgfpathlineto{\pgfqpoint{2.910526in}{0.641430in}}%
\pgfpathlineto{\pgfqpoint{2.912217in}{0.646187in}}%
\pgfpathlineto{\pgfqpoint{2.913063in}{0.646187in}}%
\pgfpathlineto{\pgfqpoint{2.914754in}{0.650104in}}%
\pgfpathlineto{\pgfqpoint{2.916446in}{0.650104in}}%
\pgfpathlineto{\pgfqpoint{2.918137in}{0.640335in}}%
\pgfpathlineto{\pgfqpoint{2.919828in}{0.648948in}}%
\pgfpathlineto{\pgfqpoint{2.924056in}{0.647662in}}%
\pgfpathlineto{\pgfqpoint{2.925747in}{0.640713in}}%
\pgfpathlineto{\pgfqpoint{2.927439in}{0.650892in}}%
\pgfpathlineto{\pgfqpoint{2.928284in}{0.650892in}}%
\pgfpathlineto{\pgfqpoint{2.929976in}{0.644791in}}%
\pgfpathlineto{\pgfqpoint{2.931667in}{0.644791in}}%
\pgfpathlineto{\pgfqpoint{2.933358in}{0.642460in}}%
\pgfpathlineto{\pgfqpoint{2.935049in}{0.651519in}}%
\pgfpathlineto{\pgfqpoint{2.935895in}{0.651519in}}%
\pgfpathlineto{\pgfqpoint{2.937586in}{0.642238in}}%
\pgfpathlineto{\pgfqpoint{2.939277in}{0.642238in}}%
\pgfpathlineto{\pgfqpoint{2.940969in}{0.645106in}}%
\pgfpathlineto{\pgfqpoint{2.942660in}{0.645106in}}%
\pgfpathlineto{\pgfqpoint{2.944351in}{0.650681in}}%
\pgfpathlineto{\pgfqpoint{2.946042in}{0.650681in}}%
\pgfpathlineto{\pgfqpoint{2.947734in}{0.640670in}}%
\pgfpathlineto{\pgfqpoint{2.949425in}{0.647947in}}%
\pgfpathlineto{\pgfqpoint{2.953653in}{0.648610in}}%
\pgfpathlineto{\pgfqpoint{2.955344in}{0.640494in}}%
\pgfpathlineto{\pgfqpoint{2.957035in}{0.650232in}}%
\pgfpathlineto{\pgfqpoint{2.957881in}{0.650232in}}%
\pgfpathlineto{\pgfqpoint{2.959572in}{0.645856in}}%
\pgfpathlineto{\pgfqpoint{2.961264in}{0.645856in}}%
\pgfpathlineto{\pgfqpoint{2.962955in}{0.641751in}}%
\pgfpathlineto{\pgfqpoint{2.964646in}{0.651367in}}%
\pgfpathlineto{\pgfqpoint{2.965492in}{0.651367in}}%
\pgfpathlineto{\pgfqpoint{2.967183in}{0.643141in}}%
\pgfpathlineto{\pgfqpoint{2.972257in}{0.644101in}}%
\pgfpathlineto{\pgfqpoint{2.973948in}{0.651068in}}%
\pgfpathlineto{\pgfqpoint{2.977330in}{0.651068in}}%
\pgfpathlineto{\pgfqpoint{2.979022in}{0.641175in}}%
\pgfpathlineto{\pgfqpoint{2.980713in}{0.641175in}}%
\pgfpathlineto{\pgfqpoint{2.982404in}{0.646917in}}%
\pgfpathlineto{\pgfqpoint{2.983250in}{0.646917in}}%
\pgfpathlineto{\pgfqpoint{2.984941in}{0.649425in}}%
\pgfpathlineto{\pgfqpoint{2.986632in}{0.649425in}}%
\pgfpathlineto{\pgfqpoint{2.988324in}{0.640472in}}%
\pgfpathlineto{\pgfqpoint{2.990015in}{0.649451in}}%
\pgfpathlineto{\pgfqpoint{2.990860in}{0.649451in}}%
\pgfpathlineto{\pgfqpoint{2.992552in}{0.646879in}}%
\pgfpathlineto{\pgfqpoint{2.994243in}{0.646879in}}%
\pgfpathlineto{\pgfqpoint{2.995934in}{0.641214in}}%
\pgfpathlineto{\pgfqpoint{2.997625in}{0.651037in}}%
\pgfpathlineto{\pgfqpoint{2.998471in}{0.651037in}}%
\pgfpathlineto{\pgfqpoint{3.000162in}{0.644101in}}%
\pgfpathlineto{\pgfqpoint{3.005236in}{0.643198in}}%
\pgfpathlineto{\pgfqpoint{3.006927in}{0.651267in}}%
\pgfpathlineto{\pgfqpoint{3.009464in}{0.651267in}}%
\pgfpathlineto{\pgfqpoint{3.011155in}{0.641819in}}%
\pgfpathlineto{\pgfqpoint{3.012001in}{0.641819in}}%
\pgfpathlineto{\pgfqpoint{3.013692in}{0.645894in}}%
\pgfpathlineto{\pgfqpoint{3.014538in}{0.645894in}}%
\pgfpathlineto{\pgfqpoint{3.016229in}{0.650093in}}%
\pgfpathlineto{\pgfqpoint{3.017920in}{0.650093in}}%
\pgfpathlineto{\pgfqpoint{3.019612in}{0.640633in}}%
\pgfpathlineto{\pgfqpoint{3.021303in}{0.648582in}}%
\pgfpathlineto{\pgfqpoint{3.025531in}{0.647834in}}%
\pgfpathlineto{\pgfqpoint{3.027222in}{0.640852in}}%
\pgfpathlineto{\pgfqpoint{3.028914in}{0.650551in}}%
\pgfpathlineto{\pgfqpoint{3.029759in}{0.650551in}}%
\pgfpathlineto{\pgfqpoint{3.031450in}{0.645087in}}%
\pgfpathlineto{\pgfqpoint{3.033142in}{0.645087in}}%
\pgfpathlineto{\pgfqpoint{3.034833in}{0.642415in}}%
\pgfpathlineto{\pgfqpoint{3.036524in}{0.651287in}}%
\pgfpathlineto{\pgfqpoint{3.037370in}{0.651287in}}%
\pgfpathlineto{\pgfqpoint{3.039061in}{0.642575in}}%
\pgfpathlineto{\pgfqpoint{3.040752in}{0.642575in}}%
\pgfpathlineto{\pgfqpoint{3.042443in}{0.644904in}}%
\pgfpathlineto{\pgfqpoint{3.044135in}{0.644904in}}%
\pgfpathlineto{\pgfqpoint{3.045826in}{0.650608in}}%
\pgfpathlineto{\pgfqpoint{3.048363in}{0.650608in}}%
\pgfpathlineto{\pgfqpoint{3.050054in}{0.640958in}}%
\pgfpathlineto{\pgfqpoint{3.050900in}{0.640958in}}%
\pgfpathlineto{\pgfqpoint{3.052591in}{0.647653in}}%
\pgfpathlineto{\pgfqpoint{3.057665in}{0.648701in}}%
\pgfpathlineto{\pgfqpoint{3.059356in}{0.640663in}}%
\pgfpathlineto{\pgfqpoint{3.061047in}{0.649931in}}%
\pgfpathlineto{\pgfqpoint{3.061893in}{0.649931in}}%
\pgfpathlineto{\pgfqpoint{3.063584in}{0.646073in}}%
\pgfpathlineto{\pgfqpoint{3.065275in}{0.646073in}}%
\pgfpathlineto{\pgfqpoint{3.066967in}{0.641766in}}%
\pgfpathlineto{\pgfqpoint{3.068658in}{0.651138in}}%
\pgfpathlineto{\pgfqpoint{3.069503in}{0.651138in}}%
\pgfpathlineto{\pgfqpoint{3.071195in}{0.643417in}}%
\pgfpathlineto{\pgfqpoint{3.076268in}{0.643972in}}%
\pgfpathlineto{\pgfqpoint{3.077960in}{0.650963in}}%
\pgfpathlineto{\pgfqpoint{3.079651in}{0.650963in}}%
\pgfpathlineto{\pgfqpoint{3.081342in}{0.641432in}}%
\pgfpathlineto{\pgfqpoint{3.082188in}{0.641432in}}%
\pgfpathlineto{\pgfqpoint{3.083879in}{0.646692in}}%
\pgfpathlineto{\pgfqpoint{3.084725in}{0.646692in}}%
\pgfpathlineto{\pgfqpoint{3.086416in}{0.649461in}}%
\pgfpathlineto{\pgfqpoint{3.088107in}{0.649461in}}%
\pgfpathlineto{\pgfqpoint{3.089798in}{0.640643in}}%
\pgfpathlineto{\pgfqpoint{3.091490in}{0.649199in}}%
\pgfpathlineto{\pgfqpoint{3.092335in}{0.649199in}}%
\pgfpathlineto{\pgfqpoint{3.094027in}{0.647034in}}%
\pgfpathlineto{\pgfqpoint{3.095718in}{0.647034in}}%
\pgfpathlineto{\pgfqpoint{3.097409in}{0.641259in}}%
\pgfpathlineto{\pgfqpoint{3.099100in}{0.650830in}}%
\pgfpathlineto{\pgfqpoint{3.099946in}{0.650830in}}%
\pgfpathlineto{\pgfqpoint{3.101637in}{0.644321in}}%
\pgfpathlineto{\pgfqpoint{3.105020in}{0.643117in}}%
\pgfpathlineto{\pgfqpoint{3.106711in}{0.651158in}}%
\pgfpathlineto{\pgfqpoint{3.107557in}{0.651158in}}%
\pgfpathlineto{\pgfqpoint{3.109248in}{0.642037in}}%
\pgfpathlineto{\pgfqpoint{3.110093in}{0.642037in}}%
\pgfpathlineto{\pgfqpoint{3.111785in}{0.645721in}}%
\pgfpathlineto{\pgfqpoint{3.112630in}{0.645721in}}%
\pgfpathlineto{\pgfqpoint{3.114322in}{0.650103in}}%
\pgfpathlineto{\pgfqpoint{3.116013in}{0.650103in}}%
\pgfpathlineto{\pgfqpoint{3.117704in}{0.640787in}}%
\pgfpathlineto{\pgfqpoint{3.119395in}{0.648375in}}%
\pgfpathlineto{\pgfqpoint{3.123623in}{0.647949in}}%
\pgfpathlineto{\pgfqpoint{3.125315in}{0.640903in}}%
\pgfpathlineto{\pgfqpoint{3.127006in}{0.650374in}}%
\pgfpathlineto{\pgfqpoint{3.127851in}{0.650374in}}%
\pgfpathlineto{\pgfqpoint{3.129543in}{0.645265in}}%
\pgfpathlineto{\pgfqpoint{3.131234in}{0.645265in}}%
\pgfpathlineto{\pgfqpoint{3.132925in}{0.642357in}}%
\pgfpathlineto{\pgfqpoint{3.134616in}{0.651190in}}%
\pgfpathlineto{\pgfqpoint{3.135462in}{0.651190in}}%
\pgfpathlineto{\pgfqpoint{3.137153in}{0.642757in}}%
\pgfpathlineto{\pgfqpoint{3.138845in}{0.642757in}}%
\pgfpathlineto{\pgfqpoint{3.140536in}{0.644763in}}%
\pgfpathlineto{\pgfqpoint{3.142227in}{0.644763in}}%
\pgfpathlineto{\pgfqpoint{3.143918in}{0.650613in}}%
\pgfpathlineto{\pgfqpoint{3.146455in}{0.650613in}}%
\pgfpathlineto{\pgfqpoint{3.148146in}{0.641089in}}%
\pgfpathlineto{\pgfqpoint{3.148992in}{0.641089in}}%
\pgfpathlineto{\pgfqpoint{3.150683in}{0.647479in}}%
\pgfpathlineto{\pgfqpoint{3.154911in}{0.648799in}}%
\pgfpathlineto{\pgfqpoint{3.156603in}{0.640704in}}%
\pgfpathlineto{\pgfqpoint{3.158294in}{0.649781in}}%
\pgfpathlineto{\pgfqpoint{3.159140in}{0.649781in}}%
\pgfpathlineto{\pgfqpoint{3.160831in}{0.646228in}}%
\pgfpathlineto{\pgfqpoint{3.162522in}{0.646228in}}%
\pgfpathlineto{\pgfqpoint{3.164213in}{0.641709in}}%
\pgfpathlineto{\pgfqpoint{3.165905in}{0.651059in}}%
\pgfpathlineto{\pgfqpoint{3.166750in}{0.651059in}}%
\pgfpathlineto{\pgfqpoint{3.168441in}{0.643576in}}%
\pgfpathlineto{\pgfqpoint{3.173515in}{0.643841in}}%
\pgfpathlineto{\pgfqpoint{3.175206in}{0.650978in}}%
\pgfpathlineto{\pgfqpoint{3.177743in}{0.650978in}}%
\pgfpathlineto{\pgfqpoint{3.179435in}{0.641542in}}%
\pgfpathlineto{\pgfqpoint{3.180280in}{0.641542in}}%
\pgfpathlineto{\pgfqpoint{3.181971in}{0.646531in}}%
\pgfpathlineto{\pgfqpoint{3.182817in}{0.646531in}}%
\pgfpathlineto{\pgfqpoint{3.184508in}{0.649563in}}%
\pgfpathlineto{\pgfqpoint{3.186200in}{0.649563in}}%
\pgfpathlineto{\pgfqpoint{3.187891in}{0.640668in}}%
\pgfpathlineto{\pgfqpoint{3.189582in}{0.649064in}}%
\pgfpathlineto{\pgfqpoint{3.190428in}{0.649064in}}%
\pgfpathlineto{\pgfqpoint{3.192965in}{0.647187in}}%
\pgfpathlineto{\pgfqpoint{3.193810in}{0.647187in}}%
\pgfpathlineto{\pgfqpoint{3.195501in}{0.641191in}}%
\pgfpathlineto{\pgfqpoint{3.197193in}{0.650766in}}%
\pgfpathlineto{\pgfqpoint{3.198038in}{0.650766in}}%
\pgfpathlineto{\pgfqpoint{3.199729in}{0.644475in}}%
\pgfpathlineto{\pgfqpoint{3.203112in}{0.642977in}}%
\pgfpathlineto{\pgfqpoint{3.204803in}{0.651188in}}%
\pgfpathlineto{\pgfqpoint{3.205649in}{0.651188in}}%
\pgfpathlineto{\pgfqpoint{3.207340in}{0.642140in}}%
\pgfpathlineto{\pgfqpoint{3.209031in}{0.642140in}}%
\pgfpathlineto{\pgfqpoint{3.210723in}{0.645554in}}%
\pgfpathlineto{\pgfqpoint{3.212414in}{0.645554in}}%
\pgfpathlineto{\pgfqpoint{3.214105in}{0.650220in}}%
\pgfpathlineto{\pgfqpoint{3.216642in}{0.650220in}}%
\pgfpathlineto{\pgfqpoint{3.218333in}{0.640801in}}%
\pgfpathlineto{\pgfqpoint{3.219179in}{0.640801in}}%
\pgfpathlineto{\pgfqpoint{3.220870in}{0.648237in}}%
\pgfpathlineto{\pgfqpoint{3.225944in}{0.648118in}}%
\pgfpathlineto{\pgfqpoint{3.227635in}{0.640819in}}%
\pgfpathlineto{\pgfqpoint{3.229326in}{0.650311in}}%
\pgfpathlineto{\pgfqpoint{3.230172in}{0.650311in}}%
\pgfpathlineto{\pgfqpoint{3.231863in}{0.645434in}}%
\pgfpathlineto{\pgfqpoint{3.233554in}{0.645434in}}%
\pgfpathlineto{\pgfqpoint{3.235246in}{0.642197in}}%
\pgfpathlineto{\pgfqpoint{3.236937in}{0.651229in}}%
\pgfpathlineto{\pgfqpoint{3.237783in}{0.651229in}}%
\pgfpathlineto{\pgfqpoint{3.239474in}{0.642871in}}%
\pgfpathlineto{\pgfqpoint{3.241165in}{0.642871in}}%
\pgfpathlineto{\pgfqpoint{3.243702in}{0.644573in}}%
\pgfpathlineto{\pgfqpoint{3.244548in}{0.644573in}}%
\pgfpathlineto{\pgfqpoint{3.246239in}{0.650748in}}%
\pgfpathlineto{\pgfqpoint{3.247930in}{0.650748in}}%
\pgfpathlineto{\pgfqpoint{3.249621in}{0.641106in}}%
\pgfpathlineto{\pgfqpoint{3.250467in}{0.641106in}}%
\pgfpathlineto{\pgfqpoint{3.252158in}{0.647317in}}%
\pgfpathlineto{\pgfqpoint{3.253004in}{0.647317in}}%
\pgfpathlineto{\pgfqpoint{3.255541in}{0.648995in}}%
\pgfpathlineto{\pgfqpoint{3.256386in}{0.648995in}}%
\pgfpathlineto{\pgfqpoint{3.258078in}{0.640611in}}%
\pgfpathlineto{\pgfqpoint{3.259769in}{0.649702in}}%
\pgfpathlineto{\pgfqpoint{3.260614in}{0.649702in}}%
\pgfpathlineto{\pgfqpoint{3.262306in}{0.646428in}}%
\pgfpathlineto{\pgfqpoint{3.263997in}{0.646428in}}%
\pgfpathlineto{\pgfqpoint{3.265688in}{0.641526in}}%
\pgfpathlineto{\pgfqpoint{3.267379in}{0.651093in}}%
\pgfpathlineto{\pgfqpoint{3.268225in}{0.651093in}}%
\pgfpathlineto{\pgfqpoint{3.269916in}{0.643720in}}%
\pgfpathlineto{\pgfqpoint{3.273299in}{0.643616in}}%
\pgfpathlineto{\pgfqpoint{3.274990in}{0.651125in}}%
\pgfpathlineto{\pgfqpoint{3.275836in}{0.651125in}}%
\pgfpathlineto{\pgfqpoint{3.277527in}{0.641582in}}%
\pgfpathlineto{\pgfqpoint{3.278372in}{0.641582in}}%
\pgfpathlineto{\pgfqpoint{3.280064in}{0.646329in}}%
\pgfpathlineto{\pgfqpoint{3.280909in}{0.646329in}}%
\pgfpathlineto{\pgfqpoint{3.282601in}{0.649788in}}%
\pgfpathlineto{\pgfqpoint{3.284292in}{0.649788in}}%
\pgfpathlineto{\pgfqpoint{3.285983in}{0.640582in}}%
\pgfpathlineto{\pgfqpoint{3.287674in}{0.648947in}}%
\pgfpathlineto{\pgfqpoint{3.291902in}{0.647430in}}%
\pgfpathlineto{\pgfqpoint{3.293594in}{0.640993in}}%
\pgfpathlineto{\pgfqpoint{3.295285in}{0.650775in}}%
\pgfpathlineto{\pgfqpoint{3.296131in}{0.650775in}}%
\pgfpathlineto{\pgfqpoint{3.297822in}{0.644667in}}%
\pgfpathlineto{\pgfqpoint{3.299513in}{0.644667in}}%
\pgfpathlineto{\pgfqpoint{3.301204in}{0.642717in}}%
\pgfpathlineto{\pgfqpoint{3.302896in}{0.651330in}}%
\pgfpathlineto{\pgfqpoint{3.303741in}{0.651330in}}%
\pgfpathlineto{\pgfqpoint{3.305432in}{0.642222in}}%
\pgfpathlineto{\pgfqpoint{3.307124in}{0.642222in}}%
\pgfpathlineto{\pgfqpoint{3.308815in}{0.645301in}}%
\pgfpathlineto{\pgfqpoint{3.310506in}{0.645301in}}%
\pgfpathlineto{\pgfqpoint{3.312197in}{0.650465in}}%
\pgfpathlineto{\pgfqpoint{3.314734in}{0.650465in}}%
\pgfpathlineto{\pgfqpoint{3.316426in}{0.640742in}}%
\pgfpathlineto{\pgfqpoint{3.317271in}{0.640742in}}%
\pgfpathlineto{\pgfqpoint{3.318962in}{0.648064in}}%
\pgfpathlineto{\pgfqpoint{3.323191in}{0.648405in}}%
\pgfpathlineto{\pgfqpoint{3.324882in}{0.640623in}}%
\pgfpathlineto{\pgfqpoint{3.326573in}{0.650273in}}%
\pgfpathlineto{\pgfqpoint{3.327419in}{0.650273in}}%
\pgfpathlineto{\pgfqpoint{3.329110in}{0.645685in}}%
\pgfpathlineto{\pgfqpoint{3.330801in}{0.645685in}}%
\pgfpathlineto{\pgfqpoint{3.332492in}{0.641909in}}%
\pgfpathlineto{\pgfqpoint{3.334184in}{0.651345in}}%
\pgfpathlineto{\pgfqpoint{3.335029in}{0.651345in}}%
\pgfpathlineto{\pgfqpoint{3.336721in}{0.643015in}}%
\pgfpathlineto{\pgfqpoint{3.341794in}{0.644267in}}%
\pgfpathlineto{\pgfqpoint{3.343486in}{0.650997in}}%
\pgfpathlineto{\pgfqpoint{3.345177in}{0.650997in}}%
\pgfpathlineto{\pgfqpoint{3.346868in}{0.641097in}}%
\pgfpathlineto{\pgfqpoint{3.347714in}{0.641097in}}%
\pgfpathlineto{\pgfqpoint{3.349405in}{0.647077in}}%
\pgfpathlineto{\pgfqpoint{3.350251in}{0.647077in}}%
\pgfpathlineto{\pgfqpoint{3.351942in}{0.649317in}}%
\pgfpathlineto{\pgfqpoint{3.353633in}{0.649317in}}%
\pgfpathlineto{\pgfqpoint{3.355324in}{0.640440in}}%
\pgfpathlineto{\pgfqpoint{3.357015in}{0.649596in}}%
\pgfpathlineto{\pgfqpoint{3.357861in}{0.649596in}}%
\pgfpathlineto{\pgfqpoint{3.359552in}{0.646742in}}%
\pgfpathlineto{\pgfqpoint{3.361244in}{0.646742in}}%
\pgfpathlineto{\pgfqpoint{3.362935in}{0.641229in}}%
\pgfpathlineto{\pgfqpoint{3.364626in}{0.651159in}}%
\pgfpathlineto{\pgfqpoint{3.365472in}{0.651159in}}%
\pgfpathlineto{\pgfqpoint{3.367163in}{0.643940in}}%
\pgfpathlineto{\pgfqpoint{3.370545in}{0.643266in}}%
\pgfpathlineto{\pgfqpoint{3.372237in}{0.651353in}}%
\pgfpathlineto{\pgfqpoint{3.373082in}{0.651353in}}%
\pgfpathlineto{\pgfqpoint{3.374774in}{0.641644in}}%
\pgfpathlineto{\pgfqpoint{3.375619in}{0.641644in}}%
\pgfpathlineto{\pgfqpoint{3.377310in}{0.646017in}}%
\pgfpathlineto{\pgfqpoint{3.378156in}{0.646017in}}%
\pgfpathlineto{\pgfqpoint{3.379847in}{0.650127in}}%
\pgfpathlineto{\pgfqpoint{3.382384in}{0.650127in}}%
\pgfpathlineto{\pgfqpoint{3.384075in}{0.640461in}}%
\pgfpathlineto{\pgfqpoint{3.385767in}{0.648758in}}%
\pgfpathlineto{\pgfqpoint{3.389995in}{0.647798in}}%
\pgfpathlineto{\pgfqpoint{3.391686in}{0.640711in}}%
\pgfpathlineto{\pgfqpoint{3.393377in}{0.650766in}}%
\pgfpathlineto{\pgfqpoint{3.394223in}{0.650766in}}%
\pgfpathlineto{\pgfqpoint{3.395914in}{0.644968in}}%
\pgfpathlineto{\pgfqpoint{3.397605in}{0.644968in}}%
\pgfpathlineto{\pgfqpoint{3.399297in}{0.642339in}}%
\pgfpathlineto{\pgfqpoint{3.400988in}{0.651511in}}%
\pgfpathlineto{\pgfqpoint{3.401834in}{0.651511in}}%
\pgfpathlineto{\pgfqpoint{3.403525in}{0.642374in}}%
\pgfpathlineto{\pgfqpoint{3.405216in}{0.642374in}}%
\pgfpathlineto{\pgfqpoint{3.406907in}{0.644923in}}%
\pgfpathlineto{\pgfqpoint{3.408599in}{0.644923in}}%
\pgfpathlineto{\pgfqpoint{3.410290in}{0.650798in}}%
\pgfpathlineto{\pgfqpoint{3.411981in}{0.650798in}}%
\pgfpathlineto{\pgfqpoint{3.413672in}{0.640697in}}%
\pgfpathlineto{\pgfqpoint{3.415364in}{0.640697in}}%
\pgfpathlineto{\pgfqpoint{3.417055in}{0.647785in}}%
\pgfpathlineto{\pgfqpoint{3.421283in}{0.648810in}}%
\pgfpathlineto{\pgfqpoint{3.422974in}{0.640385in}}%
\pgfpathlineto{\pgfqpoint{3.424665in}{0.650171in}}%
\pgfpathlineto{\pgfqpoint{3.425511in}{0.650171in}}%
\pgfpathlineto{\pgfqpoint{3.427202in}{0.646061in}}%
\pgfpathlineto{\pgfqpoint{3.428894in}{0.646061in}}%
\pgfpathlineto{\pgfqpoint{3.430585in}{0.641530in}}%
\pgfpathlineto{\pgfqpoint{3.432276in}{0.651452in}}%
\pgfpathlineto{\pgfqpoint{3.433122in}{0.651452in}}%
\pgfpathlineto{\pgfqpoint{3.434813in}{0.643264in}}%
\pgfpathlineto{\pgfqpoint{3.439887in}{0.643840in}}%
\pgfpathlineto{\pgfqpoint{3.441578in}{0.651293in}}%
\pgfpathlineto{\pgfqpoint{3.444960in}{0.651293in}}%
\pgfpathlineto{\pgfqpoint{3.446652in}{0.641148in}}%
\pgfpathlineto{\pgfqpoint{3.447497in}{0.641148in}}%
\pgfpathlineto{\pgfqpoint{3.449188in}{0.646711in}}%
\pgfpathlineto{\pgfqpoint{3.450034in}{0.646711in}}%
\pgfpathlineto{\pgfqpoint{3.451725in}{0.649734in}}%
\pgfpathlineto{\pgfqpoint{3.453417in}{0.649734in}}%
\pgfpathlineto{\pgfqpoint{3.455108in}{0.640273in}}%
\pgfpathlineto{\pgfqpoint{3.456799in}{0.649389in}}%
\pgfpathlineto{\pgfqpoint{3.457645in}{0.649389in}}%
\pgfpathlineto{\pgfqpoint{3.459336in}{0.647178in}}%
\pgfpathlineto{\pgfqpoint{3.461027in}{0.647178in}}%
\pgfpathlineto{\pgfqpoint{3.462718in}{0.640878in}}%
\pgfpathlineto{\pgfqpoint{3.464410in}{0.651168in}}%
\pgfpathlineto{\pgfqpoint{3.465255in}{0.651168in}}%
\pgfpathlineto{\pgfqpoint{3.466947in}{0.644284in}}%
\pgfpathlineto{\pgfqpoint{3.470329in}{0.642815in}}%
\pgfpathlineto{\pgfqpoint{3.472020in}{0.651584in}}%
\pgfpathlineto{\pgfqpoint{3.472866in}{0.651584in}}%
\pgfpathlineto{\pgfqpoint{3.474557in}{0.641805in}}%
\pgfpathlineto{\pgfqpoint{3.476248in}{0.641805in}}%
\pgfpathlineto{\pgfqpoint{3.477940in}{0.645579in}}%
\pgfpathlineto{\pgfqpoint{3.479631in}{0.645579in}}%
\pgfpathlineto{\pgfqpoint{3.481322in}{0.650527in}}%
\pgfpathlineto{\pgfqpoint{3.483859in}{0.650527in}}%
\pgfpathlineto{\pgfqpoint{3.485550in}{0.640391in}}%
\pgfpathlineto{\pgfqpoint{3.486396in}{0.640391in}}%
\pgfpathlineto{\pgfqpoint{3.488087in}{0.648446in}}%
\pgfpathlineto{\pgfqpoint{3.492315in}{0.648269in}}%
\pgfpathlineto{\pgfqpoint{3.494007in}{0.640421in}}%
\pgfpathlineto{\pgfqpoint{3.495698in}{0.650662in}}%
\pgfpathlineto{\pgfqpoint{3.496543in}{0.650662in}}%
\pgfpathlineto{\pgfqpoint{3.498235in}{0.645394in}}%
\pgfpathlineto{\pgfqpoint{3.499926in}{0.645394in}}%
\pgfpathlineto{\pgfqpoint{3.501617in}{0.641896in}}%
\pgfpathlineto{\pgfqpoint{3.503308in}{0.651648in}}%
\pgfpathlineto{\pgfqpoint{3.504154in}{0.651648in}}%
\pgfpathlineto{\pgfqpoint{3.505845in}{0.642647in}}%
\pgfpathlineto{\pgfqpoint{3.507537in}{0.642647in}}%
\pgfpathlineto{\pgfqpoint{3.510073in}{0.644436in}}%
\pgfpathlineto{\pgfqpoint{3.510919in}{0.644436in}}%
\pgfpathlineto{\pgfqpoint{3.512610in}{0.651146in}}%
\pgfpathlineto{\pgfqpoint{3.514301in}{0.651146in}}%
\pgfpathlineto{\pgfqpoint{3.515993in}{0.640742in}}%
\pgfpathlineto{\pgfqpoint{3.516838in}{0.640742in}}%
\pgfpathlineto{\pgfqpoint{3.518530in}{0.647378in}}%
\pgfpathlineto{\pgfqpoint{3.519375in}{0.647378in}}%
\pgfpathlineto{\pgfqpoint{3.521066in}{0.649287in}}%
\pgfpathlineto{\pgfqpoint{3.522758in}{0.649287in}}%
\pgfpathlineto{\pgfqpoint{3.524449in}{0.640184in}}%
\pgfpathlineto{\pgfqpoint{3.526140in}{0.649949in}}%
\pgfpathlineto{\pgfqpoint{3.526986in}{0.649949in}}%
\pgfpathlineto{\pgfqpoint{3.528677in}{0.646549in}}%
\pgfpathlineto{\pgfqpoint{3.530368in}{0.646549in}}%
\pgfpathlineto{\pgfqpoint{3.532060in}{0.641128in}}%
\pgfpathlineto{\pgfqpoint{3.533751in}{0.651475in}}%
\pgfpathlineto{\pgfqpoint{3.534596in}{0.651475in}}%
\pgfpathlineto{\pgfqpoint{3.536288in}{0.643641in}}%
\pgfpathlineto{\pgfqpoint{3.539670in}{0.643335in}}%
\pgfpathlineto{\pgfqpoint{3.541361in}{0.651560in}}%
\pgfpathlineto{\pgfqpoint{3.542207in}{0.651560in}}%
\pgfpathlineto{\pgfqpoint{3.543898in}{0.641317in}}%
\pgfpathlineto{\pgfqpoint{3.544744in}{0.641317in}}%
\pgfpathlineto{\pgfqpoint{3.546435in}{0.646229in}}%
\pgfpathlineto{\pgfqpoint{3.547281in}{0.646229in}}%
\pgfpathlineto{\pgfqpoint{3.548972in}{0.650181in}}%
\pgfpathlineto{\pgfqpoint{3.550663in}{0.650181in}}%
\pgfpathlineto{\pgfqpoint{3.552355in}{0.640186in}}%
\pgfpathlineto{\pgfqpoint{3.554046in}{0.649052in}}%
\pgfpathlineto{\pgfqpoint{3.558274in}{0.647697in}}%
\pgfpathlineto{\pgfqpoint{3.559965in}{0.640552in}}%
\pgfpathlineto{\pgfqpoint{3.561656in}{0.651065in}}%
\pgfpathlineto{\pgfqpoint{3.562502in}{0.651065in}}%
\pgfpathlineto{\pgfqpoint{3.564193in}{0.644747in}}%
\pgfpathlineto{\pgfqpoint{3.565885in}{0.644747in}}%
\pgfpathlineto{\pgfqpoint{3.567576in}{0.642327in}}%
\pgfpathlineto{\pgfqpoint{3.569267in}{0.651742in}}%
\pgfpathlineto{\pgfqpoint{3.570113in}{0.651742in}}%
\pgfpathlineto{\pgfqpoint{3.571804in}{0.642095in}}%
\pgfpathlineto{\pgfqpoint{3.573495in}{0.642095in}}%
\pgfpathlineto{\pgfqpoint{3.575186in}{0.645051in}}%
\pgfpathlineto{\pgfqpoint{3.576878in}{0.645051in}}%
\pgfpathlineto{\pgfqpoint{3.578569in}{0.650910in}}%
\pgfpathlineto{\pgfqpoint{3.579415in}{0.650910in}}%
\pgfpathlineto{\pgfqpoint{3.581106in}{0.640431in}}%
\pgfpathlineto{\pgfqpoint{3.581951in}{0.640431in}}%
\pgfpathlineto{\pgfqpoint{3.583643in}{0.648011in}}%
\pgfpathlineto{\pgfqpoint{3.587871in}{0.648785in}}%
\pgfpathlineto{\pgfqpoint{3.589562in}{0.640197in}}%
\pgfpathlineto{\pgfqpoint{3.591253in}{0.650431in}}%
\pgfpathlineto{\pgfqpoint{3.592099in}{0.650431in}}%
\pgfpathlineto{\pgfqpoint{3.593790in}{0.645916in}}%
\pgfpathlineto{\pgfqpoint{3.595481in}{0.645916in}}%
\pgfpathlineto{\pgfqpoint{3.597173in}{0.641460in}}%
\pgfpathlineto{\pgfqpoint{3.598864in}{0.651681in}}%
\pgfpathlineto{\pgfqpoint{3.599709in}{0.651681in}}%
\pgfpathlineto{\pgfqpoint{3.601401in}{0.643044in}}%
\pgfpathlineto{\pgfqpoint{3.606474in}{0.643897in}}%
\pgfpathlineto{\pgfqpoint{3.608166in}{0.651436in}}%
\pgfpathlineto{\pgfqpoint{3.610703in}{0.651436in}}%
\pgfpathlineto{\pgfqpoint{3.612394in}{0.640913in}}%
\pgfpathlineto{\pgfqpoint{3.613239in}{0.640913in}}%
\pgfpathlineto{\pgfqpoint{3.614931in}{0.646869in}}%
\pgfpathlineto{\pgfqpoint{3.615776in}{0.646869in}}%
\pgfpathlineto{\pgfqpoint{3.617468in}{0.649764in}}%
\pgfpathlineto{\pgfqpoint{3.619159in}{0.649764in}}%
\pgfpathlineto{\pgfqpoint{3.620850in}{0.640083in}}%
\pgfpathlineto{\pgfqpoint{3.622541in}{0.649600in}}%
\pgfpathlineto{\pgfqpoint{3.623387in}{0.649600in}}%
\pgfpathlineto{\pgfqpoint{3.625078in}{0.647097in}}%
\pgfpathlineto{\pgfqpoint{3.626769in}{0.647097in}}%
\pgfpathlineto{\pgfqpoint{3.628461in}{0.640777in}}%
\pgfpathlineto{\pgfqpoint{3.630152in}{0.651374in}}%
\pgfpathlineto{\pgfqpoint{3.630998in}{0.651374in}}%
\pgfpathlineto{\pgfqpoint{3.632689in}{0.644123in}}%
\pgfpathlineto{\pgfqpoint{3.636071in}{0.642820in}}%
\pgfpathlineto{\pgfqpoint{3.637763in}{0.651733in}}%
\pgfpathlineto{\pgfqpoint{3.638608in}{0.651733in}}%
\pgfpathlineto{\pgfqpoint{3.640299in}{0.641612in}}%
\pgfpathlineto{\pgfqpoint{3.641991in}{0.641612in}}%
\pgfpathlineto{\pgfqpoint{3.643682in}{0.645679in}}%
\pgfpathlineto{\pgfqpoint{3.645373in}{0.645679in}}%
\pgfpathlineto{\pgfqpoint{3.647064in}{0.650587in}}%
\pgfpathlineto{\pgfqpoint{3.649601in}{0.650587in}}%
\pgfpathlineto{\pgfqpoint{3.651293in}{0.640219in}}%
\pgfpathlineto{\pgfqpoint{3.652138in}{0.640219in}}%
\pgfpathlineto{\pgfqpoint{3.653829in}{0.648605in}}%
\pgfpathlineto{\pgfqpoint{3.658058in}{0.648235in}}%
\pgfpathlineto{\pgfqpoint{3.659749in}{0.640312in}}%
\pgfpathlineto{\pgfqpoint{3.661440in}{0.650833in}}%
\pgfpathlineto{\pgfqpoint{3.662286in}{0.650833in}}%
\pgfpathlineto{\pgfqpoint{3.663977in}{0.645284in}}%
\pgfpathlineto{\pgfqpoint{3.665668in}{0.645284in}}%
\pgfpathlineto{\pgfqpoint{3.667359in}{0.641872in}}%
\pgfpathlineto{\pgfqpoint{3.669051in}{0.651784in}}%
\pgfpathlineto{\pgfqpoint{3.669896in}{0.651784in}}%
\pgfpathlineto{\pgfqpoint{3.671587in}{0.642498in}}%
\pgfpathlineto{\pgfqpoint{3.673279in}{0.642498in}}%
\pgfpathlineto{\pgfqpoint{3.674970in}{0.644495in}}%
\pgfpathlineto{\pgfqpoint{3.676661in}{0.644495in}}%
\pgfpathlineto{\pgfqpoint{3.678352in}{0.651216in}}%
\pgfpathlineto{\pgfqpoint{3.680044in}{0.651216in}}%
\pgfpathlineto{\pgfqpoint{3.681735in}{0.640598in}}%
\pgfpathlineto{\pgfqpoint{3.682581in}{0.640598in}}%
\pgfpathlineto{\pgfqpoint{3.684272in}{0.647494in}}%
\pgfpathlineto{\pgfqpoint{3.685117in}{0.647494in}}%
\pgfpathlineto{\pgfqpoint{3.687654in}{0.649278in}}%
\pgfpathlineto{\pgfqpoint{3.688500in}{0.649278in}}%
\pgfpathlineto{\pgfqpoint{3.690191in}{0.640084in}}%
\pgfpathlineto{\pgfqpoint{3.691882in}{0.650083in}}%
\pgfpathlineto{\pgfqpoint{3.692728in}{0.650083in}}%
\pgfpathlineto{\pgfqpoint{3.694419in}{0.646474in}}%
\pgfpathlineto{\pgfqpoint{3.696111in}{0.646474in}}%
\pgfpathlineto{\pgfqpoint{3.697802in}{0.641095in}}%
\pgfpathlineto{\pgfqpoint{3.699493in}{0.651588in}}%
\pgfpathlineto{\pgfqpoint{3.700339in}{0.651588in}}%
\pgfpathlineto{\pgfqpoint{3.702030in}{0.643528in}}%
\pgfpathlineto{\pgfqpoint{3.705412in}{0.643371in}}%
\pgfpathlineto{\pgfqpoint{3.707104in}{0.651622in}}%
\pgfpathlineto{\pgfqpoint{3.707949in}{0.651622in}}%
\pgfpathlineto{\pgfqpoint{3.709641in}{0.641204in}}%
\pgfpathlineto{\pgfqpoint{3.710486in}{0.641204in}}%
\pgfpathlineto{\pgfqpoint{3.712177in}{0.646315in}}%
\pgfpathlineto{\pgfqpoint{3.713023in}{0.646315in}}%
\pgfpathlineto{\pgfqpoint{3.714714in}{0.650180in}}%
\pgfpathlineto{\pgfqpoint{3.716406in}{0.650180in}}%
\pgfpathlineto{\pgfqpoint{3.718097in}{0.640106in}}%
\pgfpathlineto{\pgfqpoint{3.719788in}{0.649157in}}%
\pgfpathlineto{\pgfqpoint{3.724016in}{0.647640in}}%
\pgfpathlineto{\pgfqpoint{3.725707in}{0.640526in}}%
\pgfpathlineto{\pgfqpoint{3.727399in}{0.651153in}}%
\pgfpathlineto{\pgfqpoint{3.728244in}{0.651153in}}%
\pgfpathlineto{\pgfqpoint{3.729936in}{0.644658in}}%
\pgfpathlineto{\pgfqpoint{3.731627in}{0.644658in}}%
\pgfpathlineto{\pgfqpoint{3.733318in}{0.642358in}}%
\pgfpathlineto{\pgfqpoint{3.735009in}{0.651787in}}%
\pgfpathlineto{\pgfqpoint{3.735855in}{0.651787in}}%
\pgfpathlineto{\pgfqpoint{3.737546in}{0.642006in}}%
\pgfpathlineto{\pgfqpoint{3.739237in}{0.642006in}}%
\pgfpathlineto{\pgfqpoint{3.740929in}{0.645124in}}%
\pgfpathlineto{\pgfqpoint{3.742620in}{0.645124in}}%
\pgfpathlineto{\pgfqpoint{3.744311in}{0.650901in}}%
\pgfpathlineto{\pgfqpoint{3.746848in}{0.650901in}}%
\pgfpathlineto{\pgfqpoint{3.748539in}{0.640373in}}%
\pgfpathlineto{\pgfqpoint{3.750231in}{0.640373in}}%
\pgfpathlineto{\pgfqpoint{3.751922in}{0.648097in}}%
\pgfpathlineto{\pgfqpoint{3.756150in}{0.648729in}}%
\pgfpathlineto{\pgfqpoint{3.757841in}{0.640187in}}%
\pgfpathlineto{\pgfqpoint{3.759532in}{0.650500in}}%
\pgfpathlineto{\pgfqpoint{3.760378in}{0.650500in}}%
\pgfpathlineto{\pgfqpoint{3.762069in}{0.645835in}}%
\pgfpathlineto{\pgfqpoint{3.763760in}{0.645835in}}%
\pgfpathlineto{\pgfqpoint{3.765452in}{0.641501in}}%
\pgfpathlineto{\pgfqpoint{3.767143in}{0.651707in}}%
\pgfpathlineto{\pgfqpoint{3.767989in}{0.651707in}}%
\pgfpathlineto{\pgfqpoint{3.769680in}{0.642968in}}%
\pgfpathlineto{\pgfqpoint{3.774754in}{0.643972in}}%
\pgfpathlineto{\pgfqpoint{3.776445in}{0.651412in}}%
\pgfpathlineto{\pgfqpoint{3.778982in}{0.651412in}}%
\pgfpathlineto{\pgfqpoint{3.780673in}{0.640870in}}%
\pgfpathlineto{\pgfqpoint{3.781519in}{0.640870in}}%
\pgfpathlineto{\pgfqpoint{3.783210in}{0.646953in}}%
\pgfpathlineto{\pgfqpoint{3.784055in}{0.646953in}}%
\pgfpathlineto{\pgfqpoint{3.785747in}{0.649696in}}%
\pgfpathlineto{\pgfqpoint{3.787438in}{0.649696in}}%
\pgfpathlineto{\pgfqpoint{3.789129in}{0.640090in}}%
\pgfpathlineto{\pgfqpoint{3.790820in}{0.649660in}}%
\pgfpathlineto{\pgfqpoint{3.791666in}{0.649660in}}%
\pgfpathlineto{\pgfqpoint{3.793357in}{0.647008in}}%
\pgfpathlineto{\pgfqpoint{3.795049in}{0.647008in}}%
\pgfpathlineto{\pgfqpoint{3.796740in}{0.640835in}}%
\pgfpathlineto{\pgfqpoint{3.798431in}{0.651388in}}%
\pgfpathlineto{\pgfqpoint{3.799277in}{0.651388in}}%
\pgfpathlineto{\pgfqpoint{3.800968in}{0.644044in}}%
\pgfpathlineto{\pgfqpoint{3.804350in}{0.642911in}}%
\pgfpathlineto{\pgfqpoint{3.806042in}{0.651692in}}%
\pgfpathlineto{\pgfqpoint{3.806887in}{0.651692in}}%
\pgfpathlineto{\pgfqpoint{3.808579in}{0.641572in}}%
\pgfpathlineto{\pgfqpoint{3.810270in}{0.641572in}}%
\pgfpathlineto{\pgfqpoint{3.811961in}{0.645775in}}%
\pgfpathlineto{\pgfqpoint{3.813652in}{0.645775in}}%
\pgfpathlineto{\pgfqpoint{3.815344in}{0.650500in}}%
\pgfpathlineto{\pgfqpoint{3.817880in}{0.650500in}}%
\pgfpathlineto{\pgfqpoint{3.819572in}{0.640236in}}%
\pgfpathlineto{\pgfqpoint{3.820417in}{0.640236in}}%
\pgfpathlineto{\pgfqpoint{3.822109in}{0.648673in}}%
\pgfpathlineto{\pgfqpoint{3.827182in}{0.648126in}}%
\pgfpathlineto{\pgfqpoint{3.828874in}{0.640386in}}%
\pgfpathlineto{\pgfqpoint{3.830565in}{0.650847in}}%
\pgfpathlineto{\pgfqpoint{3.831410in}{0.650847in}}%
\pgfpathlineto{\pgfqpoint{3.833102in}{0.645188in}}%
\pgfpathlineto{\pgfqpoint{3.834793in}{0.645188in}}%
\pgfpathlineto{\pgfqpoint{3.836484in}{0.641986in}}%
\pgfpathlineto{\pgfqpoint{3.838175in}{0.651734in}}%
\pgfpathlineto{\pgfqpoint{3.839021in}{0.651734in}}%
\pgfpathlineto{\pgfqpoint{3.840712in}{0.642446in}}%
\pgfpathlineto{\pgfqpoint{3.842403in}{0.642446in}}%
\pgfpathlineto{\pgfqpoint{3.844095in}{0.644616in}}%
\pgfpathlineto{\pgfqpoint{3.845786in}{0.644616in}}%
\pgfpathlineto{\pgfqpoint{3.847477in}{0.651109in}}%
\pgfpathlineto{\pgfqpoint{3.849168in}{0.651109in}}%
\pgfpathlineto{\pgfqpoint{3.850860in}{0.640613in}}%
\pgfpathlineto{\pgfqpoint{3.851705in}{0.640613in}}%
\pgfpathlineto{\pgfqpoint{3.853397in}{0.647584in}}%
\pgfpathlineto{\pgfqpoint{3.854242in}{0.647584in}}%
\pgfpathlineto{\pgfqpoint{3.856779in}{0.649143in}}%
\pgfpathlineto{\pgfqpoint{3.857625in}{0.649143in}}%
\pgfpathlineto{\pgfqpoint{3.859316in}{0.640169in}}%
\pgfpathlineto{\pgfqpoint{3.861007in}{0.650112in}}%
\pgfpathlineto{\pgfqpoint{3.861853in}{0.650112in}}%
\pgfpathlineto{\pgfqpoint{3.863544in}{0.646348in}}%
\pgfpathlineto{\pgfqpoint{3.865235in}{0.646348in}}%
\pgfpathlineto{\pgfqpoint{3.866927in}{0.641231in}}%
\pgfpathlineto{\pgfqpoint{3.868618in}{0.651540in}}%
\pgfpathlineto{\pgfqpoint{3.869463in}{0.651540in}}%
\pgfpathlineto{\pgfqpoint{3.871155in}{0.643450in}}%
\pgfpathlineto{\pgfqpoint{3.874537in}{0.643523in}}%
\pgfpathlineto{\pgfqpoint{3.876228in}{0.651503in}}%
\pgfpathlineto{\pgfqpoint{3.877074in}{0.651503in}}%
\pgfpathlineto{\pgfqpoint{3.878765in}{0.641201in}}%
\pgfpathlineto{\pgfqpoint{3.879611in}{0.641201in}}%
\pgfpathlineto{\pgfqpoint{3.881302in}{0.646440in}}%
\pgfpathlineto{\pgfqpoint{3.882148in}{0.646440in}}%
\pgfpathlineto{\pgfqpoint{3.883839in}{0.650019in}}%
\pgfpathlineto{\pgfqpoint{3.885530in}{0.650019in}}%
\pgfpathlineto{\pgfqpoint{3.887222in}{0.640187in}}%
\pgfpathlineto{\pgfqpoint{3.888913in}{0.649216in}}%
\pgfpathlineto{\pgfqpoint{3.889758in}{0.649216in}}%
\pgfpathlineto{\pgfqpoint{3.892295in}{0.647478in}}%
\pgfpathlineto{\pgfqpoint{3.893141in}{0.647478in}}%
\pgfpathlineto{\pgfqpoint{3.894832in}{0.640676in}}%
\pgfpathlineto{\pgfqpoint{3.896523in}{0.651124in}}%
\pgfpathlineto{\pgfqpoint{3.897369in}{0.651124in}}%
\pgfpathlineto{\pgfqpoint{3.899060in}{0.644540in}}%
\pgfpathlineto{\pgfqpoint{3.900752in}{0.644540in}}%
\pgfpathlineto{\pgfqpoint{3.902443in}{0.642541in}}%
\pgfpathlineto{\pgfqpoint{3.904134in}{0.651670in}}%
\pgfpathlineto{\pgfqpoint{3.904980in}{0.651670in}}%
\pgfpathlineto{\pgfqpoint{3.906671in}{0.641969in}}%
\pgfpathlineto{\pgfqpoint{3.908362in}{0.641969in}}%
\pgfpathlineto{\pgfqpoint{3.910053in}{0.645291in}}%
\pgfpathlineto{\pgfqpoint{3.911745in}{0.645291in}}%
\pgfpathlineto{\pgfqpoint{3.913436in}{0.650722in}}%
\pgfpathlineto{\pgfqpoint{3.915973in}{0.650722in}}%
\pgfpathlineto{\pgfqpoint{3.917664in}{0.640434in}}%
\pgfpathlineto{\pgfqpoint{3.918510in}{0.640434in}}%
\pgfpathlineto{\pgfqpoint{3.920201in}{0.648201in}}%
\pgfpathlineto{\pgfqpoint{3.924429in}{0.648531in}}%
\pgfpathlineto{\pgfqpoint{3.926120in}{0.640337in}}%
\pgfpathlineto{\pgfqpoint{3.927811in}{0.650507in}}%
\pgfpathlineto{\pgfqpoint{3.928657in}{0.650507in}}%
\pgfpathlineto{\pgfqpoint{3.930348in}{0.645671in}}%
\pgfpathlineto{\pgfqpoint{3.932040in}{0.645671in}}%
\pgfpathlineto{\pgfqpoint{3.933731in}{0.641707in}}%
\pgfpathlineto{\pgfqpoint{3.935422in}{0.651609in}}%
\pgfpathlineto{\pgfqpoint{3.936268in}{0.651609in}}%
\pgfpathlineto{\pgfqpoint{3.937959in}{0.642883in}}%
\pgfpathlineto{\pgfqpoint{3.943033in}{0.644181in}}%
\pgfpathlineto{\pgfqpoint{3.944724in}{0.651228in}}%
\pgfpathlineto{\pgfqpoint{3.947261in}{0.651228in}}%
\pgfpathlineto{\pgfqpoint{3.948952in}{0.640894in}}%
\pgfpathlineto{\pgfqpoint{3.949798in}{0.640894in}}%
\pgfpathlineto{\pgfqpoint{3.951489in}{0.647109in}}%
\pgfpathlineto{\pgfqpoint{3.952335in}{0.647109in}}%
\pgfpathlineto{\pgfqpoint{3.954026in}{0.649469in}}%
\pgfpathlineto{\pgfqpoint{3.955717in}{0.649469in}}%
\pgfpathlineto{\pgfqpoint{3.957408in}{0.640223in}}%
\pgfpathlineto{\pgfqpoint{3.959100in}{0.649718in}}%
\pgfpathlineto{\pgfqpoint{3.959945in}{0.649718in}}%
\pgfpathlineto{\pgfqpoint{3.961636in}{0.646796in}}%
\pgfpathlineto{\pgfqpoint{3.963328in}{0.646796in}}%
\pgfpathlineto{\pgfqpoint{3.965019in}{0.641049in}}%
\pgfpathlineto{\pgfqpoint{3.966710in}{0.651327in}}%
\pgfpathlineto{\pgfqpoint{3.967556in}{0.651327in}}%
\pgfpathlineto{\pgfqpoint{3.969247in}{0.643903in}}%
\pgfpathlineto{\pgfqpoint{3.972630in}{0.643155in}}%
\pgfpathlineto{\pgfqpoint{3.974321in}{0.651522in}}%
\pgfpathlineto{\pgfqpoint{3.975166in}{0.651522in}}%
\pgfpathlineto{\pgfqpoint{3.976858in}{0.641543in}}%
\pgfpathlineto{\pgfqpoint{3.978549in}{0.641543in}}%
\pgfpathlineto{\pgfqpoint{3.980240in}{0.645986in}}%
\pgfpathlineto{\pgfqpoint{3.981931in}{0.645986in}}%
\pgfpathlineto{\pgfqpoint{3.983623in}{0.650258in}}%
\pgfpathlineto{\pgfqpoint{3.986160in}{0.650258in}}%
\pgfpathlineto{\pgfqpoint{3.987851in}{0.640333in}}%
\pgfpathlineto{\pgfqpoint{3.988696in}{0.640333in}}%
\pgfpathlineto{\pgfqpoint{3.990388in}{0.648793in}}%
\pgfpathlineto{\pgfqpoint{3.995461in}{0.647873in}}%
\pgfpathlineto{\pgfqpoint{3.997153in}{0.640590in}}%
\pgfpathlineto{\pgfqpoint{3.998844in}{0.650841in}}%
\pgfpathlineto{\pgfqpoint{3.999689in}{0.650841in}}%
\pgfpathlineto{\pgfqpoint{4.001381in}{0.644987in}}%
\pgfpathlineto{\pgfqpoint{4.003072in}{0.644987in}}%
\pgfpathlineto{\pgfqpoint{4.004763in}{0.642250in}}%
\pgfpathlineto{\pgfqpoint{4.006454in}{0.651597in}}%
\pgfpathlineto{\pgfqpoint{4.007300in}{0.651597in}}%
\pgfpathlineto{\pgfqpoint{4.008991in}{0.642351in}}%
\pgfpathlineto{\pgfqpoint{4.010683in}{0.642351in}}%
\pgfpathlineto{\pgfqpoint{4.012374in}{0.644875in}}%
\pgfpathlineto{\pgfqpoint{4.014065in}{0.644875in}}%
\pgfpathlineto{\pgfqpoint{4.015756in}{0.650872in}}%
\pgfpathlineto{\pgfqpoint{4.017448in}{0.650872in}}%
\pgfpathlineto{\pgfqpoint{4.019139in}{0.640655in}}%
\pgfpathlineto{\pgfqpoint{4.019984in}{0.640655in}}%
\pgfpathlineto{\pgfqpoint{4.021676in}{0.647770in}}%
\pgfpathlineto{\pgfqpoint{4.025904in}{0.648862in}}%
\pgfpathlineto{\pgfqpoint{4.027595in}{0.640342in}}%
\pgfpathlineto{\pgfqpoint{4.029286in}{0.650173in}}%
\pgfpathlineto{\pgfqpoint{4.030132in}{0.650173in}}%
\pgfpathlineto{\pgfqpoint{4.031823in}{0.646092in}}%
\pgfpathlineto{\pgfqpoint{4.033514in}{0.646092in}}%
\pgfpathlineto{\pgfqpoint{4.035206in}{0.641498in}}%
\pgfpathlineto{\pgfqpoint{4.036897in}{0.651457in}}%
\pgfpathlineto{\pgfqpoint{4.037743in}{0.651457in}}%
\pgfpathlineto{\pgfqpoint{4.039434in}{0.643285in}}%
\pgfpathlineto{\pgfqpoint{4.042816in}{0.643817in}}%
\pgfpathlineto{\pgfqpoint{4.044508in}{0.651292in}}%
\pgfpathlineto{\pgfqpoint{4.045353in}{0.651292in}}%
\pgfpathlineto{\pgfqpoint{4.047044in}{0.641173in}}%
\pgfpathlineto{\pgfqpoint{4.047890in}{0.641173in}}%
\pgfpathlineto{\pgfqpoint{4.049581in}{0.646688in}}%
\pgfpathlineto{\pgfqpoint{4.050427in}{0.646688in}}%
\pgfpathlineto{\pgfqpoint{4.052118in}{0.649729in}}%
\pgfpathlineto{\pgfqpoint{4.053809in}{0.649729in}}%
\pgfpathlineto{\pgfqpoint{4.055501in}{0.640309in}}%
\pgfpathlineto{\pgfqpoint{4.057192in}{0.649352in}}%
\pgfpathlineto{\pgfqpoint{4.058038in}{0.649352in}}%
\pgfpathlineto{\pgfqpoint{4.059729in}{0.647179in}}%
\pgfpathlineto{\pgfqpoint{4.061420in}{0.647179in}}%
\pgfpathlineto{\pgfqpoint{4.063111in}{0.640922in}}%
\pgfpathlineto{\pgfqpoint{4.064803in}{0.651110in}}%
\pgfpathlineto{\pgfqpoint{4.065648in}{0.651110in}}%
\pgfpathlineto{\pgfqpoint{4.067339in}{0.644308in}}%
\pgfpathlineto{\pgfqpoint{4.070722in}{0.642852in}}%
\pgfpathlineto{\pgfqpoint{4.072413in}{0.651507in}}%
\pgfpathlineto{\pgfqpoint{4.073259in}{0.651507in}}%
\pgfpathlineto{\pgfqpoint{4.074950in}{0.641864in}}%
\pgfpathlineto{\pgfqpoint{4.076641in}{0.641864in}}%
\pgfpathlineto{\pgfqpoint{4.078332in}{0.645590in}}%
\pgfpathlineto{\pgfqpoint{4.080024in}{0.645590in}}%
\pgfpathlineto{\pgfqpoint{4.081715in}{0.650445in}}%
\pgfpathlineto{\pgfqpoint{4.084252in}{0.650445in}}%
\pgfpathlineto{\pgfqpoint{4.085943in}{0.640487in}}%
\pgfpathlineto{\pgfqpoint{4.086789in}{0.640487in}}%
\pgfpathlineto{\pgfqpoint{4.088480in}{0.648411in}}%
\pgfpathlineto{\pgfqpoint{4.092708in}{0.648208in}}%
\pgfpathlineto{\pgfqpoint{4.094399in}{0.640540in}}%
\pgfpathlineto{\pgfqpoint{4.096091in}{0.650574in}}%
\pgfpathlineto{\pgfqpoint{4.096936in}{0.650574in}}%
\pgfpathlineto{\pgfqpoint{4.098627in}{0.645380in}}%
\pgfpathlineto{\pgfqpoint{4.100319in}{0.645380in}}%
\pgfpathlineto{\pgfqpoint{4.102010in}{0.642011in}}%
\pgfpathlineto{\pgfqpoint{4.103701in}{0.651512in}}%
\pgfpathlineto{\pgfqpoint{4.104547in}{0.651512in}}%
\pgfpathlineto{\pgfqpoint{4.106238in}{0.642699in}}%
\pgfpathlineto{\pgfqpoint{4.107929in}{0.642699in}}%
\pgfpathlineto{\pgfqpoint{4.110466in}{0.644514in}}%
\pgfpathlineto{\pgfqpoint{4.111312in}{0.644514in}}%
\pgfpathlineto{\pgfqpoint{4.113003in}{0.650988in}}%
\pgfpathlineto{\pgfqpoint{4.114694in}{0.650988in}}%
\pgfpathlineto{\pgfqpoint{4.116386in}{0.640866in}}%
\pgfpathlineto{\pgfqpoint{4.117231in}{0.640866in}}%
\pgfpathlineto{\pgfqpoint{4.118922in}{0.647386in}}%
\pgfpathlineto{\pgfqpoint{4.119768in}{0.647386in}}%
\pgfpathlineto{\pgfqpoint{4.122305in}{0.649143in}}%
\pgfpathlineto{\pgfqpoint{4.123151in}{0.649143in}}%
\pgfpathlineto{\pgfqpoint{4.124842in}{0.640362in}}%
\pgfpathlineto{\pgfqpoint{4.126533in}{0.649868in}}%
\pgfpathlineto{\pgfqpoint{4.127379in}{0.649868in}}%
\pgfpathlineto{\pgfqpoint{4.129070in}{0.646463in}}%
\pgfpathlineto{\pgfqpoint{4.130761in}{0.646463in}}%
\pgfpathlineto{\pgfqpoint{4.132452in}{0.641324in}}%
\pgfpathlineto{\pgfqpoint{4.134144in}{0.651311in}}%
\pgfpathlineto{\pgfqpoint{4.134989in}{0.651311in}}%
\pgfpathlineto{\pgfqpoint{4.136681in}{0.643647in}}%
\pgfpathlineto{\pgfqpoint{4.140063in}{0.643499in}}%
\pgfpathlineto{\pgfqpoint{4.141754in}{0.651340in}}%
\pgfpathlineto{\pgfqpoint{4.142600in}{0.651340in}}%
\pgfpathlineto{\pgfqpoint{4.144291in}{0.641429in}}%
\pgfpathlineto{\pgfqpoint{4.145137in}{0.641429in}}%
\pgfpathlineto{\pgfqpoint{4.146828in}{0.646313in}}%
\pgfpathlineto{\pgfqpoint{4.147674in}{0.646313in}}%
\pgfpathlineto{\pgfqpoint{4.149365in}{0.649954in}}%
\pgfpathlineto{\pgfqpoint{4.151902in}{0.649954in}}%
\pgfpathlineto{\pgfqpoint{4.153593in}{0.640392in}}%
\pgfpathlineto{\pgfqpoint{4.155284in}{0.649022in}}%
\pgfpathlineto{\pgfqpoint{4.159512in}{0.647520in}}%
\pgfpathlineto{\pgfqpoint{4.161204in}{0.640812in}}%
\pgfpathlineto{\pgfqpoint{4.162895in}{0.650912in}}%
\pgfpathlineto{\pgfqpoint{4.163740in}{0.650912in}}%
\pgfpathlineto{\pgfqpoint{4.165432in}{0.644672in}}%
\pgfpathlineto{\pgfqpoint{4.167123in}{0.644672in}}%
\pgfpathlineto{\pgfqpoint{4.168814in}{0.642581in}}%
\pgfpathlineto{\pgfqpoint{4.170505in}{0.651491in}}%
\pgfpathlineto{\pgfqpoint{4.171351in}{0.651491in}}%
\pgfpathlineto{\pgfqpoint{4.173042in}{0.642154in}}%
\pgfpathlineto{\pgfqpoint{4.174734in}{0.642154in}}%
\pgfpathlineto{\pgfqpoint{4.176425in}{0.645232in}}%
\pgfpathlineto{\pgfqpoint{4.178116in}{0.645232in}}%
\pgfpathlineto{\pgfqpoint{4.179807in}{0.650614in}}%
\pgfpathlineto{\pgfqpoint{4.181499in}{0.650614in}}%
\pgfpathlineto{\pgfqpoint{4.183190in}{0.640626in}}%
\pgfpathlineto{\pgfqpoint{4.185727in}{0.640626in}}%
\pgfpathlineto{\pgfqpoint{4.187418in}{0.648065in}}%
\pgfpathlineto{\pgfqpoint{4.192492in}{0.648513in}}%
\pgfpathlineto{\pgfqpoint{4.194183in}{0.640491in}}%
\pgfpathlineto{\pgfqpoint{4.195874in}{0.650332in}}%
\pgfpathlineto{\pgfqpoint{4.196720in}{0.650332in}}%
\pgfpathlineto{\pgfqpoint{4.198411in}{0.645739in}}%
\pgfpathlineto{\pgfqpoint{4.200102in}{0.645739in}}%
\pgfpathlineto{\pgfqpoint{4.201794in}{0.641790in}}%
\pgfpathlineto{\pgfqpoint{4.203485in}{0.651438in}}%
\pgfpathlineto{\pgfqpoint{4.204330in}{0.651438in}}%
\pgfpathlineto{\pgfqpoint{4.206022in}{0.643015in}}%
\pgfpathlineto{\pgfqpoint{4.211095in}{0.644181in}}%
\pgfpathlineto{\pgfqpoint{4.212787in}{0.651100in}}%
\pgfpathlineto{\pgfqpoint{4.216169in}{0.651100in}}%
\pgfpathlineto{\pgfqpoint{4.217860in}{0.641054in}}%
\pgfpathlineto{\pgfqpoint{4.218706in}{0.641054in}}%
\pgfpathlineto{\pgfqpoint{4.220397in}{0.647032in}}%
\pgfpathlineto{\pgfqpoint{4.221243in}{0.647032in}}%
\pgfpathlineto{\pgfqpoint{4.222934in}{0.649409in}}%
\pgfpathlineto{\pgfqpoint{4.224625in}{0.649409in}}%
\pgfpathlineto{\pgfqpoint{4.226317in}{0.640373in}}%
\pgfpathlineto{\pgfqpoint{4.228008in}{0.649590in}}%
\pgfpathlineto{\pgfqpoint{4.228853in}{0.649590in}}%
\pgfpathlineto{\pgfqpoint{4.230545in}{0.646811in}}%
\pgfpathlineto{\pgfqpoint{4.232236in}{0.646811in}}%
\pgfpathlineto{\pgfqpoint{4.233927in}{0.641154in}}%
\pgfpathlineto{\pgfqpoint{4.235618in}{0.651183in}}%
\pgfpathlineto{\pgfqpoint{4.236464in}{0.651183in}}%
\pgfpathlineto{\pgfqpoint{4.238155in}{0.643982in}}%
\pgfpathlineto{\pgfqpoint{4.241538in}{0.643196in}}%
\pgfpathlineto{\pgfqpoint{4.243229in}{0.651395in}}%
\pgfpathlineto{\pgfqpoint{4.244075in}{0.651395in}}%
\pgfpathlineto{\pgfqpoint{4.245766in}{0.641661in}}%
\pgfpathlineto{\pgfqpoint{4.247457in}{0.641661in}}%
\pgfpathlineto{\pgfqpoint{4.249148in}{0.645959in}}%
\pgfpathlineto{\pgfqpoint{4.250840in}{0.645959in}}%
\pgfpathlineto{\pgfqpoint{4.252531in}{0.650177in}}%
\pgfpathlineto{\pgfqpoint{4.255068in}{0.650177in}}%
\pgfpathlineto{\pgfqpoint{4.256759in}{0.640459in}}%
\pgfpathlineto{\pgfqpoint{4.257605in}{0.640459in}}%
\pgfpathlineto{\pgfqpoint{4.259296in}{0.648714in}}%
\pgfpathlineto{\pgfqpoint{4.263524in}{0.647849in}}%
\pgfpathlineto{\pgfqpoint{4.265215in}{0.640695in}}%
\pgfpathlineto{\pgfqpoint{4.266907in}{0.650734in}}%
\pgfpathlineto{\pgfqpoint{4.267752in}{0.650734in}}%
\pgfpathlineto{\pgfqpoint{4.269443in}{0.645019in}}%
\pgfpathlineto{\pgfqpoint{4.271135in}{0.645019in}}%
\pgfpathlineto{\pgfqpoint{4.272826in}{0.642312in}}%
\pgfpathlineto{\pgfqpoint{4.274517in}{0.651490in}}%
\pgfpathlineto{\pgfqpoint{4.275363in}{0.651490in}}%
\pgfpathlineto{\pgfqpoint{4.277054in}{0.642426in}}%
\pgfpathlineto{\pgfqpoint{4.278745in}{0.642426in}}%
\pgfpathlineto{\pgfqpoint{4.280437in}{0.644884in}}%
\pgfpathlineto{\pgfqpoint{4.282128in}{0.644884in}}%
\pgfpathlineto{\pgfqpoint{4.283819in}{0.650789in}}%
\pgfpathlineto{\pgfqpoint{4.285510in}{0.650789in}}%
\pgfpathlineto{\pgfqpoint{4.287202in}{0.640748in}}%
\pgfpathlineto{\pgfqpoint{4.288047in}{0.640748in}}%
\pgfpathlineto{\pgfqpoint{4.289738in}{0.647733in}}%
\pgfpathlineto{\pgfqpoint{4.293967in}{0.648818in}}%
\pgfpathlineto{\pgfqpoint{4.295658in}{0.640430in}}%
\pgfpathlineto{\pgfqpoint{4.297349in}{0.650106in}}%
\pgfpathlineto{\pgfqpoint{4.298195in}{0.650106in}}%
\pgfpathlineto{\pgfqpoint{4.299886in}{0.646092in}}%
\pgfpathlineto{\pgfqpoint{4.301577in}{0.646092in}}%
\pgfpathlineto{\pgfqpoint{4.303268in}{0.641561in}}%
\pgfpathlineto{\pgfqpoint{4.304960in}{0.651380in}}%
\pgfpathlineto{\pgfqpoint{4.305805in}{0.651380in}}%
\pgfpathlineto{\pgfqpoint{4.307496in}{0.643321in}}%
\pgfpathlineto{\pgfqpoint{4.310879in}{0.643846in}}%
\pgfpathlineto{\pgfqpoint{4.312570in}{0.651224in}}%
\pgfpathlineto{\pgfqpoint{4.313416in}{0.651224in}}%
\pgfpathlineto{\pgfqpoint{4.315107in}{0.641230in}}%
\pgfpathlineto{\pgfqpoint{4.315953in}{0.641230in}}%
\pgfpathlineto{\pgfqpoint{4.317644in}{0.646682in}}%
\pgfpathlineto{\pgfqpoint{4.318490in}{0.646682in}}%
\pgfpathlineto{\pgfqpoint{4.320181in}{0.649684in}}%
\pgfpathlineto{\pgfqpoint{4.321872in}{0.649684in}}%
\pgfpathlineto{\pgfqpoint{4.323563in}{0.640369in}}%
\pgfpathlineto{\pgfqpoint{4.325255in}{0.649320in}}%
\pgfpathlineto{\pgfqpoint{4.326100in}{0.649320in}}%
\pgfpathlineto{\pgfqpoint{4.327791in}{0.647162in}}%
\pgfpathlineto{\pgfqpoint{4.329483in}{0.647162in}}%
\pgfpathlineto{\pgfqpoint{4.331174in}{0.640971in}}%
\pgfpathlineto{\pgfqpoint{4.332865in}{0.651066in}}%
\pgfpathlineto{\pgfqpoint{4.333711in}{0.651066in}}%
\pgfpathlineto{\pgfqpoint{4.335402in}{0.644316in}}%
\pgfpathlineto{\pgfqpoint{4.338785in}{0.642882in}}%
\pgfpathlineto{\pgfqpoint{4.340476in}{0.651464in}}%
\pgfpathlineto{\pgfqpoint{4.341321in}{0.651464in}}%
\pgfpathlineto{\pgfqpoint{4.343013in}{0.641889in}}%
\pgfpathlineto{\pgfqpoint{4.344704in}{0.641889in}}%
\pgfpathlineto{\pgfqpoint{4.346395in}{0.645598in}}%
\pgfpathlineto{\pgfqpoint{4.348086in}{0.645598in}}%
\pgfpathlineto{\pgfqpoint{4.349778in}{0.650413in}}%
\pgfpathlineto{\pgfqpoint{4.350623in}{0.650413in}}%
\pgfpathlineto{\pgfqpoint{4.352315in}{0.640516in}}%
\pgfpathlineto{\pgfqpoint{4.353160in}{0.640516in}}%
\pgfpathlineto{\pgfqpoint{4.354851in}{0.648404in}}%
\pgfpathlineto{\pgfqpoint{4.359080in}{0.648191in}}%
\pgfpathlineto{\pgfqpoint{4.360771in}{0.640565in}}%
\pgfpathlineto{\pgfqpoint{4.362462in}{0.650558in}}%
\pgfpathlineto{\pgfqpoint{4.363308in}{0.650558in}}%
\pgfpathlineto{\pgfqpoint{4.364999in}{0.645376in}}%
\pgfpathlineto{\pgfqpoint{4.366690in}{0.645376in}}%
\pgfpathlineto{\pgfqpoint{4.368381in}{0.642027in}}%
\pgfpathlineto{\pgfqpoint{4.370073in}{0.651498in}}%
\pgfpathlineto{\pgfqpoint{4.370918in}{0.651498in}}%
\pgfpathlineto{\pgfqpoint{4.372610in}{0.642702in}}%
\pgfpathlineto{\pgfqpoint{4.374301in}{0.642702in}}%
\pgfpathlineto{\pgfqpoint{4.376838in}{0.644521in}}%
\pgfpathlineto{\pgfqpoint{4.377683in}{0.644521in}}%
\pgfpathlineto{\pgfqpoint{4.379375in}{0.650979in}}%
\pgfpathlineto{\pgfqpoint{4.381911in}{0.650979in}}%
\pgfpathlineto{\pgfqpoint{4.383603in}{0.640868in}}%
\pgfpathlineto{\pgfqpoint{4.384448in}{0.640868in}}%
\pgfpathlineto{\pgfqpoint{4.386140in}{0.647389in}}%
\pgfpathlineto{\pgfqpoint{4.386985in}{0.647389in}}%
\pgfpathlineto{\pgfqpoint{4.389522in}{0.649141in}}%
\pgfpathlineto{\pgfqpoint{4.390368in}{0.649141in}}%
\pgfpathlineto{\pgfqpoint{4.392059in}{0.640360in}}%
\pgfpathlineto{\pgfqpoint{4.393750in}{0.649873in}}%
\pgfpathlineto{\pgfqpoint{4.394596in}{0.649873in}}%
\pgfpathlineto{\pgfqpoint{4.396287in}{0.646463in}}%
\pgfpathlineto{\pgfqpoint{4.397978in}{0.646463in}}%
\pgfpathlineto{\pgfqpoint{4.399669in}{0.641317in}}%
\pgfpathlineto{\pgfqpoint{4.401361in}{0.651322in}}%
\pgfpathlineto{\pgfqpoint{4.402206in}{0.651322in}}%
\pgfpathlineto{\pgfqpoint{4.403898in}{0.643641in}}%
\pgfpathlineto{\pgfqpoint{4.407280in}{0.643491in}}%
\pgfpathlineto{\pgfqpoint{4.408971in}{0.651358in}}%
\pgfpathlineto{\pgfqpoint{4.409817in}{0.651358in}}%
\pgfpathlineto{\pgfqpoint{4.411508in}{0.641414in}}%
\pgfpathlineto{\pgfqpoint{4.413199in}{0.641414in}}%
\pgfpathlineto{\pgfqpoint{4.414891in}{0.646311in}}%
\pgfpathlineto{\pgfqpoint{4.416582in}{0.646311in}}%
\pgfpathlineto{\pgfqpoint{4.418273in}{0.649976in}}%
\pgfpathlineto{\pgfqpoint{4.420810in}{0.649976in}}%
\pgfpathlineto{\pgfqpoint{4.422501in}{0.640365in}}%
\pgfpathlineto{\pgfqpoint{4.423347in}{0.640365in}}%
\pgfpathlineto{\pgfqpoint{4.425038in}{0.649033in}}%
\pgfpathlineto{\pgfqpoint{4.429266in}{0.647536in}}%
\pgfpathlineto{\pgfqpoint{4.430958in}{0.640777in}}%
\pgfpathlineto{\pgfqpoint{4.432649in}{0.650940in}}%
\pgfpathlineto{\pgfqpoint{4.433494in}{0.650940in}}%
\pgfpathlineto{\pgfqpoint{4.435186in}{0.644674in}}%
\pgfpathlineto{\pgfqpoint{4.436877in}{0.644674in}}%
\pgfpathlineto{\pgfqpoint{4.438568in}{0.642547in}}%
\pgfpathlineto{\pgfqpoint{4.440259in}{0.651534in}}%
\pgfpathlineto{\pgfqpoint{4.441105in}{0.651534in}}%
\pgfpathlineto{\pgfqpoint{4.442796in}{0.642135in}}%
\pgfpathlineto{\pgfqpoint{4.444488in}{0.642135in}}%
\pgfpathlineto{\pgfqpoint{4.446179in}{0.645212in}}%
\pgfpathlineto{\pgfqpoint{4.447870in}{0.645212in}}%
\pgfpathlineto{\pgfqpoint{4.449561in}{0.650663in}}%
\pgfpathlineto{\pgfqpoint{4.451253in}{0.650663in}}%
\pgfpathlineto{\pgfqpoint{4.452944in}{0.640584in}}%
\pgfpathlineto{\pgfqpoint{4.453789in}{0.640584in}}%
\pgfpathlineto{\pgfqpoint{4.455481in}{0.648068in}}%
\pgfpathlineto{\pgfqpoint{4.459709in}{0.648556in}}%
\pgfpathlineto{\pgfqpoint{4.461400in}{0.640433in}}%
\pgfpathlineto{\pgfqpoint{4.463091in}{0.650363in}}%
\pgfpathlineto{\pgfqpoint{4.463937in}{0.650363in}}%
\pgfpathlineto{\pgfqpoint{4.465628in}{0.645761in}}%
\pgfpathlineto{\pgfqpoint{4.467319in}{0.645761in}}%
\pgfpathlineto{\pgfqpoint{4.469011in}{0.641728in}}%
\pgfpathlineto{\pgfqpoint{4.470702in}{0.651496in}}%
\pgfpathlineto{\pgfqpoint{4.471547in}{0.651496in}}%
\pgfpathlineto{\pgfqpoint{4.473239in}{0.643006in}}%
\pgfpathlineto{\pgfqpoint{4.476621in}{0.644132in}}%
\pgfpathlineto{\pgfqpoint{4.478312in}{0.651174in}}%
\pgfpathlineto{\pgfqpoint{4.479158in}{0.651174in}}%
\pgfpathlineto{\pgfqpoint{4.480849in}{0.641009in}}%
\pgfpathlineto{\pgfqpoint{4.481695in}{0.641009in}}%
\pgfpathlineto{\pgfqpoint{4.483386in}{0.647013in}}%
\pgfpathlineto{\pgfqpoint{4.484232in}{0.647013in}}%
\pgfpathlineto{\pgfqpoint{4.485923in}{0.649482in}}%
\pgfpathlineto{\pgfqpoint{4.487614in}{0.649482in}}%
\pgfpathlineto{\pgfqpoint{4.489306in}{0.640298in}}%
\pgfpathlineto{\pgfqpoint{4.490997in}{0.649611in}}%
\pgfpathlineto{\pgfqpoint{4.491842in}{0.649611in}}%
\pgfpathlineto{\pgfqpoint{4.493534in}{0.646863in}}%
\pgfpathlineto{\pgfqpoint{4.495225in}{0.646863in}}%
\pgfpathlineto{\pgfqpoint{4.496916in}{0.641065in}}%
\pgfpathlineto{\pgfqpoint{4.498607in}{0.651244in}}%
\pgfpathlineto{\pgfqpoint{4.499453in}{0.651244in}}%
\pgfpathlineto{\pgfqpoint{4.501144in}{0.643996in}}%
\pgfpathlineto{\pgfqpoint{4.504527in}{0.643114in}}%
\pgfpathlineto{\pgfqpoint{4.506218in}{0.651486in}}%
\pgfpathlineto{\pgfqpoint{4.507064in}{0.651486in}}%
\pgfpathlineto{\pgfqpoint{4.508755in}{0.641629in}}%
\pgfpathlineto{\pgfqpoint{4.510446in}{0.641629in}}%
\pgfpathlineto{\pgfqpoint{4.512137in}{0.645908in}}%
\pgfpathlineto{\pgfqpoint{4.513829in}{0.645908in}}%
\pgfpathlineto{\pgfqpoint{4.515520in}{0.650278in}}%
\pgfpathlineto{\pgfqpoint{4.518057in}{0.650278in}}%
\pgfpathlineto{\pgfqpoint{4.519748in}{0.640382in}}%
\pgfpathlineto{\pgfqpoint{4.522285in}{0.640382in}}%
\pgfpathlineto{\pgfqpoint{4.523976in}{0.648709in}}%
\pgfpathlineto{\pgfqpoint{4.528204in}{0.647936in}}%
\pgfpathlineto{\pgfqpoint{4.529896in}{0.640588in}}%
\pgfpathlineto{\pgfqpoint{4.531587in}{0.650782in}}%
\pgfpathlineto{\pgfqpoint{4.532432in}{0.650782in}}%
\pgfpathlineto{\pgfqpoint{4.534124in}{0.645068in}}%
\pgfpathlineto{\pgfqpoint{4.535815in}{0.645068in}}%
\pgfpathlineto{\pgfqpoint{4.537506in}{0.642200in}}%
\pgfpathlineto{\pgfqpoint{4.539197in}{0.651585in}}%
\pgfpathlineto{\pgfqpoint{4.540043in}{0.651585in}}%
\pgfpathlineto{\pgfqpoint{4.541734in}{0.642419in}}%
\pgfpathlineto{\pgfqpoint{4.543426in}{0.642419in}}%
\pgfpathlineto{\pgfqpoint{4.545117in}{0.644795in}}%
\pgfpathlineto{\pgfqpoint{4.546808in}{0.644795in}}%
\pgfpathlineto{\pgfqpoint{4.548499in}{0.650911in}}%
\pgfpathlineto{\pgfqpoint{4.551036in}{0.650911in}}%
\pgfpathlineto{\pgfqpoint{4.552727in}{0.640683in}}%
\pgfpathlineto{\pgfqpoint{4.553573in}{0.640683in}}%
\pgfpathlineto{\pgfqpoint{4.555264in}{0.647692in}}%
\pgfpathlineto{\pgfqpoint{4.559492in}{0.648938in}}%
\pgfpathlineto{\pgfqpoint{4.561184in}{0.640318in}}%
\pgfpathlineto{\pgfqpoint{4.562875in}{0.650128in}}%
\pgfpathlineto{\pgfqpoint{4.563720in}{0.650128in}}%
\pgfpathlineto{\pgfqpoint{4.565412in}{0.646180in}}%
\pgfpathlineto{\pgfqpoint{4.567103in}{0.646180in}}%
\pgfpathlineto{\pgfqpoint{4.568794in}{0.641426in}}%
\pgfpathlineto{\pgfqpoint{4.570485in}{0.651463in}}%
\pgfpathlineto{\pgfqpoint{4.571331in}{0.651463in}}%
\pgfpathlineto{\pgfqpoint{4.573022in}{0.643352in}}%
\pgfpathlineto{\pgfqpoint{4.576405in}{0.643721in}}%
\pgfpathlineto{\pgfqpoint{4.578096in}{0.651353in}}%
\pgfpathlineto{\pgfqpoint{4.578942in}{0.651353in}}%
\pgfpathlineto{\pgfqpoint{4.580633in}{0.641192in}}%
\pgfpathlineto{\pgfqpoint{4.582324in}{0.641192in}}%
\pgfpathlineto{\pgfqpoint{4.584015in}{0.646599in}}%
\pgfpathlineto{\pgfqpoint{4.585707in}{0.646599in}}%
\pgfpathlineto{\pgfqpoint{4.587398in}{0.649829in}}%
\pgfpathlineto{\pgfqpoint{4.589935in}{0.649829in}}%
\pgfpathlineto{\pgfqpoint{4.591626in}{0.640267in}}%
\pgfpathlineto{\pgfqpoint{4.592472in}{0.640267in}}%
\pgfpathlineto{\pgfqpoint{4.594163in}{0.649303in}}%
\pgfpathlineto{\pgfqpoint{4.595009in}{0.649303in}}%
\pgfpathlineto{\pgfqpoint{4.596700in}{0.647289in}}%
\pgfpathlineto{\pgfqpoint{4.598391in}{0.647289in}}%
\pgfpathlineto{\pgfqpoint{4.600082in}{0.640826in}}%
\pgfpathlineto{\pgfqpoint{4.601774in}{0.651123in}}%
\pgfpathlineto{\pgfqpoint{4.602619in}{0.651123in}}%
\pgfpathlineto{\pgfqpoint{4.604310in}{0.644391in}}%
\pgfpathlineto{\pgfqpoint{4.606002in}{0.644391in}}%
\pgfpathlineto{\pgfqpoint{4.607693in}{0.642727in}}%
\pgfpathlineto{\pgfqpoint{4.609384in}{0.651586in}}%
\pgfpathlineto{\pgfqpoint{4.610230in}{0.651586in}}%
\pgfpathlineto{\pgfqpoint{4.611921in}{0.641890in}}%
\pgfpathlineto{\pgfqpoint{4.613612in}{0.641890in}}%
\pgfpathlineto{\pgfqpoint{4.615304in}{0.645473in}}%
\pgfpathlineto{\pgfqpoint{4.616995in}{0.645473in}}%
\pgfpathlineto{\pgfqpoint{4.618686in}{0.650573in}}%
\pgfpathlineto{\pgfqpoint{4.620377in}{0.650573in}}%
\pgfpathlineto{\pgfqpoint{4.622069in}{0.640440in}}%
\pgfpathlineto{\pgfqpoint{4.622914in}{0.640440in}}%
\pgfpathlineto{\pgfqpoint{4.624605in}{0.648341in}}%
\pgfpathlineto{\pgfqpoint{4.628833in}{0.648350in}}%
\pgfpathlineto{\pgfqpoint{4.630525in}{0.640425in}}%
\pgfpathlineto{\pgfqpoint{4.632216in}{0.650576in}}%
\pgfpathlineto{\pgfqpoint{4.633062in}{0.650576in}}%
\pgfpathlineto{\pgfqpoint{4.634753in}{0.645496in}}%
\pgfpathlineto{\pgfqpoint{4.636444in}{0.645496in}}%
\pgfpathlineto{\pgfqpoint{4.638135in}{0.641856in}}%
\pgfpathlineto{\pgfqpoint{4.639827in}{0.651596in}}%
\pgfpathlineto{\pgfqpoint{4.640672in}{0.651596in}}%
\pgfpathlineto{\pgfqpoint{4.642363in}{0.642751in}}%
\pgfpathlineto{\pgfqpoint{4.644055in}{0.642751in}}%
\pgfpathlineto{\pgfqpoint{4.645746in}{3.376050in}}%
\pgfpathlineto{\pgfqpoint{4.695638in}{3.372542in}}%
\pgfpathlineto{\pgfqpoint{4.748912in}{3.371818in}}%
\pgfpathlineto{\pgfqpoint{4.808106in}{3.374135in}}%
\pgfpathlineto{\pgfqpoint{4.878292in}{3.380251in}}%
\pgfpathlineto{\pgfqpoint{4.922265in}{3.382719in}}%
\pgfpathlineto{\pgfqpoint{4.957781in}{3.381422in}}%
\pgfpathlineto{\pgfqpoint{5.078705in}{3.374151in}}%
\pgfpathlineto{\pgfqpoint{5.138745in}{3.375247in}}%
\pgfpathlineto{\pgfqpoint{5.188636in}{3.378036in}}%
\pgfpathlineto{\pgfqpoint{5.188636in}{3.378036in}}%
\pgfusepath{stroke}%
\end{pgfscope}%
\begin{pgfscope}%
\pgfsetrectcap%
\pgfsetmiterjoin%
\pgfsetlinewidth{0.803000pt}%
\definecolor{currentstroke}{rgb}{0.000000,0.000000,0.000000}%
\pgfsetstrokecolor{currentstroke}%
\pgfsetdash{}{0pt}%
\pgfpathmoveto{\pgfqpoint{0.750000in}{0.500000in}}%
\pgfpathlineto{\pgfqpoint{0.750000in}{3.520000in}}%
\pgfusepath{stroke}%
\end{pgfscope}%
\begin{pgfscope}%
\pgfsetrectcap%
\pgfsetmiterjoin%
\pgfsetlinewidth{0.803000pt}%
\definecolor{currentstroke}{rgb}{0.000000,0.000000,0.000000}%
\pgfsetstrokecolor{currentstroke}%
\pgfsetdash{}{0pt}%
\pgfpathmoveto{\pgfqpoint{5.400000in}{0.500000in}}%
\pgfpathlineto{\pgfqpoint{5.400000in}{3.520000in}}%
\pgfusepath{stroke}%
\end{pgfscope}%
\begin{pgfscope}%
\pgfsetrectcap%
\pgfsetmiterjoin%
\pgfsetlinewidth{0.803000pt}%
\definecolor{currentstroke}{rgb}{0.000000,0.000000,0.000000}%
\pgfsetstrokecolor{currentstroke}%
\pgfsetdash{}{0pt}%
\pgfpathmoveto{\pgfqpoint{0.750000in}{0.500000in}}%
\pgfpathlineto{\pgfqpoint{5.400000in}{0.500000in}}%
\pgfusepath{stroke}%
\end{pgfscope}%
\begin{pgfscope}%
\pgfsetrectcap%
\pgfsetmiterjoin%
\pgfsetlinewidth{0.803000pt}%
\definecolor{currentstroke}{rgb}{0.000000,0.000000,0.000000}%
\pgfsetstrokecolor{currentstroke}%
\pgfsetdash{}{0pt}%
\pgfpathmoveto{\pgfqpoint{0.750000in}{3.520000in}}%
\pgfpathlineto{\pgfqpoint{5.400000in}{3.520000in}}%
\pgfusepath{stroke}%
\end{pgfscope}%
\end{pgfpicture}%
\makeatother%
\endgroup%

    \caption{PEC Data Set Three Phase Grid Fault}
    \label{fig:pec_three_phase_grid_fault}
\end{figure}


\subsubsection{Cyber Security BETH Dataset}
\label{ref_beth_dataset}

The BPF-extended tracking honeypot (BETH) cyber security dataset was released in 2021 and is still under active development and testing. This makes it an attractive dataset for this study since it reflects the current state of the art in cybersecurity and is ``the first cyberscurity dataset for uncertainty and robustness benchmarking \parencite{beth-dataset}.'' A cybersecurity honeypot is a set of computer resources that present a benefit to a hacker if they are exploited but are actively being monitored by an organization or individual. For this dataset, an ssh vulnerability is exploited where any password entered will allow a user to login. The system is running two auxillary containers to monitor traffic. The first is the Berkely Packet Filter (BPF) which examines OS process management calls. The second monitor logs DNS activity from the system. This data is collected and parsed over a series of trials to form the dataset.

The authors present a number of advantages of this dataset that make it attractive for modern machine learning and anomaly detection research \parencite{beth-dataset}:

\begin{enumerate}
    \item At over eight million data points, this is one of the largest cyber security datasets available
    \item It contains modern host activity and attacks
    \item It is fully labelled
    \item It contains highly structured but heterogeneous features 
    \item Each host contains benign activity and at most a single attack, which is ideal for behavioural analysis and other research tasks.
    \item Further data is currently being collected and analysed to add alternative attack vectors to the dataset
\end{enumerate}

Figure \ref{fig:beth_userid_all} shows the samples from the BETH dataset where there are labeled outliers. This study is examining the UserId parameter and using it as a continuous data stream. The spikes in the UserId parameter generally correspond with the labeled outliers.

\begin{figure}[H]
    %\centering
    %% Creator: Matplotlib, PGF backend
%%
%% To include the figure in your LaTeX document, write
%%   \input{<filename>.pgf}
%%
%% Make sure the required packages are loaded in your preamble
%%   \usepackage{pgf}
%%
%% Also ensure that all the required font packages are loaded; for instance,
%% the lmodern package is sometimes necessary when using math font.
%%   \usepackage{lmodern}
%%
%% Figures using additional raster images can only be included by \input if
%% they are in the same directory as the main LaTeX file. For loading figures
%% from other directories you can use the `import` package
%%   \usepackage{import}
%%
%% and then include the figures with
%%   \import{<path to file>}{<filename>.pgf}
%%
%% Matplotlib used the following preamble
%%
\begingroup%
\makeatletter%
\begin{pgfpicture}%
\pgfpathrectangle{\pgfpointorigin}{\pgfqpoint{6.000000in}{4.000000in}}%
\pgfusepath{use as bounding box, clip}%
\begin{pgfscope}%
\pgfsetbuttcap%
\pgfsetmiterjoin%
\pgfsetlinewidth{0.000000pt}%
\definecolor{currentstroke}{rgb}{1.000000,1.000000,1.000000}%
\pgfsetstrokecolor{currentstroke}%
\pgfsetstrokeopacity{0.000000}%
\pgfsetdash{}{0pt}%
\pgfpathmoveto{\pgfqpoint{0.000000in}{0.000000in}}%
\pgfpathlineto{\pgfqpoint{6.000000in}{0.000000in}}%
\pgfpathlineto{\pgfqpoint{6.000000in}{4.000000in}}%
\pgfpathlineto{\pgfqpoint{0.000000in}{4.000000in}}%
\pgfpathlineto{\pgfqpoint{0.000000in}{0.000000in}}%
\pgfpathclose%
\pgfusepath{}%
\end{pgfscope}%
\begin{pgfscope}%
\pgfsetbuttcap%
\pgfsetmiterjoin%
\definecolor{currentfill}{rgb}{1.000000,1.000000,1.000000}%
\pgfsetfillcolor{currentfill}%
\pgfsetlinewidth{0.000000pt}%
\definecolor{currentstroke}{rgb}{0.000000,0.000000,0.000000}%
\pgfsetstrokecolor{currentstroke}%
\pgfsetstrokeopacity{0.000000}%
\pgfsetdash{}{0pt}%
\pgfpathmoveto{\pgfqpoint{0.750000in}{0.500000in}}%
\pgfpathlineto{\pgfqpoint{5.400000in}{0.500000in}}%
\pgfpathlineto{\pgfqpoint{5.400000in}{3.520000in}}%
\pgfpathlineto{\pgfqpoint{0.750000in}{3.520000in}}%
\pgfpathlineto{\pgfqpoint{0.750000in}{0.500000in}}%
\pgfpathclose%
\pgfusepath{fill}%
\end{pgfscope}%
\begin{pgfscope}%
\pgfsetbuttcap%
\pgfsetroundjoin%
\definecolor{currentfill}{rgb}{0.000000,0.000000,0.000000}%
\pgfsetfillcolor{currentfill}%
\pgfsetlinewidth{0.803000pt}%
\definecolor{currentstroke}{rgb}{0.000000,0.000000,0.000000}%
\pgfsetstrokecolor{currentstroke}%
\pgfsetdash{}{0pt}%
\pgfsys@defobject{currentmarker}{\pgfqpoint{0.000000in}{-0.048611in}}{\pgfqpoint{0.000000in}{0.000000in}}{%
\pgfpathmoveto{\pgfqpoint{0.000000in}{0.000000in}}%
\pgfpathlineto{\pgfqpoint{0.000000in}{-0.048611in}}%
\pgfusepath{stroke,fill}%
}%
\begin{pgfscope}%
\pgfsys@transformshift{0.961364in}{0.500000in}%
\pgfsys@useobject{currentmarker}{}%
\end{pgfscope}%
\end{pgfscope}%
\begin{pgfscope}%
\definecolor{textcolor}{rgb}{0.000000,0.000000,0.000000}%
\pgfsetstrokecolor{textcolor}%
\pgfsetfillcolor{textcolor}%
\pgftext[x=0.961364in,y=0.402778in,,top]{\color{textcolor}\rmfamily\fontsize{10.000000}{12.000000}\selectfont \(\displaystyle {0}\)}%
\end{pgfscope}%
\begin{pgfscope}%
\pgfsetbuttcap%
\pgfsetroundjoin%
\definecolor{currentfill}{rgb}{0.000000,0.000000,0.000000}%
\pgfsetfillcolor{currentfill}%
\pgfsetlinewidth{0.803000pt}%
\definecolor{currentstroke}{rgb}{0.000000,0.000000,0.000000}%
\pgfsetstrokecolor{currentstroke}%
\pgfsetdash{}{0pt}%
\pgfsys@defobject{currentmarker}{\pgfqpoint{0.000000in}{-0.048611in}}{\pgfqpoint{0.000000in}{0.000000in}}{%
\pgfpathmoveto{\pgfqpoint{0.000000in}{0.000000in}}%
\pgfpathlineto{\pgfqpoint{0.000000in}{-0.048611in}}%
\pgfusepath{stroke,fill}%
}%
\begin{pgfscope}%
\pgfsys@transformshift{1.905826in}{0.500000in}%
\pgfsys@useobject{currentmarker}{}%
\end{pgfscope}%
\end{pgfscope}%
\begin{pgfscope}%
\definecolor{textcolor}{rgb}{0.000000,0.000000,0.000000}%
\pgfsetstrokecolor{textcolor}%
\pgfsetfillcolor{textcolor}%
\pgftext[x=1.905826in,y=0.402778in,,top]{\color{textcolor}\rmfamily\fontsize{10.000000}{12.000000}\selectfont \(\displaystyle {200000}\)}%
\end{pgfscope}%
\begin{pgfscope}%
\pgfsetbuttcap%
\pgfsetroundjoin%
\definecolor{currentfill}{rgb}{0.000000,0.000000,0.000000}%
\pgfsetfillcolor{currentfill}%
\pgfsetlinewidth{0.803000pt}%
\definecolor{currentstroke}{rgb}{0.000000,0.000000,0.000000}%
\pgfsetstrokecolor{currentstroke}%
\pgfsetdash{}{0pt}%
\pgfsys@defobject{currentmarker}{\pgfqpoint{0.000000in}{-0.048611in}}{\pgfqpoint{0.000000in}{0.000000in}}{%
\pgfpathmoveto{\pgfqpoint{0.000000in}{0.000000in}}%
\pgfpathlineto{\pgfqpoint{0.000000in}{-0.048611in}}%
\pgfusepath{stroke,fill}%
}%
\begin{pgfscope}%
\pgfsys@transformshift{2.850289in}{0.500000in}%
\pgfsys@useobject{currentmarker}{}%
\end{pgfscope}%
\end{pgfscope}%
\begin{pgfscope}%
\definecolor{textcolor}{rgb}{0.000000,0.000000,0.000000}%
\pgfsetstrokecolor{textcolor}%
\pgfsetfillcolor{textcolor}%
\pgftext[x=2.850289in,y=0.402778in,,top]{\color{textcolor}\rmfamily\fontsize{10.000000}{12.000000}\selectfont \(\displaystyle {400000}\)}%
\end{pgfscope}%
\begin{pgfscope}%
\pgfsetbuttcap%
\pgfsetroundjoin%
\definecolor{currentfill}{rgb}{0.000000,0.000000,0.000000}%
\pgfsetfillcolor{currentfill}%
\pgfsetlinewidth{0.803000pt}%
\definecolor{currentstroke}{rgb}{0.000000,0.000000,0.000000}%
\pgfsetstrokecolor{currentstroke}%
\pgfsetdash{}{0pt}%
\pgfsys@defobject{currentmarker}{\pgfqpoint{0.000000in}{-0.048611in}}{\pgfqpoint{0.000000in}{0.000000in}}{%
\pgfpathmoveto{\pgfqpoint{0.000000in}{0.000000in}}%
\pgfpathlineto{\pgfqpoint{0.000000in}{-0.048611in}}%
\pgfusepath{stroke,fill}%
}%
\begin{pgfscope}%
\pgfsys@transformshift{3.794751in}{0.500000in}%
\pgfsys@useobject{currentmarker}{}%
\end{pgfscope}%
\end{pgfscope}%
\begin{pgfscope}%
\definecolor{textcolor}{rgb}{0.000000,0.000000,0.000000}%
\pgfsetstrokecolor{textcolor}%
\pgfsetfillcolor{textcolor}%
\pgftext[x=3.794751in,y=0.402778in,,top]{\color{textcolor}\rmfamily\fontsize{10.000000}{12.000000}\selectfont \(\displaystyle {600000}\)}%
\end{pgfscope}%
\begin{pgfscope}%
\pgfsetbuttcap%
\pgfsetroundjoin%
\definecolor{currentfill}{rgb}{0.000000,0.000000,0.000000}%
\pgfsetfillcolor{currentfill}%
\pgfsetlinewidth{0.803000pt}%
\definecolor{currentstroke}{rgb}{0.000000,0.000000,0.000000}%
\pgfsetstrokecolor{currentstroke}%
\pgfsetdash{}{0pt}%
\pgfsys@defobject{currentmarker}{\pgfqpoint{0.000000in}{-0.048611in}}{\pgfqpoint{0.000000in}{0.000000in}}{%
\pgfpathmoveto{\pgfqpoint{0.000000in}{0.000000in}}%
\pgfpathlineto{\pgfqpoint{0.000000in}{-0.048611in}}%
\pgfusepath{stroke,fill}%
}%
\begin{pgfscope}%
\pgfsys@transformshift{4.739214in}{0.500000in}%
\pgfsys@useobject{currentmarker}{}%
\end{pgfscope}%
\end{pgfscope}%
\begin{pgfscope}%
\definecolor{textcolor}{rgb}{0.000000,0.000000,0.000000}%
\pgfsetstrokecolor{textcolor}%
\pgfsetfillcolor{textcolor}%
\pgftext[x=4.739214in,y=0.402778in,,top]{\color{textcolor}\rmfamily\fontsize{10.000000}{12.000000}\selectfont \(\displaystyle {800000}\)}%
\end{pgfscope}%
\begin{pgfscope}%
\definecolor{textcolor}{rgb}{0.000000,0.000000,0.000000}%
\pgfsetstrokecolor{textcolor}%
\pgfsetfillcolor{textcolor}%
\pgftext[x=3.075000in,y=0.223766in,,top]{\color{textcolor}\rmfamily\fontsize{10.000000}{12.000000}\selectfont Time (s)}%
\end{pgfscope}%
\begin{pgfscope}%
\pgfsetbuttcap%
\pgfsetroundjoin%
\definecolor{currentfill}{rgb}{0.000000,0.000000,0.000000}%
\pgfsetfillcolor{currentfill}%
\pgfsetlinewidth{0.803000pt}%
\definecolor{currentstroke}{rgb}{0.000000,0.000000,0.000000}%
\pgfsetstrokecolor{currentstroke}%
\pgfsetdash{}{0pt}%
\pgfsys@defobject{currentmarker}{\pgfqpoint{-0.048611in}{0.000000in}}{\pgfqpoint{-0.000000in}{0.000000in}}{%
\pgfpathmoveto{\pgfqpoint{-0.000000in}{0.000000in}}%
\pgfpathlineto{\pgfqpoint{-0.048611in}{0.000000in}}%
\pgfusepath{stroke,fill}%
}%
\begin{pgfscope}%
\pgfsys@transformshift{0.750000in}{0.637273in}%
\pgfsys@useobject{currentmarker}{}%
\end{pgfscope}%
\end{pgfscope}%
\begin{pgfscope}%
\definecolor{textcolor}{rgb}{0.000000,0.000000,0.000000}%
\pgfsetstrokecolor{textcolor}%
\pgfsetfillcolor{textcolor}%
\pgftext[x=0.583333in, y=0.589047in, left, base]{\color{textcolor}\rmfamily\fontsize{10.000000}{12.000000}\selectfont \(\displaystyle {0}\)}%
\end{pgfscope}%
\begin{pgfscope}%
\pgfsetbuttcap%
\pgfsetroundjoin%
\definecolor{currentfill}{rgb}{0.000000,0.000000,0.000000}%
\pgfsetfillcolor{currentfill}%
\pgfsetlinewidth{0.803000pt}%
\definecolor{currentstroke}{rgb}{0.000000,0.000000,0.000000}%
\pgfsetstrokecolor{currentstroke}%
\pgfsetdash{}{0pt}%
\pgfsys@defobject{currentmarker}{\pgfqpoint{-0.048611in}{0.000000in}}{\pgfqpoint{-0.000000in}{0.000000in}}{%
\pgfpathmoveto{\pgfqpoint{-0.000000in}{0.000000in}}%
\pgfpathlineto{\pgfqpoint{-0.048611in}{0.000000in}}%
\pgfusepath{stroke,fill}%
}%
\begin{pgfscope}%
\pgfsys@transformshift{0.750000in}{1.185815in}%
\pgfsys@useobject{currentmarker}{}%
\end{pgfscope}%
\end{pgfscope}%
\begin{pgfscope}%
\definecolor{textcolor}{rgb}{0.000000,0.000000,0.000000}%
\pgfsetstrokecolor{textcolor}%
\pgfsetfillcolor{textcolor}%
\pgftext[x=0.444444in, y=1.137590in, left, base]{\color{textcolor}\rmfamily\fontsize{10.000000}{12.000000}\selectfont \(\displaystyle {200}\)}%
\end{pgfscope}%
\begin{pgfscope}%
\pgfsetbuttcap%
\pgfsetroundjoin%
\definecolor{currentfill}{rgb}{0.000000,0.000000,0.000000}%
\pgfsetfillcolor{currentfill}%
\pgfsetlinewidth{0.803000pt}%
\definecolor{currentstroke}{rgb}{0.000000,0.000000,0.000000}%
\pgfsetstrokecolor{currentstroke}%
\pgfsetdash{}{0pt}%
\pgfsys@defobject{currentmarker}{\pgfqpoint{-0.048611in}{0.000000in}}{\pgfqpoint{-0.000000in}{0.000000in}}{%
\pgfpathmoveto{\pgfqpoint{-0.000000in}{0.000000in}}%
\pgfpathlineto{\pgfqpoint{-0.048611in}{0.000000in}}%
\pgfusepath{stroke,fill}%
}%
\begin{pgfscope}%
\pgfsys@transformshift{0.750000in}{1.734357in}%
\pgfsys@useobject{currentmarker}{}%
\end{pgfscope}%
\end{pgfscope}%
\begin{pgfscope}%
\definecolor{textcolor}{rgb}{0.000000,0.000000,0.000000}%
\pgfsetstrokecolor{textcolor}%
\pgfsetfillcolor{textcolor}%
\pgftext[x=0.444444in, y=1.686132in, left, base]{\color{textcolor}\rmfamily\fontsize{10.000000}{12.000000}\selectfont \(\displaystyle {400}\)}%
\end{pgfscope}%
\begin{pgfscope}%
\pgfsetbuttcap%
\pgfsetroundjoin%
\definecolor{currentfill}{rgb}{0.000000,0.000000,0.000000}%
\pgfsetfillcolor{currentfill}%
\pgfsetlinewidth{0.803000pt}%
\definecolor{currentstroke}{rgb}{0.000000,0.000000,0.000000}%
\pgfsetstrokecolor{currentstroke}%
\pgfsetdash{}{0pt}%
\pgfsys@defobject{currentmarker}{\pgfqpoint{-0.048611in}{0.000000in}}{\pgfqpoint{-0.000000in}{0.000000in}}{%
\pgfpathmoveto{\pgfqpoint{-0.000000in}{0.000000in}}%
\pgfpathlineto{\pgfqpoint{-0.048611in}{0.000000in}}%
\pgfusepath{stroke,fill}%
}%
\begin{pgfscope}%
\pgfsys@transformshift{0.750000in}{2.282900in}%
\pgfsys@useobject{currentmarker}{}%
\end{pgfscope}%
\end{pgfscope}%
\begin{pgfscope}%
\definecolor{textcolor}{rgb}{0.000000,0.000000,0.000000}%
\pgfsetstrokecolor{textcolor}%
\pgfsetfillcolor{textcolor}%
\pgftext[x=0.444444in, y=2.234675in, left, base]{\color{textcolor}\rmfamily\fontsize{10.000000}{12.000000}\selectfont \(\displaystyle {600}\)}%
\end{pgfscope}%
\begin{pgfscope}%
\pgfsetbuttcap%
\pgfsetroundjoin%
\definecolor{currentfill}{rgb}{0.000000,0.000000,0.000000}%
\pgfsetfillcolor{currentfill}%
\pgfsetlinewidth{0.803000pt}%
\definecolor{currentstroke}{rgb}{0.000000,0.000000,0.000000}%
\pgfsetstrokecolor{currentstroke}%
\pgfsetdash{}{0pt}%
\pgfsys@defobject{currentmarker}{\pgfqpoint{-0.048611in}{0.000000in}}{\pgfqpoint{-0.000000in}{0.000000in}}{%
\pgfpathmoveto{\pgfqpoint{-0.000000in}{0.000000in}}%
\pgfpathlineto{\pgfqpoint{-0.048611in}{0.000000in}}%
\pgfusepath{stroke,fill}%
}%
\begin{pgfscope}%
\pgfsys@transformshift{0.750000in}{2.831442in}%
\pgfsys@useobject{currentmarker}{}%
\end{pgfscope}%
\end{pgfscope}%
\begin{pgfscope}%
\definecolor{textcolor}{rgb}{0.000000,0.000000,0.000000}%
\pgfsetstrokecolor{textcolor}%
\pgfsetfillcolor{textcolor}%
\pgftext[x=0.444444in, y=2.783217in, left, base]{\color{textcolor}\rmfamily\fontsize{10.000000}{12.000000}\selectfont \(\displaystyle {800}\)}%
\end{pgfscope}%
\begin{pgfscope}%
\pgfsetbuttcap%
\pgfsetroundjoin%
\definecolor{currentfill}{rgb}{0.000000,0.000000,0.000000}%
\pgfsetfillcolor{currentfill}%
\pgfsetlinewidth{0.803000pt}%
\definecolor{currentstroke}{rgb}{0.000000,0.000000,0.000000}%
\pgfsetstrokecolor{currentstroke}%
\pgfsetdash{}{0pt}%
\pgfsys@defobject{currentmarker}{\pgfqpoint{-0.048611in}{0.000000in}}{\pgfqpoint{-0.000000in}{0.000000in}}{%
\pgfpathmoveto{\pgfqpoint{-0.000000in}{0.000000in}}%
\pgfpathlineto{\pgfqpoint{-0.048611in}{0.000000in}}%
\pgfusepath{stroke,fill}%
}%
\begin{pgfscope}%
\pgfsys@transformshift{0.750000in}{3.379985in}%
\pgfsys@useobject{currentmarker}{}%
\end{pgfscope}%
\end{pgfscope}%
\begin{pgfscope}%
\definecolor{textcolor}{rgb}{0.000000,0.000000,0.000000}%
\pgfsetstrokecolor{textcolor}%
\pgfsetfillcolor{textcolor}%
\pgftext[x=0.374999in, y=3.331759in, left, base]{\color{textcolor}\rmfamily\fontsize{10.000000}{12.000000}\selectfont \(\displaystyle {1000}\)}%
\end{pgfscope}%
\begin{pgfscope}%
\definecolor{textcolor}{rgb}{0.000000,0.000000,0.000000}%
\pgfsetstrokecolor{textcolor}%
\pgfsetfillcolor{textcolor}%
\pgftext[x=0.319444in,y=2.010000in,,bottom,rotate=90.000000]{\color{textcolor}\rmfamily\fontsize{10.000000}{12.000000}\selectfont User ID}%
\end{pgfscope}%
\begin{pgfscope}%
\pgfpathrectangle{\pgfqpoint{0.750000in}{0.500000in}}{\pgfqpoint{4.650000in}{3.020000in}}%
\pgfusepath{clip}%
\pgfsetrectcap%
\pgfsetroundjoin%
\pgfsetlinewidth{1.505625pt}%
\definecolor{currentstroke}{rgb}{0.121569,0.466667,0.705882}%
\pgfsetstrokecolor{currentstroke}%
\pgfsetdash{}{0pt}%
\pgfpathmoveto{\pgfqpoint{0.961364in}{0.914287in}}%
\pgfpathlineto{\pgfqpoint{0.961619in}{0.914287in}}%
\pgfpathlineto{\pgfqpoint{0.962941in}{3.379985in}}%
\pgfpathlineto{\pgfqpoint{0.961652in}{0.637273in}}%
\pgfpathlineto{\pgfqpoint{0.963158in}{0.637273in}}%
\pgfpathlineto{\pgfqpoint{0.963234in}{0.637273in}}%
\pgfpathlineto{\pgfqpoint{0.963342in}{3.379985in}}%
\pgfpathlineto{\pgfqpoint{0.964773in}{0.637273in}}%
\pgfpathlineto{\pgfqpoint{0.964778in}{0.637273in}}%
\pgfpathlineto{\pgfqpoint{0.964783in}{0.919772in}}%
\pgfpathlineto{\pgfqpoint{0.966317in}{0.637273in}}%
\pgfpathlineto{\pgfqpoint{0.966383in}{0.637273in}}%
\pgfpathlineto{\pgfqpoint{0.967885in}{3.379985in}}%
\pgfpathlineto{\pgfqpoint{0.967923in}{0.637273in}}%
\pgfpathlineto{\pgfqpoint{0.967998in}{0.637273in}}%
\pgfpathlineto{\pgfqpoint{0.968688in}{3.379985in}}%
\pgfpathlineto{\pgfqpoint{0.969538in}{0.637273in}}%
\pgfpathlineto{\pgfqpoint{0.980560in}{0.637273in}}%
\pgfpathlineto{\pgfqpoint{0.980588in}{0.911544in}}%
\pgfpathlineto{\pgfqpoint{0.982099in}{0.637273in}}%
\pgfpathlineto{\pgfqpoint{0.986146in}{0.637273in}}%
\pgfpathlineto{\pgfqpoint{0.986222in}{0.914287in}}%
\pgfpathlineto{\pgfqpoint{0.987686in}{0.637273in}}%
\pgfpathlineto{\pgfqpoint{0.991804in}{0.637273in}}%
\pgfpathlineto{\pgfqpoint{0.992432in}{0.914287in}}%
\pgfpathlineto{\pgfqpoint{0.993343in}{0.637273in}}%
\pgfpathlineto{\pgfqpoint{0.998656in}{0.637273in}}%
\pgfpathlineto{\pgfqpoint{0.998684in}{0.911544in}}%
\pgfpathlineto{\pgfqpoint{1.000195in}{0.637273in}}%
\pgfpathlineto{\pgfqpoint{1.004356in}{0.637273in}}%
\pgfpathlineto{\pgfqpoint{1.004360in}{0.919772in}}%
\pgfpathlineto{\pgfqpoint{1.005895in}{0.637273in}}%
\pgfpathlineto{\pgfqpoint{1.010537in}{0.637273in}}%
\pgfpathlineto{\pgfqpoint{1.010565in}{0.911544in}}%
\pgfpathlineto{\pgfqpoint{1.012077in}{0.637273in}}%
\pgfpathlineto{\pgfqpoint{1.017847in}{0.637273in}}%
\pgfpathlineto{\pgfqpoint{1.017857in}{0.911544in}}%
\pgfpathlineto{\pgfqpoint{1.019387in}{0.637273in}}%
\pgfpathlineto{\pgfqpoint{1.023259in}{0.637273in}}%
\pgfpathlineto{\pgfqpoint{1.023287in}{0.914287in}}%
\pgfpathlineto{\pgfqpoint{1.024798in}{0.637273in}}%
\pgfpathlineto{\pgfqpoint{1.029091in}{0.637273in}}%
\pgfpathlineto{\pgfqpoint{1.029119in}{0.911544in}}%
\pgfpathlineto{\pgfqpoint{1.030631in}{0.637273in}}%
\pgfpathlineto{\pgfqpoint{1.034531in}{0.637273in}}%
\pgfpathlineto{\pgfqpoint{1.034597in}{0.914287in}}%
\pgfpathlineto{\pgfqpoint{1.036071in}{0.637273in}}%
\pgfpathlineto{\pgfqpoint{1.045364in}{0.637273in}}%
\pgfpathlineto{\pgfqpoint{1.045421in}{0.914287in}}%
\pgfpathlineto{\pgfqpoint{1.046904in}{0.637273in}}%
\pgfpathlineto{\pgfqpoint{1.063011in}{0.637273in}}%
\pgfpathlineto{\pgfqpoint{1.063248in}{0.914287in}}%
\pgfpathlineto{\pgfqpoint{1.064551in}{0.637273in}}%
\pgfpathlineto{\pgfqpoint{1.069335in}{0.637273in}}%
\pgfpathlineto{\pgfqpoint{1.069363in}{0.911544in}}%
\pgfpathlineto{\pgfqpoint{1.070874in}{0.637273in}}%
\pgfpathlineto{\pgfqpoint{1.074841in}{0.637273in}}%
\pgfpathlineto{\pgfqpoint{1.074926in}{0.914287in}}%
\pgfpathlineto{\pgfqpoint{1.076380in}{0.637273in}}%
\pgfpathlineto{\pgfqpoint{1.080375in}{0.637273in}}%
\pgfpathlineto{\pgfqpoint{1.080404in}{0.911544in}}%
\pgfpathlineto{\pgfqpoint{1.081915in}{0.637273in}}%
\pgfpathlineto{\pgfqpoint{1.085882in}{0.637273in}}%
\pgfpathlineto{\pgfqpoint{1.085938in}{0.914287in}}%
\pgfpathlineto{\pgfqpoint{1.087421in}{0.637273in}}%
\pgfpathlineto{\pgfqpoint{1.091619in}{0.637273in}}%
\pgfpathlineto{\pgfqpoint{1.091648in}{0.911544in}}%
\pgfpathlineto{\pgfqpoint{1.093159in}{0.637273in}}%
\pgfpathlineto{\pgfqpoint{1.097036in}{0.637273in}}%
\pgfpathlineto{\pgfqpoint{1.097064in}{0.911544in}}%
\pgfpathlineto{\pgfqpoint{1.098575in}{0.637273in}}%
\pgfpathlineto{\pgfqpoint{1.102353in}{0.637273in}}%
\pgfpathlineto{\pgfqpoint{1.102362in}{0.914287in}}%
\pgfpathlineto{\pgfqpoint{1.103892in}{0.637273in}}%
\pgfpathlineto{\pgfqpoint{1.110754in}{0.637273in}}%
\pgfpathlineto{\pgfqpoint{1.110990in}{0.936228in}}%
\pgfpathlineto{\pgfqpoint{1.112293in}{0.637273in}}%
\pgfpathlineto{\pgfqpoint{1.113380in}{0.637273in}}%
\pgfpathlineto{\pgfqpoint{1.113847in}{0.936228in}}%
\pgfpathlineto{\pgfqpoint{1.114919in}{0.637273in}}%
\pgfpathlineto{\pgfqpoint{1.119533in}{0.637273in}}%
\pgfpathlineto{\pgfqpoint{1.119646in}{0.936228in}}%
\pgfpathlineto{\pgfqpoint{1.121072in}{0.637273in}}%
\pgfpathlineto{\pgfqpoint{1.121077in}{0.637273in}}%
\pgfpathlineto{\pgfqpoint{1.121157in}{0.936228in}}%
\pgfpathlineto{\pgfqpoint{1.122616in}{0.637273in}}%
\pgfpathlineto{\pgfqpoint{1.132288in}{0.637273in}}%
\pgfpathlineto{\pgfqpoint{1.132802in}{0.914287in}}%
\pgfpathlineto{\pgfqpoint{1.133827in}{0.637273in}}%
\pgfpathlineto{\pgfqpoint{1.138890in}{0.637273in}}%
\pgfpathlineto{\pgfqpoint{1.138918in}{0.911544in}}%
\pgfpathlineto{\pgfqpoint{1.140429in}{0.637273in}}%
\pgfpathlineto{\pgfqpoint{1.144330in}{0.637273in}}%
\pgfpathlineto{\pgfqpoint{1.144419in}{0.914287in}}%
\pgfpathlineto{\pgfqpoint{1.145869in}{0.637273in}}%
\pgfpathlineto{\pgfqpoint{1.149855in}{0.637273in}}%
\pgfpathlineto{\pgfqpoint{1.149883in}{0.911544in}}%
\pgfpathlineto{\pgfqpoint{1.151394in}{0.637273in}}%
\pgfpathlineto{\pgfqpoint{1.155267in}{0.637273in}}%
\pgfpathlineto{\pgfqpoint{1.155352in}{0.914287in}}%
\pgfpathlineto{\pgfqpoint{1.156806in}{0.637273in}}%
\pgfpathlineto{\pgfqpoint{1.160707in}{0.637273in}}%
\pgfpathlineto{\pgfqpoint{1.161750in}{0.914287in}}%
\pgfpathlineto{\pgfqpoint{1.162246in}{0.637273in}}%
\pgfpathlineto{\pgfqpoint{1.167422in}{0.637273in}}%
\pgfpathlineto{\pgfqpoint{1.167516in}{0.914287in}}%
\pgfpathlineto{\pgfqpoint{1.168961in}{0.637273in}}%
\pgfpathlineto{\pgfqpoint{1.172881in}{0.637273in}}%
\pgfpathlineto{\pgfqpoint{1.172909in}{0.911544in}}%
\pgfpathlineto{\pgfqpoint{1.174420in}{0.637273in}}%
\pgfpathlineto{\pgfqpoint{1.178297in}{0.637273in}}%
\pgfpathlineto{\pgfqpoint{1.178392in}{0.914287in}}%
\pgfpathlineto{\pgfqpoint{1.179837in}{0.637273in}}%
\pgfpathlineto{\pgfqpoint{1.183756in}{0.637273in}}%
\pgfpathlineto{\pgfqpoint{1.183785in}{0.911544in}}%
\pgfpathlineto{\pgfqpoint{1.185296in}{0.637273in}}%
\pgfpathlineto{\pgfqpoint{1.189428in}{0.637273in}}%
\pgfpathlineto{\pgfqpoint{1.190519in}{0.914287in}}%
\pgfpathlineto{\pgfqpoint{1.190967in}{0.637273in}}%
\pgfpathlineto{\pgfqpoint{1.196209in}{0.637273in}}%
\pgfpathlineto{\pgfqpoint{1.196294in}{0.911544in}}%
\pgfpathlineto{\pgfqpoint{1.197748in}{0.637273in}}%
\pgfpathlineto{\pgfqpoint{1.201659in}{0.637273in}}%
\pgfpathlineto{\pgfqpoint{1.201687in}{0.914287in}}%
\pgfpathlineto{\pgfqpoint{1.203198in}{0.637273in}}%
\pgfpathlineto{\pgfqpoint{1.207188in}{0.637273in}}%
\pgfpathlineto{\pgfqpoint{1.207217in}{0.911544in}}%
\pgfpathlineto{\pgfqpoint{1.208728in}{0.637273in}}%
\pgfpathlineto{\pgfqpoint{1.212534in}{0.637273in}}%
\pgfpathlineto{\pgfqpoint{1.212562in}{0.914287in}}%
\pgfpathlineto{\pgfqpoint{1.214073in}{0.637273in}}%
\pgfpathlineto{\pgfqpoint{1.218257in}{0.637273in}}%
\pgfpathlineto{\pgfqpoint{1.218286in}{0.911544in}}%
\pgfpathlineto{\pgfqpoint{1.219797in}{0.637273in}}%
\pgfpathlineto{\pgfqpoint{1.223679in}{0.637273in}}%
\pgfpathlineto{\pgfqpoint{1.223773in}{0.914287in}}%
\pgfpathlineto{\pgfqpoint{1.225218in}{0.637273in}}%
\pgfpathlineto{\pgfqpoint{1.229138in}{0.637273in}}%
\pgfpathlineto{\pgfqpoint{1.229166in}{0.911544in}}%
\pgfpathlineto{\pgfqpoint{1.230677in}{0.637273in}}%
\pgfpathlineto{\pgfqpoint{1.234597in}{0.637273in}}%
\pgfpathlineto{\pgfqpoint{1.234644in}{0.914287in}}%
\pgfpathlineto{\pgfqpoint{1.236136in}{0.637273in}}%
\pgfpathlineto{\pgfqpoint{1.240089in}{0.637273in}}%
\pgfpathlineto{\pgfqpoint{1.240117in}{0.911544in}}%
\pgfpathlineto{\pgfqpoint{1.241628in}{0.637273in}}%
\pgfpathlineto{\pgfqpoint{1.246015in}{0.637273in}}%
\pgfpathlineto{\pgfqpoint{1.246081in}{0.914287in}}%
\pgfpathlineto{\pgfqpoint{1.247555in}{0.637273in}}%
\pgfpathlineto{\pgfqpoint{1.251455in}{0.637273in}}%
\pgfpathlineto{\pgfqpoint{1.251484in}{0.911544in}}%
\pgfpathlineto{\pgfqpoint{1.252995in}{0.637273in}}%
\pgfpathlineto{\pgfqpoint{1.256839in}{0.637273in}}%
\pgfpathlineto{\pgfqpoint{1.256933in}{0.914287in}}%
\pgfpathlineto{\pgfqpoint{1.258378in}{0.637273in}}%
\pgfpathlineto{\pgfqpoint{1.262251in}{0.637273in}}%
\pgfpathlineto{\pgfqpoint{1.262279in}{0.911544in}}%
\pgfpathlineto{\pgfqpoint{1.263790in}{0.637273in}}%
\pgfpathlineto{\pgfqpoint{1.267625in}{0.637273in}}%
\pgfpathlineto{\pgfqpoint{1.267719in}{0.914287in}}%
\pgfpathlineto{\pgfqpoint{1.269164in}{0.637273in}}%
\pgfpathlineto{\pgfqpoint{1.273287in}{0.637273in}}%
\pgfpathlineto{\pgfqpoint{1.273315in}{0.911544in}}%
\pgfpathlineto{\pgfqpoint{1.274826in}{0.637273in}}%
\pgfpathlineto{\pgfqpoint{1.278675in}{0.637273in}}%
\pgfpathlineto{\pgfqpoint{1.278750in}{0.914287in}}%
\pgfpathlineto{\pgfqpoint{1.280214in}{0.637273in}}%
\pgfpathlineto{\pgfqpoint{1.284200in}{0.637273in}}%
\pgfpathlineto{\pgfqpoint{1.284228in}{0.911544in}}%
\pgfpathlineto{\pgfqpoint{1.285739in}{0.637273in}}%
\pgfpathlineto{\pgfqpoint{1.289593in}{0.637273in}}%
\pgfpathlineto{\pgfqpoint{1.289687in}{0.914287in}}%
\pgfpathlineto{\pgfqpoint{1.291132in}{0.637273in}}%
\pgfpathlineto{\pgfqpoint{1.295052in}{0.637273in}}%
\pgfpathlineto{\pgfqpoint{1.295623in}{0.914287in}}%
\pgfpathlineto{\pgfqpoint{1.296591in}{0.637273in}}%
\pgfpathlineto{\pgfqpoint{1.301757in}{0.637273in}}%
\pgfpathlineto{\pgfqpoint{1.301786in}{0.911544in}}%
\pgfpathlineto{\pgfqpoint{1.303297in}{0.637273in}}%
\pgfpathlineto{\pgfqpoint{1.307183in}{0.637273in}}%
\pgfpathlineto{\pgfqpoint{1.307212in}{0.914287in}}%
\pgfpathlineto{\pgfqpoint{1.308723in}{0.637273in}}%
\pgfpathlineto{\pgfqpoint{1.312680in}{0.637273in}}%
\pgfpathlineto{\pgfqpoint{1.312708in}{0.911544in}}%
\pgfpathlineto{\pgfqpoint{1.314220in}{0.637273in}}%
\pgfpathlineto{\pgfqpoint{1.323438in}{0.637273in}}%
\pgfpathlineto{\pgfqpoint{1.324528in}{0.914287in}}%
\pgfpathlineto{\pgfqpoint{1.324977in}{0.637273in}}%
\pgfpathlineto{\pgfqpoint{1.335583in}{0.637273in}}%
\pgfpathlineto{\pgfqpoint{1.335678in}{0.914287in}}%
\pgfpathlineto{\pgfqpoint{1.337123in}{0.637273in}}%
\pgfpathlineto{\pgfqpoint{1.341066in}{0.637273in}}%
\pgfpathlineto{\pgfqpoint{1.341094in}{0.911544in}}%
\pgfpathlineto{\pgfqpoint{1.342605in}{0.637273in}}%
\pgfpathlineto{\pgfqpoint{1.346497in}{0.637273in}}%
\pgfpathlineto{\pgfqpoint{1.346525in}{0.914287in}}%
\pgfpathlineto{\pgfqpoint{1.348036in}{0.637273in}}%
\pgfpathlineto{\pgfqpoint{1.351918in}{0.637273in}}%
\pgfpathlineto{\pgfqpoint{1.352281in}{0.911544in}}%
\pgfpathlineto{\pgfqpoint{1.353457in}{0.637273in}}%
\pgfpathlineto{\pgfqpoint{1.357618in}{0.637273in}}%
\pgfpathlineto{\pgfqpoint{1.357627in}{0.914287in}}%
\pgfpathlineto{\pgfqpoint{1.359157in}{0.637273in}}%
\pgfpathlineto{\pgfqpoint{1.363133in}{0.637273in}}%
\pgfpathlineto{\pgfqpoint{1.363162in}{0.911544in}}%
\pgfpathlineto{\pgfqpoint{1.364673in}{0.637273in}}%
\pgfpathlineto{\pgfqpoint{1.368536in}{0.637273in}}%
\pgfpathlineto{\pgfqpoint{1.368545in}{0.914287in}}%
\pgfpathlineto{\pgfqpoint{1.370075in}{0.637273in}}%
\pgfpathlineto{\pgfqpoint{1.375780in}{0.637273in}}%
\pgfpathlineto{\pgfqpoint{1.375808in}{0.911544in}}%
\pgfpathlineto{\pgfqpoint{1.377319in}{0.637273in}}%
\pgfpathlineto{\pgfqpoint{1.381144in}{0.637273in}}%
\pgfpathlineto{\pgfqpoint{1.381210in}{0.914287in}}%
\pgfpathlineto{\pgfqpoint{1.382684in}{0.637273in}}%
\pgfpathlineto{\pgfqpoint{1.386938in}{0.637273in}}%
\pgfpathlineto{\pgfqpoint{1.387033in}{0.914287in}}%
\pgfpathlineto{\pgfqpoint{1.388478in}{0.637273in}}%
\pgfpathlineto{\pgfqpoint{1.392397in}{0.637273in}}%
\pgfpathlineto{\pgfqpoint{1.392426in}{0.911544in}}%
\pgfpathlineto{\pgfqpoint{1.393937in}{0.637273in}}%
\pgfpathlineto{\pgfqpoint{1.397856in}{0.637273in}}%
\pgfpathlineto{\pgfqpoint{1.397951in}{0.914287in}}%
\pgfpathlineto{\pgfqpoint{1.399396in}{0.637273in}}%
\pgfpathlineto{\pgfqpoint{1.408699in}{0.637273in}}%
\pgfpathlineto{\pgfqpoint{1.409233in}{0.914287in}}%
\pgfpathlineto{\pgfqpoint{1.410238in}{0.637273in}}%
\pgfpathlineto{\pgfqpoint{1.415338in}{0.637273in}}%
\pgfpathlineto{\pgfqpoint{1.415447in}{0.911544in}}%
\pgfpathlineto{\pgfqpoint{1.416878in}{0.637273in}}%
\pgfpathlineto{\pgfqpoint{1.420816in}{0.637273in}}%
\pgfpathlineto{\pgfqpoint{1.420845in}{0.914287in}}%
\pgfpathlineto{\pgfqpoint{1.422356in}{0.637273in}}%
\pgfpathlineto{\pgfqpoint{1.426313in}{0.637273in}}%
\pgfpathlineto{\pgfqpoint{1.426341in}{0.911544in}}%
\pgfpathlineto{\pgfqpoint{1.427853in}{0.637273in}}%
\pgfpathlineto{\pgfqpoint{1.431739in}{0.637273in}}%
\pgfpathlineto{\pgfqpoint{1.431767in}{0.914287in}}%
\pgfpathlineto{\pgfqpoint{1.433279in}{0.637273in}}%
\pgfpathlineto{\pgfqpoint{1.437415in}{0.637273in}}%
\pgfpathlineto{\pgfqpoint{1.437444in}{0.911544in}}%
\pgfpathlineto{\pgfqpoint{1.438955in}{0.637273in}}%
\pgfpathlineto{\pgfqpoint{1.442836in}{0.637273in}}%
\pgfpathlineto{\pgfqpoint{1.442903in}{0.914287in}}%
\pgfpathlineto{\pgfqpoint{1.444376in}{0.637273in}}%
\pgfpathlineto{\pgfqpoint{1.448362in}{0.637273in}}%
\pgfpathlineto{\pgfqpoint{1.448390in}{0.911544in}}%
\pgfpathlineto{\pgfqpoint{1.449901in}{0.637273in}}%
\pgfpathlineto{\pgfqpoint{1.453754in}{0.637273in}}%
\pgfpathlineto{\pgfqpoint{1.453821in}{0.914287in}}%
\pgfpathlineto{\pgfqpoint{1.455294in}{0.637273in}}%
\pgfpathlineto{\pgfqpoint{1.459242in}{0.637273in}}%
\pgfpathlineto{\pgfqpoint{1.459270in}{0.911544in}}%
\pgfpathlineto{\pgfqpoint{1.460781in}{0.637273in}}%
\pgfpathlineto{\pgfqpoint{1.464890in}{0.637273in}}%
\pgfpathlineto{\pgfqpoint{1.464984in}{0.914287in}}%
\pgfpathlineto{\pgfqpoint{1.466429in}{0.637273in}}%
\pgfpathlineto{\pgfqpoint{1.470301in}{0.637273in}}%
\pgfpathlineto{\pgfqpoint{1.470330in}{0.911544in}}%
\pgfpathlineto{\pgfqpoint{1.471841in}{0.637273in}}%
\pgfpathlineto{\pgfqpoint{1.475675in}{0.637273in}}%
\pgfpathlineto{\pgfqpoint{1.475808in}{0.914287in}}%
\pgfpathlineto{\pgfqpoint{1.477215in}{0.637273in}}%
\pgfpathlineto{\pgfqpoint{1.481153in}{0.637273in}}%
\pgfpathlineto{\pgfqpoint{1.481182in}{0.911544in}}%
\pgfpathlineto{\pgfqpoint{1.482693in}{0.637273in}}%
\pgfpathlineto{\pgfqpoint{1.486499in}{0.637273in}}%
\pgfpathlineto{\pgfqpoint{1.487590in}{0.914287in}}%
\pgfpathlineto{\pgfqpoint{1.488038in}{0.637273in}}%
\pgfpathlineto{\pgfqpoint{1.494706in}{0.637273in}}%
\pgfpathlineto{\pgfqpoint{1.494735in}{0.911544in}}%
\pgfpathlineto{\pgfqpoint{1.496246in}{0.637273in}}%
\pgfpathlineto{\pgfqpoint{1.500080in}{0.637273in}}%
\pgfpathlineto{\pgfqpoint{1.500175in}{0.914287in}}%
\pgfpathlineto{\pgfqpoint{1.501620in}{0.637273in}}%
\pgfpathlineto{\pgfqpoint{1.505558in}{0.637273in}}%
\pgfpathlineto{\pgfqpoint{1.505587in}{0.911544in}}%
\pgfpathlineto{\pgfqpoint{1.507098in}{0.637273in}}%
\pgfpathlineto{\pgfqpoint{1.510951in}{0.637273in}}%
\pgfpathlineto{\pgfqpoint{1.511046in}{0.914287in}}%
\pgfpathlineto{\pgfqpoint{1.512491in}{0.637273in}}%
\pgfpathlineto{\pgfqpoint{1.516410in}{0.637273in}}%
\pgfpathlineto{\pgfqpoint{1.516438in}{0.911544in}}%
\pgfpathlineto{\pgfqpoint{1.517950in}{0.637273in}}%
\pgfpathlineto{\pgfqpoint{1.522204in}{0.637273in}}%
\pgfpathlineto{\pgfqpoint{1.522322in}{0.914287in}}%
\pgfpathlineto{\pgfqpoint{1.523744in}{0.637273in}}%
\pgfpathlineto{\pgfqpoint{1.528570in}{0.637273in}}%
\pgfpathlineto{\pgfqpoint{1.528598in}{0.911544in}}%
\pgfpathlineto{\pgfqpoint{1.530110in}{0.637273in}}%
\pgfpathlineto{\pgfqpoint{1.533935in}{0.637273in}}%
\pgfpathlineto{\pgfqpoint{1.533963in}{0.914287in}}%
\pgfpathlineto{\pgfqpoint{1.535474in}{0.637273in}}%
\pgfpathlineto{\pgfqpoint{1.540829in}{0.637273in}}%
\pgfpathlineto{\pgfqpoint{1.540858in}{0.911544in}}%
\pgfpathlineto{\pgfqpoint{1.542369in}{0.637273in}}%
\pgfpathlineto{\pgfqpoint{1.546175in}{0.637273in}}%
\pgfpathlineto{\pgfqpoint{1.546203in}{0.914287in}}%
\pgfpathlineto{\pgfqpoint{1.547714in}{0.637273in}}%
\pgfpathlineto{\pgfqpoint{1.551851in}{0.637273in}}%
\pgfpathlineto{\pgfqpoint{1.551879in}{0.911544in}}%
\pgfpathlineto{\pgfqpoint{1.553391in}{0.637273in}}%
\pgfpathlineto{\pgfqpoint{1.557272in}{0.637273in}}%
\pgfpathlineto{\pgfqpoint{1.557367in}{0.914287in}}%
\pgfpathlineto{\pgfqpoint{1.558812in}{0.637273in}}%
\pgfpathlineto{\pgfqpoint{1.562854in}{0.637273in}}%
\pgfpathlineto{\pgfqpoint{1.562882in}{0.911544in}}%
\pgfpathlineto{\pgfqpoint{1.564394in}{0.637273in}}%
\pgfpathlineto{\pgfqpoint{1.568219in}{0.637273in}}%
\pgfpathlineto{\pgfqpoint{1.568285in}{0.914287in}}%
\pgfpathlineto{\pgfqpoint{1.569758in}{0.637273in}}%
\pgfpathlineto{\pgfqpoint{1.573985in}{0.637273in}}%
\pgfpathlineto{\pgfqpoint{1.574008in}{0.911544in}}%
\pgfpathlineto{\pgfqpoint{1.575524in}{0.637273in}}%
\pgfpathlineto{\pgfqpoint{1.579349in}{0.637273in}}%
\pgfpathlineto{\pgfqpoint{1.579415in}{0.914287in}}%
\pgfpathlineto{\pgfqpoint{1.580889in}{0.637273in}}%
\pgfpathlineto{\pgfqpoint{1.584822in}{0.637273in}}%
\pgfpathlineto{\pgfqpoint{1.584851in}{0.911544in}}%
\pgfpathlineto{\pgfqpoint{1.586362in}{0.637273in}}%
\pgfpathlineto{\pgfqpoint{1.590177in}{0.637273in}}%
\pgfpathlineto{\pgfqpoint{1.590243in}{0.914287in}}%
\pgfpathlineto{\pgfqpoint{1.591717in}{0.637273in}}%
\pgfpathlineto{\pgfqpoint{1.595693in}{0.637273in}}%
\pgfpathlineto{\pgfqpoint{1.595721in}{0.911544in}}%
\pgfpathlineto{\pgfqpoint{1.597232in}{0.637273in}}%
\pgfpathlineto{\pgfqpoint{1.601058in}{0.637273in}}%
\pgfpathlineto{\pgfqpoint{1.601124in}{0.914287in}}%
\pgfpathlineto{\pgfqpoint{1.602597in}{0.637273in}}%
\pgfpathlineto{\pgfqpoint{1.606795in}{0.637273in}}%
\pgfpathlineto{\pgfqpoint{1.606824in}{0.911544in}}%
\pgfpathlineto{\pgfqpoint{1.608335in}{0.637273in}}%
\pgfpathlineto{\pgfqpoint{1.612188in}{0.637273in}}%
\pgfpathlineto{\pgfqpoint{1.612283in}{0.914287in}}%
\pgfpathlineto{\pgfqpoint{1.613728in}{0.637273in}}%
\pgfpathlineto{\pgfqpoint{1.617775in}{0.637273in}}%
\pgfpathlineto{\pgfqpoint{1.617803in}{0.911544in}}%
\pgfpathlineto{\pgfqpoint{1.619314in}{0.637273in}}%
\pgfpathlineto{\pgfqpoint{1.623158in}{0.637273in}}%
\pgfpathlineto{\pgfqpoint{1.623205in}{0.914287in}}%
\pgfpathlineto{\pgfqpoint{1.624697in}{0.637273in}}%
\pgfpathlineto{\pgfqpoint{1.628905in}{0.637273in}}%
\pgfpathlineto{\pgfqpoint{1.628933in}{0.911544in}}%
\pgfpathlineto{\pgfqpoint{1.630445in}{0.637273in}}%
\pgfpathlineto{\pgfqpoint{1.634270in}{0.637273in}}%
\pgfpathlineto{\pgfqpoint{1.634279in}{0.914287in}}%
\pgfpathlineto{\pgfqpoint{1.635809in}{0.637273in}}%
\pgfpathlineto{\pgfqpoint{1.639785in}{0.637273in}}%
\pgfpathlineto{\pgfqpoint{1.639917in}{0.914287in}}%
\pgfpathlineto{\pgfqpoint{1.641325in}{0.637273in}}%
\pgfpathlineto{\pgfqpoint{1.646557in}{0.637273in}}%
\pgfpathlineto{\pgfqpoint{1.646949in}{0.936228in}}%
\pgfpathlineto{\pgfqpoint{1.648097in}{0.637273in}}%
\pgfpathlineto{\pgfqpoint{1.652417in}{0.637273in}}%
\pgfpathlineto{\pgfqpoint{1.652512in}{0.914287in}}%
\pgfpathlineto{\pgfqpoint{1.653957in}{0.637273in}}%
\pgfpathlineto{\pgfqpoint{1.657848in}{0.637273in}}%
\pgfpathlineto{\pgfqpoint{1.658845in}{0.914287in}}%
\pgfpathlineto{\pgfqpoint{1.659388in}{0.637273in}}%
\pgfpathlineto{\pgfqpoint{1.664582in}{0.637273in}}%
\pgfpathlineto{\pgfqpoint{1.664677in}{0.914287in}}%
\pgfpathlineto{\pgfqpoint{1.666122in}{0.637273in}}%
\pgfpathlineto{\pgfqpoint{1.670107in}{0.637273in}}%
\pgfpathlineto{\pgfqpoint{1.670136in}{0.911544in}}%
\pgfpathlineto{\pgfqpoint{1.671647in}{0.637273in}}%
\pgfpathlineto{\pgfqpoint{1.675845in}{0.637273in}}%
\pgfpathlineto{\pgfqpoint{1.675939in}{0.914287in}}%
\pgfpathlineto{\pgfqpoint{1.677384in}{0.637273in}}%
\pgfpathlineto{\pgfqpoint{1.681309in}{0.637273in}}%
\pgfpathlineto{\pgfqpoint{1.681337in}{0.911544in}}%
\pgfpathlineto{\pgfqpoint{1.682848in}{0.637273in}}%
\pgfpathlineto{\pgfqpoint{1.689780in}{0.637273in}}%
\pgfpathlineto{\pgfqpoint{1.690026in}{0.936228in}}%
\pgfpathlineto{\pgfqpoint{1.691320in}{0.637273in}}%
\pgfpathlineto{\pgfqpoint{1.691490in}{0.637273in}}%
\pgfpathlineto{\pgfqpoint{1.691570in}{0.936228in}}%
\pgfpathlineto{\pgfqpoint{1.693029in}{0.637273in}}%
\pgfpathlineto{\pgfqpoint{1.693294in}{0.637273in}}%
\pgfpathlineto{\pgfqpoint{1.693605in}{0.936228in}}%
\pgfpathlineto{\pgfqpoint{1.694833in}{0.637273in}}%
\pgfpathlineto{\pgfqpoint{1.700292in}{0.637273in}}%
\pgfpathlineto{\pgfqpoint{1.700599in}{0.936228in}}%
\pgfpathlineto{\pgfqpoint{1.701832in}{0.637273in}}%
\pgfpathlineto{\pgfqpoint{1.702030in}{0.637273in}}%
\pgfpathlineto{\pgfqpoint{1.702110in}{0.936228in}}%
\pgfpathlineto{\pgfqpoint{1.703570in}{0.637273in}}%
\pgfpathlineto{\pgfqpoint{1.707905in}{0.637273in}}%
\pgfpathlineto{\pgfqpoint{1.708018in}{0.914287in}}%
\pgfpathlineto{\pgfqpoint{1.709444in}{0.637273in}}%
\pgfpathlineto{\pgfqpoint{1.713335in}{0.637273in}}%
\pgfpathlineto{\pgfqpoint{1.713364in}{0.911544in}}%
\pgfpathlineto{\pgfqpoint{1.714875in}{0.637273in}}%
\pgfpathlineto{\pgfqpoint{1.720513in}{0.637273in}}%
\pgfpathlineto{\pgfqpoint{1.720584in}{0.914287in}}%
\pgfpathlineto{\pgfqpoint{1.722053in}{0.637273in}}%
\pgfpathlineto{\pgfqpoint{1.726156in}{0.637273in}}%
\pgfpathlineto{\pgfqpoint{1.726185in}{0.911544in}}%
\pgfpathlineto{\pgfqpoint{1.727696in}{0.637273in}}%
\pgfpathlineto{\pgfqpoint{1.731549in}{0.637273in}}%
\pgfpathlineto{\pgfqpoint{1.731625in}{0.914287in}}%
\pgfpathlineto{\pgfqpoint{1.733089in}{0.637273in}}%
\pgfpathlineto{\pgfqpoint{1.742373in}{0.637273in}}%
\pgfpathlineto{\pgfqpoint{1.742448in}{0.914287in}}%
\pgfpathlineto{\pgfqpoint{1.743912in}{0.637273in}}%
\pgfpathlineto{\pgfqpoint{1.747827in}{0.637273in}}%
\pgfpathlineto{\pgfqpoint{1.747959in}{0.911544in}}%
\pgfpathlineto{\pgfqpoint{1.749367in}{0.637273in}}%
\pgfpathlineto{\pgfqpoint{1.753654in}{0.637273in}}%
\pgfpathlineto{\pgfqpoint{1.753772in}{0.914287in}}%
\pgfpathlineto{\pgfqpoint{1.755194in}{0.637273in}}%
\pgfpathlineto{\pgfqpoint{1.760011in}{0.637273in}}%
\pgfpathlineto{\pgfqpoint{1.760039in}{0.911544in}}%
\pgfpathlineto{\pgfqpoint{1.761550in}{0.637273in}}%
\pgfpathlineto{\pgfqpoint{1.765375in}{0.637273in}}%
\pgfpathlineto{\pgfqpoint{1.765403in}{0.914287in}}%
\pgfpathlineto{\pgfqpoint{1.766915in}{0.637273in}}%
\pgfpathlineto{\pgfqpoint{1.770905in}{0.637273in}}%
\pgfpathlineto{\pgfqpoint{1.770933in}{0.911544in}}%
\pgfpathlineto{\pgfqpoint{1.772444in}{0.637273in}}%
\pgfpathlineto{\pgfqpoint{1.776251in}{0.637273in}}%
\pgfpathlineto{\pgfqpoint{1.776274in}{0.914287in}}%
\pgfpathlineto{\pgfqpoint{1.777790in}{0.637273in}}%
\pgfpathlineto{\pgfqpoint{1.781984in}{0.637273in}}%
\pgfpathlineto{\pgfqpoint{1.782012in}{0.911544in}}%
\pgfpathlineto{\pgfqpoint{1.783523in}{0.637273in}}%
\pgfpathlineto{\pgfqpoint{1.787348in}{0.637273in}}%
\pgfpathlineto{\pgfqpoint{1.787376in}{0.914287in}}%
\pgfpathlineto{\pgfqpoint{1.788888in}{0.637273in}}%
\pgfpathlineto{\pgfqpoint{1.792873in}{0.637273in}}%
\pgfpathlineto{\pgfqpoint{1.792902in}{0.911544in}}%
\pgfpathlineto{\pgfqpoint{1.794413in}{0.637273in}}%
\pgfpathlineto{\pgfqpoint{1.798266in}{0.637273in}}%
\pgfpathlineto{\pgfqpoint{1.798294in}{0.914287in}}%
\pgfpathlineto{\pgfqpoint{1.799806in}{0.637273in}}%
\pgfpathlineto{\pgfqpoint{1.803753in}{0.637273in}}%
\pgfpathlineto{\pgfqpoint{1.803782in}{0.911544in}}%
\pgfpathlineto{\pgfqpoint{1.805293in}{0.637273in}}%
\pgfpathlineto{\pgfqpoint{1.809397in}{0.637273in}}%
\pgfpathlineto{\pgfqpoint{1.809425in}{0.914287in}}%
\pgfpathlineto{\pgfqpoint{1.810936in}{0.637273in}}%
\pgfpathlineto{\pgfqpoint{1.814841in}{0.637273in}}%
\pgfpathlineto{\pgfqpoint{1.814870in}{0.911544in}}%
\pgfpathlineto{\pgfqpoint{1.816381in}{0.637273in}}%
\pgfpathlineto{\pgfqpoint{1.820225in}{0.637273in}}%
\pgfpathlineto{\pgfqpoint{1.820253in}{0.914287in}}%
\pgfpathlineto{\pgfqpoint{1.821764in}{0.637273in}}%
\pgfpathlineto{\pgfqpoint{1.825646in}{0.637273in}}%
\pgfpathlineto{\pgfqpoint{1.825674in}{0.911544in}}%
\pgfpathlineto{\pgfqpoint{1.827186in}{0.637273in}}%
\pgfpathlineto{\pgfqpoint{1.831105in}{0.637273in}}%
\pgfpathlineto{\pgfqpoint{1.832177in}{0.914287in}}%
\pgfpathlineto{\pgfqpoint{1.832645in}{0.637273in}}%
\pgfpathlineto{\pgfqpoint{1.837792in}{0.637273in}}%
\pgfpathlineto{\pgfqpoint{1.837801in}{0.911544in}}%
\pgfpathlineto{\pgfqpoint{1.839331in}{0.637273in}}%
\pgfpathlineto{\pgfqpoint{1.843232in}{0.637273in}}%
\pgfpathlineto{\pgfqpoint{1.843307in}{0.914287in}}%
\pgfpathlineto{\pgfqpoint{1.844771in}{0.637273in}}%
\pgfpathlineto{\pgfqpoint{1.854107in}{0.637273in}}%
\pgfpathlineto{\pgfqpoint{1.854155in}{0.914287in}}%
\pgfpathlineto{\pgfqpoint{1.855647in}{0.637273in}}%
\pgfpathlineto{\pgfqpoint{1.859595in}{0.637273in}}%
\pgfpathlineto{\pgfqpoint{1.859623in}{0.911544in}}%
\pgfpathlineto{\pgfqpoint{1.861134in}{0.637273in}}%
\pgfpathlineto{\pgfqpoint{1.865238in}{0.637273in}}%
\pgfpathlineto{\pgfqpoint{1.865285in}{0.914287in}}%
\pgfpathlineto{\pgfqpoint{1.866777in}{0.637273in}}%
\pgfpathlineto{\pgfqpoint{1.870678in}{0.637273in}}%
\pgfpathlineto{\pgfqpoint{1.870706in}{0.911544in}}%
\pgfpathlineto{\pgfqpoint{1.872217in}{0.637273in}}%
\pgfpathlineto{\pgfqpoint{1.876161in}{0.637273in}}%
\pgfpathlineto{\pgfqpoint{1.876231in}{0.914287in}}%
\pgfpathlineto{\pgfqpoint{1.877700in}{0.637273in}}%
\pgfpathlineto{\pgfqpoint{1.881554in}{0.637273in}}%
\pgfpathlineto{\pgfqpoint{1.881686in}{0.911544in}}%
\pgfpathlineto{\pgfqpoint{1.883093in}{0.637273in}}%
\pgfpathlineto{\pgfqpoint{1.887022in}{0.637273in}}%
\pgfpathlineto{\pgfqpoint{1.887069in}{0.914287in}}%
\pgfpathlineto{\pgfqpoint{1.888561in}{0.637273in}}%
\pgfpathlineto{\pgfqpoint{1.892788in}{0.637273in}}%
\pgfpathlineto{\pgfqpoint{1.892816in}{0.911544in}}%
\pgfpathlineto{\pgfqpoint{1.894327in}{0.637273in}}%
\pgfpathlineto{\pgfqpoint{1.898152in}{0.637273in}}%
\pgfpathlineto{\pgfqpoint{1.898162in}{0.914287in}}%
\pgfpathlineto{\pgfqpoint{1.899692in}{0.637273in}}%
\pgfpathlineto{\pgfqpoint{1.903706in}{0.637273in}}%
\pgfpathlineto{\pgfqpoint{1.903734in}{0.911544in}}%
\pgfpathlineto{\pgfqpoint{1.905245in}{0.637273in}}%
\pgfpathlineto{\pgfqpoint{1.909070in}{0.637273in}}%
\pgfpathlineto{\pgfqpoint{1.909080in}{0.914287in}}%
\pgfpathlineto{\pgfqpoint{1.910610in}{0.637273in}}%
\pgfpathlineto{\pgfqpoint{1.914794in}{0.637273in}}%
\pgfpathlineto{\pgfqpoint{1.914869in}{0.914287in}}%
\pgfpathlineto{\pgfqpoint{1.916333in}{0.637273in}}%
\pgfpathlineto{\pgfqpoint{1.920234in}{0.637273in}}%
\pgfpathlineto{\pgfqpoint{1.920262in}{0.911544in}}%
\pgfpathlineto{\pgfqpoint{1.921773in}{0.637273in}}%
\pgfpathlineto{\pgfqpoint{1.926977in}{0.637273in}}%
\pgfpathlineto{\pgfqpoint{1.927015in}{0.936228in}}%
\pgfpathlineto{\pgfqpoint{1.928517in}{0.637273in}}%
\pgfpathlineto{\pgfqpoint{1.930519in}{0.637273in}}%
\pgfpathlineto{\pgfqpoint{1.931520in}{0.936228in}}%
\pgfpathlineto{\pgfqpoint{1.932059in}{0.637273in}}%
\pgfpathlineto{\pgfqpoint{1.935629in}{0.637273in}}%
\pgfpathlineto{\pgfqpoint{1.935851in}{0.936228in}}%
\pgfpathlineto{\pgfqpoint{1.937168in}{0.637273in}}%
\pgfpathlineto{\pgfqpoint{1.938566in}{0.637273in}}%
\pgfpathlineto{\pgfqpoint{1.938571in}{0.936228in}}%
\pgfpathlineto{\pgfqpoint{1.940105in}{0.637273in}}%
\pgfpathlineto{\pgfqpoint{1.940139in}{0.637273in}}%
\pgfpathlineto{\pgfqpoint{1.940176in}{0.936228in}}%
\pgfpathlineto{\pgfqpoint{1.941678in}{0.637273in}}%
\pgfpathlineto{\pgfqpoint{1.943029in}{0.637273in}}%
\pgfpathlineto{\pgfqpoint{1.943033in}{0.936228in}}%
\pgfpathlineto{\pgfqpoint{1.944568in}{0.637273in}}%
\pgfpathlineto{\pgfqpoint{1.944601in}{0.637273in}}%
\pgfpathlineto{\pgfqpoint{1.944639in}{0.936228in}}%
\pgfpathlineto{\pgfqpoint{1.946141in}{0.637273in}}%
\pgfpathlineto{\pgfqpoint{1.946528in}{0.637273in}}%
\pgfpathlineto{\pgfqpoint{1.946533in}{0.936228in}}%
\pgfpathlineto{\pgfqpoint{1.948067in}{0.637273in}}%
\pgfpathlineto{\pgfqpoint{1.953352in}{0.637273in}}%
\pgfpathlineto{\pgfqpoint{1.953389in}{0.936228in}}%
\pgfpathlineto{\pgfqpoint{1.954891in}{0.637273in}}%
\pgfpathlineto{\pgfqpoint{1.955217in}{0.637273in}}%
\pgfpathlineto{\pgfqpoint{1.956690in}{0.936228in}}%
\pgfpathlineto{\pgfqpoint{1.956756in}{0.637273in}}%
\pgfpathlineto{\pgfqpoint{1.957904in}{0.637273in}}%
\pgfpathlineto{\pgfqpoint{1.958064in}{3.382727in}}%
\pgfpathlineto{\pgfqpoint{1.959443in}{0.637273in}}%
\pgfpathlineto{\pgfqpoint{1.962815in}{0.637273in}}%
\pgfpathlineto{\pgfqpoint{1.963179in}{0.919772in}}%
\pgfpathlineto{\pgfqpoint{1.964355in}{0.637273in}}%
\pgfpathlineto{\pgfqpoint{1.967353in}{0.637273in}}%
\pgfpathlineto{\pgfqpoint{1.968897in}{3.382727in}}%
\pgfpathlineto{\pgfqpoint{1.981388in}{3.382727in}}%
\pgfpathlineto{\pgfqpoint{1.981582in}{0.637273in}}%
\pgfpathlineto{\pgfqpoint{1.982927in}{3.382727in}}%
\pgfpathlineto{\pgfqpoint{1.985369in}{3.382727in}}%
\pgfpathlineto{\pgfqpoint{1.986238in}{0.637273in}}%
\pgfpathlineto{\pgfqpoint{1.986908in}{3.382727in}}%
\pgfpathlineto{\pgfqpoint{1.986951in}{3.382727in}}%
\pgfpathlineto{\pgfqpoint{1.988495in}{0.637273in}}%
\pgfpathlineto{\pgfqpoint{1.988514in}{0.637273in}}%
\pgfpathlineto{\pgfqpoint{1.988523in}{0.919772in}}%
\pgfpathlineto{\pgfqpoint{1.990053in}{0.637273in}}%
\pgfpathlineto{\pgfqpoint{2.050079in}{0.637273in}}%
\pgfpathlineto{\pgfqpoint{2.050400in}{3.382727in}}%
\pgfpathlineto{\pgfqpoint{2.051618in}{0.637273in}}%
\pgfpathlineto{\pgfqpoint{2.052572in}{0.637273in}}%
\pgfpathlineto{\pgfqpoint{2.053965in}{3.382727in}}%
\pgfpathlineto{\pgfqpoint{2.054112in}{0.637273in}}%
\pgfpathlineto{\pgfqpoint{2.054135in}{0.637273in}}%
\pgfpathlineto{\pgfqpoint{2.054367in}{0.919772in}}%
\pgfpathlineto{\pgfqpoint{2.055675in}{0.637273in}}%
\pgfpathlineto{\pgfqpoint{2.056199in}{0.637273in}}%
\pgfpathlineto{\pgfqpoint{2.056208in}{0.919772in}}%
\pgfpathlineto{\pgfqpoint{2.057738in}{0.637273in}}%
\pgfpathlineto{\pgfqpoint{2.074554in}{0.637273in}}%
\pgfpathlineto{\pgfqpoint{2.074739in}{3.382727in}}%
\pgfpathlineto{\pgfqpoint{2.076094in}{0.637273in}}%
\pgfpathlineto{\pgfqpoint{2.080339in}{0.637273in}}%
\pgfpathlineto{\pgfqpoint{2.081850in}{3.382727in}}%
\pgfpathlineto{\pgfqpoint{2.081879in}{0.637273in}}%
\pgfpathlineto{\pgfqpoint{2.081950in}{0.637273in}}%
\pgfpathlineto{\pgfqpoint{2.083192in}{3.382727in}}%
\pgfpathlineto{\pgfqpoint{2.083489in}{0.637273in}}%
\pgfpathlineto{\pgfqpoint{2.089689in}{0.637273in}}%
\pgfpathlineto{\pgfqpoint{2.089737in}{0.914287in}}%
\pgfpathlineto{\pgfqpoint{2.091229in}{0.637273in}}%
\pgfpathlineto{\pgfqpoint{2.095432in}{0.637273in}}%
\pgfpathlineto{\pgfqpoint{2.095460in}{0.911544in}}%
\pgfpathlineto{\pgfqpoint{2.096971in}{0.637273in}}%
\pgfpathlineto{\pgfqpoint{2.100891in}{0.637273in}}%
\pgfpathlineto{\pgfqpoint{2.100947in}{0.914287in}}%
\pgfpathlineto{\pgfqpoint{2.102430in}{0.637273in}}%
\pgfpathlineto{\pgfqpoint{2.107979in}{0.637273in}}%
\pgfpathlineto{\pgfqpoint{2.108007in}{0.911544in}}%
\pgfpathlineto{\pgfqpoint{2.109518in}{0.637273in}}%
\pgfpathlineto{\pgfqpoint{2.113490in}{0.637273in}}%
\pgfpathlineto{\pgfqpoint{2.113547in}{0.914287in}}%
\pgfpathlineto{\pgfqpoint{2.115029in}{0.637273in}}%
\pgfpathlineto{\pgfqpoint{2.118911in}{0.637273in}}%
\pgfpathlineto{\pgfqpoint{2.118939in}{0.911544in}}%
\pgfpathlineto{\pgfqpoint{2.120451in}{0.637273in}}%
\pgfpathlineto{\pgfqpoint{2.124729in}{0.637273in}}%
\pgfpathlineto{\pgfqpoint{2.124847in}{0.914287in}}%
\pgfpathlineto{\pgfqpoint{2.126268in}{0.637273in}}%
\pgfpathlineto{\pgfqpoint{2.130972in}{0.637273in}}%
\pgfpathlineto{\pgfqpoint{2.131000in}{0.911544in}}%
\pgfpathlineto{\pgfqpoint{2.132511in}{0.637273in}}%
\pgfpathlineto{\pgfqpoint{2.136436in}{0.637273in}}%
\pgfpathlineto{\pgfqpoint{2.136492in}{0.914287in}}%
\pgfpathlineto{\pgfqpoint{2.137975in}{0.637273in}}%
\pgfpathlineto{\pgfqpoint{2.141895in}{0.637273in}}%
\pgfpathlineto{\pgfqpoint{2.142017in}{0.911544in}}%
\pgfpathlineto{\pgfqpoint{2.143434in}{0.637273in}}%
\pgfpathlineto{\pgfqpoint{2.147382in}{0.637273in}}%
\pgfpathlineto{\pgfqpoint{2.148454in}{0.914287in}}%
\pgfpathlineto{\pgfqpoint{2.148921in}{0.637273in}}%
\pgfpathlineto{\pgfqpoint{2.154088in}{0.637273in}}%
\pgfpathlineto{\pgfqpoint{2.154116in}{0.911544in}}%
\pgfpathlineto{\pgfqpoint{2.155627in}{0.637273in}}%
\pgfpathlineto{\pgfqpoint{2.159528in}{0.637273in}}%
\pgfpathlineto{\pgfqpoint{2.159603in}{0.914287in}}%
\pgfpathlineto{\pgfqpoint{2.161067in}{0.637273in}}%
\pgfpathlineto{\pgfqpoint{2.164944in}{0.637273in}}%
\pgfpathlineto{\pgfqpoint{2.164973in}{0.911544in}}%
\pgfpathlineto{\pgfqpoint{2.166484in}{0.637273in}}%
\pgfpathlineto{\pgfqpoint{2.170403in}{0.637273in}}%
\pgfpathlineto{\pgfqpoint{2.170498in}{0.914287in}}%
\pgfpathlineto{\pgfqpoint{2.171943in}{0.637273in}}%
\pgfpathlineto{\pgfqpoint{2.175919in}{0.637273in}}%
\pgfpathlineto{\pgfqpoint{2.175947in}{0.911544in}}%
\pgfpathlineto{\pgfqpoint{2.177458in}{0.637273in}}%
\pgfpathlineto{\pgfqpoint{2.181562in}{0.637273in}}%
\pgfpathlineto{\pgfqpoint{2.181628in}{0.914287in}}%
\pgfpathlineto{\pgfqpoint{2.183102in}{0.637273in}}%
\pgfpathlineto{\pgfqpoint{2.187049in}{0.637273in}}%
\pgfpathlineto{\pgfqpoint{2.187078in}{0.911544in}}%
\pgfpathlineto{\pgfqpoint{2.188589in}{0.637273in}}%
\pgfpathlineto{\pgfqpoint{2.192485in}{0.637273in}}%
\pgfpathlineto{\pgfqpoint{2.192551in}{0.914287in}}%
\pgfpathlineto{\pgfqpoint{2.194024in}{0.637273in}}%
\pgfpathlineto{\pgfqpoint{2.197868in}{0.637273in}}%
\pgfpathlineto{\pgfqpoint{2.197897in}{0.911544in}}%
\pgfpathlineto{\pgfqpoint{2.199408in}{0.637273in}}%
\pgfpathlineto{\pgfqpoint{2.205060in}{0.637273in}}%
\pgfpathlineto{\pgfqpoint{2.205155in}{0.914287in}}%
\pgfpathlineto{\pgfqpoint{2.206600in}{0.637273in}}%
\pgfpathlineto{\pgfqpoint{2.210798in}{0.637273in}}%
\pgfpathlineto{\pgfqpoint{2.210826in}{0.911544in}}%
\pgfpathlineto{\pgfqpoint{2.212337in}{0.637273in}}%
\pgfpathlineto{\pgfqpoint{2.216191in}{0.637273in}}%
\pgfpathlineto{\pgfqpoint{2.216257in}{0.914287in}}%
\pgfpathlineto{\pgfqpoint{2.217730in}{0.637273in}}%
\pgfpathlineto{\pgfqpoint{2.221688in}{0.637273in}}%
\pgfpathlineto{\pgfqpoint{2.221716in}{0.911544in}}%
\pgfpathlineto{\pgfqpoint{2.223227in}{0.637273in}}%
\pgfpathlineto{\pgfqpoint{2.227109in}{0.637273in}}%
\pgfpathlineto{\pgfqpoint{2.227203in}{0.914287in}}%
\pgfpathlineto{\pgfqpoint{2.228648in}{0.637273in}}%
\pgfpathlineto{\pgfqpoint{2.232539in}{0.637273in}}%
\pgfpathlineto{\pgfqpoint{2.232568in}{0.911544in}}%
\pgfpathlineto{\pgfqpoint{2.234079in}{0.637273in}}%
\pgfpathlineto{\pgfqpoint{2.238211in}{0.637273in}}%
\pgfpathlineto{\pgfqpoint{2.238305in}{0.914287in}}%
\pgfpathlineto{\pgfqpoint{2.239750in}{0.637273in}}%
\pgfpathlineto{\pgfqpoint{2.243670in}{0.637273in}}%
\pgfpathlineto{\pgfqpoint{2.243774in}{0.911544in}}%
\pgfpathlineto{\pgfqpoint{2.245209in}{0.637273in}}%
\pgfpathlineto{\pgfqpoint{2.250423in}{0.637273in}}%
\pgfpathlineto{\pgfqpoint{2.250749in}{0.936228in}}%
\pgfpathlineto{\pgfqpoint{2.251962in}{0.637273in}}%
\pgfpathlineto{\pgfqpoint{2.256415in}{0.637273in}}%
\pgfpathlineto{\pgfqpoint{2.256444in}{0.911544in}}%
\pgfpathlineto{\pgfqpoint{2.257955in}{0.637273in}}%
\pgfpathlineto{\pgfqpoint{2.261799in}{0.637273in}}%
\pgfpathlineto{\pgfqpoint{2.261846in}{0.914287in}}%
\pgfpathlineto{\pgfqpoint{2.263338in}{0.637273in}}%
\pgfpathlineto{\pgfqpoint{2.267503in}{0.637273in}}%
\pgfpathlineto{\pgfqpoint{2.267532in}{0.911544in}}%
\pgfpathlineto{\pgfqpoint{2.269043in}{0.637273in}}%
\pgfpathlineto{\pgfqpoint{2.272840in}{0.637273in}}%
\pgfpathlineto{\pgfqpoint{2.272887in}{0.914287in}}%
\pgfpathlineto{\pgfqpoint{2.274379in}{0.637273in}}%
\pgfpathlineto{\pgfqpoint{2.278388in}{0.637273in}}%
\pgfpathlineto{\pgfqpoint{2.278417in}{0.911544in}}%
\pgfpathlineto{\pgfqpoint{2.279928in}{0.637273in}}%
\pgfpathlineto{\pgfqpoint{2.283791in}{0.637273in}}%
\pgfpathlineto{\pgfqpoint{2.283819in}{0.914287in}}%
\pgfpathlineto{\pgfqpoint{2.285330in}{0.637273in}}%
\pgfpathlineto{\pgfqpoint{2.289538in}{0.637273in}}%
\pgfpathlineto{\pgfqpoint{2.289566in}{0.911544in}}%
\pgfpathlineto{\pgfqpoint{2.291077in}{0.637273in}}%
\pgfpathlineto{\pgfqpoint{2.294902in}{0.637273in}}%
\pgfpathlineto{\pgfqpoint{2.294949in}{0.914287in}}%
\pgfpathlineto{\pgfqpoint{2.296442in}{0.637273in}}%
\pgfpathlineto{\pgfqpoint{2.300423in}{0.637273in}}%
\pgfpathlineto{\pgfqpoint{2.300451in}{0.911544in}}%
\pgfpathlineto{\pgfqpoint{2.301962in}{0.637273in}}%
\pgfpathlineto{\pgfqpoint{2.305797in}{0.637273in}}%
\pgfpathlineto{\pgfqpoint{2.305825in}{0.914287in}}%
\pgfpathlineto{\pgfqpoint{2.307336in}{0.637273in}}%
\pgfpathlineto{\pgfqpoint{2.311246in}{0.637273in}}%
\pgfpathlineto{\pgfqpoint{2.311275in}{0.911544in}}%
\pgfpathlineto{\pgfqpoint{2.312786in}{0.637273in}}%
\pgfpathlineto{\pgfqpoint{2.316564in}{0.637273in}}%
\pgfpathlineto{\pgfqpoint{2.317654in}{0.914287in}}%
\pgfpathlineto{\pgfqpoint{2.318103in}{0.637273in}}%
\pgfpathlineto{\pgfqpoint{2.323321in}{0.637273in}}%
\pgfpathlineto{\pgfqpoint{2.323416in}{0.914287in}}%
\pgfpathlineto{\pgfqpoint{2.324861in}{0.637273in}}%
\pgfpathlineto{\pgfqpoint{2.328733in}{0.637273in}}%
\pgfpathlineto{\pgfqpoint{2.328761in}{0.911544in}}%
\pgfpathlineto{\pgfqpoint{2.330272in}{0.637273in}}%
\pgfpathlineto{\pgfqpoint{2.334145in}{0.637273in}}%
\pgfpathlineto{\pgfqpoint{2.334239in}{0.914287in}}%
\pgfpathlineto{\pgfqpoint{2.335684in}{0.637273in}}%
\pgfpathlineto{\pgfqpoint{2.339556in}{0.637273in}}%
\pgfpathlineto{\pgfqpoint{2.339585in}{0.911544in}}%
\pgfpathlineto{\pgfqpoint{2.341096in}{0.637273in}}%
\pgfpathlineto{\pgfqpoint{2.345289in}{0.637273in}}%
\pgfpathlineto{\pgfqpoint{2.345677in}{0.914287in}}%
\pgfpathlineto{\pgfqpoint{2.346829in}{0.637273in}}%
\pgfpathlineto{\pgfqpoint{2.351494in}{0.637273in}}%
\pgfpathlineto{\pgfqpoint{2.351523in}{0.911544in}}%
\pgfpathlineto{\pgfqpoint{2.353034in}{0.637273in}}%
\pgfpathlineto{\pgfqpoint{2.356868in}{0.637273in}}%
\pgfpathlineto{\pgfqpoint{2.356925in}{0.914287in}}%
\pgfpathlineto{\pgfqpoint{2.358408in}{0.637273in}}%
\pgfpathlineto{\pgfqpoint{2.362365in}{0.637273in}}%
\pgfpathlineto{\pgfqpoint{2.362394in}{0.911544in}}%
\pgfpathlineto{\pgfqpoint{2.363905in}{0.637273in}}%
\pgfpathlineto{\pgfqpoint{2.367786in}{0.637273in}}%
\pgfpathlineto{\pgfqpoint{2.367843in}{0.914287in}}%
\pgfpathlineto{\pgfqpoint{2.369326in}{0.637273in}}%
\pgfpathlineto{\pgfqpoint{2.373510in}{0.637273in}}%
\pgfpathlineto{\pgfqpoint{2.373538in}{0.911544in}}%
\pgfpathlineto{\pgfqpoint{2.375049in}{0.637273in}}%
\pgfpathlineto{\pgfqpoint{2.378931in}{0.637273in}}%
\pgfpathlineto{\pgfqpoint{2.380423in}{0.914287in}}%
\pgfpathlineto{\pgfqpoint{2.380471in}{0.637273in}}%
\pgfpathlineto{\pgfqpoint{2.380475in}{0.637273in}}%
\pgfpathlineto{\pgfqpoint{2.380480in}{0.914287in}}%
\pgfpathlineto{\pgfqpoint{2.382015in}{0.637273in}}%
\pgfpathlineto{\pgfqpoint{2.385849in}{0.637273in}}%
\pgfpathlineto{\pgfqpoint{2.385878in}{0.911544in}}%
\pgfpathlineto{\pgfqpoint{2.387389in}{0.637273in}}%
\pgfpathlineto{\pgfqpoint{2.391266in}{0.637273in}}%
\pgfpathlineto{\pgfqpoint{2.391360in}{0.914287in}}%
\pgfpathlineto{\pgfqpoint{2.392805in}{0.637273in}}%
\pgfpathlineto{\pgfqpoint{2.396678in}{0.637273in}}%
\pgfpathlineto{\pgfqpoint{2.396706in}{0.911544in}}%
\pgfpathlineto{\pgfqpoint{2.398217in}{0.637273in}}%
\pgfpathlineto{\pgfqpoint{2.402099in}{0.637273in}}%
\pgfpathlineto{\pgfqpoint{2.402439in}{0.914287in}}%
\pgfpathlineto{\pgfqpoint{2.403638in}{0.637273in}}%
\pgfpathlineto{\pgfqpoint{2.409140in}{0.637273in}}%
\pgfpathlineto{\pgfqpoint{2.409489in}{0.936228in}}%
\pgfpathlineto{\pgfqpoint{2.410679in}{0.637273in}}%
\pgfpathlineto{\pgfqpoint{2.415047in}{0.637273in}}%
\pgfpathlineto{\pgfqpoint{2.415113in}{0.914287in}}%
\pgfpathlineto{\pgfqpoint{2.416587in}{0.637273in}}%
\pgfpathlineto{\pgfqpoint{2.422688in}{0.637273in}}%
\pgfpathlineto{\pgfqpoint{2.423420in}{0.936228in}}%
\pgfpathlineto{\pgfqpoint{2.424228in}{0.637273in}}%
\pgfpathlineto{\pgfqpoint{2.426183in}{0.637273in}}%
\pgfpathlineto{\pgfqpoint{2.427684in}{0.936228in}}%
\pgfpathlineto{\pgfqpoint{2.427722in}{0.637273in}}%
\pgfpathlineto{\pgfqpoint{2.430546in}{0.637273in}}%
\pgfpathlineto{\pgfqpoint{2.430551in}{0.936228in}}%
\pgfpathlineto{\pgfqpoint{2.432085in}{0.637273in}}%
\pgfpathlineto{\pgfqpoint{2.432119in}{0.637273in}}%
\pgfpathlineto{\pgfqpoint{2.432156in}{0.936228in}}%
\pgfpathlineto{\pgfqpoint{2.433658in}{0.637273in}}%
\pgfpathlineto{\pgfqpoint{2.434900in}{0.637273in}}%
\pgfpathlineto{\pgfqpoint{2.434905in}{0.936228in}}%
\pgfpathlineto{\pgfqpoint{2.436439in}{0.637273in}}%
\pgfpathlineto{\pgfqpoint{2.441842in}{0.637273in}}%
\pgfpathlineto{\pgfqpoint{2.441880in}{0.936228in}}%
\pgfpathlineto{\pgfqpoint{2.443381in}{0.637273in}}%
\pgfpathlineto{\pgfqpoint{2.444595in}{0.637273in}}%
\pgfpathlineto{\pgfqpoint{2.444600in}{0.936228in}}%
\pgfpathlineto{\pgfqpoint{2.446134in}{0.637273in}}%
\pgfpathlineto{\pgfqpoint{2.446167in}{0.637273in}}%
\pgfpathlineto{\pgfqpoint{2.446205in}{0.936228in}}%
\pgfpathlineto{\pgfqpoint{2.447707in}{0.637273in}}%
\pgfpathlineto{\pgfqpoint{2.448925in}{0.637273in}}%
\pgfpathlineto{\pgfqpoint{2.450399in}{0.936228in}}%
\pgfpathlineto{\pgfqpoint{2.450465in}{0.637273in}}%
\pgfpathlineto{\pgfqpoint{2.453147in}{0.637273in}}%
\pgfpathlineto{\pgfqpoint{2.453152in}{0.936228in}}%
\pgfpathlineto{\pgfqpoint{2.454686in}{0.637273in}}%
\pgfpathlineto{\pgfqpoint{2.454719in}{0.637273in}}%
\pgfpathlineto{\pgfqpoint{2.454757in}{0.936228in}}%
\pgfpathlineto{\pgfqpoint{2.456259in}{0.637273in}}%
\pgfpathlineto{\pgfqpoint{2.457421in}{0.637273in}}%
\pgfpathlineto{\pgfqpoint{2.457685in}{0.936228in}}%
\pgfpathlineto{\pgfqpoint{2.458960in}{0.637273in}}%
\pgfpathlineto{\pgfqpoint{2.459253in}{0.637273in}}%
\pgfpathlineto{\pgfqpoint{2.459291in}{0.936228in}}%
\pgfpathlineto{\pgfqpoint{2.460792in}{0.637273in}}%
\pgfpathlineto{\pgfqpoint{2.462011in}{0.637273in}}%
\pgfpathlineto{\pgfqpoint{2.462015in}{0.936228in}}%
\pgfpathlineto{\pgfqpoint{2.463550in}{0.637273in}}%
\pgfpathlineto{\pgfqpoint{2.463612in}{0.637273in}}%
\pgfpathlineto{\pgfqpoint{2.463649in}{0.936228in}}%
\pgfpathlineto{\pgfqpoint{2.465151in}{0.637273in}}%
\pgfpathlineto{\pgfqpoint{2.472216in}{0.637273in}}%
\pgfpathlineto{\pgfqpoint{2.473689in}{0.936228in}}%
\pgfpathlineto{\pgfqpoint{2.473755in}{0.637273in}}%
\pgfpathlineto{\pgfqpoint{2.476541in}{0.637273in}}%
\pgfpathlineto{\pgfqpoint{2.478015in}{0.936228in}}%
\pgfpathlineto{\pgfqpoint{2.478081in}{0.637273in}}%
\pgfpathlineto{\pgfqpoint{2.480735in}{0.637273in}}%
\pgfpathlineto{\pgfqpoint{2.482208in}{0.936228in}}%
\pgfpathlineto{\pgfqpoint{2.482274in}{0.637273in}}%
\pgfpathlineto{\pgfqpoint{2.485093in}{0.637273in}}%
\pgfpathlineto{\pgfqpoint{2.485098in}{0.936228in}}%
\pgfpathlineto{\pgfqpoint{2.486633in}{0.637273in}}%
\pgfpathlineto{\pgfqpoint{2.486666in}{0.637273in}}%
\pgfpathlineto{\pgfqpoint{2.486704in}{0.936228in}}%
\pgfpathlineto{\pgfqpoint{2.488205in}{0.637273in}}%
\pgfpathlineto{\pgfqpoint{2.489617in}{0.637273in}}%
\pgfpathlineto{\pgfqpoint{2.489622in}{0.936228in}}%
\pgfpathlineto{\pgfqpoint{2.491157in}{0.637273in}}%
\pgfpathlineto{\pgfqpoint{2.491190in}{0.637273in}}%
\pgfpathlineto{\pgfqpoint{2.491228in}{0.936228in}}%
\pgfpathlineto{\pgfqpoint{2.492729in}{0.637273in}}%
\pgfpathlineto{\pgfqpoint{2.493976in}{0.637273in}}%
\pgfpathlineto{\pgfqpoint{2.493981in}{0.936228in}}%
\pgfpathlineto{\pgfqpoint{2.495516in}{0.637273in}}%
\pgfpathlineto{\pgfqpoint{2.495544in}{0.637273in}}%
\pgfpathlineto{\pgfqpoint{2.495586in}{0.936228in}}%
\pgfpathlineto{\pgfqpoint{2.497083in}{0.637273in}}%
\pgfpathlineto{\pgfqpoint{2.503765in}{0.637273in}}%
\pgfpathlineto{\pgfqpoint{2.503770in}{0.936228in}}%
\pgfpathlineto{\pgfqpoint{2.505305in}{0.637273in}}%
\pgfpathlineto{\pgfqpoint{2.505338in}{0.637273in}}%
\pgfpathlineto{\pgfqpoint{2.505376in}{0.936228in}}%
\pgfpathlineto{\pgfqpoint{2.506877in}{0.637273in}}%
\pgfpathlineto{\pgfqpoint{2.508096in}{0.637273in}}%
\pgfpathlineto{\pgfqpoint{2.508101in}{0.936228in}}%
\pgfpathlineto{\pgfqpoint{2.509635in}{0.637273in}}%
\pgfpathlineto{\pgfqpoint{2.509668in}{0.637273in}}%
\pgfpathlineto{\pgfqpoint{2.509706in}{0.936228in}}%
\pgfpathlineto{\pgfqpoint{2.511208in}{0.637273in}}%
\pgfpathlineto{\pgfqpoint{2.512426in}{0.637273in}}%
\pgfpathlineto{\pgfqpoint{2.512431in}{0.936228in}}%
\pgfpathlineto{\pgfqpoint{2.513966in}{0.637273in}}%
\pgfpathlineto{\pgfqpoint{2.513989in}{0.637273in}}%
\pgfpathlineto{\pgfqpoint{2.514036in}{0.936228in}}%
\pgfpathlineto{\pgfqpoint{2.515529in}{0.637273in}}%
\pgfpathlineto{\pgfqpoint{2.516785in}{0.637273in}}%
\pgfpathlineto{\pgfqpoint{2.516790in}{0.936228in}}%
\pgfpathlineto{\pgfqpoint{2.518324in}{0.637273in}}%
\pgfpathlineto{\pgfqpoint{2.518423in}{0.637273in}}%
\pgfpathlineto{\pgfqpoint{2.518461in}{0.936228in}}%
\pgfpathlineto{\pgfqpoint{2.519963in}{0.637273in}}%
\pgfpathlineto{\pgfqpoint{2.521181in}{0.637273in}}%
\pgfpathlineto{\pgfqpoint{2.522655in}{0.936228in}}%
\pgfpathlineto{\pgfqpoint{2.522721in}{0.637273in}}%
\pgfpathlineto{\pgfqpoint{2.525403in}{0.637273in}}%
\pgfpathlineto{\pgfqpoint{2.526876in}{0.936228in}}%
\pgfpathlineto{\pgfqpoint{2.526943in}{0.637273in}}%
\pgfpathlineto{\pgfqpoint{2.535051in}{0.637273in}}%
\pgfpathlineto{\pgfqpoint{2.535055in}{0.936228in}}%
\pgfpathlineto{\pgfqpoint{2.536590in}{0.637273in}}%
\pgfpathlineto{\pgfqpoint{2.536628in}{0.637273in}}%
\pgfpathlineto{\pgfqpoint{2.536666in}{0.936228in}}%
\pgfpathlineto{\pgfqpoint{2.538167in}{0.637273in}}%
\pgfpathlineto{\pgfqpoint{2.539518in}{0.637273in}}%
\pgfpathlineto{\pgfqpoint{2.540991in}{0.936228in}}%
\pgfpathlineto{\pgfqpoint{2.541058in}{0.637273in}}%
\pgfpathlineto{\pgfqpoint{2.543707in}{0.637273in}}%
\pgfpathlineto{\pgfqpoint{2.545218in}{0.936228in}}%
\pgfpathlineto{\pgfqpoint{2.545246in}{0.637273in}}%
\pgfpathlineto{\pgfqpoint{2.547754in}{0.637273in}}%
\pgfpathlineto{\pgfqpoint{2.548023in}{0.936228in}}%
\pgfpathlineto{\pgfqpoint{2.549293in}{0.637273in}}%
\pgfpathlineto{\pgfqpoint{2.549454in}{0.637273in}}%
\pgfpathlineto{\pgfqpoint{2.549492in}{0.936228in}}%
\pgfpathlineto{\pgfqpoint{2.550993in}{0.637273in}}%
\pgfpathlineto{\pgfqpoint{2.552207in}{0.637273in}}%
\pgfpathlineto{\pgfqpoint{2.553680in}{0.936228in}}%
\pgfpathlineto{\pgfqpoint{2.553746in}{0.637273in}}%
\pgfpathlineto{\pgfqpoint{2.556424in}{0.637273in}}%
\pgfpathlineto{\pgfqpoint{2.557897in}{0.936228in}}%
\pgfpathlineto{\pgfqpoint{2.557963in}{0.637273in}}%
\pgfpathlineto{\pgfqpoint{2.566185in}{0.637273in}}%
\pgfpathlineto{\pgfqpoint{2.566190in}{0.936228in}}%
\pgfpathlineto{\pgfqpoint{2.567724in}{0.637273in}}%
\pgfpathlineto{\pgfqpoint{2.567786in}{0.637273in}}%
\pgfpathlineto{\pgfqpoint{2.567824in}{0.936228in}}%
\pgfpathlineto{\pgfqpoint{2.569325in}{0.637273in}}%
\pgfpathlineto{\pgfqpoint{2.570544in}{0.637273in}}%
\pgfpathlineto{\pgfqpoint{2.570548in}{0.936228in}}%
\pgfpathlineto{\pgfqpoint{2.572083in}{0.637273in}}%
\pgfpathlineto{\pgfqpoint{2.572116in}{0.637273in}}%
\pgfpathlineto{\pgfqpoint{2.572154in}{0.936228in}}%
\pgfpathlineto{\pgfqpoint{2.573656in}{0.637273in}}%
\pgfpathlineto{\pgfqpoint{2.575025in}{0.637273in}}%
\pgfpathlineto{\pgfqpoint{2.576498in}{0.936228in}}%
\pgfpathlineto{\pgfqpoint{2.576565in}{0.637273in}}%
\pgfpathlineto{\pgfqpoint{2.579030in}{0.637273in}}%
\pgfpathlineto{\pgfqpoint{2.579313in}{0.936228in}}%
\pgfpathlineto{\pgfqpoint{2.580569in}{0.637273in}}%
\pgfpathlineto{\pgfqpoint{2.580744in}{0.637273in}}%
\pgfpathlineto{\pgfqpoint{2.580782in}{0.936228in}}%
\pgfpathlineto{\pgfqpoint{2.582283in}{0.637273in}}%
\pgfpathlineto{\pgfqpoint{2.583639in}{0.637273in}}%
\pgfpathlineto{\pgfqpoint{2.585140in}{0.936228in}}%
\pgfpathlineto{\pgfqpoint{2.585178in}{0.637273in}}%
\pgfpathlineto{\pgfqpoint{2.587856in}{0.637273in}}%
\pgfpathlineto{\pgfqpoint{2.589329in}{0.936228in}}%
\pgfpathlineto{\pgfqpoint{2.589395in}{0.637273in}}%
\pgfpathlineto{\pgfqpoint{2.597503in}{0.637273in}}%
\pgfpathlineto{\pgfqpoint{2.598977in}{0.936228in}}%
\pgfpathlineto{\pgfqpoint{2.599043in}{0.637273in}}%
\pgfpathlineto{\pgfqpoint{2.601692in}{0.637273in}}%
\pgfpathlineto{\pgfqpoint{2.603165in}{0.936228in}}%
\pgfpathlineto{\pgfqpoint{2.603232in}{0.637273in}}%
\pgfpathlineto{\pgfqpoint{2.605881in}{0.637273in}}%
\pgfpathlineto{\pgfqpoint{2.607354in}{0.936228in}}%
\pgfpathlineto{\pgfqpoint{2.607420in}{0.637273in}}%
\pgfpathlineto{\pgfqpoint{2.609215in}{0.637273in}}%
\pgfpathlineto{\pgfqpoint{2.609229in}{0.911544in}}%
\pgfpathlineto{\pgfqpoint{2.610754in}{0.637273in}}%
\pgfpathlineto{\pgfqpoint{2.627953in}{0.637273in}}%
\pgfpathlineto{\pgfqpoint{2.627958in}{0.936228in}}%
\pgfpathlineto{\pgfqpoint{2.629492in}{0.637273in}}%
\pgfpathlineto{\pgfqpoint{2.653472in}{0.637273in}}%
\pgfpathlineto{\pgfqpoint{2.653510in}{0.936228in}}%
\pgfpathlineto{\pgfqpoint{2.655012in}{0.637273in}}%
\pgfpathlineto{\pgfqpoint{2.663238in}{0.637273in}}%
\pgfpathlineto{\pgfqpoint{2.664740in}{0.936228in}}%
\pgfpathlineto{\pgfqpoint{2.664777in}{0.637273in}}%
\pgfpathlineto{\pgfqpoint{2.667715in}{0.637273in}}%
\pgfpathlineto{\pgfqpoint{2.667719in}{0.936228in}}%
\pgfpathlineto{\pgfqpoint{2.669254in}{0.637273in}}%
\pgfpathlineto{\pgfqpoint{2.669292in}{0.637273in}}%
\pgfpathlineto{\pgfqpoint{2.669330in}{0.936228in}}%
\pgfpathlineto{\pgfqpoint{2.670831in}{0.637273in}}%
\pgfpathlineto{\pgfqpoint{2.677523in}{0.637273in}}%
\pgfpathlineto{\pgfqpoint{2.677528in}{0.936228in}}%
\pgfpathlineto{\pgfqpoint{2.679062in}{0.637273in}}%
\pgfpathlineto{\pgfqpoint{2.679129in}{0.637273in}}%
\pgfpathlineto{\pgfqpoint{2.679166in}{0.936228in}}%
\pgfpathlineto{\pgfqpoint{2.680668in}{0.637273in}}%
\pgfpathlineto{\pgfqpoint{2.681882in}{0.637273in}}%
\pgfpathlineto{\pgfqpoint{2.683355in}{0.936228in}}%
\pgfpathlineto{\pgfqpoint{2.683421in}{0.637273in}}%
\pgfpathlineto{\pgfqpoint{2.686070in}{0.637273in}}%
\pgfpathlineto{\pgfqpoint{2.687548in}{0.936228in}}%
\pgfpathlineto{\pgfqpoint{2.687610in}{0.637273in}}%
\pgfpathlineto{\pgfqpoint{2.690283in}{0.637273in}}%
\pgfpathlineto{\pgfqpoint{2.691756in}{0.936228in}}%
\pgfpathlineto{\pgfqpoint{2.691822in}{0.637273in}}%
\pgfpathlineto{\pgfqpoint{2.694504in}{0.637273in}}%
\pgfpathlineto{\pgfqpoint{2.696006in}{0.936228in}}%
\pgfpathlineto{\pgfqpoint{2.696044in}{0.637273in}}%
\pgfpathlineto{\pgfqpoint{2.698721in}{0.637273in}}%
\pgfpathlineto{\pgfqpoint{2.700195in}{0.936228in}}%
\pgfpathlineto{\pgfqpoint{2.700261in}{0.637273in}}%
\pgfpathlineto{\pgfqpoint{2.708388in}{0.637273in}}%
\pgfpathlineto{\pgfqpoint{2.708393in}{0.936228in}}%
\pgfpathlineto{\pgfqpoint{2.709927in}{0.637273in}}%
\pgfpathlineto{\pgfqpoint{2.709965in}{0.637273in}}%
\pgfpathlineto{\pgfqpoint{2.710003in}{0.936228in}}%
\pgfpathlineto{\pgfqpoint{2.711505in}{0.637273in}}%
\pgfpathlineto{\pgfqpoint{2.712747in}{0.637273in}}%
\pgfpathlineto{\pgfqpoint{2.712751in}{0.936228in}}%
\pgfpathlineto{\pgfqpoint{2.714286in}{0.637273in}}%
\pgfpathlineto{\pgfqpoint{2.714314in}{0.637273in}}%
\pgfpathlineto{\pgfqpoint{2.714357in}{0.936228in}}%
\pgfpathlineto{\pgfqpoint{2.715854in}{0.637273in}}%
\pgfpathlineto{\pgfqpoint{2.717077in}{0.637273in}}%
\pgfpathlineto{\pgfqpoint{2.718550in}{0.936228in}}%
\pgfpathlineto{\pgfqpoint{2.718616in}{0.637273in}}%
\pgfpathlineto{\pgfqpoint{2.720416in}{0.637273in}}%
\pgfpathlineto{\pgfqpoint{2.721511in}{0.936228in}}%
\pgfpathlineto{\pgfqpoint{2.721955in}{0.637273in}}%
\pgfpathlineto{\pgfqpoint{2.722947in}{0.637273in}}%
\pgfpathlineto{\pgfqpoint{2.722985in}{0.936228in}}%
\pgfpathlineto{\pgfqpoint{2.724486in}{0.637273in}}%
\pgfpathlineto{\pgfqpoint{2.725728in}{0.637273in}}%
\pgfpathlineto{\pgfqpoint{2.725733in}{0.936228in}}%
\pgfpathlineto{\pgfqpoint{2.727268in}{0.637273in}}%
\pgfpathlineto{\pgfqpoint{2.727329in}{0.637273in}}%
\pgfpathlineto{\pgfqpoint{2.727367in}{0.936228in}}%
\pgfpathlineto{\pgfqpoint{2.728869in}{0.637273in}}%
\pgfpathlineto{\pgfqpoint{2.730082in}{0.637273in}}%
\pgfpathlineto{\pgfqpoint{2.730087in}{0.936228in}}%
\pgfpathlineto{\pgfqpoint{2.731622in}{0.637273in}}%
\pgfpathlineto{\pgfqpoint{2.731650in}{0.637273in}}%
\pgfpathlineto{\pgfqpoint{2.731697in}{0.936228in}}%
\pgfpathlineto{\pgfqpoint{2.733190in}{0.637273in}}%
\pgfpathlineto{\pgfqpoint{2.739891in}{0.637273in}}%
\pgfpathlineto{\pgfqpoint{2.741364in}{0.936228in}}%
\pgfpathlineto{\pgfqpoint{2.741430in}{0.637273in}}%
\pgfpathlineto{\pgfqpoint{2.744108in}{0.637273in}}%
\pgfpathlineto{\pgfqpoint{2.745586in}{0.936228in}}%
\pgfpathlineto{\pgfqpoint{2.745647in}{0.637273in}}%
\pgfpathlineto{\pgfqpoint{2.748339in}{0.637273in}}%
\pgfpathlineto{\pgfqpoint{2.749812in}{0.936228in}}%
\pgfpathlineto{\pgfqpoint{2.749878in}{0.637273in}}%
\pgfpathlineto{\pgfqpoint{2.752546in}{0.637273in}}%
\pgfpathlineto{\pgfqpoint{2.754024in}{0.936228in}}%
\pgfpathlineto{\pgfqpoint{2.754086in}{0.637273in}}%
\pgfpathlineto{\pgfqpoint{2.756768in}{0.637273in}}%
\pgfpathlineto{\pgfqpoint{2.758270in}{0.936228in}}%
\pgfpathlineto{\pgfqpoint{2.758308in}{0.637273in}}%
\pgfpathlineto{\pgfqpoint{2.760990in}{0.637273in}}%
\pgfpathlineto{\pgfqpoint{2.762463in}{0.936228in}}%
\pgfpathlineto{\pgfqpoint{2.762529in}{0.637273in}}%
\pgfpathlineto{\pgfqpoint{2.765179in}{0.637273in}}%
\pgfpathlineto{\pgfqpoint{2.765183in}{0.936228in}}%
\pgfpathlineto{\pgfqpoint{2.766718in}{0.637273in}}%
\pgfpathlineto{\pgfqpoint{2.771998in}{0.637273in}}%
\pgfpathlineto{\pgfqpoint{2.772035in}{0.936228in}}%
\pgfpathlineto{\pgfqpoint{2.773537in}{0.637273in}}%
\pgfpathlineto{\pgfqpoint{2.774755in}{0.637273in}}%
\pgfpathlineto{\pgfqpoint{2.776257in}{0.936228in}}%
\pgfpathlineto{\pgfqpoint{2.776295in}{0.637273in}}%
\pgfpathlineto{\pgfqpoint{2.779123in}{0.637273in}}%
\pgfpathlineto{\pgfqpoint{2.780597in}{0.936228in}}%
\pgfpathlineto{\pgfqpoint{2.780663in}{0.637273in}}%
\pgfpathlineto{\pgfqpoint{2.782457in}{0.637273in}}%
\pgfpathlineto{\pgfqpoint{2.783421in}{0.936228in}}%
\pgfpathlineto{\pgfqpoint{2.783997in}{0.637273in}}%
\pgfpathlineto{\pgfqpoint{2.784856in}{0.637273in}}%
\pgfpathlineto{\pgfqpoint{2.784894in}{0.936228in}}%
\pgfpathlineto{\pgfqpoint{2.786396in}{0.637273in}}%
\pgfpathlineto{\pgfqpoint{2.787609in}{0.637273in}}%
\pgfpathlineto{\pgfqpoint{2.789102in}{0.936228in}}%
\pgfpathlineto{\pgfqpoint{2.789149in}{0.637273in}}%
\pgfpathlineto{\pgfqpoint{2.791874in}{0.637273in}}%
\pgfpathlineto{\pgfqpoint{2.791878in}{0.936228in}}%
\pgfpathlineto{\pgfqpoint{2.793413in}{0.637273in}}%
\pgfpathlineto{\pgfqpoint{2.793451in}{0.637273in}}%
\pgfpathlineto{\pgfqpoint{2.793489in}{0.936228in}}%
\pgfpathlineto{\pgfqpoint{2.794990in}{0.637273in}}%
\pgfpathlineto{\pgfqpoint{2.796204in}{0.637273in}}%
\pgfpathlineto{\pgfqpoint{2.797677in}{0.936228in}}%
\pgfpathlineto{\pgfqpoint{2.797744in}{0.637273in}}%
\pgfpathlineto{\pgfqpoint{2.806012in}{0.637273in}}%
\pgfpathlineto{\pgfqpoint{2.807486in}{0.936228in}}%
\pgfpathlineto{\pgfqpoint{2.807552in}{0.637273in}}%
\pgfpathlineto{\pgfqpoint{2.810366in}{0.637273in}}%
\pgfpathlineto{\pgfqpoint{2.811844in}{0.936228in}}%
\pgfpathlineto{\pgfqpoint{2.811906in}{0.637273in}}%
\pgfpathlineto{\pgfqpoint{2.814560in}{0.637273in}}%
\pgfpathlineto{\pgfqpoint{2.814564in}{0.936228in}}%
\pgfpathlineto{\pgfqpoint{2.816099in}{0.637273in}}%
\pgfpathlineto{\pgfqpoint{2.816132in}{0.637273in}}%
\pgfpathlineto{\pgfqpoint{2.816170in}{0.936228in}}%
\pgfpathlineto{\pgfqpoint{2.817672in}{0.637273in}}%
\pgfpathlineto{\pgfqpoint{2.818885in}{0.637273in}}%
\pgfpathlineto{\pgfqpoint{2.818890in}{0.936228in}}%
\pgfpathlineto{\pgfqpoint{2.820425in}{0.637273in}}%
\pgfpathlineto{\pgfqpoint{2.821544in}{0.637273in}}%
\pgfpathlineto{\pgfqpoint{2.821582in}{0.936228in}}%
\pgfpathlineto{\pgfqpoint{2.823084in}{0.637273in}}%
\pgfpathlineto{\pgfqpoint{2.824495in}{0.637273in}}%
\pgfpathlineto{\pgfqpoint{2.824500in}{0.936228in}}%
\pgfpathlineto{\pgfqpoint{2.826035in}{0.637273in}}%
\pgfpathlineto{\pgfqpoint{2.826318in}{0.637273in}}%
\pgfpathlineto{\pgfqpoint{2.826356in}{0.936228in}}%
\pgfpathlineto{\pgfqpoint{2.827858in}{0.637273in}}%
\pgfpathlineto{\pgfqpoint{2.829076in}{0.637273in}}%
\pgfpathlineto{\pgfqpoint{2.830549in}{0.936228in}}%
\pgfpathlineto{\pgfqpoint{2.830616in}{0.637273in}}%
\pgfpathlineto{\pgfqpoint{2.838648in}{0.637273in}}%
\pgfpathlineto{\pgfqpoint{2.840126in}{0.936228in}}%
\pgfpathlineto{\pgfqpoint{2.840188in}{0.637273in}}%
\pgfpathlineto{\pgfqpoint{2.842870in}{0.637273in}}%
\pgfpathlineto{\pgfqpoint{2.844343in}{0.936228in}}%
\pgfpathlineto{\pgfqpoint{2.844409in}{0.637273in}}%
\pgfpathlineto{\pgfqpoint{2.847106in}{0.637273in}}%
\pgfpathlineto{\pgfqpoint{2.847111in}{0.936228in}}%
\pgfpathlineto{\pgfqpoint{2.848645in}{0.637273in}}%
\pgfpathlineto{\pgfqpoint{2.848683in}{0.637273in}}%
\pgfpathlineto{\pgfqpoint{2.848721in}{0.936228in}}%
\pgfpathlineto{\pgfqpoint{2.850223in}{0.637273in}}%
\pgfpathlineto{\pgfqpoint{2.851441in}{0.637273in}}%
\pgfpathlineto{\pgfqpoint{2.852943in}{0.936228in}}%
\pgfpathlineto{\pgfqpoint{2.852980in}{0.637273in}}%
\pgfpathlineto{\pgfqpoint{2.855823in}{0.637273in}}%
\pgfpathlineto{\pgfqpoint{2.855828in}{0.936228in}}%
\pgfpathlineto{\pgfqpoint{2.857363in}{0.637273in}}%
\pgfpathlineto{\pgfqpoint{2.857401in}{0.637273in}}%
\pgfpathlineto{\pgfqpoint{2.857438in}{0.936228in}}%
\pgfpathlineto{\pgfqpoint{2.858940in}{0.637273in}}%
\pgfpathlineto{\pgfqpoint{2.860154in}{0.637273in}}%
\pgfpathlineto{\pgfqpoint{2.860158in}{0.936228in}}%
\pgfpathlineto{\pgfqpoint{2.861693in}{0.637273in}}%
\pgfpathlineto{\pgfqpoint{2.861726in}{0.637273in}}%
\pgfpathlineto{\pgfqpoint{2.861764in}{0.936228in}}%
\pgfpathlineto{\pgfqpoint{2.863266in}{0.637273in}}%
\pgfpathlineto{\pgfqpoint{2.869910in}{0.637273in}}%
\pgfpathlineto{\pgfqpoint{2.869915in}{0.936228in}}%
\pgfpathlineto{\pgfqpoint{2.871449in}{0.637273in}}%
\pgfpathlineto{\pgfqpoint{2.871487in}{0.637273in}}%
\pgfpathlineto{\pgfqpoint{2.871525in}{0.936228in}}%
\pgfpathlineto{\pgfqpoint{2.873027in}{0.637273in}}%
\pgfpathlineto{\pgfqpoint{2.874269in}{0.637273in}}%
\pgfpathlineto{\pgfqpoint{2.875742in}{0.936228in}}%
\pgfpathlineto{\pgfqpoint{2.875808in}{0.637273in}}%
\pgfpathlineto{\pgfqpoint{2.878273in}{0.637273in}}%
\pgfpathlineto{\pgfqpoint{2.878538in}{0.936228in}}%
\pgfpathlineto{\pgfqpoint{2.879813in}{0.637273in}}%
\pgfpathlineto{\pgfqpoint{2.880110in}{0.637273in}}%
\pgfpathlineto{\pgfqpoint{2.880148in}{0.936228in}}%
\pgfpathlineto{\pgfqpoint{2.881650in}{0.637273in}}%
\pgfpathlineto{\pgfqpoint{2.882863in}{0.637273in}}%
\pgfpathlineto{\pgfqpoint{2.882868in}{0.936228in}}%
\pgfpathlineto{\pgfqpoint{2.884403in}{0.637273in}}%
\pgfpathlineto{\pgfqpoint{2.884469in}{0.637273in}}%
\pgfpathlineto{\pgfqpoint{2.884507in}{0.936228in}}%
\pgfpathlineto{\pgfqpoint{2.886008in}{0.637273in}}%
\pgfpathlineto{\pgfqpoint{2.887387in}{0.637273in}}%
\pgfpathlineto{\pgfqpoint{2.887392in}{0.936228in}}%
\pgfpathlineto{\pgfqpoint{2.888927in}{0.637273in}}%
\pgfpathlineto{\pgfqpoint{2.888960in}{0.637273in}}%
\pgfpathlineto{\pgfqpoint{2.888998in}{0.936228in}}%
\pgfpathlineto{\pgfqpoint{2.890499in}{0.637273in}}%
\pgfpathlineto{\pgfqpoint{2.891855in}{0.637273in}}%
\pgfpathlineto{\pgfqpoint{2.891859in}{0.936228in}}%
\pgfpathlineto{\pgfqpoint{2.893394in}{0.637273in}}%
\pgfpathlineto{\pgfqpoint{2.898763in}{0.637273in}}%
\pgfpathlineto{\pgfqpoint{2.898801in}{0.936228in}}%
\pgfpathlineto{\pgfqpoint{2.900303in}{0.637273in}}%
\pgfpathlineto{\pgfqpoint{2.901658in}{0.637273in}}%
\pgfpathlineto{\pgfqpoint{2.903131in}{0.936228in}}%
\pgfpathlineto{\pgfqpoint{2.903198in}{0.637273in}}%
\pgfpathlineto{\pgfqpoint{2.906050in}{0.637273in}}%
\pgfpathlineto{\pgfqpoint{2.906055in}{0.936228in}}%
\pgfpathlineto{\pgfqpoint{2.907589in}{0.637273in}}%
\pgfpathlineto{\pgfqpoint{2.907674in}{0.637273in}}%
\pgfpathlineto{\pgfqpoint{2.907712in}{0.936228in}}%
\pgfpathlineto{\pgfqpoint{2.909214in}{0.637273in}}%
\pgfpathlineto{\pgfqpoint{2.910569in}{0.637273in}}%
\pgfpathlineto{\pgfqpoint{2.910574in}{0.936228in}}%
\pgfpathlineto{\pgfqpoint{2.912109in}{0.637273in}}%
\pgfpathlineto{\pgfqpoint{2.912170in}{0.637273in}}%
\pgfpathlineto{\pgfqpoint{2.912208in}{0.936228in}}%
\pgfpathlineto{\pgfqpoint{2.913709in}{0.637273in}}%
\pgfpathlineto{\pgfqpoint{2.915093in}{0.637273in}}%
\pgfpathlineto{\pgfqpoint{2.915098in}{0.936228in}}%
\pgfpathlineto{\pgfqpoint{2.916633in}{0.637273in}}%
\pgfpathlineto{\pgfqpoint{2.916666in}{0.637273in}}%
\pgfpathlineto{\pgfqpoint{2.916703in}{0.936228in}}%
\pgfpathlineto{\pgfqpoint{2.918205in}{0.637273in}}%
\pgfpathlineto{\pgfqpoint{2.919560in}{0.637273in}}%
\pgfpathlineto{\pgfqpoint{2.919565in}{0.936228in}}%
\pgfpathlineto{\pgfqpoint{2.921100in}{0.637273in}}%
\pgfpathlineto{\pgfqpoint{2.926469in}{0.637273in}}%
\pgfpathlineto{\pgfqpoint{2.926507in}{0.936228in}}%
\pgfpathlineto{\pgfqpoint{2.928009in}{0.637273in}}%
\pgfpathlineto{\pgfqpoint{2.929364in}{0.637273in}}%
\pgfpathlineto{\pgfqpoint{2.929369in}{0.936228in}}%
\pgfpathlineto{\pgfqpoint{2.930903in}{0.637273in}}%
\pgfpathlineto{\pgfqpoint{2.930951in}{0.637273in}}%
\pgfpathlineto{\pgfqpoint{2.930988in}{0.936228in}}%
\pgfpathlineto{\pgfqpoint{2.932490in}{0.637273in}}%
\pgfpathlineto{\pgfqpoint{2.933685in}{0.637273in}}%
\pgfpathlineto{\pgfqpoint{2.933954in}{0.936228in}}%
\pgfpathlineto{\pgfqpoint{2.935224in}{0.637273in}}%
\pgfpathlineto{\pgfqpoint{2.935517in}{0.637273in}}%
\pgfpathlineto{\pgfqpoint{2.935560in}{0.936228in}}%
\pgfpathlineto{\pgfqpoint{2.937057in}{0.637273in}}%
\pgfpathlineto{\pgfqpoint{2.938417in}{0.637273in}}%
\pgfpathlineto{\pgfqpoint{2.938421in}{0.936228in}}%
\pgfpathlineto{\pgfqpoint{2.939956in}{0.637273in}}%
\pgfpathlineto{\pgfqpoint{2.940017in}{0.637273in}}%
\pgfpathlineto{\pgfqpoint{2.940055in}{0.936228in}}%
\pgfpathlineto{\pgfqpoint{2.941557in}{0.637273in}}%
\pgfpathlineto{\pgfqpoint{2.942941in}{0.637273in}}%
\pgfpathlineto{\pgfqpoint{2.942945in}{0.936228in}}%
\pgfpathlineto{\pgfqpoint{2.944480in}{0.637273in}}%
\pgfpathlineto{\pgfqpoint{2.944513in}{0.637273in}}%
\pgfpathlineto{\pgfqpoint{2.944551in}{0.936228in}}%
\pgfpathlineto{\pgfqpoint{2.946053in}{0.637273in}}%
\pgfpathlineto{\pgfqpoint{2.947408in}{0.637273in}}%
\pgfpathlineto{\pgfqpoint{2.947413in}{0.936228in}}%
\pgfpathlineto{\pgfqpoint{2.948947in}{0.637273in}}%
\pgfpathlineto{\pgfqpoint{2.954317in}{0.637273in}}%
\pgfpathlineto{\pgfqpoint{2.954354in}{0.936228in}}%
\pgfpathlineto{\pgfqpoint{2.955856in}{0.637273in}}%
\pgfpathlineto{\pgfqpoint{2.957070in}{0.637273in}}%
\pgfpathlineto{\pgfqpoint{2.957074in}{0.936228in}}%
\pgfpathlineto{\pgfqpoint{2.958609in}{0.637273in}}%
\pgfpathlineto{\pgfqpoint{2.958647in}{0.637273in}}%
\pgfpathlineto{\pgfqpoint{2.958685in}{0.936228in}}%
\pgfpathlineto{\pgfqpoint{2.960186in}{0.637273in}}%
\pgfpathlineto{\pgfqpoint{2.961613in}{0.637273in}}%
\pgfpathlineto{\pgfqpoint{2.961617in}{0.936228in}}%
\pgfpathlineto{\pgfqpoint{2.963152in}{0.637273in}}%
\pgfpathlineto{\pgfqpoint{2.963190in}{0.637273in}}%
\pgfpathlineto{\pgfqpoint{2.963228in}{0.936228in}}%
\pgfpathlineto{\pgfqpoint{2.964729in}{0.637273in}}%
\pgfpathlineto{\pgfqpoint{2.966080in}{0.637273in}}%
\pgfpathlineto{\pgfqpoint{2.966085in}{0.936228in}}%
\pgfpathlineto{\pgfqpoint{2.967619in}{0.637273in}}%
\pgfpathlineto{\pgfqpoint{2.967685in}{0.637273in}}%
\pgfpathlineto{\pgfqpoint{2.967723in}{0.936228in}}%
\pgfpathlineto{\pgfqpoint{2.969225in}{0.637273in}}%
\pgfpathlineto{\pgfqpoint{2.970689in}{0.637273in}}%
\pgfpathlineto{\pgfqpoint{2.970694in}{0.936228in}}%
\pgfpathlineto{\pgfqpoint{2.972228in}{0.637273in}}%
\pgfpathlineto{\pgfqpoint{2.972261in}{0.637273in}}%
\pgfpathlineto{\pgfqpoint{2.972299in}{0.936228in}}%
\pgfpathlineto{\pgfqpoint{2.973801in}{0.637273in}}%
\pgfpathlineto{\pgfqpoint{2.975184in}{0.637273in}}%
\pgfpathlineto{\pgfqpoint{2.975189in}{0.936228in}}%
\pgfpathlineto{\pgfqpoint{2.976724in}{0.637273in}}%
\pgfpathlineto{\pgfqpoint{2.976757in}{0.637273in}}%
\pgfpathlineto{\pgfqpoint{2.976795in}{0.936228in}}%
\pgfpathlineto{\pgfqpoint{2.978296in}{0.637273in}}%
\pgfpathlineto{\pgfqpoint{2.985082in}{0.637273in}}%
\pgfpathlineto{\pgfqpoint{2.986556in}{0.936228in}}%
\pgfpathlineto{\pgfqpoint{2.986622in}{0.637273in}}%
\pgfpathlineto{\pgfqpoint{2.989413in}{0.637273in}}%
\pgfpathlineto{\pgfqpoint{2.990957in}{0.936228in}}%
\pgfpathlineto{\pgfqpoint{2.990962in}{0.936228in}}%
\pgfpathlineto{\pgfqpoint{2.992506in}{0.637273in}}%
\pgfpathlineto{\pgfqpoint{2.993705in}{0.637273in}}%
\pgfpathlineto{\pgfqpoint{2.993710in}{0.936228in}}%
\pgfpathlineto{\pgfqpoint{2.995245in}{0.637273in}}%
\pgfpathlineto{\pgfqpoint{2.995283in}{0.637273in}}%
\pgfpathlineto{\pgfqpoint{2.995320in}{0.936228in}}%
\pgfpathlineto{\pgfqpoint{2.996822in}{0.637273in}}%
\pgfpathlineto{\pgfqpoint{2.998064in}{0.637273in}}%
\pgfpathlineto{\pgfqpoint{2.998069in}{0.936228in}}%
\pgfpathlineto{\pgfqpoint{2.999604in}{0.637273in}}%
\pgfpathlineto{\pgfqpoint{2.999637in}{0.637273in}}%
\pgfpathlineto{\pgfqpoint{2.999679in}{0.936228in}}%
\pgfpathlineto{\pgfqpoint{3.001176in}{0.637273in}}%
\pgfpathlineto{\pgfqpoint{3.002531in}{0.637273in}}%
\pgfpathlineto{\pgfqpoint{3.004076in}{0.936228in}}%
\pgfpathlineto{\pgfqpoint{3.004094in}{0.936228in}}%
\pgfpathlineto{\pgfqpoint{3.005639in}{0.637273in}}%
\pgfpathlineto{\pgfqpoint{3.006975in}{0.637273in}}%
\pgfpathlineto{\pgfqpoint{3.006980in}{0.936228in}}%
\pgfpathlineto{\pgfqpoint{3.008515in}{0.637273in}}%
\pgfpathlineto{\pgfqpoint{3.010748in}{0.637273in}}%
\pgfpathlineto{\pgfqpoint{3.010786in}{0.936228in}}%
\pgfpathlineto{\pgfqpoint{3.012288in}{0.637273in}}%
\pgfpathlineto{\pgfqpoint{3.016826in}{0.637273in}}%
\pgfpathlineto{\pgfqpoint{3.018299in}{0.936228in}}%
\pgfpathlineto{\pgfqpoint{3.018365in}{0.637273in}}%
\pgfpathlineto{\pgfqpoint{3.021062in}{0.637273in}}%
\pgfpathlineto{\pgfqpoint{3.021066in}{0.936228in}}%
\pgfpathlineto{\pgfqpoint{3.022601in}{0.637273in}}%
\pgfpathlineto{\pgfqpoint{3.026799in}{0.637273in}}%
\pgfpathlineto{\pgfqpoint{3.026847in}{0.914287in}}%
\pgfpathlineto{\pgfqpoint{3.028339in}{0.637273in}}%
\pgfpathlineto{\pgfqpoint{3.032301in}{0.637273in}}%
\pgfpathlineto{\pgfqpoint{3.032329in}{0.911544in}}%
\pgfpathlineto{\pgfqpoint{3.033840in}{0.637273in}}%
\pgfpathlineto{\pgfqpoint{3.039465in}{0.637273in}}%
\pgfpathlineto{\pgfqpoint{3.039512in}{0.914287in}}%
\pgfpathlineto{\pgfqpoint{3.041004in}{0.637273in}}%
\pgfpathlineto{\pgfqpoint{3.045202in}{0.637273in}}%
\pgfpathlineto{\pgfqpoint{3.045231in}{0.911544in}}%
\pgfpathlineto{\pgfqpoint{3.046742in}{0.637273in}}%
\pgfpathlineto{\pgfqpoint{3.050553in}{0.637273in}}%
\pgfpathlineto{\pgfqpoint{3.050628in}{0.914287in}}%
\pgfpathlineto{\pgfqpoint{3.052092in}{0.637273in}}%
\pgfpathlineto{\pgfqpoint{3.055983in}{0.637273in}}%
\pgfpathlineto{\pgfqpoint{3.055993in}{0.911544in}}%
\pgfpathlineto{\pgfqpoint{3.057523in}{0.637273in}}%
\pgfpathlineto{\pgfqpoint{3.061452in}{0.637273in}}%
\pgfpathlineto{\pgfqpoint{3.061527in}{0.914287in}}%
\pgfpathlineto{\pgfqpoint{3.062991in}{0.637273in}}%
\pgfpathlineto{\pgfqpoint{3.067194in}{0.637273in}}%
\pgfpathlineto{\pgfqpoint{3.067222in}{0.911544in}}%
\pgfpathlineto{\pgfqpoint{3.068733in}{0.637273in}}%
\pgfpathlineto{\pgfqpoint{3.072559in}{0.637273in}}%
\pgfpathlineto{\pgfqpoint{3.072568in}{0.914287in}}%
\pgfpathlineto{\pgfqpoint{3.074098in}{0.637273in}}%
\pgfpathlineto{\pgfqpoint{3.078074in}{0.637273in}}%
\pgfpathlineto{\pgfqpoint{3.078103in}{0.911544in}}%
\pgfpathlineto{\pgfqpoint{3.079614in}{0.637273in}}%
\pgfpathlineto{\pgfqpoint{3.083477in}{0.637273in}}%
\pgfpathlineto{\pgfqpoint{3.083486in}{0.914287in}}%
\pgfpathlineto{\pgfqpoint{3.085016in}{0.637273in}}%
\pgfpathlineto{\pgfqpoint{3.088992in}{0.637273in}}%
\pgfpathlineto{\pgfqpoint{3.089087in}{0.914287in}}%
\pgfpathlineto{\pgfqpoint{3.090532in}{0.637273in}}%
\pgfpathlineto{\pgfqpoint{3.094451in}{0.637273in}}%
\pgfpathlineto{\pgfqpoint{3.094480in}{0.911544in}}%
\pgfpathlineto{\pgfqpoint{3.095991in}{0.637273in}}%
\pgfpathlineto{\pgfqpoint{3.100080in}{0.637273in}}%
\pgfpathlineto{\pgfqpoint{3.100175in}{0.914287in}}%
\pgfpathlineto{\pgfqpoint{3.101620in}{0.637273in}}%
\pgfpathlineto{\pgfqpoint{3.105492in}{0.637273in}}%
\pgfpathlineto{\pgfqpoint{3.105520in}{0.911544in}}%
\pgfpathlineto{\pgfqpoint{3.107031in}{0.637273in}}%
\pgfpathlineto{\pgfqpoint{3.110875in}{0.637273in}}%
\pgfpathlineto{\pgfqpoint{3.110998in}{0.914287in}}%
\pgfpathlineto{\pgfqpoint{3.112415in}{0.637273in}}%
\pgfpathlineto{\pgfqpoint{3.116344in}{0.637273in}}%
\pgfpathlineto{\pgfqpoint{3.116372in}{0.911544in}}%
\pgfpathlineto{\pgfqpoint{3.117883in}{0.637273in}}%
\pgfpathlineto{\pgfqpoint{3.121968in}{0.637273in}}%
\pgfpathlineto{\pgfqpoint{3.122034in}{0.914287in}}%
\pgfpathlineto{\pgfqpoint{3.123508in}{0.637273in}}%
\pgfpathlineto{\pgfqpoint{3.127352in}{0.637273in}}%
\pgfpathlineto{\pgfqpoint{3.127380in}{0.911544in}}%
\pgfpathlineto{\pgfqpoint{3.128891in}{0.637273in}}%
\pgfpathlineto{\pgfqpoint{3.132726in}{0.637273in}}%
\pgfpathlineto{\pgfqpoint{3.132858in}{0.914287in}}%
\pgfpathlineto{\pgfqpoint{3.134265in}{0.637273in}}%
\pgfpathlineto{\pgfqpoint{3.138175in}{0.637273in}}%
\pgfpathlineto{\pgfqpoint{3.138260in}{0.911544in}}%
\pgfpathlineto{\pgfqpoint{3.139715in}{0.637273in}}%
\pgfpathlineto{\pgfqpoint{3.143577in}{0.637273in}}%
\pgfpathlineto{\pgfqpoint{3.144668in}{0.914287in}}%
\pgfpathlineto{\pgfqpoint{3.145117in}{0.637273in}}%
\pgfpathlineto{\pgfqpoint{3.150311in}{0.637273in}}%
\pgfpathlineto{\pgfqpoint{3.150340in}{0.911544in}}%
\pgfpathlineto{\pgfqpoint{3.151851in}{0.637273in}}%
\pgfpathlineto{\pgfqpoint{3.155704in}{0.637273in}}%
\pgfpathlineto{\pgfqpoint{3.155799in}{0.914287in}}%
\pgfpathlineto{\pgfqpoint{3.157244in}{0.637273in}}%
\pgfpathlineto{\pgfqpoint{3.162042in}{0.637273in}}%
\pgfpathlineto{\pgfqpoint{3.163548in}{0.922515in}}%
\pgfpathlineto{\pgfqpoint{3.163581in}{0.637273in}}%
\pgfpathlineto{\pgfqpoint{3.164389in}{0.637273in}}%
\pgfpathlineto{\pgfqpoint{3.165664in}{0.922515in}}%
\pgfpathlineto{\pgfqpoint{3.165928in}{0.637273in}}%
\pgfpathlineto{\pgfqpoint{3.166334in}{0.637273in}}%
\pgfpathlineto{\pgfqpoint{3.167501in}{0.922515in}}%
\pgfpathlineto{\pgfqpoint{3.167874in}{0.637273in}}%
\pgfpathlineto{\pgfqpoint{3.168252in}{0.637273in}}%
\pgfpathlineto{\pgfqpoint{3.169456in}{0.922515in}}%
\pgfpathlineto{\pgfqpoint{3.169791in}{0.637273in}}%
\pgfpathlineto{\pgfqpoint{3.170065in}{0.637273in}}%
\pgfpathlineto{\pgfqpoint{3.171576in}{0.922515in}}%
\pgfpathlineto{\pgfqpoint{3.171604in}{0.637273in}}%
\pgfpathlineto{\pgfqpoint{3.172143in}{0.637273in}}%
\pgfpathlineto{\pgfqpoint{3.173238in}{0.922515in}}%
\pgfpathlineto{\pgfqpoint{3.173682in}{0.637273in}}%
\pgfpathlineto{\pgfqpoint{3.173706in}{0.637273in}}%
\pgfpathlineto{\pgfqpoint{3.174565in}{0.922515in}}%
\pgfpathlineto{\pgfqpoint{3.175245in}{0.637273in}}%
\pgfpathlineto{\pgfqpoint{3.175297in}{0.637273in}}%
\pgfpathlineto{\pgfqpoint{3.176549in}{0.922515in}}%
\pgfpathlineto{\pgfqpoint{3.176837in}{0.637273in}}%
\pgfpathlineto{\pgfqpoint{3.177210in}{0.637273in}}%
\pgfpathlineto{\pgfqpoint{3.178343in}{0.922515in}}%
\pgfpathlineto{\pgfqpoint{3.178749in}{0.637273in}}%
\pgfpathlineto{\pgfqpoint{3.178995in}{0.637273in}}%
\pgfpathlineto{\pgfqpoint{3.180378in}{0.922515in}}%
\pgfpathlineto{\pgfqpoint{3.180534in}{0.637273in}}%
\pgfpathlineto{\pgfqpoint{3.181181in}{0.637273in}}%
\pgfpathlineto{\pgfqpoint{3.182390in}{0.922515in}}%
\pgfpathlineto{\pgfqpoint{3.182721in}{0.637273in}}%
\pgfpathlineto{\pgfqpoint{3.183103in}{0.637273in}}%
\pgfpathlineto{\pgfqpoint{3.184558in}{0.922515in}}%
\pgfpathlineto{\pgfqpoint{3.184643in}{0.637273in}}%
\pgfpathlineto{\pgfqpoint{3.185275in}{0.637273in}}%
\pgfpathlineto{\pgfqpoint{3.186536in}{0.922515in}}%
\pgfpathlineto{\pgfqpoint{3.186815in}{0.637273in}}%
\pgfpathlineto{\pgfqpoint{3.187089in}{0.637273in}}%
\pgfpathlineto{\pgfqpoint{3.187736in}{0.922515in}}%
\pgfpathlineto{\pgfqpoint{3.188628in}{0.637273in}}%
\pgfpathlineto{\pgfqpoint{3.188732in}{0.637273in}}%
\pgfpathlineto{\pgfqpoint{3.189965in}{0.922515in}}%
\pgfpathlineto{\pgfqpoint{3.190272in}{0.637273in}}%
\pgfpathlineto{\pgfqpoint{3.190730in}{0.637273in}}%
\pgfpathlineto{\pgfqpoint{3.191991in}{0.922515in}}%
\pgfpathlineto{\pgfqpoint{3.192269in}{0.637273in}}%
\pgfpathlineto{\pgfqpoint{3.192614in}{0.637273in}}%
\pgfpathlineto{\pgfqpoint{3.193913in}{0.922515in}}%
\pgfpathlineto{\pgfqpoint{3.194153in}{0.637273in}}%
\pgfpathlineto{\pgfqpoint{3.194659in}{0.637273in}}%
\pgfpathlineto{\pgfqpoint{3.195934in}{0.922515in}}%
\pgfpathlineto{\pgfqpoint{3.196198in}{0.637273in}}%
\pgfpathlineto{\pgfqpoint{3.196500in}{0.637273in}}%
\pgfpathlineto{\pgfqpoint{3.198016in}{0.922515in}}%
\pgfpathlineto{\pgfqpoint{3.198040in}{0.637273in}}%
\pgfpathlineto{\pgfqpoint{3.198045in}{0.637273in}}%
\pgfpathlineto{\pgfqpoint{3.199447in}{0.922515in}}%
\pgfpathlineto{\pgfqpoint{3.199584in}{0.637273in}}%
\pgfpathlineto{\pgfqpoint{3.199655in}{0.637273in}}%
\pgfpathlineto{\pgfqpoint{3.201199in}{0.922515in}}%
\pgfpathlineto{\pgfqpoint{3.201209in}{0.922515in}}%
\pgfpathlineto{\pgfqpoint{3.202753in}{0.637273in}}%
\pgfpathlineto{\pgfqpoint{3.202767in}{0.637273in}}%
\pgfpathlineto{\pgfqpoint{3.204302in}{0.922515in}}%
\pgfpathlineto{\pgfqpoint{3.204306in}{0.637273in}}%
\pgfpathlineto{\pgfqpoint{3.204330in}{0.637273in}}%
\pgfpathlineto{\pgfqpoint{3.205874in}{0.922515in}}%
\pgfpathlineto{\pgfqpoint{3.205884in}{0.922515in}}%
\pgfpathlineto{\pgfqpoint{3.207414in}{0.637273in}}%
\pgfpathlineto{\pgfqpoint{3.207423in}{0.922515in}}%
\pgfpathlineto{\pgfqpoint{3.207433in}{0.922515in}}%
\pgfpathlineto{\pgfqpoint{3.208977in}{0.637273in}}%
\pgfpathlineto{\pgfqpoint{3.210502in}{0.922515in}}%
\pgfpathlineto{\pgfqpoint{3.210516in}{0.637273in}}%
\pgfpathlineto{\pgfqpoint{3.210710in}{0.637273in}}%
\pgfpathlineto{\pgfqpoint{3.212235in}{0.922515in}}%
\pgfpathlineto{\pgfqpoint{3.212249in}{0.637273in}}%
\pgfpathlineto{\pgfqpoint{3.212263in}{0.637273in}}%
\pgfpathlineto{\pgfqpoint{3.213793in}{0.922515in}}%
\pgfpathlineto{\pgfqpoint{3.213803in}{0.637273in}}%
\pgfpathlineto{\pgfqpoint{3.213822in}{0.637273in}}%
\pgfpathlineto{\pgfqpoint{3.215248in}{0.922515in}}%
\pgfpathlineto{\pgfqpoint{3.215361in}{0.637273in}}%
\pgfpathlineto{\pgfqpoint{3.215385in}{0.637273in}}%
\pgfpathlineto{\pgfqpoint{3.216929in}{0.922515in}}%
\pgfpathlineto{\pgfqpoint{3.216948in}{0.922515in}}%
\pgfpathlineto{\pgfqpoint{3.218492in}{0.637273in}}%
\pgfpathlineto{\pgfqpoint{3.219966in}{0.922515in}}%
\pgfpathlineto{\pgfqpoint{3.220032in}{0.637273in}}%
\pgfpathlineto{\pgfqpoint{3.220164in}{0.637273in}}%
\pgfpathlineto{\pgfqpoint{3.221699in}{0.922515in}}%
\pgfpathlineto{\pgfqpoint{3.221703in}{0.637273in}}%
\pgfpathlineto{\pgfqpoint{3.221732in}{0.637273in}}%
\pgfpathlineto{\pgfqpoint{3.223276in}{0.922515in}}%
\pgfpathlineto{\pgfqpoint{3.224787in}{0.637273in}}%
\pgfpathlineto{\pgfqpoint{3.224815in}{0.922515in}}%
\pgfpathlineto{\pgfqpoint{3.224820in}{0.922515in}}%
\pgfpathlineto{\pgfqpoint{3.226364in}{0.637273in}}%
\pgfpathlineto{\pgfqpoint{3.226374in}{0.637273in}}%
\pgfpathlineto{\pgfqpoint{3.227918in}{0.922515in}}%
\pgfpathlineto{\pgfqpoint{3.229443in}{0.637273in}}%
\pgfpathlineto{\pgfqpoint{3.229457in}{0.922515in}}%
\pgfpathlineto{\pgfqpoint{3.229481in}{0.922515in}}%
\pgfpathlineto{\pgfqpoint{3.230983in}{0.637273in}}%
\pgfpathlineto{\pgfqpoint{3.231020in}{0.922515in}}%
\pgfpathlineto{\pgfqpoint{3.231025in}{0.922515in}}%
\pgfpathlineto{\pgfqpoint{3.232569in}{0.637273in}}%
\pgfpathlineto{\pgfqpoint{3.232584in}{0.637273in}}%
\pgfpathlineto{\pgfqpoint{3.234109in}{0.922515in}}%
\pgfpathlineto{\pgfqpoint{3.234123in}{0.637273in}}%
\pgfpathlineto{\pgfqpoint{3.234142in}{0.637273in}}%
\pgfpathlineto{\pgfqpoint{3.235686in}{0.922515in}}%
\pgfpathlineto{\pgfqpoint{3.235696in}{0.922515in}}%
\pgfpathlineto{\pgfqpoint{3.237207in}{0.637273in}}%
\pgfpathlineto{\pgfqpoint{3.237235in}{0.922515in}}%
\pgfpathlineto{\pgfqpoint{3.237254in}{0.922515in}}%
\pgfpathlineto{\pgfqpoint{3.238784in}{0.637273in}}%
\pgfpathlineto{\pgfqpoint{3.238793in}{0.922515in}}%
\pgfpathlineto{\pgfqpoint{3.238831in}{0.922515in}}%
\pgfpathlineto{\pgfqpoint{3.240347in}{0.637273in}}%
\pgfpathlineto{\pgfqpoint{3.240371in}{0.922515in}}%
\pgfpathlineto{\pgfqpoint{3.240375in}{0.922515in}}%
\pgfpathlineto{\pgfqpoint{3.241896in}{0.637273in}}%
\pgfpathlineto{\pgfqpoint{3.241915in}{0.922515in}}%
\pgfpathlineto{\pgfqpoint{3.241943in}{0.922515in}}%
\pgfpathlineto{\pgfqpoint{3.243473in}{0.637273in}}%
\pgfpathlineto{\pgfqpoint{3.243483in}{0.922515in}}%
\pgfpathlineto{\pgfqpoint{3.243520in}{0.922515in}}%
\pgfpathlineto{\pgfqpoint{3.245060in}{0.637273in}}%
\pgfpathlineto{\pgfqpoint{3.246590in}{0.922515in}}%
\pgfpathlineto{\pgfqpoint{3.246599in}{0.637273in}}%
\pgfpathlineto{\pgfqpoint{3.246618in}{0.637273in}}%
\pgfpathlineto{\pgfqpoint{3.248162in}{0.922515in}}%
\pgfpathlineto{\pgfqpoint{3.248167in}{0.922515in}}%
\pgfpathlineto{\pgfqpoint{3.249707in}{0.637273in}}%
\pgfpathlineto{\pgfqpoint{3.251218in}{0.922515in}}%
\pgfpathlineto{\pgfqpoint{3.251246in}{0.637273in}}%
\pgfpathlineto{\pgfqpoint{3.251251in}{0.637273in}}%
\pgfpathlineto{\pgfqpoint{3.252620in}{0.922515in}}%
\pgfpathlineto{\pgfqpoint{3.252790in}{0.637273in}}%
\pgfpathlineto{\pgfqpoint{3.252823in}{0.637273in}}%
\pgfpathlineto{\pgfqpoint{3.254372in}{3.379985in}}%
\pgfpathlineto{\pgfqpoint{3.255917in}{0.637273in}}%
\pgfpathlineto{\pgfqpoint{3.257352in}{0.919772in}}%
\pgfpathlineto{\pgfqpoint{3.257456in}{0.637273in}}%
\pgfpathlineto{\pgfqpoint{3.263633in}{0.637273in}}%
\pgfpathlineto{\pgfqpoint{3.265177in}{3.379985in}}%
\pgfpathlineto{\pgfqpoint{3.265271in}{3.379985in}}%
\pgfpathlineto{\pgfqpoint{3.266816in}{0.637273in}}%
\pgfpathlineto{\pgfqpoint{3.271009in}{0.637273in}}%
\pgfpathlineto{\pgfqpoint{3.271037in}{0.911544in}}%
\pgfpathlineto{\pgfqpoint{3.272549in}{0.637273in}}%
\pgfpathlineto{\pgfqpoint{3.276756in}{0.637273in}}%
\pgfpathlineto{\pgfqpoint{3.276888in}{0.914287in}}%
\pgfpathlineto{\pgfqpoint{3.278296in}{0.637273in}}%
\pgfpathlineto{\pgfqpoint{3.282555in}{0.637273in}}%
\pgfpathlineto{\pgfqpoint{3.283141in}{0.914287in}}%
\pgfpathlineto{\pgfqpoint{3.284095in}{0.637273in}}%
\pgfpathlineto{\pgfqpoint{3.289596in}{0.637273in}}%
\pgfpathlineto{\pgfqpoint{3.289624in}{0.911544in}}%
\pgfpathlineto{\pgfqpoint{3.291136in}{0.637273in}}%
\pgfpathlineto{\pgfqpoint{3.295593in}{0.637273in}}%
\pgfpathlineto{\pgfqpoint{3.295598in}{0.919772in}}%
\pgfpathlineto{\pgfqpoint{3.297133in}{0.637273in}}%
\pgfpathlineto{\pgfqpoint{3.301714in}{0.637273in}}%
\pgfpathlineto{\pgfqpoint{3.301742in}{0.911544in}}%
\pgfpathlineto{\pgfqpoint{3.303253in}{0.637273in}}%
\pgfpathlineto{\pgfqpoint{3.307173in}{0.637273in}}%
\pgfpathlineto{\pgfqpoint{3.307201in}{0.911544in}}%
\pgfpathlineto{\pgfqpoint{3.308712in}{0.637273in}}%
\pgfpathlineto{\pgfqpoint{3.309052in}{0.637273in}}%
\pgfpathlineto{\pgfqpoint{3.309080in}{0.914287in}}%
\pgfpathlineto{\pgfqpoint{3.310591in}{0.637273in}}%
\pgfpathlineto{\pgfqpoint{3.314417in}{0.637273in}}%
\pgfpathlineto{\pgfqpoint{3.314445in}{0.911544in}}%
\pgfpathlineto{\pgfqpoint{3.315956in}{0.637273in}}%
\pgfpathlineto{\pgfqpoint{3.320154in}{0.637273in}}%
\pgfpathlineto{\pgfqpoint{3.320253in}{0.914287in}}%
\pgfpathlineto{\pgfqpoint{3.321694in}{0.637273in}}%
\pgfpathlineto{\pgfqpoint{3.325599in}{0.637273in}}%
\pgfpathlineto{\pgfqpoint{3.325627in}{0.911544in}}%
\pgfpathlineto{\pgfqpoint{3.327138in}{0.637273in}}%
\pgfpathlineto{\pgfqpoint{3.331030in}{0.637273in}}%
\pgfpathlineto{\pgfqpoint{3.331134in}{0.914287in}}%
\pgfpathlineto{\pgfqpoint{3.332569in}{0.637273in}}%
\pgfpathlineto{\pgfqpoint{3.336574in}{0.637273in}}%
\pgfpathlineto{\pgfqpoint{3.336602in}{0.911544in}}%
\pgfpathlineto{\pgfqpoint{3.338113in}{0.637273in}}%
\pgfpathlineto{\pgfqpoint{3.354164in}{0.637273in}}%
\pgfpathlineto{\pgfqpoint{3.354372in}{0.914287in}}%
\pgfpathlineto{\pgfqpoint{3.355704in}{0.637273in}}%
\pgfpathlineto{\pgfqpoint{3.360535in}{0.637273in}}%
\pgfpathlineto{\pgfqpoint{3.360563in}{0.911544in}}%
\pgfpathlineto{\pgfqpoint{3.362074in}{0.637273in}}%
\pgfpathlineto{\pgfqpoint{3.366017in}{0.637273in}}%
\pgfpathlineto{\pgfqpoint{3.366064in}{0.914287in}}%
\pgfpathlineto{\pgfqpoint{3.367557in}{0.637273in}}%
\pgfpathlineto{\pgfqpoint{3.371571in}{0.637273in}}%
\pgfpathlineto{\pgfqpoint{3.371599in}{0.911544in}}%
\pgfpathlineto{\pgfqpoint{3.373110in}{0.637273in}}%
\pgfpathlineto{\pgfqpoint{3.376964in}{0.637273in}}%
\pgfpathlineto{\pgfqpoint{3.377039in}{0.914287in}}%
\pgfpathlineto{\pgfqpoint{3.378503in}{0.637273in}}%
\pgfpathlineto{\pgfqpoint{3.382767in}{0.637273in}}%
\pgfpathlineto{\pgfqpoint{3.382796in}{0.911544in}}%
\pgfpathlineto{\pgfqpoint{3.384307in}{0.637273in}}%
\pgfpathlineto{\pgfqpoint{3.388165in}{0.637273in}}%
\pgfpathlineto{\pgfqpoint{3.388231in}{0.914287in}}%
\pgfpathlineto{\pgfqpoint{3.389704in}{0.637273in}}%
\pgfpathlineto{\pgfqpoint{3.393647in}{0.637273in}}%
\pgfpathlineto{\pgfqpoint{3.393676in}{0.911544in}}%
\pgfpathlineto{\pgfqpoint{3.395187in}{0.637273in}}%
\pgfpathlineto{\pgfqpoint{3.399078in}{0.637273in}}%
\pgfpathlineto{\pgfqpoint{3.399154in}{0.914287in}}%
\pgfpathlineto{\pgfqpoint{3.400618in}{0.637273in}}%
\pgfpathlineto{\pgfqpoint{3.404570in}{0.637273in}}%
\pgfpathlineto{\pgfqpoint{3.404599in}{0.911544in}}%
\pgfpathlineto{\pgfqpoint{3.406110in}{0.637273in}}%
\pgfpathlineto{\pgfqpoint{3.409916in}{0.637273in}}%
\pgfpathlineto{\pgfqpoint{3.410449in}{0.914287in}}%
\pgfpathlineto{\pgfqpoint{3.411455in}{0.637273in}}%
\pgfpathlineto{\pgfqpoint{3.416470in}{0.637273in}}%
\pgfpathlineto{\pgfqpoint{3.416499in}{0.911544in}}%
\pgfpathlineto{\pgfqpoint{3.418010in}{0.637273in}}%
\pgfpathlineto{\pgfqpoint{3.421882in}{0.637273in}}%
\pgfpathlineto{\pgfqpoint{3.422014in}{0.914287in}}%
\pgfpathlineto{\pgfqpoint{3.423422in}{0.637273in}}%
\pgfpathlineto{\pgfqpoint{3.427388in}{0.637273in}}%
\pgfpathlineto{\pgfqpoint{3.427417in}{0.911544in}}%
\pgfpathlineto{\pgfqpoint{3.428928in}{0.637273in}}%
\pgfpathlineto{\pgfqpoint{3.432829in}{0.637273in}}%
\pgfpathlineto{\pgfqpoint{3.432928in}{0.914287in}}%
\pgfpathlineto{\pgfqpoint{3.434368in}{0.637273in}}%
\pgfpathlineto{\pgfqpoint{3.438269in}{0.637273in}}%
\pgfpathlineto{\pgfqpoint{3.439341in}{0.914287in}}%
\pgfpathlineto{\pgfqpoint{3.439808in}{0.637273in}}%
\pgfpathlineto{\pgfqpoint{3.444988in}{0.637273in}}%
\pgfpathlineto{\pgfqpoint{3.445017in}{0.911544in}}%
\pgfpathlineto{\pgfqpoint{3.446528in}{0.637273in}}%
\pgfpathlineto{\pgfqpoint{3.450396in}{0.637273in}}%
\pgfpathlineto{\pgfqpoint{3.450462in}{0.914287in}}%
\pgfpathlineto{\pgfqpoint{3.451935in}{0.637273in}}%
\pgfpathlineto{\pgfqpoint{3.455817in}{0.637273in}}%
\pgfpathlineto{\pgfqpoint{3.455930in}{0.911544in}}%
\pgfpathlineto{\pgfqpoint{3.457356in}{0.637273in}}%
\pgfpathlineto{\pgfqpoint{3.461295in}{0.637273in}}%
\pgfpathlineto{\pgfqpoint{3.461361in}{0.914287in}}%
\pgfpathlineto{\pgfqpoint{3.462834in}{0.637273in}}%
\pgfpathlineto{\pgfqpoint{3.467141in}{0.637273in}}%
\pgfpathlineto{\pgfqpoint{3.467453in}{0.914287in}}%
\pgfpathlineto{\pgfqpoint{3.468680in}{0.637273in}}%
\pgfpathlineto{\pgfqpoint{3.473445in}{0.637273in}}%
\pgfpathlineto{\pgfqpoint{3.473473in}{0.911544in}}%
\pgfpathlineto{\pgfqpoint{3.474985in}{0.637273in}}%
\pgfpathlineto{\pgfqpoint{3.478838in}{0.637273in}}%
\pgfpathlineto{\pgfqpoint{3.478923in}{0.914287in}}%
\pgfpathlineto{\pgfqpoint{3.480377in}{0.637273in}}%
\pgfpathlineto{\pgfqpoint{3.484391in}{0.637273in}}%
\pgfpathlineto{\pgfqpoint{3.484420in}{0.911544in}}%
\pgfpathlineto{\pgfqpoint{3.485931in}{0.637273in}}%
\pgfpathlineto{\pgfqpoint{3.489784in}{0.637273in}}%
\pgfpathlineto{\pgfqpoint{3.489841in}{0.914287in}}%
\pgfpathlineto{\pgfqpoint{3.491324in}{0.637273in}}%
\pgfpathlineto{\pgfqpoint{3.500882in}{0.637273in}}%
\pgfpathlineto{\pgfqpoint{3.500910in}{0.914287in}}%
\pgfpathlineto{\pgfqpoint{3.502421in}{0.637273in}}%
\pgfpathlineto{\pgfqpoint{3.507715in}{0.637273in}}%
\pgfpathlineto{\pgfqpoint{3.507809in}{0.911544in}}%
\pgfpathlineto{\pgfqpoint{3.509254in}{0.637273in}}%
\pgfpathlineto{\pgfqpoint{3.513174in}{0.637273in}}%
\pgfpathlineto{\pgfqpoint{3.513183in}{0.914287in}}%
\pgfpathlineto{\pgfqpoint{3.514713in}{0.637273in}}%
\pgfpathlineto{\pgfqpoint{3.518940in}{0.637273in}}%
\pgfpathlineto{\pgfqpoint{3.519039in}{0.914287in}}%
\pgfpathlineto{\pgfqpoint{3.520479in}{0.637273in}}%
\pgfpathlineto{\pgfqpoint{3.524380in}{0.637273in}}%
\pgfpathlineto{\pgfqpoint{3.524408in}{0.911544in}}%
\pgfpathlineto{\pgfqpoint{3.525919in}{0.637273in}}%
\pgfpathlineto{\pgfqpoint{3.529806in}{0.637273in}}%
\pgfpathlineto{\pgfqpoint{3.529938in}{0.914287in}}%
\pgfpathlineto{\pgfqpoint{3.531345in}{0.637273in}}%
\pgfpathlineto{\pgfqpoint{3.535270in}{0.637273in}}%
\pgfpathlineto{\pgfqpoint{3.535298in}{0.911544in}}%
\pgfpathlineto{\pgfqpoint{3.536809in}{0.637273in}}%
\pgfpathlineto{\pgfqpoint{3.540663in}{0.637273in}}%
\pgfpathlineto{\pgfqpoint{3.540766in}{0.914287in}}%
\pgfpathlineto{\pgfqpoint{3.542202in}{0.637273in}}%
\pgfpathlineto{\pgfqpoint{3.546131in}{0.637273in}}%
\pgfpathlineto{\pgfqpoint{3.546159in}{0.911544in}}%
\pgfpathlineto{\pgfqpoint{3.547670in}{0.637273in}}%
\pgfpathlineto{\pgfqpoint{3.551798in}{0.637273in}}%
\pgfpathlineto{\pgfqpoint{3.551869in}{0.914287in}}%
\pgfpathlineto{\pgfqpoint{3.553337in}{0.637273in}}%
\pgfpathlineto{\pgfqpoint{3.557233in}{0.637273in}}%
\pgfpathlineto{\pgfqpoint{3.557261in}{0.911544in}}%
\pgfpathlineto{\pgfqpoint{3.558773in}{0.637273in}}%
\pgfpathlineto{\pgfqpoint{3.562683in}{0.637273in}}%
\pgfpathlineto{\pgfqpoint{3.562815in}{0.914287in}}%
\pgfpathlineto{\pgfqpoint{3.564222in}{0.637273in}}%
\pgfpathlineto{\pgfqpoint{3.568184in}{0.637273in}}%
\pgfpathlineto{\pgfqpoint{3.568213in}{0.911544in}}%
\pgfpathlineto{\pgfqpoint{3.569724in}{0.637273in}}%
\pgfpathlineto{\pgfqpoint{3.573596in}{0.637273in}}%
\pgfpathlineto{\pgfqpoint{3.574130in}{0.914287in}}%
\pgfpathlineto{\pgfqpoint{3.575135in}{0.637273in}}%
\pgfpathlineto{\pgfqpoint{3.580226in}{0.637273in}}%
\pgfpathlineto{\pgfqpoint{3.580254in}{0.911544in}}%
\pgfpathlineto{\pgfqpoint{3.581766in}{0.637273in}}%
\pgfpathlineto{\pgfqpoint{3.585610in}{0.637273in}}%
\pgfpathlineto{\pgfqpoint{3.585742in}{0.914287in}}%
\pgfpathlineto{\pgfqpoint{3.587149in}{0.637273in}}%
\pgfpathlineto{\pgfqpoint{3.591116in}{0.637273in}}%
\pgfpathlineto{\pgfqpoint{3.591144in}{0.911544in}}%
\pgfpathlineto{\pgfqpoint{3.592655in}{0.637273in}}%
\pgfpathlineto{\pgfqpoint{3.596532in}{0.637273in}}%
\pgfpathlineto{\pgfqpoint{3.596664in}{0.914287in}}%
\pgfpathlineto{\pgfqpoint{3.598072in}{0.637273in}}%
\pgfpathlineto{\pgfqpoint{3.601953in}{0.637273in}}%
\pgfpathlineto{\pgfqpoint{3.603025in}{0.914287in}}%
\pgfpathlineto{\pgfqpoint{3.603493in}{0.637273in}}%
\pgfpathlineto{\pgfqpoint{3.608659in}{0.637273in}}%
\pgfpathlineto{\pgfqpoint{3.608687in}{0.911544in}}%
\pgfpathlineto{\pgfqpoint{3.610199in}{0.637273in}}%
\pgfpathlineto{\pgfqpoint{3.614099in}{0.637273in}}%
\pgfpathlineto{\pgfqpoint{3.614203in}{0.914287in}}%
\pgfpathlineto{\pgfqpoint{3.615639in}{0.637273in}}%
\pgfpathlineto{\pgfqpoint{3.625017in}{0.637273in}}%
\pgfpathlineto{\pgfqpoint{3.625121in}{0.914287in}}%
\pgfpathlineto{\pgfqpoint{3.626557in}{0.637273in}}%
\pgfpathlineto{\pgfqpoint{3.630457in}{0.637273in}}%
\pgfpathlineto{\pgfqpoint{3.630486in}{0.911544in}}%
\pgfpathlineto{\pgfqpoint{3.631997in}{0.637273in}}%
\pgfpathlineto{\pgfqpoint{3.636105in}{0.637273in}}%
\pgfpathlineto{\pgfqpoint{3.636209in}{0.914287in}}%
\pgfpathlineto{\pgfqpoint{3.637645in}{0.637273in}}%
\pgfpathlineto{\pgfqpoint{3.641545in}{0.637273in}}%
\pgfpathlineto{\pgfqpoint{3.641678in}{0.911544in}}%
\pgfpathlineto{\pgfqpoint{3.643085in}{0.637273in}}%
\pgfpathlineto{\pgfqpoint{3.647080in}{0.637273in}}%
\pgfpathlineto{\pgfqpoint{3.647127in}{0.914287in}}%
\pgfpathlineto{\pgfqpoint{3.648619in}{0.637273in}}%
\pgfpathlineto{\pgfqpoint{3.654338in}{0.637273in}}%
\pgfpathlineto{\pgfqpoint{3.654366in}{0.911544in}}%
\pgfpathlineto{\pgfqpoint{3.655878in}{0.637273in}}%
\pgfpathlineto{\pgfqpoint{3.659981in}{0.637273in}}%
\pgfpathlineto{\pgfqpoint{3.660028in}{0.914287in}}%
\pgfpathlineto{\pgfqpoint{3.661521in}{0.637273in}}%
\pgfpathlineto{\pgfqpoint{3.665469in}{0.637273in}}%
\pgfpathlineto{\pgfqpoint{3.665497in}{0.911544in}}%
\pgfpathlineto{\pgfqpoint{3.667008in}{0.637273in}}%
\pgfpathlineto{\pgfqpoint{3.670861in}{0.637273in}}%
\pgfpathlineto{\pgfqpoint{3.670909in}{0.914287in}}%
\pgfpathlineto{\pgfqpoint{3.672401in}{0.637273in}}%
\pgfpathlineto{\pgfqpoint{3.676387in}{0.637273in}}%
\pgfpathlineto{\pgfqpoint{3.676415in}{0.911544in}}%
\pgfpathlineto{\pgfqpoint{3.677926in}{0.637273in}}%
\pgfpathlineto{\pgfqpoint{3.681794in}{0.637273in}}%
\pgfpathlineto{\pgfqpoint{3.681841in}{0.914287in}}%
\pgfpathlineto{\pgfqpoint{3.683333in}{0.637273in}}%
\pgfpathlineto{\pgfqpoint{3.687531in}{0.637273in}}%
\pgfpathlineto{\pgfqpoint{3.687951in}{0.914287in}}%
\pgfpathlineto{\pgfqpoint{3.689071in}{0.637273in}}%
\pgfpathlineto{\pgfqpoint{3.693854in}{0.637273in}}%
\pgfpathlineto{\pgfqpoint{3.693883in}{0.911544in}}%
\pgfpathlineto{\pgfqpoint{3.695394in}{0.637273in}}%
\pgfpathlineto{\pgfqpoint{3.699294in}{0.637273in}}%
\pgfpathlineto{\pgfqpoint{3.699394in}{0.914287in}}%
\pgfpathlineto{\pgfqpoint{3.700834in}{0.637273in}}%
\pgfpathlineto{\pgfqpoint{3.704820in}{0.637273in}}%
\pgfpathlineto{\pgfqpoint{3.704848in}{0.911544in}}%
\pgfpathlineto{\pgfqpoint{3.706359in}{0.637273in}}%
\pgfpathlineto{\pgfqpoint{3.710212in}{0.637273in}}%
\pgfpathlineto{\pgfqpoint{3.710316in}{0.914287in}}%
\pgfpathlineto{\pgfqpoint{3.711752in}{0.637273in}}%
\pgfpathlineto{\pgfqpoint{3.715908in}{0.637273in}}%
\pgfpathlineto{\pgfqpoint{3.715936in}{0.911544in}}%
\pgfpathlineto{\pgfqpoint{3.717447in}{0.637273in}}%
\pgfpathlineto{\pgfqpoint{3.721239in}{0.637273in}}%
\pgfpathlineto{\pgfqpoint{3.721367in}{0.914287in}}%
\pgfpathlineto{\pgfqpoint{3.722779in}{0.637273in}}%
\pgfpathlineto{\pgfqpoint{3.726707in}{0.637273in}}%
\pgfpathlineto{\pgfqpoint{3.726736in}{0.911544in}}%
\pgfpathlineto{\pgfqpoint{3.728247in}{0.637273in}}%
\pgfpathlineto{\pgfqpoint{3.732063in}{0.637273in}}%
\pgfpathlineto{\pgfqpoint{3.732195in}{0.914287in}}%
\pgfpathlineto{\pgfqpoint{3.733602in}{0.637273in}}%
\pgfpathlineto{\pgfqpoint{3.737559in}{0.637273in}}%
\pgfpathlineto{\pgfqpoint{3.737588in}{0.911544in}}%
\pgfpathlineto{\pgfqpoint{3.739099in}{0.637273in}}%
\pgfpathlineto{\pgfqpoint{3.743207in}{0.637273in}}%
\pgfpathlineto{\pgfqpoint{3.743273in}{0.914287in}}%
\pgfpathlineto{\pgfqpoint{3.744747in}{0.637273in}}%
\pgfpathlineto{\pgfqpoint{3.748690in}{0.637273in}}%
\pgfpathlineto{\pgfqpoint{3.748718in}{0.911544in}}%
\pgfpathlineto{\pgfqpoint{3.750229in}{0.637273in}}%
\pgfpathlineto{\pgfqpoint{3.754083in}{0.637273in}}%
\pgfpathlineto{\pgfqpoint{3.754158in}{0.914287in}}%
\pgfpathlineto{\pgfqpoint{3.755622in}{0.637273in}}%
\pgfpathlineto{\pgfqpoint{3.759608in}{0.637273in}}%
\pgfpathlineto{\pgfqpoint{3.759636in}{0.911544in}}%
\pgfpathlineto{\pgfqpoint{3.761147in}{0.637273in}}%
\pgfpathlineto{\pgfqpoint{3.765005in}{0.637273in}}%
\pgfpathlineto{\pgfqpoint{3.766153in}{0.914287in}}%
\pgfpathlineto{\pgfqpoint{3.766545in}{0.637273in}}%
\pgfpathlineto{\pgfqpoint{3.773166in}{0.637273in}}%
\pgfpathlineto{\pgfqpoint{3.773194in}{0.911544in}}%
\pgfpathlineto{\pgfqpoint{3.774705in}{0.637273in}}%
\pgfpathlineto{\pgfqpoint{3.778606in}{0.637273in}}%
\pgfpathlineto{\pgfqpoint{3.778634in}{0.911544in}}%
\pgfpathlineto{\pgfqpoint{3.780145in}{0.637273in}}%
\pgfpathlineto{\pgfqpoint{3.783970in}{0.637273in}}%
\pgfpathlineto{\pgfqpoint{3.784017in}{0.914287in}}%
\pgfpathlineto{\pgfqpoint{3.785510in}{0.637273in}}%
\pgfpathlineto{\pgfqpoint{3.789495in}{0.637273in}}%
\pgfpathlineto{\pgfqpoint{3.789599in}{0.914287in}}%
\pgfpathlineto{\pgfqpoint{3.791035in}{0.637273in}}%
\pgfpathlineto{\pgfqpoint{3.794935in}{0.637273in}}%
\pgfpathlineto{\pgfqpoint{3.794964in}{0.911544in}}%
\pgfpathlineto{\pgfqpoint{3.796475in}{0.637273in}}%
\pgfpathlineto{\pgfqpoint{3.800734in}{0.637273in}}%
\pgfpathlineto{\pgfqpoint{3.800853in}{0.914287in}}%
\pgfpathlineto{\pgfqpoint{3.802274in}{0.637273in}}%
\pgfpathlineto{\pgfqpoint{3.806968in}{0.637273in}}%
\pgfpathlineto{\pgfqpoint{3.807095in}{0.914287in}}%
\pgfpathlineto{\pgfqpoint{3.808507in}{0.637273in}}%
\pgfpathlineto{\pgfqpoint{3.812436in}{0.637273in}}%
\pgfpathlineto{\pgfqpoint{3.812465in}{0.911544in}}%
\pgfpathlineto{\pgfqpoint{3.813976in}{0.637273in}}%
\pgfpathlineto{\pgfqpoint{3.819298in}{0.637273in}}%
\pgfpathlineto{\pgfqpoint{3.819430in}{0.914287in}}%
\pgfpathlineto{\pgfqpoint{3.820837in}{0.637273in}}%
\pgfpathlineto{\pgfqpoint{3.824766in}{0.637273in}}%
\pgfpathlineto{\pgfqpoint{3.825120in}{0.911544in}}%
\pgfpathlineto{\pgfqpoint{3.826306in}{0.637273in}}%
\pgfpathlineto{\pgfqpoint{3.830490in}{0.637273in}}%
\pgfpathlineto{\pgfqpoint{3.830556in}{0.914287in}}%
\pgfpathlineto{\pgfqpoint{3.832029in}{0.637273in}}%
\pgfpathlineto{\pgfqpoint{3.841219in}{0.637273in}}%
\pgfpathlineto{\pgfqpoint{3.841408in}{0.914287in}}%
\pgfpathlineto{\pgfqpoint{3.842758in}{0.637273in}}%
\pgfpathlineto{\pgfqpoint{3.846711in}{0.637273in}}%
\pgfpathlineto{\pgfqpoint{3.846739in}{0.911544in}}%
\pgfpathlineto{\pgfqpoint{3.848250in}{0.637273in}}%
\pgfpathlineto{\pgfqpoint{3.852104in}{0.637273in}}%
\pgfpathlineto{\pgfqpoint{3.852482in}{0.914287in}}%
\pgfpathlineto{\pgfqpoint{3.853643in}{0.637273in}}%
\pgfpathlineto{\pgfqpoint{3.857841in}{0.637273in}}%
\pgfpathlineto{\pgfqpoint{3.857870in}{0.911544in}}%
\pgfpathlineto{\pgfqpoint{3.859381in}{0.637273in}}%
\pgfpathlineto{\pgfqpoint{3.863300in}{0.637273in}}%
\pgfpathlineto{\pgfqpoint{3.863409in}{0.914287in}}%
\pgfpathlineto{\pgfqpoint{3.864840in}{0.637273in}}%
\pgfpathlineto{\pgfqpoint{3.868807in}{0.637273in}}%
\pgfpathlineto{\pgfqpoint{3.868911in}{0.911544in}}%
\pgfpathlineto{\pgfqpoint{3.870346in}{0.637273in}}%
\pgfpathlineto{\pgfqpoint{3.874280in}{0.637273in}}%
\pgfpathlineto{\pgfqpoint{3.874355in}{0.914287in}}%
\pgfpathlineto{\pgfqpoint{3.875819in}{0.637273in}}%
\pgfpathlineto{\pgfqpoint{3.879748in}{0.637273in}}%
\pgfpathlineto{\pgfqpoint{3.879777in}{0.911544in}}%
\pgfpathlineto{\pgfqpoint{3.881288in}{0.637273in}}%
\pgfpathlineto{\pgfqpoint{3.885415in}{0.637273in}}%
\pgfpathlineto{\pgfqpoint{3.885486in}{0.914287in}}%
\pgfpathlineto{\pgfqpoint{3.886954in}{0.637273in}}%
\pgfpathlineto{\pgfqpoint{3.890879in}{0.637273in}}%
\pgfpathlineto{\pgfqpoint{3.890907in}{0.911544in}}%
\pgfpathlineto{\pgfqpoint{3.892418in}{0.637273in}}%
\pgfpathlineto{\pgfqpoint{3.896333in}{0.637273in}}%
\pgfpathlineto{\pgfqpoint{3.896399in}{0.914287in}}%
\pgfpathlineto{\pgfqpoint{3.897872in}{0.637273in}}%
\pgfpathlineto{\pgfqpoint{3.901797in}{0.637273in}}%
\pgfpathlineto{\pgfqpoint{3.901825in}{0.911544in}}%
\pgfpathlineto{\pgfqpoint{3.903336in}{0.637273in}}%
\pgfpathlineto{\pgfqpoint{3.907416in}{0.637273in}}%
\pgfpathlineto{\pgfqpoint{3.907492in}{0.914287in}}%
\pgfpathlineto{\pgfqpoint{3.908956in}{0.637273in}}%
\pgfpathlineto{\pgfqpoint{3.912837in}{0.637273in}}%
\pgfpathlineto{\pgfqpoint{3.912866in}{0.911544in}}%
\pgfpathlineto{\pgfqpoint{3.914377in}{0.637273in}}%
\pgfpathlineto{\pgfqpoint{3.918174in}{0.637273in}}%
\pgfpathlineto{\pgfqpoint{3.918301in}{0.914287in}}%
\pgfpathlineto{\pgfqpoint{3.919713in}{0.637273in}}%
\pgfpathlineto{\pgfqpoint{3.923708in}{0.637273in}}%
\pgfpathlineto{\pgfqpoint{3.923737in}{0.911544in}}%
\pgfpathlineto{\pgfqpoint{3.925248in}{0.637273in}}%
\pgfpathlineto{\pgfqpoint{3.929054in}{0.637273in}}%
\pgfpathlineto{\pgfqpoint{3.929158in}{0.914287in}}%
\pgfpathlineto{\pgfqpoint{3.930593in}{0.637273in}}%
\pgfpathlineto{\pgfqpoint{3.934447in}{0.637273in}}%
\pgfpathlineto{\pgfqpoint{3.935519in}{0.914287in}}%
\pgfpathlineto{\pgfqpoint{3.935986in}{0.637273in}}%
\pgfpathlineto{\pgfqpoint{3.941185in}{0.637273in}}%
\pgfpathlineto{\pgfqpoint{3.941214in}{0.911544in}}%
\pgfpathlineto{\pgfqpoint{3.942725in}{0.637273in}}%
\pgfpathlineto{\pgfqpoint{3.946578in}{0.637273in}}%
\pgfpathlineto{\pgfqpoint{3.946701in}{0.914287in}}%
\pgfpathlineto{\pgfqpoint{3.948118in}{0.637273in}}%
\pgfpathlineto{\pgfqpoint{3.952160in}{0.637273in}}%
\pgfpathlineto{\pgfqpoint{3.952188in}{0.911544in}}%
\pgfpathlineto{\pgfqpoint{3.953700in}{0.637273in}}%
\pgfpathlineto{\pgfqpoint{3.957553in}{0.637273in}}%
\pgfpathlineto{\pgfqpoint{3.957619in}{0.914287in}}%
\pgfpathlineto{\pgfqpoint{3.959093in}{0.637273in}}%
\pgfpathlineto{\pgfqpoint{3.963012in}{0.637273in}}%
\pgfpathlineto{\pgfqpoint{3.963380in}{0.911544in}}%
\pgfpathlineto{\pgfqpoint{3.964552in}{0.637273in}}%
\pgfpathlineto{\pgfqpoint{3.968745in}{0.637273in}}%
\pgfpathlineto{\pgfqpoint{3.968811in}{0.914287in}}%
\pgfpathlineto{\pgfqpoint{3.970284in}{0.637273in}}%
\pgfpathlineto{\pgfqpoint{3.974232in}{0.637273in}}%
\pgfpathlineto{\pgfqpoint{3.974261in}{0.911544in}}%
\pgfpathlineto{\pgfqpoint{3.975772in}{0.637273in}}%
\pgfpathlineto{\pgfqpoint{3.979663in}{0.637273in}}%
\pgfpathlineto{\pgfqpoint{3.979729in}{0.914287in}}%
\pgfpathlineto{\pgfqpoint{3.981202in}{0.637273in}}%
\pgfpathlineto{\pgfqpoint{3.986879in}{0.637273in}}%
\pgfpathlineto{\pgfqpoint{3.986907in}{0.911544in}}%
\pgfpathlineto{\pgfqpoint{3.988418in}{0.637273in}}%
\pgfpathlineto{\pgfqpoint{3.992475in}{0.637273in}}%
\pgfpathlineto{\pgfqpoint{3.992541in}{0.914287in}}%
\pgfpathlineto{\pgfqpoint{3.994014in}{0.637273in}}%
\pgfpathlineto{\pgfqpoint{3.997915in}{0.637273in}}%
\pgfpathlineto{\pgfqpoint{3.997943in}{0.911544in}}%
\pgfpathlineto{\pgfqpoint{3.999454in}{0.637273in}}%
\pgfpathlineto{\pgfqpoint{4.003260in}{0.637273in}}%
\pgfpathlineto{\pgfqpoint{4.003326in}{0.914287in}}%
\pgfpathlineto{\pgfqpoint{4.004800in}{0.637273in}}%
\pgfpathlineto{\pgfqpoint{4.008710in}{0.637273in}}%
\pgfpathlineto{\pgfqpoint{4.008738in}{0.911544in}}%
\pgfpathlineto{\pgfqpoint{4.010249in}{0.637273in}}%
\pgfpathlineto{\pgfqpoint{4.014098in}{0.637273in}}%
\pgfpathlineto{\pgfqpoint{4.014192in}{0.914287in}}%
\pgfpathlineto{\pgfqpoint{4.015637in}{0.637273in}}%
\pgfpathlineto{\pgfqpoint{4.019925in}{0.637273in}}%
\pgfpathlineto{\pgfqpoint{4.020043in}{0.914287in}}%
\pgfpathlineto{\pgfqpoint{4.021465in}{0.637273in}}%
\pgfpathlineto{\pgfqpoint{4.026159in}{0.637273in}}%
\pgfpathlineto{\pgfqpoint{4.026187in}{0.911544in}}%
\pgfpathlineto{\pgfqpoint{4.027698in}{0.637273in}}%
\pgfpathlineto{\pgfqpoint{4.031575in}{0.637273in}}%
\pgfpathlineto{\pgfqpoint{4.031623in}{0.914287in}}%
\pgfpathlineto{\pgfqpoint{4.033115in}{0.637273in}}%
\pgfpathlineto{\pgfqpoint{4.037077in}{0.637273in}}%
\pgfpathlineto{\pgfqpoint{4.037105in}{0.911544in}}%
\pgfpathlineto{\pgfqpoint{4.038616in}{0.637273in}}%
\pgfpathlineto{\pgfqpoint{4.042470in}{0.637273in}}%
\pgfpathlineto{\pgfqpoint{4.042479in}{0.914287in}}%
\pgfpathlineto{\pgfqpoint{4.044009in}{0.637273in}}%
\pgfpathlineto{\pgfqpoint{4.048221in}{0.637273in}}%
\pgfpathlineto{\pgfqpoint{4.048354in}{0.914287in}}%
\pgfpathlineto{\pgfqpoint{4.049761in}{0.637273in}}%
\pgfpathlineto{\pgfqpoint{4.053695in}{0.637273in}}%
\pgfpathlineto{\pgfqpoint{4.053723in}{0.911544in}}%
\pgfpathlineto{\pgfqpoint{4.055234in}{0.637273in}}%
\pgfpathlineto{\pgfqpoint{4.059097in}{0.637273in}}%
\pgfpathlineto{\pgfqpoint{4.059224in}{0.914287in}}%
\pgfpathlineto{\pgfqpoint{4.060636in}{0.637273in}}%
\pgfpathlineto{\pgfqpoint{4.064594in}{0.637273in}}%
\pgfpathlineto{\pgfqpoint{4.064622in}{0.911544in}}%
\pgfpathlineto{\pgfqpoint{4.066133in}{0.637273in}}%
\pgfpathlineto{\pgfqpoint{4.070298in}{0.637273in}}%
\pgfpathlineto{\pgfqpoint{4.070327in}{0.914287in}}%
\pgfpathlineto{\pgfqpoint{4.071838in}{0.637273in}}%
\pgfpathlineto{\pgfqpoint{4.075743in}{0.637273in}}%
\pgfpathlineto{\pgfqpoint{4.075771in}{0.911544in}}%
\pgfpathlineto{\pgfqpoint{4.077283in}{0.637273in}}%
\pgfpathlineto{\pgfqpoint{4.081136in}{0.637273in}}%
\pgfpathlineto{\pgfqpoint{4.081212in}{0.914287in}}%
\pgfpathlineto{\pgfqpoint{4.082675in}{0.637273in}}%
\pgfpathlineto{\pgfqpoint{4.086666in}{0.637273in}}%
\pgfpathlineto{\pgfqpoint{4.086694in}{0.911544in}}%
\pgfpathlineto{\pgfqpoint{4.088205in}{0.637273in}}%
\pgfpathlineto{\pgfqpoint{4.092016in}{0.637273in}}%
\pgfpathlineto{\pgfqpoint{4.092082in}{0.914287in}}%
\pgfpathlineto{\pgfqpoint{4.093556in}{0.637273in}}%
\pgfpathlineto{\pgfqpoint{4.097433in}{0.637273in}}%
\pgfpathlineto{\pgfqpoint{4.098505in}{0.914287in}}%
\pgfpathlineto{\pgfqpoint{4.098972in}{0.637273in}}%
\pgfpathlineto{\pgfqpoint{4.104110in}{0.637273in}}%
\pgfpathlineto{\pgfqpoint{4.104138in}{0.911544in}}%
\pgfpathlineto{\pgfqpoint{4.105649in}{0.637273in}}%
\pgfpathlineto{\pgfqpoint{4.109531in}{0.637273in}}%
\pgfpathlineto{\pgfqpoint{4.109654in}{0.914287in}}%
\pgfpathlineto{\pgfqpoint{4.111071in}{0.637273in}}%
\pgfpathlineto{\pgfqpoint{4.115028in}{0.637273in}}%
\pgfpathlineto{\pgfqpoint{4.115056in}{0.911544in}}%
\pgfpathlineto{\pgfqpoint{4.116567in}{0.637273in}}%
\pgfpathlineto{\pgfqpoint{4.120449in}{0.637273in}}%
\pgfpathlineto{\pgfqpoint{4.120572in}{0.914287in}}%
\pgfpathlineto{\pgfqpoint{4.121989in}{0.637273in}}%
\pgfpathlineto{\pgfqpoint{4.125908in}{0.637273in}}%
\pgfpathlineto{\pgfqpoint{4.125937in}{0.911544in}}%
\pgfpathlineto{\pgfqpoint{4.127448in}{0.637273in}}%
\pgfpathlineto{\pgfqpoint{4.131537in}{0.637273in}}%
\pgfpathlineto{\pgfqpoint{4.131660in}{0.914287in}}%
\pgfpathlineto{\pgfqpoint{4.133077in}{0.637273in}}%
\pgfpathlineto{\pgfqpoint{4.136949in}{0.637273in}}%
\pgfpathlineto{\pgfqpoint{4.136977in}{0.911544in}}%
\pgfpathlineto{\pgfqpoint{4.138488in}{0.637273in}}%
\pgfpathlineto{\pgfqpoint{4.142361in}{0.637273in}}%
\pgfpathlineto{\pgfqpoint{4.142484in}{0.914287in}}%
\pgfpathlineto{\pgfqpoint{4.143900in}{0.637273in}}%
\pgfpathlineto{\pgfqpoint{4.147773in}{0.637273in}}%
\pgfpathlineto{\pgfqpoint{4.147886in}{0.911544in}}%
\pgfpathlineto{\pgfqpoint{4.149312in}{0.637273in}}%
\pgfpathlineto{\pgfqpoint{4.153453in}{0.637273in}}%
\pgfpathlineto{\pgfqpoint{4.153520in}{0.914287in}}%
\pgfpathlineto{\pgfqpoint{4.154993in}{0.637273in}}%
\pgfpathlineto{\pgfqpoint{4.158809in}{0.637273in}}%
\pgfpathlineto{\pgfqpoint{4.158837in}{0.911544in}}%
\pgfpathlineto{\pgfqpoint{4.160348in}{0.637273in}}%
\pgfpathlineto{\pgfqpoint{4.164183in}{0.637273in}}%
\pgfpathlineto{\pgfqpoint{4.164305in}{0.914287in}}%
\pgfpathlineto{\pgfqpoint{4.165722in}{0.637273in}}%
\pgfpathlineto{\pgfqpoint{4.169590in}{0.637273in}}%
\pgfpathlineto{\pgfqpoint{4.169623in}{0.911544in}}%
\pgfpathlineto{\pgfqpoint{4.171129in}{0.637273in}}%
\pgfpathlineto{\pgfqpoint{4.175006in}{0.637273in}}%
\pgfpathlineto{\pgfqpoint{4.175129in}{0.914287in}}%
\pgfpathlineto{\pgfqpoint{4.176546in}{0.637273in}}%
\pgfpathlineto{\pgfqpoint{4.180800in}{0.637273in}}%
\pgfpathlineto{\pgfqpoint{4.180829in}{0.911544in}}%
\pgfpathlineto{\pgfqpoint{4.182340in}{0.637273in}}%
\pgfpathlineto{\pgfqpoint{4.186193in}{0.637273in}}%
\pgfpathlineto{\pgfqpoint{4.186259in}{0.914287in}}%
\pgfpathlineto{\pgfqpoint{4.187733in}{0.637273in}}%
\pgfpathlineto{\pgfqpoint{4.191681in}{0.637273in}}%
\pgfpathlineto{\pgfqpoint{4.191709in}{0.911544in}}%
\pgfpathlineto{\pgfqpoint{4.193220in}{0.637273in}}%
\pgfpathlineto{\pgfqpoint{4.197111in}{0.637273in}}%
\pgfpathlineto{\pgfqpoint{4.197206in}{0.914287in}}%
\pgfpathlineto{\pgfqpoint{4.198651in}{0.637273in}}%
\pgfpathlineto{\pgfqpoint{4.202570in}{0.637273in}}%
\pgfpathlineto{\pgfqpoint{4.202599in}{0.911544in}}%
\pgfpathlineto{\pgfqpoint{4.204110in}{0.637273in}}%
\pgfpathlineto{\pgfqpoint{4.208242in}{0.637273in}}%
\pgfpathlineto{\pgfqpoint{4.208308in}{0.914287in}}%
\pgfpathlineto{\pgfqpoint{4.209781in}{0.637273in}}%
\pgfpathlineto{\pgfqpoint{4.213701in}{0.637273in}}%
\pgfpathlineto{\pgfqpoint{4.213729in}{0.911544in}}%
\pgfpathlineto{\pgfqpoint{4.215240in}{0.637273in}}%
\pgfpathlineto{\pgfqpoint{4.220694in}{0.637273in}}%
\pgfpathlineto{\pgfqpoint{4.220742in}{0.914287in}}%
\pgfpathlineto{\pgfqpoint{4.222234in}{0.637273in}}%
\pgfpathlineto{\pgfqpoint{4.226172in}{0.637273in}}%
\pgfpathlineto{\pgfqpoint{4.226201in}{0.911544in}}%
\pgfpathlineto{\pgfqpoint{4.227712in}{0.637273in}}%
\pgfpathlineto{\pgfqpoint{4.231612in}{0.637273in}}%
\pgfpathlineto{\pgfqpoint{4.231660in}{0.914287in}}%
\pgfpathlineto{\pgfqpoint{4.233152in}{0.637273in}}%
\pgfpathlineto{\pgfqpoint{4.237459in}{0.637273in}}%
\pgfpathlineto{\pgfqpoint{4.237577in}{0.914287in}}%
\pgfpathlineto{\pgfqpoint{4.238998in}{0.637273in}}%
\pgfpathlineto{\pgfqpoint{4.243673in}{0.637273in}}%
\pgfpathlineto{\pgfqpoint{4.243720in}{0.911544in}}%
\pgfpathlineto{\pgfqpoint{4.245213in}{0.637273in}}%
\pgfpathlineto{\pgfqpoint{4.249109in}{0.637273in}}%
\pgfpathlineto{\pgfqpoint{4.249175in}{0.914287in}}%
\pgfpathlineto{\pgfqpoint{4.250648in}{0.637273in}}%
\pgfpathlineto{\pgfqpoint{4.254629in}{0.637273in}}%
\pgfpathlineto{\pgfqpoint{4.254657in}{0.911544in}}%
\pgfpathlineto{\pgfqpoint{4.256168in}{0.637273in}}%
\pgfpathlineto{\pgfqpoint{4.260022in}{0.637273in}}%
\pgfpathlineto{\pgfqpoint{4.260097in}{0.914287in}}%
\pgfpathlineto{\pgfqpoint{4.261561in}{0.637273in}}%
\pgfpathlineto{\pgfqpoint{4.265490in}{0.637273in}}%
\pgfpathlineto{\pgfqpoint{4.266562in}{0.914287in}}%
\pgfpathlineto{\pgfqpoint{4.267030in}{0.637273in}}%
\pgfpathlineto{\pgfqpoint{4.272262in}{0.637273in}}%
\pgfpathlineto{\pgfqpoint{4.272290in}{0.911544in}}%
\pgfpathlineto{\pgfqpoint{4.273802in}{0.637273in}}%
\pgfpathlineto{\pgfqpoint{4.277608in}{0.637273in}}%
\pgfpathlineto{\pgfqpoint{4.277674in}{0.914287in}}%
\pgfpathlineto{\pgfqpoint{4.279147in}{0.637273in}}%
\pgfpathlineto{\pgfqpoint{4.283086in}{0.637273in}}%
\pgfpathlineto{\pgfqpoint{4.283114in}{0.911544in}}%
\pgfpathlineto{\pgfqpoint{4.284625in}{0.637273in}}%
\pgfpathlineto{\pgfqpoint{4.288483in}{0.637273in}}%
\pgfpathlineto{\pgfqpoint{4.288545in}{0.914287in}}%
\pgfpathlineto{\pgfqpoint{4.290023in}{0.637273in}}%
\pgfpathlineto{\pgfqpoint{4.294216in}{0.637273in}}%
\pgfpathlineto{\pgfqpoint{4.294244in}{0.911544in}}%
\pgfpathlineto{\pgfqpoint{4.295756in}{0.637273in}}%
\pgfpathlineto{\pgfqpoint{4.299609in}{0.637273in}}%
\pgfpathlineto{\pgfqpoint{4.299675in}{0.914287in}}%
\pgfpathlineto{\pgfqpoint{4.301149in}{0.637273in}}%
\pgfpathlineto{\pgfqpoint{4.305134in}{0.637273in}}%
\pgfpathlineto{\pgfqpoint{4.305162in}{0.911544in}}%
\pgfpathlineto{\pgfqpoint{4.306674in}{0.637273in}}%
\pgfpathlineto{\pgfqpoint{4.310527in}{0.637273in}}%
\pgfpathlineto{\pgfqpoint{4.310536in}{0.914287in}}%
\pgfpathlineto{\pgfqpoint{4.312066in}{0.637273in}}%
\pgfpathlineto{\pgfqpoint{4.317743in}{0.637273in}}%
\pgfpathlineto{\pgfqpoint{4.318116in}{0.914287in}}%
\pgfpathlineto{\pgfqpoint{4.319282in}{0.637273in}}%
\pgfpathlineto{\pgfqpoint{4.323457in}{0.637273in}}%
\pgfpathlineto{\pgfqpoint{4.323480in}{0.911544in}}%
\pgfpathlineto{\pgfqpoint{4.324996in}{0.637273in}}%
\pgfpathlineto{\pgfqpoint{4.328873in}{0.637273in}}%
\pgfpathlineto{\pgfqpoint{4.328996in}{0.914287in}}%
\pgfpathlineto{\pgfqpoint{4.330413in}{0.637273in}}%
\pgfpathlineto{\pgfqpoint{4.334332in}{0.637273in}}%
\pgfpathlineto{\pgfqpoint{4.334361in}{0.911544in}}%
\pgfpathlineto{\pgfqpoint{4.335872in}{0.637273in}}%
\pgfpathlineto{\pgfqpoint{4.339791in}{0.637273in}}%
\pgfpathlineto{\pgfqpoint{4.339928in}{0.914287in}}%
\pgfpathlineto{\pgfqpoint{4.341331in}{0.637273in}}%
\pgfpathlineto{\pgfqpoint{4.345269in}{0.637273in}}%
\pgfpathlineto{\pgfqpoint{4.345297in}{0.911544in}}%
\pgfpathlineto{\pgfqpoint{4.346809in}{0.637273in}}%
\pgfpathlineto{\pgfqpoint{4.350893in}{0.637273in}}%
\pgfpathlineto{\pgfqpoint{4.351016in}{0.914287in}}%
\pgfpathlineto{\pgfqpoint{4.352433in}{0.637273in}}%
\pgfpathlineto{\pgfqpoint{4.356305in}{0.637273in}}%
\pgfpathlineto{\pgfqpoint{4.356418in}{0.911544in}}%
\pgfpathlineto{\pgfqpoint{4.357845in}{0.637273in}}%
\pgfpathlineto{\pgfqpoint{4.363067in}{0.637273in}}%
\pgfpathlineto{\pgfqpoint{4.363266in}{0.936228in}}%
\pgfpathlineto{\pgfqpoint{4.364607in}{0.637273in}}%
\pgfpathlineto{\pgfqpoint{4.368881in}{0.637273in}}%
\pgfpathlineto{\pgfqpoint{4.368909in}{0.911544in}}%
\pgfpathlineto{\pgfqpoint{4.370420in}{0.637273in}}%
\pgfpathlineto{\pgfqpoint{4.374245in}{0.637273in}}%
\pgfpathlineto{\pgfqpoint{4.374599in}{0.914287in}}%
\pgfpathlineto{\pgfqpoint{4.375785in}{0.637273in}}%
\pgfpathlineto{\pgfqpoint{4.379888in}{0.637273in}}%
\pgfpathlineto{\pgfqpoint{4.379917in}{0.911544in}}%
\pgfpathlineto{\pgfqpoint{4.381428in}{0.637273in}}%
\pgfpathlineto{\pgfqpoint{4.385328in}{0.637273in}}%
\pgfpathlineto{\pgfqpoint{4.385428in}{0.914287in}}%
\pgfpathlineto{\pgfqpoint{4.386868in}{0.637273in}}%
\pgfpathlineto{\pgfqpoint{4.390821in}{0.637273in}}%
\pgfpathlineto{\pgfqpoint{4.390849in}{0.911544in}}%
\pgfpathlineto{\pgfqpoint{4.392360in}{0.637273in}}%
\pgfpathlineto{\pgfqpoint{4.396261in}{0.637273in}}%
\pgfpathlineto{\pgfqpoint{4.396365in}{0.914287in}}%
\pgfpathlineto{\pgfqpoint{4.397800in}{0.637273in}}%
\pgfpathlineto{\pgfqpoint{4.401701in}{0.637273in}}%
\pgfpathlineto{\pgfqpoint{4.402083in}{0.911544in}}%
\pgfpathlineto{\pgfqpoint{4.403240in}{0.637273in}}%
\pgfpathlineto{\pgfqpoint{4.407467in}{0.637273in}}%
\pgfpathlineto{\pgfqpoint{4.407528in}{0.914287in}}%
\pgfpathlineto{\pgfqpoint{4.409006in}{0.637273in}}%
\pgfpathlineto{\pgfqpoint{4.412817in}{0.637273in}}%
\pgfpathlineto{\pgfqpoint{4.412836in}{0.911544in}}%
\pgfpathlineto{\pgfqpoint{4.414357in}{0.637273in}}%
\pgfpathlineto{\pgfqpoint{4.418295in}{0.637273in}}%
\pgfpathlineto{\pgfqpoint{4.418399in}{0.914287in}}%
\pgfpathlineto{\pgfqpoint{4.419834in}{0.637273in}}%
\pgfpathlineto{\pgfqpoint{4.429175in}{0.637273in}}%
\pgfpathlineto{\pgfqpoint{4.430275in}{0.914287in}}%
\pgfpathlineto{\pgfqpoint{4.430715in}{0.637273in}}%
\pgfpathlineto{\pgfqpoint{4.435862in}{0.637273in}}%
\pgfpathlineto{\pgfqpoint{4.435890in}{0.911544in}}%
\pgfpathlineto{\pgfqpoint{4.437401in}{0.637273in}}%
\pgfpathlineto{\pgfqpoint{4.441302in}{0.637273in}}%
\pgfpathlineto{\pgfqpoint{4.441411in}{0.914287in}}%
\pgfpathlineto{\pgfqpoint{4.442842in}{0.637273in}}%
\pgfpathlineto{\pgfqpoint{4.446889in}{0.637273in}}%
\pgfpathlineto{\pgfqpoint{4.446917in}{0.911544in}}%
\pgfpathlineto{\pgfqpoint{4.448428in}{0.637273in}}%
\pgfpathlineto{\pgfqpoint{4.452281in}{0.637273in}}%
\pgfpathlineto{\pgfqpoint{4.452324in}{0.914287in}}%
\pgfpathlineto{\pgfqpoint{4.453821in}{0.637273in}}%
\pgfpathlineto{\pgfqpoint{4.457722in}{0.637273in}}%
\pgfpathlineto{\pgfqpoint{4.458534in}{0.914287in}}%
\pgfpathlineto{\pgfqpoint{4.459261in}{0.637273in}}%
\pgfpathlineto{\pgfqpoint{4.465905in}{0.637273in}}%
\pgfpathlineto{\pgfqpoint{4.465934in}{0.911544in}}%
\pgfpathlineto{\pgfqpoint{4.467445in}{0.637273in}}%
\pgfpathlineto{\pgfqpoint{4.471303in}{0.637273in}}%
\pgfpathlineto{\pgfqpoint{4.471374in}{0.914287in}}%
\pgfpathlineto{\pgfqpoint{4.472842in}{0.637273in}}%
\pgfpathlineto{\pgfqpoint{4.476762in}{0.637273in}}%
\pgfpathlineto{\pgfqpoint{4.476790in}{0.911544in}}%
\pgfpathlineto{\pgfqpoint{4.478301in}{0.637273in}}%
\pgfpathlineto{\pgfqpoint{4.482126in}{0.637273in}}%
\pgfpathlineto{\pgfqpoint{4.482226in}{0.914287in}}%
\pgfpathlineto{\pgfqpoint{4.483666in}{0.637273in}}%
\pgfpathlineto{\pgfqpoint{4.487567in}{0.637273in}}%
\pgfpathlineto{\pgfqpoint{4.487935in}{0.911544in}}%
\pgfpathlineto{\pgfqpoint{4.489106in}{0.637273in}}%
\pgfpathlineto{\pgfqpoint{4.493299in}{0.637273in}}%
\pgfpathlineto{\pgfqpoint{4.493375in}{0.914287in}}%
\pgfpathlineto{\pgfqpoint{4.494839in}{0.637273in}}%
\pgfpathlineto{\pgfqpoint{4.500170in}{0.637273in}}%
\pgfpathlineto{\pgfqpoint{4.500312in}{0.911544in}}%
\pgfpathlineto{\pgfqpoint{4.501710in}{0.637273in}}%
\pgfpathlineto{\pgfqpoint{4.505681in}{0.637273in}}%
\pgfpathlineto{\pgfqpoint{4.505747in}{0.914287in}}%
\pgfpathlineto{\pgfqpoint{4.507221in}{0.637273in}}%
\pgfpathlineto{\pgfqpoint{4.511169in}{0.637273in}}%
\pgfpathlineto{\pgfqpoint{4.511197in}{0.911544in}}%
\pgfpathlineto{\pgfqpoint{4.512708in}{0.637273in}}%
\pgfpathlineto{\pgfqpoint{4.516769in}{0.637273in}}%
\pgfpathlineto{\pgfqpoint{4.516840in}{0.914287in}}%
\pgfpathlineto{\pgfqpoint{4.518309in}{0.637273in}}%
\pgfpathlineto{\pgfqpoint{4.522252in}{0.637273in}}%
\pgfpathlineto{\pgfqpoint{4.523621in}{0.936228in}}%
\pgfpathlineto{\pgfqpoint{4.523791in}{0.637273in}}%
\pgfpathlineto{\pgfqpoint{4.523891in}{0.637273in}}%
\pgfpathlineto{\pgfqpoint{4.523895in}{0.936228in}}%
\pgfpathlineto{\pgfqpoint{4.525430in}{0.637273in}}%
\pgfpathlineto{\pgfqpoint{4.529402in}{0.637273in}}%
\pgfpathlineto{\pgfqpoint{4.529468in}{0.914287in}}%
\pgfpathlineto{\pgfqpoint{4.530941in}{0.637273in}}%
\pgfpathlineto{\pgfqpoint{4.534922in}{0.637273in}}%
\pgfpathlineto{\pgfqpoint{4.534950in}{0.911544in}}%
\pgfpathlineto{\pgfqpoint{4.536461in}{0.637273in}}%
\pgfpathlineto{\pgfqpoint{4.540315in}{0.637273in}}%
\pgfpathlineto{\pgfqpoint{4.540390in}{0.914287in}}%
\pgfpathlineto{\pgfqpoint{4.541854in}{0.637273in}}%
\pgfpathlineto{\pgfqpoint{4.546057in}{0.637273in}}%
\pgfpathlineto{\pgfqpoint{4.546085in}{0.911544in}}%
\pgfpathlineto{\pgfqpoint{4.547597in}{0.637273in}}%
\pgfpathlineto{\pgfqpoint{4.551455in}{0.637273in}}%
\pgfpathlineto{\pgfqpoint{4.551521in}{0.914287in}}%
\pgfpathlineto{\pgfqpoint{4.552994in}{0.637273in}}%
\pgfpathlineto{\pgfqpoint{4.556937in}{0.637273in}}%
\pgfpathlineto{\pgfqpoint{4.556966in}{0.911544in}}%
\pgfpathlineto{\pgfqpoint{4.558477in}{0.637273in}}%
\pgfpathlineto{\pgfqpoint{4.562368in}{0.637273in}}%
\pgfpathlineto{\pgfqpoint{4.562444in}{0.914287in}}%
\pgfpathlineto{\pgfqpoint{4.563907in}{0.637273in}}%
\pgfpathlineto{\pgfqpoint{4.565263in}{0.637273in}}%
\pgfpathlineto{\pgfqpoint{4.565343in}{0.936228in}}%
\pgfpathlineto{\pgfqpoint{4.566802in}{0.637273in}}%
\pgfpathlineto{\pgfqpoint{4.572223in}{0.637273in}}%
\pgfpathlineto{\pgfqpoint{4.572304in}{0.936228in}}%
\pgfpathlineto{\pgfqpoint{4.573763in}{0.637273in}}%
\pgfpathlineto{\pgfqpoint{4.573961in}{0.637273in}}%
\pgfpathlineto{\pgfqpoint{4.574273in}{0.936228in}}%
\pgfpathlineto{\pgfqpoint{4.575501in}{0.637273in}}%
\pgfpathlineto{\pgfqpoint{4.580025in}{0.637273in}}%
\pgfpathlineto{\pgfqpoint{4.581470in}{0.936228in}}%
\pgfpathlineto{\pgfqpoint{4.581564in}{0.637273in}}%
\pgfpathlineto{\pgfqpoint{4.584450in}{0.637273in}}%
\pgfpathlineto{\pgfqpoint{4.584530in}{0.936228in}}%
\pgfpathlineto{\pgfqpoint{4.585989in}{0.637273in}}%
\pgfpathlineto{\pgfqpoint{4.586060in}{0.637273in}}%
\pgfpathlineto{\pgfqpoint{4.586409in}{0.936228in}}%
\pgfpathlineto{\pgfqpoint{4.587599in}{0.637273in}}%
\pgfpathlineto{\pgfqpoint{4.593134in}{0.637273in}}%
\pgfpathlineto{\pgfqpoint{4.593323in}{0.936228in}}%
\pgfpathlineto{\pgfqpoint{4.594673in}{0.637273in}}%
\pgfpathlineto{\pgfqpoint{4.594810in}{0.637273in}}%
\pgfpathlineto{\pgfqpoint{4.595127in}{0.936228in}}%
\pgfpathlineto{\pgfqpoint{4.596354in}{0.914287in}}%
\pgfpathlineto{\pgfqpoint{4.597899in}{0.637273in}}%
\pgfpathlineto{\pgfqpoint{4.601965in}{0.637273in}}%
\pgfpathlineto{\pgfqpoint{4.602158in}{0.936228in}}%
\pgfpathlineto{\pgfqpoint{4.603504in}{0.637273in}}%
\pgfpathlineto{\pgfqpoint{4.603679in}{0.637273in}}%
\pgfpathlineto{\pgfqpoint{4.603759in}{0.936228in}}%
\pgfpathlineto{\pgfqpoint{4.605218in}{0.637273in}}%
\pgfpathlineto{\pgfqpoint{4.605454in}{0.637273in}}%
\pgfpathlineto{\pgfqpoint{4.605676in}{0.936228in}}%
\pgfpathlineto{\pgfqpoint{4.606994in}{0.637273in}}%
\pgfpathlineto{\pgfqpoint{4.612528in}{0.637273in}}%
\pgfpathlineto{\pgfqpoint{4.612779in}{0.936228in}}%
\pgfpathlineto{\pgfqpoint{4.614068in}{0.637273in}}%
\pgfpathlineto{\pgfqpoint{4.614266in}{0.637273in}}%
\pgfpathlineto{\pgfqpoint{4.614346in}{0.936228in}}%
\pgfpathlineto{\pgfqpoint{4.615806in}{0.637273in}}%
\pgfpathlineto{\pgfqpoint{4.617534in}{0.637273in}}%
\pgfpathlineto{\pgfqpoint{4.617747in}{0.936228in}}%
\pgfpathlineto{\pgfqpoint{4.619073in}{0.637273in}}%
\pgfpathlineto{\pgfqpoint{4.624679in}{0.637273in}}%
\pgfpathlineto{\pgfqpoint{4.624872in}{0.936228in}}%
\pgfpathlineto{\pgfqpoint{4.626218in}{0.637273in}}%
\pgfpathlineto{\pgfqpoint{4.626369in}{0.637273in}}%
\pgfpathlineto{\pgfqpoint{4.626610in}{0.936228in}}%
\pgfpathlineto{\pgfqpoint{4.627909in}{0.637273in}}%
\pgfpathlineto{\pgfqpoint{4.628107in}{0.637273in}}%
\pgfpathlineto{\pgfqpoint{4.628145in}{0.936228in}}%
\pgfpathlineto{\pgfqpoint{4.629647in}{0.637273in}}%
\pgfpathlineto{\pgfqpoint{4.631238in}{0.637273in}}%
\pgfpathlineto{\pgfqpoint{4.631470in}{0.936228in}}%
\pgfpathlineto{\pgfqpoint{4.632778in}{0.637273in}}%
\pgfpathlineto{\pgfqpoint{4.632900in}{0.637273in}}%
\pgfpathlineto{\pgfqpoint{4.634379in}{0.936228in}}%
\pgfpathlineto{\pgfqpoint{4.634440in}{0.637273in}}%
\pgfpathlineto{\pgfqpoint{4.636164in}{0.637273in}}%
\pgfpathlineto{\pgfqpoint{4.637642in}{0.936228in}}%
\pgfpathlineto{\pgfqpoint{4.637703in}{0.637273in}}%
\pgfpathlineto{\pgfqpoint{4.640862in}{0.637273in}}%
\pgfpathlineto{\pgfqpoint{4.641963in}{0.936228in}}%
\pgfpathlineto{\pgfqpoint{4.642402in}{0.637273in}}%
\pgfpathlineto{\pgfqpoint{4.645211in}{0.637273in}}%
\pgfpathlineto{\pgfqpoint{4.646751in}{0.936228in}}%
\pgfpathlineto{\pgfqpoint{4.648295in}{0.637273in}}%
\pgfpathlineto{\pgfqpoint{4.649230in}{0.637273in}}%
\pgfpathlineto{\pgfqpoint{4.649268in}{0.936228in}}%
\pgfpathlineto{\pgfqpoint{4.650770in}{0.637273in}}%
\pgfpathlineto{\pgfqpoint{4.652177in}{0.637273in}}%
\pgfpathlineto{\pgfqpoint{4.652380in}{0.936228in}}%
\pgfpathlineto{\pgfqpoint{4.653716in}{0.637273in}}%
\pgfpathlineto{\pgfqpoint{4.654052in}{0.637273in}}%
\pgfpathlineto{\pgfqpoint{4.654089in}{0.936228in}}%
\pgfpathlineto{\pgfqpoint{4.655591in}{0.637273in}}%
\pgfpathlineto{\pgfqpoint{4.661215in}{0.637273in}}%
\pgfpathlineto{\pgfqpoint{4.661678in}{0.936228in}}%
\pgfpathlineto{\pgfqpoint{4.662755in}{0.637273in}}%
\pgfpathlineto{\pgfqpoint{4.663114in}{0.637273in}}%
\pgfpathlineto{\pgfqpoint{4.663152in}{0.936228in}}%
\pgfpathlineto{\pgfqpoint{4.664653in}{0.637273in}}%
\pgfpathlineto{\pgfqpoint{4.665040in}{0.637273in}}%
\pgfpathlineto{\pgfqpoint{4.666547in}{0.936228in}}%
\pgfpathlineto{\pgfqpoint{4.666580in}{0.637273in}}%
\pgfpathlineto{\pgfqpoint{4.669640in}{0.637273in}}%
\pgfpathlineto{\pgfqpoint{4.671028in}{3.382727in}}%
\pgfpathlineto{\pgfqpoint{4.671179in}{0.637273in}}%
\pgfpathlineto{\pgfqpoint{4.671359in}{0.637273in}}%
\pgfpathlineto{\pgfqpoint{4.671401in}{0.919772in}}%
\pgfpathlineto{\pgfqpoint{4.672898in}{0.637273in}}%
\pgfpathlineto{\pgfqpoint{4.675770in}{0.637273in}}%
\pgfpathlineto{\pgfqpoint{4.676105in}{0.919772in}}%
\pgfpathlineto{\pgfqpoint{4.677309in}{0.637273in}}%
\pgfpathlineto{\pgfqpoint{4.680166in}{0.637273in}}%
\pgfpathlineto{\pgfqpoint{4.681710in}{3.382727in}}%
\pgfpathlineto{\pgfqpoint{4.690428in}{3.382727in}}%
\pgfpathlineto{\pgfqpoint{4.690905in}{0.637273in}}%
\pgfpathlineto{\pgfqpoint{4.691967in}{3.382727in}}%
\pgfpathlineto{\pgfqpoint{4.694621in}{3.382727in}}%
\pgfpathlineto{\pgfqpoint{4.694739in}{0.637273in}}%
\pgfpathlineto{\pgfqpoint{4.696161in}{3.382727in}}%
\pgfpathlineto{\pgfqpoint{4.697714in}{3.382727in}}%
\pgfpathlineto{\pgfqpoint{4.698432in}{0.637273in}}%
\pgfpathlineto{\pgfqpoint{4.699254in}{3.382727in}}%
\pgfpathlineto{\pgfqpoint{4.699778in}{3.382727in}}%
\pgfpathlineto{\pgfqpoint{4.701317in}{0.637273in}}%
\pgfpathlineto{\pgfqpoint{4.701350in}{3.382727in}}%
\pgfpathlineto{\pgfqpoint{4.702857in}{0.637273in}}%
\pgfpathlineto{\pgfqpoint{4.763312in}{0.637273in}}%
\pgfpathlineto{\pgfqpoint{4.764856in}{3.382727in}}%
\pgfpathlineto{\pgfqpoint{4.768454in}{3.382727in}}%
\pgfpathlineto{\pgfqpoint{4.769805in}{0.637273in}}%
\pgfpathlineto{\pgfqpoint{4.769994in}{3.382727in}}%
\pgfpathlineto{\pgfqpoint{4.770065in}{3.382727in}}%
\pgfpathlineto{\pgfqpoint{4.771609in}{0.637273in}}%
\pgfpathlineto{\pgfqpoint{4.773281in}{0.637273in}}%
\pgfpathlineto{\pgfqpoint{4.773635in}{0.936228in}}%
\pgfpathlineto{\pgfqpoint{4.774820in}{0.637273in}}%
\pgfpathlineto{\pgfqpoint{4.780577in}{0.637273in}}%
\pgfpathlineto{\pgfqpoint{4.780657in}{0.936228in}}%
\pgfpathlineto{\pgfqpoint{4.782116in}{0.637273in}}%
\pgfpathlineto{\pgfqpoint{4.782296in}{0.637273in}}%
\pgfpathlineto{\pgfqpoint{4.782376in}{0.936228in}}%
\pgfpathlineto{\pgfqpoint{4.783835in}{0.637273in}}%
\pgfpathlineto{\pgfqpoint{4.783939in}{0.637273in}}%
\pgfpathlineto{\pgfqpoint{4.784298in}{0.936228in}}%
\pgfpathlineto{\pgfqpoint{4.785478in}{0.637273in}}%
\pgfpathlineto{\pgfqpoint{4.795579in}{0.637273in}}%
\pgfpathlineto{\pgfqpoint{4.795674in}{0.914287in}}%
\pgfpathlineto{\pgfqpoint{4.797119in}{0.637273in}}%
\pgfpathlineto{\pgfqpoint{4.801544in}{0.637273in}}%
\pgfpathlineto{\pgfqpoint{4.801553in}{0.911544in}}%
\pgfpathlineto{\pgfqpoint{4.803083in}{0.637273in}}%
\pgfpathlineto{\pgfqpoint{4.807083in}{0.637273in}}%
\pgfpathlineto{\pgfqpoint{4.807092in}{0.914287in}}%
\pgfpathlineto{\pgfqpoint{4.808622in}{0.637273in}}%
\pgfpathlineto{\pgfqpoint{4.812712in}{0.637273in}}%
\pgfpathlineto{\pgfqpoint{4.812721in}{0.911544in}}%
\pgfpathlineto{\pgfqpoint{4.814251in}{0.637273in}}%
\pgfpathlineto{\pgfqpoint{4.818284in}{0.637273in}}%
\pgfpathlineto{\pgfqpoint{4.818294in}{0.914287in}}%
\pgfpathlineto{\pgfqpoint{4.819824in}{0.637273in}}%
\pgfpathlineto{\pgfqpoint{4.825694in}{0.637273in}}%
\pgfpathlineto{\pgfqpoint{4.825703in}{0.911544in}}%
\pgfpathlineto{\pgfqpoint{4.827233in}{0.637273in}}%
\pgfpathlineto{\pgfqpoint{4.831228in}{0.637273in}}%
\pgfpathlineto{\pgfqpoint{4.831516in}{0.914287in}}%
\pgfpathlineto{\pgfqpoint{4.832768in}{0.637273in}}%
\pgfpathlineto{\pgfqpoint{4.837107in}{0.637273in}}%
\pgfpathlineto{\pgfqpoint{4.837117in}{0.911544in}}%
\pgfpathlineto{\pgfqpoint{4.838647in}{0.637273in}}%
\pgfpathlineto{\pgfqpoint{4.842642in}{0.637273in}}%
\pgfpathlineto{\pgfqpoint{4.842651in}{0.914287in}}%
\pgfpathlineto{\pgfqpoint{4.844181in}{0.637273in}}%
\pgfpathlineto{\pgfqpoint{4.848309in}{0.637273in}}%
\pgfpathlineto{\pgfqpoint{4.848318in}{0.911544in}}%
\pgfpathlineto{\pgfqpoint{4.849848in}{0.637273in}}%
\pgfpathlineto{\pgfqpoint{4.853857in}{0.637273in}}%
\pgfpathlineto{\pgfqpoint{4.853867in}{0.914287in}}%
\pgfpathlineto{\pgfqpoint{4.855397in}{0.637273in}}%
\pgfpathlineto{\pgfqpoint{4.859486in}{0.637273in}}%
\pgfpathlineto{\pgfqpoint{4.859496in}{0.911544in}}%
\pgfpathlineto{\pgfqpoint{4.861026in}{0.637273in}}%
\pgfpathlineto{\pgfqpoint{4.865271in}{0.637273in}}%
\pgfpathlineto{\pgfqpoint{4.865281in}{0.914287in}}%
\pgfpathlineto{\pgfqpoint{4.866811in}{0.637273in}}%
\pgfpathlineto{\pgfqpoint{4.876416in}{0.637273in}}%
\pgfpathlineto{\pgfqpoint{4.876482in}{0.914287in}}%
\pgfpathlineto{\pgfqpoint{4.877955in}{0.637273in}}%
\pgfpathlineto{\pgfqpoint{4.882073in}{0.637273in}}%
\pgfpathlineto{\pgfqpoint{4.882083in}{0.911544in}}%
\pgfpathlineto{\pgfqpoint{4.883613in}{0.637273in}}%
\pgfpathlineto{\pgfqpoint{4.887816in}{0.637273in}}%
\pgfpathlineto{\pgfqpoint{4.887825in}{0.914287in}}%
\pgfpathlineto{\pgfqpoint{4.889355in}{0.637273in}}%
\pgfpathlineto{\pgfqpoint{4.893397in}{0.637273in}}%
\pgfpathlineto{\pgfqpoint{4.893407in}{0.911544in}}%
\pgfpathlineto{\pgfqpoint{4.894937in}{0.637273in}}%
\pgfpathlineto{\pgfqpoint{4.898885in}{0.637273in}}%
\pgfpathlineto{\pgfqpoint{4.898922in}{0.914287in}}%
\pgfpathlineto{\pgfqpoint{4.900424in}{0.637273in}}%
\pgfpathlineto{\pgfqpoint{4.904504in}{0.637273in}}%
\pgfpathlineto{\pgfqpoint{4.904514in}{0.911544in}}%
\pgfpathlineto{\pgfqpoint{4.906044in}{0.637273in}}%
\pgfpathlineto{\pgfqpoint{4.909992in}{0.637273in}}%
\pgfpathlineto{\pgfqpoint{4.910010in}{0.914287in}}%
\pgfpathlineto{\pgfqpoint{4.911531in}{0.637273in}}%
\pgfpathlineto{\pgfqpoint{4.915871in}{0.637273in}}%
\pgfpathlineto{\pgfqpoint{4.915880in}{0.911544in}}%
\pgfpathlineto{\pgfqpoint{4.917410in}{0.637273in}}%
\pgfpathlineto{\pgfqpoint{4.921405in}{0.637273in}}%
\pgfpathlineto{\pgfqpoint{4.921415in}{0.914287in}}%
\pgfpathlineto{\pgfqpoint{4.922945in}{0.637273in}}%
\pgfpathlineto{\pgfqpoint{4.927034in}{0.637273in}}%
\pgfpathlineto{\pgfqpoint{4.927044in}{0.911544in}}%
\pgfpathlineto{\pgfqpoint{4.928574in}{0.637273in}}%
\pgfpathlineto{\pgfqpoint{4.932607in}{0.637273in}}%
\pgfpathlineto{\pgfqpoint{4.932616in}{0.914287in}}%
\pgfpathlineto{\pgfqpoint{4.934146in}{0.637273in}}%
\pgfpathlineto{\pgfqpoint{4.939719in}{0.637273in}}%
\pgfpathlineto{\pgfqpoint{4.939728in}{0.911544in}}%
\pgfpathlineto{\pgfqpoint{4.941258in}{0.637273in}}%
\pgfpathlineto{\pgfqpoint{4.946632in}{0.637273in}}%
\pgfpathlineto{\pgfqpoint{4.947704in}{0.914287in}}%
\pgfpathlineto{\pgfqpoint{4.948171in}{0.637273in}}%
\pgfpathlineto{\pgfqpoint{4.953536in}{0.637273in}}%
\pgfpathlineto{\pgfqpoint{4.953545in}{0.911544in}}%
\pgfpathlineto{\pgfqpoint{4.955075in}{0.637273in}}%
\pgfpathlineto{\pgfqpoint{4.959071in}{0.637273in}}%
\pgfpathlineto{\pgfqpoint{4.959080in}{0.914287in}}%
\pgfpathlineto{\pgfqpoint{4.960610in}{0.637273in}}%
\pgfpathlineto{\pgfqpoint{4.964752in}{0.637273in}}%
\pgfpathlineto{\pgfqpoint{4.964761in}{0.911544in}}%
\pgfpathlineto{\pgfqpoint{4.966291in}{0.637273in}}%
\pgfpathlineto{\pgfqpoint{4.970286in}{0.637273in}}%
\pgfpathlineto{\pgfqpoint{4.970296in}{0.914287in}}%
\pgfpathlineto{\pgfqpoint{4.971826in}{0.637273in}}%
\pgfpathlineto{\pgfqpoint{4.976274in}{0.637273in}}%
\pgfpathlineto{\pgfqpoint{4.976392in}{0.914287in}}%
\pgfpathlineto{\pgfqpoint{4.977813in}{0.637273in}}%
\pgfpathlineto{\pgfqpoint{4.982654in}{0.637273in}}%
\pgfpathlineto{\pgfqpoint{4.982663in}{0.911544in}}%
\pgfpathlineto{\pgfqpoint{4.984193in}{0.637273in}}%
\pgfpathlineto{\pgfqpoint{4.988141in}{0.637273in}}%
\pgfpathlineto{\pgfqpoint{4.988151in}{0.914287in}}%
\pgfpathlineto{\pgfqpoint{4.989681in}{0.637273in}}%
\pgfpathlineto{\pgfqpoint{4.993761in}{0.637273in}}%
\pgfpathlineto{\pgfqpoint{4.993770in}{0.911544in}}%
\pgfpathlineto{\pgfqpoint{4.995300in}{0.637273in}}%
\pgfpathlineto{\pgfqpoint{4.999295in}{0.637273in}}%
\pgfpathlineto{\pgfqpoint{4.999324in}{0.914287in}}%
\pgfpathlineto{\pgfqpoint{5.000835in}{0.637273in}}%
\pgfpathlineto{\pgfqpoint{5.005189in}{0.637273in}}%
\pgfpathlineto{\pgfqpoint{5.005255in}{0.914287in}}%
\pgfpathlineto{\pgfqpoint{5.006728in}{0.637273in}}%
\pgfpathlineto{\pgfqpoint{5.010752in}{0.637273in}}%
\pgfpathlineto{\pgfqpoint{5.010761in}{0.911544in}}%
\pgfpathlineto{\pgfqpoint{5.012291in}{0.637273in}}%
\pgfpathlineto{\pgfqpoint{5.016390in}{0.637273in}}%
\pgfpathlineto{\pgfqpoint{5.016456in}{0.914287in}}%
\pgfpathlineto{\pgfqpoint{5.017929in}{0.637273in}}%
\pgfpathlineto{\pgfqpoint{5.021953in}{0.637273in}}%
\pgfpathlineto{\pgfqpoint{5.021962in}{0.911544in}}%
\pgfpathlineto{\pgfqpoint{5.023492in}{0.637273in}}%
\pgfpathlineto{\pgfqpoint{5.027558in}{0.637273in}}%
\pgfpathlineto{\pgfqpoint{5.027875in}{0.914287in}}%
\pgfpathlineto{\pgfqpoint{5.029098in}{0.637273in}}%
\pgfpathlineto{\pgfqpoint{5.033371in}{0.637273in}}%
\pgfpathlineto{\pgfqpoint{5.033381in}{0.911544in}}%
\pgfpathlineto{\pgfqpoint{5.034911in}{0.637273in}}%
\pgfpathlineto{\pgfqpoint{5.038972in}{0.637273in}}%
\pgfpathlineto{\pgfqpoint{5.039038in}{0.914287in}}%
\pgfpathlineto{\pgfqpoint{5.040512in}{0.637273in}}%
\pgfpathlineto{\pgfqpoint{5.044573in}{0.637273in}}%
\pgfpathlineto{\pgfqpoint{5.044582in}{0.911544in}}%
\pgfpathlineto{\pgfqpoint{5.046112in}{0.637273in}}%
\pgfpathlineto{\pgfqpoint{5.050173in}{0.637273in}}%
\pgfpathlineto{\pgfqpoint{5.050240in}{0.914287in}}%
\pgfpathlineto{\pgfqpoint{5.051713in}{0.637273in}}%
\pgfpathlineto{\pgfqpoint{5.055736in}{0.637273in}}%
\pgfpathlineto{\pgfqpoint{5.055746in}{0.911544in}}%
\pgfpathlineto{\pgfqpoint{5.057276in}{0.637273in}}%
\pgfpathlineto{\pgfqpoint{5.061587in}{0.637273in}}%
\pgfpathlineto{\pgfqpoint{5.061653in}{0.914287in}}%
\pgfpathlineto{\pgfqpoint{5.063127in}{0.637273in}}%
\pgfpathlineto{\pgfqpoint{5.067150in}{0.637273in}}%
\pgfpathlineto{\pgfqpoint{5.067160in}{0.911544in}}%
\pgfpathlineto{\pgfqpoint{5.068690in}{0.637273in}}%
\pgfpathlineto{\pgfqpoint{5.072746in}{0.637273in}}%
\pgfpathlineto{\pgfqpoint{5.072812in}{0.914287in}}%
\pgfpathlineto{\pgfqpoint{5.074286in}{0.637273in}}%
\pgfpathlineto{\pgfqpoint{5.078370in}{0.637273in}}%
\pgfpathlineto{\pgfqpoint{5.078380in}{0.911544in}}%
\pgfpathlineto{\pgfqpoint{5.079910in}{0.637273in}}%
\pgfpathlineto{\pgfqpoint{5.083905in}{0.637273in}}%
\pgfpathlineto{\pgfqpoint{5.084193in}{0.914287in}}%
\pgfpathlineto{\pgfqpoint{5.085444in}{0.637273in}}%
\pgfpathlineto{\pgfqpoint{5.089728in}{0.637273in}}%
\pgfpathlineto{\pgfqpoint{5.089737in}{0.911544in}}%
\pgfpathlineto{\pgfqpoint{5.091267in}{0.637273in}}%
\pgfpathlineto{\pgfqpoint{5.094700in}{0.637273in}}%
\pgfpathlineto{\pgfqpoint{5.095413in}{0.914287in}}%
\pgfpathlineto{\pgfqpoint{5.096240in}{0.637273in}}%
\pgfpathlineto{\pgfqpoint{5.101156in}{0.637273in}}%
\pgfpathlineto{\pgfqpoint{5.101165in}{0.911544in}}%
\pgfpathlineto{\pgfqpoint{5.102695in}{0.637273in}}%
\pgfpathlineto{\pgfqpoint{5.106761in}{0.637273in}}%
\pgfpathlineto{\pgfqpoint{5.107923in}{0.914287in}}%
\pgfpathlineto{\pgfqpoint{5.108300in}{0.637273in}}%
\pgfpathlineto{\pgfqpoint{5.113958in}{0.637273in}}%
\pgfpathlineto{\pgfqpoint{5.113967in}{0.911544in}}%
\pgfpathlineto{\pgfqpoint{5.115497in}{0.637273in}}%
\pgfpathlineto{\pgfqpoint{5.119554in}{0.637273in}}%
\pgfpathlineto{\pgfqpoint{5.119563in}{0.911544in}}%
\pgfpathlineto{\pgfqpoint{5.121093in}{0.637273in}}%
\pgfpathlineto{\pgfqpoint{5.122175in}{0.637273in}}%
\pgfpathlineto{\pgfqpoint{5.122269in}{0.936228in}}%
\pgfpathlineto{\pgfqpoint{5.123714in}{0.637273in}}%
\pgfpathlineto{\pgfqpoint{5.128229in}{0.637273in}}%
\pgfpathlineto{\pgfqpoint{5.128238in}{0.936228in}}%
\pgfpathlineto{\pgfqpoint{5.129768in}{0.637273in}}%
\pgfpathlineto{\pgfqpoint{5.134339in}{0.637273in}}%
\pgfpathlineto{\pgfqpoint{5.134349in}{0.911544in}}%
\pgfpathlineto{\pgfqpoint{5.135879in}{0.637273in}}%
\pgfpathlineto{\pgfqpoint{5.139883in}{0.637273in}}%
\pgfpathlineto{\pgfqpoint{5.139940in}{0.914287in}}%
\pgfpathlineto{\pgfqpoint{5.141423in}{0.637273in}}%
\pgfpathlineto{\pgfqpoint{5.142509in}{0.637273in}}%
\pgfpathlineto{\pgfqpoint{5.142528in}{0.911544in}}%
\pgfpathlineto{\pgfqpoint{5.144048in}{0.637273in}}%
\pgfpathlineto{\pgfqpoint{5.145814in}{0.637273in}}%
\pgfpathlineto{\pgfqpoint{5.145843in}{0.911544in}}%
\pgfpathlineto{\pgfqpoint{5.147354in}{0.637273in}}%
\pgfpathlineto{\pgfqpoint{5.151415in}{0.637273in}}%
\pgfpathlineto{\pgfqpoint{5.151425in}{0.914287in}}%
\pgfpathlineto{\pgfqpoint{5.152955in}{0.637273in}}%
\pgfpathlineto{\pgfqpoint{5.162664in}{0.637273in}}%
\pgfpathlineto{\pgfqpoint{5.162673in}{0.914287in}}%
\pgfpathlineto{\pgfqpoint{5.164203in}{0.637273in}}%
\pgfpathlineto{\pgfqpoint{5.166045in}{0.637273in}}%
\pgfpathlineto{\pgfqpoint{5.166073in}{0.911544in}}%
\pgfpathlineto{\pgfqpoint{5.167584in}{0.637273in}}%
\pgfpathlineto{\pgfqpoint{5.170030in}{0.637273in}}%
\pgfpathlineto{\pgfqpoint{5.170040in}{0.911544in}}%
\pgfpathlineto{\pgfqpoint{5.171570in}{0.637273in}}%
\pgfpathlineto{\pgfqpoint{5.175815in}{0.637273in}}%
\pgfpathlineto{\pgfqpoint{5.175825in}{0.914287in}}%
\pgfpathlineto{\pgfqpoint{5.177355in}{0.637273in}}%
\pgfpathlineto{\pgfqpoint{5.183579in}{0.637273in}}%
\pgfpathlineto{\pgfqpoint{5.184627in}{0.922515in}}%
\pgfpathlineto{\pgfqpoint{5.185118in}{0.637273in}}%
\pgfpathlineto{\pgfqpoint{5.185331in}{0.637273in}}%
\pgfpathlineto{\pgfqpoint{5.186417in}{0.922515in}}%
\pgfpathlineto{\pgfqpoint{5.186870in}{0.637273in}}%
\pgfpathlineto{\pgfqpoint{5.187196in}{0.637273in}}%
\pgfpathlineto{\pgfqpoint{5.188381in}{0.922515in}}%
\pgfpathlineto{\pgfqpoint{5.188636in}{0.637273in}}%
\pgfusepath{stroke}%
\end{pgfscope}%
\begin{pgfscope}%
\pgfsetrectcap%
\pgfsetmiterjoin%
\pgfsetlinewidth{0.803000pt}%
\definecolor{currentstroke}{rgb}{0.000000,0.000000,0.000000}%
\pgfsetstrokecolor{currentstroke}%
\pgfsetdash{}{0pt}%
\pgfpathmoveto{\pgfqpoint{0.750000in}{0.500000in}}%
\pgfpathlineto{\pgfqpoint{0.750000in}{3.520000in}}%
\pgfusepath{stroke}%
\end{pgfscope}%
\begin{pgfscope}%
\pgfsetrectcap%
\pgfsetmiterjoin%
\pgfsetlinewidth{0.803000pt}%
\definecolor{currentstroke}{rgb}{0.000000,0.000000,0.000000}%
\pgfsetstrokecolor{currentstroke}%
\pgfsetdash{}{0pt}%
\pgfpathmoveto{\pgfqpoint{5.400000in}{0.500000in}}%
\pgfpathlineto{\pgfqpoint{5.400000in}{3.520000in}}%
\pgfusepath{stroke}%
\end{pgfscope}%
\begin{pgfscope}%
\pgfsetrectcap%
\pgfsetmiterjoin%
\pgfsetlinewidth{0.803000pt}%
\definecolor{currentstroke}{rgb}{0.000000,0.000000,0.000000}%
\pgfsetstrokecolor{currentstroke}%
\pgfsetdash{}{0pt}%
\pgfpathmoveto{\pgfqpoint{0.750000in}{0.500000in}}%
\pgfpathlineto{\pgfqpoint{5.400000in}{0.500000in}}%
\pgfusepath{stroke}%
\end{pgfscope}%
\begin{pgfscope}%
\pgfsetrectcap%
\pgfsetmiterjoin%
\pgfsetlinewidth{0.803000pt}%
\definecolor{currentstroke}{rgb}{0.000000,0.000000,0.000000}%
\pgfsetstrokecolor{currentstroke}%
\pgfsetdash{}{0pt}%
\pgfpathmoveto{\pgfqpoint{0.750000in}{3.520000in}}%
\pgfpathlineto{\pgfqpoint{5.400000in}{3.520000in}}%
\pgfusepath{stroke}%
\end{pgfscope}%
\end{pgfpicture}%
\makeatother%
\endgroup%

    \caption{BETH Data Set Signal}
    \label{fig:beth_userid_all}
\end{figure}


\subsection{Algorithm Development}

The code development and experimentation for this work is contained in a public Github Repository \parencite{BeattieGithub2022}. Most of the figures and results presented in section \ref{ref_results} will be generated from this code. This section outlines how the final outlier detector was developed and some of the unsuccessful detection attempts along the way. Some of the datasets in the repository are private and as such cannot be included in the repository. The code to perform the analysis is included in the repository and can be adapted to fit similar datasets or problems.

\subsubsection{Unsuccessful Algorithms}

The author experimented with many of the libraries and algorithms outlined in section \ref{ref_code_libraries}. The first iteration of development involved using techniques from the river ML \parencite{2020river} and pysad \parencite{pysad} libraries.  The tested algorithms in the pysad library include IForestASD, LODA, RSHash, xStream, and Robust Random Cut Forest. The tested (and only) algorithm from the river library is Half Space Trees. Initial experimentation was performed and the algorithms demonstrated good performance for point wise outliers on a simulated noisy sine wave. Following this, the algorithms were tested against the Power Electronics Dataset and all techniques performed poorly. These techniques are all well suited to point wise outliers. At this point, the author realized that the kind of anomalies present in the Power Electronics Dataset (and other datasets in this study) are not pointwise but are instead contextual shaplet outliers. A new detection methodology would be needed as the tested algorithms were unable to detect the contextual outliers.

The author then reexamined the available libraries and searched for ones that do not require training of a model or previous knowledge of the data, operate in a streaming context, and can use a windowed or fix memory and processing allocation. This significantly reduces the list of available algorithms and libraries as many required model training and do not operate well outside a batch context. The author evaluated the banpei \parencite{banpei} library's Singular Spectrum Transformation (SST) and Hotelling methods and the library was unable to detect the desired anomalies. The author also evaluated the Gradient and Difference detectors in TSOD \parencite{tsod} and those algorithms were also unable to detect the contextual anomalies.

\subsubsection{Matrix Profile}

Finally, the author found a suitable algorithm for contextual outlier detection called Matrix Profile from the stumpy library \parencite{law2019stumpy}. Initial testing with the multivariate matrix profile demonstrated good performance but was very computationally expensive and slow. Additionally, the library didn't facilitate an iterative computation for this method so each iteration would need to recompute the entire matrix profile across the whole window. Subsequently, the author selected to use the univariate matrix profile, which was faster, and allowed for iterative updates to the matrix profile which can be preformed efficiently. Following successful initial experimentation, the author designed an outlier model class which would compute the matrix profile using the stumpy library on every iteration, and predict whether the current event was an outlier or not. 

Initially the prediction metric was based on determining if the current mean value is greater then a multiplier of the standard deviation of the matrix profile. This worked well but often triggered false positives. This class was parameterized so the user can set the data window size, analysis window size, standard deviation multiplier, and other parameters. Filters were added to not trigger when the algorithm was still loading data into the analysis window and had recently triggered a fault to avoid redundant detection. An additional filter was added to ensure that the currently calculated metric minus the current max value is greater than zero. If it is not there is not enough information to conclude definitely the point is an outlier and it should not be flagged. This technique is used in section \ref{ref_results} to detect outliers in a variety of situations. It only requires adjusting three parameters based on the characteristics of the data being analyzed, much less than is traditionally required by machine learning techniques. Additionally it relies completely on mathematical calculations and statics and is therefore explainable in a way many machine learning techniques are not. 

\subsubsection{Rolling Range Detector}

The detector was fairly well tuned but in certain datasets, like the BETH dataset, performance was inadequate. There were many false positives being reported from the detector. After analysis of the values computed during each iteration,
a rolling range calculation and filter was implemented. With this technique, every iteration the range in the window is computed by subtracting the maximum and minimum value. A ring buffer is implemented to store the rolling range values. In this buffer, only a set number of elements are allowed in the buffer and when a new element is inserted the oldest element is purged from the buffer. If the current range is greater than the average of the rolling range ring buffer multiplied by the standard deviation multiplier, then it is considered an anomaly. If this is the case the current range is appended to the rolling range ring buffer. This improved performance significantly for datasets like the hydraulic simulation and BETH dataset.