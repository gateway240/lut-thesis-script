\section{Background}

\subsection{FIREMAN Technologies}

This section will outline the progress and overall goal of the three milestones that the current work will utilize in creating demonstrations.

\subsubsection{Large Scale Data Acquisition (WP3)}

According to \cite{wp3.1}, the primary aim of this work package is ``event-based modelling and traffic characterization techniques aiming for a reduced use of communication and storage resources. This pre-processing task is expected -among others- to optimize the subsequent data transmissions by injecting only relevant data in the network to reduce overhead and increase spectral efficiency.'' In this portion of Fireman, the researchers examined a  variety of industrial data sets including the Tennessee Eastman Process, Electricity Metering, IEC-61850 Distribution Automation, and the EPFL Smart Grid Pilot. Modeling of transmission reduction techniques centered around the electricity modeling data set. Three approaches for reducing transmissions were presented including transmission on large spikes, transmissions on accumulated variance, and time interval defined transmission with a timeout to compensate for meter or network failure. These simulations have shown a large potential for reducing transmissions without significantly affecting the accuracy of the measured signal. The researchers then modeled a Markov-Modulated Poission Process to simulate these industrial processes for use in further stages of the project.

\begin{figure}[H]
    \centering
    \includegraphics[width=0.8\textwidth]{Images/three-layer-system.PNG}
    \caption{Three layer framework for rare event detection - \cite{three-layer-approach-to-detect-anomalies-in-industrial-environments-based-on-machine-learning}}
    \label{fig:three-layer}
\end{figure}
\cite{three-layer-approach-to-detect-anomalies-in-industrial-environments-based-on-machine-learning} explain the importance of data pre-processing in the three-layer framework shown in figure \ref{fig:three-layer}. The sensors traditionally used in Cyber-Physical Systems do not have much compute or storage capacity which makes pre-processing of data at the sensor level challenging. \cite{compression} states that ``because of this compression is essential for reducing the challenge of data storage, collection, transmission, processing, and analysis at the local level.'' \cite{wp3.2} have shown that it is possible to achieve a 92.60\% compression rate which means that only 7.4\% of samples are transmitted. The researchers have also used different methods including linear interpolation to de-compress the data and have measure the error using the root mean square technique. This data set is not heavily reliant on relational constraints or ACID compliance. Because of this, the researchers have selected a non-relational Postgres database for storage because of its superior throughput and increased performance in data processing applications.

\subsubsection{Big Data Fusion (WP4)}
\label{ref_wp4}
\cite{wp4.1} state that the goal of this task is ``to show how a large number of sensors and their corresponding data can be  accommodated and aggregated at small cost for reliably detecting rare events. This task focuses on three primary objects: heterogeneous data aggregation in machine-type communications, signature-based cluster formation, and IoT platform and database selection. It is challenging to simultaneously connect a massive number of devices which is why data aggregation is important. This strategy outlined by \cite{massive-machine} relies on the principals of:
\begin{enumerate}
  \item Shortening the distance in communication, while diminishing the power consumption of the machine type devices (MTDs)
  \item Reducing the number of connections to the core, thus decreasing the congestion
  \item Extending the network coverage. 
\end{enumerate}
The researchers also introduced a cluster formation scheme based on signatures that reduces the signalling overhead required for pee-discovery in the network. This technique utilizes signal aggregates to reduce traffic congestion to the central node. Additionally the researchers surveyed IoT experts to determine the most important properties and components of an IoT system. Then they compared the five most popular cloud IoT providers: AWS, Azure, GCP, IBM Watson, and Oracle IoT to create a comparison that can be used by businesses when selecting a cloud provider that suits their business needs. SEAT, an automotive component manufacturer, provided two use cases and accompanying data sets to the researchers. The first involves early failure detection of mechanical components in the drive chain in the Paint shop which causes axial displacement. The second involves detecting early failure of the spindle on a CNC machine which ensures lineal movement over a surface. In both of these situations detecting failure early will help SEAT detect problems and solve them before they begin to impact production components and eventually reduce production downtime to zero. 

\subsubsection{Machine Intelligence for Industrial Rare-event Provisioning (WP5)}

The goal of this task is to utilize machine intelligence techniques to predict indicators of health for a machine, component, or entire industrial process to determine health and detect premature failure. The main technique proposed in this task is QARMA and the researchers applied it to the datasets discussed in \ref{ref_wp4}. According to \cite{wp5.1} ``QARMA is a family of algorithms for extracting all (or, depending on user inputs, an important subset of) valid non-dominated quantitative association rules that hold in a dataset, that can then be used for further data analysis such as deriving rule-based classifier ensembles or as explainers of classification results of other black-box classifiers.'' This technique is one of the most human understandable techniques for machine intelligence and has shown promise for this application. The researchers have analyzed traditional methods for pruning the rules created by the algorithm (as many of them can be extraneous or unreliable). They found that the fastest open source Mixed-Integer Programming (MIP) tool was able to solve the problem in 19 seconds while their parallelized hybrid search algorithm solved the problem 240 times faster when using 24 threads. \cite{wp5.1} used a breadth first search approach to compute a ``small'' subset of rules that cover the majority of instances in a training dataset covered by the totality of rules extracted by QARMA applied on it. The researchers also experimented with Deep Neural Networks on a synthetic power grid fault diagnostic dataset and discovered that QARMA is able to handle noisy data better than certain neural networks because of overfitting of training data. The rule based methods generated by QARMA produce human understandable rules which is beneficial in the present push for explainable AI.


% \subsection{FIREMAN Demonstrations }

% The COVID-19 pandemic has shown the importance of the accessibility of remote and virtual solutions. While people are unable to travel and gather in public spaces, it is increasingly difficult to create demonstrations and physical collaborative solutions. Given this situation, this work will propose two demonstrations: a virtual one in \ref{ref_virtual_demo} and a physical one in \ref{ref_physical_demo}. Both demonstrations will focus on maximizing accessibility and understanding for those outside of the scientific community, like industry professionals and lay people who may not fully understand all of the intricacies and novel accomplishments of the FIREMAN project. 

\subsubsection{Virtual Demo}
\label{ref_virtual_demo}

Many portions of the project already include simulations and analysis on sample datasets. In one way this bolsters the claim on the importance of virtual demonstrations, especially given the pandemic situation. The problem this section hopes to address is that nearly all of the aforementioned simulations are utilized in a very specific and technical context which requires a relatively intimate understanding of the field and project. An interactive online demonstration of the overall components of the system, their interconnections, and some basic explanations of the underlying logic would serve as an entry point of understanding for people all over the world. 

This proposed demonstration would consist of a web application that would contain a brief overview of the topic and what the simulation situation will entail. Once the user has read the overview they are prompted to click the 'begin simulation' button which will commence the demo. There are many potential scenarios that could be implemented in the demonstration and as an extension the user could select a demonstration based on the field that they are involved in. For example an industrial automotive plant manager may select the SEAT dataset demonstration while a researcher planning to attend a conference discussing FIREMAN techniques may choose the more theoretical Tennessee Eastman demonstration. 

The Tennessee Eastman is a commonly used reference industrial process simulation that is visualized by \cite{tennessee-eastman-medium} in figure \ref{fig:tennessee-eastman-medium}. \cite{tennessee-eastman-model} states that ``the Tennessee Eastman process is a typical industrial process that consists of five main process units: a two-phase reactor where an exothermic reaction occurs, a separator, a stripper, a compressor, and a mixer. This is a nonlinear open-loop unstable process that has been used in many studies as a case study for plant-wide control, statistical process monitoring, sensor fault detection, and identification of data-driven network models.''
\begin{figure}[H]
    \centering
    \includegraphics[width=0.8\textwidth]{Images/tennessee_eastman.jpg}
    \caption{Tennessee Eastman Process Visualization}
    \label{fig:tennessee-eastman-medium}
\end{figure}

A simulation of the Tennessee Eastman Process shown in figure \ref{fig:tennessee-eastman-medium} would involve a 3-D model similar to that shown in the figure where the user could navigate and pan around the different elements. When a user hovered over the different components a popup would appear that explains what that component is. At the top of the screen there would be a button to start the simulation. When the simulation begins there will be a time accelerated visualization of the liquid flowing through the system. The initial run would be without any fault detection and the simulation will highlight the places in the process where faults occur and the negative repercussions of those faults. Then on the second run the user will be able to enable FIREMAN to detect the process faults and will receive a notification of what is going to fail and what action is recommended to be taken. The user will also have the option to halt the process to make the necessary repairs and then resume the process. At the end of both simulations the user is presented with metrics obtained from both simulation including number of faults present compared to faults detected, estimated system downtime, and overall efficiency and throughput of the process. This simulation will concretely show the benefits of the FIREMAN project in an easy to understand and use web based simulation.


\subsection{Black Box Algorithm Explainability}

Certain high-risk use cases for machine learning algorithms demand a high level of explainability and confidence in the algorithm outputs to use in fields of decision making. Usually an algorithm will make a classification or decision but there will not be a clear explanation as to why the algorithm made that decision. In the field of power electronics, understanding why a data-driven controller is making a decision is critical to incorporating it into real world power distribution scenarios. The fundamental questions for the field of power electronics are posed by \cite{black-box-explainability} as the following:
\begin{enumerate}
    \item How can the prediction of an AI/ML algorithm be \textit{trusted} for power electronics?
    \item How can the physical insights behind these black box tools be \textit{explained} for power electronics?
\end{enumerate}

Answering these questions is complex and there are a few approaches to achieving this goal. \cite{black-box-explainability} propose using conditional entropy to determine how each input is related to each output. The generated plots are then compared to the physical insights of the system and outlier, adversarial data that falls outside the plot is identified and removed. The model is then retrained. This technique is helpful for identifying and removing adversarial data that falls outside the range of accepted values but it would not identify data overloading in a specific portion of the graph which would create an invalid classification.

\subsubsection{Model Uncertainty}
In deep learning, there are two types of uncertainty that researchers are concerned about. \cite{uncertainity-towards-data-science} explains each type:
\blockquote{
\textbf{Epistemic uncertainty} describes what the model does not know because training data was not appropriate. Epistemic uncertainty is due to limited data and knowledge. Given enough training samples, epistemic uncertainty will decrease. Epistemic uncertainty can arise in areas where there are fewer samples for training.

\textbf{Aleatoric uncertainty} is the uncertainty arising from the natural stochasticity of observations. Aleatoric uncertainty cannot be reduced even when more data is provided. When it comes to measurement errors, we call it homoscedastic uncertainty because it is constant for all samples. Input data-dependent uncertainty is known as heteroscedastic uncertainty.
}

With deep learning algorithms, it is important to know the level of confidence the model is predicting the outcome with. \cite{explaining-adversarial-examples} explain that adding simple adversarial data (like small noise to a photo) will make an image recognition algorithm incorrectly classify one animal as a completely different one. This is further concerning since adversarial data does not need to be tailored to a specific algorithm. The transferability of this type of adversarial data allows it to be applied to many black box algorithms to achieve an unintended or potentially malicious result. 

A malfunctioning sensor or bad connection can generate noise (a type of aleatoric uncertainty), which cannot be fixed by more measurements or more data. Bayseaian techniques can be used to create a probability distribution over the weights to determine a level of uncertainty for the weights. Unfortunately, retraining a large number of models on a variety of data sets is computationally expensive and time consuming. \cite{gal2016dropout} proposed using a dropout technique to approximate the Bayseaian representation. In this technique, which avoids over-fitting, network nodes are randomly sampled and dropped across many different training iterations. This dropout will randomly zero neurons in accordance with a Bernoulli distribution which was shown to be equivalent to the Bayesian approximation of the Gaussian process. It is important to perform the dropout technique while training and testing the algorithm and then compute the variance to determine the uncertainty. This enables researchers to determine that for specific values the algorithm is providing a best guess answer with high levels of uncertainty, which could then signal the need for human intervention or review in decision making.  

\subsection{Types of Outliers}

The term `outlier' or `anomaly' can have a variety of interpretations and meanings depending on the context. In order to select and evaluate appropriate techniques for outlier detection, it is essential to understand and define the various types of outliers that can be present in a dataset. There are conventionally three ways that outliers are categorized in literature as follows \cite{lai2021revisiting} and visualized graphically in figure \ref{fig:outliers-graphic}:
\blockquote{
\textbf{Point outliers} are defined as the individual instances that are anomalous with respect to the rest of the
data. The extreme values could lead to serious consequences, and therefore point outliers are often
the focus of sequential outlier detection research. These outliers can be non-temporal or temporal in nature and often represent phenomenon like intermittent sensor failure. 

\textbf{Contextual outliers}  are the individual instances that are anomalous under a specific context, such as
the discord points within the same harmonic pattern. Contextual outliers usually have relatively
larger/smaller values in their own context but not globally.

\textbf{Collective outliers} are defined as a collection of related data instances that are anomalous with respect
to the entire data set. Specifically, the individual points of a collective outlier may not be anomalous
by themselves but the co-occurrence of them becomes an outlier. Collective outliers are ubiquitous
in sequential data since there are often strong dependencies among time points.
}

\begin{figure}[H]
    \centering
    \includegraphics[width=0.8\textwidth]{Images/outliers_graphic.PNG}
    \caption{Point (left), Contextual (middle), and Collective (right) outliers \cite{lai2021revisiting}}
    \label{fig:outliers-graphic}
\end{figure}

It is useful in analyzing data to be more specific with the specific phenomenon and outlier detection desired. As such, the taxonomy of outliers will be outlined in the following sections for this work \cite{lai2021revisiting}.

% TODO: Add graph of outliers related to Subham collaboration illustrating collective outlier 
\subsubsection{Point-wise Outliers}
Point-wise outliers can fall into 2 categories: global and contextual.
\begin{itemize}[leftmargin=1cm]
    \item \textbf{Global point outliers} are single data points that significantly deviate from the overall distribution of the rest of the data points.
    \item \textbf{Contextual point outliers} are points that significantly deviate from the overall distribution of their context (ex. a  given wavelength in a signal).
\end{itemize} 
\subsubsection{Pattern-wise Outliers}
Pattern-wise outliers can fall into 3 categories: shaplet, seasonal, and trend outliers. Figure \ref{fig:contextual-outliers} shows these phenomenon visually. This subset of anomalies represents various anomalous subsequences of the data in a given context. 
\begin{itemize}[leftmargin=1cm]
    \item \textbf{Shaplet outliers} are one or a series of subsequences that has a dissimilar basic shaplet (pattern) compared with the standard data shaplet (pattern).
    \item \textbf{Seasonal outliers} are subsequences with abnormal seasonalities compared with the general data seasonality. This often represents unusual phenonmenon (ex. a spike in web traffic due to a holiday, or an increase in residential power demand because to a major televised sporting event). 
    \item \textbf{Trend outliers} are abnormal subsequenes that resultingly alter the overal distribution of the following data. This is the type of outlier that is present in one of the data sets analyzed in this work and will be a focus of the detection strategies presented. 
\end{itemize} 

\begin{figure}[H]
    \centering
    \includegraphics[width=0.8\textwidth]{Images/contextual_outliers_graphic.PNG}
    \caption{Three types of Pattern-wise Outliers \cite{lai2021revisiting}}
    \label{fig:contextual-outliers}
\end{figure}

\subsection{Current Time Series Techniques}
Numerous outlier detection techniques currently exist each is traditionally applied to one domain specific problem. These techniques are very versatile (especially unsupervised ones) and as such can be applied in a much more general sense. This section will focus on current developments and advances across all disciplines in anomaly detection techniques.

While there are many new and state-of-the-art developments, corresponding code to a publication is often not included. This makes reproducing the results presented in the paper difficult and in order to utilize these techniques in practice, the algorithm has to be implemented based on pseudo-code, a time consuming process. As such, this will serve primarily as a survey of promising techniques. Section \ref{ref_code_libraries} will examine existing implementations of some of these techniques. If one of the non-implemented techniques is proposed to have superior performance, the researchers will examine implementing and adding it to an appropriate existing open source machine learning library.

\subsubsection{Streaming Half Space Trees (HST)}
Streaming Half-Space Trees (HST) are a fast one-class anomaly detector for evolving data streams \cite{fast-anomaly-detection-streaming}. It is an efficient method for point-wise outlier detection and is implemented in a variety of anomaly detection libraries. This technique relies on ensembleing HS-Trees which are full binary trees, where all leaves are at the same depth. It is fast, relatively efficient with a time and space complexity of $O(1)$, and is suitable for handling streaming data.

% \subsubsection{RS Hash}

% \subsubsection{Isolation Forests}

\subsubsection{Local Outlier Factor (LOF)}

The Local Outlier Factor (LOF) technique assigns a degree of `outlierness' to each datapoint as opposed to a binary classifier. The traditional LOF method is used to detect outliers in static datasets. Since it must compute neighborhood distances for each data point, which must be recomputed entirely when a new datapoint is added, it is not well suited for streaming data. Many researchers have proposed streaming adaptations to the base LOF algorithm \cite{dilof-data-streams}, \cite{fast-memory-efficent-lof-milof}. The extensions modify the algorithm to be suitable for stream learning as well as improve memory and computational performance and speed. This algorithm is well suited to point-wise contextual outlier detection.

\subsubsection{Matrix Profile} % TODO: improve matrix profile section
\label{ref_matrix-profile-alg}
The Matrix Profile is defined as a vector that represents the distances between all subsequences within a time seires and their nearest neighbors \cite{yeh2016matrix-profile-1}.

% \subsubsection{LSTM}

% \subsubsection{One-Class SVM}

% \subsubsection{Vector Autoregressor}

% \subsubsection{STL}

% \subsubsection{Clustering}

% \subsubsection{Deep Neural Network (DNN)}

% \subsubsection{Inlier Priority of Discriminate Network}

% \subsubsection{Set-Based Processing}

\subsection{Anomaly Detection Libraries}
\label{ref_code_libraries}
 Techniques were surveyed from a variety of github repositories and online sources using comprehensive tools and frameworks lists \cite{medico2020-ts-list}. The most relevant and promising libraries are selected and their performance will be evaluated with the sample data sets. Certain exclusion criteria was applied to narrow the list of frameworks.This list includes:
 \begin{itemize}[leftmargin=1cm]
     \item Not supporting python
     \item Does not run on specific platforms (ex. only Ubuntu)
     \item No longer being maintained (explicitly or no recent git commits)
     \item Non-suitability for the provided task (focused on forecasting, point-wise outliers, etc. )
     \item Focused on a specific non-applicable domain (like image processing)
     \item Designed primarily for creating a production workflow

 \end{itemize}
 In further research the production oriented libraries may be considered to implement a production outlier detection pipeline as many of them use the underlying libraries described here for detection. 
 
 Table \ref{tab:anom_detect_lib} at the end of the section presents a summary and comparison of the algorithms and libraries discussed in the following subsections. It also highlights the techniques that work well for point-wise outlier detection, context-wise time-series detection, drift detection, and segmentation. 

\subsubsection{PySAD}

PySAD is an open-source python framework for anomaly detection on streaming multivariate data \cite{pysad}. PySAD can operate in a streaming context so models can be updated as new data points arrive. This is both computational and memory efficient. PySAD also provides a variety of pre-processing, post-processing, and probability calculation tools. This library provides both univariate and multivariate prediction models for supervised and unsupervised data. 

\subsubsection{PyOD}

PyOD is a python framework for anomaly detection on non-streaming data. PySAD provides an interface for PyOD for certain batch processing tasks. PyOD includes more than 30 detection algorithms, from classical LOF (SIGMOD 2000) to the latest COPOD (ICDM 2020) and SUOD (MLSys 2021) algorithms \cite{zhao2019pyod}. Its features include:
 \begin{itemize}[leftmargin=1cm]
     \item Unified APIs, detailed documentation, and interactive examples across various algorithms.
     \item Advanced models, including classical ones from scikit-learn, latest deep learning methods, and emerging algorithms like COPOD.
     \item Optimized performance with JIT and parallelization when possible, using numba and joblib.
 \end{itemize}

\subsubsection{River}

River is a streaming machine learning library that provides tools for anomaly detection, drift detection, and more. It is a fast and efficient library that can learn and predict with single instances. Some of its features include \cite{2020river}:
 \begin{itemize}[leftmargin=1cm]
     \item Linear models with a wide array of optimizers
     \item Nearest neighbors, decision trees, naïve Bayes
     \item Anomaly detection
     \item Time series forecasting
     \item Imbalanced learning
 \end{itemize}


\subsubsection{STUMPY}
STUMPY is a library used to compute the uni-variate or multi-variate matrix profile for provided time series data created by U.S. based investment firm TD Ameritrade. The naive approach of of comparing all pairwise distances between each sub-sequence, known as a self-similarity join, is very computationally expensive with a complexity of $O(n^2m)$. STUMPY introduces a computationally efficient way to compute the matrix profile defined in section \ref{ref_matrix-profile-alg}. The library incorporates various gpu-accelerated methods to further improve performance. This can be used for a variety of time series data mining tasks including: \cite{law2019stumpy}
\begin{itemize}[leftmargin=1cm]
    \item pattern/motif (approximately repeated subsequences within a longer time series) discovery
    \item anomaly/novelty (discord) discovery
    \item shapelet discovery
    \item semantic segmentation
    \item streaming (on-line) data
    \item fast approximate matrix profiles
    \item time series chains (temporally ordered set of subsequence patterns)
    \item snippets for summarizing long time series
    \item pan matrix profiles for selecting the best subsequence window size(s)
\end{itemize}
STUMPY is well suited to analyzing pattern-wise outliers in the dataset and finding anomalies, called discords in this discipline. This library will be used to create and analyze a uni-variate and multi-variate Matrix Profile for the sample dataset. 

\subsubsection{tsod}

The tsod package aims to provide examples and algorithms for detecting anomalies in time series data specifically tailored to the water domain \cite{tsod}. Although it is designed for the water domain, the techniques can be analyzed and used in a variety of other domains. It includes a gradient detection technique for detecting abrupt changes that will be explored.

\subsubsection{TODS}
TODS provides a varity of automated machine learning tools for outlier detection on multivariate time-series data \cite{Lai_2021_TODS}. It includes many state-of-the-art algorithms and anomaly detection techniques. It is focused on the following three tasks \cite{Lai_2021_TODS}
\begin{itemize}[leftmargin=1cm]
    \item \textbf{Full Stack Machine Learning System} with support for prepossessing, feature extraction, detection algorithms, and human-in-the-loop interfaces.
    \item \textbf{Extensive Algorithms}, including PyOD's point-wise detection algorithms, pattern-wise detection algorithms, and ensemble algorithms for global anomaly detection.
    \item \textbf{Automated Machine Learning} to provide knowledge-free processes that construct optimal pipelines based on the provided data by automatically searching for the best combination of techniques from existing modules.
\end{itemize}

\subsubsection{Alibi Detect}
Alibi Detect is a library developed by U.K. based Seldon for outlier, adversarial, and drift data detection compatible with TensorFlow and PyTorch backends \cite{alibi-detect}. There are a variety of supported algorithms for these three types of data detection tasks and the package is actively maintained. The company also provide and maintain an enterprise machine learning deployment platform that integrates with Alibi Detect. The researchers will explore this further in subsequent work when examining the production pipeline. The framework handles streaming and offline data detection for time series, tabular, text, and image data. 

\subsubsection{banpei}

Banpei is a Python package of the anomaly detection \cite{banpei}. It implements the Singular spectrum transformation (SST) algorithm for change point detection and Hotelling's theory for outlier detection. It was designed to offer real time monitoring functionality and operates on streaming data.

\subsubsection{SaxPy}

Symbolic Aggregate approximation (SAX) is used to transform a series of numerical data points into a sequence of letters \cite{senin2018grammarviz-saxpy}. The library implements HOT-SAX, a time series anomaly detection algorithm. 

% \subsubsection{RNN-Time-Series}

\subsubsection{Darts}
Darts is a machine learning library primarily for time series data forecasting developed by the Swiss AI company Unit8. The library provides a variety of advanced and classic models with a sci-kit learn compatible interface. \enquote{The emphasis of the library is on offering modern machine learning functionalities such as supporting multidimensional series, meta-learning on multiple series, training on large datasets, incorporating external data, ensembling models, and providing a rich support for probabilistic forecasting \cite{herzen2021darts}.}

\subsubsection{SalesForce's Merlion}
Merlion \cite{bhatnagar2021merlion} is a Python library for time series intelligence. It provides an end-to-end machine learning framework that includes loading and transforming data, building and training models, post-processing model outputs, and evaluating model performance. It supports various time series learning tasks, including forecasting and anomaly detection for both univariate and multivariate time series.
The key features of this library are:
\begin{itemize}[leftmargin=1cm]
    \item Standardized and easily extensible data loading \& benchmarking for a wide range of forecasting and anomaly detection datasets.
    \item A library of diverse models for both anomaly detection and forecasting, unified under a shared interface. Models include classic statistical methods, tree ensembles, and deep learning approaches. Advanced users may fully configure each model as desired.
    \item AutoML for automated hyperaparameter tuning and model selection.
    \item Practical, industry-inspired post-processing rules for anomaly detectors that make anomaly scores more interpretable, while also reducing the number of false positives.
    \item Flexible evaluation pipelines that simulate the live deployment \& re-training of a model in production, and evaluate performance on both forecasting and anomaly detection.
\end{itemize}

\subsubsection{Zillow's Luminaire}

Luminaire \cite{chakraborty2020building-luminaire} is a python package that provides ML-driven solutions for monitoring time series data. Luminaire provides several anomaly detection and forecasting capabilities that incorporate correlational and seasonal patterns as well as uncontrollable variations in the data over time. This library provides a technique to monitor data points over a window of time which is helpful in detecting context-wise outliers in a streaming context.

\subsection{Algorithm and Library Comparison}

Table \ref{tab:anom_detect_lib} compares the various outlier detection methodologies described in section \ref{ref_code_libraries} a systematic way. Each specific sub-technique for each library is classified into a category with a description. Additionally the method is classified according to how suitable it is for online learning. For the online learning category to be a yes, the technique must be able to iteratively update the model efficiently, and predict specific instance one at a time. Partially online means that there is a pre-trained model, with individual instance detection. This table can be used to quickly compare and evaluate techniques for algorithm implementation.

\begin{tabularx}{\linewidth}{ llllX }
    \caption{Outlier Detection Overview((\textbf{PA}): Point-wise Anomaly, (\textbf{CA}): Context-wise Anomaly, (\textbf{DD}): Drift Detection, (\textbf{S}): Segmentation)} \\
    \toprule 
    Library & Algorithm & Focus & Online & Description \\
    \midrule 
    PySAD \cite{pysad} & Individual Models &  \textbf{PA} & yes & Score anomalies \\
    & Probability Calibrators & \textbf{S} & yes & Convert raw scores to probabilities\\
    \midrule 
    PyOD \cite{zhao2019pyod} & Individual Models &  \textbf{PA} & no & Score anomalies\\
    \midrule 
    River \cite{2020river} & Individual Models &  \textbf{PA} & yes & Score anomalies\\
    & Drift models & \textbf{DD} & yes & Detect concept drift \\
    & Time-series models & \textbf{S} & yes & Segment time-series data \\
    \midrule
    STUMPY \cite{law2019stumpy} & Matrix Profile \cite{yeh2016matrix-profile-1} &  \textbf{CA} & yes & Time-series subsequence analysis\\
    & Time Series Chains \cite{Zhu2017-time-series-chains} & \textbf{CA, DD} & no  & Motif drift detection \\
    & FLUSS \& FLOSS \cite{2017-fluss-floss} & \textbf{CA, S} & yes & Time-series segmentation\\
    \midrule 
    tsod \cite{tsod} & Simple Detectors &  \textbf{CA, PA} & partial & Gradient and range detectors \\
    \midrule 
    TODS \cite{Lai_2021_TODS} & Detection Algorithms &  \textbf{CA, PA} & partial & Detect anomalies\\
    & Featuer Analysis & \textbf{S} & partial & Statistics based segmentation\\
    \midrule 

    Alibi Detect \cite{alibi-detect} & Outlier Detection & \textbf{CA, PA} & partial & Anomaly detection\\
    & Adversarial Detection & \textbf{CA} & partial & Detect adversarial data \\
    & Drift Detection & \textbf{DD} & partial & Concept drift identifications \\
    \midrule 
    banpei \cite{banpei} & Hotelling's Theory &  \textbf{PA} & yes & Outlier detection\\
    & SST & \textbf{S} & yes & Change point detection \\
    \midrule 
    SaxPy \cite{senin2018grammarviz-saxpy} & SAX &  \textbf{PA, S} & no & Symbolic Aggregate approXimation \\
    & EMMA & \textbf{CA, S} & no & Time-series motif discovery\\
    \midrule 
    Darts \cite{herzen2021darts} & Filtering Models & \textbf{CA} & no & Gaussian, Kalman, Moving Average Filter \\
    & Forecasting Models & \textbf{S} & no & Predictive forecasting \\
    \midrule 
    Merlion \cite{bhatnagar2021merlion} & Anomaly Algorithms &  \textbf{CA, PA} & yes & Anomaly Detection strategies \\
    & Forecasting & \textbf{S} & yes & Predictive forecasting \\
    \midrule 
    Luminaire \cite{chakraborty2020building-luminaire} & Structural Modeling &  \textbf{ PA } & yes & Anomaly detection \\
    & Windowed Density Model & \textbf{CA, DD} & yes & Anomalous window detection using KL divergence \\
    \bottomrule 
    \label{tab:anom_detect_lib}
\end{tabularx}