\bigskip
\begin{longtable}{P{3cm} P{11cm}}
\caption{Context-wise Anomaly Classifications} \\
\toprule
\textbf{Class} & \textbf{Description} \\
\midrule
\endfirsthead
\multicolumn{2}{c}%
{\tablename\ \thetable\ -- \textit{Continued from previous page}} \\
\hline
\textbf{Class} & \textbf{Description} \\
\hline
\endhead
\hline \multicolumn{2}{r}{\textit{Continued on next page}} \\
\endfoot
\hline
\endlastfoot
    Shaplet & Shaplet outliers are classified by a shaplet or pattern that significantly differs from the normal data pattern. This outlier type classifies abrupt faults in a system and is an important outlier type for this study.
      \\
    \midrule
   Seasonal & Seasonal outliers are classified by an increased or decrease pattern frequency during a specific time period. Identifying seasonal outliers is important in understanding specific phenomenon like a spike in web traffic related to a major holiday. Another example is an increased demand in residential electrical demand because of a large televised sporting event.
      \\
    \midrule
    Trend & Trend outliers are classified by a sub-sequence of the dataset that modifies the underlying distribution of the data. Trend outliers are present in certain faults in the PEC dataset and in certain attacks in the BETH dataset. This is discussed further in Section \ref{ref_datasets}.
      \\
\label{tab:context-wise-anomalies}
\end{longtable}





