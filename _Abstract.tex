\section*{\MakeUppercase{Abstract}}
\phantomsection
\addcontentsline{toc}{section}{\numberline{}Abstract}%
\thispagestyle{empty}
\begin{singlespace} 
{
Lappeenranta-Lahti University of Technology LUT \bigskip

\parskip=0pt % Overide the paragraph skipping formatting

LUT School of Engineering Science \bigskip

Mechatronic System Design \bigskip

Alexander Beattie \bigskip \bigskip

\textbf{\MakeUppercase{Detecting Temporal Anomalies in Time Series Data}} \\
\textbf{\MakeUppercase{Utilizing the Matrix Profile}} \bigskip\bigskip

Master's thesis \bigskip

\the\year{} \bigskip

\pageref{LastPage} pages, \TotalValue{figure} figures, \TotalValue{table} tables \bigskip

Examiners: Dr. Pedro H. J. Nardelli and Dr. Heikki Handroos \bigskip\bigskip

Keywords: Anomaly Detection, Time Series, Matrix Profile \bigskip
}

This work presents a review of anomaly detection algorithms and libraries and the strengths and weaknesses of the most commonly used benchmarking datasets. With this information, the experimental datasets are selected and a practical implementation of an anomaly detector is created.

The matrix profile algorithm is selected for implementation because of its generalizeable approach for detecting real-time anomalies in streaming time series data.
The STUMPY python library implementation of the iterative matrix profile is used for the creation of the detector.
A series of custom filters is created and added to the detector to tune its sensitivity, recall, and detection accuracy. 

Three experiments with significantly different conditions are presented to demonstrate the generalizability and performance of the detector.
The experimential datasets used in this study include a hydraulic simulation, Power Electronic Converter (PEC), and cyber-security intrusion detection dataset.
With simple parameter tuning, the detector provides high accuracy and performance in a variety of difficult circumstances.
In the future, the detector can be improved to further increase precision and accuracy in more complex circumstances.
Additionally, the detector can be developed for use in a real-time, production system for monitoring and decision making.
\end{singlespace}